\setcounter{section}{0}

\begin{enumerate}[label=\bfseries Câu \arabic*:]
	%1
	\item \mkstar{1}
	
	\cauhoi{Tính chất nào sau đây \textbf{không phải} là tính chất chung của các tia $\alpha$, $\beta$, $\gamma$?
		\begin{mcq}
			\item Có khả năng ion hoá không khí.	
			\item Bị lệch trong điện trường hoặc từ trường.
			\item Có tác dụng làm đen kính ảnh.
			\item Có mang năng lượng.
		\end{mcq}}	
		
		\loigiai{\textbf{Đáp án: B.}
			
			Tia $\gamma$ không bị lệch trong điện trường hoặc từ trường, do đó đây không phải là tính chất chung của các tia $\alpha$, $\beta$, $\gamma$.
			
			}
	%2
	\item \mkstar{1}
	
	\cauhoi{Phát biểu nào sau đây \textbf{không đúng}?
		\begin{mcq}
			\item Tia $\alpha$ bị lệch về bản âm của tụ điện.	
			\item Tia $\alpha$ là hạt nhân nguyên tử $\ce{He}$.
			\item Tia $\beta ^-$ phát ra từ lớp vỏ của nguyên tử vì nó là electron.
			\item Tia $\gamma$ là sóng điện từ.
		\end{mcq}}	
		
		\loigiai{\textbf{Đáp án: C.}
			
			Thực chất của phóng xạ $\beta ^-$ là 1 nơtron biến thành 1 proton, 1 electron và 1 nơtrinô. Vậy tia $\beta ^-$ không phải phát ra từ lớp vỏ của nguyên tử mà là từ hạt nhân.
			
			}
	%3
	\item \mkstar{1}
	
	\cauhoi{Thực chất của phóng xạ $\gamma$ là
		\begin{mcq}
			\item hạt nhân bị kích thích bức xạ photon.	
			\item dịch chuyển giữa các mức năng lượng ở trạng thái dừng trong nguyên tử.
			\item do tương tác giữa electron và hạt nhân làm phát ra bức xạ hãm.
			\item do electron trong nguyên tử dao động bức xạ ra dưới dạng sóng điện từ.
		\end{mcq}	}
		
		\loigiai{\textbf{Đáp án: A.}
			
			Thực chất của phóng xạ $\gamma$ là hạt nhân bị kích thích bức xạ photon.
			
			}	
	%4
	\item \mkstar{1}
	
	\cauhoi{Các tia nào sau đây xuyên qua được tấm chì dày cỡ cen-ti-mét?
		\begin{mcq}(2)
			\item Tia tử ngoại và tia hồng ngoại.	
			\item Tia $X$ và tia gamma.
			\item Tia gamma.
			\item Tia $X$ và tia tử ngoại.
		\end{mcq}	}
		
		\loigiai{\textbf{Đáp án: C.}
			
			Tia $\gamma$ xuyên qua được tấm chì dày cỡ $\SI{}{\centi \meter}$. Tia $X$ bị chì cản lại.
			
			}	
	%5
	\item \mkstar{1}
	
	\cauhoi{Biến đổi của proton thành nơtron xảy ra trong lòng hạt nhân là bản chất của quá trình phóng xạ
		\begin{mcq}(4)
			\item $\beta^+$.	
			\item $\beta^-$.
			\item $\gamma$.
			\item $\alpha$.
		\end{mcq}	}
		
		\loigiai{	\textbf{Đáp án: A.}
			
			Thực chất của quá trình phóng xạ $\beta ^+$ là biến đổi proton thành nơtron, phản electron và phản nơtrinô.
			
			}	
	%6
	
	\item \mkstar{1}
	
	\cauhoi{Có thể tăng hằng số phóng xạ $\lambda$ của đồng vị phóng xạ bằng cách nào?
		\begin{mcq}
			\item Đặt nguồn phóng xạ đó vào trong từ trường mạnh.	
			\item Đặt nguồn phóng xạ đó vào trong điện trường mạnh.
			\item Đặt nguồn phóng xạ đó vào nơi có nhiệt độ cao.
			\item Hiện nay chưa có cách nào để thay đổi hằng số phóng xạ.
		\end{mcq}	}
		
		\loigiai{	\textbf{Đáp án: D.}
			
			Hiện nay, chưa có cách nào để thay đổi hằng số phóng xạ của các đồng vị phóng xạ.
			
		}
	%7
	\item \mkstar{2}
	
	\cauhoi{Một hạt nhân nguyên tử phóng xạ lần lượt một tia $\alpha$, rồi một tia $\beta$ thì hạt nhân nguyên tử sẽ biến đổi như thế nào?
		\begin{mcq}(2)
			\item Số khối giảm 4, số proton giảm 2.	
			\item Số khối giảm 4, số proton giảm 1.
			\item Số khối tăng 4, số proton giảm 1.
			\item Số khối giảm 3, số proton tăng 1.
		\end{mcq}	}
		
		\loigiai{\textbf{Đáp án: B.}
			
			Sau phóng xạ $\alpha$, số khối giảm 4, số proton giảm 2. Sau phóng xạ $\beta$ (gọi tắt của $\beta ^-$), số proton tăng 1. Vậy sau phóng xạ $\alpha$ và $\beta$, số khối giảm 4, số proton giảm 1.
			
			}
	%8
	\item \mkstar{2}
	
	\cauhoi{Phát biểu nào sau đây là \textbf{sai} khi nói về hiện tượng phóng xạ?
		\begin{mcq}
			\item Trong phóng xạ $\alpha$, hạt nhân con có số nơtron nhỏ hơn số nơtron của hạt nhân mẹ.
			\item Trong phóng xạ $\beta ^-$, hạt nhân mẹ và hạt nhân con có số khối bằng nhau, số proton khác nhau.
			\item Trong phóng xạ $\beta$, có sự bảo toàn điện tích nên số proton được bảo toàn.
			\item Trong phóng xạ $\beta ^+$, hạt nhân mẹ và hạt nhân con có số khối bằng nhau, số nơtron khác nhau.
		\end{mcq}}
		
		\loigiai{\textbf{Đáp án: C.}
			
			Định luật bảo toàn điện tích không đồng nghĩa số proton được bảo toàn.
			
			}
	
	
\end{enumerate}

\loigiai{\textbf{Đáp án}
	\begin{center}
		\begin{tabular}{|m{2.8em}|m{2.8em}|m{2.8em}|m{2.8em}|m{2.8em}|m{2.8em}|m{2.8em}|m{2.8em}|m{2.8em}|m{2.8em}|}
			\hline
			1. B & 2. C & 3. A & 4. C & 5. A & 6. D & 7. B & 8. C & & \\
			\hline
			
		\end{tabular}
\end{center}}