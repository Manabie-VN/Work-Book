\begin{enumerate}[label=\bfseries Câu \arabic*:]
	
\item \mkstar{1}

\cauhoi{Trong máy phát điện xoay chiều 3 pha
	\begin{mcq}
		\item Stato là phần cảm, rôto là phần ứng.
		\item Phần nào quay là phần ứng.
		\item Stato là phần ứng, rôto là phần cảm.
		\item Phần nào đứng yên là phần tạo ra từ trường.
	\end{mcq}
}
	\loigiai{\textbf{Đáp án: C.}
		
		Máy phát điện xoay chiều có ba phần gồm ba cuộn dây (phần ứng) mắc trên một vành tròn tại ba vị trí đối xứng, trục của ba vòng dây lệch nhau một góc $120^\circ$.
		
		Nam châm (phần cảm) quay quanh một trục đóng vai trò là roto.
		
		}
\item \mkstar{1}

\cauhoi{Máy phát điện xoay chiều 3 pha hoạt động dựa trên hiện tượng gì?
	\begin{mcq}(2)
		\item Nhiệt điện.
		\item Tự cảm.
		\item Cảm ứng điện từ.
		\item Siêu dẫn.
	\end{mcq}
}
\loigiai{\textbf{Đáp án: C.}
	
	Máy phát điện xoay chiều 3 pha hoạt động dựa trên hiện tượng cảm ứng điện từ.
	
}
\item \mkstar{2}

\cauhoi{Một máy phát điện xoay chiều 1 pha có rôto gồm 4 cặp cực từ, muốn tần số dòng điện xoay chiều mà máyphát ra là $\SI{50}{Hz}$ thì rôto phải quay với tốc độ là bao nhiêu?
	\begin{mcq}(2)
		\item 3000 vòng/phút.
		\item 1500 vòng/phút.
		\item 750 vòng/phút.
		\item 500 vòng/phút.
	\end{mcq}
}
	\loigiai{\textbf{Đáp án: C.}
		
		Tốc độ quay của roto $n=\dfrac{60f}{p}=750 \textrm{vòng/phút}$.
		
		}
\item \mkstar{2}

\cauhoi{Một máy phát điện xoay chiều 1 pha có rôto gồm 4 cực từ (2 nam, 2 bắc), muốn tần số dòng điện xoay chiều mà máyphát ra là $\SI{50}{Hz}$ thì rôto phải quay với tốc độ là bao nhiêu?
	\begin{mcq}(2)
		\item 3000 vòng/phút.
		\item 1500 vòng/phút.
		\item 750 vòng/phút.
		\item 500 vòng/phút.
	\end{mcq}
}
\loigiai{\textbf{Đáp án: B.}
	
	Tốc độ quay của roto $n=\dfrac{60f}{p}=1500 \textrm{vòng/phút}$.
	
}
\item \mkstar{2}

\cauhoi{Một máy phát điện xoay chiều một pha có phần cảm là rôto gồm 10 cặp cực (10 cực nam và 10 cực bắc). Rô to quay với tốc độ 300 vòng/phút. Suất điện động do máy sinh ra có tần số bằng
	\begin{mcq}(4)
		\item $\SI{5}{Hz}$.
		\item $\SI{30}{Hz}$.
		\item $\SI{300}{Hz}$.
		\item $\SI{50}{Hz}$.
	\end{mcq}
}
	\loigiai{\textbf{Đáp án: D.}
		
		Tần số của suất điện động $f=\dfrac{pn}{60}=\SI{50}{Hz}$.
		
		}
	\item \mkstar{2}
	
	\cauhoi{Một máy phát điện xoay chiều một pha có phần cảm là rôto gồm 10 cặp cực (10 cực nam và 10 cực bắc). Rô to quay với tốc độ $3600\pi\ \text{rad/phút}$. Suất điện động do máy sinh ra có tần số bằng
		\begin{mcq}(4)
			\item $\SI{5}{Hz}$.
			\item $\SI{30}{Hz}$.
			\item $\SI{300}{Hz}$.
			\item $\SI{50}{Hz}$.
		\end{mcq}
	}
	\loigiai{\textbf{Đáp án: C.}
		
		Đổi $n=3600\pi\ \text{rad/phút} = 1800\ \text{vòng/phút}$
		Tần số của suất điện động $f=\dfrac{pn}{60}=\SI{300}{Hz}$.
		
	}
\item \mkstar{2}

\cauhoi{Suất điện động do một máy phát điện xoay chiều một pha tạo ra có biểu thức $e=120\sqrt{2}\cos(100\pi t)\,\text{V}$. Giá trị hiệu dụng của suất điện động này bằng
	\begin{mcq} (4)
		\item $\SI{100}{\volt}$.
		\item $\SI{120}{\volt}$.
		\item $120\sqrt{2}\,\text{V}$.
		\item $100\pi\,\text{V}$.
	\end{mcq}
}
\loigiai{\textbf{Đáp án: B.}
	
Giá trị hiệu dụng của suất điện động này bằng
\begin{equation*}
	e=\dfrac{E_0}{\sqrt{2}}=\dfrac{120\sqrt{2}\,\text{V}}{\sqrt{2}}=\SI{120}{\volt}.
\end{equation*}
	
}

\item \mkstar{3}

\cauhoi{Một máy phát điện xoay chiều một pha có điện trở không đáng kể. Nối 2 cực của máy với cuộn dây thuần cảm. Khi roto quay với tốc độ $n$ vòng/s thì cường độ dòng điện hiệu dụng qua cuộn cảm là $I$. Hỏi khi roto quay với tốc độ $3n$ vòng/s thì cường độ dòng điện hiệu dụng qua cuộn cảm bao nhiêu?
	\begin{mcq}(4)
		\item $I$.
		\item $2I$.
		\item $3I$.
		\item $\dfrac{I}{3}$.
	\end{mcq}
}	
	\loigiai{\textbf{Đáp án: A.}
		
		Khi mạch ngoài của máy phát nối với cuộn cảm thì dòng điện qua cuộn cảm không phụ thuộc vào tốc độ quay của roto nên khi roto quay với tốc độ $n$ và $3n$ thì dòng trong mạch luôn là $I$.
		
		}
\item \mkstar{4}

\cauhoi{Hai máy phát điện xoay chiều một pha đang hoạt động bình thường và tạo ra hai suất điện động có cùng tần số $f$. Rôto của máy thứ nhất có $p_1$ cặp cực và quay với tốc độ $n_1=1800\,\text{vòng/phút}$. Rôto của máy thứ hai có $p_2=4$ cặp cực và quay với tốc độ $n_2$ . Biết $n_2$ có giá trị trong khoảng từ $12\,\text{vòng/giây}$ đến $18\,\text{vòng/giây}$. Giá trị của $f$ là
	\begin{mcq} (4)
		\item $\SI{54}{\hertz}$.
		\item $\SI{60}{\hertz}$.
		\item $\SI{48}{\hertz}$.
		\item $\SI{50}{\hertz}$.
	\end{mcq}
}
\loigiai{\textbf{Đáp án: B.}
	
	Hai máy có cùng tần số $f$ nên
	\begin{equation*}
		f=f_1=f_2\Rightarrow p_1n_1=p_2n_2\Rightarrow n_2= n_1\cdot\dfrac{p_1}{p_2}.
	\end{equation*}
	
	Vì $n_2$ có giá trị trong khoảng từ $12\,\text{vòng/giây}$ đến $18\,\text{vòng/giây}$ nên
	\begin{equation*}
		12\,\text{vòng/giây} \leq n_2 \leq 18\,\text{vòng/giây}
	\end{equation*}
	\begin{equation*}
		\Rightarrow 12\,\text{vòng/giây} \leq n_1\cdot\dfrac{p_1}{p_2} \leq 18\,\text{vòng/giây}
	\end{equation*}
	\begin{equation*}
		\Rightarrow 12\,\text{vòng/giây} \leq 30\,\text{vòng/giây}\cdot\dfrac{p_1}{4} \leq 18\,\text{vòng/giây}
	\end{equation*}
	\begin{equation*}
		1,6 \leq p_1 \leq 2,4.
	\end{equation*}
	
	Vì $p$ nguyên nên chọn $p_1=2$.
	Tần số $f$ của hai máy là
	\begin{equation*}
		f=p_1 n_1= 2\cdot 30\,\text{vòng/giây} = \SI{60}{\hertz}.
	\end{equation*}
	
}
\item \mkstar{4}

\cauhoi{Một máy phát điện xoay chiều ba pha đang hoạt động ổng định. Suất điện động trong ba cuộn dây của phần ứng có giá trị  $e_1$, $e_2$ và $e_3$. Ở thời điểm mà $e_1=\SI{30}{\volt}$ thì $\left|e_2-e_3\right|=\SI{30}{\volt}$. Giá trị cực đại của $e_1$ là
	\begin{mcq} (4)
		\item $\SI{40,2}{\volt}$.
		\item $\SI{51,9}{\volt}$.
		\item $\SI{34,6}{\volt}$.
		\item $\SI{45,1}{\volt}$.
	\end{mcq}
}
\loigiai{\textbf{Đáp án: C.}
	
	Giả sử suất điện động trong khung dây có dạng
	\begin{equation*}
		\left\{\begin{array}{ll}{{{e}_{1}}={{E}_{0}}\cos \omega t}&\\{{{e}_{2}}={{E}_{0}}\cos \left( \omega t+\dfrac{2\pi}{3}\right)}&\\{{{e}_{2}}={{E}_{0}}\cos \left( \omega t-\dfrac{2\pi }{3} \right)}&\end{array}\right.
	\end{equation*}
	
	Theo đề bài
	\begin{equation*}
		\left| {{e}_{2}}-{{e}_{3}} \right|=\SI{30}{\volt}
		\Rightarrow {{e}_{2}}-{{e}_{3}}=\pm \SI{30}{\volt}
		\Rightarrow 2{{E}_{0}}\sin \omega t\sin \frac{2\pi }{3}=\pm \SI{30}{\volt}.
	\end{equation*}
	
	Ngoài ra, cũng theo đề bài
	\begin{equation*}
		{{e}_{1}}={{E}_{0}}\cos \omega t =\SI{30}{\volt}.
	\end{equation*}
	
	Ta có hệ phương trình sau
	\begin{equation*}
		\left\{\begin{array}{ll}{-2{{E}_{0}}\sin \omega t\sin \dfrac{2\pi}{3}=\pm \SI{30}{\volt}}&\\{{{E}_{0}}\cos \omega t=\SI{30}{\volt}}&\end{array}\right.
		\Rightarrow \left\{\begin{array}{ll}{{{E}_{0}}\sin \omega t=\pm 10\sqrt{3}\,\text{V}}&\\{{{E}_{0}}\cos \omega t=\SI{30}{\volt}}&\end{array}\right.
		\Rightarrow {{E}_{0}}=20\sqrt{3}\approx \SI{34,6}{\volt}.
	\end{equation*}
}
\end{enumerate}
\loigiai{\textbf{Đáp án}
	\begin{center}
		\begin{tabular}{|m{2.8em}|m{2.8em}|m{2.8em}|m{2.8em}|m{2.8em}|m{2.8em}|m{2.8em}|m{2.8em}|m{2.8em}|m{2.8em}|}
			\hline
			1. C & 2. C & 3. C & 4. B & 5. D & 6. C  & 7. B  & 8. A & 9. B & 10. C\\
			\hline
		\end{tabular}
\end{center}}