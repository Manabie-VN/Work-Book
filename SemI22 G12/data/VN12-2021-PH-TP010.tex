\begin{enumerate}[label=\bfseries Câu \arabic*:]
	\item \mkstar{1}
	
	\cauhoi{
		
		Bước sóng là khoảng cách giữa hai điểm:
		\begin{mcq}
			\item trên cùng một phương truyền sóng mà dao động tại hai điểm đó ngược pha.
			\item gần nhau nhất trên cùng một phương truyền sóng mà dao động tại hai điểm đó cùng pha.
			\item gần nhau nhất mà dao động tại hai điểm đó cùng pha.
			\item trên cùng một phương truyền sóng mà dao động tại hai điểm đó cùng pha.
		\end{mcq}
		
	}
	\loigiai{
		\textbf{Đáp án B.}
		
		Bước sóng là khoảng cách giữa hai điểm gần nhau nhất trên cùng một phương truyền sóng mà dao động tại hai điểm đó cùng pha.
	}
	
		\item \mkstar{2}
		
		\cauhoi{
			
			Một quan sát viên khí tượng quan sát mặt biển. Nếu trên mặt biển người quan sát thấy được 10 ngọn sóng trước mắt và cách nhau 90 m. Hãy xác định bước sóng của sóng trên mặt biển.
			\begin{mcq}(4)
				\item 9 m.
				\item 10 m.
				\item 8 m.
				\item 11 m.
			\end{mcq}
		}
		\loigiai{
			\textbf{Đáp án B.}
			
			Ta có 10 ngọn sóng suy ra có $9\lambda$.
			
			$9\lambda =\SI{90}{m} \Rightarrow \lambda =\SI{10}{m}$.
		}
		\item \mkstar{2}
		
		\cauhoi{
			
			Quan sát sóng cơ trên mặt nước, ta thấy cứ 2 ngọn sóng liên tiếp cách nhau 40 cm. Nguồn sóng dao động với tần số $f=\SI{20}{Hz}$. Xác định vận tốc truyền sóng trên môi trường.
			\begin{mcq}(4)
				\item 80 cm/s.
				\item 80 m/s.
				\item 4 m/s.
				\item 8 m/s.
			\end{mcq}
		}
		\loigiai{
			\textbf{Đáp án D.}
			
			Ta có $v=\lambda f = 8\ \text{m/s}$.
		}
		\item \mkstar{2}
		
		\cauhoi{
			
			Một nguồn sóng cơ có phương trình $u_\text{O} = 4\cos 20\pi t\ \text{cm}$. Sóng truyền theo phương ON với vận tốc 20 cm/s. Hãy xác định phương trình sóng tại điểm N cách nguồn O 5 cm.
			\begin{mcq}(2)
				\item $u_\text{N} = 4\cos (20\pi t -5\pi)\ \text{cm}$.
				\item $u_\text{N} = 4\cos (20\pi t -\pi)\ \text{cm}$.
				\item $u_\text{N} = 4\cos (20\pi t -\text{2,5}\pi)\ \text{cm}$.
				\item $u_\text{N} = 4\cos (20\pi t -\text{5,5}\pi)\ \text{cm}$.
			\end{mcq}
		}
		\loigiai{
			\textbf{Đáp án A.}
			
			
			Bước sóng $\lambda =\dfrac{v}{f} = \SI{2}{cm}$.
			
			$\Delta \varphi = \dfrac{2\pi d}{2} = 5\pi \ \text{rad/s}$.
			
			Phương trình sóng có dạng: 
			
			$u_\text{N} = 4\cos (20\pi t -5\pi)\ \text{cm}$.
		}
		\item \mkstar{2}
		
		\cauhoi{ 
			
			Một nguồn sóng cơ có phương trình $u_\text{O} = 4\cos 20\pi t\ \text{cm}$. Sóng truyền theo phương ONM với vận tốc 20 cm/s. Hãy xác định độ lệch pha giữa hai điểm MN, biết $\text{MN} =\SI{1}{cm}$.
			
			\begin{mcq}(4)
				\item $\xsi{2\pi}{rad}$.
				\item $\xsi{\pi}{rad}$.
				\item $\dfrac{\pi}{2}\ \text{rad}$.
				\item $\dfrac{\pi}{3}\ \text{rad}$.
			\end{mcq}
		}
		\loigiai{
			\textbf{Đáp án B.}
			
			$\lambda = \dfrac{v}{f}=\SI{2}{cm}$
			
			$\Delta \varphi = \dfrac{2\pi d}{\lambda}=\pi\ \text{rad}$.
			
		}
		\item \mkstar{2}
		
		\cauhoi{
			
			Tại hai điểm AB trên phương truyền sóng cách nhau 4 cm có phương trình lần lượt như sau: $u_\text{M} =2\cos \left(4\pi t + \dfrac{\pi}{6}\right)\ \text{cm}$; $u_\text{N} =2\cos \left(4\pi t + \dfrac{\pi}{3}\right)\ \text{cm}$. Hãy xác định sóng truyền như thế nào?
			\begin{mcq}
				\item Truyền từ N đến M với vận tốc 96 m/s.
				\item Truyền từ N đến M với vận tốc 0,96 m/s.
				\item Truyền từ M đến N với vận tốc 96 m/s.
				\item Truyền từ M đến N với vận tốc 0,96 m/s.
			\end{mcq}
			
		}
		\loigiai{
			\textbf{Đáp án B.}
			
			Vì N nhanh pha hơn M nên sóng truyền từ N đến M
			
			$\Delta \varphi =\dfrac{2\pi d}{\lambda} =\dfrac{\pi}{6} \Rightarrow \lambda = 12d= \SI{48}{cm}$.
			
			$v=\lambda f = 96\ \text{m/s}$.
			
			
		}
		
		\item \mkstar{2}
		
		\cauhoi{
			
			Một sóng cơ truyền với phương trình $u = 5\cos \left (20\pi t - \dfrac{\pi \cdot x}{2}\right)\ \text{cm}$ (trong đó $x$ tính bằng m, $t$ tính bằng giây). Xác định vận tốc truyền sóng trong môi trường.
			\begin{mcq}(4)
				\item 20 m/s.
				\item 40 cm/s.
				\item 30 cm/s.
				\item 40 m/s.
			\end{mcq}
			
		}
		\loigiai{
			\textbf{Đáp án D.}
			
			Ta có: $\Delta \varphi =\dfrac{2\pi x}{\lambda} = \dfrac{\pi}{x}{2} \Rightarrow \lambda =\SI{4}{m} \Rightarrow v=\lambda f =40\ \text{m/s}$.
			
			
		}
		
		\item \mkstar{2}
		
		\cauhoi{
			
			Một sóng cơ truyền với phương trình  $u = 5\cos \left (20\pi t - \dfrac{\pi \cdot x}{2}\right)\ \text{cm}$ (trong đó $x$ tính bằng m, $t$ tính bằng giây). Tại $t_1$ thì $u=\SI{4}{cm}$. Hỏi tại $t=(t_1+2)\ \text{s}$ thì độ dời của sóng là bao nhiêu?
			\begin{mcq}(4)
				\item - 4 cm.
				\item 2 cm.
				\item 4 cm.
				\item - 2 cm.
			\end{mcq}
			
		}
		\loigiai{
			\textbf{Đáp án C.}
			
			
			Tại $t_1$ thì $u=5\cos \left (20\pi t - \dfrac{\pi \cdot x}{2}\right)= \SI{4}{cm}$.
			
			Tại $t=t_1+2\ \text{s}$ thì $u_2= 5\cos \left (20\pi (t+2) - \dfrac{\pi \cdot x}{2}\right)= 5\cos \left (20\pi t - \dfrac{\pi \cdot x}{2}+40\pi\right)= 5\cos \left (20\pi t - \dfrac{\pi \cdot x}{2}\right)=\SI{4}{cm}$.
			
			
		}
		\item \mkstar{3}
		
		\cauhoi{
			
			Một mũi nhọn S chạm nhẹ vào mặt nước dao động điều hòa với tần số 20 Hz thì thấy hai điểm A và B trên mặt nước cùng nằm trên một phương truyền sóng cách nhau một khoảng $d=\SI{10}{cm}$ luôn luôn dao động ngược pha với nhau. Tốc độ truyền sóng có giá trị ($\text{0,8}\ \text{m/s} \leq v \leq 1\ \text{m/s}$) là
			\begin{mcq}(4)
				\item 0,8 m/s.
				\item 1 m/s.
				\item 0,9 m/s.
				\item 0,7 m/s.
			\end{mcq}
			
		}
		\loigiai{
			\textbf{Đáp án A.}
			
			$\Delta \varphi =\dfrac{2\pi d}{\lambda} =\dfrac{2\pi fd}{v} =(2k+1)\pi \Rightarrow v = \dfrac{2fd}{2k+1}(1)$ (theo đề bài $\text{0,8}\ \text{m/s} \leq v \leq 1\ \text{m/s}$).
			
			$\Rightarrow 80 \leq \dfrac{2fd}{2k+1} \leq 100$ giải ra ta được $\text{1,5} \leq k \leq 2$ 
			
			Chọn $k=2$ thay $k$ vào (1) ta có: $v=80\ \text{cm/s}$.
		}
		\item \mkstar{3}
		
		\cauhoi{
			
			Một nguồn sóng O dao động với phương trình $x=A\cos \left(\omega t +\dfrac{\pi}{2}\right)\ \text{cm}$. Tại điểm M cách O một khoảng $\dfrac{\lambda}{2}$ điểm $\dfrac{T}{2}$ dao động với li độ $2\sqrt 3\ \text{cm}$. Hãy xác định biên độ sóng.
			\begin{mcq}(4)
				\item $2\sqrt 3$ cm.
				\item 4 cm.
				\item 8 cm.
				\item $4\sqrt 3$ cm.
			\end{mcq}
			
		}
		\loigiai{
			\textbf{Đáp án B.}
			
			
			Ta có $u_\text{M} = A\cos \left (\omega t +\dfrac{\pi}{2}-\dfrac{2\pi d}{\lambda}\right)\ \text{cm}$.
			
			Suy ra: $u_\text{M} = A\cos \left(\omega t +\dfrac{\pi}{2} -\pi\right)\ \text{cm}$.
			
			Ở thời điểm $t=\dfrac{T}{3} \Rightarrow u_\text{M} = A \cos \dfrac{\pi}{6} = 2\sqrt 3 \Rightarrow A =\SI{4}{cm}$.
			
		}
	
\end{enumerate}
\loigiai{\textbf{Đáp án}
	\begin{center}
		\begin{tabular}{|m{2.8em}|m{2.8em}|m{2.8em}|m{2.8em}|m{2.8em}|m{2.8em}|m{2.8em}|m{2.8em}|m{2.8em}|m{2.8em}|}
			\hline
			1. B & 2. B & 3. D & 4. A & 5. B & 6. B  & 7. D  & 8. C & 9. A & 10. B\\
			\hline
		\end{tabular}
\end{center}}