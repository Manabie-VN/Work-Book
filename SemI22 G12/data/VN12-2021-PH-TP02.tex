\setcounter{section}{0}

\begin{enumerate}[label=\bfseries Câu \arabic*:]
	%1
	\item \mkstar{1}
	
	\cauhoi{Chọn câu \textbf{sai}.
		\begin{mcq}
			\item Các hạt nhân nặng trung bình (có số khối trung bình) là bền vững nhất.	
			\item Các hạt $p$, $e$ thì bền vững.	
			\item Hạt nhân có năng lượng liên kết càng lớn thì càng bền vững.	
			\item Hạt nhân có năng lượng liên kết riêng càng lớn thì càng bền vững.
		\end{mcq}}
	
		\loigiai{\textbf{Đáp án: C.}
			
			Độ bền vững của hạt nhân phụ thuộc năng lượng liên kết riêng, không phải năng lượng liên kết.
			
				}
	%2
	\item \mkstar{1}
	
	\cauhoi{Tìm phát biểu \textbf{sai} về năng lượng liên kết.
		\begin{mcq}
			\item Muốn phá vỡ hạt nhân có khối lượng $m$ thành các nuclôn có tổng khối lượng $m_0>m$ thì ta phải tốn năng lượng $\Delta E = (m_0-m)c^2$ để thắng lực hạt nhân.	
			\item Hạt nhân có năng lượng liên kết $\Delta E$ càng lớn thì càng bền vững.	
			\item Năng lượng liên kết tính cho một nuclôn được gọi là năng lượng liên kết riêng.	
			\item Hạt nhân có năng lượng liên kết riêng càng nhỏ thì càng kém bền vững.
		\end{mcq}}
	
		\loigiai{\textbf{Đáp án: B.}
			
			Độ bền vững của hạt nhân phụ thuộc năng lượng liên kết riêng, không phải năng lượng liên kết.
			
				}
	%3
	\item \mkstar{1}
	
	\cauhoi{Hạt nhân $X$ bền vững hơn hạt nhân $Y$ là vì
		\begin{mcq}
			\item độ hụt khối của $X$ lớn hơn $Y$.	
			\item độ hụt khối của $X$ nhỏ hơn $Y$.	
			\item năng lượng liên kết của $X$ lớn hơn $Y$.	
			\item năng lượng liên kết riêng của $X$ lớn hơn $Y$.
		\end{mcq}}
	
		\loigiai{\textbf{Đáp án: D.}
			
				}
	%4
	\item \mkstar{2}
	
	\cauhoi{Giả sử hai hạt nhân $X$ và $Y$ có độ hụt khối bằng nhau và số nuclôn của hạt nhân $X$ lớn hơn số nuclôn của hạt nhân $Y$ thì
		\begin{mcq}
			\item hạt nhân $Y$ bền vững hơn hạt nhân $X$.	
			\item hạt nhân $X$ bền vững hơn hạt nhân $Y$.	
			\item năng lượng liên kết riêng của $X$ và $Y$ bằng nhau.	
			\item năng lượng liên kết của $X$ và $Y$ khác nhau.
		\end{mcq}}
	
		\loigiai{\textbf{Đáp án: A.}
			
			Do $X$ và $Y$ có độ hụt khối bằng nhau nên năng lượng liên kết bằng nhau: $W_{\text{lk }X} = W_{\text{lk }Y}$.
			
			Mà $A_X > A_Y$ nên $W_{\text{lkr }X} < W_{\text{lkr }Y}$, hạt nhân $Y$ bền vững hơn hạt nhân $X$.
			
				}
	%5
	\item \mkstar{2}
	
	\cauhoi{Hạt nhân $^2_1\text D$ (đơtêri) có khối lượng $m=\SI{2.00136}{u}$. Biết $m_p=\SI{1.0073}{u}$, $m_n=\SI{1.0087}{u}$, $c=\SI{3e8}{m/s}$. Hãy xác định năng lượng liên kết của hạt nhân $\text D$.
		\begin{mcq}(4)
			\item $\SI{1.364}{MeV}$.	
			\item $\SI{1.643}{MeV}$.	
			\item $\SI{13.64}{MeV}$ .	
			\item $\SI{14.64}{MeV}$.
		\end{mcq}}
	
		\loigiai{\textbf{Đáp án: C.}
			
			Năng lượng liên kết:
			
			\begin{equation*}
				W_\text{lk}=\Delta m c^2 \approx \SI{13.64}{MeV} .
			\end{equation*}	
			
				}
	%6
	\item \mkstar{2}
	
	\cauhoi{Hạt nhân $^2_1\text D$ (đơtêri) có khối lượng $m=\SI{2.00136}{u}$. Biết $m_p=\SI{1.0073}{u}$, $m_n=\SI{1.0087}{u}$, $c=\SI{3e8}{m/s}$. Hãy xác định năng lượng liên kết riêng của hạt nhân $\text D$.
		\begin{mcq}(4)
			\item $\SI{1.364}{MeV}$.	
			\item $\SI{6.82}{MeV}$.	
			\item $\SI{13.64}{MeV}$ .	
			\item $\SI{14.64}{MeV}$.
		\end{mcq}}
	
		\loigiai{\textbf{Đáp án: B.}
			
			Năng lượng liên kết riêng:
			
			\begin{equation*}
				W_\text{lkr}=\dfrac{W_\text{lk}}{A}=\dfrac{\Delta m c^2}{A} \approx \SI{6.82}{MeV} .
			\end{equation*}	
			
				}
	%7
	\item \mkstar{2}
	
	\cauhoi{Trong phản ứng hạt nhân $^{A}_{Z} \text{B} \longrightarrow ^{A}_{Z+1} \text{C} + X$ thì $X$ là
		\begin{mcq}(4)
			\item hạt $\alpha$.	
			\item hạt $\beta^{-}$.	
			\item hạt $\beta^{+}$ .	
			\item hạt phôtôn.
		\end{mcq}}
	
		\loigiai{\textbf{Đáp án: B.}
			
			Áp dụng định luật bảo toàn số khối, điện tích, tìm được $X$ có $A=0$, $Z=-1$. Vậy $X$ là hạt $\beta ^-$.
			
				}
	%8
	\item \mkstar{2}
	
	\cauhoi{Trong phản ứng hạt nhân toả năng lượng thì
		\begin{mcq}
			\item khối lượng các hạt ban đầu nhỏ hơn khối lượng các hạt tạo thành.	
			\item độ hụt khối của các hạt ban đầu nhỏ hơn độ hụt khối của các hạt tạo thành.
			\item năng lượng liên kết của các hạt ban đầu lớn hơn năng lượng liên kết của các hạt tạo thành.
			\item năng lượng liên kết riêng của các hạt ban đầu lớn hơn năng lượng liên kết riêng của các hạt tạo thành.
		\end{mcq}}
	
		\loigiai{\textbf{Đáp án: B.}
			
			Trong phản ứng hạt nhân toả năng lượng thì $W>0$, suy ra $m_\text{trước} > m_\text{sau}$, $\Delta m_\text{trước} < \Delta m_\text{sau}$.
			
				}
	%9
	
	\item \mkstar{2}
	
	\cauhoi{Cho phản ứng hạt nhân $^9_4 \text{Be} + \alpha \longrightarrow ^{12}_6 \text C + n$, trong đó khối lượng các hạt tham gia và tạo thành trong phản ứng là $m_{\alpha} = \text{4,0015}\ \text{u}$; $m_\text {Be} = \text{9,0122}\ \text{u}$; $m_\text C =\text{12,0000}\ \text{u}$; $m_n = \text{1,0087}\ \text{u}$ và $1\ \text{u} =\text{931,5}\ \text{MeV/c}^2$. Phản ứng hạt nhân này
		\begin{mcq}(2)
			\item thu vào 4,66 MeV.	
			\item tỏa ra 4,66 MeV.	
			\item thu vào 6,46 MeV.	
			\item tỏa ra 6,46 MeV.
		\end{mcq}}
	
		\loigiai{\textbf{Đáp án: B.}
			
			Năng lượng phản ứng:
			\begin{equation*}
				W=(m_\text{trước}-m_\text{sau})c^2\approx\SI{4.66}{MeV}.
			\end{equation*}
			Vì $W>0$ nên phản ứng toả năng lượng.
			
				}
	%10
	
	\item \mkstar{2}
	
	\cauhoi{Cho phản ứng hạt nhân $^{27}_{13} \text{Al} +\alpha \longrightarrow ^{30}_{15}\text P +n$, trong đó khối lượng các hạt tham gia và tạo thành trong phản ứng là $m_{\alpha} =\text{4,0016}\ \text{u}$; $m_\text{Al} =\text{26,9743}\ \text{u}$; $m_\text P=\text{29,9701}\ \text{u}$; $m_n=\text{1,0087}\ \text{u}$ và $1\ \text{u} =\text{931,5}\ \text{MeV/c}^2$. Phản ứng hạt nhân này
		\begin{mcq}(2)
			\item thu vào 2,7 MeV.	
			\item tỏa ra 2,7 MeV.	
			\item thu vào 4,3 MeV.	
			\item tỏa ra 4,3 MeV.
		\end{mcq}}
	
		\loigiai{\textbf{Đáp án: A.}
			
			Năng lượng phản ứng:
			\begin{equation*}
				W=(m_\text{trước}-m_\text{sau})c^2\approx\SI{-2.7}{MeV}.
			\end{equation*}
			Vì $W<0$ nên phản ứng thu năng lượng.
			
				}
	%11
	
	
	\item \mkstar{2}
	
	\cauhoi{\textbf{[TSĐH 2009]} Cho phản ứng hạt nhân $^3_1\text T+^2_1\text D \longrightarrow ^4_2\text{He} + X$. Lấy độ hụt khối của hạt nhân $\text T$, hạt nhân $\text D$, hạt nhân He lần lượt là 0,009106 u; 0,002491 u; 0,030382 u và $1\ \text{u} =\text{931,5}\ \text{MeV/c}^2$. Năng lượng tỏa ra của phản ứng xấp xỉ bằng
		\begin{mcq}(4)
			\item 21,076 MeV.	
			\item 200,025 MeV.	
			\item 17,498 MeV.	
			\item 15,017 MeV.
		\end{mcq}}
	
		\loigiai{\textbf{Đáp án: C.}
			
			Phản ứng hạt nhân:
			\begin{equation*}
				^3_1\text T+^2_1\text D \longrightarrow ^4_2\text{He} + ^1_0 n
			\end{equation*}
			thì $n$ có độ hụt khối bằng 0.
			
			Năng lượng phản ứng:
			\begin{equation*}
				W=(\Delta m_\text{trước}-\Delta m_\text{sau})c^2\approx - \SI{17.498}{MeV}.
			\end{equation*}
			Vì $W<0$ nên phản ứng thu năng lượng.
			
				}
	%12
	\item \mkstar{2}
	
	\cauhoi{Cho phản ứng hạt nhân $^{235}_{92} \text U + n \longrightarrow ^{94}_{38} \text{Sr} + ^{140}_{54} \text{Xe} + 2n$. Biết năng lượng liên kết riêng của hạt nhân $\text U$ bằng 7,59 MeV; hạt nhân $\text{Sr}$ bằng 8,59 MeV và hạt nhân $\text{Xe}$ bằng 8,29 MeV. Năng lượng tỏa ra của phản ứng là
		\begin{mcq}(4)
			\item 148,4 MeV.	
			\item 144,8 MeV.	
			\item 418,4 MeV.	
			\item 184,4 MeV.
		\end{mcq}}
	
		\loigiai{\textbf{Đáp án: D.}
			
			Năng lượng phản ứng:
			\begin{equation*}
				W=W_\text{lk sau}-\Delta W_\text{lk trước} =W_\text{lkr Sr} \cdot A_\text{Sr} + W_\text{lkr Xe} \cdot A_\text{Xe} -W_\text{lkr U} \cdot A_\text U   \approx\SI{184.4}{MeV}.
			\end{equation*}
			Vì $W>0$ nên phản ứng thu năng lượng.
			
				}
	%13
	\item \mkstar{3}
	
	\cauhoi{Năng lượng liên kết riêng của hạt nhân $^7_3 \ce{Li}$ là $\SI{5.11}{MeV}$. Khối lượng của proton và nơtron lần lượt là $m_p=\SI{1.0073}{u}$, $m_n=\SI{1.0087}{u}$, cho $\SI{1}{u}=\SI{931.5}{MeV/c^2}$. Khối lượng của hạt nhân $^7_3 \ce{Li}$ là
		\begin{mcq}(4)
			\item $\SI{7.0125}{u}$.	
			\item $\SI{7.0383}{u}$.	
			\item $\SI{7.0183}{u}$.	
			\item $\SI{7.0112}{u}$.
		\end{mcq}}
	
		\loigiai{\textbf{Đáp án: C.}
			
			Năng lượng liên kết:
			\begin{equation*}
				W_\text{lk}=W_\text{lkr} A = \SI{35.77}{MeV}.
			\end{equation*}
			
			mà $W_\text{lk}=\Delta m c^2$,
			suy ra $m_\text{Li} \approx \SI{7.0183}{u}$.
			
				}	
	%14
	\item \mkstar{3}
	
	\cauhoi{Cho phản ứng hạt nhân sau $^2_1\text D+^2_1\text D \longrightarrow ^3_2\text{He} + n + \text{3,25}\ \text{MeV}$. Biết độ hụt khối của $^2_1\text{D}$ là $\Delta m_\text D = \text{0,0024}\ \text{u}$; và $1\ \text{u} = 931\ \text{MeV/c}^2$. Năng lượng liên kết của hạt nhân $^3_2\text{He}$ là
		\begin{mcq}(4)
			\item 7,7188 MeV.	
			\item 77,188 MeV.	
			\item 771,88 MeV.	
			\item 7,7188 eV.
		\end{mcq}}
		\loigiai{\textbf{Đáp án: A.}
			
			Năng lượng phản ứng:
			\begin{equation*}
				W=(\Delta m_\text{sau}-\Delta m_\text{trước})c^2 = (\Delta m_\text{He} - 2 \Delta m_\text D) c^2=\SI{3.25}{MeV},
			\end{equation*}
			suy ra $\Delta m_\text{He} \approx \SI{0.00829}{u}$ và $W_\text{lk He}=\Delta m_\text{He}c^2\approx\SI{7.7188}{MeV}$.
			
				}
	%15
	\item \mkstar{3}
	
	\cauhoi{\textbf{[TSĐH 2012]} Tổng hợp hạt nhân heli $^4_2 \text{He}$ từ phản ứng $^1_1 \text H + ^7_3 \text{Li} \longrightarrow ^4_2 \text{He} + X$. Mỗi phản ứng trên tỏa năng lượng 17,3 MeV. Năng lượng tỏa ra khi tổng hợp được 0,5 mol heli là
		\begin{mcq}(4)
			\item 2,6$\cdot 10^{24}$ MeV.	
			\item 2,4$\cdot 10^{24}$ MeV.	
			\item 5,2$\cdot 10^{24}$ MeV.	
			\item 1,3$\cdot 10^{24}$ MeV.
		\end{mcq}}
		\loigiai{\textbf{Đáp án: A.}
			
			Phản ứng hạt nhân:
			\begin{equation*}
				^1_1 \text H + ^7_3 \text{Li} \longrightarrow ^4_2 \text{He} + ^4_2 \text{He}
			\end{equation*}
			Mỗi phản ứng tạo ra 2 hạt heli.
			
			Số hạt heli có trong $\SI{0.5}{mol}$ là
			\begin{equation*}
				N=\dfrac{N_\text A}{2} = \SI{3.01e23}{}
			\end{equation*}
			
			Năng lượng toả ra khi tổng hợp được $\SI{0.5}{mol}$ heli:
			\begin{equation*}
				W=\dfrac{N}{2} \cdot \SI{17.3}{MeV}\approx\SI{2.6e24}{MeV}.
			\end{equation*}	
				}
\end{enumerate}

\loigiai{\textbf{Đáp án}
	\begin{center}
		\begin{tabular}{|m{2.8em}|m{2.8em}|m{2.8em}|m{2.8em}|m{2.8em}|m{2.8em}|m{2.8em}|m{2.8em}|m{2.8em}|m{2.8em}|}
			\hline
			1. C & 2. B & 3. D & 4. A & 5. C & 6. B & 7. B & 8. B & 9. B & 10. A \\
			\hline
			11. C & 12. D & 13. C & 14. A & 15. A &  & &  &  & \\
			\hline
		\end{tabular}
\end{center}}