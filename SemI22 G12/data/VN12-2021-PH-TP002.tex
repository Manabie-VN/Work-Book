\begin{enumerate}[label=\bfseries Câu \arabic*:]
	\item \mkstar{1}
	
	\cauhoi{
		
		Đối với dao động cơ điều hòa của một chất điểm thì khi chất điểm đi đến vị trí biên nó có
		\begin{mcq}
			\item tốc độ bằng không và gia tốc cực đại.
			\item tốc độ bằng không và gia tốc bằng không.
			\item tốc độ cực đại và gia tốc cực đại.
			\item tốc độ cực đại và gia tốc bằng không.
		\end{mcq}
	}
	\loigiai{
		\textbf{Đáp án A.}
		
		Khi chất điểm đến vị trí biên thì động năng của nó bằng 0 nên suy ra vận tốc bằng 0 và gia tốc cực đại.
	}
\item \mkstar{1}

\cauhoi{
	
	Trong phương trình dao động điều hòa $x=A \cos (\omega t + \varphi)$. Chọn câu phát biểu \textbf{sai}.
	\begin{mcq}
		\item Pha ban đầu $\varphi$ chỉ phụ thuộc vào gốc thời gian.
		\item Biên độ $A$ không phụ thuộc vào gốc thời gian.
		\item Tần số góc có phụ thuộc vào các đặc tính của hệ.
		\item Biên độ $A$ không phụ thuộc vào cách kích thích dao động.
	\end{mcq}
}
\loigiai{
	\textbf{Đáp án D.}
	
	Biên độ $A$ phụ thuộc vào cách kích thích dao động.
}
	\item \mkstar{1}
	
	\cauhoi{
		
		Pha của dao động được dùng để xác định
		\begin{mcq}(2)
			\item trạng thái dao động.
			\item biên độ dao động.
			\item chu kì dao động.
			\item tần số dao động.
		\end{mcq}
	}
	\loigiai{
		\textbf{Đáp án A.}
		
		$(\omega t + \varphi)$ - Pha của dao động cho biết trạng thái dao động (gồm li độ $x$ và chiều chuyển động $\vec{v}$).
		
	}
\item \mkstar{1}

\cauhoi{
	
Mối liên hệ giữa tần số góc $\omega$ và tần số $f$ của một dao động điều hòa là
\begin{mcq}(4)
	\item $\omega = \dfrac {f}{2 \pi}$.
	\item $ \omega = \pi f$.
	\item $\omega = 2 \pi f$.
	\item $\omega = \dfrac {1}{2 \pi f}$.
\end{mcq}
}
\loigiai{
	\textbf{Đáp án C.}
	
	Mối liên hệ giữa tần số góc $\omega$ và tần số $f$ của một dao động điều hòa là $\omega = 2 \pi f$.	
	
}
\item \mkstar{1}

\cauhoi{
	
	Phương trình vận tốc của vật là: $v=A\omega \cos \omega t$. Phát biểu nào sau đây là đúng?
	\begin{mcq}
		\item Gốc thời gian lúc vật có li độ $x=-A$.
		\item Gốc thời gian lúc vật có li độ $x=A$.
		\item Gốc thời gian lúc vật đi qua VTCB theo chiều dương.
		\item Gốc thời gian lúc vật đi qua VTCB theo chiều âm.
	\end{mcq}
}
\loigiai{
	\textbf{Đáp án C.}
	
	Với $t_0=0 \Rightarrow v=\omega A\cos 0 = \omega A =v_{\text{max}}>0$ do đó gốc thời gian được chọn lúc vật đi qua VTCB theo chiều dương.
}
\item \mkstar{1}

\cauhoi{
	
	Gia tốc tức thời trong dao động điều hòa biến đổi
	\begin{mcq}(2)
		\item lệch pha$\dfrac{\pi}{4}$ so với li độ.
		\item ngược pha với li độ.
		\item lệch pha vuông góc so với li độ.
		\item cùng pha với li độ.
	\end{mcq}
}
\loigiai{
	\textbf{Đáp án B.}
	
	Ta có:
	\begin{equation*}
		x=A\cos (\omega t + \varphi).
	\end{equation*}
	\begin{equation*}
		a=-A\omega^2 \cos(\omega t + \varphi) = A\omega^2 \cos (\omega t + \varphi -\pi).
	\end{equation*}
	
	Suy ra gia tốc tức thời trong dao động điều hòa biến đổi ngược pha với li độ.	
}
\item \mkstar{1}

\cauhoi{
	
	Li độ và gia tốc trong dao động điều hòa luôn
	
	\begin{mcq}(2)
		\item cùng pha nhau.
		\item lệch pha nhau $\dfrac{\pi}{3}$.
		\item ngược pha nhau.
		\item lệch pha nhau $\dfrac{\pi}{2}$.
	\end{mcq}
}
\loigiai{
	\textbf{Đáp án C.}
	
	Mối liên hệ giữa gia tốc và li độ của một vật dao động điều hòa bất kì là
	\begin{equation*}
		a=-\omega^2x.
	\end{equation*}
	
	Do đó, li độ và gia tốc trong dao động điều hòa luôn ngược pha nhau.
}
	\item \mkstar{2}
	
	\cauhoi{Một vật dao động điều hòa theo phương trình $x=A \cos (2\omega t + \varphi)$. Tần số góc của dao động là
		\begin{mcq}(4)
			\item $\omega t$.
			\item $\omega$.
			\item $2 \omega$.
			\item $2\omega t + \varphi$.
	\end{mcq}}
	\loigiai{\textbf{Đáp án: C.}
		
		Vật dao động theo phương trình $x=A \cos (2\omega t + \varphi)$, do đó tần số góc là $2\omega$.
	}
	\item \mkstar{2}

\cauhoi{Trong các phương trình sau, phương trình nào \textbf{không} biểu thị dao động điều hòa?
	\begin{mcq}(2)
		\item $x=5\cos \pi t + 1\ \text{cm}$.
		\item $x = 3t \cos \left(100\pi t + \dfrac{\pi}{6}\right)\ \text{cm}$. 
		\item $x=2\sin^2 \left(2\pi t +\dfrac{\pi}{6}\right)\ \text{cm}$.
		\item $x=3\sin 5\pi t + 3\cos 5\pi t\ \text{cm}$. 
	\end{mcq}
}
\loigiai{\textbf{Đáp án: B.}
	
	\begin{itemize}
		\item $x=5\cos \pi t + 1\ \text{cm}$: chính xác là hàm điều hòa, trong đó $x_0 =\SI{1}{cm}$ là tọa độ ban đầu.
		\item $x=3\sin 5\pi t + 3\cos 5\pi t\ \text{cm} = 3\sqrt 2 \cos \left(5\pi t -\dfrac{\pi}{4}\right)\ \text{cm}$ cũng là một hàm điều hoà.
		\item $x=2\sin^2 \left(2\pi t +\dfrac{\pi}{6}\right)\ \text{cm} = 1+ \cos \left(4\pi t +\dfrac{\pi}{3}\right)\ \text{cm}$ cũng là một hàm điều hòa.
		\item Loại ba phương án trên, ta thấy B là lựa chọn chính xác, do có biến số $t$ trong biên độ dao động.
	\end{itemize}
}
	\item \mkstar{2}
	
	\cauhoi{
		
		Một vật dao động điều hòa với biên độ $A=\SI{2}{cm}$, tần số $f=\SI{5}{Hz}$. Tại thời điểm ban đầu vật có li độ $x_0 = \SI{-1}{cm}$ và đang chuyển động ra xa vị trí cân bằng. Phương trình dao động của vật có dạng
		
		\begin{mcq}(2)
			\item $x=2\cos \left(10\pi t - \dfrac{2\pi}{3}\right)\ \text{cm}$.			
			\item $x=2\cos \left(10\pi t + \dfrac{2\pi}{3}\right)\ \text{cm}$.	
			\item $x=2\cos \left(10\pi t + \dfrac{\pi}{6}\right)\ \text{cm}$.	
			\item $x=2\cos \left(10\pi t - \dfrac{\pi}{6}\right)\ \text{cm}$.	
		\end{mcq}
	}
	\loigiai{
		\textbf{Đáp án B.}
		
		Ta có 
		\begin{equation*}
			\omega =2\pi f =10\pi\ \text{rad/s}.
		\end{equation*}
		Biên độ 
		\begin{equation*}
			A=\SI{2}{cm}.
		\end{equation*}
		Lúc $t=0$ vật ở vị trí $M_0$ có 
		\begin{equation*}
			x=-\SI{1}{cm}; v<0.
		\end{equation*}
		Từ đường tròn lượng giác suy ra 
		\begin{equation*}
			\varphi = \dfrac{2\pi}{3}.
		\end{equation*}
		Phương trình dao động của vật có dạng
		\begin{equation*}
			x=2\cos \left(10\pi t +\dfrac{2\pi}{3}\right)\ \text{cm}.
		\end{equation*}
	}
\item \mkstar{2}

\cauhoi{Một vật dao động điều hòa có quỹ đạo là một đoạn thẳng dài $\SI{12}{\centi \meter}$. Biên độ dao động của vật là bao nhiêu?
	\begin{mcq}(4)
		\item $\SI{12}{\centi \meter}$.
		\item $\SI{-12}{\centi \meter}$.
		\item $\SI{6}{\centi \meter}$.
		\item $\SI{-6}{\centi \meter}$.
\end{mcq}}
\loigiai{\textbf{Đáp án: C.}

Mối liên hệ giữa biên độ $A$ và chiều dài quỹ đạo $L$ là $A = \dfrac {L}{2} = \dfrac {\SI{12}{\centi \meter}} {2} = \SI{6}{\centi \meter}$.	
	}
	\item \mkstar{2}
	
	\cauhoi{
		
		Một vật dao động điều hòa, biết rằng: Khi vật có li độ $x_1=\SI{6}{cm}$ thì vận tốc của nó là $v_1=80\ \text{cm/s}$; khi vật có ly độ $x_2=5\sqrt 2\ \text{cm}$ thì vận tốc của nó là $v_2=50\sqrt2\ \text{cm/s}$. Tần số góc và biên độ dao động của vật là
		\begin{mcq}(2)
			\item $\omega =10\ \text{rad/s}; A = \SI{10}{cm}$.
			\item $\omega =10\pi \ \text{rad/s}; A = \SI{3,18}{cm}$.
			\item $\omega =8\sqrt 2\ \text{rad/s}; A = \SI{3,14}{cm}$.
			\item $\omega =10\pi\ \text{rad/s}; A = \SI{5}{cm}$.
		\end{mcq}	
	}
	\loigiai{
		\textbf{Đáp án A.}
		
		Ta có: 
		\begin{equation*}
			v^2_1 =\omega^2 (A^2-x^2_1)\ (1).
		\end{equation*}
		và 
		\begin{equation*}
			v^2_2 =\omega^2 (A^2-x^2_2)\ (2).
		\end{equation*}
		Lập tỉ số $(1)$ và $(2)$:
		\begin{equation*}
			\left|\dfrac{v_2}{v_1}\right| =\sqrt {\dfrac{A^2-x^2_2}{A^2-x^2_1}} \Rightarrow A =\SI{10}{cm}.
		\end{equation*}
		Thay vào phương trình (1) suy ra  $\omega =10\ \text{rad/s}$.
	}
	\item \mkstar{2}
	
	\cauhoi{
		
		Một vật dao động điều hòa theo phương trình $x=5\cos \left(2\pi t -\dfrac{\pi}{3}\right)\ \text{cm}$. Vận tốc và gia tốc của vật khi pha dao động của vật có giá trị bằng $\dfrac{17\pi}{6}\ \text{rad}$ là
		\begin{mcq}(2)
			\item $\text{27,2}\ \text{cm/s}$ và $-\text{170,9}\ \text{cm/s}^2$.
			\item $-5\pi\ \text{cm/s}$ và $-\text{170,9}\ \text{cm/s}^2$.
			\item $31\ \text{cm/s}$ và $-\text{30,5}\ \text{cm/s}^2$.
			\item $31\ \text{cm/s}$ và $\text{30,5}\ \text{cm/s}^2$.
		\end{mcq}	
	}
	\loigiai{
		\textbf{Đáp án B.}
		
		Ta có phương trình 
		\begin{equation*}
			x=5\cos \left(2\pi t -\dfrac{\pi}{3}\right)\ \text{cm}.
		\end{equation*}
		Phương trình vận tốc 
		\begin{equation*}
			v=-10\pi \sin \left (2\pi t -\dfrac{\pi}{3}\right)\ \text{cm/s},
		\end{equation*}
		thay pha dao động bằng $\dfrac{17\pi}{6}\ \text{rad}$ vào phương trình vận tốc 
		
		\begin{equation*}
			v=-10\pi \sin \dfrac{17\pi}{6} = -5\pi\ \text{cm/s},
		\end{equation*}
		tương tự đối với phương trình gia tốc 
		\begin{equation*}
			a=-5 \cdot (2\pi)^2 \cos \dfrac{17\pi}{6} =-\text{170,9}\ \text{cm/s}^2.
		\end{equation*}
	}
	\item \mkstar{2}
	
	\cauhoi{
		
		Dao động điều hòa có vận tốc cực đại là $v_\text{max} =8\pi\ \text{cm/s}$ và gia tốc cực đại $a_\text{max} =16\pi^2\ \text{cm/s}^2$ thì tần số góc của dao động là
		\begin{mcq}(4)
			\item $\pi\ \text{rad/s}$.
			\item $2\pi\ \text{rad/s}$.
			\item $\dfrac{\pi}{2}\ \text{rad/s}$.
			\item $\xsi{2\pi}{Hz}$.
		\end{mcq}	
	}	
	\loigiai{
		\textbf{Đáp án B.}
		
		Ta có:
		\begin{equation*}
			v_\text{max} = \omega A =8\pi\ \text{cm/s}\ \text{và}\ a_\text{max} =\omega^2 A = 16\pi^2\ \text{cm/s}^2.
		\end{equation*}
		Suy ra:
		\begin{equation*}
			\omega = \dfrac{a_\text{max}}{v_\text{max}} = 2\pi\ \text{rad/s}.
		\end{equation*}
	}
	\item \mkstar{2}

\cauhoi{Một chất điểm dao động điều hòa với chu kỳ $\text{0,5}\pi\ \text{s}$ và biên độ $\SI{2}{cm}$. Vận tốc của chất điểm tại vị trí cân bằng có độ lớn bằng
	\begin{mcq}(4)
		\item $\SI{3}{\centi\meter/\second}$.
		\item $\SI{0,5}{\centi\meter/\second}$.
		\item $\SI{4}{\centi\meter/\second}$.
		\item $\SI{8}{\centi\meter/\second}$.
	\end{mcq}
}
\loigiai{\textbf{Đáp án: D.}
	
	Tốc độ cực đại là
	\begin{equation*}
		v_\text{max}=\omega A=\dfrac{2\pi}{T}A=\SI{8}{\centi\meter/\second}.
	\end{equation*}
}
	\item \mkstar{2}
	
	\cauhoi{
		
		Tại thời điểm khi thực hiện dao động điều hòa với vận tốc bằng $\dfrac{1}{2}$ vận tốc cực đại. Vật xuất hiện tại li độ bằng bao nhiêu?
		\begin{mcq}(4)
			\item $A\dfrac{\sqrt 3}{2}$.
			\item $A\sqrt 2$.
			\item $\dfrac{A}{\sqrt 3}$.
			\item $\dfrac{A}{\sqrt 2}$.
		\end{mcq}	
	}
	\loigiai{
		\textbf{Đáp án A.}
		
		Ta có: 
		\begin{equation*}
			v=\dfrac{v_\text{max}}{2} =\dfrac{\omega A}{2},
		\end{equation*}
		mà 
		\begin{equation*}
			v^2 = \omega^2 (A^2 -x^2).
		\end{equation*}
		Thay số vào ta có:
		\begin{equation*}
			x = \pm \dfrac{A\sqrt 3}{2}.
		\end{equation*}
	}
	\item \mkstar{2}
	
	\cauhoi{
		
		Xét một con lắc dao động điều hòa. Biết rằng mỗi phút con lắc thực hiện 360 dao động. Tần số dao động của con lắc là
		\begin{mcq}(4)
			\item $\dfrac{1}{6}\ \text{Hz}$.
			\item $60\ \text{Hz}$.
			\item $6\ \text{Hz}$.
			\item $120\ \text{Hz}$.
		\end{mcq}	
	}
	\loigiai{
		\textbf{Đáp án C.}
		
		Một phút thực hiện 360 dao động $n=\dfrac{t}{T} \Leftrightarrow T =\dfrac{1}{6}\ \text{s}$.
		
		Tần số dao động $f= \dfrac{1}{T} = \SI{6}{Hz}$.
	}
	\item \mkstar{2}

\cauhoi{	Vật $m$ dao động điều hòa với phương trình $x=20\cos 2\pi t\ \text{cm}$. Gia tốc tại li độ $\SI{10}{cm}$ là
	\begin{mcq}(4)
		\item $\SI{-4}{\meter/\second^2}$.
		\item $\SI{-2}{\meter/\second^2}$.
		\item $\SI{9,8}{\meter/\second^2}$.
		\item $\SI{10}{\meter/\second^2}$.
	\end{mcq}
}
\loigiai{\textbf{Đáp án: A.}
	
	Ta có:
	\begin{equation*}
		a=-\omega^2 x=\SI{-400}{\centi\meter/\second^2}=\SI{-4}{\meter/\second^2}.
	\end{equation*}
}
	\item \mkstar{2}

\cauhoi{	Một chất điểm dao động điều hòa theo phương trình $x=5\cos (2\pi t)\ \text{cm}$. Chu kỳ dao động của chất điểm là
	
	\begin{mcq}(4)
		\item $\SI{1}{\second}$.
		\item $\SI{2}{\second}$. 
		\item $\SI{0,5}{\second}$.
		\item $\SI{1,5}{\second}$.
	\end{mcq}
}
\loigiai{\textbf{Đáp án: A.}
	
	Vật dao động điều hòa theo phương trình $x=5\cos (2\pi t)\ \text{cm}$ nên tần số góc của dao động là $\omega=2\pi\,\text{rad/s}$.
	
	Chu kì của dao động là
	\begin{equation*}
		\omega=\dfrac{2\pi}{T}\Rightarrow T=\dfrac{2\pi}{\omega}=\dfrac{2\pi}{2\pi\,\text{rad/s}}=\SI{1}{\second}.
	\end{equation*}
	
}
	\item \mkstar{3}
	
	\cauhoi{
		
		Một vật dao động điều hòa với phương trình $x=4\cos \left(4\pi t + \dfrac{\pi}{3}\right)$. Tính quãng đường lớn nhất mà vật đi được trong khoảng thời gian $\dfrac{1}{6}\ \text{s}$.
		\begin{mcq}(4)
			\item $\sqrt 3\ \text{cm}$.
			\item $2\sqrt 3\ \text{cm}$.
			\item $3\sqrt 3\ \text{cm}$.
			\item $4\sqrt 3\ \text{cm}$.
		\end{mcq}	
	}
	\loigiai{
		\textbf{Đáp án D.}
		
		Ta có $T=\dfrac{\omega}{2\pi} =\SI{0,5}{s}$; thời gian chuyển động $\Delta t = \dfrac{1}{6}\ \text{s} < \dfrac{T}{2}$.
		
		Trong thời gian $\Delta t = \dfrac{1}{6}\ \text{s}$ thì góc quét $\Delta \varphi =\omega \Delta t =\dfrac{2\pi}{3}\ \text{rad}$.
		
		Để vật đi được quãng đường lớn nhất thì $\Delta \varphi$ phải đối xứng qua trục tung. Từ đường tròn lượng giác
		
		$S'_\text{max} =\dfrac{A\sqrt 3}{2}+\dfrac{A\sqrt 3}{2} = A\sqrt 3= 4\sqrt 3\ \text{cm}$
	}
	
	\item \mkstar{3}
	
	\cauhoi{
		
		Một vật dao động điều hòa với phương trình $x=4\cos \left(4\pi t + \dfrac{\pi}{3}\right)$. Tính quãng đường nhỏ nhất mà vật đi được trong khoảng thời gian $\dfrac{1}{6}\ \text{s}$.
		\begin{mcq}(4)
			\item $2(4-2\sqrt 3)\ \text{cm}$.
			\item $2\sqrt 3\ \text{cm}$.
			\item $\SI{4}{cm}$.
			\item $4\sqrt 3\ \text{cm}$.
		\end{mcq}	
	}
	\loigiai{
		\textbf{Đáp án C.}
		
		Ta có $T=\dfrac{\omega}{2\pi} =\SI{0,5}{s}$.
		
		Thời gian chuyển động $\Delta t =\dfrac{1}{6}\ \text{s} <\dfrac{T}{2}$.
		
		Trong thời gian $\Delta t = \dfrac{1}{6}\ \text{s}$ thì góc quét $\Delta \varphi =\omega \Delta t =\dfrac{2\pi}{3}\ \text{rad}$.
		
		Để vật đi qua quãng đường nhỏ nhất thì $\Delta \varphi$ phải đối xứng qua trục hoành. Từ đường tròn lượng giác
		
		$S'_\text{min} = \dfrac{A}{2} + \dfrac{A}{2} =A = \SI{4}{cm}$.
	}
\item \mkstar{3}

\cauhoi{
	
	Một vật dao động điều hòa với biên độ $A=\SI{5}{cm}$. Trong 10 giây vật thực hiện được 20 dao động. Xác định phương trình dao động của vật biết rằng tại thời điểm ban đầu vật qua vị trí cân bằng theo chiều dương.
	
	\begin{mcq}(2)
		\item $x= 5\cos \left (4\pi t +\dfrac{\pi}{2}\right)\ \text{cm}$.
		\item $x= 5\cos \left (4\pi t -\dfrac{\pi}{2}\right)\ \text{cm}$.
		\item $x= 5\cos \left (2\pi t +\dfrac{\pi}{2}\right)\ \text{cm}$.
		\item $x= 5\cos \left (2\pi t +\dfrac{\pi}{2}\right)\ \text{cm}$.
	\end{mcq}
}
\loigiai{
	\textbf{Đáp án B.}
	
	Ta có
	\begin{equation*}
		f = \dfrac{N}{t} = \SI{2}{Hz} \Rightarrow \omega =2\pi f = 4\pi\ \text{rad/s}.
	\end{equation*}
	
	Biên độ dao động
	\begin{equation*}
		A = \SI{5}{cm}.
	\end{equation*}
	Khi $t=0$ vật đang ở vị trí cân bằng theo chiều dương.
	
	\begin{equation*}
		x=5\cos \varphi =0; v>0.
	\end{equation*}
	
	Suy ra
	\begin{equation*}
		\varphi = - \dfrac{\pi}{2}.
	\end{equation*}
	
	Phương trình dao động của vật là: 
	
	\begin{equation*}
		x=5\cos \left(4\pi t - \dfrac{\pi}{2}\right)\ \text{cm}.
	\end{equation*}
}
\item \mkstar{3}

\cauhoi{
	
	Một chất điểm dao động điều hòa trên trục Ox. Trong thời gian $\SI{31,4}{s}$ chất điểm thực hiện được 100 dao động toàn phần. Gốc thời gian là lúc chất điểm đi qua vị trí có li độ $\SI{2}{cm}$ theo chiều âm với tốc độ là $40\sqrt 3\ \text{cm/s}$. Lấy $\pi = \text{3,14}$. Phương trình dao động của chất điểm là
	
	\begin{mcq}(2)
		\item $x= 6\cos \left (20t -\dfrac{\pi}{6}\right)\ \text{cm}$.
		\item $x= 4\cos \left (20t +\dfrac{\pi}{3}\right)\ \text{cm}$.
		\item $x= 4\cos \left (20t -\dfrac{\pi}{3}\right)\ \text{cm}$.
		\item $x= 6\cos \left (20t +\dfrac{\pi}{6}\right)\ \text{cm}$.
	\end{mcq}
}
\loigiai{
	\textbf{Đáp án B.}
	
	Ta có
	\begin{equation*}
		T =\dfrac{\Delta t}{n} = \SI{0,314}{s} \Rightarrow \omega =20\ \text{rad/s}.
	\end{equation*}
	
	Biên độ dao động
	\begin{equation*}
		A = \sqrt {x^2_0 + \left(\dfrac{v_0}{\omega}\right)^2} = \SI{4}{cm}.
	\end{equation*}
	Khi $t=0$ thì $x_0 = \pm \SI{2}{cm}$ và $v<0 \Rightarrow$ pha ban đầu
	\begin{equation*}
		\varphi_0 =\dfrac{\pi}{3}.
	\end{equation*}
	
	Phương trình dao động của chất điểm là 
	\begin{equation*}
		x = 4 \cos \left(20t + \dfrac{\pi}{3}\right)\ \text{cm}.
	\end{equation*}
}
	\item \mkstar{3}
	
\cauhoi{
	
	Một chất điểm dao động điều hòa có phương trình $x=6 \cos \left(10t - \dfrac{\pi}{2}\right)\ \text{cm}$. Vận tốc của chất điểm có phương trình
	\begin{mcq}(2)
		\item $v=-60 \cos (10t)\ \text{cm/s}.$
		\item $v= 60 \cos \left(10t - \dfrac{\pi}{2}\right)\ \text{cm/s}.$
		\item $v= 60 \cos (10t)\ \text{cm/s}.$
		\item $v= 60 \cos \left(10t + \dfrac{\pi}{2}\right)\ \text{cm/s}.$
	\end{mcq}
}
\loigiai{
	\textbf{Đáp án C.}
	
	Phương trình vận tốc 
	\begin{equation*}
		v=x'=-60\sin \left (10t-\dfrac{\pi}{2}\right) = 60 \cos (10t)\ \text{cm/s}.
	\end{equation*}
}
	\item \mkstar{3}
	
	\cauhoi{
		
		Vật dao động điều hòa với phương trình $x=10\cos \left(4\pi t - \dfrac{\pi}{6}\right)\ \text{cm}$. Tính quãng đường vật đi được từ $t=0$ đến $t=\dfrac{5}{6}\ \text{s}$.
		\begin{mcq}(4)
			\item $\SI{26,68}{cm}$.
			\item $\SI{68,62}{cm}$.
			\item $\SI{82,68}{cm}$.
			\item $\SI{62,68}{cm}$.
		\end{mcq}
	}
	\loigiai{
		\textbf{Đáp án D.}
		
		Ta có: $T =\SI{0,5}{s}$; $\Delta t=\dfrac{5}{6} =\dfrac{5}{3}T=T+\dfrac{2}{3}T$.
		
		Suy ra: $S =4A +S'$.
		
		Tại $t=0$ ta có $x_1 =5\sqrt 3; v>0$ ứng với vị trí M$_0$.
		
		Tại $t=\dfrac{5}{6}\ \text{s}$ ta có $x_2 =-5\sqrt 3; v>0$ ứng với vị trí M.
		
		Quãng đường đi của vật đi được $S= 4\cdot 10 + (10-5\sqrt 3) + 20 + (10-5\sqrt 3) \approx \SI{62,68}{cm}$.
	}
	
	\item \mkstar{3}
	
	\cauhoi{
		
		Một vật dao động điều hòa theo phương trình $x=10\cos \left (10\pi t + \dfrac{\pi}{2}\right)\ \text{cm}$. Xác định thời điểm vật qua vị trí $x=\SI{5}{cm}$ lần thứ 2008
		\begin{mcq}(4)
			\item $\SI{208}{s}$.
			\item $\SI{201}{s}$.
			\item $\SI{207}{s}$.
			\item $\SI{205}{s}$.
		\end{mcq}
	}
	\loigiai{
		\textbf{Đáp án B.}
		
		Ta có: $5 =10 \cos \left (10\pi t +\dfrac{\pi}{2}\right) \Leftrightarrow \cos \left(10\pi t +\dfrac{\pi}{2}\right) = \dfrac{1}{2} = \cos \left(\pm \dfrac{\pi}{3}\right)$.
		
		$\Rightarrow 10\pi t +\dfrac{\pi}{2} = \pm \dfrac{\pi}{3} + k2\pi$.
		
		Vì $t>0$ nên khi vật qua vị trí $x=\SI{5}{cm}$ lần thứ 2008 ứng với $k=1004$.
		
		Vậy $t = -\dfrac{1}{60} +\dfrac{k}{5} = \SI{201}{s}$.
		
	}
\item \mkstar{3}

\cauhoi{
	
	Phương trình dao động của chất điểm là $x=A\cos(\omega t+\varphi)$. Biểu thức gia tốc của chất điểm là
	\begin{mcq}(2)
		\item $a=\omega A\cos\left(\omega t+\varphi\right)$.
		\item $a=\omega^2 A\cos\left(\omega t+\varphi\right)$.
		\item $a=-\omega A\cos\left(\omega t+\varphi\right)$.
		\item $a=-\omega^2 A\cos\left(\omega t+\varphi\right)$.
	\end{mcq}
}
\loigiai{
	\textbf{Đáp án D.}
	
	Đạo hàm bậc hai phương trình li độ là phương trình gia tốc trong dao động điều hòa
	\begin{equation*}
		a=x''=-\omega^2 A\cos\left(\omega t+\varphi\right);
	\end{equation*}
}
\item \mkstar{3}

\cauhoi{
	
	Phương trình dao động của chất điểm là $x=2\cos\left( 2\pi t+\dfrac{\pi}{2}\right)$ ($x$ tính bằng cm, $t$ tính bằng s). Lấy $\pi^2=10$. Biểu thức gia tốc của chất điểm là
	\begin{mcq}(2)
		\item $a=80\cos\left( 2\pi t+\dfrac{\pi}{2}\right)\ \text{cm/s}^2$.
		\item $a=60\sin\left( 2\pi t+\dfrac{\pi}{2}\right)\ \text{cm/s}^2$.
		\item $a=-80\cos\left( 2\pi t+\dfrac{\pi}{2}\right)\ \text{cm/s}^2$.
		\item $a=60\sin\left( 2\pi t+\dfrac{\pi}{2}\right)\ \text{cm/s}^2$.
	\end{mcq}
}
\loigiai{
	\textbf{Đáp án C.}
	
	Đạo hàm bậc hai phương trình li độ là phương trình gia tốc trong dao động điều hòa
	\begin{equation*}
		a=x''=-\omega^2 A\cos\left(\omega t+\varphi\right)=-(2\pi)^2\cdot 2\cos\left( 2\pi t+\dfrac{\pi}{2}\right)=-80\cos\left( 2\pi t+\dfrac{\pi}{2}\right)\ \text{cm/s}^2.
	\end{equation*}
}
	\item \mkstar{3}
	
	\cauhoi{
		
		Vật dao động điều hòa theo phương trình $x=5\cos \pi t\ \text{cm}$ sẽ qua vị trí cân bằng lần thứ ba (kể từ lúc $t=0$) vào thời điểm nào?
		\begin{mcq}(4)
			\item $\SI{2,5}{s}$.
			\item $\SI{4,5}{s}$.
			\item $\SI{2}{s}$.
			\item $\SI{6,5}{s}$.
		\end{mcq}
	}
	\loigiai{
		\textbf{Đáp án A.}
		
		Ta có: $0 =5\cos \pi t \Rightarrow \cos \pi t =0 \Rightarrow \pi t = \dfrac{\pi}{2}+k\pi \Rightarrow t =\dfrac{1}{2}+ k.$
		
		Vì $t>0$ nên $k=0,1,2,3,...$
		
		Vật qua vị trí cân bằng lần thứ 3 ứng với $k=2$.
		
		Vậy $t = \dfrac{1}{2} +2 =\SI{2,5}{s}$.
		
		
		
	}
	
	\item \mkstar{3}
	
	\cauhoi{
		
		Một chất điểm dao động điều hòa theo phương trình $x=4\cos \dfrac{2\pi}{3}t\ \text{cm}$ ($x$ tính bằng $\text{cm}$; $t$ tính bằng s). Kể từ $t=0$, chất điểm đi qua vị trí có li độ $x=\SI{-2}{cm}$ lần thứ 2017 tại thời điểm
		\begin{mcq}(4)
			\item $\SI{3026}{s}$.
			\item $\SI{5230}{s}$.
			\item $\SI{3025}{s}$.
			\item $\SI{5231}{s}$.
		\end{mcq}
	}
	\loigiai{
		\textbf{Đáp án C.}
		
		Chu kì dao động $\omega =\dfrac{2\pi}{3}=  \dfrac{2\pi}{T} \Rightarrow T=\SI{3}{s}.$
		
		Trong một chu kì, chất điểm đi qua vị trí có li độ $x=-\SI{2}{cm}$ tất cả 2 lần nên $n_0=2$.
		
		Theo đề bài: $n=2017$ (lẻ) nên $m=1$, khi đó: 
		
		- Tại $t=0$; $x_0 = 4\cos \left(\dfrac{2\pi}{3}\cdot 0\right) =\SI{4}{cm}$. 
		
		- Khoảng thời gian từ vị trí ban đầu qua vị tró $x$ lần thứ 1 suy ra $t'=\dfrac{T}{12} + \dfrac{T}{4} =\dfrac{T}{3}.$
		
		Thời điểm vật qua vị trí $x=-\SI{2}{cm}$ lần thứ $n=2017$:
		
		$t=\dfrac{n-m}{n_0} \cdot T + t' = \dfrac{2017 -1}{2}T +\dfrac{T}{3} =\dfrac{3025}{3}T =\SI{3025}{s}$.
	}
	
	\item \mkstar{3}
	
	\cauhoi{
		
		Một chất điểm dao động điều hòa theo phương trình $x=2\cos (\pi t + \pi)\ \text{cm}$. Thời gian ngắn nhất vật đi từ lúc bắt đầu dao động đến lúc vật có li độ $x=\sqrt 3\ \text{cm}$ là
		\begin{mcq}(4)
			\item $\SI{2,4}{s}$.
			\item $\SI{1,2}{s}$.
			\item $\dfrac{5}{6}\ \text{s}$.
			\item $\dfrac{5}{12}\ \text{s}$.
		\end{mcq}
	}
	\loigiai{
		\textbf{Đáp án C.}
		
		Tại thời điểm $t=0 \Rightarrow x =-\SI{2}{cm}$.
		
		Khoảng thời gian ngắn nhất ứng với vật chuyển động qua vị trí $x=\sqrt 3\ \text{cm}$ theo chiều dương lần đầu tiên. 
		
		Khoảng thời gian tương ứng $t=\dfrac{\Delta \varphi}{\omega} = \dfrac{\dfrac{\pi}{2} + \dfrac{\pi}{3}}{\pi} = \dfrac{5}{6}\ \text{s}$.
	}
	\item \mkstar{3}
	
	\cauhoi{
		
		Vật dao động điều hòa theo phương trình $x=-5\cos 10\pi t\ \text{cm}$. Thời gian vật đi quãng đường dài $\SI{12,5}{cm}$ kể từ lúc bắt đầu chuyển động là
		
		\begin{mcq}(4)
			\item $\dfrac{1}{15}\ \text{s}$.
			\item $\dfrac{2}{15}\ \text{s}$.
			\item $\dfrac{1}{30}\ \text{s}$.
			\item $\dfrac{1}{12}\ \text{s}$.
		\end{mcq}
	}	
	\loigiai{
		\textbf{Đáp án B.}
		
		Ta có quãng đường $S=\SI{12,5}{cm} = \dfrac{5A}{2} = 2A+\dfrac{A}{2} \Rightarrow \Delta \varphi = \pi + \varphi_{\dfrac{A}{2}}$.
		
		Tại $t=0$, ta có $x=5\cos \pi =-5$; $v=0 \Rightarrow$ để đi được quãng đường $\dfrac{A}{2}$ vật quét một góc $\dfrac{\pi}{3}$.
		
		Vậy tổng góc quét $\Delta \varphi =\pi + \dfrac{\pi}{3} = \dfrac{4\pi}{3} \Rightarrow \Delta t =\dfrac{\Delta \varphi}{\omega} = \dfrac{2}{15}\ \text{s}$.
		
	} 
	
	\item \mkstar{3}
	
	\cauhoi{
		
		Một chất điểm dao động điều hòa theo phương trình $x=4\cos 5\pi t\ \text{cm}$. Thời giạn ngắn nhất vật đi từ lúc bắt đầu chuyển động đến khi vật đi được quãng đường $\SI{6}{cm}$ là
		\begin{mcq}(4)
			\item $\SI{0,15}{s}$.
			\item $\dfrac{2}{15}\ \text{s}$.
			\item $\SI{0,2}{s}$.
			\item $\SI{0,3}{s}$.
		\end{mcq}
		
	}
	\loigiai{
		\textbf{Đáp án B.}
		
		Tại thời điểm $t=0$ vật đang ở vị trí biên dương, ta thấy $\SI{6}{cm} =\text{1,5} A$. Vậy từ thời điểm $t=0$ vật đi được quãng đường $\SI{6}{cm}$ khi vật đến vị trí có ly độ $x=-\dfrac{A}{2}$ lần đầu tiên. 
		
		Thời gian ngắn nhất từ khi bắt đầu dao động đến khi vật đi được quãng đường là $\SI{6}{cm}$ là $\Delta t =\dfrac{T}{3} = \dfrac{2}{15}\ \text{s}$.
	}
	
	\item \mkstar{4}
	
	\cauhoi{
		
		Hai vật dao động điều hòa dọc theo các trục song song với nhau. Phương trình dao động của các vật lần lượt là: $x_1= 3\cos \left(5\pi t-\dfrac{\pi}{3}\right)$ và $x_2=
		\sqrt 3 \cos \left(5\pi t- \dfrac{\pi}{6}\right)$ ($x$ tính bằng cm; $t$ tính bằng s). Trong khoảng thời gian $\SI{1}{s}$ đầu tiên thì hai vật gặp nhau mấy lần?
		\begin{mcq}(4)
			\item 2.
			\item 3.
			\item 5.
			\item 6.
		\end{mcq}
	}
	\loigiai{
		\textbf{Đáp án D.}
		
		Ta thấy hai vật gặp nhau tại thời điểm ban đầu $t_1=0$:
		
		$x_1 = 3\cos \left(\dfrac{\pi}{3}\right) = \dfrac{3}{2}$.
		
		$x_2=\sqrt 3\cos \left(-\dfrac{\pi}{3}\right) = \dfrac{3}{2}$.
		
		Chu kì $T=\dfrac{2\pi}{\omega}=\SI{0,4}{s}$.
		
		Trong $\SI{1}{s}$ có: $t=(n-1)\dfrac{T}{2}+t_1 \Rightarrow n=6\ \text{lần}$.
	}
	\item \mkstar{4}
	
	\cauhoi{
		
		Một chất điểm dao động điều hòa theo phương trình $x=10\cos\left(5\pi t-\dfrac{\pi}{3}\right)$ ($x$ tính bằng cm, $t$ tính bằng s). Sau khoảng thời gian $\SI{4,2}{\second}$ kể từ $t=0$ chất điểm đi qua vị trí có li độ $x=\SI{-5}{\centi\meter}$ bao nhiêu lần?
		\begin{mcq}(4)
			\item 20 lần.
			\item 10 lần. 
			\item 21 lần.
			\item 11 lần.
		\end{mcq}
	}
	\loigiai{
		\textbf{Đáp án C.}
		
		Chu kì chất điểm dao động điều hòa là
		\begin{equation*}
			T=\dfrac{2\pi}{\omega}=\dfrac{2\pi}{5\pi\,\text{rad/s}}=\SI{0,4}{\second}.
		\end{equation*}
		
		Phân tích $\SI{4,2}{\second}$ theo chu kì $T$ ta được 
		\begin{equation*}
			\SI{4,2}{\second}=10T+\SI{0,2}{\second}.
		\end{equation*}
		
		Vì chất điểm dao động với chu kì $A=\SI{10}{\centi\meter}$ nên vị trí $x=\SI{-5}{\centi\meter}$ không phải là vị trí biên. Do đó, mỗi chu kì, số lần chất điểm dao động điều hòa qua vị trí có li độ $x=\SI{-5}{\centi\meter}$ là 2 lần.
		
		Số lần chất điểm đi qua vị trí có li độ $x=\SI{-5}{\centi\meter}$ là
		\begin{equation*}
			N=20+m;
		\end{equation*}
		với $m$ là số lần vật đi qua vị trí $x_0=\SI{5}{\centi\meter}$ trong thời gian $\SI{0,2}{\second}$ kể từ $t=0$.
		
		Để xác định $m$ ta căn cứ vào đường tròn lượng giác.
		
		Góc quay vật dao động điều hòa trong khoảng thời gian $\SI{0,2}{\second}$ là
		\begin{equation*}
			\Delta\alpha=\omega t=5\pi\,\text{rad/s}\cdot\SI{0,2}{\second}=\pi\,\text{rad}.
		\end{equation*}
		
		Góc quay vật dao động điều hòa từ vị trí ban đầu $x_0=\SI{5}{\centi\meter}$ đến vị trí $x=\SI{-5}{\centi\meter}$ là $\pi$.
		
		Dựa vào đường tròn lượng giác ta suy ra trong khoảng thời gian $\SI{0,2}{\second}$, kể từ từ vị trí ban đầu $x_0=\SI{-5}{\centi\meter}$ vật đi qua vị trí  $x=\SI{5}{\centi\meter}$ 1 lần. Do đó, $m=1$.
		
		Vậy số lần chất điểm đi qua vị trí có li độ $x=\SI{-5}{\centi\meter}$ kể từ $t=0$ là
		\begin{equation*}
			N=20+1=21.
		\end{equation*}
	}
	\item \mkstar{4}

\cauhoi{
	
	Một chất điểm dao động điều hòa theo phương trình $x=3 \sin\left(5\pi t +\dfrac{\pi}{6}\right)\ \text{cm}$ ($x$ tính bằng cm và $t$ tính bằng giây). Trong một giây đầu tiên từ thời điểm $t=\SI{0,2}{s}$, chất điểm đi qua vị trí có li độ $x=+ \SI{1}{cm}$ là
	\begin{mcq}(4)
		\item 7 lần. 	       		
		\item 6 lần. 		
		\item 4 lần. 			
		\item 5 lần.
	\end{mcq}
}
\loigiai{
	\textbf{Đáp án D.}
	
	Ta có
	\begin{equation*}
		x=3\sin \left (5\pi t +\dfrac{\pi}{6}\right) = 3 \cos \left (5\pi t + \dfrac{\pi}{6}- \dfrac{\pi}{2}\right) = 3\cos \left (5\pi t - \dfrac{\pi}{3}\right)\ \text{cm}.
	\end{equation*}
	Chu kỳ dao động: 
	\begin{equation*} 
		T =\dfrac{2\pi}{\omega} =\SI{0,4}{s}.
	\end{equation*}
	Tại $t =\SI{0,2}{s}$:
	
	\begin{equation*}
		x=3\cos \left(5\pi \text{0,2} -\dfrac{\pi}{3}\right).
	\end{equation*}
	
	\begin{equation*}
		v= - A\omega \sin \left(5\pi \text{0,2}-\dfrac{\pi}{3}\right).
	\end{equation*}
	Suy ra:
	\begin{equation*}
		x= -\SI{1,5}{cm}; v<0.
	\end{equation*}
	Ta có: 
	\begin{equation*}
		\SI{1}{s} = 2T+\dfrac{T}{2}.
	\end{equation*}
	Trong một chu kỳ, vật đi qua vị trí $+\SI{1}{cm}$ 2 lần.
	
	Trong khoảng thời gian $\dfrac{T}{2}$ vật qua vị trí $+ \SI{1}{cm}$ 1 lần kể từ $t=\SI{0,2}{s}$.
	
	Vậy trong $\SI{1}{s}$ đầu tiên kể từ $t =\SI{0,2}{s}$, vật qua vị trí $+ \SI{1}{cm}$ số lần là $2 \cdot 2 + 1 = 5$ lần.
}
	\item \mkstar{4}
	
	\cauhoi{
		
		
		Hai con lắc lò xo đặt cạnh nhau, song song với nhau trên mặt phẳng nằm ngang có chu kì dao động lần lượt là $\SI{1,4}{s}$ và $\SI{1,8}{s}$. Kéo các quả cầu con lắc ra khỏi vị trí cân bằng một đoạn như nhau rồi đồng thời buông nhẹ thì hai con lắc đồng thời trở lại vị trí này sau thời gian ngắn nhất bằng
		\begin{mcq}(4)
			\item $\SI{8,8}{s}$.
			\item $\SI{12,6}{s}$.
			\item $\SI{6,3}{s}$.
			\item $\SI{24}{s}$.
		\end{mcq}
	}
	\loigiai{
		\textbf{Đáp án B.}
		
		Gọi $x_1 = A\cos \dfrac{10\pi}{7}t$ và $x_2=A\cos \dfrac{10\pi}{9}t$.
		
		Hai con lắc trở về trạng thái ban đầu thì:
		
		$x_1=x_2=A$.
		
		Khi đó $\dfrac{10\pi}{7}t =k_12\pi$ và $\dfrac{10\pi}{9} t =k_2 2\pi$.
		
		Do 2 con lắc được kích thích đồng thời nên trạng thái lặp lại đầu tiên thì thời gian chúng dao động là như nhau. Do đó:
		
		$\dfrac{k_1}{k_2}=\dfrac{9}{7}$.
		
		Thời điểm đầu tiên nên chọn $k_1=9$; $k_2=7$.
		
		$t=9 \cdot \text{1,4} = \SI{12,6}{s}$.
	}
	
	\item \mkstar{4}
	
	\cauhoi{
		
		Một vật dao động theo phương trình $x=4\cos \left(10\pi t - \dfrac{\pi}{6}\right)\ \text{cm}$. Thời điểm vật đi qua vị trí có vận tốc $20\pi \sqrt 2\ \text{cm/s}$ lần thứ 2012 là
		
		\begin{mcq}(4)
			\item $\SI{201,19}{s}.$
			\item $\SI{201,11}{s}$.
			\item $\SI{201,12}{s}$.
			\item $\SI{201,21}{s}$.
		\end{mcq}
		
	}
	\loigiai{
		\textbf{Đáp án A.}
		
		Ta có lúc $t=0$ vật ở vị trí $M_0$.
		
		Từ đường tròn lượng giác. Trong 1 chu kỳ vật đi qua $v=20\pi \sqrt 2\ \text{cm/s}$ là 2 lần.
		
		Để đi qua vị trí có vận tốc $v=20\pi \sqrt 2\ \text{cm/s}$  lần thứ 2012 thì
		
		$t=1006T - \left(\dfrac{T}{6}- \dfrac{T}{8}\right)= \SI{201,19}{s}$.
	}
	\item \mkstar{4}
	
	\cauhoi{
		
		Một chất điểm dao động điều hòa theo phương trình $x=4\cos\dfrac{2\pi}{3}t$ ($x$ tính bằng cm; $t$ tính bằng s). Kể từ $t=0$, chất điểm đi qua vị trí có li độ $x=\SI{-2}{\centi\meter}$ lần thứ 2011 tại thời điểm 
		\begin{mcq}(4)
			\item $\SI{3015}{\second}$.
			\item $\SI{6030}{\second}$.
			\item $\SI{3016}{\second}$.
			\item $\SI{6031}{\second}$.
		\end{mcq}
		
	}
	\loigiai{
		\textbf{Đáp án C.}
		
		Chu kì chất điểm dao động điều hòa là
		\begin{equation*}
			T=\dfrac{2\pi}{\omega}=\dfrac{2\pi}{\dfrac{2\pi}{3}\,\text{rad}/\text{s}}=\SI{3}{\second}.
		\end{equation*}
		
		Chất điểm dao động với chu kì $A=\SI{4}{\centi\meter}$ nên vị trí $x=\SI{-2}{\centi\meter}$ không phải là vị trí biên. Do đó, mỗi chu kì, số lần chất điểm dao động điều hòa qua vị trí có li độ $x=\SI{-2}{\centi\meter}$ là 2 lần.
		
		Thời điểm chất điểm dao động điều hòa qua vị trí $x=\SI{-2}{\centi\meter}$ lần thứ 2011 là
		\begin{equation*}
			t=\dfrac{N-1}{2}T+\Delta t_1=1005T+\Delta t_1;
		\end{equation*}
		với $\Delta t_1$ là khoảng thời gian vật đi từ vị trí có li độ $x_0=\SI{4}{\centi\meter}$ đến vị trí $x=\SI{-2}{\centi\meter}$ lần thứ nhất.
		
		Dựa vào đường tròn lượng giác tai có góc quay từ vị trí vị trí có li độ $x_0=0$ đến vị trí $x=\SI{-2}{\centi\meter}$ lần thứ nhất là $\Delta\alpha=\dfrac{2\pi}{3}\,\text{rad}$, do đó
		\begin{equation*}
			\omega=\dfrac{\Delta\alpha}{\Delta t_1}\Rightarrow\Delta t_1=\dfrac{\Delta\alpha}{	\omega}=\dfrac{\dfrac{2\pi}{3}\,\text{rad}}{\dfrac{2\pi}{3}\,\text{rad}/\text{s}}=\SI{1}{\second}.
		\end{equation*}
		
		Thế $T=\SI{3}{\second}$ và $\Delta t_1=\SI{1}{\second}$ vào biểu thức tính thời điểm chất điểm dao động điều hòa qua vị trí $x=\SI{-2}{\centi\meter}$ lần thứ 2011 ta được
		\begin{equation*}
			t=1005T+\Delta t_1=1005\cdot\SI{3}{\second} +\SI{1}{\second}=\SI{3016}{\second}.
		\end{equation*}
	}
		\item \mkstar{4}
	
	\cauhoi{
		
		Một vật dao động điều hòa theo phương trình $x=10 \cos(10\pi t) \ \text{cm}$. Thời điểm vật đi qua vị trí N có li độ $x=\SI{5}{cm}$ lần thứ 2009 theo chiều dương là
		\begin{mcq}(4)
			\item $\SI{401,8}{s}$. 		      	
			\item $\SI{408,1}{s}$.			
			\item $\SI{410,8}{s}$.	.	 	
			\item $\SI{401,77}{s}$.
		\end{mcq}
		
	}
	\loigiai{
		\textbf{Đáp án D.}
		
		Chu kỳ dao động 
		\begin{equation*}
			T =\SI{0,2}{s}.
		\end{equation*}
		Ta có $t=0$
		\begin{equation*}
			x= 10 \cos 0 =\SI{10}{cm}= + A.
		\end{equation*}
		
		Thời gian vật đi từ vị trí ban đầu $x=+A$ tới $x =\SI{5}{cm} =\dfrac{A}{2}$ chuyển động theo chiều dương lần thứ nhất là:
		
		\begin{equation*}
			\dfrac{T}{2} + \dfrac{T}{4} + \dfrac{T}{12} = \dfrac{5T}{6}.
		\end{equation*}
		
		Còn 2008 lần sau đó, cứ một chu kì vật lại qua $x=\dfrac{A}{2}$ theo chiều dương một lần nên cần thời gian $2008T$.
		
		Thời điểm vật đi qua vị trí li độ $x=\SI{5}{cm}$ lần thứ 2009 theo chiều dương:
		
		\begin{equation*}
			t =t_1+ 2008T = \SI{401,77}{s}
		\end{equation*}
	}
\end{enumerate}
\loigiai{\textbf{Đáp án}
\begin{center}
	\begin{tabular}{|m{2.8em}|m{2.8em}|m{2.8em}|m{2.8em}|m{2.8em}|m{2.8em}|m{2.8em}|m{2.8em}|m{2.8em}|m{2.8em}|}
		\hline
		1. A & 2. D & 3. A & 4. C & 5. C & 6. B & 7. C & 8. C & 9. B & 10. B \\
		\hline
		11. C & 12. A & 13. B & 14. B & 15. D & 16. A & 17. C & 18. A & 19. A & 20. D\\
		\hline
		21. C & 22. B & 23. B & 24. C & 25. D & 26. B & 27. D & 28. C & 29. A & 30. C\\
		\hline
		31. C & 32. B & 33. B & 34. D & 35. C & 36. D & 37. B & 38. A & 39. C & 40. D\\
		\hline
	\end{tabular}
\end{center}}