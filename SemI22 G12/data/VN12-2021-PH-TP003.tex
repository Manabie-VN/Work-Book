\begin{enumerate}[label=\bfseries Câu \arabic*:]
	\item \mkstar{1}
	
\cauhoi{
	
	Một con lắc lò xo gồm vật nhỏ có khối lượng $m$ và lò xo nhẹ có độ cứng $k$. Con lắc dao động điều hòa với chu kì là
	\begin{mcq}(4)
		\item $2\pi \sqrt {\dfrac {k}{m}}$.
		\item $\sqrt {\dfrac {k}{m}}$.
		\item $\sqrt {\dfrac {m}{k}}$.
		\item $2 \pi \sqrt {\dfrac {m}{k}}$.
	\end{mcq}
}
\loigiai{
	\textbf{Đáp án D.}
	
	Chu kì dao động của một con lắc lò xo gồm vật nhỏ khối lượng $m$ và lò xo nhẹ có độ cứng $k$ là $T = 2\pi \sqrt {\dfrac {m}{k}}.$
}	
	\item \mkstar{1}
	
\cauhoi{
	
	Một con lắc lò xo gồm vật nhỏ có khối lượng $m$ và lò xo nhẹ có độ cứng $k$. Con lắc dao động điều hòa với tần số góc là
	\begin{mcq}(4)
		\item $2\pi \sqrt {\dfrac {k}{m}}$.
		\item $\sqrt {\dfrac {k}{m}}$.
		\item $\sqrt {\dfrac {m}{k}}$.
		\item $2 \pi \sqrt {\dfrac {m}{k}}$.
	\end{mcq}
}
\loigiai{
	\textbf{Đáp án B.}
	
	Tần số góc dao động của một con lắc lò xo gồm vật nhỏ khối lượng $m$ và lò xo nhẹ có độ cứng $k$ là $\omega = \sqrt {\dfrac {k}{m}}.$
}	

	\item \mkstar{1}
	
\cauhoi{
	
	Một con lắc lò xo dao động điều hòa theo phương nằm ngang. Nếu biên độ dao động tăng gấp ba thì tần số dao động điều hòa của con lắc
	
	\begin{mcq}(4)
		\item tăng $\sqrt{3}$ lần.
		\item giảm 3 lần.
		\item không đổi.
		\item tăng 3 lần.
	\end{mcq}
}
\loigiai{
	\textbf{Đáp án C.}
	
	Tần số dao động của con lắc lò xo $f = \dfrac{1}{2 \pi} \sqrt {\dfrac{k}{m}}$ chỉ phụ thuộc vào khối lượng  $m$ và độ cứng $k$ của lò xo nên tần số không đổi.
}	
	\item \mkstar{2}
	
	\cauhoi{
		
		Một con lắc lò xo có vật nặng 400 g dao động điều hòa. Vật thực hiện được 50 dao động trong thời gian 20 s. Lấy $g=10\ \text{m/s}^2$. Độ cứng của lò xo là
		\begin{mcq}(4)
			\item 50 N/m.
			\item 100 N/m.
			\item 200 N/m.
			\item 300 N/m.
		\end{mcq}
	}
	\loigiai{
		\textbf{Đáp án B.}
		
		$\Delta t=NT\Rightarrow T=\dfrac{\Delta t}{N}=\text{0,4}\ \text{s}$.
		
		$T = 2 \pi \sqrt {\dfrac{m}{k}}\Rightarrow k= 4\pi^2\dfrac{m}{T^2}=100\ \text{N/m}$.
	}
	\item \mkstar{2}
	
\cauhoi{
	
	Một con lắc lò xo có độ cứng $k$ mắc với vật nặng $m_1$ có chu kì dao động $T_1 = \text{0,1}\ \text{s}$. Nếu mắc lò xo đó với vật nặng $m_2$ thì chu kì dao động $T_2 = \text{0,2}\ \text{s}$. Chu kì dao động gắn vật có khối lượng $ m = m_1 +2m_2$ vào lò xo là
	\begin{mcq}(4)
		\item $T = \text{0,25}\ \text{s}$.
		\item $T = \text{0,22}\ \text{s}$.
		\item $T = \text{0,36}\ \text{s}$.
		\item $T = \text{0,3}\ \text{s}$.
	\end{mcq}
}
\loigiai{
	\textbf{Đáp án D.}
	
	Do độ cứng $k$ không đổi nên $T \ \text{tỉ lệ thuận}\ \sqrt{m}$.
	
	Với $m=\alpha m_1+\beta m_2$ thì chu kỳ $T^2=\alpha T_1^2+\beta T_2^2$.
	
	Trong trường hợp này, $\alpha=1, \ \beta=2$ nên $T^2=T_1^2+2T_2^2\Rightarrow T=\text{0,3}\ \text{s}$.
}
	\item \mkstar{2}
	
\cauhoi{
	
Một con lắc lò xo có độ cứng $k=100\ \text{N/m}$. Vật nặng dao động với biên độ $A=20\ \text{cm}$, khi vật đi qua li độ $x=12\ \text{cm}$ thì động năng của vật bằng
\begin{mcq}(4)
	\item $\text{1,44}\ \text{J}$.
	\item $\text{1,28}\ \text{J}$.
	\item $\text{2,56}\ \text{J}$.
	\item $\text{0,72}\ \text{J}$.
\end{mcq}
}
\loigiai{
	\textbf{Đáp án B.}
	
	Động năng của vật là $$W_{\text {đ}}=W-W_{\text {t}}=\dfrac{1}{2}kA^2-\dfrac{1}{2}kx^2=\text{1,28}\ \text{J}$$
}
	\item \mkstar{2}
	
\cauhoi{
	
	Một con lắc lò xo gồm một vật nặng có khối lương 500 g treo vào đầu lò xo có độ cứng $k = \text{2,5}\ \text{N/cm}$. Kích thước cho vật dao động, vật có gia tốc cực đại $5\ \text{m/s}^2$. Biên độ dao động của vật là
	\begin{mcq}(4)
		\item  2 cm.
		\item  5 cm.
		\item  1 cm.
		\item 10 cm.
	\end{mcq}
}
\loigiai{
	\textbf{Đáp án C.}
	
	
	Độ cứng của lò xo là $k = \text{2,5}\ \text{N/cm}=250\ \text{N/m}\Rightarrow \omega=\sqrt{\dfrac{k}{m}}=10\sqrt{5}\ \text{rad/s}$.
	
	Gia tốc cực đại là $a_\text{max}=\omega^2A\Rightarrow A=\dfrac{a_\text{max}}{\omega^2}=\text{}0,01\ \text{m}=1\ \text{cm}$.
	
}
	\item \mkstar{2}
	
\cauhoi{
	
	Một vật nặng treo vào một lò xo làm cho lò xo dãn ra $\SI{0.8}{\centi \meter}$. Kích thích cho vật dao động, tìm chu kì dao động ấy. Cho $g=\SI{10}{\meter / \second^2}$.
	\begin{mcq}(4)
		\item $\SI{0.138}{\second}$.
		\item $\SI{0.158}{\second}$.
		\item $\SI{0.178}{\second}$.
		\item $\SI{0.198}{\second}$.
	\end{mcq}
}
\loigiai{
	\textbf{Đáp án C.}
	
	
Độ biến dạng của lò xo tại vị trí cân bằng:
$$\Delta l_0 = \SI{0.8}{\centi \meter} = \SI{8e-3}{\meter}.$$

Chu kì dao động của con lắc lò xo:
$$T=2 \pi \sqrt{\dfrac{\Delta l_0}{g} }= 2\pi \sqrt{ \dfrac {\SI{8e-3}{\meter}}{\SI{10}{\meter / \second^2}} }= \SI{0.178}{\second}.$$
	
}

	\item \mkstar{2}
	
\cauhoi{
	
	Một con lắc lò xo dao động điều hòa. Lò xo có độ cứng $k=\SI{40}{\newton / \meter}$. Khi vật $m$ của con lắc đang qua vị trí có li độ $x=\SI{-2}{\centi \meter}$ thì thế năng của con lắc là bao nhiêu?
	\begin{mcq}(4)
		\item $\SI{-0.016}{\joule}$.
		\item $\SI{-0.008}{\joule}$.
		\item $\SI{0.016}{\joule}$.
		\item $\SI{0.008}{\joule}$.
	\end{mcq}
}
\loigiai{
	\textbf{Đáp án D.}
	
	
	Công thức tính thế năng của con lắc lò xo: $$W_\text t = \dfrac {1}{2}kx^2 = \dfrac {1}{2} \cdot \SI{40}{\newton / \meter} \cdot (\SI{-2e-2}{\meter})^2 = \SI{0.008}{\joule}.$$	
	
}

	\item \mkstar{2}

\cauhoi{
	
Một vật có khối lượng $\SI{750}{\gram}$ dao động điều hòa với biên độ $\SI{4}{\centi \meter}$ và chu kì $T=\SI{2}{\second}$. Tính năng lượng của dao động.
\begin{mcq}(4)
	\item $\SI{5.92e-3}{\joule}$.
	\item $\SI{6.92e-3}{\joule}$.
	\item $\SI{7.92e-3}{\joule}$.
	\item $\SI{8.92e-3}{\joule}$.
\end{mcq}
}
\loigiai{
	\textbf{Đáp án A.}
	
	
	Tần số góc của dao động:
	$$\omega = \dfrac {2\pi}{T} = \dfrac {2\pi}{\SI{2}{\second}} = \xsi{\pi}{\radian / \second}.$$
	
	Năng lượng của dao động:
	$$W=\dfrac{1}{2}m \omega ^2 A^2 = \dfrac {1}{2} \cdot \SI{0.75}{\kilogram} \cdot (\xsi{\pi}{\radian / \second})^2 \cdot (\SI{0.04}{\meter})^2 = \SI{5.92e-3}{\joule}.$$
	
}

	\item \mkstar{3}

	\cauhoi{
		
		Một con lắc lò xo gồm lò xo có độ cứng $k = 80\ \text{N/m}$ và vật nặng có khối lượng $m = 200\ \text{g}$ treo thẳng đứng. Từ vị trí cân bằng, người ta đưa vật dọc theo trục lò xo đến vị trí lò xo bị nén đoạn $\text{2,5}\ \text{cm}$ rồi buông nhẹ cho vật dao động điều hòa. Lấy $g = \pi^2 = 10\ \text{m/s}^2$. Tính từ thời điểm buông vật, thời điểm đầu tiên lực đàn hồi của lò xo có độ lớn bằng $\dfrac{1}{5}$  giá trị cực đại là
		
		\begin{mcq}(4)
			\item  $\text{0,226}\ \text{s}$.
			\item  $\text{0,088}\ \text{s}$.
			\item  $\text{0,032}\ \text{s}$.
			\item  $\text{0,324}\ \text{s}$.
		\end{mcq}
	}	
	\loigiai{
		\textbf{Đáp án C.}
		
		Tại vị trí cân bằng, lò xo bị dãn một đoạn là: $\Delta l_0=\dfrac{mg}{k}=\text{0,025}\ \text{m}$. 
		
		Lò xo nén $\text{2,5}\ \text{cm}$ rồi thả nhẹ nên biên độ dao động của vật là: $A=\Delta l+\text{2,5}\ \text{cm}=5\ \text{cm}$.
		
		Lực đàn hồi của lò xo có độ lớn bằng $\dfrac{1}{5}$ giá trị cực đại $F=\dfrac{1}{5}F_\text{max}$.
		
		Suy ra $k\Delta l'=\dfrac{1}{5}\cdot k\cdot\left( \Delta l+A\right) \Rightarrow \Delta l'=\text{1,5}\ \text{cm}$.
		
		Như vậy $x=\Delta l'-\Delta l=-1\ \text{cm}$ hoặc  $ x=-\Delta l'-\Delta l=-4\ \text{cm}$.
		
		Do đó $\Delta t=t_{\left( -5 \rightarrow  -4\right) }=\dfrac{1}{\omega}\cdot \arccos \dfrac{4}{5}=\text{0,032}\ \text{s}$.	
		
	}
	\item \mkstar{3}
	
\cauhoi{
	
	
	Một con lắc lò xo dao động điều hoà với biên độ $A$ trên mặt phẳng nằm ngang. Khi động năng của vật gấp 2 lần thế năng thì vận tốc của vật là $10\ \text{cm/s}$. Vận tốc cực đại của vật trong quá trình dao động là
	
	\begin{mcq}(4)
		\item $10\sqrt{2}\ \text{cm/s}$.
		\item $5\sqrt{6}\ \text{cm/s}$.
		\item $10\ \text{cm/s}$.
		\item $20\sqrt{3}\ \text{cm/s}$.
	\end{mcq}
}
\loigiai{
	\textbf{Đáp án B.}
	
	Động năng của vật gấp 2 lần thế năng nên $W_{\text {đ}}=2W_{\text {t}}$ với $n=2$.
	
	Khi đó, vận tốc của vật là $$v=\pm \sqrt{\dfrac{n}{n+1}}v_\text{max}=\pm \sqrt{\dfrac{2}{2+1}}v_\text{max}$$
	
	$$\Rightarrow v_\text{max}= \dfrac{\sqrt{6}}{2}v=5\sqrt{6}\ \text{cm/s}$$	
}

	\item \mkstar{3}

\cauhoi{
	
	Một con lắc lò xo dao động điều hoà theo phương nằm ngang. Biết rằng khi tốc độ của vật là $\text{48}\pi\ \text{cm/s}$ thì động năng bằng n lần thế năng, còn khi vật có li độ $x=4\ \text{cm}$ thì thế năng bằng n lần động năng. Chu kỳ dao động của con lắc là
	\begin{mcq}(4)
		\item $\text{0,52}\ \text{s}$.
		\item $\text{0,6}\ \text{s}$.
		\item $\text{0,167}\ \text{s}$. 
		\item $\text{0,256}\ \text{s}$.
	\end{mcq}
}
\loigiai{
	\textbf{Đáp án C.}
	
Khi $W_{\text {đ}}=nW_{\text {t}}$ thì  $v=\pm \sqrt{\dfrac{n}{n+1}}v_\text{max}$.

Do đó $$\left( \dfrac{v}{v_\text{max}} \right)^2=\dfrac{n}{n+1}\Rightarrow \left( \dfrac{\text{48}\pi\ \text{cm/s}}{\omega A} \right)^2=\dfrac{n}{n+1}$$

Khi $W_{\text {t}}=nW_{\text {đ}}$ thì $x=\pm \sqrt{\dfrac{n}{n+1}}A\Rightarrow \left( \dfrac{\text{4}\ \text{cm}}{A} \right)^2=\dfrac{n}{n+1} $.


Do đó $$\left( \dfrac{\text{48}\pi\ \text{cm/s}}{\omega A} \right)^2=\left( \dfrac{\text{4}\ \text{cm}}{A} \right)^2\Rightarrow \omega= 12\pi \ \text{rad/s}\Rightarrow T=\text{0,167}\ \text{s}$$
}	

	\item \mkstar{3}

\cauhoi{
	
Con lắc lò xo gồm vật có khối lượng $m=100\ \text{g}$, lò xo có độ cứng $k=40\ \text{N/m}$. Thời điểm ban đầu kéo vật lệch ra khỏi vị trí cân bằng theo chiều âm một đoạn $10\ \text{cm}$ rồi thả nhẹ. Viết phương trình dao động của vật.

\begin{mcq}(2)
	\item $x=10\cos \left( 20t+\dfrac{2\pi}{3}\right)$.
	\item $x=10\cos \left( 20t+\dfrac{\pi}{3}\right)$.
	\item $x=10\cos \left( 20t+\pi\right)$.
	\item $x=10\cos \left( 20t-\dfrac{2\pi}{3}\right)$.
\end{mcq}
}
\loigiai{
	\textbf{Đáp án C.}
	
	Tần số góc dao động của vật là $$ \omega = \sqrt {\dfrac {k}{m}}=20\ \text{rad/s}$$
	
	Thời điểm ban đầu kéo vật lệch khỏi vị trí cân bằng theo một đoạn 10 cm rồi thả nhẹ  nên ta có: $x_0=-10\ \text{cm}$, $v_0=0$. Do đó, $$A=\sqrt{x^2+\dfrac{v^2}{\omega^2}}=10\ \text{cm}$$
	
	Dựa vào đường tròn lượng giác, ta xác định được pha ban đầu là $\varphi=\pi$.
	
	Phương trình dao động của vật là $$x=10\cos \left( 20t+\pi\right)$$
}	

	\item \mkstar{3}

\cauhoi{
	
	Một con lắc lò xo gồm một vật nặng có khối lương 500 g treo vào đầu lò xo có độ cứng $k = \text{90}\ \text{N/m}$. Thời điểm ban đầu kéo vật lệch ra khỏi vị trí cân bằng theo chiều âm một đoạn $10\ \text{cm}$ rồi truyền cho vật một vận tốc ban đầu là $300\sqrt{3}\ \text{cm/s}$ theo chiều dương. Viết phương trình dao động của vât.
	\begin{mcq}(2)
		\item  $x=20\cos \left( 30t-\dfrac{2\pi}{3}\right)$.
		\item  $x=20\cos \left( 30t+\dfrac{2\pi}{3}\right)$.
		\item $x=20\cos \left( 30t+\dfrac{2\pi}{3}\right)$.
		\item $x=20\cos \left( 30t+\dfrac{\pi}{3}\right)$.
	\end{mcq}
}
\loigiai{
	\textbf{Đáp án A.}
	
	 Tần số góc dao động của vật là 
	\begin{equation*}
		\omega = \sqrt {\dfrac {k}{m}}=30\ \text{rad/s}.
	\end{equation*}
	Thời điểm ban đầu kéo vật lệch khỏi vị trí cân bằng theo một đoạn 10 cm theo chiều âm nên $x_0=-10\ \text{cm}$.
	
	Thời điểm ban đầu vật đi theo chiều dương nên 
	\begin{equation*}
		v_0=300\sqrt{3}\ \text{cm/s}. 
	\end{equation*}	
	Do đó, 
	\begin{equation*}
		A=\sqrt{x^2+\dfrac{v^2}{\omega^2}}=20\ \text{cm}.
	\end{equation*}	
	
	Dựa vào đường tròn lượng giác, tại thời điểm ban đầu vật đi qua vị trí $x_0=-10\ \text{cm}=-\dfrac{A}{2}$ theo chiều dương nên pha ban đầu là $\varphi=-\dfrac{2\pi}{3}$.
	
	Phương trình dao động của vật là 
	\begin{equation*}
		x=20\cos \left( 30t-\dfrac{2\pi}{3}\right).
	\end{equation*}	
}	
	\item \mkstar{3}

	\cauhoi{
		
		Một con lắc lò xo gồm lò xo có độ cứng $k = 50\ \text{N/m}$ và vật nặng có khối lượng $m = 200\ \text{g}$ treo thẳng đứng. Từ vị trí cân bằng, người ta đưa vật dọc theo trục lò xo đến vị trí lò xo bị nén $4\ \text{cm}$ rồi buông nhẹ cho vật dao động điều hòa. Lấy $g = \pi^2 = 10\ \text{m/s}^2$. Tính từ thời điểm buông vật, thời điểm đầu tiên lực đàn hồi của lò xo có độ lớn bằng nửa giá trị cực đại và đang giảm là
		\begin{mcq}(4)
			\item $\text{0,016}\ \text{s}$.
			\item $\text{0,100}\ \text{s}$.
			\item $\text{0,300}\ \text{s}$.
			\item $\text{0,284}\ \text{s}$.
		\end{mcq}
	}	
	\loigiai{
		\textbf{Đáp án D.}
		
		Tại vị trí cân bằng, lò xo đã bị dãn một đoạn là: $\Delta l_0=\dfrac{mg}{k}=\text{0,04}\ \text{m}$.
		
		Lò xo nén 4 cm rồi thả nhẹ nên biên độ dao động của vật là: $A = \Delta l_0 + 4 = 8\ \text{cm}$.
		
		Lực đàn hồi của lò xo có độ lớn bằng $\dfrac{1}{2}$ giá trị cực đại $F=\dfrac{1}{2}F_\text{max}$.
		
		Suy ra $k\Delta l'=\dfrac{1}{2}\cdot k\cdot\left( \Delta l+A\right) \Rightarrow \Delta l'=6\ \text{cm}$.
		
		Như vậy $x=\Delta l'-\Delta l=2\ \text{cm}$ hoặc  $ x=-\Delta l'-\Delta l=-10\ \text{cm}\ \left( \text{loại}\right) $.
		
		Do lực đàn hồi đang giảm nên vật ở vị trí li độ $x = 2\ \text{cm}$ và tiến về vị trí không biến dạng. 
		
		Do đó $\Delta t=t_{\left( -8 \rightarrow  8\right) }+ t_{\left( 8 \rightarrow  2\right) }=\dfrac{T}{2}+\dfrac{1}{\omega}\cdot \arccos \dfrac{2}{8}=\text{0,284}\ \text{s}$.
	}
	\item \mkstar{3}
	
	\cauhoi{
		
		Một con lắc lò xo treo thẳng đứng gồm vật nhỏ có $m = 200\ \text{g}$ treo phía dưới một lò xo nhẹ có $k = 50\ \text{N/m}$ . Từ vị trí cân bằng, kéo vật xuống dưới một đoạn sao cho lò xo dãn $8\ \text{cm}$ và truyền cho vật vận tốc $v=20\pi\sqrt{3}\ \text{cm/s}$. Lấy $g = \pi^2 = 10\ \text{m/s}^2$. Tỉ số giữa thời gian lò xo dãn và thời gian lò xo nén trong một chu kỳ dao động là
		
		\begin{mcq}(4)
			\item $\text{0,5}$.
			\item $\text{2,0}$.	
			\item $\text{0,4}$.
			\item $\text{2,5}$.
		\end{mcq}
	}
	\loigiai{
		\textbf{Đáp án B.}
		
		- Khoảng thời gian lò xo nén là: $\Delta t=\dfrac{2\alpha}{\omega}=2\cdot \dfrac{1}{\omega}\cdot \arccos \dfrac{\Delta l_0}{A}$.
		
		- Khoảng thời gian lò xo dãn là: $T-\Delta t$.
		
		Ta có: $\Delta l_0=\dfrac{mg}{k}=\text{0,04}\ \text{m}$.
		
		Biên độ dao động của vật là: $A=\sqrt{x^2+\dfrac{v^2}{\omega^2}}=\sqrt{\left(\Delta l-\Delta l_0 \right)^2+\dfrac{mv^2}{k}}=8\ \text{cm}$.	
		
		Thời gian lo xo nén là $2\cdot \dfrac{T}{6}=\dfrac{T}{3}$, thời gian lò xo dãn là $T-\dfrac{T}{3}=\dfrac{2T}{3}\Rightarrow \dfrac{t_\text{dãn}}{t_\text{nén}}=2$.  

	}
	
	\item \mkstar{3}

	\cauhoi{
		
		Cho ba con lắc lò xo dao động điều hòa theo phương nằm ngang. Biết ba lò xo giống hệt nhau và vật nặng có khối lượng tương ứng $m_1, \ m_2, \ m_3$. Lần lượt kéo ba con lắc lò xo dãn cùng một đoạn $A$ như nhau rồi thả nhẹ cho ba vật dao động điều hòa. Khi qua vị trí cân bằng vận tốc của hai vật $m_1, \ m_2$ có vận tốc là  $v_1= 20\ \text{cm/s}$ và $v_2 = 10\ \text{cm/s}$. Biết $m_3=9m_1+4m_2$, độ lớn vận tốc cực đại của vật $m_3$ là
		\begin{mcq}(4)
			\item $9\ \text{cm/s}$.
			\item $5\ \text{cm/s}$.
			\item $10\ \text{cm/s}$.
			\item $4\ \text{cm/s}$.
		\end{mcq}	
	}	
	\loigiai{
		\textbf{Đáp án D.}
		
		Do $3$ con lắc đặt nằm ngang nên khi kéo cùng một đoạn A rồi thả nhẹ thì $A_1=A_2=A_3=A$.
		
		Mặt khác $v_\text{3max}=\omega_3 A=\sqrt{\dfrac{k}{9m_1+4m_2}}\cdot A$  suy ra  $\dfrac{1}{v_\text{3max}^2}=\dfrac{9}{kA^2}+\dfrac{4m_2}{kA^2}=\dfrac{9}{v_\text{1max}^2}+\dfrac{4}{v_\text{2max}^2}$.
		
		Do đó  $v_\text{3max}=4\ \text{cm/s}$
	}
	
	\item \mkstar{3}
	
	\cauhoi{
		
		Một lò xo có độ cứng $k$. Lần lượt gắn vào lò xo các vật $m_1$, $m_2$, $m_3= m_1 +2m_2$ và $m_4 = 2m_1 - m_2$ $(2m_1 > m_2)$. Ta thấy chu kì dao động của các vật trên lần lượt là $T_1, \ T_2, \ T_3 = 8\ \text{s}$, $T_4 = 5\ \text{s}$. Khi đó $T_1, \ T_2$ có giá trị là
		\begin{mcq}(2)
			\item $T_1=\text{9,43}\ \text{s}, \ T_2=\text{6,25}\ \text{s}$.
			\item $T_1=\text{6,67}\ \text{s}, \ T_2=\text{1,56}\ \text{s}$.
			\item $T_1=\text{10,67}\ \text{s}, \ T_2=\text{10,15}\ \text{s}$.
			\item $T_1=\text{4,77}\ \text{s}, \ T_2=\text{4,54}\ \text{s}$.
		\end{mcq}	
	}
	\loigiai{
		\textbf{Đáp án D.}
		
		Ta có: $T_1=2\pi \sqrt{\dfrac{m_1}{k}}, \ T_2=2\pi \sqrt{\dfrac{m_2}{k}}, \ T_3=2\pi \sqrt{\dfrac{m_1+2m_2}{k}}, \ T_4=2\pi \sqrt{\dfrac{2m_1-m_2}{k}}$.
		
		Do đó: $T_1^2+2T_2^2=T_3^2=64, \ 2T_1^2-T_2^2=T_4^2=25$.
		
		Suy ra: $T_1=\text{4,77}\ \text{s}, \ T_2=\text{4,54}\ \text{s}$.
		
		
	}
	
	\item \mkstar{3}
	
	\cauhoi{
		
		Một chất điểm dao động điều hoà không ma sát trên trục O$x$, mốc thế năng ở vị trí cân bằng O. Biết trong quá trình khảo sát chất điểm không đổi chiều chuyển động. Khi vừa rời khỏi vị trí cân bằng môt đoạn $s$ thì động  năng của chất điểm là $\text{13,95}\ \text{mJ}$, đi tiếp một đoạn $s$ nữa thì động năng của chất điểm chỉ còn $\text{12,60}\ \text{mJ}$. Nếu chất điểm đi tiếp một đoạn $s$ nữa thì động năng của chất điểm khi đó bằng
		\begin{mcq}(4)
			\item $\text{11,25}\ \text{mJ}$.
			\item $\text{10,35}\ \text{mJ}$.
			\item $\text{8,95}\ \text{mJ}$.
			\item $\text{6,68}\ \text{mJ}$.
		\end{mcq}	
	}
	\loigiai{
		\textbf{Đáp án B.} 
		
		Theo giả thiết ta có:$W_\text{đ1}=w-\dfrac{1}{2}ks^2=\text{13,95}\ \left( 1\right) $,  (với $W$ là cơ năng của chất điểm).
		
		Lại có: $W_2=W-\dfrac{1}{2}\cdot k\left( 2s\right)^2=\text{12,6}\ \text{s}\ \left(2 \right)  $.
		
		Giải $(1)$ và $(2)$ suy ra: $W=\text{14,4}\ \text{mJ}$ và $kS^2=\text{0,9}\ \text{mJ}$.
		
		Ta cần tìm:  $W_\text{đ}=W-\dfrac{1}{2}\cdot k\left( 3s\right)^2= \text{10,35}\ \text{mJ}$.
	}
	
	
	\item \mkstar{3}
	
	\cauhoi{
		
		Một vật dao động điều hoà với biên độ $A$, đang đi tới vị trí cân bằng ($t = 0$, vật ở vị trí biên), sau đó một khoảng thời gian $t$   thì vật có thế năng bằng $\text{30}\ \text{J}$, đi tiếp một khoảng thời gian $3t$ nữa thì chỉ còn cách vị trí cân bằng một khoảng bằng $\dfrac{A}{7}$. Biết $4t<\dfrac{T}{4}$. Hỏi khi tiếp tục đi một thời gian $\dfrac{T}{4}$ thì thế năng của vật bằng bao nhiêu?
		
		\begin{mcq}(4)
			\item $\text{33,5}\ \text{J}$.
			\item $\text{0,8}\ \text{J}$.
			\item $\text{45,1}\ \text{J}$. 
			\item $\text{0,7}\ \text{J}$.
		\end{mcq}	
	} 
	\loigiai{
		\textbf{Đáp án A.}
		
		Ta có  $\cos \left( 4\omega t \right)=\pm \dfrac{1}{7}\Rightarrow \cos\left( \omega t \right) =\text{0,937}\Rightarrow \dfrac{E_\text{t1}}{E}=\left( \dfrac{x}{A}\right)^2=\text{0,878} \Rightarrow E=\text{34,17}\ \text{J}$.
		
		Khi $x=\dfrac{A}{7}\Rightarrow E_\text{t}=\text{0,679}\ \text{J}\Rightarrow E_\text{đ}=\text{33,5}\ \text{J}$.
		
		Sau khoảng thời gian $\dfrac{T}{4}$ thì $E_\text{đ2}=E_\text{t3}=\text{33,5}\ \text{J}$. 

	}
	\item \mkstar{3}

	\cauhoi{
		
		Một con lắc lò xo treo thẳng đứng. Trong quá trình vật dao động người ta thấy tỷ số độ lớn giữa lực đàn hồi cực đại và lực đàn hồi cực tiểu tác dụng lên vật bằng 3. Lấy $g =\pi^2 = 10\ \text{m/s}^2$, chọn gốc tọa độ O tại vị trí cân bằng, chiều dương hướng xuống. Biết phương trình dao động của vật là $X=A\cos \left(5\pi t-\pi \right)$. Tốc độ cực đại của vật trong quá trình dao động là
		\begin{mcq}(4)
			\item $10\pi\ \text{cm/s}$.
			\item $10\ \text{cm/s}$.
			\item $20\pi\ \text{cm/s}$.
			\item $20\ \text{cm/s}$.
		\end{mcq}
	}	
	\loigiai{
		\textbf{Đáp án C.}
		
		Trong trường hợp $\Delta l_0<A$ thì $F_{\text{đh}\ \text{min}}=0$, do đó trong bài toán này $\dfrac{F_{\text{đh}\ \text{max}}}{F_{\text{đh}\ \text{min}}}=3\Rightarrow \Delta l_0>A $.
		
		Tỷ số độ lớn giữa lực đàn hồi cực đại và lực đàn hồi cực tiểu tác dụng lên vật bằng 3 nên:
		
		$$\dfrac{F_{\text{đh}\ \text{max}}}{F_{\text{đh}\ \text{min}}}=\dfrac{k\left(\Delta l_0+A \right)}{k\left(\Delta l_0-A \right)}=3\Rightarrow A=\dfrac{\Delta l_0}{2}=\dfrac{g}{2\omega^2}=\text{0,02}\ \text{m}$$
		
		Do đó, tốc độ cực đại của vật là $$v_\text{max}=\omega A=10\pi\ \text{cm/s}$$
		
		
	}
	\item \mkstar{3}

\cauhoi{
	
	Con lắc lò xo treo thẳng đứng, độ cứng $k=50\ \text{N/m}$, vật nặng khối lượng $m = 200\ \text{g}$ dao động điều hòa  theo phương thẳng đứng với biên độ $A =4\sqrt{2}\ \text{cm}$, lấy $g =\pi^2= 10\ \text{m/s}^2$. Trong một chu kỳ, thì thời gian lò xo nén bằng
	\begin{mcq}(4)
		\item  $\text{0,2}\ \text{s}$.
		\item  $\text{0,1}\ \text{s}$.
		\item $\text{0,3}\ \text{s}$.
		\item $\text{0,5}\ \text{s}$.
	\end{mcq}
}
\loigiai{
	\textbf{Đáp án B.}
	
	Tần số góc $$\omega=\sqrt{\dfrac{k}{m}}=5\pi\ \text{rad/s}$$
	
	Độ biến dạng của lò xo khi vật ở vị trí cân bằng $$\Delta l_0=\dfrac{g}{\omega^2}=\text{4}\ \text{cm}$$
	
	$$\cos \alpha=\dfrac{\Delta l_0}{A}\Rightarrow \alpha=\arccos \dfrac{\Delta l_0}{A}=\dfrac{\pi}{4}$$
	
	Thời gian lò xo nén là $$t_\text{nén}=\dfrac{\varphi_\text{nén}}{\omega}=\dfrac{2\alpha}{\omega}=\text{0,1}\ \text{s}$$

}
	\item \mkstar{3}

\cauhoi{
	
	Một con lắc lò xo treo thẳng đứng có vật nặng có khối lượng $m=200\ \text{g}$. Kích thích cho con lắc dao động theo phương thẳng đứng thì nó dao động điều hòa với chu kỳ $\text{0,4}\ \text{s}$ và trong quá trình dao động, chiều dài của lò xo thay đổi từ $l_1= 16\ \text{cm}$ đến $l_2 = 22\ \text{cm}$. Lấy $g =\pi^2= 10\ \text{m/s}^2$. Lực đàn hồi cực tiểu của lò xo trong quá trình vật dao động là
	
	\begin{mcq}(4)
		\item  $\text{1,5}\ \text{N}$.
		\item  $\text{0}\ \text{N}$.
		\item $\text{2,5}\ \text{N}$.
		\item $\text{0,5}\ \text{N}$.
	\end{mcq}	
}
\loigiai{
	\textbf{Đáp án D.}
	
	Biên độ dao động của vật là $$A=\dfrac{l_2-l_1}{2}=3\ \text{cm}$$
	
	Tần số góc $$\omega=\dfrac{2\pi}{T}=5\pi\ \text{rad/s}$$
	
	Mặt khác $$\Delta l_0=\dfrac{g}{\omega^2}=\text{4}\ \text{cm}$$
	
	Độ cứng của lò xo là $$k=m\omega^2=50\ \text{N/m}$$
	
	Do $\Delta l_0>A$ nên lực đàn hồi cực tiểu là $$F_\text{min}=k\left( \Delta l_0-A\right)=\SI{0.5}{N} $$
	
	
}
	\item \mkstar{4}

	\cauhoi{
		
		Một chất điểm dao động điều hoà không ma sát theo trục Ox với biên độ $A$, mốc thế năng tại vị trí cân bằng. Khi vừa rời khỏi vị trí cân bằng một đoạn $S$ thì động năng của chất điểm là $\text{10,92}\ \text{J}$. Đi tiếp một đoạn $S$ nữa thì động năng của chất điểm chỉ còn $\text{7,68}\ \text{J}$, nếu vật tiếp tục đi một đoạn $\dfrac{2S}{3}$ nữa thì động năng của chất điểm bằng bao nhiêu, biết vật chưa đổi chiều chuyển động?
		
		\begin{mcq}(4)
			\item  $\text{5,75}\ \text{J}$.
			\item  $\text{4,32}\ \text{J}$.
			\item  $\text{5,25}\ \text{J}$.
			\item  $\text{4,84}\ \text{J}$.
		\end{mcq}	
	}
	\loigiai{
		\textbf{Đáp án B.}
		
		Khi vật vừa đi qua vị trí cân bằng đoạn $S$: $E_\text{t}=E-\text{10,92}=\dfrac{1}{2}kS^2$.
		
		Khi vật đi thêm một đoạn $S$ nữa, tương tự ta có: $E'_\text{t}=E-\text{7,68}=\dfrac{1}{2}k(2S)^2$.
		
		Suy ra, $\dfrac{E'_\text{t}}{E_\text{t}}=\dfrac{E-\text{7,68}}{E-\text{10,92}}=4\Rightarrow E=12\ \text{J}\Rightarrow \dfrac{S}{A}=\sqrt{\dfrac{E-E_\text{đ1}}{E}}=\text{0,3}\Rightarrow \dfrac{8}{3}\cdot S<A$.
		
		Khi vật đi tiếp $\dfrac{2S}{3}$:  $E_{\text{t3}}=\left( \dfrac{8}{3}\right)^2\cdot E_\text{t1}=\text{7,68}\ \text{J}\Rightarrow E_\text{đ3}=12-\text{7,68}=\text{4,32}\ \text{J}$.
		
		
	}
	
	\item \mkstar{4}

	\cauhoi{
		
		Một con lắc lò xo treo thẳng đứng, đầu trên lò xo gắn cố định, đầu dưới lò xo gắn với vật nặng có khối lượng $100\ \text{g}$. Kích thích cho vật dao động điều hòa dọc theo trục Ox có phương thẳng đứng, chiều dương hướng xuống dưới, gốc O tại vị trí cân bằng của vật. Phương trình dao động của vật có dạng  $x=A\cos \left(\omega t+\varphi \right) $, $t$ tính bằng giây, thì lực kéo về có phương trình  $F=2\cos\left(5\pi t-\dfrac{5\pi}{6} \right) \ \text{N} $. Lấy  $g = \pi^2 = 10\ \text{m/s}^2$. Thời điểm có độ lớn lực đàn hồi bằng $\text{0,5}\ \text{N}$ lần thứ 2018 (tính từ lúc $t = 0$) có giá trị gần đúng bằng
		\begin{mcq}(4)
			\item $\text{201,72}\ \text{s}$.
			\item $\text{161,27}\ \text{s}$.
			\item $\text{242,45}\ \text{s}$.
			\item $\text{120,724}\ \text{s}$.
		\end{mcq}	
	}	
	\loigiai{
		\textbf{Đáp án A.}
		
		Ta có: $k=m\omega^2=25\ \text{N/m}, \ \Delta l=\dfrac{mg}{k}=\text{0,04} \ \text{m}, \ T=\text{0,4}\ \text{s}$.
		
		$F=kx\Rightarrow x=-\dfrac{F}{k}=\text{0,08}\cos \left( 5\pi t+\dfrac{\pi}{6} \right) $.
		
		$\left| F_\text{đh}\right| =\left| k\cdot \left(\Delta l +x \right)\right| =\left| 25\cdot  \left(\text{0,04} +x\right)\right|   $.
		
		Suy ra $x=-\text{0,02}\ \text{m}$ hoặc $x=-\text{0,04}\ \text{m}$.
		
		Một chu kỳ vật qua vị trí lực đàn hồi có độ lớn bằng $\text{0,5}\ \text{N}$ $4$ lần tại $L_1, \ L_2, \ L_3, \ L_4$.
		
		Tách $2018$ lần, suy ra $t=504T+\text{0,3}T=\text{201,72}\ \text{s}$ (với $\alpha=108^\circ\approx \text{0,3}\ \text{T}$).
		
		
	}
	
	\item \mkstar{4}
	
	\cauhoi{
		
		Một con lắc lò xo treo thẳng đứng, đầu trên lò xo gắn cố định, đầu dưới lò xo gắn với vật nặng. Kích thích cho vật dao động điều hòa dọc theo trục Ox có phương thẳng đứng, chiều dương hướng xuống dưới, gốc O tại vị trí cân bằng của vật, năng lượng vật dao động bằng $\text{67,5}\ \text{mJ}$. Độ lớn lực đàn hồi cực đại bằng $\text{3,75}\ \text{N}$. Khoảng thời gian ngắn nhất vật đi từ vị trí biên dương đến vị trí có độ lớn lực đàn hồi bằng $3\ \text{N}$ là $\Delta t_1$. Khoảng thời gian lò xo nén trong một chu kì là  $\Delta t_2$, với  $\Delta t_2=2\Delta t_1$. Lấy  $g = \pi^2 = 10\ \text{m/s}^2$. Khoảng thời gian lò xo bị dãn trong một chu kỳ có giá trị gần đúng bằng 
		
		\begin{mcq}(4)
			\item  $\text{0,182}\ \text{s}$.
			\item  $\text{0,293}\ \text{s}$.
			\item  $\text{0,346}\ \text{s}$.
			\item  $\text{0,212}\ \text{s}$.
		\end{mcq}
	}
	\loigiai{
		\textbf{Đáp án B.}
		
		Cơ năng: $W=\dfrac{1}{2}kA^2=\text{67,5}\ \cdot 10^{-3}\ \text{J}\ (1)$.
		
		Lực đàn hồi cực đại: $F_\text{đhmax}=k\left(\Delta l_0+A \right)=\text{3,75}\ \text{N}\ (2)$.
		
		Gọi H là điểm tại đó $F_\text{đh}=3\ \text{N}\Rightarrow \Delta_1$ là thời gian vật đi từ H đến A. 
		
		$\Delta t_2$ là thời gian lò xo bị nén, vật đi từ I tới A và từ A tới I.
		
		Do $\Delta t_2=2\Delta t_1$, suy ra H,I  đối xứng qua O, suy ra $HI=2\Delta l_0$.
		
		Lực đàn hồi tại H: $F_\text{H}=k\cdot \text{IH}=k\cdot 2\Delta l_0=3\Rightarrow k\cdot \Delta l_0=\text{1,5}\ \text{N}\ (3) $.
		
		Từ $(2), \ (3)$, ta tìm được $kA=\text{2,25}\ \text{N} \ (4)$.
		
		Và từ $(1), \ (4)$ giá trị $A=\text{6}\ \text{cm}, \ k=\text{37,5}\ \text{N/m}$.
		
		Thay lên $(3)$, ta tìm được $\Delta l_0=4\ \text{cm}$.
		
		Ta có: $\alpha= \arccos \left( \dfrac{4}{6}\right)=\text{48,19}^\circ\Rightarrow \varphi_\text{nén}=\text{86,38}^\circ \Rightarrow \varphi_\text{giãn}=\text{263,62}^\circ$.
		
		Vậy $\Delta t_\text{giãn}=\dfrac{\varphi_\text{giãn}}{\omega}=\text{0,29}\ \text{s}$.
		
	}
	
	\item \mkstar{4}

\cauhoi{
	
	Một con lắc lò xo có độ cứng của lò xo $k = 100\ \text{N/m}$, vật nặng có khối lượng $m=500\ \text{g}$, lấy $g=10\ \text{m/s}^2$. Kéo vật ra khỏi vị trí cân bằng một đoạn 8 cm rồi thả không vận tốc. Trong quá trình dao động thực tế có ma sát, với hệ số ma sát là $\mu=\text{0,02}$. Số chu kì dao động cho đến lúc vật dừng lại là 
	\begin{mcq}(4)
		\item $50$.
		\item $5$.
		\item $20$.
		\item $2$.
	\end{mcq}
}
\loigiai{
	\textbf{Đáp án C.}
	
	Biên độ dao động ban đầu là $ A = 8 cm$.
	
	Độ giảm biên độ sau một chu kì: $$\Delta A=\dfrac{4F_\text{ms}}{k}=\dfrac{4\mu mg}{k}=4\cdot 10^{-3}\ \text{m}=\text{0,4}\ \text{cm}$$
	
	Số chu kì dao động cho đến khi vật dừng lại là: $$N=\dfrac{A}{\Delta A}=20$$
	
	
}	
	\item \mkstar{4}

\cauhoi{
	
	Một con lắc lò xo có đô cứng $k = 100\ \text{N/m}$, khối lượng $m =250\ \text{g}$ dao động tắt dần trên mặt phẳng nằm ngang có ma sát, hệ số ma sát là $\mu=\text{0,04}$. Ban đầu vật ở vị trí có biên độ  $A=5\ \text{cm}$, lấy gia tốc trọng trường $g=10\ \text{m/s}^2$. Quãng đường vật đi được đến khi dừng lại hẳn là 
	\begin{mcq}(4)
		\item  $120\ \text{cm}$.
		\item  $60\ \text{cm}$.
		\item $125\ \text{cm}$.
		\item $250\ \text{cm}$.
	\end{mcq}
}
\loigiai{
	\textbf{Đáp án C.}
	
	Gọi $S$ là quãng đường vật đi được từ lúc bắt đầu dao động đến khi dừng hẳn.
	Theo định luật bảo toàn năng lượng ta có:
	
	$$\dfrac{1}{2}kA^2=F_\text{ms}S\Rightarrow S=\dfrac{kA^2}{2\mu mg}=\text{1,25}\ \text{m}=\text{125}\ \text{cm}$$
	
}		
	\item \mkstar{4}

\cauhoi{
	
	Một con lắc lò xo gồm vật nhỏ khối lượng $\text{0,02}\ \text{kg}$ và lò xo có độ cứng $k=1\ \text{N/m}$. Vật nhỏ được đặt trên giá đỡ cố định nằm ngang dọc theo trục lò xo. Hệ số ma sát trượt giữa giá đỡ và vật nhỏ là $\text{0,1}$. Ban đầu giữ vật ở vị trí lò xo bị nén 10 cm rồi buông nhẹ để con lắc dao động tắt dần. Lấy $g=10\ \text{m/s}^2$. Tốc độ lớn nhất vật nhỏ đạt được trong quá trình dao động là 
	\begin{mcq}(4)
		\item  $40\ \text{cm/s}$.
		\item  $40\sqrt{2}\ \text{cm/s}$.
		\item  $20\ \text{cm/s}$.
		\item  $20\sqrt{2}\ \text{cm/s}$.
	\end{mcq}
}
\loigiai{
	\textbf{Đáp án B.}
	
	Tốc độ lớn nhất vật nhỏ đạt được trong quá trình dao động khi vật ở vị trí lực đàn hồi cân bằng với lực ma sát, tức là khi đó vật ở vị trí cân bằng mới.
	
	Định luật bảo toàn năng lượng trong quá trình dao động:
	
	$$\dfrac{1}{2}kA^2=\dfrac{1}{2}kx_0^2+\dfrac{1}{2}mv_\text{max}^2+\mu mg(A-x_0)$$
	
	Với $$x_0=\dfrac{\mu mg}{k}=\text{0,02}\ \text{m}$$
	
	$$\Rightarrow v_\text{max}=\sqrt{\dfrac{k}{m}}\left(A-x_0\right)=\omega \left(A-x_0\right)=40\sqrt{2}\ \text{cm/s}$$
	
}	
	
\end{enumerate}
\loigiai{\textbf{Đáp án}
	\begin{center}
		\begin{tabular}{|m{2.8em}|m{2.8em}|m{2.8em}|m{2.8em}|m{2.8em}|m{2.8em}|m{2.8em}|m{2.8em}|m{2.8em}|m{2.8em}|}
			\hline
			1. D & 2. B & 3. C & 4. B & 5. D & 6. B & 7. C & 8. C & 9. D & 10. A \\
			\hline
			11. C & 12. B & 13. C & 14. C & 15. A & 16. D & 17. B & 18. D & 19. D & 20. B\\
			\hline
			21. A & 22. C & 23. B & 24. D & 25. B & 26. A & 27. B & 28. C & 29. C & 30. B\\
			\hline
		\end{tabular}
\end{center}}