\begin{enumerate}[label=\bfseries Câu \arabic*:]

\item \mkstar{1}

\cauhoi{
	
	Trong các phát biểu sau phát biểu nào \textbf{không} đúng về con lắc đơn dao động điều hòa?
	
	\begin{mcq}
		\item Chu kỳ của con lắc đơn phụ thuộc vào chiều dài dây treo.
		\item Chu kỳ của con lắc đơn không phụ thuộc vào khối lượng của vật nặng.
		\item Chu kỳ của con lắc đơn phụ thuộc vào biên độ của dao động.
		\item Chu kỳ của con lắc đơn phụ thuộc vào vị trí thực hiện thí nghiệm.
	\end{mcq}
}
\loigiai{
	\textbf{Đáp án C.}
	
	Chu kỳ của con lắc đơn
	
	\begin{equation*}
		T=2\pi \sqrt {\dfrac{l}{g}}.
	\end{equation*}
	
	Biểu thức trên cho thấy chu kỳ của con lắc đơn không phụ thuộc vào biên độ dao động.
}

\item \mkstar{2}

\cauhoi{
	
Một con lắc đơn gồm sợ dây có chiều dài $20\ \text{cm}$ treo tại một điểm cố định. Kéo con lắc khỏi phương thẳng đứng một góc bằng 0,1 rad về phía bên phải, rồi truyền cho con lắc một tốc độ bằng $14\sqrt 3\ \text{cm/s}$ theo phương vuông góc với dây. Coi con lắc dao động điều hòa. Cho gia tốc trọng trường $9,8\ \text {m/s}^2$. Biên độ dài của con lắc là 

\begin{mcq}(4)
	\item $0,04\ \text{m}$.
	\item $0,02\ \text{m}$..
	\item $0,4\ \text{m}$.
	\item $0,2\ \text{m}$.
\end{mcq}
}
\loigiai{
	\textbf{Đáp án A.}
	
	\begin{itemize}
		\item Li độ của con lắc
		\begin{equation*}
			s = l \alpha = 0,02\ \text{m}. 
		\end{equation*}
		\item Tốc độ góc của con lắc
		\begin{equation*}
			\omega = \sqrt{\dfrac{g}{l}} = 7\ \text{rad/s}.
		\end{equation*}
		\item Áp dụng hệ thức độc lập
		\begin{equation*}
			s^2_0 =s^2+\left(\dfrac{v}{\omega}\right)^2 \Rightarrow s_0 = 0,04\ \text{m}.
		\end{equation*}
	\end{itemize}
}
\item \mkstar{2}

\cauhoi{
	
	Con lắc đơn dao động với chu kỳ 2 s khi treo trong thang máy đứng yên. Nếu thang máy đi xuống chậm dần đều và có gia tốc bằng $\dfrac{g}{10}$ thì chu kỳ dao động của con lắc là bao nhiêu?
	
	\begin{mcq}(4)
		\item $T=\text{0,907}\ \text{s}$.
		\item $T=\text{1,907}\ \text{s}$.
		\item $T=\text{2,907}\ \text{s}$.
		\item $T=\text{3,907}\ \text{s}$.
	\end{mcq}
}
\loigiai{
	\textbf{Đáp án B.}
	
	Thang máy đi xuống chậm dần đều 
	
	\begin{equation*}
		T=2\pi\sqrt{\dfrac{l}{g+a} }= 2\pi \sqrt {\dfrac{l}{\dfrac{11g}{10}}} = \sqrt{\dfrac{10}{11}}2\pi \sqrt{\dfrac{l}{g}} =2 \sqrt{\dfrac{10}{11}} \approx \text{1,907}\ \text{s}.
	\end{equation*}
}
\item \mkstar{2}

\cauhoi{
	
	Con lắc đơn chiều dài $l_1$ dao động điều hoà tại một nơi với chu kỳ $T_1 = \text{1,5}\ \text{s}$. Con lắc đơn chiều dài $l_2$ cũng dao động điều hoà tại nơi đó với chu kỳ $T_2 =\text{0,9}\ \text{s}$. Tính chu kỳ của con lắc chiều dài $l$ dao động điều hoà ở nơi trên với $l = l_1+l_2$.
	
	\begin{mcq}(4)
		\item $T=\text{0,75}\ \text{s}$.
		\item $T=\text{1,25}\ \text{s}$.
		\item $T=\text{1,75}\ \text{s}$.
		\item $T=\text{2,00}\ \text{s}$.
	\end{mcq}
}
\loigiai{
	\textbf{Đáp án C.}
	
	 Với $l=l_1+l_2$ sử dụng công thức $T=\sqrt {T^2_1+T^2_2}$.
		
		Thay số vào 
		\begin{equation*}
			T=\text{1,75}\ \text{s}.
		\end{equation*}
}
\item \mkstar{2}

\cauhoi{
	
	Con lắc đơn chiều dài $l_1$ dao động điều hoà tại một nơi với chu kỳ $T_1 = \text{1,5}\ \text{s}$. Con lắc đơn chiều dài $l_2$ cũng dao động điều hoà tại nơi đó với chu kỳ $T_2 =\text{0,9}\ \text{s}$. Tính chu kỳ của con lắc chiều dài $l$ dao động điều hoà ở nơi trên với $l = l_1-l_2$.
	
	\begin{mcq}(4)
		\item $T=\text{0,8}\ \text{s}$.
		\item $T=\text{1,0}\ \text{s}$.
		\item $T=\text{1,1}\ \text{s}$.
		\item $T=\text{1,2}\ \text{s}$.
	\end{mcq}
}
\loigiai{
	\textbf{Đáp án D.}
	
	Với $l=l_1-l_2$ sử dụng công thức $T=\sqrt {T^2_1 - T^2_2}.$
	
	Thay số vào 
	\begin{equation*}
		T= \text{1,2}\ \text{s}.
	\end{equation*}
}
\item \mkstar{2}

\cauhoi{
	
	Một đồng hồ quả lắc chạy đúng giờ trên mặt đất. Biết bán kính Trái Đất là 6400 km và coi nhiệt độ không ảnh hưởng đến chu kỳ của con lắc. Đưa đồng hồ lên đỉnh núi cao $640\ \text m$ so với mặt đất thì mỗi ngày đồng hồ chạy nhanh hay chậm bao nhiêu?
	
	\begin{mcq}(2)
		\item Chạy nhanh $\text{4,84}\ \text{s}$.
		\item Chạy chậm $\text{4,84}\ \text{s}$.
		\item Chạy nhanh $\text{8,64}\ \text{s}$.
		\item Chạy chậm $\text{8,64}\ \text{s}$.
	\end{mcq}
}
\loigiai{
	\textbf{Đáp án D.}
	
	\begin{itemize}
		\item Với chu kỳ của con lắc một ngày $T= \SI{86400}{s}$.
		\item Áp dụng công thức 
		\begin{equation*}
			\dfrac{\Delta T}{T} =\dfrac{h}{R} \Rightarrow \Delta T = T \dfrac{h}{R}.
		\end{equation*}
		\item Thay số vào tìm được 
		\begin{equation*}
			\Delta T = T \dfrac{h}{R}=\text{8,64}\ \text{s} >0.
		\end{equation*}
		\item Đồng hồ chạy chậm $\text{8,64}\ \text{s}$
	\end{itemize}
}
\item \mkstar{2}

\cauhoi{
	
	Một đồng hồ quả lắc chạy đúng giờ trên mặt đất. Đưa đồng hồ xuống giếng sâu $\SI{400}{m}$ so với mặt đất. Coi nhiệt độ không đổi. Bán kính Trái Đất $\SI{6400}{km}$. Sau một ngày đêm đồng hồ đó chạy nhanh hay chậm bao nhiêu?
	
	\begin{mcq}(2)
		\item Chạy nhanh $\text{4,5}\ \text{s}$.
		\item Chạy chậm $\text{4,5}\ \text{s}$.
		\item Chạy nhanh $\text{5,4}\ \text{s}$.
		\item Chạy chậm $\text{5,4}\ \text{s}$.
	\end{mcq}
}
\loigiai{
	\textbf{Đáp án C.}
	
	\begin{itemize}
		\item Với chu kỳ của con lắc một ngày $T= 86400\ \text{s}$.
		\item Áp dụng công thức 
		\begin{equation*}
			\dfrac{\Delta T}{T} =-\dfrac{h}{R} \Rightarrow \Delta T = -T \dfrac{h}{R}.
		\end{equation*}
		\item Thay số vào tìm được 
		\begin{equation*}
			\Delta T = -T \dfrac{h}{R}=-\text{5,4}\ \text{s} <0.
		\end{equation*}
		\item Đồng hồ chạy nhanh $\text{5,4}\ \text{s}$
	\end{itemize}
}
\item \mkstar{2}

\cauhoi{
	
Một đồng hồ quả lắc chạy đúng giờ trên mặt đất ở nhiệt độ $\SI{25}{\celsius}$. Biết hệ số nở dài dây treo con lắc là $\alpha = 2 \cdot 10^{-5}\ \text{K}^{-1}$. Khi nhiệt độ ở đó $\SI{20}{\celsius}$ thì sau một ngày một đêm, đồng hồ sẽ chạy như thế nào?
	\begin{mcq}(2)
		\item Chạy nhanh $\text{4,32}\ \text{s}$.
		\item Chạy chậm $\text{4,32}\ \text{s}$.
		\item Chạy nhanh $\text{3,42}\ \text{s}$.
		\item Chạy chậm $\text{3,42}\ \text{s}$.
	\end{mcq}
}
\loigiai{
	\textbf{Đáp án A.}
	
	\begin{itemize}
		\item Với chu kỳ của con lắc một ngày $T= 86400\ \text{s}$.
		\item Áp dụng công thức
		\begin{equation*}
			\dfrac{\Delta T}{T_1} = \dfrac{T_2-T_1}{T_1} =\dfrac{1}{2} \alpha (t_2 -t_1).
		\end{equation*}
		\item Thay số vào tìm được
		\begin{equation*}
			\Delta T = T_1 \dfrac{1}{2} \alpha (t_2 -t_1)=-\text{4,32}\ \text{s} <0.
		\end{equation*}
		\item Đồng hồ chạy nhanh $\text{4,32}\ \text{s}$
	\end{itemize}
}
\item \mkstar{2}

\cauhoi{
	
	Một con lắc đơn có chiều dài $l = 1\ \text{m}$, đầu trên treo vào trần nhà, đầu dưới gắn với vật có khối lượng $m = \text{0,1}\ \text{kg}$. Kéo vật ra khỏi vị trí cân bằng một góc $\alpha = 45^\circ$ và buông tay không vận tốc đầu cho vật dao động. Biết $g = 10\ \text{m/s}^2$. Hãy xác định động năng của vật khi vật đi qua vị trí có $\alpha = 30^\circ$.
	
	\begin{mcq}(4)
		\item $W_{\text{đ}} =\text{0,133}\ \text{J}$.
		\item $W_{\text{đ}} =\text{0,145}\ \text{J}$.
		\item $W_{\text{đ}} =\text{0,149}\ \text{J}$.
		\item $W_{\text{đ}} =\text{0,159}\ \text{J}$.
	\end{mcq}
}
\loigiai{
	\textbf{Đáp án D.}
	
	\begin{itemize}
		\item Thế năng của con lắc
		\begin{equation*}
			W_{\text{t}} =mgl(1-\cos \alpha).
		\end{equation*}
		\item Cơ năng của con lắc
		\begin{equation*}
			W=mgl (1-\cos \alpha_0).
		\end{equation*}
		\item Động năng của con lắc đơn
		\begin{equation*}
			W_{\text{đ}} =W-W_{\text{t}} = mgl (\cos \alpha - \cos \alpha_0)= \text{0,159}\ \text{J}.
		\end{equation*}
	\end{itemize}
}
\item \mkstar{2}

\cauhoi{
	
	Một con lắc đơn có chiều dài $l = 1\ \text{m}$, đầu trên treo vào trần nhà, đầu dưới gắn với vật có khối lượng $m = \text{0,1}\ \text{kg}$. Kéo vật ra khỏi vị trí cân bằng một góc $\alpha = 45^\circ$ và buông tay không vận tốc đầu cho vật dao động. Biết $g = 10\ \text{m/s}^2$. Hãy xác định cơ năng của vật khi vật đi qua vị trí có $\alpha = 30^\circ$.
	
	\begin{mcq}(4)
		\item $W =\text{0,133}\ \text{J}$.
		\item $W =\text{0,263}\ \text{J}$.
		\item $W =\text{0,293}\ \text{J}$.
		\item $W =\text{0,423}\ \text{J}$.
	\end{mcq}
}
\loigiai{
	\textbf{Đáp án C.}
	
		Cơ năng của con lắc
		\begin{equation*}
			W=mgl (1-\cos \alpha_0)=\text{0,293}\ \text{J}.
		\end{equation*}
}
	\item \mkstar{3}

	\cauhoi{
		
		Chiều dài con lắc đơn $1\ \text{m}$. Phía điểm treo O trên phương thẳng đứng có một chiếc đinh đóng vào điểm O' cách O một khoảng $\text{OO'}=50\ \text{cm}$. Kéo con lắc lệch khỏi phương thẳng đứng một góc $\alpha=30^\circ$ rồi thả nhẹ. Bỏ qua ma sát. Biên độ cong trước và sau khi vướng đinh là
		
		\begin{mcq}(2)
			\item $\text{5,2}\ \text{cm}$ và $\text{3,7}\ \text{cm}$.
			\item $\text{3,0}\ \text{cm}$ và $\text{2,1}\ \text{cm}$.
			\item $\text{5,2}\ \text{cm}$ và $\text{2,7}\ \text{cm}$.
			\item $\text{2,1}\ \text{cm}$ và $\text{3,1}\ \text{cm}$.
		\end{mcq}
	}
	\loigiai{
		\textbf{Đáp án A.}
		
		Biên độ cong ban đầu: $A_1=l_1a_\text{max1}=100\cdot \dfrac{3\pi}{180}=\text{5,2}\ \text{cm}$.
		
		Dao động của con lắc gồm $2$ nửa, một nửa là con lắc có chiều dài $l_1$ và biên độ dài $A_1$, một nửa là con lắc có chiều dài $l_2$  và biên độ dài $A_2$. Vì cơ năng bảo toàn nên:
		
		$W_2=W_1\Leftrightarrow \dfrac{mg}{2l_2}\cdot A_2^2=\dfrac{mg}{2l_1}\cdot A_1^2\Rightarrow  A_2=A_1\cdot \sqrt{\dfrac{l_2}{l_1}}=\text{3,7}\ \text{cm}$.
		
		
	}
	\item \mkstar{3}

	\cauhoi{ 
		Một con lắc đơn dao động điều hòa với biên độ góc $5^\circ$. Khi vật nặng đi qua vị trí cân bằng thì người ta giữ chặt điểm chính giữa của đây treo, sau đó vật tiếp tục dao động điều hòa với biên độ góc $\alpha_0$. Giá trị của $\alpha_0$ là
		\begin{mcq}(4) 
			\item $\text{10,5}^\circ$.
			\item $\text{7,1}^\circ$.
			\item $\text{3,8}^\circ$.
			\item $\text{2,8}^\circ$.
		\end{mcq}
	}
	\loigiai{
		\textbf{Đáp án B.}
		
		Ta có: $v_\text{max}=\sqrt{2gl\left(1-\cos 5^\circ  \right) }$.
		
		Khi qua VTCB ta giữ điểm chính giữa, vận tốc cực đại không đổi.
		
		Khi đó $v_\text{max}=\sqrt{2gl'\left(1-\cos \alpha  \right) }=\sqrt{2g\dfrac{l}{2}\left(1-\cos \alpha  \right) }\Rightarrow \alpha=\text{7,1}^\circ$. 
		
		
		
	}
	\item \mkstar{3}
	
	\cauhoi{
		
		Một con lắc đơn dao động bé có chu kỳ T. Đặt con lắc trong điện trường đều có phương thẳng đứng hướng xuống dưới. Khi quả cầu của con lắc tích điện $q_1$  thì chu kỳ của con lắc là $T_1=4T$. Khi quả cầu của con lắc tích điện $q_2$  thì chu kỳ $T_2=2T/3$. Tỉ số giữa hai điện tích $q_1/q_2$ là
		
		\begin{mcq}(4)
			\item $\dfrac{q_1}{q_2}=-\dfrac{3}{4}$.
			\item $\dfrac{q_1}{q_2}=\dfrac{3}{4}$.
			\item $\dfrac{q_1}{q_2}=-\dfrac{4}{3}$.
			\item $\dfrac{q_1}{q_2}=\dfrac{4}{3}$.
		\end{mcq}
	}
	\loigiai{
		\textbf{Đáp án A.}
		
		Do véc-tơ cường độ điện trường hướng xuống nên ta có: $g'=g+\dfrac{qE}{m}$.
		
		Ta có: $\dfrac{T_1}{T}=\sqrt{\dfrac{g}{g_1}}=\sqrt{\dfrac{g}{g+\dfrac{q_1E}{m}}}\Rightarrow \dfrac{T^2}{T_1^2}=1+\dfrac{q_1E}{m}$.
		
		Tương tự ta có:  $\dfrac{T^2}{T_2^2}=1+\dfrac{q_2E}{m}$.
		
		Do đó: $\dfrac{q_1}{q_2}=\dfrac{\dfrac{T^2}{T_1^2} -1}{\dfrac{T^2}{T_2^2}-1}=-\dfrac{3}{4}$. 	
	}
	
	\item \mkstar{3}
	
	\cauhoi{
		
		Một con lắc đơn vật nhỏ có khối lượng $m$ mang điện tích $q>0$ được coi là điện tích điểm. Ban đầu con lắc dao động dưới tác dụng chỉ của trọng trường có biên độ $\alpha_0$. Khi con lắc có li độ góc $\dfrac{\alpha_0}{2}$, tác dụng điện trường đều mà véc-tơ cường độ điện trường có độ lớn $E$ và hướng thẳng đứng xuống dưới. Biết $qE=mg$. Biên độ góc của con lắc sau đó tăng hay giảm bao nhiêu \% so với ban đầu ?
		
		\begin{mcq}(4)
			\item Giảm $19 \%$.
			\item Tăng $19\%$.
			\item Tăng $21\%$.
			\item Giảm $21\%$.
		\end{mcq}
		
	}
	\loigiai{
		\textbf{Đáp án D.}
		
		Tại  $\alpha=\text{0,5}\alpha_0\Rightarrow E_\text{t}=\dfrac{E}{4}, \ E_\text{đ}=\dfrac{3E}{4}$.
		
		Khi tác dụng điện trường đều thẳng đứng hướng xuống $qE=mg$ thì con lắc dao động với 
		
		$g_\text{hd}=g+a=g+\dfrac{qE}{m}=2g\Rightarrow E'_\text{t}=2E_\text{t}\Rightarrow E'= E'_\text{t}+ E_\text{đ}=\dfrac{5}{4}E_\text{t}$.
		
		$\Rightarrow \dfrac{E'}{E}=\dfrac{mg'l\alpha'}{mgl\alpha}=\dfrac{5}{4}\Leftrightarrow \dfrac{\alpha'}{\alpha}=\sqrt{\dfrac{5}{8}}=\text{0,79}$.
		
		Vậy Biên độ góc của con lắc sau đó giảm $21\%$. 
		
	}
	
	\item \mkstar{3}
	
	\cauhoi{
		
		Một con lắc đơn gồm dây treo có độ dài $25\ \text{cm}$ và một vật nhỏ, treo tại nơi có gia tốc trọng trường bằng $10\ \text{m/s}^2$. Lấy gần đúng $\pi^2=10$. Đưa vật theo chiều dương tới vị trí dây treo lệch với phương thẳng đứng một góc bằng $8^\circ$ rồi buông nhẹ. Thời điểm ban đầu, $t = 0$ được chọn sau thời điểm vật bắt đầu chuyển động $\dfrac{1}{3}\ \text{s}$. Phương trình li độ góc của chất điểm là
		
		\begin{mcq}(2)
			\item $\alpha=4^\circ \cos \left( 2\pi t+\dfrac{2\pi}{3}\right) $.
			\item $\alpha=8^\circ \cos \left( 2\pi t+\dfrac{\pi}{2}\right) $.
			\item $\alpha=8^\circ \cos \left( 2\pi t-\dfrac{2\pi}{3}\right) $.
			\item $\alpha=8^\circ \cos \left( 2\pi t+\dfrac{2\pi}{3}\right) $.
		\end{mcq}
		
	}
	\loigiai{
		\textbf{Đáp án D.}
		
		Kéo con lắc ra $8^\circ$, rồi buông, suy ra trong quá trình dao động con lắc lệch góc lớn nhất bằng $8^\circ$, suy ra $\alpha_0=8^\circ$.
		
		$\omega=\sqrt{\dfrac{g}{l}}=2\pi \ \text{rad/s}$.
		
		Ban đầu kéo ra $8^\circ$ theo chiều dương, suy ra vật ở biên dương. Sau một khoảng thời gian $\dfrac{1}{3}\ \text{s}$, M quay được một góc $\Delta \varphi=\omega t=\dfrac{2\pi}{3}\ \text{rad/s}$.
		
		Lúc này tính thời gian dao động $(t = 0)$ nên  $\varphi=\dfrac{2\pi}{3}\ \text{rad/s}$.
		
		Vậy phương trình li độ góc của chất điểm là  $\alpha=8^\circ \cos \left( 2\pi t+\dfrac{2\pi}{3}\right) $.	
	}
	
	\item \mkstar{3}
	
	\cauhoi{
		
		Một con lắc đơn dao động điều hòa trong một thang máy đứng yên tại nơi có gia tốc $g=\text{9,8}\ \text{m/s}^2$ với năng lượng dao động $150\ \text{mJ}$. Thang máy bắt đầu chuyển động nhanh dần đều lên trên với gia tốc $\text{2,5}\ \text{m/s}^2$. Biết thời điểm thang máy bắt đầu chuyển động là lúc con lắc có li độ bằng nửa li độ cực đại. Con lắc sẽ tiếp tục dao động trong thang máy với năng lượng
		\begin{mcq}(4)
			\item $\text{140,4}\ \text{mJ}$.
			\item $\text{188}\ \text{mJ}$.
			\item $\text{112}\ \text{mJ}$.
			\item $\text{159,6}\ \text{mJ}$.
		\end{mcq}
		
	}
	\loigiai{
		\textbf{Đáp án A.}
		
		$g'=g-a=\text{7,3}\ \text{m/s}^2$.
		
		$\Delta W_\text{t}=\dfrac{m\Delta g l}{2}\cdot a^2=\dfrac{m\left( g'-g\right) }{2}\cdot \dfrac{a_\text{max}^2}{4}=-\dfrac{25}{392}$.
		
		$W'=W+\Delta W_\text{t}=150-\dfrac{25}{392}\cdot 150=\text{140,4}\ \text{mJ}$.
		
		
	}
	\item \mkstar{3}
	
	\cauhoi{
		
		Một con lắc đơn có chiều dài dây treo là $l$ và vật nặng có khối lượng $m$, khối lượng riêng của vật là $D=8\ \text{g/cm}^3$. Khi đặt trong chân không con lắc đơn dao động với chu kì $T = 2\ \text{s}$. Lấy $g=\text{9,8}\ \text{m/s}^2$. Tìm chu kì dao động của con lắc khi nó dao động trong nước. Biết khối lượng riêng của nước là  $D=1000\ \text{kg/m}^3$.
		\begin{mcq}(4)
			\item $\text{2,309}\ \text{s}$.
			\item $\text{2,138}\ \text{s}$.
			\item $\text{1,875}\ \text{s}$.
			\item $\text{1,678}\ \text{s}$.
		\end{mcq}
		
	}
	\loigiai{
		\textbf{Đáp án B.}
		
		Lực đẩy Acsimet hướng lên tác dụng lên vật là  $F_\text{A}=DVg$.
		
		Ta có: $P'=P-F_\text{A}\Rightarrow g'=g-\dfrac{DVg}{\rho V}=\left(1-\dfrac{D}{\rho} \right)\rho\Rightarrow \dfrac{T'}{T}=\sqrt{\dfrac{1}{1-\dfrac{D}{\rho}}}$.
		
		$\Rightarrow T'=T\sqrt{\dfrac{1}{1-\dfrac{D}{\rho}}}=\text{2,138}\ \text{s}$.
		
		
	}
	\item \mkstar{3}
	
	\cauhoi{
		
		Một con lắc đơn được treo trên trần một thang máy. Khi thang máy chuyển động thẳng đứng đi xuống nhanh dần đều với gia tốc có độ lớn $a$ thì chu kì dao động điều hòa của con lắc là $4\ \text{s}$. Khi thang máy chuyển động thẳng đứng đi xuống chậm dần đều với gia tốc có cùng độ lớn $a$ thì chu kì dao động điều hòa của con lắc là $2\ \text{s}$. Khi thang máy đứng yên thì chu kì dao động điều hòa của con lắc là
		
		\begin{mcq}(4)
			\item $\text{4,32}\ \text{s}$.
			\item $\text{3,16}\ \text{s}$.
			\item $\text{2,53}\ \text{s}$.
			\item $\text{2,66}\ \text{s}$.
		\end{mcq}
		
	}
	\loigiai{
		\textbf{Đáp án C.}
		
		
		Khi thang máy chuyển động thẳng đứng đi lên nhanh dần đều với gia tốc có độ lớn $a$:
		
		$T_1=2\pi\sqrt{\dfrac{l}{g-a}}=4\ \text{s}$.
		
		Khi thang máy chuyển động thẳng đứng đi lên chậm dần đều với gia tốc có cùng độ lớn $a$:
		
		$T_2=2\pi\sqrt{\dfrac{l}{g+a}}=2\ \text{s}$.
		
		$\Rightarrow \dfrac{T_1}{T_2}=\sqrt{\dfrac{g+a}{g-a}}=2\Rightarrow a=\dfrac{3g}{5}$.
		
		Khi thang máy đứng yên: $T_3=2\pi\sqrt{\dfrac{l}{g}}\Rightarrow \dfrac{T_1}{T_3}=\sqrt{\dfrac{g}{g-a}}\Rightarrow T_3= \text{2,53}\ \text{s}$.  
		
	}
	\item \mkstar{4}
	
	\cauhoi{
		
		Một con lắc đơn gồm dây treo chiều dài $1\ \text{m}$, vật nặng khối lượng $100\ \text{g}$ được tích điện $q=10\ \mu \text{C}$. Con lắc đơn được đặt vào một điện trường đều có véc-tơ cường độ điện trường nằm ngang, độ lớn cường độ điện trường $E = \SI{26795}{V/m}$. Con lắc đơn đang đứng yên ở vị trí cân bằng, người ta kéo con lắc đến vị trí sao cho dây treo hợp với phương thẳng đứng góc $30^\circ$ rồi thả nhẹ. Tốc độ cực đại của con lắc trong quá trình dao động có thể là       
		\begin{mcq}(4)
			\item $\text{0,76}\ \text{m/s}$.
			\item $\text{1,06}\ \text{m/s}$.
			\item $\text{2,46}\ \text{m/s}$.
			\item $\text{1,66}\ \text{m/s}$.
		\end{mcq}
		
	}
	\loigiai{
		\textbf{Đáp án C.}
		
		Ta có: $F_\text{đ}=qE=\text{0,26795}\ \text{N}$.
		
		$P=mg=1\ \text{N}$.
		
		Đặt con lắc vào trong điện trường nằm ngang thì vị trí cân bằng  của con lắc bị lệch so với phương thẳng đứng góc $\theta$ sao cho: $\tan \theta=\dfrac{F_\text{đ}}{P}\Rightarrow \theta \approx 15^\circ$.
		
		Và con lắc đơn dao động với $g'=\dfrac{g}{\cos \theta}\approx \text{10,35}\ \text{m/s}^2$.
		
		Kích thích dao động bằng cách kéo theo phương thẳng đứng góc $30^\circ$, ta có hai trường hợp xảy ra: 
		
		Trường hợp 1: Kéo cùng bên so với bên lệch của vị trí cân bằng, suy ra $\alpha_0=15^\circ$.
		
		$v_\text{max}=\sqrt{2g'\left(1-\cos \alpha_0 \right) }\approx \text{0,84}\ \text{m/s}$.
		
		Trường hợp 2: Kéo khác bên so với bên lệch của vị trí cân bằng, suy ra $\alpha_0=45^\circ$.
		
		$v_\text{max}=\sqrt{2g'\left(1-\cos \alpha_0 \right) }\approx \text{2,46}\ \text{m/s}$.
		
	}
	
	\item \mkstar{4}
	
	\cauhoi{ 
		
		Một con lắc đơn gồm một vật nhỏ khối lượng $m=2\ \text{g}$ và một dây treo mảnh, chiều dài $l$, được kích thích cho dao động điều hòa. Trong khoảng thời gian $\Delta t$ con lắc thực hiện được $40$ dao động. Khi tăng chiều dài con lắc thêm một đoạn $\text{7,9}\ \text{cm}$ thì cũng trong khoảng thời gian $\Delta t$  con lắc thực hiện được $39$ dao động. Lấy gia tốc trọng trường $g=\text{9,8}\ \text{m/s}^2$. Để con lắc với chiều dài tăng thêm có cùng chu kỳ dao động với con lắc chiều dài $l$, người ta truyền cho vật điện tích $q=\text{0,5}\cdot 10^{-8}\ \text{C}$ rồi cho nó dao động điều hòa trong một điện trường đều có đường sức thẳng đứng. Véc-tơ cường độ điện trường này có chiều
		
		
		\begin{mcq}(2)
			\item hướng lên, độ lớn $\text{1,02}\cdot 10^5\ \text{V/m}$.    
			\item hướng xuống, độ lớn $\text{1,02}\cdot 10^5\ \text{V/m}$. 
			\item hướng xuống, độ lớn $\text{2,04}\cdot 10^5\ \text{V/m}$. 
			\item hướng lên, độ lớn $\text{2,04}\cdot 10^5\ \text{V/m}$. 
		\end{mcq}
	}
	\loigiai{
		\textbf{Đáp án C.}
		
		Ta có: $T_1=\dfrac{\Delta t}{40}=2\pi \cdot \sqrt{\dfrac{l}{g}}$  và $T_2=\dfrac{\Delta t}{39}=2\pi \cdot \sqrt{\dfrac{l+\text{0,079}}{g}}$.
		
		Suy ra $\dfrac{T_1}{T_2}=\dfrac{39}{40}=\sqrt{\dfrac{l}{l+\text{l+\text{0,079}}}}$.
		
		Lại có: $T_q=2\pi\cdot \sqrt{\dfrac{l+\text{0,079}}{g+\dfrac{qE}{m}}}=T_1=2\pi \cdot \sqrt{\dfrac{l}{g}} \Leftrightarrow \dfrac{l+\text{0,079}}{g+\dfrac{qE}{m}} = \dfrac{l}{g} \Rightarrow E= \text{2,04}\cdot 10^5\ \text{V/m} $. 
		
		
		Do đó $E$ có chiều hướng xuống và độ lớn $E= \text{2,04}\cdot 10^5\ \text{V/m}$.	
		
	}
	
\end{enumerate}
\loigiai{\textbf{Đáp án}
	\begin{center}
		\begin{tabular}{|m{2.8em}|m{2.8em}|m{2.8em}|m{2.8em}|m{2.8em}|m{2.8em}|m{2.8em}|m{2.8em}|m{2.8em}|m{2.8em}|}
			\hline
			1. C & 2. A & 3. B & 4. C & 5. D & 6. D & 7. C & 8. A & 9. D & 10. C \\
			\hline
			11. A & 12. B & 13. A & 14. D & 15. D & 16. A & 17. B & 18. C & 19. C & 20. C\\
			\hline
		\end{tabular}
\end{center}}