\begin{enumerate}[label=\bfseries Câu \arabic*:]
	
	\item \mkstar{1}
	
\cauhoi{Một mạch điện xoay chiều chỉ có cuộn thuần cảm, mối quan hệ về pha của $u$ và $i$ trong mạch là
		\begin{mcq}(2)
			\item $u$ và $i$ cùng pha với nhau.
			\item $u$ sớm pha hơn $i$ góc $\pi/2$.
			\item $u$ và $i$ ngược pha với nhau.
			\item $i$ sớm pha hơn $u$ góc $\pi/2$.
		\end{mcq}
}		
		\loigiai{
		\textbf{Đáp án: B.}
			
			Một mạch điện xoay chiều chỉ có cuộn thuần cảm, mối quan hệ về pha của $u$ và $i$ trong mạch là $u$ sớm pha hơn $i$ góc $\pi/2$.
			
			}
	
	%%%%%%%%%%%%%%CÂU2%%%%%%%%%%
	\item \mkstar{2}
	
	\cauhoi{Mạch điện $X$ chỉ có một phần tử có phương trình dòng điện và hiệu điện thế lần lượt như sau $i=2\sqrt{2}\cos\left(100\pi t+\dfrac{\pi}{6}\right)\,\text{A}$ và $u=200\sqrt{2}\cos\left(100\pi t+\dfrac{\pi}{6}\right)\,\text{V}$. Hãy xác định đó là phần tử gì và độ lớn bao nhiêu?
		\begin{mcq}(4)
			\item $Z_L=\SI{100}{\ohm}$.
			\item $Z_C=\SI{100}{\ohm}$.
			\item $R=\SI{100}{\ohm}$.
			\item $R=100\sqrt{2}\SI{}{\ohm}$.
		\end{mcq}
}		
		\loigiai{
			\textbf{Đáp án: C.}
			
			Vì $u$ và $i$ cùng pha nên đây là $R$, $R=\dfrac{U_0}{I_0}=\SI{100}{\ohm}.$
			
		}
	
	
	%%%%%%%%%%%%%%CÂU3%%%%%%%%%%
	\item \mkstar{2}
	
	\cauhoi{Một mạch điện chỉ có cuộn cảm có hệ số tự cảm $L=\dfrac{1}{\pi}\,\text{H}$ mắc vào mạng điện và có phương trình dòng điện $i=2\cos\left(100\pi t+\dfrac{\pi}{6}\right)\,\text{A}$. Hãy viết phương trình hiệu điện thế giữa hai đầu mạch điện.
		\begin{mcq}(2)
			\item $u_L=200\cos\left(100\pi t+\dfrac{2\pi}{3}\right)\,\text{V}$.
			\item $u_L=200\cos\left(100\pi t+\dfrac{\pi}{6}\right)\,\text{V}$.
			\item $u_L=200\sqrt{2}\cos\left(100\pi t+\dfrac{2\pi}{3}\right)\,\text{V}$.
			\item $u_L=200\sqrt{2}\cos\left(100\pi t+\dfrac{\pi}{6}\right)\,\text{V}$.
		\end{mcq}
}		
		\loigiai{
				\textbf{Đáp án: A.}
			
			$u_L$ có dạng $u_L=U_{0L}\cos\left(100\pi t+\dfrac{\pi}{6}+\dfrac{\pi}{2}\right)\,\text{V}$.
			
			Trong đó
			$U_{0L}=I_0Z_L=\SI{200}{V}$.
			
			Vậy $u_L=200\cos\left(100\pi t+\dfrac{2\pi}{3}\right)\,\text{V}$.
			
		}
	
	%%%%%%%%%%%%%%CÂU4%%%%%%%%%%
	\item \mkstar{2}
	
	\cauhoi{Mạch điện $X$ chỉ có tụ điện $C$, biết $C=\dfrac{10^{-4}}{\pi}\,\text{F}$, mắc mạch điện trên vào mạng điện có phương trình $u=100\sqrt{2}\cos\left(100\pi t+\dfrac{\pi}{6}\right)\,\text{V}$. Xác định phương trình dòng điện trong mạch.
		\begin{mcq}(2)
			\item $i=\sqrt{2}\cos\left(100\pi t+\dfrac{2\pi}{3}\right)\,\text{A}$.
			\item $i=\sqrt{2}\cos\left(100\pi t+\dfrac{\pi}{6}\right)\,\text{A}$.
			\item $i=\cos\left(100\pi t+\dfrac{2\pi}{3}\right)\,\text{A}$.
			\item $i=\cos\left(100\pi t+\dfrac{\pi}{6}\right)\,\text{A}$.
		\end{mcq}
}		
		\loigiai{
			\textbf{Đáp án: A.}
			
			Phương trình dòng điện có dạng $i=I_0\cos\left(100\pi t+\dfrac{\pi}{6}+\dfrac{\pi}{2}\right)\,\text{A}$.
			
			Trong đó
			$I_0=\sqrt{2}\,\text{A}$.
			
			Vậy $i=\sqrt{2}\cos\left(100\pi t+\dfrac{2\pi}{3}\right)\,\text{A}$.		
			
			}	
	
	
	%%%%%%%%%%%%%%CÂU5%%%%%%%%%%
	\item \mkstar{2}
	
	\cauhoi{Cho một cuộn dây có điện trở thuần $\SI{40}{\ohm}$ và có độ tự cảm $\dfrac{0,4}{\pi}\,\text{H}$. Đặt vào hai đầu cuộn dây điện áp xoay chiều có biểu thức $u=U_0\cos\left(100\pi t-\dfrac{\pi}{2}\right)\,\text{V}$. Khi $t=\SI{0,1}{s}$ dòng điện có giá trị $2,75\sqrt{2}\,\text{A}$. Giá trị của $U_0$ là
		\begin{mcq}(4)
			\item $\SI{220}{V}$.
			\item $110\sqrt{2}\,\SI{}{V}$.
			\item $220\sqrt{2}\,\SI{}{V}$.
			\item $440\sqrt{2}\,\SI{}{V}$.
		\end{mcq}
}	
		\loigiai{
		\textbf{Đáp án: B.}
			
			Tổng trở của mạch là $Z=\sqrt{R^2+Z_L^2}=40\sqrt{2}\,\Omega$.
			
			Phương trình dòng điện có dạng $i=I_0\cos\left(100\pi t-\pi\right)\,\text{A}$.
			
			Do đó $I=I_0\cos0=2,75\sqrt{2}\,\text{A}\Rightarrow I_0=-2,75\sqrt{2}\,\text{A}$.
			
			Giá trị của $U_0$ là $U_0=110\sqrt{2}\,\SI{}{V}$.
			
			}
	\item \mkstar{2}	
	
	\cauhoi{		
		Đặt điện áp xoay chiều $u = 120 \sqrt 2 \cos(100 \pi t + \dfrac{\pi}{6})$ (V) vào hai đầu đoạn mạch chỉ có tụ điện $C = \dfrac{10^{-4}}{\pi}$ F. Dòng điện qua tụ có biểu thức
		\begin{mcq}(2)
			\item $i = 1,2 \sqrt 2 \cos \left( 100 \pi t + \dfrac{2 \pi }{3}\right) $ A.
			\item $i = 1,2 \cos \left( 100 \pi t - \dfrac{2 \pi }{3}\right) $ A.
			\item $i = 1,2 \sqrt 2 \cos \left( 100 \pi t + \dfrac{ \pi }{3}\right) $ A.
			\item $i = 1,2 \sqrt 2 \cos \left( 100 \pi t - \dfrac{ \pi }{3}\right) $ A.
		\end{mcq}		
	}
	
	\loigiai{
		\textbf{Đáp án A.}
		
		Tổng trở của mạch: $Z=Z_C =100\ \Omega$ 
		
		Cường độ dòng điện cực đại: $I_0=\dfrac {U_0}{Z}=1,2\sqrt 2$ A.
		
		Trong đoạn mạch chỉ chứa tụ điện, dòng điện luôn sớm pha hơn điện áp một góc $\dfrac {\pi}{2}$: 
		$$\varphi _i = \varphi _u + \dfrac{\pi}{2} = \frac {\pi}{6}+ \dfrac {\pi}{2}=\frac {2\pi}{3}\,\text{rad}.$$
		
		Vậy $i = 1,2 \sqrt 2 \cos \left( 100 \pi t + \dfrac{2 \pi }{3}\right) $ A.
	}
	
	%%%%%%%%%%%Câu 2%%%%%%%%%%%%%%
	\item \mkstar{3}
	
	\cauhoi{	
		Một đoạn mạch gồm các phần tử: điện trở thuần $R$, cuộn cảm thuần $L$ và tụ điện $C$ mắc nối tiếp. Đặt vào hai đầu đoạn mạch một điện áp xoay chiều. Điện áp hiệu dụng trên các phần tử $U_R=U_L=\dfrac {1}{2}U_C$. So với điện áp giữa hai đầu đoạn mạch, dòng điện qua mạch
		\begin{mcq}(4)
			\item sớm pha $\dfrac {\pi}{4}$.
			\item chậm pha $\dfrac {\pi}{4}$.
			\item chậm pha $\dfrac {\pi}{4}$.
			\item chậm pha $\dfrac {\pi}{6}$.
		\end{mcq}
	}
	
	\loigiai{
		\textbf{Đáp án: A}
		
		Độ lệch pha giữa $u$ và $i$:
		$$\tan \varphi =\frac{U_L-U_C}{U_R}=\frac{U_R-2U_C}{U_R}=-1 \\ \Rightarrow \varphi=-\frac {\pi}{4}\ \text {rad}$$
		
		Vậy $i$ sớm pha hơn $u$ một góc $\dfrac {\pi}{4}$
	}
	
	
	
	%%%%%%%%%%%Câu 3%%%%%%%%%%%%%%
	\item \mkstar{3}
	
	\cauhoi{	
		Đặt điện áp $u = 100 \sqrt 2 \cos 100 \pi t\,\text{V}$ ($t$ tính bằng giây) vào hai đầu đoạn mạch mắc nối tiếp gồm điện trở $80\, \Omega$, tụ điện có điện dung $\dfrac{10^{-4}}{2 \pi}\,\text{F}$, cuộn dây có độ tự cảm $\dfrac{1}{\pi}\text{H}$. Khi đó, cường độ dòng điện trong đoạn mạch sớm pha $\dfrac{\pi}{4}$ so với điện áp giữa hai đầu đoạn mạch. Điện trở cuộn dây có giá trị là	
		\begin{mcq}(4)
			\item $100\, \Omega$.
			\item $80\, \Omega$.
			\item $20\,\Omega$.
			\item $40\, \Omega$.
		\end{mcq}
		
	}
	
	\loigiai{
		\textbf{Đáp án C.}
		
		Dung kháng của tụ điện $$Z_C=\dfrac{1}{\omega C}=200\,\Omega.$$ 
		Cảm kháng của cuộn dây $$Z_L=\omega L=100\,\Omega.$$
		Cường độ dòng điện sớm pha hơn điện áp nên $\varphi<0$ (do $\varphi=\varphi_u-\varphi_i$)
		$ \Rightarrow \varphi=-\pi/4\,\textrm{rad}$.
		
		Áp dụng công thức:
		$$\tan \varphi=\frac{Z_{L}-Z_{C}}{R+r}\Rightarrow r =20\, \Omega.$$
	}
	\item \mkstar{3} 
	
	\cauhoi{	
		Một đoạn mạch gồm một điện trở thuần $R = 25\ \Omega$, mắc nối tiếp với tụ điện có điện dung $C = \dfrac{10^{-4}}{\pi}$ F và cuộn dây thuần cảm có hệ số tự cảm L. Đặt vào hai đầu đoạn mạch đó một điện áp xoay chiều có tần số $f = 50$ Hz thì điện áp giữa hai đầu điện trở thuần R sớm pha $\dfrac{\pi}{4}$ so với điện áp giữa hai đầu đoạn mạch. Giá trị cảm kháng của cuộn dây là
		\begin{mcq}(4)
			\item $125\ \Omega$.
			\item $75\ \Omega$.
			\item $125\ \Omega$.
			\item $100\ \Omega$.
		\end{mcq}
		
	}
	
	\loigiai{
		\textbf{Đáp án B.}
		
		Tổng trở của mạch: $$Z=\sqrt {R^2+(Z_L-Z_C)^2} =\sqrt {25^2+(Z_L-100)^2}\ \Omega.$$ 
		
		Độ lệch pha giữa dòng điện và điện áp: $$\tan \varphi=\frac {Z_L-Z_C}{R} =\frac {Z_L-100}{25}.$$ 
		
		Điện áp giữa hai đầu điện trở (cũng là cường độ dòng điện) sớm pha $\dfrac{\pi}{4}$ so với điện áp giữa hai đầu đoạn mạch nên
		
		$$\tan \varphi = -1 \Rightarrow Z_L=75\ \Omega.$$
		
	}
	
	%%%%%%%%%%%Câu 10%%%%%%%%%%%%%%
	\item \mkstar{4}
	
	\cauhoi{
		Cho nhiều hộp kín giống nhau, trong mỗi hộp chứa một trong ba phần tử $R_0, L_0$ hoặc $C_0$. Lấy một hộp bất kì mắc nối tiếp với một cuộn dây thuần cảm có độ tự cảm $L = \dfrac{\sqrt 3}{\pi}\ \text H$. Đặt vào hai đầu đoạn mạch điện áp xoay chiều có biểu thức dạng $u = 200 \sqrt 2 \cos (100 \pi t)\ \text{(V)}$ thì dòng điện trong mạch có biểu thức $i = I_0 \cos (100 \pi t - \pi/3)\ \text{(A)}$. Phần tử trong hộp kín đó là
		
		\begin{mcq}(4)
			\item $C_0 = \dfrac{100}{\pi}\ \mu F$.
			\item $L = \dfrac{1}{\sqrt 3 \pi}\ H$.
			\item $R_o = 100 \sqrt 3\ \Omega$.
			\item $R_0 = 100\ \Omega$.
		\end{mcq}
		
	}
	
	\loigiai{
		\textbf{Đáp án D.}
		
		Do điện áp nhanh pha hơn dòng điện một góc $\dfrac{\pi}{3}\ \text{rad}$, không phải góc $\dfrac {\pi}{2}$ nên trong mạch chứa hai phần tử là $L$ và $R$.
		
		Áp dụng công thức: $$\tan \varphi =\dfrac {Z_L}{R}=\tan \dfrac{\pi}{3}=\sqrt 3\Rightarrow R=\dfrac{Z_L}{\sqrt 3}=100\ \Omega.$$
	}
\end{enumerate}
\loigiai{\textbf{Đáp án}
	\begin{center}
		\begin{tabular}{|m{2.8em}|m{2.8em}|m{2.8em}|m{2.8em}|m{2.8em}|m{2.8em}|m{2.8em}|m{2.8em}|m{2.8em}|m{2.8em}|}
			\hline
			1. B & 2. C & 3. A & 4. A & 5. B & 6. A  & 7. A  & 8. C & 9. B & 10. D\\
			\hline
		\end{tabular}
\end{center}}