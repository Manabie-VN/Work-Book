\setcounter{section}{0}

\begin{enumerate}[label=\bfseries Câu \arabic*:]
	%1
		\item \mkstar{1}
	\cauhoi{
		
		
		Trong dao động điều hoà của con lắc lò xo, phát biểu nào sau đây là \textbf{không} đúng?
		
		\begin{mcq}
			\item Lực kéo về phụ thuộc vào khối lượng của vật nặng.
			\item Lực  kéo về phụ thuộc vào độ cứng của lò xo.
			\item Gia tốc của vật phụ thuộc vào khối lượng của vật.
			\item Tần số góc của vật phụ thuộc vào khối lượng của vật.
		\end{mcq}
		
	}
	
	\loigiai{
		\textbf{Đáp án A.}
		
		Lực kéo về $F=-kx$ không phụ thuộc vào khối lượng.
	}

	\item \mkstar{2}
	\cauhoi{
		
		Một con lắc lò xo nằm ngang có độ cứng $k = 100\ \text{N/m}$ được gắn vào vật nặng có khối lượng $m=\SI{0,1}{kg}$. Kích thích cho vật dao động điều hòa, xác định chu kỳ của con lắc lò xo. Lấy $\pi^2 =10$.
		\begin{mcq}(4)
			\item 0,1 s.
			\item 5 s.
			\item 0,2 s.
			\item 0,3 s.
		\end{mcq}
	}
	\loigiai{
		\textbf{Đáp án C.}
		
		Ta có: $T=2\pi \sqrt {\dfrac{m}{k}} = \SI{0,2}{s}$.
	}
	%2
	\item \mkstar{2}
	\cauhoi{
		
		Một con lắc lò xo có khối lượng không đáng kể, độ cứng là $k$, lò xo treo thẳng đứng, bên dưới treo vật nặng có khối lượng $m$. Ta thấy ở vị trí cân bằng lò xo giãn ra một đoạn 16 cm, kích thích cho vật dao động điều hòa. Xác định tần số của con lắc lò xo.
		\begin{mcq}(4)
			\item 2,5 Hz.
			\item 5 Hz.
			\item 3 Hz.
			\item 1,25 Hz.
		\end{mcq}
	}
	\loigiai{
		\textbf{Đáp án D.}
		
		Tần số của vật
		
		$$f=\dfrac{1}{2\pi} \sqrt{\dfrac{g}{\Delta l}}=\SI{1,25}{Hz}.$$
		
	}
	%3
	\item \mkstar{2}
	\cauhoi{
		
		Một con lắc lò xo có độ cứng là $k$. Một đầu gắn cố định, một đầu gắn với vật nặng có khối lượng $m$. Kích thích cho vật dao động, nó dao động điều hòa với chu kỳ $T$. Hỏi nếu tăng gấp đôi khối lượng của vật và giảm độ cứng đi 2 lần thì chu kỳ của con lắc lò xo sẽ thay đổi như thế nào?
		
		\begin{mcq}(2)
			\item Không đổi
			\item Tăng lên 2 lần.	
			\item Giảm đi 2 lần.
			\item Giảm đi 4 lần.
		\end{mcq}
	}
	\loigiai{
		\textbf{Đáp án B.}
		
		Gọi chu kỳ ban đầu của con lắc lò xo là $T=2\pi \sqrt{\dfrac{m}{k}}$.
		
		Gọi $T'$ là chu kỳ của con lắc sau khi thay đổi khối lượng và độ cứng của lò xo
		
		$$T'=2\pi \sqrt{\dfrac{m'}{k'}}$$
		
		Trong đó $m'=2m$; $k'=\dfrac{k}{2} \Rightarrow T'=2T$.
		
		Chu kỳ dao động tăng lên 2 lần.
	}
	%4
	\item \mkstar{2}
	\cauhoi{
		
		Một lò xo có độ cứng $k$. Khi gắn vật $m_1$ vào lò xo và cho dao động thì chu kỳ dao động là $\SI{0,3}{s}$. Khi gắn vật có khối lượng $m_2$ vào lò xo trên và kích thích cho dao động thì nó dao động và chu kỳ là 0,4 s. Hỏi nếu khi gắn vật có khối lượng $m=2m_1+3m_2$ thì nó dao động với chu kỳ là bao nhiêu?
		\begin{mcq}(4)
			\item 0,25 s.
			\item 0,4 s.
			\item 0,812 s.
			\item 0,3 s.
		\end{mcq}	
	}
	\loigiai{
		\textbf{Đáp án C.}
		
		$$T^2=2T_1^2+3T^2_2 \Rightarrow T=\SI{0,812}{s}.$$
	}
	

	%2
	\item \mkstar{3}
	\cauhoi{
		
		Một con lắc lò xo có chiều dài tự nhiên $l_0 =\SI{30}{cm}$, độ cứng của lò xo là $k=10\ \text{N/m}$. Treo vật nặng có khối lượng 0,1 kg vào lò xo và kích thích lò xo dao động điều hòa theo phương thẳng đứng với biên độ $A=\SI{5}{cm}$. Xác định lực đàn hồi cực đại, cực tiểu của lò xo trong quá trình dao động của vật.
		\begin{mcq}(2)
			\item 1,5 N; 0,5 N.
			\item 2 N; 1,5 N.
			\item 2,5 N; 0,5 N.
			\item Không có đáp án đúng.
		\end{mcq}
	}
	\loigiai{
		\textbf{Đáp án A.}
		
		Ta có $\Delta l = \SI{0,1}{m} <A$ nên $F_{\text{đh}_\text{max}} = k(A+\Delta l) = \SI{1,5}{N}.$
		
		và $F_{\text{đh}_\text{min}} =\SI{0,5}{N}$.
		
		
		
	}
	%3
	\item \mkstar{3}
	
	\cauhoi{
		
		Một con lắc lò xo có chiều dài tự nhiên $l_0 =\SI{30}{cm}$, độ cứng của lò xo là $k=10\ \text{N/m}$. Treo vật nặng có khối lượng 0,1 kg vào lò xo và kích thích lò xo dao động điều hòa theo phương thẳng đứng với biên độ $A=\SI{20}{cm}$. Xác định lực đàn hồi cực đại, cực tiểu của lò xo trong quá trình dao động của vật.
		
		\begin{mcq}(2)
			\item 1,5 N; 0 N.
			\item 2 N; 0 N.
			\item 3 N; 0 N.
			\item Không có đáp án đúng.
		\end{mcq}
	}	
	\loigiai{
		\textbf{Đáp án C.}
		
		Ta có $\Delta l = \SI{0,1}{m} <A$ nên $F_{\text{đh}_\text{max}} = k(A+\Delta l) = \SI{3}{N}.$
		
		và $F_{\text{đh}_\text{min}} =0 $ vì $\Delta l<A$.
		
	} 
	
	%4
	
	\item \mkstar{3}
	
	\cauhoi{
		
		Con lắc lò xo treo thẳng đứng, tại vị trí cân bằng lò xo dãn $\Delta l_0$. Kích thích để quả nặng dao động điều hoà theo phương thẳng đứng với chu kỳ $T$. Thời gian lò xo bị nén trong một chu kỳ là $\dfrac{T}{4}$. Biên độ dao động của vật là
		
		\begin{mcq}(4)
			\item $A=\sqrt{2}\Delta l_0$.
			\item $A=2\Delta l_0$.
			\item $A=\sqrt{3}\Delta l_0$.
			\item $A=3\Delta l_0$.
		\end{mcq}
		
	}
	
	\loigiai{
		\textbf{Đáp án A.}
		
		Khoảng thời gian lò xo bị nén trong một chu kì là $\Delta t= \dfrac{T}{4} =\dfrac{2\alpha}{\omega}\Rightarrow \alpha $.
		
		Khi đó biên độ dao động là $\cos\alpha=\dfrac{\Delta l_0}{A}\Rightarrow A$.
	}
	%5
	\item \mkstar{3}
	\cauhoi{
		
		Một con lắc lò xo có chiều dài tự nhiên là $l_0 =\SI{30}{cm}$, độ cứng của lò xo là $k =10\ \text{N/m}$. Treo vật nặng có khối lượng 0,1 kg vào lò xo và kích thích cho lò xo dao động điều hòa theo phương thẳng đứng với biên độ $A = \SI{20}{cm}$. Xác định tỉ số thời gian lò xo bị nén và giãn.
		\begin{mcq}(4)
			\item 0,5.
			\item 1.
			\item 2.
			\item 0,25.
		\end{mcq}
	}
	\loigiai{
		\textbf{Đáp án A.}
		
		
		Gọi $H$ là tỉ số thời gian lò xo bị nén và giãn trong một chu kỳ.
		
		Ta có $$H = \dfrac{t_\text{nén} }{t_\text{giãn}} = \dfrac{\varphi_\text{nén}}{\omega_\text{nén}} \cdot \dfrac{\omega_\text{giãn}}{\varphi_\text{giãn}} = \dfrac{\varphi_\text{nén}}{\varphi_\text{giãn}}$$
		
		Trong đó 
		
		+ $\varphi_\text{nén} =2\varphi'$; $\cos \varphi' = \dfrac{1}{2} \Rightarrow \varphi' =\dfrac{\pi}{3}$. Suy ra $\varphi_\text{nén} =\dfrac{2\pi}{3}.$
		
		+ $\varphi_\text{giãn} =2\pi - \dfrac{2\pi}{3} = \dfrac{4\pi}{3}$.
		
		Suy ra
		
		$$H = \dfrac{\varphi_\text{nén}}{\varphi_\text{dãn}}=\dfrac{1}{2}.$$
		
		
		
	}
	%6
	\item \mkstar{3}
	\cauhoi{
		
		Một con lắc lò xo có chiều dài tự nhiên là $l_0 =\SI{30}{cm}$, độ cứng của lò xo là $k =10\ \text{N/m}$. Treo vật nặng có khối lượng 0,1 kg vào lò xo và kích thích cho lò xo dao động điều hòa theo phương thẳng đứng với biên độ $A = \SI{20}{cm}$. Xác định thời gian lò xo bị nén trong 1 chu kỳ.
		\begin{mcq}(4)
			\item $\dfrac{\pi}{15}\ \text{s}$.
			\item $\dfrac{\pi}{10}\ \text{s}$.
			\item $\dfrac{\pi}{5}\ \text{s}$.
			\item $\pi\ \text{s}$.
		\end{mcq}
	}
	\loigiai{
		\textbf{Đáp án A.}
		
		Ta có $$t_\text{nén} =\dfrac{\varphi}{\omega}$$
		
		Trong đó:
		
		+ $\cos \varphi ' = \dfrac{\Delta l}{A} = \dfrac{1}{2} \Rightarrow \varphi' =\dfrac{\pi}{3} \Rightarrow \varphi = 2\varphi' =\dfrac{2\pi}{3} $.
		
		+ $\omega =\sqrt{\dfrac{k}{m}} =10\ \text{rad/s}$.
		
		Suy ra: $$t_\text{nén} =\dfrac{\varphi}{\omega} = \dfrac{\pi}{15}\ \text{s}$$
	}
	
	
\end{enumerate}

\loigiai{\textbf{Đáp án}
	\begin{center}
		\begin{tabular}{|m{2.8em}|m{2.8em}|m{2.8em}|m{2.8em}|m{2.8em}|m{2.8em}|m{2.8em}|m{2.8em}|m{2.8em}|m{2.8em}|}
			\hline
			1. A & 2. C & 3. D & 4. B & 5. C & 6. A & 7. C & 8. A & 9. A & 10. A \\
			\hline
		\end{tabular}
\end{center}}