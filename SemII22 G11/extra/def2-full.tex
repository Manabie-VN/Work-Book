%============ DECLARATION OF DEFAULT VALUE =====
	\newcommand{\outfooter}{
		Đề cương HK II - Vật lý 11 % Tên tài liệu ở footer
	}
	
	\graphicspath{{../figs/}{../extra/}} % các thư mục chứa hình ảnh 
	
% ========== PAPER FORMAT ====================
% --- paper size 
	\geometry{
		a4paper	% khổ giấy A4
		,total={170mm,247mm} % kích thước văn bản 170mmx247mm 
		,left=20mm % canh lề trái 
		,top=20mm % canh lề trên 
		,footskip=1.5cm % khoảng cách từ văn bản đến footer 
	}	
% --- line spacing -- choose 1 in 2 choices 
 	\onehalfspacing			% cách dòng đơn
	%\doublespacing			% cách dòng đôi  


% ========== DEFINE LEVELS  
% --- define \mychapter level, between \part and \chapter
	\titleclass{\mychapter}{top}[\part] 
	\newcounter{mychapter}[part]
	\renewcommand{\themychapter}{\arabic{mychapter}}
	%\titlespacing{\mychapter}{1pc}{*4}{*2.3}
	\titlespacing{\mychapter}{0pt}{1cm}{2.3em}
	
	
% ========== MAIN TABLE OF CONTENTS =========
	\contentsmargin{1.5em}
% Part text styling
%	\texorpdfstring{\setlength\fboxsep{0pt}\noindent\protect\colorbox{ocre!35}{\strut\protect\parbox[c][.8cm]{\linewidth}{\Large\sffamily\protect\centering #1\quad\mbox{}}}	
	\titlecontents{part}
		[3cm] % Left indentation
		{\addvspace{20pt}
			\begin{tikzpicture}[remember picture, overlay]%
			\draw[fill=black!60,draw=black!60] (-3.,-.1) ;%
			\pgftext[left,x=-2.5cm,y=0.23cm]{\color{black}\large\sc\bfseries Phần \thecontentslabel.};%
			\end{tikzpicture}\color{black!60}\large\bfseries
				} % Spacing and font options for parts
		{}
		{}
		{}
		[\addvspace{5pt}\color{black}]
% My chapter text styling 
	\titlecontents{mychapter}
		[1.25em]
		{}
		{\bfseries Bài \thecontentslabel.\enspace}
		{}
		{\bfseries\titlerule*[0.75pc]{.}\contentspage} %bad formatting, but only here to produce a content line in ToC	
% Chapter text styling
	\titlecontents{chapter}
		[1.25em] % Left indentation
		{} % Spacing and font options for chapters
		{} % Formatting of numbered sections of this type
		{} % Formatting of numberless sections of this type
		{\bfseries\titlerule*[0.75pc]{.}\contentspage} % Formatting of the filler to the right of the heading and the page number

\makeatletter
% that a section has Level 2, rather than Level 1.
\renewcommand{\l@section}{\@dottedtocline{2}{1.5em}{2.3em}}
\makeatother
\setcounter{tocdepth}{1}


% ========== TITLE FORMATTING 
% --- draw circle around chapter number 
	\newcommand*{\circhap}[1]{
		\tikz[baseline=(char.base)] % tạo ô bao quanh chữ
		{
			\node[
			shape=circle % hình tròn
			,fill=gray!50 % màu nền 
			,inner sep=5pt % khoảng cách chữ và hình 
			] 
			(char)
			{#1};
		}
	}	
% --- 
\AtBeginDocument{
% --- PART format (level -1)
	\titleclass{\part}{top}
	\renewcommand{\thepart}{\Roman{part}}
	\titleformat{\part}
		[display]
		{\bfseries\Huge}
		{\filleft \LARGE PHẦN  \Huge\thepart}
		{4ex}
		{\titlerule
		 %\pagecolor{green}	 
		 \vspace{2ex}%
		 \filcenter
	 	}
		[\vspace{2ex}%
		 \titlerule
		 %\pagecolor{white}
		]
		
% --- MYCHAPTER format (level 0) 
	\titleformat{\mychapter}
		[display]
		{\bfseries\huge}
		{\filleft \LARGE\itshape Bài \Huge\themychapter}
		{4ex}
		{%\titlerule
			\vspace{2ex}%
			\filcenter}
		[\vspace{2ex}%
		%\titlerule
		]

% --- CHAPTER format (level 1)
	\titleformat{\chapter}
		[display]
		{\normalfont\large\bfseries\centering}
		{}
		{-2cm}
		{\Large}	
		%\titleformat % định dạng tiêu đề 
	\renewcommand{\thesection} % định dạng chỉ số subsection
		{
			\arabic{section} % kiểu số Ả rập 
		}
	\titleformat{\section} % chỉnh tiêu đề section  
		[hang]
		{\large\bfseries} % format 
		{\thesection\!\!.} % không ghi chỉ mục 
		{0.5em} % khoảng cách đến tiêu đề 
		{} % trước khi bắt đầu 
\renewcommand{\thesubsection} % định dạng chỉ số subsection
{
	\thesection\!\!.\,\arabic{subsection} % kiểu số Ả rập 
}
\titleformat % định dạng tiêu đề 
{\subsection} %command 
{\normalsize\bfseries} % format 
{\thesubsection\!\!.} % đánh số 1,2,3,...
{0.5em} % khoảng cách đến tiêu đề 
{} % trước tiêu đề 
[] % sau tiêu đề 	
\renewcommand{\thesubsubsection} % định dạng chỉ số subsubsection 
{
	\thesubsection\!\!.\,\arabic{subsubsection} % kiểu số Ả rập 
}
\titleformat % định dạng tiêu đề 
{\subsubsection} % command 
{\normalsize\bfseries} % format 
{\thesubsubsection\!\!.} % 1.1, 1.2
{0.25em} % khoảng cách sau 1.1 
{ } % trước tiêu đề 
[\vspace*{-3mm}] % sau tiêu đề  	
}


% ----- định dạng Header và footer
% --- trang văn bản thông thường  
\pagestyle{fancy} 
\fancyhf{}
\renewcommand{\headrulewidth}
{0pt} % độ dày đường kẻ ở header 
\newcommand*\cirpage[1] % tạo hình tròn quanh số trang 
{\tikz[baseline=(char.base)]
	{
		\node[
		shape=circle
		,draw=black
		,fill=gray!0
		,inner sep=2pt
		]
		(char)
		{#1};
	}
}
\fancyfoot[LO,RE] % footer - lề trong 
{	
	\small \outfooter % tên tài liệu lấy từ phần khai báo đầu file 
}  
\fancyfoot[CO,CE] % footer - giữa trang 
{
	\small \cirpage{\thepage}
} 
\fancyfoot[LE] % footer - lề ngoài 
{
	\vspace*{-11pt}
	\hspace*{-1.1pt}\includegraphics[scale=0.03]{../extra/Logo.png} 
}
\fancyfoot[RO] % footer - lề ngoài 
{
	\vspace*{-11pt}
	\includegraphics[scale=0.03]{../extra/Logo.png}\hspace*{-5pt}
}
% --- trang Part và Chapter title 
\fancypagestyle{plain} % mặc định của trang Part và Chapter title 
{
	\fancyfoot[LO,RE]
	{
		\small \outfooter
	}
	\fancyfoot[CO,CE]
	{
		\small \cirpage{\thepage}
	} % footer - giữa trang chẵn và lẻ 
	\fancyfoot[LE]
	{
		\vspace*{-11pt}
		\hspace*{-1.1pt}\includegraphics[scale=0.03]{../extra/Logo.png}
	}
	\fancyfoot[RO]
	{
		\vspace*{-11pt}
		\includegraphics[scale=0.03]{../extra/Logo.png}\hspace*{-5pt}
	}
} % giống với trang thường 


% --- Định dạng cấu trúc 
\setcounter{secnumdepth}{4} % đánh số đến cấp thứ 4 của chỉ mục (subsubsection sẽ được đánh số)






% ======== MANUAL DEFINITIONS VERSION 3.1415 ===
% --- chừa chỗ trống tương ứng - văn bản gốc 
\newcommand{\bltext}[1]{#1}
\newcommand{\xtrule}{ }
\newcommand{\phantomeqn}[2][b]{
	#2
}


% --- tạo hộp Tóm tắt lý thuyết  
\newcommand{\hops}[1] %lệnh hộp (tóm tắt lý thuyết)
{	
	\begin{flushright}
		\leavevmode 
		\begin{tcolorbox}
			[
			standard jigsaw
			,opacityback=0
			,opacityframe=1
			,breakable
			,pad at break*=2mm
			%						,colback=white!20!white,
			,colframe=black!70!white
			,width=\textwidth
			,before upper={\parindent15pt}
			%						,watermark color=blue!3!white
			%						,watermark text=\arabic{tcbbreakpart}
			]
			{
				#1
			}
		\end{tcolorbox}
	\end{flushright}
}

% --- định nghĩa môi trường mới - Dạng 
\newcounter{dang} % định nghĩa chỉ số cho Dạng 
[section] % chỉ số sẽ reset mỗi section 
\newenvironment % định nghĩa môi trường mới 
{dang} % tên môi trường 
[1] % số thành phần phải có 
{
	\refstepcounter{dang} % chỉ số tương ứng 
	\leavevmode
	\begin{center}
		\leavevmode \vspace{-0.6cm}
		\begin{tcolorbox}
			[
			bicolor
			,sidebyside
			,width=0.93\textwidth
			,lefthand width=2.7cm
			,arc=0.5cm
			%	,rounded corners
			,colback=green!5
			,colbacklower=white
			,segmentation engine=path
			,segmentation style=
			{
				line width=1.5pt
				,solid
			}
			%						,borderline={0.3mm}{0.3mm}{black}
			]
			\large
			{
				\bf Mục tiêu \thedang
			}
			\tcblower
			\centering\large
			{
				\bf #1
			}
		\end{tcolorbox}
	\end{center}
	
} 

{
	
	\par
	\medskip	
}

% --- Định nghĩa lệnh tạo box Phương pháp giải 
\newcommand{\ppgiai}[1] %lệnh hộp (tóm tắt lý thuyết)
{
	\leavevmode 
	\begin{center}
		\leavevmode 
		\begin{tcolorbox}
			[
			%standard jigsaw
			,enhanced
			,opacityback=0
			,opacityfill=1
			,attach boxed title to top left={yshift=-3mm,yshifttext=-1mm}
			,boxed title style=
			{
				size=small
				,boxrule=1.5pt
				,colframe=black!70!white
				,colback=white!40!black
			}
			,title=\textbf{Phương pháp giải}
			%						,opacityback=0
			,opacityframe=1
			,breakable
			,pad at break*=2mm
			,colback=white!20!white,
			,colframe=black!70!white
			,width=0.85\textwidth
			,before upper={\parindent15pt}
			%						,watermark color=blue!3!white
			%						,watermark text=\arabic{tcbbreakpart}
			]
			{
				#1
			}
		\end{tcolorbox}
	\end{center}
}

% --- Định nghĩa lệnh tạo box Manatips
\newcommand{\manatip}[1] %lệnh hộp (tóm tắt lý thuyết)
{	\begin{center}
		\leavevmode 
		\begin{tcolorbox}
			[
			%standard jigsaw
			,enhanced
			,opacityback=0
			,opacityfill=1
			,attach boxed title to top left={yshift=-0.5mm,yshifttext=0mm}
			,boxed title style=
			{
				size=small
				,boxrule=1.5pt
				,colframe=black!70!white
				,colback=white!40!black
			}
			,title=\textbf{Manatip}
			%						,opacityback=0
			,opacityframe=1
			,breakable
			,pad at break*=2mm
			,colback=white!20!white,
			,colframe=black!70!white
			,width=0.85\textwidth
			,before upper={\parindent15pt}
			%						,watermark color=blue!3!white
			%						,watermark text=\arabic{tcbbreakpart}
			]
			{
				#1
			}
		\end{tcolorbox}
	\end{center}
}
% --- Định nghĩa lệnh tạo box Lưu ý khi dàn trang
\newcommand{\notebox}[1] %lệnh hộp (tóm tắt lý thuyết)
{	\begin{center}
		\leavevmode 
		\begin{tcolorbox}
			[
			%standard jigsaw
			,enhanced
			,opacityback=0
			,opacityfill=1
			,attach boxed title to top center={yshift=-0.5mm,yshifttext=0mm}
			,boxed title style=
			{
				size=small
				,boxrule=1.5pt
				,colframe=red!90!black
				,colback=red!90!black
			}
			,title=\textbf{Lưu ý khi dàn trang}
			%						,opacityback=0
			,opacityframe=1
			,breakable
			,pad at break*=2mm
			,colback=yellow!60!white,
			,colframe=red!90!black
			,width=0.85\textwidth
			,before upper={\parindent15pt}
			%						,watermark color=blue!3!white
			%						,watermark text=\arabic{tcbbreakpart}
			]
			{
				#1
			}
		\end{tcolorbox}
	\end{center}
}

% --- Định nghĩa lệnh tạo box Lưu ý 
\newcommand{\luuy}[1] %lệnh hộp (tóm tắt lý thuyết)
{	
	\vspace*{-0.7cm}
	\begin{center}
		\leavevmode 
		\begin{tcolorbox}
			[
			%standard jigsaw
			,enhanced
			,opacityback=0
			,opacityfill=1
			,attach boxed title to top right={yshift=-0.5mm,yshifttext=0mm}
			,boxed title style=
			{
				size=small
				,boxrule=1.5pt
				,colframe=black!70!white
				,colback=white!40!black
			}
			,title=\textbf{Lưu ý}
			%						,opacityback=0
			,opacityframe=1
			,breakable
			,pad at break*=2mm
			,colback=white!20!white,
			,colframe=black!70!white
			,width=0.85\textwidth
			,before upper={\parindent15pt}
			%						,watermark color=blue!3!white
			%						,watermark text=\arabic{tcbbreakpart}
			]
			{
				#1
			}
		\end{tcolorbox}
	\end{center}
}



% --- Tạo môi trường các đáp án trắc nghiệm 
\makeatletter
\@ifpackagelater{tasks}{2019/10/04}
{
	\NewTasksEnvironment[style=enumerate,label=\Alph*.,label-format={\bfseries},label-width=2ex,label-offset=1ex,item-indent=1.8cm]{mcq}[\item](1)
	% Code which runs if the package date is 2019/10/04 or later
}
{
	\NewTasks[style=enumerate,counter-format={\bfseries tsk[A].},label-width=2ex,label-offset=1.5ex,item-indent=1.8cm]{mcq}[\item](1)
	% Code which runs if the package date is older than 2019/10/04
}
\makeatother
%	\NewEnviron{mcq}[1][]
%		{
% Misc. stuff to preceed the tasks env here
%			\def\tempbegin
%				{%\vspace{1cm}
%					\begin{twopartasks}
%				}%
%					\expandafter\tempbegin\BODY
%					\end{twopartasks}
% Misc. stuff to follow
%		}

% -- insert stars
\newcommand\score[2]{%
	\pgfmathsetmacro\pgfxa{#1 + 1}%
	\tikzstyle{scorestars}=[star, star points=5, star point ratio=2.25, draw, inner sep=1.75pt, anchor=outer point 3]%
	\begin{tikzpicture}[baseline]
	\foreach \i in {1, ..., #2} {
		\pgfmathparse{\i<=#1 ? "gray" : "white"}
		\edef\starcolor{\pgfmathresult}
		\draw (\i*2.5ex, 0ex) node[name=star\i, scorestars, fill=\starcolor]  {};
	}
	\end{tikzpicture}%
}
\newcommand{\mkstar}[1]{\protect\score{#1}{4}}

\newcommand{\whiteBGstarBegin}{}
\newcommand{\whiteBGstarEnd}{}

% --- Định nghĩa môi trường ví dụ 
\newcommand{\vidu}[3] % -- không đánh số, có lời giải 
{
	%\vspace{0.3cm}
	\noindent\textbf{Ví dụ}\quad\mkstar{#1}
	\needspace{4\baselineskip}
	\begin{flushright}
		\leavevmode\vspace{-15pt}
		\begin{tcolorbox}[
			standard jigsaw
			,opacityback=0
			,opacityframe=0
			,width=0.95\textwidth
			,breakable
			,right=-4pt,top=-4pt,left=-4pt
			,colframe=white
			,colback=white
			,before upper={\parindent15pt}
			]
			
			{#2}
			\needspace{4\baselineskip}
%			\begin{center}
%				\textbf{Giải:}
%			\end{center}
			
			{#3}	
		\end{tcolorbox}
	\end{flushright}	
}

\newcommand{\viduon}[2] % không đánh số, không lời giải
{
	%\vspace{0.3cm}
	\noindent\textbf{Ví dụ \quad\mkstar{#1}}
	\needspace{4\baselineskip}
	\begin{flushright}
		\leavevmode\vspace{-15pt}
		\begin{tcolorbox}[
			standard jigsaw
			,opacityback=0
			,opacityframe=0
			,width=0.95\textwidth
			,breakable
			,right=-4pt,top=-4pt,left=-4pt
			,colframe=white
			,colback=white
			,before upper={\parindent15pt}
			]
			
			
			{#2}
			
		\end{tcolorbox}
	\end{flushright}	
}

\newcounter{viduii}[dang] % chỉ số của ví dụ, reset khi bắt đầu dạng mới 

\newcommand{\viduii}[3] % có đánh số, có lời giải 
{
	%\vspace{0.3cm}
	\refstepcounter{viduii}
	\needspace{4\baselineskip}
	\noindent\textbf{Câu \theviduii~ \quad\mkstar{#1}}
	\begin{flushright}
		\leavevmode\vspace{-15pt}
		\begin{tcolorbox}[
			standard jigsaw
			,opacityback=0
			,opacityframe=0
			,width=0.95\textwidth
			,breakable
			,right=-4pt,top=-4pt,left=-4pt
			,colframe=white
			,colback=white
			,before upper={\parindent15pt}
			]
			
			
			{#2}
			\needspace{4\baselineskip}
%			\begin{center}
%				\textbf{Giải:}
%			\end{center}
			
			{#3}	
		\end{tcolorbox}
	\end{flushright}	
}

\newcounter{viduiii}[dang] % chỉ số của ví dụ, reset khi bắt đầu dạng mới 

\newcommand{\viduiii}[3] % có đánh số, có lời giải 
{
	%\vspace{0.3cm}
	\refstepcounter{viduiii}
	\needspace{4\baselineskip}
	\noindent\textbf{Câu \theviduiii~ \quad\mkstar{#1}}
	\begin{flushright}
		\leavevmode\vspace{-15pt}
		\begin{tcolorbox}[
			standard jigsaw
			,opacityback=0
			,opacityframe=0
			,width=0.95\textwidth
			,breakable
			,right=-4pt,top=-4pt,left=-4pt
			,colframe=white
			,colback=white
			,before upper={\parindent15pt}
			]
			
			
			{#2}
			\needspace{4\baselineskip}
			%			\begin{center}
			%				\textbf{Giải:}
			%			\end{center}
			
			{#3}	
		\end{tcolorbox}
	\end{flushright}	
}

\newcommand{\viduin}[2] % có đánh số, không lời giải 
{
	%\vspace{0.3cm}
	\refstepcounter{viduii}
	\needspace{4\baselineskip}
	\noindent\textbf{Ví dụ \theviduii~ \quad\mkstar{#1}}
	\begin{flushright}
		\leavevmode\vspace{-10pt}
		\begin{tcolorbox}[
			standard jigsaw
			,opacityback=0
			,opacityframe=0
			,width=0.95\textwidth
			,breakable
			,right=-4pt,top=-4pt,left=-4pt
			,colframe=white
			,colback=white
			,before upper={\parindent15pt}
			]
			
			
			{#2}	
			
		\end{tcolorbox}
	\end{flushright}	
}

% --- các ký tự tạo thêm 
% --- ký hiệu song song 
\newcommand{\parallelsum} % tên lệnh tạo ký hiệu song song 
{
	{\mathbin{\!/\mkern-5mu/\!}}
}
\newcommand{\dpara}{\parallelsum}
% --- đồng nhất kí hiệu độ (đơn vị góc) thành ^\circ
\renewcommand{\ang}[1]{#1^\circ}
% --- ký hiệu suất điện động và công suất như sgk
% ký hiệu từ font calligra 
%	\DeclareFontFamily{U}{calligra}{}
%	\DeclareFontShape{U}{calligra}{m}{n}{<->callig15}{}
%	\newcommand{\calE} % lệnh tạo ký hiệu sđđ 
%		{
%			{\!\!\text{\usefont{U}{calligra}{m}{n}\textbf{E}}\,\,}
%		}
%	\newcommand{\calP} % lệnh tạo ký hiệu công suất 
%		{
%			{\!\!\text{\usefont{U}{calligra}{m}{n}P}\,\,}
%		}
% ký hiệu từ font Boondox 
\DeclareFontFamily{U}{BOONDOX-cal}{\skewchar\font=45 }
\DeclareFontShape{U}{BOONDOX-cal}{m}{n}{
	<-> s*[1.05] BOONDOX-r-cal}{}
\DeclareFontShape{U}{BOONDOX-cal}{b}{n}{
	<-> s*[1.05] BOONDOX-b-cal}{}
\DeclareMathAlphabet{\bdx}{U}{BOONDOX-cal}{m}{n}
\SetMathAlphabet{\bdx}{bold}{U}{BOONDOX-cal}{b}{n}
\DeclareMathAlphabet{\bbdx}{U}{BOONDOX-cal}{b}{n}
\newcommand{\calE}{\bdx{E}}
\newcommand{\calP}{\bdx{P}}

% lệnh siunit 
\newcommand{\xsi}[2]{\SI[parse-numbers=false]{#1}{#2}}

% --- định nghĩa môi trường định luật 
\newtheorem{thrPh}{Định luật}

%--- hỗ trợ bảng 
\renewcommand{\theadfont}
{
	\normalfont\bfseries
} % làm ô trong bảng canh giữa + in đậm 
\newcommand{\nfhead}[1] % tên lệnh 
{
	\renewcommand{\theadfont}
	{
		\normalfont
	}
	\thead{#1}
	\renewcommand{\theadfont}
	{
		\normalfont\bfseries
	} 
} % làm ô trong bảng canh giữa 
% lưu ý, khi sử dụng \thead và \nfhead thì phải xuống dòng thủ công.

% --- tạo dòng trống 
\newcommand{\Pointilles}[1]{%
	\par\nobreak
	\noindent\rule{0pt}{\baselineskip}% Provides a larger gap between the preceding paragraph and the dots
	%\doublespacing
	\multido{}{#1}{\noindent\makebox[\linewidth]{\dotfill}\endgraf}% ... dotted lines ...
	%\onehalfspacing
	\bigskip% Gap between dots and next paragraph
}

\newcommand{\Linesfill}[1]{%
	\par\nobreak
	\noindent\rule{0pt}{\baselineskip}% Provides a larger gap between the preceding paragraph and the dots
	%\doublespacing
	\multido{}{#1}{\noindent\rule{\linewidth}{0.2pt}\endgraf}% ... dotted lines ...
	%\onehalfspacing
	\bigskip% Gap between dots and next paragraph
}	

\newcommand{\Blfill}[1]{%
	\par\nobreak
	\noindent\rule{0pt}{\baselineskip}% Provides a larger gap between the preceding paragraph and the dots
	%\doublespacing
	\multido{}{#1}{\noindent\rule{\linewidth}{0pt}\endgraf}% ... dotted lines ...
	%\onehalfspacing
	\bigskip% Gap between dots and next paragraph
}	

\newcommand{\phantomline}[2][b]{
	\ifx b#1 \Blfill{#2} \else
	\ifx d#1	\Pointilles{#2} \else
	\ifx l#1 \Linesfill{#2}
	\fi\fi\fi 
}

% --- các lệnh che các đoạn văn bản 
\newlength{\saveparindent}
\AtBeginDocument{\setlength{\saveparindent}{\parindent}}

\newsavebox{\mytext}

\newcommand{\dotshide}[1]{%
	\savebox{\mytext}{%
		\parbox[t]{\columnwidth}{
			\setlength{\parindent}{\saveparindent}
			#1\par\xdef\savedprevdepth{\the\prevdepth}
		}%
	}%
	\noindent
	\pgfmathparse{int(round(\the\dp\mytext/\the\baselineskip))}
	\Pointilles{\pgfmathresult}
	\par
	% restore \prevdepth to compute correctly the interline glue
	\prevdepth\savedprevdepth
}

\newcommand{\lineshide}[1]{%
	\savebox{\mytext}{%
		\parbox[t]{\columnwidth}{
			\setlength{\parindent}{\saveparindent}
			#1\par\xdef\savedprevdepth{\the\prevdepth}
		}%
	}%
	\noindent
	%	\vrule height \ht\mytext % the height of \mytext
	%	depth \dp\mytext  % the depth of \mytext
	%	width \columnwidth
	%	\newline	
	\pgfmathparse{int(round(\the\dp\mytext/\the\baselineskip))}
	\Linesfill{\pgfmathresult}
	\par
	% restore \prevdepth to compute correctly the interline glue
	\prevdepth\savedprevdepth
}

\newcommand{\blankhide}[1]{%
	\savebox{\mytext}{%
		\parbox[t]{\columnwidth}{
			\setlength{\parindent}{\saveparindent}
			#1\par\xdef\savedprevdepth{\the\prevdepth}
		}%
	}%
	\noindent
	%	\vrule height \ht\mytext % the height of \mytext
	%	depth \dp\mytext  % the depth of \mytext
	%	width \columnwidth
	%	\newline	
	\pgfmathparse{int(round(\the\dp\mytext/\the\baselineskip))}
	\Blfill{\pgfmathresult}
	\par
	% restore \prevdepth to compute correctly the interline glue
	\prevdepth\savedprevdepth
}

\newcommand{\hide}[2][b]{
	\ifx b#1 #2
	\fi
}

% --- new commands workshop 
\DeclareSIUnit\minute{\textrm{phút}}
% 
\newcommand{\bai}[1]{\part{#1}}
\newcommand{\LO}[1]{\chapter{#1}}

% --- equation number
\renewcommand{\theequation}{\arabic{equation}}


% --- testing 
\newcommand{\hides}[1]{#1}
% --- các lệnh bật / tắt dáp án
\newcommand{\AnswersOff}{
	\def\anskey{0}
}
\newcommand{\AnswersOn}{
	\def\anskey{1}
}	
\newcommand{\loigiai}[1]{#1}
\renewcommand{\loigiai}[1]{
	\ifthenelse{\equal{\anskey}{1}}{}{#1}	
}
	\def\anskey{0}
\newcommand{\cauhoi}[1]{#1}
\renewcommand{\cauhoi}[1]{
	\ifthenelse{\equal{\anskey}{1}}{}{#1}	
}