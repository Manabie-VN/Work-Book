\whiteBGstarBegin
\setcounter{section}{0}

\begin{enumerate}[label=\bfseries Câu \arabic*:]
	
		\item \mkstar{1} [21]
		
		\cauhoi{
			Độ lớn của suất điện động cảm ứng trong mạch kín tỉ lệ với 
			
			\begin{mcq}
				\item tốc độ biến thiên từ thông qua mạch ấy.	 
				\item độ lớn từ thông qua mạch.
				\item điện trở của mạch.	
				\item diện tích của mạch.
			\end{mcq}
		}
		
		\loigiai{
			\textbf{Đáp án: A.}
			
			Độ lớn của suất điện động cảm ứng trong mạch kín tỉ lệ với tốc độ biến thiên từ thông qua mạch ấy.
		}
	
			\item \mkstar{1} [21]
		
		\cauhoi{
			Độ lớn của suất điện động cảm ứng trong một mạch kín được xác định theo công thức
			
			\begin{mcq}(2)
				\item  $e_\text c = \left|\dfrac{\Delta \Phi}{\Delta t}\right|$.
				\item  $e_\text c = |\Delta \Phi \cdot \Delta t|$.
				\item  $e_\text c = \left|\dfrac{\Delta t}{\Delta \Phi}\right|$.
				\item  $e_\text c = - \left|\dfrac{\Delta \Phi}{\Delta t}\right|$.
			\end{mcq}
		}
		
		\loigiai{
			\textbf{Đáp án: A.}
			
			Độ lớn của suất điện động cảm ứng trong một mạch kín được xác định theo công thức
			$$e_\text c = \left|\dfrac{\Delta \Phi}{\Delta t}\right|.$$
			
		}
	
		\item \mkstar{1} [7]
	
	\cauhoi{
		
		Phát biểu định luật Faraday. Viết công thức của định luật.
		
	}
	
	\loigiai{
		
		Độ lớn của suất điện động cảm ứng xuất hiện trong mạch kín tỉ lệ với tốc độ biến thiên từ thông qua mạch kín đó.
		
		$$|e_\text c|= \left|-\dfrac{\Delta \Phi}{\Delta t}\right|.$$
		
	}

	\item \mkstar{1} [35]
	
	\cauhoi{
		
		Suất điện động cảm ứng là gì? Phát biểu, viết công thức, ghi tên, đơn vị các đại lượng trong công thức của định luật Fa-ra-đây về hiện tượng cảm ứng điện từ.
		
		Vận dụng: Một khung dây phẳng diện tích $\SI{20}{cm}^2$, gồm $10$ vòng dây đặt trong từ trường đều, góc giữa $\vec B$ và mặt phẳng khung dây là $30^\circ$, $B = 4\cdot 10^{-4}\ \text T$, làm cho từ trường giảm đều về $0$ trong thời gian $\SI{0,01}{s}$. Hãy xác định độ lớn của suất điện động cảm ứng sinh ra trong khung dây.
		
		
		
	}
	
	\loigiai{
		
		Định nghĩa:
	
	Suất điện động cảm ứng là suất điện động sinh ra dòng điện cảm ứng trong mạch kín. 
	
	Định luật Fa-ra-đây về hiện tượng cảm ứng điện từ 
	
	- Phát biểu: “Độ lớn của suất điện động cảm ứng xuất hiện trong mạch kín tỉ lệ với tốc độ biến thiên từ thông qua mạch kín đó” 
	- Công thức suất điện động cảm ứng:
	
	$$e_\text c = -\dfrac{\Delta \Phi}{\Delta t}$$
	
	- Độ lớn:
	
	$$|e_\text c| = \left|-\dfrac{\Delta \Phi}{\Delta t}\right|$$
	
	trong đó:
	
	+ $\left|\dfrac{\Delta \Phi}{\Delta t}\right|$: tốc độ biến thiên từ thông (Wb/s).
	
	+ $e_\text c$: suất điện động cảm ứng (V)  
	
	Vận dụng: Suất điện động cảm ứng trong khung:($\alpha = 90^\circ - 30^\circ = 60^\circ$) 
	
	$$e_\text{c} = -\dfrac{\Delta \Phi}{\Delta t} =- \dfrac{N\Delta B S \cos \alpha}{\Delta t}= 4 \cdot 10^{-4}\ \text{V}.$$
	
		
	}

	\item \mkstar{2} [34]
	
	\cauhoi{
		
		Một khung dây phẳng, tròn, bán kính $\SI{0,1}{m}$, có $100$ vòng dây, đặt trong từ trường đều. Mặt phẳng khung dây vuông góc với các đường cảm ứng từ. Lúc đầu cảm ứng từ có giá trị $\SI{0,2}{T}$. Xác định độ lớn suất điện động cảm ứng trong cuộn dây nếu trong $\SI{0,2}{s}$ cảm ứng từ tăng đều lên gấp ba.
		
		
	}
	\loigiai{
		
	Độ biến thiên từ thông 
	
	$$\Delta \Phi = N \Delta BS \cos 0^\circ = \pi \dfrac{2}{5}\ \text{Wb}.$$
	
	Suất điện động cảm ứng trong cuộn dây
	
	$$|e_\text c| = \left|\dfrac{\Delta \Phi}{\Delta t}\right| = \pi \ \text V.$$ 

	
	}

	
	\item \mkstar{2} [10]
	
	\cauhoi{
		
		Một khung dây phẳng diện tích $\SI{20}{cm}^2$ gồm $10$ vòng dây, được đặt trong từ trường đều có vectơ cảm ứng từ hợp với mặt phẳng khung dây một góc $30^\circ$ và có độ lớn $2\cdot 10^{-4}\ \text T$. Người ta cho từ trường qua khung dây giảm đều đến không trong $\SI{0,01}{s}$. Tính độ lớn suất điện động cảm ứng xuất hiện trong thời gian từ trường biến đổi.
		
		
	}
	\loigiai{
		
		Độ lớn suất điện động cảm ứng xuất hiện trong thời gian từ trường biến đổi
		
		$$|e_\text c|= \left|-\dfrac{N\Delta B S \cos \alpha}{\Delta t}\right|= 2 \cdot 10^{-4}\ \text{V}.$$
	}

	\item \mkstar{3} [14]
	
	\cauhoi{
		Một khung dây gồm $500$ vòng dây, diện tích mỗi vòng $\SI{10}{cm}^2$, khung đặt trong từ trường đều. Cảm ứng từ  hợp với mặt phẳng khung dây góc $30^\circ$  và có độ lớn $\SI{0,06}{T}$. 
		
		\begin{enumerate}[label=\alph*)]
			\item Tính từ thông qua khung dây.
			\item Cho cảm ứng từ của từ trường tăng đều đến $\SI{0,08}{T}$ trong khoảng thời gian $\SI{0,01}{s}$. Tính độ lớn suất điện động cảm ứng xuất hiện trong khung dây.
		\end{enumerate}
			
	}
	\loigiai{
		
	\begin{enumerate}[label=\alph*)]
		\item Từ thông qua khung dây
		
		$$\Phi = NBS \cos \alpha = \text{1,5} \cdot 10^{-2}\ \text{Wb}.$$
		
		\item Độ biến thiên từ thông
		
		$$\Delta \Phi =N \Delta B S \cos \alpha = 5 \cdot 10^{-3}\ \text{Wb}.$$
		
		Độ lớn Suất điện động cảm ứng
		
		$$|e_\text c |= \left|-\dfrac{\Delta \Phi}{\Delta t}\right| = \SI{0,5}{V}.$$ 
		
	\end{enumerate}
	}
	
	\item \mkstar{3} [6]
	
	\cauhoi{
		
		Một khung dây hình tròn có đường kính $\SI{10}{cm}$ gồm 1000 vòng dây đặt trong từ trường đều có cảm ứng từ $\SI{0,5}{T}$, sao cho mặt phẳng khung vuông góc với các đường sức từ. Sau khoảng thời gian $\SI{0,1}{s}$ thì cảm ứng từ tăng đều đến giá trị $\SI{0,75}{T}$. Lấy $\pi = \text{3,14}$.
		\begin{enumerate}[label=\alph*)]
			\item Tính độ lớn suất điện động cảm ứng xuất hiện trong khung dây.
			\item Tính cường độ dòng điện cảm ứng xuất hiện trong khung dây ở khoảng thời gian trên. Biết điện trở của dây dẫn trên một mét chiều dài là $\SI{0,1}{\Omega}$.
		\end{enumerate}
	 
		
	}
	
	\loigiai{	
			
	\begin{enumerate}[label=\alph*)]
		\item Độ lớn suất điện động cảm ứng xuất hiện trong khung dây.
		
		$$e_\text c = \left|\dfrac{\Delta \Phi}{\Delta t}\right| = \dfrac{|N\Delta B S \cos 0^\circ|}{\Delta t} = \SI{19,625}{V}.$$
		
		\item Điện trở của dây
		
		$$R' = l \cdot R = N\pi d \cdot R = \SI{31,2}{\Omega}.$$
		
		Cường độ dòng điện cảm ứng xuất hiện trong khung dây
		
		$$i =\dfrac{e_\text c}{R} = \SI{0,625}{A}.$$
		
		
	\end{enumerate}
	}
	
		\item \mkstar{3} [11]
	
	\cauhoi{
		
		Một khung dây (C) phẳng có diện tích $\SI{80}{cm}^2$, gồm $250$ vòng dây đặt trong từ trường đều. Vectơ cảm ứng từ hợp với mặt phẳng khung dây góc $30^\circ$ và có độ lớn $\text{4,5}\cdot 10^{-3}\ \text T$. Vòng dây có điện trở $\SI{0,5}{\Omega}$ . Người ta làm cho từ trường giảm đều về $2\cdot 10^{-3}\ \text T$ trong khoảng thời gian $\SI{0,05}{s}$.
	
		\begin{enumerate}[label=\alph*)]
			\item Tìm độ lớn suất điện động cảm ứng xuất hiện trong (C).
			\item Tìm cường độ của dòng điện cảm ứng trong (C). 
		\end{enumerate}
	}
	\loigiai{
	
		\begin{enumerate}[label=\alph*)]
			\item Độ lớn suất điện động cảm ứng xuất hiện trong (C)
			
			$$\alpha =90^\circ-30^\circ = 60^\circ.$$ 
			
			$$|\Delta \Phi|=N|B_2-B_1|S \cos \alpha.$$
			
			$$\Rightarrow \Delta \Phi = \text{2,5}\cdot 10^{-3}\ \text{Wb}.$$
			
			\item Cường độ của dòng điện cảm ứng trong (C)
			
			$$|e_\text c|=\left|\dfrac{\Delta \Phi}{t}\right| = \SI{0,05}{V}.$$
			
			$$I=\dfrac{e_\text c}{R} = \SI{0,1}{A}. $$
		\end{enumerate}
		
		
	}

	
	\item \mkstar{3} [22]
	
	\cauhoi{
		Một khung dây dẫn gồm $500$ vòng hình vuông có cạnh $\SI{10}{cm}$. Khung được đặt cố định trong một từ trường đều, vectơ cảm ứng từ $\vec B$ hợp với pháp tuyến của  khung dây một góc $60^\circ$ và có độ lớn $\SI{2,4}{T}$.
		
		\begin{enumerate}[label=\alph*)]
			\item Tính độ lớn từ thông xuất hiện trong khung.
			\item Người ta làm cho từ trường giảm đều đến $\SI{1,5}{T}$ trong khoảng thời gian $\SI{0,1}{s}$. Tính độ lớn suất điện động cảm ứng xuất hiện trong khung.
		\end{enumerate}
	}
	
	\loigiai{
		
	\begin{enumerate}[label=\alph*)]
		\item Độ lớn từ thông xuất hiện trong khung
		
		$$\Phi = NBS \cos \alpha = \SI{1,2}{Wb}.$$
		
		\item Suất điện động cảm ứng xuất hiện trong khung
		
		$$|e_\text c| = \dfrac{|\Delta \Phi|}{\Delta t} = \dfrac{|N \Delta B S \cos \alpha|}{\Delta t} = \SI{4,5}{V}.$$
		
	\end{enumerate}
		
	}

	
	\item \mkstar{3} [24]
	
	\cauhoi{
		
		Một khung dây hình tròn kín gồm $500$ vòng dây, đường kính vòng dây $\SI{0,6}{cm}$, đặt trong một từ trường đều có vectơ cảm ứng từ $\vec B$ cùng hướng với véc tơ pháp tuyến và hợp với mặt phẳng  khung dây một góc $30^\circ$, cảm ứng từ lúc đầu có độ lớn là $\SI{0,4}{T}$. Tính suất điện động cảm ứng trong thời gian $\Delta t = \SI{0,05}{s}$, cảm ứng từ tăng lên gấp đôi .
		
	}
	
	\loigiai{
		
		Diện tích khung dây
	
	$$S = \pi r^2.$$
	
	Suất điện động cảm ứng
	
	$$e_\text c = - \dfrac{NS \cos \alpha (B_2-B_1)}{\Delta t}= \SI{0,056}{V}.$$
	
	
	}

		\item \mkstar{3} [26]
	
	\cauhoi{
		Một khung dây hình vuông có cạnh $\SI{10}{cm}$ có điện trở $R = \SI{0,5}{\Omega}$, được đặt nghiêng góc $30^\circ$ so với đường sức của một từ trường đều với $B = \SI{0,02}{T}$. 
	
		\begin{enumerate}[label=\alph*)]
			\item Tính từ thông gửi qua vòng dây đó.
			\item Từ trường tăng đều từ $B$ đến $2B$ trong khoảng thời gian $\SI{0,001}{s}$. Tính suất điện động cảm ứng xuất hiện trong vòng dây. 
			\item Tính độ lớn của dòng điện cảm ứng xuất hiện trong vòng dây.
		\end{enumerate}
		
	}
	\loigiai{
		
		\begin{enumerate}[label=\alph*)]
			\item Từ thông gửi qua vòng dây đó.
			
			$$\Phi = NBS \cos \alpha = 10^{-4}\ \text{Wb}.$$
			
			\item Suất điện động cảm ứng
			
			$$e_\text c = \dfrac{N\Delta BS \cos \alpha}{\Delta t} = \SI{0,1}{V}.$$
			
			\item Độ lớn của dòng điện cảm ứng xuất hiện trong vòng dây
			
			$$I = \dfrac{e_\text c}{R} = \SI{0,2}{A}.$$
		\end{enumerate}
			}
	
\end{enumerate}	
\whiteBGstarEnd