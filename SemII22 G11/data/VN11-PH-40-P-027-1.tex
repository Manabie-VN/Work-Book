%\setcounter{chapter}{1}
\chapter{Luyện tập: Mắt}
\section{Tiêu cự và độ tụ của mắt}
\begin{enumerate}
	\item %câu 1
	Một người có thể nhìn rõ các vật cách mắt từ $\SI{10}{\centi\meter}$ đến $\SI{100}{\centi\meter}$. Độ biến thiên độ tụ của mắt người đó từ trạng thái không điều tiết đến trạng thái điều tiết tối đa là
	\begin{mcq}(4)
		\item $\SI{12}{dp}$.
		\item $\SI{5}{dp}$.
		\item $\SI{6}{dp}$.
		\item $\SI{9}{dp}$.
	\end{mcq}
	\item %câu 2
	Một người có thể nhìn rõ các vật cách mắt $\SI{12}{\centi\meter}$ thì mắt không phải điều tiết. Lúc đó, độ tụ của thuỷ tinh thể là $\SI{62,5}{dp}$. Khoảng cách từ quang tâm thuỷ tinh thể đến võng mạc \textbf{gần giá trị nào nhất} sau đây?
	\begin{mcq}(4)
		\item $\SI{1,8}{\centi\meter}$.
		\item $\SI{1,5}{\centi\meter}$.
		\item $\SI{1,6}{\centi\meter}$.
		\item $\SI{1,9}{\centi\meter}$.
	\end{mcq}
	\item %câu 3
	Một người mắt không có tật, quang tâm nằm cách võng mạc một khoảng $\SI{2,2}{\centi\meter}$. Độ tụ của mắt khi quan sát không điều tiết \textbf{gần giá trị nào nhất} sau đây?
	\begin{mcq}(4)
		\item $\SI{42}{dp}$.
		\item $\SI{45}{dp}$.
		\item $\SI{46}{dp}$.
		\item $\SI{49}{dp}$.
	\end{mcq}
	\item %câu 4
	Một người mắt không có tật, quang tâm nằm cách võng mạc một khoảng $\SI{2,2}{\centi\meter}$. Độ tụ của mắt đó khi quan sát một vật cách mắt $\SI{20}{\centi\meter}$ \textbf{gần giá trị nào nhất} sau đây?
	\begin{mcq}(4)
		\item $\SI{42}{dp}$.
		\item $\SI{45}{dp}$.
		\item $\SI{46}{dp}$.
		\item $\SI{49}{dp}$.
	\end{mcq}
	\item %câu 5
	Mắt của một người có quang tâm cách võng mạc khoảng $\SI{1,52}{\centi\meter}$. Tiêu cự thể thủy tinh thay đổi giữa hai giá trị $f_1=\SI{1,500}{\centi\meter}$ và $f_2=\SI{1,415}{\centi\meter}$. Khoảng nhìn rõ của mắt \textbf{gần giá trị nào nhất} sau đây?
	\begin{mcq}(4)
		\item $\SI{95,8}{\centi\meter}$.
		\item $\SI{93,5}{\centi\meter}$.
		\item $\SI{97,4}{\centi\meter}$.
		\item $\SI{97,8}{\centi\meter}$.
	\end{mcq}
\end{enumerate}

\textbf{ĐÁP ÁN}
\begin{longtable}[\textwidth]{|p{0.1\textwidth}|p{0.1\textwidth}|p{0.1\textwidth}|p{0.1\textwidth}|p{0.1\textwidth}|}
	% --- first head
	\hline%\hspace{2 pt}
	\multicolumn{1}{|c}{\textbf{Câu 1}} &
	\multicolumn{1}{|c|}{\textbf{Câu 2}} &
	\multicolumn{1}{c|}{\textbf{Câu 3}} &
	\multicolumn{1}{c|}{\textbf{Câu 4}} &
	\multicolumn{1}{c|}{\textbf{Câu 5}} \\
	\hline
	D. & A.  & B. & D. & B.\\
	\hline
\end{longtable} 

\section{Sửa tật ở mắt}
\begin{enumerate}
	\item %câu 1
	Mắt của một người có điểm cực viễn cách mắt $\SI{80}{\centi\meter}$. Muốn nhìn thấy vật ở vô cực không điều tiết, người đó phải đeo kính sát mắt có độ tụ
	\begin{mcq}(4)
		\item $\SI{-4}{dp}$.
		\item $\SI{-1,25}{dp}$.
		\item $\SI{-2}{dp}$.
		\item $\SI{-2,5}{dp}$.
	\end{mcq}
	\item %câu 2
	Một người khi không đeo kính có thể nhìn rõ các vật gần nhất cách mắt $\SI{50}{\centi\meter}$. Xác định độ tụ của kính mà người đó cần đeo sát mắt để có thể nhìn rõ các vật gần nhất cách mắt $\SI{25}{\centi\meter}$.
	\begin{mcq}(4)
		\item $\SI{4,2}{dp}$.
		\item $\SI{2}{dp}$.
		\item $\SI{3}{dp}$.
		\item $\SI{1,9}{dp}$.
	\end{mcq}
	\item %câu 3
	Một người có điểm cực viễn cách mắt $\SI{25}{\centi\meter}$ và điểm cực cận cách mắt $\SI{10}{\centi\meter}$. Khi đeo kính sát mắt có độ tụ $\SI{-2}{dp}$ thì có thể nhìn rõ các vật nằm trong khoảng nào trước kính?
	\begin{mcq}(2)
		\item $\SI{10}{\centi\meter}\div\SI{50}{\centi\meter}$.
		\item $\SI{12,5}{\centi\meter}\div\SI{50}{\centi\meter}$.
		\item $\SI{10}{\centi\meter}\div\SI{40}{\centi\meter}$.
		\item $\SI{12,5}{\centi\meter}\div\SI{40}{\centi\meter}$.
	\end{mcq}
	\item %câu 4
	Một người cận thị phải kính sát mắt có độ tụ $\SI{-2,5}{dp}$. Khi đeo kính đó, người ấy có thể nhìn rõ các vật gần nhất cách kính $\SI{24}{\centi\meter}$. Khoảng nhìn rõ của mắt khi không đeo kính \textbf{gần giá trị nào nhất} sau đây?
	\begin{mcq}(4)
		\item $\SI{26}{\centi\meter}$.
		\item $\SI{15}{\centi\meter}$.
		\item $\SI{50}{\centi\meter}$.
		\item $\SI{40}{\centi\meter}$.
	\end{mcq}
	\item %câu 5
	Một người khi đeo kính có độ tụ $\SI{+2,5}{dp}$. có thể nhìn rõ các vật cách mắt từ $\SI{27}{\centi\meter}$ tới vô cùng. Biết kính đeo cách mắt $\SI{2}{\centi\meter}$. Khoảng cực cận của mắt người đó là 
	\begin{mcq}(4)
		\item $\SI{15}{\centi\meter}$.
		\item $\SI{61}{\centi\meter}$.
		\item $\SI{52}{\centi\meter}$.
		\item $\SI{40}{\centi\meter}$.
	\end{mcq}
\end{enumerate}

\textbf{ĐÁP ÁN}
\begin{longtable}[\textwidth]{|p{0.1\textwidth}|p{0.1\textwidth}|p{0.1\textwidth}|p{0.1\textwidth}|p{0.1\textwidth}|}
	% --- first head
	\hline%\hspace{2 pt}
	\multicolumn{1}{|c}{\textbf{Câu 1}} &
	\multicolumn{1}{|c|}{\textbf{Câu 2}} &
	\multicolumn{1}{c|}{\textbf{Câu 3}} &
	\multicolumn{1}{c|}{\textbf{Câu 4}} &
	\multicolumn{1}{c|}{\textbf{Câu 5}} \\
	\hline
	B. & B.  & B. & A. & C.\\
	\hline
\end{longtable}











