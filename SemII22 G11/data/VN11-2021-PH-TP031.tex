\whiteBGstarBegin
\setcounter{section}{0}
\begin{enumerate}[label=\bfseries Câu \arabic*:]
	
	\item \mkstar{1} [6]
	
	\cauhoi{
		Trình bày tác dụng của lăng kính đối với sự truyền ánh sáng qua nó. Xét hai trường hợp: Ánh sáng đơn sắc và ánh sáng trắng.
		
	}
	
	\loigiai{		
		
		- Tia sáng đơn sắc chiếu đến mặt bên của lăng kính, khi có tia ló ra khỏi lăng kính thì tia ló bao giờ cũng lệch về phía đáy lăng kính so với tia tới. 
		
		- Chùm ánh sáng trắng khi đi qua lăng kính sẽ bị phân tích thành nhiều chùm sáng đơn sắc khác nhau. 
		
	}
	
	
	\item \mkstar{1} [16]
	
	\cauhoi{
	
	Lăng kính là gì? Nêu các phần tử và đặc trưng quang học của lăng kính.
	
	}
	
	\loigiai{
		
		- Lăng kính là một khối chất trong suốt, đồng chất, thường có dạng lăng trụ tam giác.  
		
		- Các phần tử của lăng kính gồm: cạnh, đáy và hai mặt bên.  
		
		- Đặc trưng quang học của lăng kính gồm: góc chiết quang $A$ và chiết suất $n$.
		
	}

\end{enumerate}	
\whiteBGstarEnd