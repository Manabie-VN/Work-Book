\chapter{Hiện tượng tự cảm}
\section{Lý thuyết trọng tâm}
\subsection{Từ thông riêng của một mạch kín}
Giả sử có một mạch kín $(C)$, trong đó có dòng điện cường độ $i$. Dòng điện $i$ tạo ra một từ trường, từ trường này gây ra một từ thông $\Phi$ qua $(C)$ được gọi là \textit{từ thông riêng} của mạch.
\begin{equation}
\Phi=Li,
\end{equation}
trong đó, $L$ là gọi là độ tự cảm của mạch, có đơn vị henry (H). Độ tự cảm $L$  chỉ phụ thuộc vào cấu tạo và kích thước của mạch kín $(C)$. 

Độ tự cảm của ống dây được tính theo công thức:
\begin{equation}
L=4\pi \cdot 10^{-7}\cdot \dfrac{N^2}{l}\cdot S,
\end{equation}
trong đó,
\begin{itemize}
	\item $N$ là số vòng của cuộn dây,
	\item $l$ là chiều dài cuộn dây, 
	\item $S$ là diện tích ống dây.
	\item $L$ là độ tự cảm.
\end{itemize}

Ngoài ra, độ tự cảm còn được tính theo công thức:
\begin{equation}
L=4\pi \cdot 10^{-7}\cdot n^2 \cdot V,
\end{equation}
trong đó,
\begin{itemize}
	\item $n$ là số vòng trên một đơn vị chiều dài ống dây,
	\item $V$ là thể tích ống dây, 
\end{itemize}

Ống dây có độ tự cảm $L$ được gọi là cuộn cảm.  

Nếu ống dây có lõi sắt thì độ tự cảm được tính theo công thức:

\begin{equation}
L=4\pi\mu \cdot 10^{-7}\cdot \dfrac{N^2}{l}\cdot S,
\end{equation}
trong đó, $\mu$ gọi là độ từ thẩm, đặc trưng cho từ tính của lõi sắt.
\subsection{Hiện tượng tự cảm}
Hiện tượng tự cảm là hiện tượng cảm ứng điện từ  xảy ra trong một mạch có dòng điện mà sự biến thiên từ thông qua mạch được gây ra bởi sự biến thiên của cường độ dòng điện trong mạch kín.

Trong các mạch điện một chiều, hiện tượng tự cảm thường xảy ra khi đóng, ngắt mạch.

Trong các mạch điện xoay chiều, luôn luôn xảy ra hiện tượng tự cảm, vì cường độ dòng điện xoay chiều biến thiên liên tục theo thời gian.
\section{Bài tập}
\begin{dang}{Từ thông  riêng qua mạch kín, độ tự cảm}
\end{dang}

\textbf{Phương pháp giải}

Sử dụng biểu thức tính từ thông riêng của một mạch kín:
\begin{equation}
\Phi=Li,
\end{equation}


\vspace{1em}
{\viduii{2}{	
Một ống dây có chiều dài là $\text{1,5}\ \text{m}$, gồm 2000 vòng dây, ống dây có đường kính là 40 cm. Độ tự cảm của ống dây là 
	\begin{mcq}(4)
		\item $\text{0,21}\ \text{H}$.
		\item $\text{0,42}\ \text{H}$.
		\item $\text{0,21}\ \text{mH}$.
		\item $\text{0,42}\ \text{mH}$.
	\end{mcq}}
{\begin{center}
		\textbf{Hướng dẫn giải:}
\end{center}
	
	Tiết diện của ống dây là $S=\dfrac{\pi d^2}{4}=\dfrac{\pi}{25}\ \text{m}^2$.
	
	Độ tự cảm của ống dây là $L=4\pi \cdot 10^{-7}\cdot \dfrac{N^2}{l}\cdot S=\text{0,42}\ \text{H}$.
	
\textbf{	Đáp án: B.}}
}

{\viduii{2}{
	
Tìm từ thông qua ống dây biết ống dây có chiều dài là $\text{2,5}\ \text{m}$, gồm 3000 vòng dây, bán kính ống dây là 10 cm. Biết cường độ dòng điện trong ống dây là 5 A. 
	\begin{mcq}(4)
		\item $\text{0,7}\ \text{Wb}$.
		\item $\text{0,8}\ \text{Wb}$.
		\item $\text{0,9}\ \text{Wb}$.
		\item $\text{0,6}\ \text{Wb}$.
		
	\end{mcq}}
{\begin{center}
		\textbf{Hướng dẫn giải:}
\end{center}
		
		Tiết diện của ống dây là $S=\pi R^2=\dfrac{\pi}{100}\ \text{m}^2$.
	
	   Độ tự cảm của ống dây là $L=4\pi \cdot 10^{-7}\cdot \dfrac{N^2}{l}\cdot S=\text{0,14}\ \text{H}$.
		
	 Từ thông qua ống dây là $\Phi=Li=\text{0,7}\ \text{Wb}$.
	 
\textbf{		Đáp án: A.}}
	}
	
	
		
		\begin{dang}{Hiện tượng tự cảm}
		\end{dang}
	\textbf{Phương pháp giải}
		Độ tự cảm $L$ được tính theo công thức:
		\begin{equation}
		L=4\pi \cdot 10^{-7}\cdot \dfrac{N^2}{l}\cdot S,
		\end{equation}
		hay 
		\begin{equation}
		L=4\pi \cdot 10^{-7}\cdot n^2 \cdot V,
		\end{equation}
		\vspace{1em}
		\viduii{1}{
			Kết luận nào sau đây là đúng?
			\begin{mcq}
				\item Hiện tượng tự cảm không phải là hiện tượng cảm ứng điện từ.
				\item Hiện tượng tự cảm không xảy ra ở các mạch điện xoay chiều.
				\item Hiện tượng tự cảm là hiện tượng cảm ứng điện từ xảy ra trong một mạch có dòng điện mà sự biến thiên từ thông qua mạch được gây ra bởi sự biến thiên của cường độ dòng điện trong mạch.
				\item Hiện tượng tự cảm là hiện tượng cảm ứng điện từ xảy ra trong một mạch có dòng điện mà sự biến thiên từ thông qua mạch được gây ra bởi sự biến thiên của từ trường bên ngoài mạch điện.
			\end{mcq}}
		{\begin{center}
				\textbf{Hướng dẫn giải:}
		\end{center}
			
			Câu A sai vì hiện tượng tự cảm là hiện tượng cảm ứng điện từ.
			
			Câu B sai vì hiện tượng tự cảm luôn luôn xảy ra ở các mạch điện xoay chiều.
			
			Câu D sai vì hiện tượng tự cảm là hiện tượng cảm ứng điện từ xảy ra trong một mạch có dòng điện mà sự biến thiên từ thông qua mạch được gây ra bởi sự biến thiên của cường độ dòng điện trong mạch.
			
	\textbf{Đáp án: C.}
		
}		
		{\viduii{1}{	
			Trong thí nghiệm về hiện tượng tự cảm và ngắt mạch, người ta đưa lõi sắt vào trong lòng ống dây để
				\begin{mcq}
				\item tăng điện trở của ống dây
				\item tăng cường độ dòng điện qua ống dây
				\item làm cho bóng đèn mắc trong mạch không bị cháy
				\item tăng độ tự cảm của ống dây
			\end{mcq}}
{		\begin{center}
				\textbf{Hướng dẫn giải:}
		\end{center}
			
			Trong thí nghiệm về hiện tượng tự cảm và ngắt mạch, người ta đưa lõi sắt vào trong lòng ống dây để tăng độ tự cảm của ống dây.
			
		\textbf{	Đáp án: D.}}
		}
		
	