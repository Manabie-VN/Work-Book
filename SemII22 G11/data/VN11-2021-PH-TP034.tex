\whiteBGstarBegin
\setcounter{section}{0}
\begin{enumerate}[label=\bfseries Câu \arabic*:]
	
	
	
	\item \mkstar{1} [9]
	
	\cauhoi{
		\begin{minipage}[l]{11cm}
			Dùng một thấu kính để quan sát một dòng chữ nhỏ như hình  bên. Khi đặt thấu kính cách dòng chữ một khoảng thích hợp, nhìn qua thấu kính ta thấy một ảnh cùng chiều, lớn hơn vật. Hãy cho biết tính chất của ảnh (ảo hay thật) và loại thấu kính đang sử dụng. Giải thích.
		\end{minipage}
	\begin{minipage}[r]{5cm}
		\includegraphics[scale=0.9]{../figs//VN11-2021-PH-TP028-24.JPG}
	\end{minipage}
	
	}
	
	\loigiai{
		
		- Ảnh ảo.	
								
		- Thấu kính hội tụ.
				 				
		- Vì cho ảnh ảo lớn hơn vật nên đó là thấu kính hội tụ.
	}
	
	\item \mkstar{1} [10]
	
	\cauhoi{
		
		Nêu công dụng và cấu tạo của kính lúp.
		
	}
	
	\loigiai{
		
		Kính lúp là dụng cụ quang bổ trợ cho mắt để quan sát các vật nhỏ, là thấu kính hội tụ có tiêu cự nhỏ (vài cm).
		
	}
	
	\item \mkstar{2} [36]
	
	\cauhoi{
		
		Ở thành phố Hồ Chí Minh, vào những ngày hè trời nắng nóng nhiệt độ bên ngoài có thể đạt từ $35^\circ \ \text{C}$ – $40^\circ \ \text{C}$, một học sinh trường trung học phổ thông Nhân Việt trong một buổi lao động đã dùng một kính lúp có độ tụ thích hợp để ngoài trời nắng và sau đó đốt cháy lá cây trong đống rác dưới sân trường. Em hãy cho biết thí nghiệm trên liên quan đến hiện tượng nào? Hãy giải thích hiện tượng đó.
		
	}
	
	\loigiai{
				
		- Thí nghiệm trên liên quan đến thấu kính hội tụ.
		
		- Kính lúp là thấu kính hội tụ. Mặt trời phát ra chùm tia sáng nóng và song song. Mà khi chiều tia song song vào thấu kính hội tụ thì cho ta tia ló hội tụ tại 1 điểm (tiêu điểm ảnh), sự hội tụ tại 1 điểm này làm cho nhiệt độ tăng cao, có thể làm cháy lá cây.
		
	}

			\item \mkstar{2} [25]
		
		\cauhoi{
			Tại sao chúng ta không nên vứt chai, lọ thủy tinh vào rừng đặc biệt là vào mùa nắng?
			
		}
		
		\loigiai{
			
			- Chai, lọ thủy tinh đóng vai trò giống như thấu kính hội tụ.
			
			- Tính chất: Hội tụ tia sáng mặt trời, làm nhiệt độ tăng dễ gây cháy rừng.
			
			
		}
		
	
		\item \mkstar{2} [11]
	
	\cauhoi{
		
		Kính lúp là gì? Viết công thức tính số bội giác của kính lúp khi ngắm chừng ở vô cực, nêu tên gọi của các đại lượng trong công thức.
		
		Áp dụng: Một quan sát viên có mắt bình thường dùng một kính lúp để quan sát một vật nhỏ. Mắt đặt sát kính và khoảng cực cận của quan sát viên là $\SI{25}{cm}$. Để góc trông ảnh qua kính lớn gấp 5 lần góc trông vật khi ảnh hiện ra ở vô cực thì quan sát viên này phải dùng kính lúp có độ tụ bằng bao nhiêu đi-ốp?
		
		
	}
	
	\loigiai{
		
		+ Kính lúp là dụng cụ quang học bổ trợ cho mắt quan sát các vật nhỏ. Tiêu cự của kính lúp khoảng vài centimét.
		
		+ Ta có:
		
		$$G_\infty =\dfrac{\text{Đ}}{f}$$ 
		
		Với: 
		
		- $G_\infty$ là số bội giác khi ngắm chừng ở vô cực.
		
		- $\text{Đ} = \text{OC}_\text{C}$ là khoảng cực cận của mắt (cm).
		
		- $f$ là tiêu cự của kính lúp (cm).
		
		* Áp dụng: 
		
		+ Ta có: $G_\infty=5 $.
		
		$$G_\infty = \dfrac{\text{OC}_\text C}{f} \Rightarrow f= \SI{5}{cm}= \SI{0,05}{m}.$$
		
		Độ tụ của kính
		
		$$D=\dfrac{1}{f} = \SI{20}{dp}.$$
		
		
	}

	

	
	
\end{enumerate}


\whiteBGstarEnd