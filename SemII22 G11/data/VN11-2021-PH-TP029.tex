\whiteBGstarBegin
\setcounter{section}{0}

\begin{enumerate}[label=\bfseries Câu \arabic*:]
	
	\item \mkstar{2} [21]
	
	\cauhoi{
		Cho một tia sáng đi từ nước $\left(n = \dfrac{4}{3}\right)$ ra không khí. Sự phản xạ toàn phần xảy ra khi góc tới thỏa mãn:
		
		\begin{mcq}(4)
			\item $i < 49^\circ$.	
			\item $i > 42^\circ$.	
			\item $i > 49^\circ$.	
			\item $i > 43^\circ$
		\end{mcq}
	}
	
	\loigiai{
		\textbf{Đáp án: C.}
		
		Ta có
		
		$$\sin i_\text{gh} =\dfrac{n_2}{n_1} = \dfrac{3}{4} \Rightarrow i_\text{gh} = \text{48,59}^\circ.$$
		
		Để xảy ra hiện tượng phản xạ toàn phần: $i \geq i_\text{gh}$
		
		Suy ra: $i \geq \text{48,59}^\circ$
	}
	
	\item \mkstar{1} [18]
	
	\cauhoi{
		
		Lúc trời nắng, mặt đường nhựa khô ráo, nhưng nhìn từ xa ta thấy mặt đường có vẻ bị ướt nước đó là kết quả của hiện tượng gì? Tại sao có hiện tượng đó xảy ra?
		
	}
	
	\loigiai{
		
		Mặt đường nhựa nóng, không khí tại gần mặt đất có nhiệt độ cao hơn không khí trên cao, dẫn đến chiết suất không khí tăng theo độ cao, các tia sáng từ bầu trời xanh có thể được khúc xạ toàn phần đến mắt người quan sát. Do không khí luôn có các dòng đối lưu gây nhiễu loạn chiết suất, hình ảnh thu được luôn dao động như khi nhìn hình ảnh bầu trời phản xạ từ mặt nước vậy nên ta có thể nhìn như thấy vũng nước trên đường. 
		
	}
	
	\item \mkstar{1} [14]
	
	\cauhoi{
		
		Nêu công dụng của cáp quang.
		
	}
	
	\loigiai{
		Ứng dụng của cáp quang:
		
		Trong công nghệ thông tin, cáp quang được dùng để truyền thông tin, dữ liệu dưới dạng tín hiệu ánh sáng.
		
		Trong nội soi y học.
	}
		\item \mkstar{1} [15]
		
		\cauhoi{
			
		Ngày nay, cáp quang được sử dụng rộng rãi (thay thế cáp đồng) để truyền tín hiệu trong viễn thông, em hãy cho biết cáp quang là ứng dụng của hiện tượng vật lý nào? Ưu điểm và nhược điểm của cáp quang? 
			
			
		}
		
		\loigiai{
			
			Cáp quang là ứng dụng của hiện tượng phản xạ toàn phần.
					
		*Ưu điểm:
		
		- Mỏng, dung lượng tải cao hơn cho phép nhiều kênh đi qua cáp của bạn.
		
		- Suy giảm tín hiệu ít - tín hiệu bị mất trong cáp quang ít hơn trong cáp đồng.
			
			
		*Nhược điểm:
		
		- Nối cáp khó, phải thẳng, không được gập. 
		
		- Chi phí cao.
			
		}

	
	
	\item \mkstar{1} [7]
	
	\cauhoi{
		Thế nào là hiện tượng phản xạ toàn phần? Nêu điều kiện để có phản xạ toàn phần.	
	}
	
	\loigiai{
		
		Hiện tượng phản xạ toàn phần là hiện tượng phản xạ toàn bộ tia sáng tới,    xảy ra ở mặt phân cách giữa hai môi trường trong suốt.
		
		Điểu kiện xảy ra phản xạ toàn phần: 
		
		$$n_1 >n_2.$$
		
		$$i \geq i_\text{gh}.$$
		
	}
	
	\item \mkstar{2} [36]
	
	\cauhoi{
		
		Chiếu một tia sáng đến mặt tiếp xúc nước - không khí. Tìm góc tới để xảy ra hiện tượng phản xạ toàn phần. (Biết chiết suất của nước đối với tia sáng là $\dfrac{4}{3}$)
		
		
	}
	
	\loigiai{
		
		Ta có
		
		$$\sin i_\text{gh} =\dfrac{n_2}{n_1} = \dfrac{3}{4} \Rightarrow i_\text{gh} = \text{48,59}^\circ.$$
		
		Để xảy ra hiện tượng phản xạ toàn phần: $i \geq i_\text{gh}$
		
		Suy ra: $i \geq \text{48,59}^\circ$
		
	}
	\item \mkstar{2} [24]
	
	\cauhoi{
		
		Chiếu tia sáng từ nước vào không khí sao cho tia sáng tới hợp với mặt nước một góc $60^\circ$. Cho chiết suất nước là $\dfrac{4}{3}$. Hỏi có xảy ra hiện tượng phản xạ toàn phần không? 
		
	}
	\loigiai{
		
		Ta có: 
		
		$$i =90^\circ -60^\circ =30^\circ.$$
		
		$$\sin i_\text{gh} = \dfrac{n_2}{n_1} \Rightarrow i_\text{gh} = \text{48,59}^\circ.$$
		
		Lại có:
		
		$$n_2 > n_1 \ \text{và} i<i_\text{gh}.$$
		
		Nên xảy ra hiện tượng khúc xạ ánh sáng.
	}
	
	\item \mkstar{2} [18]
	
	\cauhoi{
		
		Một tia sáng truyền từ một môi trường trong suốt có chiết suất 1,5 sang không khí. Tính góc tới của tia sáng để có tia ló ra không khí? Biết chiết suất không khí bằng 1.
		
	}
	
	\loigiai{
		
		Ta có:
		
		$$\sin i_\text{gh}= \dfrac{n_2}{n_1}= \text{0,67}\Rightarrow i_\text{gh} = \text{41,8}.$$
		
		Để có tia ló ra không khí  $i < i_\text{gh}$
		
		
	}
	\item \mkstar{3} [10]
	
	\cauhoi{
		Một tia sáng truyền từ môi trường có chiết suất $n_1 = \sqrt 6$, đến gặp mặt phân cách của môi trường thứ hai có chiết suất $n_2 = \sqrt 2$.
		  
		\begin{enumerate}[label=\alph*)]
			\item Tìm điều kiện của góc tới để không có tia sáng nào ra môi trường thứ hai.
			\item Góc tới $i$ phải bằng bao nhiêu để khi truyền qua mặt phân cách, tia sáng bị lệch so với phương ban đầu một góc bằng $i$?
		\end{enumerate}
	}
	
	\loigiai{
				
		\begin{enumerate}[label=\alph*)]
			\item ABC.Để không có tia sáng ló ra môi trường 2 $\Rightarrow$ hiện tượng phản xạ tòan phần
			
			$$\sin i_\text{gh} =\dfrac{n_2}{n_1} = \text{0,577} \Rightarrow i_\text{gh}=\text{35,24}^\circ.$$
			
			$$\Rightarrow i \geq \text{35,24}^\circ.$$
			
			\item Ta có:
			
			$$n_1 \sin i = n_2 \sin r$$
			
			$$\Rightarrow n_1 \sin i = n_2 \sin 2i$$
			
			$$n_1 \sin i = n_2 2\sin i \cos i.$$
			
			$$\Rightarrow \cos i  = \dfrac{n_1}{2n_2} = \dfrac{\sqrt 3}{2}\Rightarrow i =30^\circ.$$
			
			
		\end{enumerate}
	}
	
		\item \mkstar{3} [20]
	
	\cauhoi{
		
	Một tia sáng truyền từ môi trường trong suốt có chiết suất $n$ ra không khí, dưới góc tới $30^\circ$ thì tia khúc xạ ra không khí có hướng lệch so với tia tới một góc bằng $15^\circ$.
	
	\begin{enumerate}[label=\alph*)]
		\item Xác định giá trị chiết suất $n$ của môi trường.
		\item Để không có tia khúc xạ ra không khí thì phải tăng góc tới ít nhất bao nhiêu độ? 
	\end{enumerate} 
	
		
	}
	
	\loigiai{
		
	
		\begin{enumerate}[label=\alph*)]
			\item Chiết suất $n$ của môi trường
			
			$$D = r - i  \Rightarrow r = 45^\circ.$$  
			
			$$n \sin i = \sin r \Rightarrow n = \sqrt 2.$$           
			                       
			
			\item 
			Ta có:
			
			$$\sin i_\text{gh} = \dfrac{1}{n} \Rightarrow i_\text{gh} = 45^\circ.$$
			     
			để xảy ra phản xạ toàn phần thì $i \geq i_\text{gh} \Leftrightarrow i \geq 45^\circ.$
			
			$\Rightarrow$ góc tới tăng ít nhất $15^\circ$. 
			        
		\end{enumerate} 
	}
	

	
	\item \mkstar{3} [23]
	
	\cauhoi{
		
		Tia sáng đi từ thuỷ tinh có chiết suất $\sqrt 3$ đến môi trường chất lỏng có chiết suất $\sqrt 2$.
		
		\begin{enumerate}[label=\alph*)]
			\item Tìm góc giới hạn phản xạ toàn phần.
			\item Cho góc tới $45^\circ$ thì có tia sáng đi vào chất lỏng không? Nếu có, tìm góc khúc xạ.
		\end{enumerate}
		
	}
	
	\loigiai{
		
		\begin{enumerate}[label=\alph*)]
			\item Giới hạn phản xạ toàn phần
			
			$$\sin i_\text{gh} = \dfrac{n_2}{n_1} \Rightarrow i_\text{gh} = \text{54,73}^\circ.$$
			
			\item Ta có:
			
			$i < i_\text{gh}$ nên có tia khúc xạ.
			
			$$n_1 \sin i = n_2 \sin r \Rightarrow r =60^\circ.$$
			
		\end{enumerate}
	}
	
	
	
	\item \mkstar{3} [33]
	
	\cauhoi{
		Một tia sáng truyền từ môi trường có chiết suất $\sqrt 2$ hướng tới mặt phân cách với không khí.
		
	\begin{enumerate}[label=\alph*)]
		\item Tính góc giới hạn phản xạ toàn phần.
		\item Nếu góc tới của tia sáng là $48^\circ$ thì tia sáng có bị phản xạ toàn phần không? Tại sao?
	\end{enumerate}
	}
	
	\loigiai{
	
	\begin{enumerate}[label=\alph*)]
		\item Góc giới hạn phản xạ toàn phần
		
		$$\sin i_\text{gh} = \dfrac{n_2}{n_1} \Rightarrow i_\text{gh} = 45^\circ.$$
		
		\item Nếu góc tới của tia sáng là $48^\circ$ thì tia sáng xảy phản xạ toàn phần vì $i \geq i_\text{gh}$.
	\end{enumerate}
	
	
	}
	

\end{enumerate}	
\whiteBGstarEnd