%\setcounter{chapter}{1}
\chapter{Luyện tập: Lực từ. Cảm ứng từ}
\begin{enumerate}
	
	
	\item Một dòng điện đặt trong từ trường vuông góc với đường sức từ, chiều của lực từ tác dụng vào dòng điện sẽ không thay đổi khi
	\begin{mcq}
		\item đổi chiều dòng điện ngược lại.
		\item đổi chiều cảm ứng từ ngược lại.
		\item đồng thời đổi chiều dòng điện và đổi chiều cảm ứng từ.
		\item quay dòng điện một góc $90^\circ$ xung quanh đường sức từ.
	\end{mcq}
	
	
	\item Phát biểu nào sau đây là \textbf{không đúng}?
	\begin{mcq}
		\item Lực từ tác dụng lên dòng điện có phương vuông góc với dòng điện.
		\item Lực từ tác dụng lên dòng điện có phương vuông góc với đường cảm ứng từ.
		\item Lực từ tác dụng lên dòng điện có phương vuông góc với mặt phẳng chứa dòng điện và đường cảm ứng từ.
		\item Lực từ tác dụng lên dòng điện có phương tiếp tuyến với các đường cảm ứng từ.
	\end{mcq}
	
	\item  Phát biểu nào sau đây là \textbf{không đúng}?	
	\begin{mcq}
		\item Lực từ tác dụng lên dòng điện đổi chiều khi đổi chiều dòng điện.
		\item Lực từ tác dụng lên dòng điện đổi chiều khi đổi chiều đường cảm ứng từ.
		\item Lực từ tác dụng lên dòng điện đổi chiều khi tăng cường độ dòng điện.
		\item Lực từ tác dụng lên dòng điện không đổi chiều khi đồng thời đổi chiều dòng điện và đường cảm ứng từ.
	\end{mcq}
	
	\item Phát biểu nào sau đây là \textbf{không đúng}?
	\begin{mcq}
		\item Lực từ tác dụng lên một đoạn dây dẫn mang dòng điện đặt trong từ trường đều tỉ lệ thuận với cường độ dòng điện trong đoạn dây.
		\item Lực từ tác dụng lên một đoạn dây dẫn mang dòng điện đặt trong từ trường đều tỉ lệ thuận với chiều dài của đoạn dây.
		\item  Lực từ tác dụng lên một đoạn dây dẫn mang dòng điện đặt trong từ trường đều tỉ lệ thuận với góc hợp bởi đoạn dây và đường sức từ.	
		\item Lực từ tác dụng lên một đoạn dây dẫn mang dòng điện đặt trong từ trường đều tỉ lệ thuận với cảm ứng từ tại điểm đặt đoạn dây.
	\end{mcq}
	
	\item Phát biểu nào dưới đây là đúng?
	
	Cho một đoạn dây dẫn mang dòng điện I đặt song song với đường sức từ, chiều của dòng điện ngược chiều với chiều của đường sức từ. 
	\begin{mcq}
		\item Lực từ luôn bằng không khi tăng cường độ dòng điện
		\item Lực từ tăng khi tăng cường độ dòng điện.	
		\item Lực từ giảm khi tăng cường độ dòng điện.	
		\item Lực từ đổi chiều khi ta đổi chiều dòng điện.
	\end{mcq}
	
	\item Một đoạn dây dẫn dài 5 cm đặt trong từ trường đều và vuông góc với vectơ cảm ứng từ. Dòng điện chạy qua dây có cường độ $\text{0,75}$ A. Lực từ tác dụng lên đoạn dây đó là $3\cdot 10^{-2}$ N. Cảm ứng từ của từ trường đó có độ lớn là
	\begin{mcq}(4)
		\item $\text{0,4}\ \text{T}$.
		\item $\text{0,8}\ \text{T}$.
		\item $\text{1,0}\ \text{T}$.	
		\item $\text{1,2}\ \text{T}$.
	\end{mcq}
	
	\item Phát biểu nào sau đây là \textbf{không đúng}?
	
	Một đoạn dây dẫn thẳng mang dòng điện I đặt trong từ trường đều thì 
	
	\begin{mcq}
		\item lực từ tác dụng lên mọi phần của đoạn dây.
		\item lực từ chỉ tác dụng vào trung điểm của đoạn dây.
		\item lực từ chỉ tác dụng lên đoạn dây khi nó không song song với đường sức từ.
		\item lực từ tác dụng lên đoạn dây có điểm đặt là trung điểm của đoạn dây.
	\end{mcq}
	\item Một đoạn dây dẫn thẳng MN dài 6 cm có dòng điện I = 5 A đặt trong từ trường đều có cảm ứng từ $B = \text{0,5}\ \text{T}$. Lực từ tác dụng lên đoạn dây có độ lớn $F = \text{7,5}\cdot 10^{-2}\ \text{N}$. Góc $\alpha$ hợp bởi dây MN và đường cảm ứng từ là
	
	\begin{mcq}(4)
		\item $30^\circ$.
		\item  $60^\circ$.
		\item $45^\circ$.
		\item $90^\circ$.
	\end{mcq}
	
	\item Phát biểu nào sau đây là \textbf{không đúng}? 
	
	\begin{mcq}
		\item Cảm ứng từ là đại lượng đặc trưng cho từ trường về mặt tác dụng lực. 
		\item Độ lớn của cảm ứng từ được xác định theo công thức $B=\dfrac{F}{BIl\sin \alpha}$ phụ thuộc vào cường độ dòng điện $I$ và chiều dài đoạn dây dẫn đặt trong từ trường
		\item Độ lớn của cảm ứng từ được xác định theo công thức  $B=\dfrac{F}{BIl\sin \alpha}$ không phụ thuộc vào cường độ dòng điện $I$ và chiều đài đoạn dây dẫn đặt trong từ trường.
		\item Cảm ứng từ là đại lượng véc-tơ
	\end{mcq}
	
	\item Chiều của lực từ tác dụng lên đoạn dây dẫn mang dòng điện, thường được xác định bằng quy tắc
	\begin{mcq}(2)
		\item bàn tay trái.
		\item bàn tay phải.
		\item nắm tay trái.
		\item nắm tay phải.
	\end{mcq}
\end{enumerate}
\begin{center}
	\textbf{ĐÁP ÁN}
	\begin{longtable}[\textwidth]{|p{0.15\textwidth}|p{0.15\textwidth}|p{0.15\textwidth}|p{0.15\textwidth}|p{0.15\textwidth}|}
		% --- first head
		\hline%\hspace{2 pt}
		\multicolumn{1}{|c}{\textbf{Câu 1}} & \multicolumn{1}{|c|}{\textbf{Câu 2}} & \multicolumn{1}{c|}{\textbf{Câu 3}} &
		\multicolumn{1}{c|}{\textbf{Câu 4}} &
		\multicolumn{1}{c|}{\textbf{Câu 5}} \\
		\hline
		C.&D. &C. &C. &A. 	\\
		\hline
		
		
		\multicolumn{1}{|c|}{\textbf{Câu 6}} &
		\multicolumn{1}{c|}{\textbf{Câu 7}} &
		\multicolumn{1}{c|}{\textbf{Câu 8}} &
		\multicolumn{1}{c|}{\textbf{Câu 9}} & \multicolumn{1}{c|}{\textbf{Câu 10}} \\
		\hline
		B. &B. &A.&B. &A. \\
		\hline	
		
		
	\end{longtable}
	
	
	
\end{center}









