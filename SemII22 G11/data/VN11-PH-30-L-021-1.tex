\chapter{Suất điện động cảm ứng và cường độ dòng điện cảm ứng}
\section{Lý thuyết trọng tâm}
\subsection{Định luật Fa-ra-đây về cảm ứng điện từ}
Độ lớn của suất điện động cảm ứng trong mạch kín tỉ lệ với tốc độ biến thiên của từ thông qua mạch.
\begin{equation}
\left| e_\text{c}\right| =k\cdot \left|\dfrac{\Delta \Phi}{\Delta t} \right|,
\end{equation}
Trong hệ SI, hệ số tỉ lệ $k=1$.
  
Theo định luật Len-xơ thì trong hệ SI suất điện động cảm ứng được viết dưới dạng:
\begin{equation}
e_\text{c}=-\dfrac{\Delta \Phi}{\Delta t},
\end{equation}
trong đó,
\begin{itemize}
	\item $\Delta \Phi$ là độ biến thiên từ thông,
	\item $\Delta t$ là khoảng thời gian, 
	\item $e_\text{c}$ là suất điện động cảm ứng. 
\end{itemize}

\subsection{Quan hệ giữa suất điện động cảm ứng và định luật Len-xơ}
Sự xuất hiện dấu $(-)$ trong công thức trên là phù hợp với định luật Len-xơ

Nếu $\Phi$ tăng thì $e_\text{c}<0$: Chiều của suất điện động cảm ứng (chiều của dòng điện cảm ứng) ngược chiều với chiều của mạch.

Nếu $\Phi$ giảm thì $e_\text{c}>0$: Chiều của suất điện động cảm ứng (chiều của dòng điện cảm ứng) cùng chiều với chiều của mạch.

\subsection{Cường độ dòng điện cảm ứng}

\begin{equation}
I_\text{c}=\dfrac{e_\text{c}}{R}=-\dfrac{\Delta \Phi}{R\Delta t},
\end{equation}
trong đó,
\begin{itemize}
	\item $\Delta \Phi$ là độ biến thiên từ thông,
	\item $\Delta t$ là khoảng thời gian, 
	\item $e_\text{c}$ là suất điện động cảm ứng. 
	\item $i_\text{c}$ là cường độ dòng điện cảm ứng. 
	\item $R$ là điện trở của vật dẫn.
\end{itemize}
\section{Bài tập}
%\begin{dang}{Suất điện động cảm ứng \\ và cường độ dòng điện cảm ứng}
%\end{dang}

\textbf{Phương pháp giải}

Công thức tính từ thông qua khung dây có $N$ vòng dây:
\begin{equation}
\Phi=NBS\cos \alpha.
\end{equation}

Công thức tính suất điện động cảm ứng:
\begin{equation}
e_\text{c}=-\dfrac{\Delta \Phi}{\Delta t},
\end{equation}

Công thức tính cường độ dòng điện cảm ứng:
\begin{equation}
I_\text{c}=\dfrac{e_\text{c}}{R}=-\dfrac{\Delta \Phi}{R\Delta t},
\end{equation}

\vspace{1em}
{\viduii{2}{	
Một khung dây phẳng, diện tích $20\ \text{cm}^2$, gồm 10 vòng dây đặt trong từ trường đều, góc giữa $\vec{B}$ và vectơ pháp tuyến là $30^\circ$, độ lớn cảm ứng từ $B=2\cdot 10^{-4}\ \text{T}$, làm cho từ trường giảm đều về 0 trong thời giam $\text{0,01}\ \text{s}$. Hãy xác định độ lớn suất điện động cảm ứng sinh ra trong khung dây?
\begin{mcq}(4)
	\item $\text{5,19}\cdot 10^{-4}\ \text{V}$.
	\item $\text{3,46}\cdot 10^{-4}\ \text{V}$.
	\item $\text{6,92}\cdot 10^{-4}\ \text{V}$.
	\item $\text{1,26}\cdot 20^{-6}\ \text{V}$.
	
\end{mcq}}{
\begin{center}
	\textbf{Hướng dẫn giải:}
\end{center}

Độ biến thiên từ thông xuyên qua khung dây là $\Delta \Phi=\Phi_2-\Phi_1=0-BS\cos \alpha=-\text{3,46}\cdot 10^{-6}\ \text{Wb}$.
 
Độ lớn suất điện động cảm ứng sinh ra trong khung dây là $\left| e_\text{c}\right| =\left| -\dfrac{\Delta \Phi}{\Delta t}\right|  =\text{3,46}\cdot 10^{-4}\ \text{V}$.

\textbf{Đáp án: B.}}
}

{\viduii{2}{	
Một khung dây dẫn đặt vuông góc với một từ trường đều, cảm ứng từ $B$ có độ lớn biến đổi theo thời gian. Tính tốc độ biến thiên của cảm ứng từ, biết rằng cường độ dòng điện cảm ứng là $I_\text{c}=\text{0,5}\ \text{A}$, điện trở của khung là $R=2\ \Omega$  và diện tích của khung là $S=100\ \text{cm}^2$.   
\begin{mcq}(4)
	\item $100\ \text{T/s}$.
	\item  $200\ \text{T/s}$.
	\item  $300\ \text{T/s}$.
	\item  $50\ \text{T/s}$.
	
\end{mcq}}{
\begin{center}
	\textbf{Hướng dẫn giải:}
\end{center}
	
$\left| I_\text{c}\right| =\left| \dfrac{e_\text{c}}{R}\right|  \Rightarrow \left| e_\text{c}\right|  =\left| I_\text{c}\right|\cdot R =1\ \text{V}$.

$\dfrac{\left| \Delta B\right|}{\Delta t}$ được gọi là tốc độ biến thiên của cảm ứng từ.

Do đó: $\left| e_\text{c}\right|=\dfrac{\left| \Delta B\right|S}{\Delta t}\Rightarrow \dfrac{\left| \Delta B\right|}{\Delta t}=\dfrac{\left| e_\text{c}\right|}{S}=100\ \text{T/s} $.


\textbf{Đáp án: A.}}
}

{\viduii{2}{	
Một ống dây hình trụ dài gồm $10^3$ vòng dây, diện tích mỗi vòng dây $S=100\ \text{cm}^2$. Ống dây có điện trở $R=16\ \Omega$, hai đầu nối đoản mạch và được đặt trong từ trường đều có véctơ cảm ứng từ song song với trục của ống dây và có độ lớn tăng đều $10^{-2}\ \text{T/s}$. Tính công suất tỏa nhiệt của ống dây.
\begin{mcq}(4)
		\item $\text{6,25}\ \text{mW}$.
		\item $\text{6,25}\cdot 10^{-4}\ \text{W}$.
		\item $\text{6,25}\ \text{W}$.
		\item $\text{6,25}\cdot 10^{-2}\ \text{W}$.		
	\end{mcq}}
{\begin{center}
		\textbf{Hướng dẫn giải:}
\end{center}
		
		Độ lớn suất điện động cảm ứng là  $\left| e_\text{c}\right|=\dfrac{N \left| \Delta B\right|S}{\Delta t}=\text{0,1}\ \text{V}$.
		  
		Cường độ dòng điện chạy qua ống dây là $I=\dfrac{\left| e_\text{c}\right| }{R}=\text{0,625}\cdot 10^{-2}\ \text{A}$.
		
	Công suất tỏa nhiệt của ống dây là  $P=I^2 R=\text{6,25}\cdot 10^{-4}\ \text{W}$.
			
	\textbf{	Đáp án: B.}}
	}






