\whiteBGstarBegin
\setcounter{section}{0}
\section{Lý thuyết: Cấu tạo mắt và sự điều tiết của mắt}
\begin{enumerate}[label=\bfseries Câu \arabic*:]
	
	\item \mkstar{1} [10]
	
	\cauhoi{
		
		Điểm cực viễn của mắt là gì? Khi quan sát vật đặt tại điểm cực viễn thì tiêu cự của thủy tinh thể mắt có giá trị như thế nào? Sự điều tiết của mắt là gì?
		
		
	}
	
	\loigiai{
		
		+ Điểm cực viễn của mắt là điểm xa nhất trên trục (chính) của mắt mà mắt còn nhìn rõ khi không điều tiết.
		
		+ Khi quan sát vật đặt tại điểm cực viễn thì tiêu cự của thủy tinh thể mắt có giá trị lớn nhất.
		
		+ Sự điều tiết của mắt là hoạt động của mắt để cho ảnh của các vật ở cách mắt những khoảng khác nhau vẫn được tạo ra ở màn lưới.
		
	}
	
	\item \mkstar{2} [10]
	
	\cauhoi{
		
		Khoảng cách từ quang tâm thấu kính mắt đến màng lưới của một người có mắt bình thường là $\SI{1,5}{cm}$. Điểm cực viễn của mắt nằm ở đâu? Tính độ tụ của mắt khi mắt quan sát vật đặt ở điểm cực viễn.
		
		
	}
	
	\loigiai{
		Điểm cực viễn nằm ở vô cùng.
		
		Độ tụ của mắt khi mắt quan sát vật đặt ở điểm cực viễn
		
		$$D=\dfrac{1}{\text{OC}_\text{V}} + \dfrac{1}{\text{OV}} =  \dfrac{1}{\text{0,015}}= \SI{67}{dp}.$$
		
		
	}
\end{enumerate}
\section{Lý thuyết: Các tật của mắt và cách khắc phục}
\begin{enumerate}[label=\bfseries Câu \arabic*:]
	
	\item \mkstar{2} [26]
	
	\cauhoi{
		Trong khoảng thời gian gần đây, mắt của bạn Lan có hiện tượng khi nhìn các vật ở xa thấy hình ảnh bị mờ, nhòe, không rõ. Khi đọc sách báo Lan thì phải để rất gần mắt. Bạn cũng ngồi sát ti vi thì mới có thể theo dõi bộ phim bạn yêu thích. Mắt của bạn Lan bị tật gì? Giải thích. Đề xuất cách khắc phục. 
		
	}
	
	\loigiai{
		
		* Mắt của Lan bị tật cận thị.
		
		* Giải thích: 
		
		+ Điểm cực viễn cách mắt khoảng không lớn nên khi nhìn xa ảnh bị nhòe.
		
		+ Điểm cực cận cách mắt gần hơn bình thường nên khi đọc sách hay xem ti vi phải để ở gần mắt.
		
		* Cách khắc phục: Đeo thấu kính phân kì có độ tụ thích hợp.
		
		
	}
	

\end{enumerate}	
\whiteBGstarEnd