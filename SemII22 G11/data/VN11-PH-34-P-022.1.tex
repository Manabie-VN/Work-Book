\chapter{Luyện tập: Khúc xạ ánh sáng}
\begin{enumerate}
	\item {\textbf{(Đề chính thức của BGDĐT - 2018)} Chiết suất của nước và của thủy tinh đối với một ánh sáng đơn sắc có giá trị lần lượt là 1,333 và 1,532. Chiết suất tỉ đối của nước đối với thủy tinh ứng với ánh sáng đơn sắc này là
		\begin{mcq}(4)
			\item 0,199.			
			\item 0,870.			
			\item 1,433.			
			\item 1,149.
		\end{mcq}
	}
	\item{\textbf{(Đề chính thức của BGDĐT - 2018)} Chiếu một tia sáng đơn sắc từ không khí tới mặt nước với góc tới $60^\circ$, tia khúc xạ đi vào trong nước với góc khúc xạ là $r$. Biết chiết suất của không khí và của nước đối với ánh sáng đơn sắc này lần lượt là 1 và 1,333. Giá trị của $r$ là
		\begin{mcq}(4)
			\item $\text{37,97}^\circ$.			
			\item $\text{22,03}^\circ$.			
			\item $\text{40,52}^\circ$.			
			\item $\text{19,48}^\circ$.
		\end{mcq}
	}
	\item{Tính tốc độ của ánh sáng trong thủy tinh. Biết thủy tinh có chiết suất $n = \text{1,6}$ và tốc độ ánh sáng trong chân không là $c = 3\cdot 10^8\ \text{m/s}$.
		\begin{mcq}(4)
			\item 2,23$\cdot 10^8\ \text{m/s}$.		
			\item 1,875$\cdot 10^8\ \text{m/s}$. 		
			\item 2,75$\cdot 10^8\ \text{m/s}$.		
			\item 1,5$\cdot 10^8\ \text{m/s}$.
		\end{mcq}
	}
	\item {Một tia sáng truyền từ môi trường A vào môi trường B dưới góc tới $6^\circ$ thì góc khúc xạ là $8^\circ$. Tính tốc độ ánh sáng trong môi trường A. Biết tốc độ ánh sáng trường môi trường B là 2$\cdot 10^5$ km/s.
		\begin{mcq}(4)
			\item 2,25$\cdot 10^5$ km/s. 		
			\item 2,3$\cdot 10^5$ km/s.		
			\item l,5$\cdot 10^5$ km/s. 		
			\item 2,5$\cdot 10^5$ km/s.
		\end{mcq}
	}
	\item{Tia sáng đi từ nước có chiết suất $n_1 = \dfrac{4}{3}$ sang thủy tinh có chiết suất $n_2 = \text{1,5}$ với góc tới $i = 30^\circ$. Góc khúc xạ và góc lệch $D$ tạo bởi tia khúc xạ và tia tới lần lượt là
		\begin{mcq}(4)
			\item $\text{26,4}^\circ$ và  $\text{3,6}^\circ$.	
			\item $\text{50,34}^\circ$ và  $\text{9,7}^\circ$.		
			\item $\text{34,23}^\circ$ và  $\text{4,23}^\circ$.			
			\item $\text{76,98}^\circ$ và  $\text{47}^\circ$.	
		\end{mcq}
	}
	\item{Tia sáng truyền trong không khí tới gặp mặt thoáng của chất lỏng có chiết suất $n =\sqrt 3$. Nếu tia phản xạ và tia khúc xạ vuông góc với nhau thì góc tới bằng 
		\begin{mcq}(4)
			\item $30^\circ$.			
			\item $60^\circ$.			
			\item $75^\circ$.		
			\item $45^\circ$.
		\end{mcq}
	}
	\item{Tia sáng truyền trong không khí tới gặp mặt thoáng của chất lỏng có chiết suất $n = \text{1,6}$. Nếu tia phản xạ và tia khúc xạ hợp với nhau một góc $100^\circ$ thì góc tới bằng
		\begin{mcq}(4)
			\item $36^\circ$.			
			\item $60^\circ$.				
			\item $72^\circ$.				
			\item $51^\circ$.
		\end{mcq}
	}
	\item{Một thợ lặn ở dưới nước nhìn thấy Mặt Trời ở độ cao $60^\circ$ so với đường chân trời. Biết chiết suất của nước là $n = \dfrac{4}{3}$. Tính độ cao thực của Mặt Trời so với đường chân trời.
		\begin{mcq}(4)
			\item $38^\circ$.			
			\item $60^\circ$.				
			\item $72^\circ$.				
			\item $48^\circ$.
		\end{mcq}
	}
	\item{Ba môi trường trong suốt (1), (2), (3) có thể đặt tiếp giáp nhau. Với cùng góc tới $i = 60^\circ$; nếu ánh sáng truyền từ (1) vào (2) thì góc khúc xạ là $45^\circ$; nếu ánh sáng truyền từ (1) vào (3) thì góc khúc xạ là $30^\circ$. Nếu ánh sáng truyền từ (2) vào (3) vẫn với góc tới $i$ thì góc khúc xạ gần giá trị nào nhất sau đây?
		\begin{mcq}(4)
			\item $36^\circ$.			
			\item $60^\circ$.				
			\item $72^\circ$.				
			\item $51^\circ$.
		\end{mcq}
	}
	\item{ Một cái gậy dài 2 m cắm thẳng đứng ở đáy hồ. Gậy nhô lên khỏi mặt nước 0,5 m. Ánh sáng Mặt Trời chiếu xuống hồ theo phương hợp với pháp tuyến của mặt nước góc $60^\circ$. Biết chiết suất của nước là $n = \dfrac{4}{3}$. Tìm chiều dài bóng của cây gậy in trên đáy hồ.
		\begin{mcq}(4)
			\item 200 cm.			
			\item 180 cm.			
			\item 175 cm.			
			\item 250 cm.
		\end{mcq}
	} 
	
	\item{ Một cây cọc dài được cắm thẳng đứng xuống một bể nước chiết suất $n = \dfrac{4}{3}$. Phần cọc nhô ra ngoài mặt nước là 30 cm, bóng của nó trên mặt nước dài 40 cm và dưới đáy bể nước dài 190 cm. Tính chiều sâu của lớp nước:
		\begin{mcq}(4)
			\item 200 cm.			
			\item 180 cm.			
			\item 175 cm.			
			\item 250 cm.
		\end{mcq}
	}
	\item{ Một cái máng nước sâu 30 cm rộng 40 cm có hai thành bên thẳng đứng. Lúc máng cạn nước thì bóng râm của thành A kéo dài tới đúng chân thành đối diện. Người ta đổ nước vào máng đến một độ cao $h$ thì bóng của thành A ngắn bớt đi 7 cm so với trước. Biết chiết suất của nước là $n = \dfrac{4}{3}$. Tính $h$.
		\begin{mcq}(4)
			\item 20 cm. 			
			\item 12 cm. 			
			\item 15 cm.			 
			\item 25 cm.	 
		\end{mcq}
	}
	\item{ Một cái đinh được cắm vuông góc vào tâm O một tấm gỗ hình tròn có bán kính $R = 5\ \text{cm}$. Tấm gỗ được thà nổi trên mặt thoáng của một chậu nước. Đầu A của đinh ở trong nước. Cho chiết suất của nước là $n = \dfrac{4}{3}$. Cho chiều dài OA của đinh ở trong nước là 8,7 cm. Hỏi mắt ở trong không khí, nhìn theo mép của tấm gỗ sẽ thấy đầu đinh ở cách mặt nước bao nhiêu centimet?
		\begin{mcq}(4)
			\item 6,5 cm.			
			\item 7,2 cm.			
			\item 4,5 cm.			
			\item 5,6 cm.
		\end{mcq}
	}
	
\end{enumerate}
\textbf{ĐÁP ÁN}
\begin{longtable}[\textwidth]{|p{0.1\textwidth}|p{0.1\textwidth}|p{0.1\textwidth}|p{0.1\textwidth}|p{0.1\textwidth}|p{0.1\textwidth}|p{0.1\textwidth}|p{0.1\textwidth}|}
	% --- first head
	\hline%\hspace{2 pt}
	\multicolumn{1}{|c}{\textbf{Câu 1}} & \multicolumn{1}{|c|}{\textbf{Câu 2}} & \multicolumn{1}{c|}{\textbf{Câu 3}} &
	\multicolumn{1}{c|}{\textbf{Câu 4}} &
	\multicolumn{1}{c|}{\textbf{Câu 5}} &
	\multicolumn{1}{c|}{\textbf{Câu 6}} &
	\multicolumn{1}{c|}{\textbf{Câu 7}} &
	\multicolumn{1}{c|}{\textbf{Câu 8}} \\
	\hline
	B.&C. &B . &C. &A. &B. &D. &D.	\\
	\hline
	
	\multicolumn{1}{|c|}{\textbf{Câu 9}} & \multicolumn{1}{c|}{\textbf{Câu 10}} & \multicolumn{1}{c|}{\textbf{Câu 11}} &
	\multicolumn{1}{c|}{\textbf{Câu 12}} &
	\multicolumn{1}{c|}{\textbf{Câu 13}} &
	\multicolumn{1}{c|}{\textbf{Câu 14}} &
	\multicolumn{1}{c|}{\textbf{Câu 15}} &
	\multicolumn{1}{c|}{}  \\
	\hline
	A. &A. &A. &B. &D. &&&	\\
	\hline
\end{longtable}
