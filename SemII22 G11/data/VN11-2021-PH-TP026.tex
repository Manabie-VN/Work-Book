\whiteBGstarBegin
\setcounter{section}{0}
\section{Lý thuyết: Hiện tượng tự cảm. Hệ số tự cảm}
\begin{enumerate}[label=\bfseries Câu \arabic*:]
	
	\item \mkstar{1} [21]
	
	\cauhoi{
		Đơn vị của hệ số tự cảm là
		\begin{mcq} (4)
			\item Vôn (V).				
			\item Tesla (T).	
			\item Vêbe (Wb).	
			\item Henry (H).	
		\end{mcq}
		
	}
	
	\loigiai{
		
		\textbf{Đáp án: D.}
		
		
		
	}

	\item \mkstar{1} [6]
	
	\cauhoi{
		Hiện tượng tự cảm là gì? Viết công thức tính độ tự cảm của ống dây.
	}
	
	\loigiai{
		
	Hiện tượng tự cảm là hiện cảm ứng điện từ xảy ra trong một mạch có dòng điện mà sự biến thiên từ thông qua mạch được gây ra bởi sự biến thiên của cường độ dòng điện trong mạch.
				
	Độ tự cảm của một ống dây
	
	$$L = 4\pi \cdot 10^{-7} \cdot \dfrac{N^2}{l} \cdot S.$$
	}
	
	
	\item \mkstar{2} [37]
	
	\cauhoi{
		
		Một ống dây lõi không khí dài $\SI{50}{cm}$, có $1000$ vòng dây. Bán kính của ống dây  là $\SI{10}{cm}$. Giả thiết rằng từ trường trong ống dây là từ trường đều. Tính độ tự cảm của ống dây.
		
	}
	
	\loigiai{
		
		Diện tích của ống dây
		
		$$S=\pi r^2=\SI{0,0314}{m}^2.$$

		Độ tự cảm của ống dây
		
		$$L = 4\pi \cdot 10^{-7} \cdot \dfrac{N^2}{l} \cdot S = \SI{0,079}{H}.$$
		
	}
	\item \mkstar{2} [13]
	
	\cauhoi{
		
		Ống dây điện hình trụ có lõi không khí, chiều dài $l = \SI{20}{cm}$ gồm $N = 500$ vòng dây, độ tự cảm của ống dây là $\SI{0,008}{H}$. Tính đường kính tiết diện của ống dây. 
	}
	\loigiai{
		
		Đường kính tiết diện của ống dây
		
		$$L = \dfrac{4\pi \cdot 10^{-7} N^2 S}{l} \Rightarrow S = \text{5,1} \cdot 10^{-3}\ \text{m}^2.$$
		
		$$d = \sqrt{\dfrac{4S}{\pi}} = \SI{0,0805}{m}.$$
		
	}
\end{enumerate}
\section{Lý thuyết: Suất điện động tự cảm}
\begin{enumerate}[label=\bfseries Câu \arabic*:]
	
	
		\item \mkstar{1} [21]
	
	\cauhoi{
		Phát biểu nào sau đây là không đúng?
		
		\begin{mcq} 
			\item Hiện tượng cảm ứng điện từ chỉ tồn tại trong khoảng thời gian từ thông qua mạch kín biến thiên.				
			\item Suất điện động được sinh ra do hiện tượng tự cảm gọi là suất điện động tự cảm.
			\item Hiện tượng tự cảm là một trường hợp đặc biệt của hiện tượng cảm ứng điện từ.	
			\item Suất điện động cảm ứng cũng là suất điện động tự cảm.
		\end{mcq}
		
	}
	
	\loigiai{
		
		\textbf{Đáp án: D.}
		
		
		
	}
	
	
	\item \mkstar{2} [37]
	
	\cauhoi{
		Một ống dây lõi không khí dài $\SI{50}{cm}$, có $1000$ vòng dây. Bán kính của ống dây  là $\SI{10}{cm}$. Giả thiết rằng từ trường trong ống dây là từ trường đều. Trong thời gian $\SI{2}{s}$ dòng điện trong ống dây tăng từ $0$ đến $\SI{4}{A}$. Hãy tính độ lớn suất điện động tự cảm trong ống dây?
		
	}
	
	\loigiai{
		
		Diện tích của ống dây
		
		$$S=\pi r^2=\SI{0,0314}{m}^2.$$
		
		Độ tự cảm của ống dây
		
		$$L = 4\pi \cdot 10^{-7} \cdot \dfrac{N^2}{l} \cdot S = \SI{0,079}{H}.$$
		
		Suất điện động tự cảm trong ống dây
		
		$$|e_\text{tc}| = L \left|\dfrac{\Delta i}{\Delta t} \right| = \SI{0,158}{V}.$$
		
	}
	
	

	\item \mkstar{2} [20]
	
	\cauhoi{
		
		Một mạch điện có độ tự cảm $L = \SI{40}{mH}$, cho dòng điện qua mạch giảm từ $\SI{0,4}{A}$ xuống $0$ trong thời gian $\SI{0,4}{s}$.Tính độ lớn suất điện động tự cảm của mạch ?
		
	}
	\loigiai{
		
		Độ lớn suất điện động tự cảm trong ống dây
		
		$$|e_\text{tc}| =\left|-L \dfrac{\Delta i}{\Delta t}\right| = \SI{0,04}{V}.$$
	}
	
	\item \mkstar{2} [16]
	
	\cauhoi{
		
		Ống dây điện hình trụ dài $\SI{40}{cm}$ gồm $1000$ vòng dây, diện tích mỗi vòng là $S = \SI{100}{cm}^2$.
		
		\begin{enumerate}[label=\alph*)]
			\item Tính hệ số tự cảm của ống dây.
			\item Dòng điện qua cuộn cảm đó tăng đều từ $0$ đến $\SI{10}{A}$ trong thời gian $\SI{0,1}{s}$. Tính độ lớn suất điện động tự cảm xuất hiện trong ống dây.
		\end{enumerate}
	}
	
	\loigiai{
		
		\begin{enumerate}[label=\alph*)]
			\item Tính hệ số tự cảm của ống dây
			
			$$L = 4\pi \cdot 10^{-7} \dfrac{N^2}{l} S = \SI{0,0314}{H}.$$
			
			\item Độ lớn suất điện động tự cảm xuất hiện trong ống dây
			
			$$|e_\text{tc}|= L \left|\dfrac{\Delta i}{\Delta t}\right| = \SI{3,14}{V}.$$ 
		\end{enumerate}
	}
	
	\item \mkstar{2} [19]
	
	\cauhoi{
		Một ống dây có độ tự cảm $L = \SI{0,2}{H}$.
		\begin{enumerate}[label=\alph*)]
			\item Cho dòng điện qua ống dây tăng từ $0$ đến $\SI{2}{A}$ trong thời gian $\SI{0,2}{s}$. Tính độ lớn suất điện động tự cảm trong ống dây. 
			\item Nếu cho dòng điện qua ống dây biến thiên một lượng $\SI{4}{A}$ trong thời gian $\Delta t$ thì suất điện động tự cảm trong ống dây có độ lớn là $\SI{8}{V}$. Tính giá trị của $\Delta t$. 
		\end{enumerate}
	
	}
	\loigiai{
		
		\begin{enumerate}[label=\alph*)]
			\item Độ lớn suất điện động tự cảm trong ống dây
			
			$$e_\text{tc} =\left|-L \dfrac{\Delta i}{\Delta t}\right| = \SI{2}{V}.$$

			\item Ta có:
			
			$$ \Delta t =\left|-L \dfrac{\Delta i}{e_\text{tc}}\right| = \SI{0,1}{s}.$$
		\end{enumerate}
	}
	
	
\end{enumerate}	
\whiteBGstarEnd