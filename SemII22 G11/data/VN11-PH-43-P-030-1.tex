%\setcounter{chapter}{1}
\chapter{Luyện tập: Kính thiên văn}
\begin{enumerate}
	\item %câu 1
	Người ta dùng kính thiên văn để quan sát những
	\begin{mcq}(2)
		\item vật rất nhỏ ở rất xa.
		\item vật nhỏ ở ngang trước vật kính
		\item thiên thể ở xa.
		\item ngôi nhà cao tầng.
	\end{mcq}
	\item %câu 2
	Kính thiên văn khúc xạ gồm hai thấu kính hội tụ:
	\begin{mcq}
		\item Vật kính có tiêu cực nhỏ, thị kính có tiêu cự lớn, khoảng cách giữa chúng là cố định.
		\item Vật kính có tiêu cực nhỏ, thị kính có thiêu cự lớn, khoảng cách giữa chúng có thể thay đổi được.
		\item Vật kính có tiêu cực lớn, thị kính có tiêu cự nhỏ, khoảng cách giữa chúng có thể thay đổi được.
		\item Vật kính và thị kính có tiêu cự bằng nhau, khoảng cách giữa chúng cố định.
	\end{mcq}
	\item %câu 3
	Khi nói về cách sử dụng kính thiên văn, phát biểu nào sau đây đúng?
	\begin{mcq}
		\item Điều chỉnh khoảng cách giữa vật và vật kính sao cho ảnh của vật qua kính nằm trong khoảng nhìn rõ của mắt.
		\item Điều chỉnh khoảng cách giữa vật kính và thị kính sao cho ảnh của vật qua kính nằm trong khoảng nhìn rõ của mắt.
		\item Giữ nguyên khoảng cách giữa vật kính và thị kính, thay đổi khoảng cách giữa kính với vật sao cho ảnh của vật qua kính nằm trong khoảng nhìn rõ của mắt.
		\item Giữ nguyên khoảng cách giữa vật kính và thị kính, thay đổi khoảng cách giữa mắt và thị kính sao cho ảnh của vật qua kính nằm trong khoảng nhìn rõ của mắt.
	\end{mcq}
	\item %câu 4
	Người ta điều chỉnh kính thiên văn theo cách nào sau đây?
	\begin{mcq}
		\item Thay đổi khoảng cách giữa vật kính và thị kính bằng cách giữ nguyên vật kính, dịch chuyển thị kính sao cho nhìn thấy ảnh của vật to và rõ nhất.
		\item Thay đổi khoảng cách giữa vật kính và thị kính bằng cách dịch chuyển thị kính sao cho nhìn thấy ảnh của vật to và rõ nhất.
		\item Thay đổi khoảng cách giữa vật kính và thị kính bằng cách giữ nguyên thị kính, dịch chuyển thị kính sao cho nhìn thấy ảnh của vật to và rõ nhất.
		\item Dịch chuyển thích hợp cả vật kính và thị kính sao cho nhìn thấy ảnh của vật to và rõ nhất.
	\end{mcq}
	\item %câu 5
	Kính thiên văn mà vật kính có tiêu cự $\SI{1,2}{\meter}$ và thị kính có tiêu cự $\SI{4}{\centi\meter}$. Người mắt tốt dùng kính thiên văn để quan sát Mặt Trăng trong trạng thái không điều tiết. Xác định khoảng cách giữa hai thấu kính và độ bội giác khi đó.
	\begin{mcq}(2)
		\item $L=\SI{124}{\centi\meter}$ và $G=30$.
		\item $L=\SI{124}{\centi\meter}$ và $G=3$.
		\item $L=\SI{116}{\centi\meter}$ và $G=30$.
		\item $L=\SI{116}{\centi\meter}$ và $G=3$.
	\end{mcq}
	\item %câu 6
	Một kính thiên văn học sinh gồm vật kính có tiêu cự $\SI{1,2}{\meter}$, thị kính. Khi ngắm chừng ở vô cực, số bội giác của kính là 30. Khoảng cách giữa vật kính và thị kính là
	\begin{mcq}(4)
		\item $\SI{120}{\centi\meter}$.
		\item $\SI{4}{\centi\meter}$.
		\item $\SI{124}{\centi\meter}$.
		\item $\SI{5,2}{\centi\meter}$. 
	\end{mcq}
	\item %câu 7
	Một người mắt bình thường khi quan sát vật ở xa bằng kính thiên văn, trong trường hợp ngắm chừng ở vô cực thấy khoảng cách giữa vật kính và thị kính là $\SI{62}{\centi\meter}$, số bội giác là 30. Tiêu cự của vật kính và thị kính lần lượt là
	\begin{mcq}(4)
		\item $\SI{2}{\centi\meter}$ và $\SI{60}{\centi\meter}$.
		\item $\SI{2}{\meter}$ và $\SI{60}{\meter}$
		\item $\SI{60}{\centi\meter}$ và $\SI{2}{\centi\meter}$
		\item $\SI{60}{\meter}$ và $\SI{2}{\meter}$
	\end{mcq}
	\item %câu 8
	Công thức số bội giác của kính thiên văn khúc xạ trong trường hợp ngắm chừng ở vô cực $G_\infty$ là
	\begin{mcq}(2)
		\item $G_\infty=\dfrac{f_1}{f_2}$.
		\item $G_\infty=f_1\cdot f_2$.
		\item $G_\infty=\text{Đ}\cdot\dfrac{f_1}{f_2}$.
		\item $G_\infty=\text{Đ}\cdot(f_1\cdot f_2)$.
	\end{mcq}
	\item %câu 9
	Kính thiên văn khúc xạ tiêu cự vật kính $f_1$ và tiêu cự thị kính $f_2$. Khoảng cách giữa vật kính và thị kính của kính thiên văn ngắm chừng ở vô cực có biểu thức nào?
	\begin{mcq}(4)
		\item $f_1+f_2$.
		\item $\dfrac{f_1}{f_2}$.
		\item  $\dfrac{f_2}{f_1}$.
		\item $f_1-f_2$.
	\end{mcq}
	\item %câu 10
	Một kính thiên văn gồm vật kính có tiêu cự $\SI{100}{\centi\meter}$ và thị kính có tiêu cự $\SI{14}{\centi\meter}$. Số bội giác của kính khi người mắt tốt quan sát Mặt Trăng trong trạng thái không điều tiết là
	\begin{mcq}(4)
		\item 20.
		\item 24.
		\item 25.
		\item 30.
	\end{mcq}
\end{enumerate}

\textbf{ĐÁP ÁN}
\begin{longtable}[\textwidth]{|p{0.1\textwidth}|p{0.1\textwidth}|p{0.1\textwidth}|p{0.1\textwidth}|p{0.1\textwidth}|}
	% --- first head
	\hline%\hspace{2 pt}
	\multicolumn{1}{|c}{\textbf{Câu 1}} &
	\multicolumn{1}{|c|}{\textbf{Câu 2}} &
	\multicolumn{1}{c|}{\textbf{Câu 3}} &
	\multicolumn{1}{c|}{\textbf{Câu 4}} &
	\multicolumn{1}{c|}{\textbf{Câu 5}} \\
	\hline
	C. & C.  & B. & A. & A.\\
	\hline
	\multicolumn{1}{|c}{\textbf{Câu 6}} &
	\multicolumn{1}{|c|}{\textbf{Câu 7}} &
	\multicolumn{1}{c|}{\textbf{Câu 8}} &
	\multicolumn{1}{c|}{\textbf{Câu 9}} &
	\multicolumn{1}{c|}{\textbf{Câu 10}} \\
	\hline
	C. & C.  & A. & A. & C.\\
	\hline
\end{longtable}

