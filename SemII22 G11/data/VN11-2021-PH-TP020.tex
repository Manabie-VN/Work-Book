\whiteBGstarBegin
\setcounter{section}{0}

\begin{enumerate}[label=\bfseries Câu \arabic*:]
	
	\item \mkstar{1} [34]
	
	\cauhoi{
		Nêu các đặc điểm của vectơ lực từ $\vec F$ tác dụng lên một phần tử dòng điện $I$ đặt trong từ trường đều có cảm ứng từ.
	}
	
	\loigiai{		
		Lực từ $\vec F$ có điểm đặt tại trung điểm của $M_1M_2$, có phương vuông góc với $\vec l$ và $\vec B$, có chiều tuân theo quy tắc bàn tay trái và có độ lớn: 
		
		$$F=IlB\sin \alpha.$$
		
		trong đó $\alpha$ là góc tạo bởi $\vec B$ và $\vec l$.
	}
	
	
	\item \mkstar{2} [34]
	
	\cauhoi{
	Một đoạn dây dẫn dài $\SI{10}{cm}$ đặt trong từ trường đều và vuông góc với vectơ cảm ứng từ. Dòng điện chạy qua dây có cường độ $\SI{1}{A}$. Lực từ tác dụng lên đoạn dây đó là $6\cdot 10^{-2}\,\ \text N$. Tính độ lớn cảm ứng từ của từ trường.
	
	}
	
	\loigiai{
		Độ lớn cảm ứng từ của từ trường 
		
		$$F = BIl\sin \alpha \Rightarrow B =\dfrac{F}{Il\sin \alpha} = \SI{0,6}{T}.$$
	}
	
	\item \mkstar{2} [23]
	
	\cauhoi{
		Một đoạn dây dẫn dài $\SI{0,5}{m}$ mang dòng điện có cường độ $\SI{10}{A}$ đặt trong một từ trường đều, cảm ứng từ có độ lớn $\SI{1,2}{T}$. Biết dòng điện chạy trong dây dẫn hợp với $\vec B$ một góc $60^\circ$.  Xác định lực từ tác dụng lên đoạn dây dẫn trên.
		
	}
	
	\loigiai{
		Lực từ tác dụng lên dây dẫn  
		
		$$F = BIl\sin \alpha = 3\sqrt 3\ \text{N}.$$
		
	}
\end{enumerate}

\whiteBGstarEnd