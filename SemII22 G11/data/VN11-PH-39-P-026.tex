%\setcounter{chapter}{1}
\chapter{Luyện tập: Giải bài toán về hệ thấu kính}
\begin{enumerate}
	
	\item Hai thấu kính có tiêu cự lần lượt là $f_1, \ f_2$ ghép đồng trục sát nhau thì ta có tiêu cự của hệ thấu kính là:
	\begin{mcq}(2)
		\item $f=f_1+f_2$.
		\item $\dfrac{1}{f}=\dfrac{1}{f_1}+\dfrac{1}{f_2}$.
		\item $f^2=f^2_1+f^2_2$.
		\item $f=\left| f_1-f_2\right| $.
	\end{mcq}
	
	
	\item Vật sáng AB đặt trước hệ hai thấu kính phân kỳ ghép đồng trục với nhau. Ảnh qua hệ thấu kính
	\begin{mcq}(2)
		\item là ảnh ảo.
		\item là ảnh thật.
		\item có thể ảnh ảo, có thể ảnh thật
		\item Chưa đủ cơ sở để kết luận.
	\end{mcq}
	
	\item Khoảng cách $l$ giữa hai thấu kính hội tụ đồng trục nhận giá trị nào sau đây để khi tia tới quang hệ song song với trục chính thì tia ló ra khỏi quang hệ song song với tia tới
	\begin{mcq}(4)
		\item $f_1+2f_2$.
		\item $f_2+2f_1$.
		\item $f_1+f_2$.
		\item $\left|f_1-f_2 \right| $.
	\end{mcq}
	
	\textbf{Sử dụng dữ kiện sau để trả lời cho câu 4 và câu 5.}
	
	Cho hai thấu kính có độ tụ $D_1=6\ \text{dp}, \ D_2=-1\ \text{dp}$ được ghép sát nhau và đồng trục. Vật thật AB đặt trước hệ thấu kính cho ảnh ảo cao gấp 4 lần vật.
	
	\item Tiêu cự của hệ thấu kính là
	\begin{mcq}(4)
		\item $\text{1}\ \text{cm}$.
		\item $\text{16,7}\ \text{cm}$.
		\item $\text{20}\ \text{cm}$.
		\item $-\text{20}\ \text{cm}$.
	\end{mcq}
	
	\item Vị trí của vật AB cách hệ thấu kính một đoạn là
	\begin{mcq}(4)
		\item $\text{10}\ \text{cm}$.
		\item $\text{15}\ \text{cm}$.	
		\item $\text{12}\ \text{cm}$.
		\item $\text{20}\ \text{cm}$.
	\end{mcq}
	
	\item Một vật AB đặt trước thấu kính $L_1$ một khoảng $d_1 = 60\ \text{cm}$ cho ảnh $\text{A}_1\text{B}_1$. Đặt sau $L_1$ một thấu kính $L_2$ đồng trục và sát vào $L_1$ thì ảnh $\text{A}_2\text{B}_2$ của AB tạo bởi hệ ghép có cùng độ lớn với $\text{A}_1\text{B}_1$. Xác định tiêu cự của hai thấu kính trên; biết rằng chúng có giá trị bằng nhau.
	\begin{mcq}(2)
		\item $f_1=f_2=-90\ \text{cm}$.
		\item $f_1=f_2=90\ \text{cm}$.
		\item $f_1=-90\ \text{cm}, \ f_1=90\ \text{cm}$.	
		\item $f_1=90\ \text{cm}, \ f_1=-90\ \text{cm}$.
	\end{mcq}
	
	\item Hai thấu kính $L_1$, $L_2$ có tiêu cự lần lượt là $f_1=20\ \text{cm}$ và $f_2=10\ \text{cm}$ đặt cách nhau một khoảng $l=55\ \text{cm}$, sao cho trục chính trùng nhau. Để qua hệ thu được 1 ảnh thật có chiều cao bằng 2 cm và cùng chiều với vật AB thì phải đặt vật AB cách thấu kính $L_1$ đoạn bằng bao nhiêu.
	\begin{mcq}(4)
		\item 40 cm.
		\item 60 cm.
		\item 20 cm.
		\item 80 cm.
	\end{mcq}
	
	\item Hai thấu kính hội tụ có các tiêu cự lần lượt là $f_1=10\ \text{cm}$ và $f_2=20\ \text{cm}$ đuợc đặt đồng trục và cách nhau $l=30\ \text{cm}$. Vật sáng AB được đặt vuông góc với trục chính trước $(L_1)$ cách quang tâm $O_1$ một đoạn 12 cm. Xác định ảnh của vật cho bởi hệ.	
	\begin{mcq} 
		\item Ảnh cuối cùng là ảnh thật cách $(O_2)$ 12 cm và cao gấp đôi vật.
		\item Ảnh cuối cùng là ảnh ảo cách $(O_2)$ 12 cm và cao gấp đôi vật.
		\item Ảnh cuối cùng là ảnh thật cách $(O_2)$ 24 cm và cao gấp đôi vật.
		\item Ảnh cuối cùng là ảnh ảo cách $(O_2)$ 24 cm và cao gấp đôi vật.
	\end{mcq}
	
	\item Đặt một vật sáng AB vuông góc với trục chính của thấu kính hội tụ $L_1$ có tiêu cự $f_1=32\ \text{cm}$ và cách thấu kính 40 cm. Sau $L_1$ ta đặt một thấu kính $L_2$ có tiêu cự $f_2 =-15\ \text{cm}$, đồng trục với $L_1$ và cách $L_2$ một đoạn $a$. Tìm $a$ để độ lớn của ảnh cuối cùng của AB không phụ thuộc khoảng cách từ vật AB tới hệ.
	\begin{mcq}(4)
		\item 32 cm.
		\item 15 cm.
		\item 47 cm.
		\item 17 cm.
	\end{mcq}
	
	\item Một hệ đồng trục gồm một thấu kính phân kỳ $O_1$ có tiêu cự $f_1 = -18\ \text{cm}$ và 1 thấu kính hội tụ $O_2$ có tiêu cự $f_2 =24\ \text{cm}$ đặt cách nhau một khoảng $a$. Vật sáng AB đặt vuông góc trục chính cách $O_1$ đoạn 18 cm.  Xác định $a$ để hệ cho ảnh cao gấp 3 lần vật.
	\begin{mcq}(2)
		\item 11 cm.
		\item 19 cm
		\item 30 cm.
		\item Cả A và B đều đúng.
	\end{mcq}
\end{enumerate}
\begin{center}
	
	\textbf{ĐÁP ÁN}
	\begin{longtable}[\textwidth]{|p{0.1\textwidth}|p{0.1\textwidth}|p{0.1\textwidth}|p{0.1\textwidth}|p{0.1\textwidth}|p{0.1\textwidth}|p{0.1\textwidth}|p{0.1\textwidth}|}
		% --- first head
		\hline%\hspace{2 pt}
		\multicolumn{1}{|c|}{\textbf{Câu 1}} & \multicolumn{1}{c|}{\textbf{Câu 2}} & \multicolumn{1}{c|}{\textbf{Câu 3}} &
		\multicolumn{1}{c|}{\textbf{Câu 4}} &
		\multicolumn{1}{c|}{\textbf{Câu 5}} &
		\multicolumn{1}{c|}{\textbf{Câu 6}} &
		\multicolumn{1}{c|}{\textbf{Câu 7}} &
		\multicolumn{1}{c|}{\textbf{Câu 8}}\\
		\hline
		B.&A. &C. &C. &B. &B. &A. &A.	\\
		\hline
		
		\multicolumn{1}{|c|}{\textbf{Câu 9}} & \multicolumn{1}{c|}{\textbf{Câu 10}} & \multicolumn{1}{c|}{\textbf{}} &
		\multicolumn{1}{c|}{\textbf{}} &
		\multicolumn{1}{c|}{\textbf{}} &
		\multicolumn{1}{c|}{\textbf{}} &
		\multicolumn{1}{c|}{\textbf{}} &
		\multicolumn{1}{c|}{} \\
		\hline
		B. &D. & & & & & &\\
		\hline	
		
		
	\end{longtable}
	
\end{center}











