%\setcounter{chapter}{1}
\chapter{Luyện tập: Thấu kính mỏng}
\section{Bài toán xác định vị trí, tính chất, độ phóng đại của ảnh và chiều cao ảnh}

\begin{enumerate}
	\item  Vật sáng nhỏ AB đặt vụông góc trục chính của một thấu kính và cách thấu kính 
	15 cm cho ảnh ảo lớn hơn vật hai lần. Tiêu cự của thấu kính là
	\begin{mcq}(4)
		\item 18 cm.
		\item 24 cm.
		\item 36 cm.
		\item 30 cm.
	\end{mcq}
	
	\item Một thấu kính hội tụ có tiêu cự 30 cm. Vật sáng AB đặt vuông góc với trục chính của thấu kính. Ảnh của vật tạo hởi thấu kính ngược chiều với vật và cao gấp ba lần vật. Vật AB cách thấu kính 
	\begin{mcq}(4)
		\item 15 cm.
		\item 20 cm.
		\item 30 cm.
		\item 40 cm.
	\end{mcq}
	
	\item Vật sáng AB vuông góc với trục chính của thấu kính hội tụ cho ảnh cách vật 20 cm. Xác định vị trí vật và ảnh. Cho tiêu cự của thấu kính là $f = 15\ \text{cm}$.
	\begin{mcq}(2)
		\item $d=10\ \text{cm}, \ d'=-\text{30}\ \text{cm}$.
		\item $d=10\ \text{cm}, \ d'=-\text{10}\ \text{cm}$.
		\item $d=10\ \text{cm}, \ d'=-\text{20}\ \text{cm}$.
		\item $d=10\ \text{cm}, \ d'=\text{40}\ \text{cm}$.
	\end{mcq}
	
	\item Vật AB cao 2 cm đặt vuông góc với trục chính của thấu kính hội tụ cho ảnh A'B' cao 4cm. Tiêu cự thấu kính là $f=20\ \text{cm}$. Xác định vị trí của vật và ảnh.
	\begin{mcq}(2)
		\item $d=15\ \text{cm}, \ d'=-\text{30}\ \text{cm}$.
		\item $d=10\ \text{cm}, \ d'=-\text{20}\ \text{cm}$.
		\item $d=5\ \text{cm}, \ d'=-\text{10}\ \text{cm}$.
		\item $d=20\ \text{cm}, \ d'=-\text{40}\ \text{cm}$.
	\end{mcq}
	
	\item Đặt một thấu kính cách một trang sách 20 cm, nhìn qua thấu kính thấy ảnh của dòng chữ cùng chiều với dòng chữ nhưng cao bằng một nửa dòng chữ thật. Tìm tiêu cự của thấu kính, suy ra thấu kính loại gì?
	
	\begin{mcq}(2)
		\item $f=-20\ \text{cm}$, thấu kính phân kỳ.
		\item $f=-10\ \text{cm}$, thấu kính phân kỳ.
		\item $f=20\ \text{cm}$, thấu kính hội tụ.
		\item $f=10\ \text{cm}$, thấu kính hội tụ.
	\end{mcq}
	
	\item Một thấu kính hội tụ có tiêu cự 30 cm. Vật sáng AB đặt vuông góc với trục chính của thấu kính. Anh của vật tạo bởi thấu kính cùng chiều với vật và cao gấp hai lần vật. Vật AB cách thấu kính 
	\begin{mcq}(4)
		\item $10\ \text{cm}$.
		\item $45\ \text{cm}$.
		\item $15\ \text{cm}$.
		\item $90\ \text{cm}$.
	\end{mcq}
	
	
\end{enumerate}

\section{Khoảng cách giữa vật và ảnh}
\begin{enumerate}
	\item Một thấu kính phân kì có độ tụ $-5\ \text{dp}$. Nếu vật sáng AB đặt vuông góc vói trục chính và cách thấu kính 30 cm thỉ ảnh cách vật một khoảng là $L$ với số phóng đại ảnh là $k$. Chọn phương án đúng.
	\begin{mcq}(4)
		\item $L = 20\ \text{cm}.$
		\item $k=-\text{0,4}.$
		\item $L = 40\ \text{cm}.$
		\item $k=\text{0,4}.$
	\end{mcq}
	
	\item Đặt vật sáng nhỏ AB vuông góc trục chính cua thấu kính có tiêu cự 16 cm, cho ảnh cao bằng nửa vật Khoảng cách giữa vật vả ảnh là
	\begin{mcq}(4)
		\item $72\ \text{cm}.$
		\item $80\ \text{cm}.$
		\item $30\ \text{cm}.$
		\item $90\ \text{cm}.$
	\end{mcq}
	
	\item Vật AB là đoạn thẳng sáng nhỏ đặt vuông góc với trục chính của một thấu kính cho ảnh ảo cao bằng 5 lần vật và cách vật 60 cm. Đầu A của vật nằm tại trục chính của thấu kính. Tiêu cực của thấu kính gần giá trị nào nhất sau đây?
	\begin{mcq} (4)
		\item $32\ \text{cm}.$
		\item $80\ \text{cm}.$
		\item $17\ \text{cm}.$
		\item $21\ \text{cm}.$
	\end{mcq}
	
	\item  Vật AB là đoạn thẳng sáng nhỏ vuông góc với trục chính của một thấu kính phân kì cho ảnh cao bằng $\text{0,5}$ lần vật và cách vật 60 cm. Đầu A của vật nằm tại trục chính của thấu kính. Tiêu cự của thấu kính gần giá trị nào nhất sau đây?
	\begin{mcq}(4)
		\item $-72\ \text{cm}.$
		\item $-80\ \text{cm}.$
		\item $-130\ \text{cm}.$ 
		\item $-90\ \text{cm}.$
	\end{mcq}
	
	\item  Một thấu kính hội tụ có tiêu cự $f = 20\ \text{cm}$. Vật sáng AB được đặt trước thấu kính và có ảnh A'B'. Cho biết khoảng cách vật và ảnh là 125 cm. Khoảng cách từ vật đến thấu kính là:
	\begin{mcq}
		\item 25 cm hoặc 100 cm.
		\item 20 cm hoặc 105 cm.
		\item 40 cm hoặc 85 cm hoặc 100 cm
		\item 25 cm hoặc 100 cm hoặc $\text{17,5}$ cm.
	\end{mcq}
	
	\item Một vật sáng 4 mm đặt thẳng góc với trục chính của một thấu kính hội tụ (có tiêu cự 40 cm), cho ảnh cách vật 36 cm. Xác định tính chất, độ lớn của ảnh và vị trí của vật.
	\begin{mcq}
		\item Ảnh thật, cao 10 mm, cách thấu kính 24 cm.
		\item Ảnh ảo, cao 10 mm, vật cách thấu kính 24 cm.	
		\item Ảnh thật, cao 5 mm, vật cách thấu kính 12 cm.
		\item Ảnh ảo, cao 5 mm, vật cách thấu kính 12 cm.
	\end{mcq}
	
	
\end{enumerate}
\section{Bài toán dời vật hoặc thấu kính theo phương trục chính}

\begin{enumerate}
	\item  Một thấu kính hội tụ tiêu cự $f$. Đặt thấu kính này giữa vật AB và màn (song song với vật) sao cho ảnh cảu AB hiện rõ nét trên màn và gấp hai lần vật. Để ảnh rõ nét của vật trên màn gấp ba lần vật, phải tăng khoảng cách vật và màn thêm 10 cm.  Tiêu cực của thấu kính bằng?
	\begin{mcq}(4)
		\item $12\ \text{cm}.$
		\item $20\ \text{cm}.$
		\item $17\ \text{cm}.$
		\item $15\ \text{cm}.$
	\end{mcq}
	
	\item Vật sáng nhỏ AB đặt vuông góc trục chính của thấu kính. Khi vật cách thấu kính 30 cm thì cho ảnh thật $\text{A}_1\text{B}_1$. Đưa vật đến vị trí khác thì cho ảnh ảo $\text{A}_2\text{B}_2$ cách thấu kính 20 cm. Nếu hai ảnh $\text{A}_1\text{B}_1$ và $\text{A}_2\text{B}_2$ có cùng độ lớn thì tiêu cự của thấu kính bằng 
	\begin{mcq}(4)
		\item $18\ \text{cm}.$
		\item $15\ \text{cm}.$
		\item $20\ \text{cm}.$
		\item $30\ \text{cm}.$
	\end{mcq}
	
	\item Một vật sáng phẳng đặt trước một thấu kính, vuông góc với trục chính. Ảnh của vật tạo bởi thấu kính bằng ba lần vật. Dời vật lại gần thấu kính một đoạn 12 cm. Ảnh của vật ở vị trí mới vẫn bằng ba lần vật. Tiêu cự của thấu kính gần giá trị nào nhất sau đây? 
	\begin{mcq} (4)
		\item $10\ \text{cm}.$
		\item $20\ \text{cm}.$
		\item $30\ \text{cm}.$
		\item $40\ \text{cm}.$
	\end{mcq}
	
	\item Một vật thật AB đặt vuông góc với trục chính của một thấu kính. Ban đầu ảnh của vật qua thấu kính là ảnh ảo và bằng nửa vật. Giữ thấu kính cố định di chuyển vật dọc trục chính 100 cm. Ảnh của vật vẫn là ảnh ảo và cao bằng $\dfrac{1}{3}$ vật. Xác định chiều dời của vật, vị trí ban đầu của vật và tiêu cự của thấu kính?
	\begin{mcq} 
		\item Vật ra xa thấu kính, vị trí ban đầu cách thấu kính 100 cm, tiêu cự $f=-100\ \text{cm}.$ 
		\item Vật lại gần thấu kính, vị trí ban đầu cách thấu kính 100 cm, tiêu cự $f=-100\ \text{cm}.$ 
		\item Vật ra xa thấu kính, vị trí ban đầu cách thấu kính 50 cm, tiêu cự $f=-50\ \text{cm}.$ 
		\item Vật lại gần thấu kính, vị trí ban đầu cách thấu kính 50 cm, tiêu cự $f=-50\ \text{cm}.$ 
	\end{mcq}
\end{enumerate}

\begin{center}
	\textbf{ĐÁP ÁN}
	
\end{center}

\textbf{1. Bài toán xác định vị trí, tính chất, độ phóng đại của ảnh và chiều cao ảnh}
\begin{longtable}[\textwidth]{|p{0.1\textwidth}|p{0.1\textwidth}|p{0.1\textwidth}|p{0.1\textwidth}|p{0.1\textwidth}|p{0.1\textwidth}|p{0.1\textwidth}|p{0.1\textwidth}|}
	% --- first head
	\hline%\hspace{2 pt}
	\multicolumn{1}{|c|}{\textbf{Câu 1}} & \multicolumn{1}{c|}{\textbf{Câu 2}} & \multicolumn{1}{c|}{\textbf{Câu 3}} &
	\multicolumn{1}{c|}{\textbf{Câu 4}} &
	\multicolumn{1}{c|}{\textbf{Câu 5}} &
	\multicolumn{1}{c|}{\textbf{Câu 6}} &
	\multicolumn{1}{c|}{\textbf{}} &
	\multicolumn{1}{c|}{\textbf{}}\\
	\hline
	D.&D. &A. &B. &A. &C. & &	\\
	\hline
	
	
	
\end{longtable}

\textbf{2. Khoảng cách giữa vật và ảnh}

\begin{longtable}[\textwidth]{|p{0.1\textwidth}|p{0.1\textwidth}|p{0.1\textwidth}|p{0.1\textwidth}|p{0.1\textwidth}|p{0.1\textwidth}|p{0.1\textwidth}|p{0.1\textwidth}|}
	% --- first head
	\hline%\hspace{2 pt}
	\multicolumn{1}{|c|}{\textbf{Câu 1}} & \multicolumn{1}{c|}{\textbf{Câu 2}} & \multicolumn{1}{c|}{\textbf{Câu 3}} &
	\multicolumn{1}{c|}{\textbf{Câu 4}} &
	\multicolumn{1}{c|}{\textbf{Câu 5}} &
	\multicolumn{1}{c|}{\textbf{Câu 6}} &
	\multicolumn{1}{c|}{\textbf{}} &
	\multicolumn{1}{c|}{\textbf{}}\\
	\hline
	D.&A. &C. &C. &D. &B. & &	\\
	\hline
	
	
	
\end{longtable}


\textbf{3. Bài toán dời vật hoặc thấu kính theo phương trục chính}

\begin{longtable}[\textwidth]{|p{0.1\textwidth}|p{0.1\textwidth}|p{0.1\textwidth}|p{0.1\textwidth}|p{0.1\textwidth}|p{0.1\textwidth}|p{0.1\textwidth}|p{0.1\textwidth}|}
	% --- first head
	\hline%\hspace{2 pt}
	\multicolumn{1}{|c|}{\textbf{Câu 1}} & \multicolumn{1}{c|}{\textbf{Câu 2}} & \multicolumn{1}{c|}{\textbf{Câu 3}} &
	\multicolumn{1}{c|}{\textbf{Câu 4}} &
	\multicolumn{1}{c|}{\textbf{}} &
	\multicolumn{1}{c|}{\textbf{}} &
	\multicolumn{1}{c|}{\textbf{}} &
	\multicolumn{1}{c|}{\textbf{}}\\
	\hline
	A.&C. &B. &A. & & & &	\\
	\hline
	
	
	
\end{longtable}










