\whiteBGstarBegin
\setcounter{section}{0}
\section{Lý thuyết: Lực Lo-ren-xơ }
\begin{enumerate}[label=\bfseries Câu \arabic*:]
	\item \mkstar{2} 
	
	\cauhoi{
		Cho electron bay vào miền có từ trường đều với vận tốc $v = 8\cdot 10^5\ \text{m/s}$ theo phương vuông góc với vectơ cảm ứng từ, độ lớn cảm ứng từ là $B = \text{9,1}\cdot 10^{-4}\ \text{T}$. Tính độ lớn lực Lo-ren-xơ tác dụng lên electron.
		\begin{mcq}(2)
			\item $\text{1,1648}\cdot 10^{-16}\ \text{N}$. 
			\item $\text{1,1648}\cdot 10^{-17}\ \text{N}$.
			\item $\text{1,1648}\cdot 10^{-19}\ \text{N}$.
			\item $\text{1,1648}\cdot 10^{-20}\ \text{N}$.
		\end{mcq}
	}
	
	\loigiai{
		\textbf{Đáp án: A.}
		
		Góc hợp bởi $\vec B$ và $\vec v$ là $90^\circ$ nên ta có độ lớn của lực Lorenxo: 
		
		$$f = |e|vB \sin \alpha = 1,1648 \cdot 10^{-16}\ \text N.$$
		
	}
	\item \mkstar{2} 
	
	\cauhoi{
		Một hạt mang điện $\text{3,2}\cdot 10^{-19}\ \text{C}$ bay vào trong từ trường đều có $B = \text{0,5}\ \text{T}$ hợp với hướng của đường sức từ $30^\circ$. Lực Lorenxơ tác dụng lên hạt có độ lớn $\text{8}\cdot 10^{-14}\ \text{N}$. Vận tốc của hạt đó khi bắt đầu vào trong từ trường là bao nhiêu?
		\begin{mcq}(2)
			\item $v = 2\cdot 10^6\ \text{m/s}$.
			\item $v = 1\cdot 10^6\ \text{m/s}$.
			\item $v = 3\cdot 10^6\ \text{m/s}$.
			\item $v = 5\cdot 10^6\ \text{m/s}$.
		\end{mcq}
		
	}
	
	\loigiai{
		\textbf{Đáp án: B.}
		
		Vận tốc của hạt đó khi bắt đầu vào trong từ trường 
		
		$$f =qvB \sin  \alpha \Rightarrow v = \dfrac{F}{|q| B \sin \alpha} = 10^6\ \text{m/s}.$$
	}
	
		\item \mkstar{3} 
	
	\cauhoi{
		Một hạt điện tích chuyên động trong từ trường đều quĩ đạo của hạt vuông góc với đường sức từ. Nếu hạt chuyển động với vận tốc $v_1=\text{1,8}\cdot 10^{6}\ \text{m/s}$ thì lực Lo-ren-xơ tác dụng lên hạt có độ lớn là $f_1=\text{2}\cdot 10^{-6}\ \text{N}$, nếu hạt chuyển động với vận tốc là $v_1=\text{4,5}\cdot 10^{7}\ \text{m/s}$ thì lực Loren tác dụng lên hạt có giá trị là
		\begin{mcq}(2)
			\item $f_2=\text{2}\cdot 10^{-5}\ \text{N}$. 
			\item $f_2=\text{3}\cdot 10^{-5}\ \text{N}$.
			\item $f_2=\text{5}\cdot 10^{-5}\ \text{N}$.
			\item $f_2=\text{6}\cdot 10^{-5}\ \text{N}$.
		\end{mcq}
	}
	
	\loigiai{
		\textbf{Đáp án: C.}
		
		Một hạt tích điện chuyển động trong từ trường đều, lực Lorenxo tác dụng lên hạt 
		
		$$f_1 = qvB_1.$$ 
		
		$$f_2 = qvB_2.$$
		
		Ta có: 
		
		$$\dfrac{f_2}{f_1} = \dfrac{v_2}{v_1} \Leftrightarrow \dfrac{f_2}{2 \cdot 10^{-6}} = \dfrac{4,5 \cdot 10^7}{1,8 \cdot 10^6} \Rightarrow f_2 = 5 \cdot 10^{-5}\ \text{N}.$$
			
	}
	
	
\end{enumerate}
\section{Lý thuyết: Chuyển động của hạt mang điện trong từ trường đều (đọc thêm)}
\begin{enumerate}[label=\bfseries Câu \arabic*:]
		\item \mkstar{2}
	
	\cauhoi{
		Một ion bay theo quỹ đạo hòn bán kính $R$ trong một mặt phẳng vuông góc với các đường sức của một từ trường đều. Khi độ lớn vận tốc tăng gấp đôi thì bán kính quỹ đạo là 
		\begin{mcq}(4)
			\item $\dfrac{R}{2}.$ 
			\item $\dfrac{R}{4}.$
			\item $2R.$
			\item $4R.$
		\end{mcq}
		
	}
	
	\loigiai{
		\textbf{Đáp án: C.}
		
		Bán kính quỹ đạo của ion chuyển động trong từ trường khi có vận tốc vuông góc với đường sức từ là: 
		
		$$R = \dfrac{mv}{|q|B}.$$
		
		Vậy khi vận tốc tăng lên gấp đôi thì bán kính cũng tăng lên 2 lần.
	}
	
	\item \mkstar{3}
	\cauhoi{
		Hạt electron với vận tốc đầu bằng không được gia tốc qua một hiệu điện thế $\SI{400}{V}$. Tiếp đó nó được dẫn vào miền có từ trường đều vuông góc hướng chuyển động. Quỹ đạo của electron là đường tròn bán kính $R = \SI{7}{cm}$. Xác định độ lớn cảm ứng từ $B$.
		\begin{mcq}(2)
			\item $\text{9,636}\cdot 10^{-4}\ \text{T}$.
			\item $\text{4,818}\cdot 10^{-4}\ \text{T}$.
			\item $\text{3,212}\cdot 10^{-4}\ \text{T}$.
			\item $\text{6,242}\cdot 10^{-4}\ \text{T}$.
		\end{mcq}
	}
	
	\loigiai{
		\textbf{Đáp án: A.}
		
		Độ biến thiên động năng bằng công ngoại lực.
		
		Vận tốc của electron thu được khi tăng tốc bằng hiệu điện thế $U$ là
		
		$$\dfrac{1}{2} m_e v^2 =|e|U \Rightarrow v = 1,186 \cdot 10^7\ \text{m/s}.$$
		
		Khi electron đi vào từ trường đều, lực Lorenxo đóng vai trò là lực hướng tâm 
		
		$$|e| vB = \dfrac{m_e v^2}{r} \Rightarrow B = 9,636 \cdot 10^{-4}\ \text{T}.$$
	}


\end{enumerate}

\whiteBGstarEnd