
\chapter{Dạng bài: Xác định vị trí, tính chất, độ lớn của vật và ảnh }
	
\section{Kiến thức cần nhớ}
	
\subsection{Công thức liên hệ vị trí vật và ảnh}
\begin{equation}\label{eq:A004-2-1}
\dfrac{1}{f}=\dfrac{1}{d}+\dfrac{1}{d'},
\end{equation}

trong đó:
\begin{itemize}
	\item $f$ là tiêu cự của thấu kính;
	\item $d$ là khoảng cách từ vật đến thấu kính;
	\item $d'$ là khoảng cách từ ảnh đến thấu kính. 
\end{itemize}
\subsection{Công thức xác định số phóng đại ảnh}
\begin{equation}\label{eq:A004-2-2}
k=-\dfrac{d'}{d}=\dfrac{f}{f-d}=\dfrac{f-d'}{f},
\end{equation}
trong đó:
\begin{itemize}
	\item $k$ là số phóng đại của ảnh;
	\item $f$ là tiêu cự của thấu kính;
	\item $d$ là khoảng cách từ vật đến thấu kính;
	\item $d'$ là khoảng cách từ ảnh đến thấu kính. 
\end{itemize}

\subsection{Công thức khoảng cách giữa vật và ảnh}
\begin{equation}\label{eq:A004-2-3}
L=\left|d+d' \right| ,
\end{equation}
trong đó:
\begin{itemize}
	\item $L$ là khoảng cách giữa vật và ảnh;
	\item $d$ là khoảng cách từ vật đến thấu kính;
	\item $d'$ là khoảng cách từ ảnh đến thấu kính. 
\end{itemize}
	
\section{Phương pháp giải}
	\begin{description}
		\item[Bước 1:] Xác định dữ kiện đề bài cho và dùng công thức phù hợp.
		
		Nếu biết vị trí của vật hoặc ảnh và số phóng đại ảnh, ta dùng công thức \eqref{eq:A004-2-1} và \eqref{eq:A004-2-2}. 
		
		Nếu biết vị trí của vật hoặc ảnh và khoảng cách giữa vật và màn, ta dùng công thức \eqref{eq:A004-2-1} và \eqref{eq:A004-2-3}.  
		
		\item[Bước 2:] Ta giải hệ phương trình từ các công thức đã xác định ở bước 1, xác định được các giá trị $d,\ d',\ f,\ k,\ L$. Từ đó, xác định vị trí, tính chất, độ lớn của vật và ảnh. 
		
	\end{description}

		\luuy{
			\begin{itemize}
				\item Vật và ảnh cùng tính chất thì trái chiều và ngược lại;
				\item Vật và ảnh không cùng tính chất thì cùng chiều và ngược lại;
				\item Thấu kính hội tụ tạo ảnh ảo lớn hơn vật thật;
				\item Thấu kính phân kỳ tạo ảnh ảo lớn hơn vật thật.
				\end{itemize}
	
	}
\section{Ví dụ minh họa}
\viduii{3}{

	Vật AB cao 2 cm đặt vuông góc với trục chính của thấu kính hội tụ cho ảnh ảo A'B' cao 4 cm. Tiêu cự của thấu kính là $f= 20\ \text{cm}$. Xác định vị trí của vật và ảnh.
	\begin{mcq}(2)
		\item $d=15\ \text{cm};\ d'=-30\ \text{cm}$.
		\item $d=10\ \text{cm};\ d'=-20\ \text{cm}$.
		\item $d=15\ \text{cm};\ d'=30\ \text{cm}$.
		\item $d=15\ \text{cm};\ d'=20\ \text{cm}$.
	\end{mcq}
}{
\begin{center}
	\textbf{Hướng dẫn giải:}
\end{center}
	
	{Ta có: $|k|=\dfrac{\text{A'B'}}{AB}=2$.
		
		Vì ảnh là ảnh ảo nên $d'<0$. Từ đó $k=-\dfrac{d'}{d}>0\Rightarrow k=2$.
		
		Ta có hệ phương trình: 
		
		$\begin{cases} k=-\dfrac{d'}{d}=2, \\ f=\dfrac{dd'}{d+d'}=20\ \text{cm}.\end{cases}$
		
		Từ đó, ta tìm được: $d=10\ \text{cm};\ d'=-20\ \text{cm}$. 
		
		Vậy vật cách thấu kính 10 cm và ảnh ảo cách thấu kính 20 cm. 
		
	\textbf{	Đáp án: B.}
		
	
	}}
\viduii{3}{		Một điểm sáng nằm trên trục chính của một thấu kính phân kỳ(độ lớn tiêu cự bằng 15 cm) cho ảnh cách vật $\text{7,5}\ \text{cm}$. Xác định tính chất, vị trí của vật, vị trí và tính chất của ảnh.
	
	\begin{mcq}
		\item Vật cách thấu kính 15 cm, ảnh ảo, cùng chiều, nhỏ hơn vật, cách thấu kính $\text{7,5}\ \text{cm}$.
		\item Vật cách thấu kính 15 cm, ảnh thật, ngược chiều, nhỏ hơn vật, cách thấu kính $\text{7,5}\ \text{cm}$.
		\item Vật cách thấu kính 15 cm, ảnh ảo, ngược chiều, lớn hơn vật, cách thấu kính $\text{7,5}\ \text{cm}$.
		\item Vật cách thấu kính 15 cm, ảnh ảo, ngược chiều, nhỏ hơn vật, cách thấu kính $\text{7,5}\ \text{cm}$.
	\end{mcq}}
{
\begin{center}
	\textbf{Hướng dẫn giải:}
\end{center}
	
	Vì là thấu kính phân kỳ nên ảnh thu được luôn là ảnh ảo.
		
	Vật thật qua thấu kính phân kỳ cho ảnh ảo nên $d>0,\ d'<0$.
	
	Do ảnh ảo ở gần thấu kính hơn so với vật nên  $d+d'=\text{7,5}\ \text{cm}$.
	
	Ta thay $f= -15\ \text{cm},\ d'=15\ \text{cm} -d$ vào công thức $\dfrac{1}{f}=\dfrac{1}{d}+\dfrac{1}{d'}$. 
	
	Ta tìm được: $d= 15\ \text{cm}$ (nhận); $d= \text{-7,5}\ \text{cm}$ (loại).
	
	Vậy vật là vật thật cách thấu kính 15 cm, ảnh là ảnh ảo, cùng chiều, nhỏ hơn vật và cách thấu kính $\text{7,5}\ \text{cm}$.
		
		\textbf{Đáp án: A.}
		

		
		
	} 

\viduii{3}{
Một vật sáng AB= 4 mm đặt thẳng góc với trục chính của một thấu kính hội tụ (có tiêu cự 40 cm), cho ảnh cách vật 36 cm. Xác định vị trí, tính chất và độ lớn của ảnh.
\begin{mcq}
	\item Ảnh thật, cao 10 mm, cách thấu kính $\text{30}\ \text{cm}$.
	\item Ảnh ảo, cao 10 mm, cách thấu kính $\text{60}\ \text{cm}$.
	\item Ảnh thật, cao 5 mm, cách thấu kính $\text{30}\ \text{cm}$.
	\item Ảnh thật, cao 5 mm, cách thấu kính $\text{60}\ \text{cm}$.
\end{mcq}}
{
\begin{center}
	\textbf{Hướng dẫn giải:}
\end{center}

{Vì là thấu kính hội tụ nên $f>0$ và $f=40 cm$.

Theo giả thiết, khoảng cách giữa vật và ảnh là $L=\left|d+d' \right|= 36 \ \text{cm}$. 

Suy ra: $\begin{cases} d+d'=36\ \text{cm}, \\ d+d'=-36\ \text{cm}\end{cases}$
	
	\textbf{Trường hợp 1:} $d+d'=36\ \text{cm}$ hay $d'=36\ \text{cm}-d$. 
	
	Ta thay $f=40\ \text{cm},\ d'=36\ \text{cm} -d$ vào công thức $\dfrac{1}{f}=\dfrac{1}{d}+\dfrac{1}{d'}$. 
	
	Ta thu được phương trình $d^2-36d+1440=0$, phương trình này vô nghiệm.
	
	\textbf{Trường hợp 2:} $d+d'=-36\ \text{cm}$ hay $d'=-36\ \text{cm}-d$. 
	
	Ta thay $f=40\ \text{cm},\ d'=-36\ \text{cm} -d$ vào công thức $\dfrac{1}{f}=\dfrac{1}{d}+\dfrac{1}{d'}$. 
	
	Ta thu được phương trình $d^2+36d-1440=0$.
	
	Giải phương trình, ta tìm được: $d= 24\ \text{cm}$ (nhận); $d= -\text{60}\ \text{cm}$ (loại).
	
	Ta có $d= 24\ \text{cm}$ nên $d'=-60 \ \text{cm}$.
	
$k=-\dfrac{d'}{d}=\text{2,5}$ nên $\left| k\right|=\dfrac{\text{A'B'}}{\text{AB}}=\text{2,5}$.

Ta tìm được $\text{A'B'}=10\ \text{cm}$.
	
	Vậy ảnh là ảnh ảo, có độ lớn 10 mm và cách thấu kính một đoạn là $60\ \text{cm}$.
	
\textbf{	Đáp án: B.}
} }

\viduii{3}{
Một thấu kính mỏng được đặt sao cho trục chính trùng với trục O$x$ của hệ trục tọa độ vuông góc O$xy$. Điểm sáng A đặt gần trục chính, trước thấu kính. A' là ảnh của A qua thấu kính. Tính tiêu cự của thấu kính.
%\begin{center}
%	\includegraphics[scale=0.7]{../figs/hinh 1.jpg}
%\end{center}

\begin{mcq}(4)
	\item  $\text{75}\ \text{cm}$.
	\item $-\text{75}\ \text{cm}$.
	\item $\text{150}\ \text{cm}$.
	\item $-\text{155}\ \text{cm}$.
\end{mcq}}
{
\begin{center}
	\textbf{Hướng dẫn giải:}
\end{center}

{	Theo hình vẽ, ta có một số dữ kiện sau:
	\begin{itemize}
		\item Ảnh và vật cùng chiều, vật cao 3 ô, ảnh cao 5 ô.
		Độ phóng đại ảnh là:
		$$k=\dfrac{5}{3}(k\textrm{ dương vì ảnh cùng chiều vật}).$$
		Từ giá trị độ phóng đại ảnh, ta suy ra tỉ lệ giữa khoảng cách từ vật đến thấu kính và khoảng cách từ ảnh đến thấu kính là:
		$$k=-\dfrac{d'}{d}=\dfrac{5}{3}\Rightarrow d'=-\dfrac{5}{3}d.$$
		\item Khoảng cách giữa vật và ảnh là $\SI{20}{\centi\meter}$ nên:
		$$\left|d+d'\right|=\SI{20}{\centi\meter}.$$
		Từ đó ta tính được khoảng cách từ vật đến thấu kính và khoảng cách từ ảnh đến thấu kính
		$$d=\SI{30}{\centi\meter} \textrm{ và } d'=\SI{-50}{\centi\meter}.$$
		Tiêu cự của thấu kính là:
		$$\dfrac{1}{f}=\dfrac{1}{d}+\dfrac{1}{d'}\Rightarrow f =\dfrac{d.d'}{d+d'}\Rightarrow f =\dfrac{\SI{30}{\centi\meter}.(\SI{-50}{\centi\meter})}{\SI{30}{\centi\meter}+(\SI{-50}{\centi\meter})}\Rightarrow f =\SI{75}{\centi\meter}.$$
	\end{itemize}
\textbf{	Đáp án: A.}
	
	
	
	
} 
}
