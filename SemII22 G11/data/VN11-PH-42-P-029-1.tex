%\setcounter{chapter}{1}
\chapter{Luyện tập: Kính hiển vi}
\begin{enumerate}
	\item %câu 1
	Vật kính của kính hiển vi tạo ảnh có các tính chất nào?
	\begin{mcq}(1)
		\item Ảnh thật, cùng chiều với vật.
		\item Ảnh ảo, ngược chiều với vật.
		\item Ảnh thật, ngược chiều với vật và lớn hơn vật.
		\item Ảnh ảo, ngược chiều với vật và lớn hơn vật.
	\end{mcq}
	\item %câu 2
	Kính hiển vi gồm vật kính và thị kính là các thấu kính hội tụ như thế nào?
	\begin{mcq}
		\item Vật kính và thị kính có tiêu cự nhỏ cỡ mm, khoảng cách giữa chúng có thể thay đổi được.
		\item Vật kính và thị kính có tiêu cự nhỏ cỡ mm, khoảng cách giữa chúng không đổi.
		\item Vật kính có tiêu cự cỡ mm, thị kính có tiêu cự nhỏ hơn, khoảng cách giữa chúng có thể thay đổi được. 
		\item Vật kính có tiêu cự cỡ mm, thị kính có tiêu cự lớn hơn, khoảng cách giữa chúng không đổi.
	\end{mcq}
	\item %câu 3
	Thị kính của kính hiển vi tạo ảnh có các tính chất nào?
	\begin{mcq}
		\item Ảnh thật, ngược chiều với vật.
		\item Ảnh ảo, ngược chiều với vật.
		\item Ảnh thật, cùng chiều với vật và lớn hơn vật.
		\item Ảnh ảo, cùng chiều với vật và lớn hơn vật.
	\end{mcq}
	\item %câu 4
	Kính hiển vi có tiêu cự vật kính là $\SI{5}{\milli\meter}$; tiêu cự của thị kính là $\SI{2,5}{\centi\meter}$ và độ dài quang học $\SI{17}{\centi\meter}$. Người quan sát có khoảng cực cận $\SI{20}{\centi\meter}$. Số bội giác của kính ngắm chừng ở vô cực có trị số là
	\begin{mcq}(4)
		\item 170.
		\item 272.
		\item 340.
		\item 550.
	\end{mcq}
	\item %câu 5
	Một kính hiển vi có vật kính có tiêu cự $f_1=\SI{1}{\centi\meter}$, thị kính có tiêu cực $f_2=\SI{4}{\centi\meter}$. Khoảng cách giữa vật kính và thị kính là $\SI{17}{\centi\meter}$. Khoảng cách rõ ngắn nhất của mắt là $\text{Đ}=\SI{25}{\centi\meter}$. Độ bội giác của kính hiển vi khi ngắm chừng ở vô cực là
	\begin{mcq}(4)
		\item 60.
		\item 85.
		\item 75.
		\item 80.
	\end{mcq}
	\item %câu 6
	Vật kính và thị kính của một kính hiển vi có tiêu cự $f_1=\SI{0,5}{\centi\meter}$ và $f_2=\SI{4}{\centi\meter}$ có độ dài quang học là $\SI{17}{\centi\meter}$. Người quan sát có khoảng cực cận là $\SI{20}{\centi\meter}$. Độ bội giác của kính khi ngắm chừng ở vô cực là
	\begin{mcq}(4)
		\item 272.
		\item 2,72.
		\item 0,272.
		\item 27,2.
	\end{mcq}
	\item %câu 7
	Một người mắt không có tật có khoảng cực cận $\SI{25}{\centi\meter}$, đặt mắt sát vào thị kính của kính hiển vi để quan sát vật $\text{AB}=\SI{1}{\micro\meter}$, khi ngắm chừng ở vô cực có số bội giác 250. Tính góc trông ảnh của AB qua kính.
	\begin{mcq}(4)
		\item $\SI{0,5}{\milli\meter}$.
		\item $\SI{1,5}{\milli\meter}$.
		\item $\SI{1}{\milli\meter}$.
		\item $\SI{2}{\milli\meter}$.
	\end{mcq}
	\item %câu 8
	Một kính hiển vi gồm vật kính có tiêu cự $\SI{5}{\milli\meter}$ và thị kính có tiêu cự $\SI{20}{\milli\meter}$. Vật AB cách vật kính $\SI{5,2}{\milli\meter}$. Vị trí ảnh của vật cho bởi vật kính là
	\begin{mcq}(4)
		\item $\SI{6,67}{\centi\meter}$.
		\item $\SI{13}{\centi\meter}$.
		\item $\SI{19,67}{\centi\meter}$.
		\item $\SI{25}{\centi\meter}$.
	\end{mcq}
	\item %câu 9
	Một người có mắt tốt có điểm cực cận cách mắt $\SI{25}{\centi\meter}$ quan sát trong trạng thái không điều tiết qua một kính hiển vi mà thị kính có tiêu cự gấp 10 lần thị kính thì thấy độ bội giác của ảnh là 150. Độ dài quang học của kính là $\SI{15}{\centi\meter}$. Tiêu cự của vật kính và thị kính lần lượt là
	\begin{mcq}(4)
		\item $\SI{5}{\centi\meter}$ và $\SI{0,5}{\centi\meter}$.
		\item $\SI{0,5}{\centi\meter}$ và $\SI{5}{\centi\meter}$.
		\item $\SI{0.8}{\centi\meter}$ và $\SI{9}{\centi\meter}$.
		\item $\SI{8}{\centi\meter}$ và $\SI{0,8}{\centi\meter}$.
	\end{mcq}
	\item %câu 10
	Một kính hiển vi vật kính có tiêu cự $\SI{2}{\centi\meter}$, thị kính có tiêu cự $\SI{10}{\centi\meter}$ đặt cách nhau $\SI{15}{\centi\meter}$. Để quan sát ảnh của vật qua kính phải đặt vật trước vật kính
	\begin{mcq}(4)
		\item $\SI{1,88}{\centi\meter}$.
		\item $\SI{1,77}{\centi\meter}$.
		\item $\SI{2,04}{\centi\meter}$.
		\item $\SI{1,99}{\centi\meter}$.
	\end{mcq}
\end{enumerate}

\textbf{ĐÁP ÁN}
\begin{longtable}[\textwidth]{|p{0.1\textwidth}|p{0.1\textwidth}|p{0.1\textwidth}|p{0.1\textwidth}|p{0.1\textwidth}|}
	% --- first head
	\hline%\hspace{2 pt}
	\multicolumn{1}{|c}{\textbf{Câu 1}} &
	\multicolumn{1}{|c|}{\textbf{Câu 2}} &
	\multicolumn{1}{c|}{\textbf{Câu 3}} &
	\multicolumn{1}{c|}{\textbf{Câu 4}} &
	\multicolumn{1}{c|}{\textbf{Câu 5}} \\
	\hline
	C. & D.  & D. & B. & C.\\
	\hline
	\multicolumn{1}{|c}{\textbf{Câu 6}} &
	\multicolumn{1}{|c|}{\textbf{Câu 7}} &
	\multicolumn{1}{c|}{\textbf{Câu 8}} &
	\multicolumn{1}{c|}{\textbf{Câu 9}} &
	\multicolumn{1}{c|}{\textbf{Câu 10}}  \\
	\hline
	A. & C.  & B. & B. & C.\\
	\hline
\end{longtable}



