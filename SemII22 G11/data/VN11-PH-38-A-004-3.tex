\chapter{Dạng bài: Dời vật hoặc thấu kính}
\section{Kiến thức cần nhớ}
	
\subsection{Công thức xác định vị trí ảnh}
\begin{equation}
\dfrac{1}{f}=\dfrac{1}{d}+\dfrac{1}{d'},
\end{equation}

trong đó:
\begin{itemize}
	\item $f$ là tiêu cự của thấu kính;
	\item $d$ là khoảng cách từ vật đến thấu kính;
	\item $d'$ là khoảng cách từ ảnh đến thấu kính. 
\end{itemize}
\subsection{Công thức xác định số phóng đại ảnh}
\begin{equation}
k=-\dfrac{d'}{d}=\dfrac{f}{f-d}=\dfrac{f-d'}{f},
\end{equation}
trong đó:
\begin{itemize}
	\item $k$ là số phóng đại của ảnh;
	\item $f$ là tiêu cự của thấu kính;
	\item $d$ là khoảng cách từ vật đến thấu kính;
	\item $d'$ là khoảng cách từ ảnh đến thấu kính. 
\end{itemize}

\subsection{Công thức khoảng cách giữa vật và ảnh}
\begin{equation}
L=\left|d+d' \right| ,
\end{equation}
trong đó:
\begin{itemize}
	\item $L$ là khoảng cách giữa vật và ảnh;
	\item $d$ là khoảng cách từ vật đến thấu kính;
	\item $d'$ là khoảng cách từ ảnh đến thấu kính. 
\end{itemize}
	
\section{Phương pháp giải}
\subsection{Thấu kính được giữ cố định, vật được dời đi}

		\begin{itemize}
			\item Trước khi vật được dời đi:  
		\begin{equation}
		\dfrac{1}{f}=\dfrac{1}{d_1}+\dfrac{1}{d'_1}.
		\end{equation}	
				
		 Độ phóng đại của ảnh: 
		 \begin{equation}
		 k_1=-\dfrac{d'_1}{d_1}=\dfrac{f}{f-d_1}=\dfrac{f-d'_1}{f}.
		 \end{equation}
		 
		 \item Sau khi vật được dời đi:
		 
		 Gọi $\Delta d, \ \Delta d'$ là độ dịch chuyển của vật và ảnh:
		 \begin{equation}
		  \Delta d=d_2-d_1\, \ \Delta d'=d'_2-d'_1.
		 \end{equation}
		 			 
		 Khi vật được dời một đoạn $\Delta d$ thì ảnh dời một đoạn $\Delta d'$. Khi đó:
		 \begin{equation}  
		  \dfrac{1}{f}=\dfrac{1}{d_2}+\dfrac{1}{d'_2}=\dfrac{1}{d_1+\Delta d}+\dfrac{1}{d'_1+\Delta d'},
		 \end{equation}
		
		 
		hay \begin{equation} \dfrac{1}{f}=\dfrac{1}{d_1}+\dfrac{1}{d'_1}=\dfrac{1}{d_1+\Delta d}+\dfrac{1}{d'_1+\Delta d'}.
		 \end{equation}
		 
		 Độ phóng đại của ảnh: 
		 \begin{equation}
		 k_2=-\dfrac{d'_2}{d_2}=\dfrac{f}{f-d_2}=\dfrac{f-d'_2}{f}.
		 \end{equation}
			\end{itemize}
			\luuy{
				
				\begin{itemize}
				\item Khi dời vật (thấu kính cố định),\textbf{ ảnh và vật luôn  di chuyển cùng chiều. }
				
				\item Nếu bài toán cho độ phóng đại $k_1$ và $k_2$, ta có thể giải như sau: 
			\begin{equation} \dfrac{\Delta d'}{\Delta d}=-k_1\cdot k_2.\end{equation} 
				\end{itemize}
		}
	
\subsection{Vật được giữa cố định, thấu kính được dời đi}
		
		Khi thấu kính được dời đi thì ta phải khảo sát khoảng cách vật - ảnh để xác định chiều chuyển động của ảnh:
	\begin{equation}L_1=\left|d_1+d'_1 \right|, \ L_2=\left|d_2+d'_2 \right|,\end{equation}
		
		với $L_1, \ L_2$ là khoảng cách giữa vật và ảnh lúc đầu và lúc sau.

	
	
\section{Ví dụ minh họa}
\viduii{3}{
	Một vật thật AB đặt vuông góc với trục chính của một thấu kính. Ban đầu ảnh của vật qua thấu kính là ảnh ảo và bằng nửa vật. Giữ thấu kính cố định, di chuyển vật dọc trục chính 100 cm. Ảnh của vật vẫn là ảnh ảo và cao bằng $\dfrac{1}{3}$ vật. Xác định chiều dời của vật, tiêu cự của thấu kính và vị trí ban đầu của vật?
	\begin{mcq}
		\item Vật ra xa thấu kính, $f=-100\ \text{cm}$, vị trí ban đầu của vật $d_1=100\ \text{cm}$.
		\item  Vật lại gần thấu kính, $f=-100\ \text{cm}$, vị trí ban đầu của vật $d_1=100\ \text{cm}$.
		\item  Vật ra xa thấu kính, $f=-50\ \text{cm}$, vị trí ban đầu của vật $d_1=100\ \text{cm}$.
		\item  Vật laị gần thấu kính, $f=-50\ \text{cm}$, vị trí ban đầu của vật $d_1=100\ \text{cm}$.
	\end{mcq}
}{
\begin{center}
	\textbf{Hướng dẫn giải:}
\end{center}
	
	{Vật thật qua thấu kính cho ảnh ảo nhỏ hơn vật nên thấu kính là thấu kính phân kỳ.
		
	Vị trí vật lúc đầu $d_1$ được tính từ độ phóng đại $k_1$:
	
	 $k_1=-\dfrac{d'_1}{d_1}=\dfrac{1}{2}\Rightarrow d_1=-2d'_1=\dfrac{2d_1f}{f-d_1}\Rightarrow d_1=-f.$
	 
	 Vị trí vật lúc sau $d_2$ được tính từ độ phóng đại $k_2$:
	 
	$k_2=-\dfrac{d'_2}{d_2}=\dfrac{1}{3}\Rightarrow d_1=-3d'_1=\dfrac{3d_2f}{f-d_2}\Rightarrow d_2=-2f.$
		
		Vì thấu kính là thấu kính phân kỳ nên $f<0$. 
		
		Do đó $d_2>d_1$, nên vật dịch chuyển ra xa thấu kính.
		
		Suy ra $d_2=d_1+100 \Rightarrow f=-100\ \text{cm}\Rightarrow d_1=-f=100\ \text{cm}$.
		
		Vậy vị trí ban đầu của vật là $d_1=100\ \text{cm}$.
		
	\textbf{	Đáp án: A.}
		}}
\viduii{3}{	Một thấu kính hội tụ có $f=12\ \text{cm}$. Điểm sáng A trên trục chính có ảnh A'. Dời A gần thấu kính thêm 6 cm, A' dời 2 cm (không đổi tính chất). Vị trí vật và ảnh lúc đầu là
		\begin{mcq}
		\item $d_1=9\ \text{cm}, \ d'_1=36\ \text{cm}$.
		\item $d_1=36\ \text{cm}, \ d'_1=9\ \text{cm}$.
		\item $d_1=18\ \text{cm}, \ d'_1=18\ \text{cm}$.
		\item $d_1=36\ \text{cm}, \ d'_1=18\ \text{cm}$.
	\end{mcq}}
	{
\begin{center}
	\textbf{Hướng dẫn giải:}
\end{center}
	
	{Ảnh và vật di chuyển cùng chiều nên $d_2=d_1-6\ (1), \ d'_2=d'_1+2\ (2)$.
	
	Công thức xác định vị trí vật và ảnh: $d'_2=\dfrac{d_2f}{d_2-f}\ (3),\ d'_1=\dfrac{d_1f}{d_1-f}\ (4)$.
	
	Ta thay $(3),\ (4)$ vào $(2)$, ta được phương trình: 
	
	$\dfrac{d_2f}{d_2-f}=\dfrac{d_1f}{d_1-f}+2\ (5)$.
	
	Ta thay $(1)$ vào $(5)$, ta được phương trình:
	
	$\dfrac{(d_1-6)f}{d_1-6-f}=\dfrac{d_1f}{d_1-f}+2 \ (6)$.
	
	Ta thay $f=12\ \text{cm}$ vào $(6)$, ta được phương trình:
	
	$\dfrac{(d_1-6)f}{d_1-6-12}=\dfrac{d_1\cdot 12}{d_1-12}+2 \ (6)$.
	
	Ta tìm được $d_1= 36\ \text{cm}\Rightarrow d'_1=18\ \text{cm}$.
	
	\textbf{	Đáp án: D.}
		} 
}
\viduii{3}{
A, B, C là ba điểm thẳng hàng. Đặt vật ở A, một thấu kính ở B thì ảnh thật hiện ở C với độ phóng đại $|k_1|=3$. Dịch thấu kính ra xa vật một đoạn 64 cm thì ảnh của vật vẫn hiện ở C với độ phóng đại $|k_2|=\dfrac{1}{3}$. Tiêu cự của thấu kính là
\begin{mcq}
	\item $f=\text{12}\ \text{cm}$.
	\item $f=\text{24}\ \text{cm}$.
	\item $f=\text{36}\ \text{cm}$.
	\item $f=\text{38}\ \text{cm}$.
\end{mcq}}
{
\begin{center}
	\textbf{Hướng dẫn giải:}
\end{center}

{Lúc đầu ảnh thật nên $k=-3\Rightarrow d'_1=3d_1\ (1)$.
	
	Khi dịch thấu kính ra xa khỏi A thêm 64 cm thì thấu kính sẽ lại gần ảnh thêm 64 cm (A và C cố định). 
	
	Suy ra: 
	$\begin{cases} d_2=d_1+64 \\ d'_2=d'_1-64\end{cases}$
	
	Vì ảnh lúc sau vẫn ở trên màn nên ảnh đó là ảnh thật. Do đó $k=-\dfrac{1}{3}\Rightarrow d'_2=\dfrac{1}{3}d_2$.
	
	$\Rightarrow \begin{cases} d_2=d_1+64 \\ \dfrac{1}{3}d_2=d'_1-64\end{cases}\Rightarrow \begin{cases} d_2=d_1+64 \\ d_2=3d'_1-3\cdot 64\end{cases}$
	
	$\Rightarrow d_1=3d'_1-4\cdot 64$, thay $(1)$ vào ta được:
	$d_1=32\ \text{cm}, \Rightarrow d'_1=96\ \text{cm}$.
	
	$\Rightarrow f=\dfrac{d_1\cdot d'_1}{d_1+d'_1}=24\ \text{cm}$.
	
\textbf{	Đáp án: B.}
} 

}