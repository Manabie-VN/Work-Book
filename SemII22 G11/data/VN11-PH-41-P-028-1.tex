%\setcounter{chapter}{1}
\chapter{Luyện tập: Kính lúp}
\begin{enumerate}
	\item %câu 1
	Một mắt không có tật có điểm cực cận cách mắt $\SI{20}{\centi\meter}$, quan sát vật AB qua một kính lúp có tiêu cực $\SI{2}{\centi\meter}$. Xác định số bội giác của kính lúp khi ngắm chừng ở vô cực
	\begin{mcq}(4)
		\item 6.
		\item 10.
		\item 15.
		\item 2,5.
	\end{mcq}
	\item %câu 2
	Một kính lúp là một thấu kính hội tụ có độ tụ $\SI{10}{dp}$. Mắt người quan sát có khoảng nhìn rõ ngắn nhất là $\SI{20}{\centi\meter}$. Độ bội giác của kính lúp khi ngắm chừng ở vô cực là 
	\begin{mcq}(4)
		\item 2,5.
		\item 4.
		\item 5.
		\item 2.
	\end{mcq}
	\item %câu 3
	Một học sinh, có mắt không bị tật, có khoảng cực cận $\text{OC}_\text{C}=\SI{25}{\centi\meter}$, dùng kính lúp có độ tụ $+\SI{10}{dp}$ để quan sát một vật nhỏ. Biết ngắm chừng kính lúp ở vô cực. Tính số bội giác. 
	\begin{mcq}(4)
		\item 6.
		\item 4.
		\item 15.
		\item 2,5.
	\end{mcq}
	\item %câu 4
	Một mắt không tật có điểm cực cận cách mắt $\SI{20}{\centi\meter}$, quan sát vật AB qua một kính lúp có tiêu cự $\SI{2}{\centi\meter}$. Xác định số bội giác của kính khi ngắm chừng ở điểm cực cận, khi mắt đặt tại tiêu điểm ảnh của kính.
	\begin{mcq}(4)
		\item 6.
		\item 4.
		\item 10.
		\item 2,5.
	\end{mcq}
	\item %câu 5
	Một người cận thị đặt mắt tại tiêu điểm ảnh của kính lúp có tiêu cực $\SI{2}{\centi\meter}$, quan sát ảnh mà không phải điều tiết mắt. Xác định số bội giác của kính đối với mắt người đó, biết rằng mắt cận có điểm cực cận cách mắt $\SI{10}{\centi\meter}$ và điểm cực viễn cách mắt $\SI{122}{\centi\meter}$.
	\begin{mcq}(4)
		\item 5.
		\item 4.
		\item 10.
		\item 2,5.
	\end{mcq}
	\item %câu 6
	An mua một chiếc kính lúp. An thấy trên vành của kính lúp có ghi $4x$. Tiêu cự của kính lúp này là
	\begin{mcq}(4)
		\item $\SI{4}{\centi\meter}$.
		\item $\SI{0,4}{\centi\meter}$.
		\item $\SI{100}{\centi\meter}$.
		\item $\SI{6,25}{\centi\meter}$.
	\end{mcq}
	\item %câu 7
	Trong các kính lúp sau, kính lúp nào khi dùng để quan sát một vật sẽ cho ảnh lớn nhất?
	\begin{mcq}(2)
		\item Kính lúp có số bội giác $G = 5$.
		\item Kính lúp có số bội giác $G = 1,5$.
		\item Kính lúp có số bội giác $G = 6$.
		\item Kính lúp có số bội giác $G = 4$.
	\end{mcq}
	\item %câu 8
	Thấu kính hội tụ có tiêu cự nào sau đây không thể dùng làm kính lúp?
	\begin{mcq}(4)
		\item $\SI{25}{\centi\meter}$.
		\item $\SI{14}{\centi\meter}$.
		\item $\SI{3}{\centi\meter}$.
		\item $\SI{9}{\centi\meter}$. 
	\end{mcq}
	\item %câu 9
	Dùng kính lúp có số bội giác $G=2,5x$ để quan sát một vật nhỏ cao $\SI{2}{\milli\meter}$. Muốn có ảnh ảo cao $\SI{8}{\milli\meter}$ thì phải đặt vật cách kính bao nhiêu? Lúc đó ảnh cách kính bao?
	\begin{mcq}(4)
		\item $\SI{30}{\centi\meter}$.
		\item $\SI{25}{\centi\meter}$.
		\item $\SI{20}{\centi\meter}$.
		\item $\SI{15}{\centi\meter}$. 
	\end{mcq}
	\item %câu 10
	Trên vành của một chiếc kính lúp có ghi $G = 5x$. Vật nhỏ S có chiều cao là $\SI{0,4}{\centi\meter}$ được đặt trước kính lúp và cách kính lúp $\SI{3}{\centi\meter}$. Ảnh của S qua kính lúp cách S bao nhiêu?
	\begin{mcq}(4)
		\item $\SI{4,5}{\centi\meter}$.
		\item $\SI{2}{\centi\meter}$.
		\item $\SI{1,5}{\centi\meter}$.
		\item $\SI{1}{\centi\meter}$. 
	\end{mcq}	
\end{enumerate}

\textbf{ĐÁP ÁN}
\begin{longtable}[\textwidth]{|p{0.1\textwidth}|p{0.1\textwidth}|p{0.1\textwidth}|p{0.1\textwidth}|p{0.1\textwidth}|}
	% --- first head
	\hline%\hspace{2 pt}
	\multicolumn{1}{|c}{\textbf{Câu 1}} &
	\multicolumn{1}{|c|}{\textbf{Câu 2}} &
	\multicolumn{1}{c|}{\textbf{Câu 3}} &
	\multicolumn{1}{c|}{\textbf{Câu 4}} &
	\multicolumn{1}{c|}{\textbf{Câu 5}} \\
	\hline
	B. & D.  & D. & C. & A.\\
	\hline
	\multicolumn{1}{|c}{\textbf{Câu 6}} &
	\multicolumn{1}{|c|}{\textbf{Câu 7}} &
	\multicolumn{1}{c|}{\textbf{Câu 8}} &
	\multicolumn{1}{c|}{\textbf{Câu 9}} &
	\multicolumn{1}{c|}{\textbf{Câu 10}} \\
	\hline
	D. & C.  & A. & A. & A.\\
	\hline
\end{longtable}










