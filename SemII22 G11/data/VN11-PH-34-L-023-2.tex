\chapter[Chiết suất của các môi trường - Tính thuận nghịch của sự truyền ánh sáng]{Chiết suất của các môi trường \\Tính thuận nghịch của sự truyền ánh sáng}
\section{Lý thuyết trọng tâm}
\subsection{Chiết suất của môi trường}
\subsubsection{Chiết suất tỉ đối}
Tỉ số không đổi $\dfrac{\sin i}{\sin r}$ trong hiện tượng khúc xạ được gọi là chiết suất tỉ đối $n_{21}$ của môi trường (2) (chứa tia khúc xạ) đối với môi trường (1) (chứa tia tới):  
\begin{equation}
\dfrac{\sin i}{\sin r}=n_{21}.
\end{equation}

\begin{itemize}
	\item Nếu $n_{21}>1$ thì $r<i$: Tia khúc xạ lệch lại gần pháp tuyến hơn. Ta nói môi trường (2) chiết quang hơn môi trường (1).
	\item Nếu $n_{21}<1$ thì $r>i$: Tia khúc xạ lệch xa pháp tuyến hơn. Ta nói môi trường (2) chiết quang kém môi trường (1).
\end{itemize}
Chiết suất tỉ đối này bằng tỉ số giữa các tốc độ $v_1$ và $v_2$ của ánh sáng khi đi trong môi trường (1) và trong môi trường (2).
\begin{equation}
n_{21}=\dfrac{n_2}{n_1}=\dfrac{v_1}{v_2},
\end{equation}
trong đó,
\begin{itemize}
	\item $n_2$ là chiết suất của môi trường (2),
	\item $n_1$ là chiết suất môi trường khúc (1). 
	\item $v_2$ là tốc độ ánh sáng trong môi trường (2),
	\item $v_1$ là tốc độ ánh sáng trong môi trường (1).
\end{itemize}
\subsubsection{Chiết suất tuyệt đối}
Chiết suất tuyệt đối (thường gọi tắt là chiết suất) của một môi trường là chiết suất tỉ đối của môi trường đó đối với chân không.

Hệ thức:
\begin{equation}
n=\dfrac{c}{v},
\end{equation}
trong đó,
\begin{itemize}
	\item $n$ là chiết suất tuyệt đối của môi trường,
	\item $c$ là tốc độ ánh sáng sáng trong chân không. 
	\item $v$ là tốc độ ánh sáng sáng trong môi trường đang xét. 
\end{itemize}

\luuy{Quy ước, chiết suất của chân không bằng 1, chiết suất của không khí thường được tính tròn là 1.

Mọi môi trường trong suốt đều có chiết suất tuyệt đối lớn hơn 1.}


 \subsection{Tính thuận nghịch của ánh sáng}
Ánh sáng truyền đi theo đường nào thì cũng truyền ngược lại theo đường đó. 

Theo tính chất thuận nghịch về sự truyền ánh sáng, ta có: 
\begin{equation}
n_{12}=\dfrac{1}{n_{21}},
\end{equation}
 
\section{Bài tập }
\begin{dang}{Chiết suất tuyệt đối, chiết suất tỉ đối}
\end{dang}
\textbf{Phương pháp giải}

Sử dụng công thức định luật khúc xạ ánh sáng:
\begin{equation}
n_1\cdot \sin i=n_2\cdot \sin r,
\end{equation}
Chiết suất tỉ đối:
\begin{equation}
n_{21}=\dfrac{n_2}{n_1}=\dfrac{v_1}{v_2},
\end{equation}
Chiết suất tuyệt đối:
\begin{equation}
n=\dfrac{c}{v},
\end{equation}


\vspace{1em}
\viduii{2}{
Chiết suất tỉ đối giữa môi trường khúc xạ với môi trường tới
\begin{mcq}
	\item luôn lớn hơn 1. 
	\item luôn nhỏ hơn 1.
	\item bằng tỉ số giữa chiết suất tuyệt đối của môi trường khúc xạ và chiết suất tuyệt đối của môi trường tới.
	\item bằng hiệu số giữa chiết suất tuyệt đối của môi trường khúc xạ và chiết suất tuyệt đối của môi trường tới.
\end{mcq}}
{\begin{center}
	\textbf{Hướng dẫn giải:}
\end{center}

{Chiết suất tỉ đối được tính theo công thức: $n_{21}=\dfrac{n_2}{n_1}$, tức là chiết suất tỉ đối giữa môi trường khúc xạ với môi trường tới bằng tỉ số giữa chiết suất tuyệt đối của môi trường khúc xạ và chiết suất tuyệt đối của môi trường tới. Nó có thể lớn hoặc nhỏ hơn 1.

\textbf{	Đáp án: C.}
}
}

\viduii{2}{
Tính vận tốc của ánh sáng trong thủy tinh. Biết thủy tinh có chiết suất $n = \text{1,5}$ và vận tốc ánh sáng trong chân không là $c = 3\cdot 10^8\ \text{m/s}$. 

\begin{mcq}(4)
	\item $5\cdot 10^8\ \text{m/s}$.
	\item $3 \cdot 10^8\ \text{m/s}$.
	\item $\text{4,5} \cdot 10^8\ \text{m/s}$.
	\item $2 \cdot 10^8\ \text{m/s}$.
\end{mcq}}
{\begin{center}
	\textbf{Hướng dẫn giải:}
\end{center}

{Chiết suất tuyệt đối: $n=\dfrac{c}{v}$
	
$\Rightarrow v=\dfrac{c}{n}=2\cdot 10^8\ \text{m/s}$.
	

\textbf{	Đáp án: D.}
	}
}
\viduii{3}{
Tính vận tốc của ánh sáng truyền trong môi trường nước. Biết tia sáng truyền từ không khí với góc tới là $i = 60^\circ$ thì góc khúc xạ trong nước là $r = 40^\circ$. Biết vận tốc ánh sáng ngoài không khí $c = 3\cdot 10^8\ \text{m/s}$.

\begin{mcq}(4)
	\item $\text{2,2}\cdot 10^8\ \text{m/s}$.
	\item $\text{3,2} \cdot 10^8\ \text{m/s}$.
	\item $\text{4,2} \cdot 10^8\ \text{m/s}$.
	\item $\text{5,2}\cdot 10^8\ \text{m/s}$.
\end{mcq}}
{\begin{center}
	\textbf{Hướng dẫn giải:}
\end{center}

{Định luật khúc xạ ánh sáng: $n_1\cdot \sin i=n_2\cdot \sin r$
	
	$\Rightarrow n_2=\dfrac{n_1\sin i}{n_2}=\text{1,38}$.
	
	Chiết suất tuyệt đối: $n_2=\dfrac{c}{v_2}$
	
	$\Rightarrow v_2=\dfrac{c}{n_2}=\text{2,2}\cdot 10^8\ \text{m/s}$.
	
	
\textbf{	Đáp án: A.}
}
}


