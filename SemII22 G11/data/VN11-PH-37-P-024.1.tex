%\setcounter{chapter}{1}
\chapter{Luyện tập: Lăng kính}
\begin{enumerate}
	
	\item Chọn câu trả lời \textbf{sai}?
	\begin{mcq}
		\item Lăng kính là môi trường trong suốt đồng tính và đẳng hướng được giới hạn bởi hai mặt phẳng không song song.
		\item Tia sáng đơn sắc qua lăng kính sẽ luôn luôn bị lệch về phía đáy.
		\item Tia sáng không đơn sắc qua lăng kính thì chùm tia ló sẽ bị tán sắc.
		\item Góc lệch của tia đơn sắc qua lăng kính là $D=i+i'-A$.
	\end{mcq}
	
	
	\item Chiếu một chùm tia sáng đỏ hẹp coi như một tia sáng vào mặt bên của một lăng kính có tiết diện thẳng là tam giác cân ABC có góc chiết quang $A = 8^\circ$ theo phương vuông góc với mặt phẳng phân giác của góc chiết quang tại một điểm tới rất gần A. Biết chiết suất của lăng kính đối với tia đỏ là $n_\text{đ}=\text{1,5}$. Góc lệch của tia ló so với tia tới là
	
	\begin{mcq}(4)
		\item $2^\circ$.
		\item $8^\circ$.
		\item $4^\circ$.
		\item $12^\circ$.
	\end{mcq}
	
	\item Lăng kính có góc chiết quang $A = 60^\circ$, chiết suất $n=\sqrt{2}$ trong không khí. Tia sáng tới mặt thứ nhất với góc tới $i$. Có tia ló ở mặt thứ hai khi
	
	\begin{mcq}(4)
		\item $i\leq 15^\circ$.
		\item $i\geq \text{21,47}^\circ$.
		\item $i\leq 15^\circ$.
		\item $i\leq \text{21,47}^\circ$.
	\end{mcq}
	
	\item Cho một chùm tia sáng chiếu vuông góc đến mặt AB của một lăng kính ABC vuông góc tại A và góc ABC bằng $30^\circ$, làm bằng thủy tinh chiết suất $n = \text{1,3}$. Tính góc lệch của tia ló so với tia tới.  
	
	\begin{mcq}(4)
		\item $\text{40,5}^\circ$.
		\item $\text{20,2}^\circ$.
		\item $\text{19,5}^\circ$.  	
		\item $\text{10,5}^\circ$. 
	\end{mcq}
	
	\item Sử dụng hình vẽ về đường đi của tia sáng qua lăng kính có góc chiết quang $A$: SI là tia tới, JR là tia ló, $D$ là góc lệch giữa tia tới và tia ló, $n$ là chiết suất của chất làm lăng kính. Cho $i_1, \ r_2$ là góc tới ở mặt bên thứ nhất và thứ hai; $r_1, \ i_2$ là góc khúc xạ ở mặt bên thứ nhất và thứ hai.  Công thức nào trong các công thức sau đây là đúng? 
	\begin{mcq}(2)
		\item $\sin i_1=n\sin r_1$.
		\item $\sin i_2=n\sin r_2$.	
		\item $D=i_1+i_2-A$.
		\item A, B và C đều đúng.
	\end{mcq}
	
	\item Điều nào sau đây là đúng khi nói về lăng kính và đường đi của một tia sáng qua lăng kính?
	\begin{mcq}
		\item Tiết diện thẳng của lăng kính là một tam giác cân.
		\item Lăng kính là một khối chất trong suốt hình lăng trụ đứng, có tiết diện thẳng là một hình tam giác.
		\item Mọi tia sáng khi quang lăng kính đều khúc xạ và cho tia ló ra khỏi lăng kính.	
		\item Cả  A và C đều đúng.
	\end{mcq}
	
	\item Điều nào sau đây là đúng khi nói về lăng kính?
	\begin{mcq}
		\item Lăng kính là một khối chất trong suốt hình lăng trụ đứng, có tiết diện thẳng là một hình tam giác
		\item Góc chiết quang của lăng kính luôn nhỏ hơn $90^\circ$.
		\item Hai mặt bên của lăng kính luôn đối xứng nhau qua mặt phẳng phân giác của góc chiết quang.
		\item Tất cả các lăng kính chỉ sử dụng hai mặt bên cho ánh sáng truyền qua.
	\end{mcq}
	
	\item Lăng kính phản xạ toàn phần là một khối lăng trụ thủy tinh có tiết diện thẳng là
	\begin{mcq} (2)
		\item một tam giác vuông cân.
		\item một hình vuông.
		\item một tam giác đều.
		\item một tam giác bất kì.
	\end{mcq}
	
	\item Một lăng kính đặt trong không khí, có góc chiết quang $A = 30^\circ$ nhận một tia sáng tới vuông góc với mặt bên AB và tia ló sát mặt bên AC của lăng kính. Chiết suất $n$ của lăng kính là
	
	\begin{mcq}(4)
		\item $\text{1,2}$.
		\item $\text{1,3}$.
		\item $\text{1,5}$.
		\item $\text{2,0}$.
	\end{mcq}
	
	\item Tiết diện thẳng của đoạn lăng kính là tam giác đều. Một tia sáng đơn sắc chiếu tới mặt bên lăng kính và cho tia ló đi ra từ một mặt bên khác. Nếu góc tới và góc ló là $45^\circ$ thì góc lệch là
	\begin{mcq}(4)
		\item $10^\circ$.
		\item $20^\circ$.
		\item $30^\circ$.
		\item $40^\circ$
	\end{mcq}
\end{enumerate}
\begin{center}
	\textbf{ĐÁP ÁN}
	
	\begin{longtable}[\textwidth]{|p{0.1\textwidth}|p{0.1\textwidth}|p{0.1\textwidth}|p{0.1\textwidth}|p{0.1\textwidth}|p{0.1\textwidth}|p{0.1\textwidth}|p{0.1\textwidth}|}
		% --- first head
		\hline%\hspace{2 pt}
		\multicolumn{1}{|c}{\textbf{Câu 1}} & \multicolumn{1}{|c|}{\textbf{Câu 2}} & \multicolumn{1}{c|}{\textbf{Câu 3}} &
		\multicolumn{1}{c|}{\textbf{Câu 4}} &
		\multicolumn{1}{c|}{\textbf{Câu 5}} &
		\multicolumn{1}{c|}{\textbf{Câu 6}} &
		\multicolumn{1}{c|}{\textbf{Câu 7}} &
		\multicolumn{1}{c|}{\textbf{Câu 8}}\\
		\hline
		B.&C. &B. &D. &D. &D. &A. &A.	\\
		\hline
		
		\multicolumn{1}{|c|}{\textbf{Câu 9}} & \multicolumn{1}{c|}{\textbf{Câu 10}} & \multicolumn{1}{c|}{\textbf{}} &
		\multicolumn{1}{c|}{\textbf{}} &
		\multicolumn{1}{c|}{\textbf{}} &
		\multicolumn{1}{c|}{\textbf{}} &
		\multicolumn{1}{c|}{\textbf{}} &
		\multicolumn{1}{c|}{} \\
		\hline
		D. &C. & & & & & &\\
		\hline	
		
		
	\end{longtable}
	
\end{center}









