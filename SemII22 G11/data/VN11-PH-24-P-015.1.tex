%\setcounter{chapter}{1}
\chapter{Luyện tập: Từ trường}
\begin{enumerate}
	\item Từ trường là dạng vật chất tồn tại trong không gian và
	\begin{mcq}
		\item tác dụng lực hút lên các vật.
		\item tác dụng lực điện lên điện tích.	
		\item tác dụng lực từ lên nam châm và dòng điện.
		\item tác dụng lực đẩy lên các vật đặt trong nó.
	\end{mcq}
	
	
	\item Tính chất cơ bản của từ trường là
	\begin{mcq}
		\item gây ra lực từ tác dụng lên nam châm hoặc lên dòng điện đặt trong nó.	
		\item gây ra lực hấp dẫn lên các vật đặt trong nó.
		\item gây ra lực đàn hồi tác dụng lên các dòng điện và nam châm đặt trong nó.
		\item gây ra sự biến đổi về tính chất điện của môi trường xung quanh.
	\end{mcq}
	
	\item  Dây dẫn mang dòng điện không tương tác với
	\begin{mcq}
		\item Các điện tích chuyển động.
		\item Nam châm đứng yên.	
		\item Các điện tích đứng yên.
		\item Nam châm chuyển động.
	\end{mcq}
	
	\item Lực nào sau đây không phải lực từ?
	\begin{mcq}
		\item Lực Trái Đất tác dụng lên vật nặng;
		\item Lực Trái Đất tác dụng lên kim nam châm ở trạng thái tự do làm nó định hướng theo phương bắc nam;
		\item Lực nam châm tác dụng lên dây dẫn bằng nhôm mang dòng điện;	
		\item Lực hai dây dẫn mang dòng điện tác dụng lên nhau.
	\end{mcq}
	
	\item Vật liệu nào sau đây không thể dùng làm nam châm?
	\begin{mcq}
		\item Sắt và hợp chất của sắt.
		\item Niken và hợp chất của niken.	
		\item Cô ban và hợp chất của cô ban.
		\item Nhôm và hợp chất của nhôm.
	\end{mcq}
	
	\item Phát biểu nào sau đây là \textbf{không đúng}?
	
	Người ta nhận ra từ trường tồn tại xung quanh dây dẫn mang dòng điện vì:
	\begin{mcq}
		\item có lực tác dụng lên một dòng điện khác đặt song song cạnh nó.	
		\item có lực tác dụng lên một kim nam châm đặt song song cạnh nó.	
		\item có lực tác dụng lên một hạt mang điện chuyển động dọc theo nó.	
		\item có lực tác dụng lên một hạt mang điện đứng yên đặt bên cạnh nó.
	\end{mcq}
	
	\item Từ phổ là
	\begin{mcq}
		\item hình ảnh của các đường mạt sắt cho ta hình ảnh của các đường sức từ của từ trường.
		\item hình ảnh tương tác của hai nam châm với nhau.
		\item hình ảnh tương tác giữa dòng điện và nam châm. 
		\item  hình ảnh tương tác của hai dòng điện chạy trong hai dây dẫn thẳng song song.
	\end{mcq}
	\item Phát biểu nào sau đây là \textbf{không đúng}?
	
	\begin{mcq}
		\item Qua bất kỳ điểm nào trong từ trường ta cũng có thể vẽ được một đường sức từ.
		\item  Đường sức từ do nam châm thẳng tạo ra xung quanh nó là những đường thẳng.
		\item Đường sức mau ở nơi có cảm ứng từ lớn, đường sức thưa ở nơi có cảm ứng từ nhỏ.
		\item Các đường sức từ là những đường cong kín.
	\end{mcq}
	
	\item Phát biểu nào sau đây là \textbf{không đúng}?
	
	Từ trường đều là từ trường có
	\begin{mcq}
		\item các đường sức song song và cách đều nhau.
		\item cảm ứng từ tại mọi nơi đều bằng nhau.
		\item lực từ tác dụng lên các dòng điện như nhau.
		\item các đặc điểm bao gồm cả phương án A và B.
	\end{mcq}
	
	\item Phát biểu nào sau đây là \textbf{đúng}? 
	\begin{mcq}
		
		\item  Các đường mạt sắt của từ phổ chính là các đường sức từ.
		\item Các đường sức từ của từ trường đều có thể là những đường cong cách đều nhau.
		\item Các đường sức từ là những đường cong kín.
		\item Một hạt mang điện chuyển động theo quỹ đạo tròn trong từ trường thì quỹ đạo chuyển động của hạt chính là một đường sức từ.
	\end{mcq}
\end{enumerate}

\begin{center}
	\textbf{ĐÁP ÁN}
	\begin{longtable}[\textwidth]{|p{0.15\textwidth}|p{0.15\textwidth}|p{0.15\textwidth}|p{0.15\textwidth}|p{0.15\textwidth}|}
		% --- first head
		\hline%\hspace{2 pt}
		\multicolumn{1}{|c}{\textbf{Câu 1}} & \multicolumn{1}{|c|}{\textbf{Câu 2}} & \multicolumn{1}{c|}{\textbf{Câu 3}} &
		\multicolumn{1}{c|}{\textbf{Câu 4}} &
		\multicolumn{1}{c|}{\textbf{Câu 5}} \\
		\hline
		D.&A. &C. &C. &B.	\\
		\hline
		
		\multicolumn{1}{|c|}{\textbf{Câu 6}} &
		\multicolumn{1}{c|}{\textbf{Câu 7}} &
		\multicolumn{1}{c|}{\textbf{Câu 8}} &
		\multicolumn{1}{c|}{\textbf{Câu 9}} & \multicolumn{1}{c|}{\textbf{Câu 10}}  \\
		\hline
		 A. &B. &D.&D. &C. \\
		\hline	
		
		
	\end{longtable}
	
\end{center}







