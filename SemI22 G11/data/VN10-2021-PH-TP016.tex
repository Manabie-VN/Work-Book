\whiteBGstarBegin
\setcounter{section}{0}
\section{Trắc nghiệm}
\begin{enumerate}[label=\bfseries Câu \arabic*:]
	
	\item \mkstar{1}
	
	\cauhoi
	{Một chất điểm chuyển động tròn đều với chu kì $T$, bán kính $R$. Công thức tính gia tốc hướng tâm của vật là
		\begin{mcq}(4)
			\item $a=4 \pi ^2 \dfrac{R^2}{T^2}$.
			\item $a=4 \pi \dfrac{R}{T^2}$.
			\item $a=4 \pi ^2 \dfrac{R}{T^2}$.
			\item $a=4 \pi ^2 \dfrac{R}{T}$.
		\end{mcq}
	}
	
	\loigiai
	{	\textbf{Đáp án: C.}
		
	Gia tốc hướng tâm:
	$$a=\omega^2 R = \left(\dfrac{2\pi}{T}\right)^2 R = 4 \pi ^2 \dfrac{R}{T^2}$$
	}

	\item \mkstar{2}
	
	\cauhoi
	{Một vật đang chuyển động tròn đều dưới tác dụng của lực hướng tâm $F$. Nếu bán kính quỹ đạo tăng gấp hai lần so với trước và đồng thời giảm tốc độ quay còn một nửa thì so với ban đầu, lực hướng tâm
		\begin{mcq}(2)
			\item giảm 8 lần.
			\item giảm 4 lần.
			\item giảm 2 lần.
			\item không thay đổi.
		\end{mcq}
	}
	
	\loigiai
	{	\textbf{Đáp án: A.}
		
	Ta lập tỉ lệ:
	$$\dfrac{F_\text{ht 1}}{F_\text{ht 2}} = \dfrac{v_1 ^2 r_2}{v_2 ^2 r_1} = 8$$
	
	Vậy lực hướng tâm giảm 8 lần.
	}
	\item \mkstar{2}
	
	\cauhoi
	{Khi ô tô chuyển động đều trên một đoạn đường có dạng cung tròn, lực tác dụng đóng vai trò lực hướng tâm là
		\begin{mcq}
			\item trọng lực của ô tô.
			\item phản lực của mặt đường.
			\item hợp lực của tất cả các lực tác dụng lên xe.
			\item lực ma sát giữa bánh xe và mặt đường.
		\end{mcq}
	}
	
	\loigiai
	{	\textbf{Đáp án: C.}
		
	Khi ô tô chuyển động đều trên một đoạn đường có dạng cung tròn, lực tác dụng đóng vai trò lực hướng tâm là hợp lực của tất cả các lực tác dụng lên xe.
	}
	\item \mkstar{3}
	
	\cauhoi
	{Một đĩa tròn nằm ngang có thể quay quanh một trục thẳng đứng. Vật $m=\SI{100}{g}$ đặt trên đĩa, nối với trục quay bởi một lò xo nằm ngang. Nếu số vòng quay không quá $\omega_1 = 2\ \text{vòng/s}$ thì lò xo không biến dạng. Nếu số vòng quay tăng dần đến $\omega_2 = 5\ \text{vòng/s}$ thì lò xo dãn gấp đôi. Tính độ cứng $k$ của lò xo.
		\begin{mcq}(4)
			\item $\SI{182}{N/m}$.
			\item $\SI{232}{N/m}$.
			\item $\SI{419}{N/m}$.
			\item $\SI{336}{N/m}$.
		\end{mcq}
	}
	
	\loigiai
	{	\textbf{Đáp án: A.}
		
	Đổi $\omega_1=\SI{4\pi}{rad/s}$, $\omega_2 = \SI{10\pi}{rad/s}$
	
	Khi lò xo chưa biến dạng:
	$$F_\text{ms} = F_\text{ht} = m \omega_1 l_0$$
	
	Khi lò xo biến dạng dãn gấp đôi:
	$$F_\text{ht} = F_\text{đh} + F-\text{ms} \Rightarrow m \omega_2 ^2 2l_0 = k (2l_0 - l_0) + m \omega_1 l_0 \Rightarrow k = m(2\omega_2 - \omega_1) = \SI{182}{N/m}$$
	}
	

	\item \mkstar{4}
	
	\cauhoi
	{Hai quả cầu $m_1=2m_2$ nối với nhau bằng dây dài $l=\SI{12}{cm}$ có thể chuyển động không ma sát trên một trục nằm ngang qua tâm của hai quả cầu. Cho hệ quay đều quanh trục thẳng đứng. Biết hai quả cầu đứng yên không trượt trên trục ngang. Tìm khoảng cách từ hai quả cầu đến trục quay.
		\begin{mcq}(2)
			\item $r_1=\SI{5}{}$, $r_2=\SI{8}{cm}$.
			\item $r_1=\SI{4}{}$, $r_2=\SI{8}{cm}$.
			\item $r_1=\SI{4}{}$, $r_2=\SI{6}{cm}$.
			\item $r_1=\SI{4}{}$, $r_2=\SI{10}{cm}$.
		\end{mcq}
	}
	
	\loigiai
	{	\textbf{Đáp án: B.}
		
	Các quả cầu chuyển động quanh trục có khoảng cách đến tâm khác nhau, nhưng tốc độ góc thì vẫn như nhau. Lực căng dây đóng vai trò lực hướng tâm.
	
	Ta có:
	$$m_1 \omega^2 r_1 = m_2 \omega^2 r_2 \Rightarrow m_1 r_1 = m_2 r_2$$
	
	Mà $$r_1+r_2 = l$$
	
	Suy ra: $r_1 = \SI{4}{cm}$, $r_2 = \SI{8}{cm}$.		
	}
	
	
\end{enumerate}

\whiteBGstarEnd

\loigiai
{
	\begin{center}
		\textbf{BẢNG ĐÁP ÁN}
	\end{center}
	\begin{center}
		\begin{tabular}{|m{2.8em}|m{2.8em}|m{2.8em}|m{2.8em}|m{2.8em}|m{2.8em}|m{2.8em}|m{2.8em}|m{2.8em}|m{2.8em}|}
			\hline
			1.C  & 2.A  & 3.C  & 4.A  & 5.B  & & & & &  \\
			\hline
			
		\end{tabular}
	\end{center}
}
\section{Tự luận}
\begin{enumerate}[label=\bfseries Câu \arabic*:]
	\item \mkstar{1}
	
	\cauhoi{
		Phát biểu và viết công thức của lực hướng tâm.
	}
	
	\loigiai{
		
	Lực hướng tâm là lực (hay hợp của các lực) tác dụng vào một vật chuyển động tròn đều và gây ra cho vật gia tốc hướng tâm.
	
	Biểu thức:
	$$F_\text{ht} = ma_\text{ht} = \dfrac{mv^2}{r}=m\omega^2r $$
	}
	
	\item \mkstar{2}
	
	\cauhoi
	{\begin{enumerate}
			\item Lực hướng tâm có phải là một loại lực mới như lực hấp dẫn hay không?
			\item Nếu nói (trong ví dụ b sách giáo khoa) vật chịu 4 lực là $\vec P$, $\vec N$, $\vec F_\text{msn}$ và $\vec F_\text{ht}$ thì đúng hay sai? Tại sao?
		\end{enumerate}
	}
	
	\loigiai
	{\begin{enumerate}
			\item Lực hướng tâm có phải là một loại lực mới như lực hấp dẫn hay không?
			
			Lực hướng tâm không phải là một loại lực mới như lực hấp dẫn, lực hướng tâm có thể là một trong hoặc là hợp lực của các lực chúng ta đã học.
			
			\item Nếu nói (trong ví dụ b sách giáo khoa) vật chịu 4 lực là $\vec P$, $\vec N$, $\vec F_\text{msn}$ và $\vec F_\text{ht}$ thì đúng hay sai? Tại sao?
			
			Sai, vì lực ma sát chính là lực hướng tâm trong trường hợp này.
		\end{enumerate}
	}
	\item \mkstar{3}
	
	\cauhoi
	{Một vật có khối lượng $m=\SI{20}{g}$ đặt ở mép một chiếc bàn quay. Hỏi phải quay bàn với tần số vòng lớn nhất là bao nhiêu để vật không bị văng ra khỏi bàn? Cho biết mặt bàn hình tròn, bán kính $\SI{1}{m}$. Lực ma sát nghỉ cực đại bằng $\SI{0.08}{N}$.
	}
	
	\loigiai
	{Điều kiện để vật không bị văng ra khỏi bàn xoay là
		$$F_\text{ht} \leq F_\text{msn max}$$
		
	Trong đó: $F_\text{ht} = m\omega^2 r = m (2\pi f)^2 r$
	
	Và $F_\text{msn max} = \SI{0.08}{N}$
	
	Suy ra $$f^2 \leq \dfrac{F_\text{msn max}}{m 4 \pi ^2 r} \Rightarrow f^2 \leq {0.10132} \Rightarrow f\leq \SI{0.32}{v/s}$$
	}
	\item \mkstar{4}
	
	\cauhoi
	{Một ô tô có khối lượng $\SI{1200}{\kilogram}$ chuyển động đều qua một đoạn cầu vượt (coi là cung tròn) với tốc độ $\SI{36}{\kilo \meter / \hour}$. Hỏi áp lực của ô tô vào mặt đường tại điểm cao nhất bằng bao nhiêu? Biết bán kính cong của đoạn cầu vượt là $\SI{50}{\meter}$. Lấy $g=\SI{10}{\meter / \second ^2}$.
	}
	
	\loigiai
	{
	Chọn chiều dương theo phương thẳng đứng là chiều hướng xuống (chiều hướng tâm).
	
	Các lực tác dụng lên vật theo phương thẳng đứng: trọng lực $\vec P$, phản lực $\vec N$.
	
	Áp dụng định luật II Newton trên phương thẳng đứng:
	$$\vec P + \vec N = m \vec a_\text{ht}.$$
	Chiếu lên chiều dương (chiều hướng tâm):
	$$P-N=ma_\text {ht} \Rightarrow N=P-ma_\text{ht} = mg - m \dfrac{v^2}{r}.$$
	Thay số:
	$$N=\SI{1200}{\kilogram} \cdot \SI{10}{\meter / \second ^2} - \SI{1200}{\kilogram} \dfrac{(\SI{10}{\meter / \second})^2}{\SI{50}{\meter}} = \SI{9600}{\newton}.$$
	
	Vậy áp lực của ô tô vào mặt đường tại điểm cao nhất bằng $\SI{9600}{\newton}$.
	}
	\item \mkstar{4}
	
	\cauhoi
	{Nếu cầu võng xuống (các số liệu vẫn giữ như trên) thì áp lực của ô tô vào mặt cầu tại điểm thấp nhất là bao nhiêu?
	}
	
	\loigiai
	{Chọn chiều dương theo phương thẳng đứng là chiều hướng lên (chiều hướng tâm).
		
		Các lực tác dụng lên vật theo phương thẳng đứng: trọng lực $\vec P$, phản lực $\vec N$.
		
		Áp dụng định luật II Newton trên phương thẳng đứng:
		$$\vec P + \vec N = m \vec a_\text{ht}.$$
		Chiếu lên chiều dương (chiều hướng tâm):
		$$N-P=ma_\text {ht} \Rightarrow N=P+ma_\text{ht} = mg + m \dfrac{v^2}{r}.$$
		Thay số:
		$$N=\SI{1200}{\kilogram} \cdot \SI{10}{\meter / \second ^2} + \SI{1200}{\kilogram} \dfrac{(\SI{10}{\meter / \second})^2}{\SI{50}{\meter}} = \SI{14400}{\newton}.$$
		
		Vậy áp lực của ô tô vào mặt cầu tại điểm thấp nhất bằng $\SI{14400}{\newton}$.
	}
\end{enumerate}