\whiteBGstarBegin
\setcounter{section}{0}
\section{Trắc nghiệm}
\begin{enumerate}[label=\bfseries Câu \arabic*:]
	
	
	\item \mkstar{1}
	
	\cauhoi
	{Khi bị đốt nóng, các hạt mang điện tự do trong không khí
		\begin{mcq}(2)
			\item chỉ là ion dương.
			\item chỉ là ion âm.
			\item là electron, ion dương, ion âm.
			\item chỉ là electron.
		\end{mcq}
		
	}
	\loigiai
	{	\textbf{Đáp án: C.}
		
		Khi bị đốt nóng, các hạt mang điện tự do trong không khí là electron, ion dương, ion âm.
	}
	\item \mkstar{1}
	
	\cauhoi
	{Dòng chuyển dời có hướng của các electron, ion âm, ion dương là dòng điện trong
		\begin{mcq}(2)
			\item chất khí.
			\item chất bán dẫn.
			\item kim loại.
			\item chất điện phân.
		\end{mcq}
		
	}
	\loigiai
	{	\textbf{Đáp án: A.}
		
	}
	\item \mkstar{1}
	
	\cauhoi
	{Hồ quang điện là
		\begin{mcq}
			\item quá trình phóng điện tự lực trong chất khí ở áp suất rất cao.
			\item quá trình phóng điện tự lực trong chất khí ở áp suất thường hoặc áp suất thấp.
			\item quá trình phóng điện không tự lực trong chất khí.
			\item quá trình phóng điện tự lực trong chất khí ở áp suất cao.
		\end{mcq}
		
	}
	\loigiai
	{	\textbf{Đáp án: B.}
		
		Hồ quang điện là quá trình phóng điện tự lực trong chất khí ở áp suất thường hoặc áp suất thấp đặt giữa hai điện cực có hiệu điện thế không lớn.
	}
	\item \mkstar{1}
	
	\cauhoi
	{Hiện tượng hồ quang điện được ứng dụng trong
		\begin{mcq}(2)
			\item kỹ thuật hàn điện.
			\item kỹ thuật mạ điện.
			\item diode bán dẫn.
			\item ống phóng điện tử.
		\end{mcq}
		
	}
	\loigiai
	{	\textbf{Đáp án: A.}
		
	}
	\item \mkstar{1}
	
	\cauhoi
	{Điện trường tối thiểu giữa hai cực để phát sinh tia lửa điện trong không khí ở điều kiện thường theo đơn vị $\SI{}{V/m}$ là
		\begin{mcq}(4)
			\item $\SI{8e3}{V/m}$.
			\item $\SI{60}{V/m}$.
			\item $\SI{10e4}{V/m}$.
			\item $\SI{3e6}{V/m}$.
		\end{mcq}
		
	}
	\loigiai
	{	\textbf{Đáp án: D.}
		
		Điện trường tối thiểu giữa hai cực để phát sinh tia lửa điện trong không khí ở điều kiện thường là $\SI{3e6}{V/m}$.
	}
	\item \mkstar{1}
	
	\cauhoi
	{Cách tạo ra tia lửa điện là
		\begin{mcq}
			\item nung nóng không khí giữa hai đầu tụ điện được tích điện.
			\item đặt vào hai đầu của hai thanh than một hiệu điện thế khoảng $\SI{40}{V}$ đến $\SI{50}{V}$.
			\item tạo ra một điện trường rất lớn khoảng $\SI{3e6}{V/m}$ trong chân không.
			\item tạo ra một điện trường rất lớn khoảng $\SI{3e6}{V/m}$ trong không khí.
		\end{mcq}
		
	}
	\loigiai
	{	\textbf{Đáp án: D.}
		
	}
	\item \mkstar{2}
	
	\cauhoi
	{Khi tạo ra hồ quang điện, ban đầu ta phải cho hai đầu thanh than chạm vào nhau là để
		\begin{mcq}
			\item tạo ra cường độ điện trường rất lớn.
			\item tăng tính dẫn điện ở chỗ tiếp xúc của hai thanh than.
			\item làm giảm điện trở ở chỗ tiếp xúc của hai thanh than.
			\item làm tăng nhiệt độ ở chỗ tiếp xúc của hai thanh than lên rất lớn.
		\end{mcq}
		
	}
	\loigiai
	{	\textbf{Đáp án: D.}
		
		Khi tạo ra hồ quang điện, ban đầu ta phải cho hai đầu thanh than chạm vào nhau là để làm tăng nhiệt độ ở chỗ tiếp xúc của hai thanh than lên rất lớn, tạo ra các hạt tải điện trong vùng không khí xung quanh hai đầu thanh than.
	}
	\item \mkstar{2}
	
	\cauhoi
	{Chọn phát biểu đúng.
		\begin{mcq}
			\item Không khí là chất điện môi trong mọi điều kiện.
			\item Không khí có thể dẫn điện trong mọi điều kiện.
			\item Không khí chỉ dẫn điện khi có tác nhân ion hóa.
			\item Không khí chỉ dẫn điện khi bị đốt nóng.
		\end{mcq}
		
	}
	\loigiai
	{	\textbf{Đáp án: C.}
		
		Chất khí chỉ dẫn điện khi có tác nhân ion hóa.
	}
	\item \mkstar{2}
	
	\cauhoi
	{Khi nói về sự phụ thuộc của cường độ dòng điện vào hiệu điện thế trong quá trình dẫn điện không tự lực của chất khí, điều nào dưới đây là \textbf{sai}?
		\begin{mcq}
			\item Khi $U$ nhỏ, $I$ tăng theo $U$.
			\item Khi $U$ đủ lớn, $I$ đạt giá trị bão hòa.
			\item Nếu $U$ quá lớn, $I$ tăng rất nhanh theo $U$.
			\item Với mọi giá trị của $U$, $U$ và $I$ tỉ lệ thuận.
		\end{mcq}
		
	}
	\loigiai
	{	\textbf{Đáp án: D.}
		
		Đặc tuyến Vôn-Ampe trong quá trình dẫn điện của chất khí không phải đường thẳng, nên $U$ và $I$ không tỉ lệ thuận với nhau. Do đó, quá trình dẫn điện không tự lực không tuân theo định luật Ôm.
	}
	\item \mkstar{2}
	
	\cauhoi
	{Cách gọi sấm, sét đúng là
		\begin{mcq}
			\item sấm là tiếng nổ khi có sự phóng điện giữa các đám mây với nhau.
			\item sét là tiếng nố khi có sự phóng điện trong tự nhiên với cường độ lớn.
			\item sấm là tiếng nổ khi có sự phóng điện trong tự nhiên với cường độ nhỏ.
			\item sét là tiếng nổ khi có sự tiếp xúc giữa các đám mây với mặt đất.
		\end{mcq}
		
	}
	\loigiai
	{	\textbf{Đáp án: A.}
		
		Khi có cơn giông, các đám mây gần mặt đất thường tích điện âm và mặt đất tích điện dương. Giữa đám mây và mặt đất có hiệu điện thế lớn. Những chỗ nhô cao trên mặt đất là nơi tập trung điện tích. Sét là tia lửa điện hình thành giữa đám mây và mặt đất, thường đánh vào các nơi cao như ngọn cây, cột thu lôi, $\ldots$.
		
		Khi có hai đám mây tích điện trái dấu lại gần nhau, hiệu điện thế giữa chúng có thể lên tới hàng triệu Vôn. Giữa hai đám mây có hiện tượng phóng tia lửa điện (sét) và khi chạm nhau tạo thành tiếng nổ (sấm).
		
		Hiện tượng phóng tia lửa điện giữa đám mây và mặt đất gọi là hiện tượng sét đánh.
	}
\end{enumerate}

\whiteBGstarEnd

\loigiai
{
	\begin{center}
		\textbf{BẢNG ĐÁP ÁN}
	\end{center}
	\begin{center}
		\begin{tabular}{|m{2.8em}|m{2.8em}|m{2.8em}|m{2.8em}|m{2.8em}|m{2.8em}|m{2.8em}|m{2.8em}|m{2.8em}|m{2.8em}|}
			\hline
			1.C  & 2.A  & 3.B  & 4.A  & 5.D  & 6.D  & 7.D  & 8.C  & 9.D  & 10.A  \\
			\hline
			
		\end{tabular}
	\end{center}
}
\section{Tự luận}
\begin{enumerate}[label=\bfseries Câu \arabic*:]
	\item \mkstar{1}
	
	\cauhoi{
		Trình bày nguyên nhân gây ra hiện tượng hồ quang điện và tia lửa điện.
	}
	
	\loigiai{
		
		Nguyên nhân gây ra hồ quang điện: do phát xạ nhiệt electron từ cathode bị đốt nóng và quá trình nhân số hạt tải điện.
		
		Nguyên nhân gây ra tia lửa điện: do quá trình ion hóa chất khí và quá trình nhân số hạt tải điện.
	}
	
	\item \mkstar{2}
	
	\cauhoi{
		Vì sao dòng điện trong hồ quang điện lại chủ yếu là dòng electron chạy từ cathode đến anode?
	}
	
	\loigiai{
		
		Dòng điện trong hồ quang điện được tạo ra do quá trình phóng điện tự lực được hình thành khi dòng điện qua chất khí có thể giữa được nhiệt độ cao của cathode để nó tự phát xạ được electron, gọi là hiện tượng phát xạ nhiệt electron.
		
		Vì vậy, dòng điện trong hồ quang điện chủ yếu là dòng electron chạy từ cathode đến anode.
	}
	\item \mkstar{4}
	
	\cauhoi{
		Cho phóng điện qua chất khí ở áp suất thấp, giữa hai điện cực cách nhau $\SI{20}{cm}$. Quãng đường chuyển động tự do của electron là $\SI{4}{cm}$. Cho rằng năng lượng mà electron nhận được trên quãng đường chuyển động tự do đủ để ion hóa chất khí. Hãy tính xem một electron đưa vào trong chất khí có thể sinh ra tối đa bao nhiêu hạt tải điện?
	}
	
	\loigiai{
		
		Vì giữa hai điện cực các nhau $\SI{20}{cm}$, quãng đường chuyển động tự do của các electron là $\SI{4}{cm}$ nên số lần ion hóa là $20/4=5$ lần.
		
		Khi va chạm với phân tử khí thì 1 electron tạo thành 1 ion dương và 1 ion âm.
	
		Sau 5 lần va chạm với phân tử khí thì tạo ra 16 ion dương và 16 ion âm.
		
		Vậy số hạt tải điện tạo ra tối đa là 32 hạt.
	}
	\item \mkstar{4}
	
	\cauhoi{
		Một dòng điện được tạo ra trong một ống chứa khí hydrogen, khi có một hiệu điện thế đủ cao giữa hai điện cực của ống thì chất khí bị ion hóa. Xác định cường độ và chiều của dòng điện chạy qua ống này khi có $\SI{4.2e18}{}$ electron và $\SI{2.2e18}{}$ protôn chuyển qua tiết diện của ống trong mỗi giây.
	}
	
	\loigiai{
		
		Chiều dòng điện qua ống phóng điện là từ cực dương sang cực âm của ống.
		
		Cường độ dòng điện qua ống là
		$$I=\dfrac{q}{t} = \dfrac{N |e|}{t} = \dfrac{(n_e + n_p) |e|}{t} = \SI{1.024}{A}.$$
	}
\end{enumerate}