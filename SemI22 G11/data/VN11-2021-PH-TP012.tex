\whiteBGstarBegin
\setcounter{section}{0}
\section{Trắc nghiệm}
\begin{enumerate}[label=\bfseries Câu \arabic*:]
	
	
	\item \mkstar{1}
	
	\cauhoi
	{Phát biểu nào sau đây đúng?
		\begin{mcq}
			\item Nguồn điện là thiết bị tạo ra và duy trì hiệu điện thế nhằm duy trì dòng điện trong mạch. Trong nguồn điện dưới tác dụng của lực lạ, các điện tích dương di chuyển từ cực dương sang cực âm.
			\item Suất điện động của nguồn điện là đại lượng đặc trưng cho khả năng sinh công của nguồn điện và được đo bằng thương số giữa công của lực lạ thực hiện khi làm dịch chuyển một điện tích dương $q$ bên trong nguồn điện từ cực âm đến cực dương và độ lớn của điện tích $q$ đó.
			\item Suất điện động của nguồn điện là đại lượng đặc trưng cho khả năng sinh công của nguồn điện và được đo bằng thương số giữa công của lực lạ thực hiện khi làm dịch chuyển một điện tích âm $q$ bên trong nguồn điện từ cực âm đến cực dương và độ lớn của điện tích $q$ đó.
			\item Suất điện động của nguồn điện là đại lượng đặc trưng cho khả năng sinh công của nguồn điện và được đo bằng thương số giữa công của lực lạ thực hiện khi làm dịch chuyển một điện tích dương $q$ bên trong nguồn điện từ cực dương đến cực âm và độ lớn của điện tích $q$ đó.
		\end{mcq}
		
	}
	\loigiai
	{	\textbf{Đáp án: B.}
		
	}
	\item \mkstar{1}
	
	\cauhoi
	{Theo quy ước, chiều dòng điện là
		\begin{mcq}
			\item chiều dịch chuyển của các electron.
			\item chiều dịch chuyển của các điện tích dương.
			\item chiều dịch chuyển của các ion.
			\item chiều dịch chuyển của điện tích.
		\end{mcq}
		
	}
	\loigiai
	{	\textbf{Đáp án: B.}
		
	}
	\item \mkstar{2}
	
	\cauhoi
	{Suất điện động của một pin là $\SI{1.5}{V}$. Công của lực lạ khi dịch chuyển điện tích 2 C từ cực âm tới cực dương bên trong nguồn điện là
		\begin{mcq}(4)
			\item $\SI{3}{J}$.
			\item $\SI{4.5}{J}$.
			\item $\SI{4.3}{J}$.
			\item $\SI{0.75}{J}$.
		\end{mcq}
		
	}
	\loigiai
	{	\textbf{Đáp án: A.}
		
		Công của lực lạ:
		$$A=qU=\SI{3}{J}.$$
	}
	\item \mkstar{2}
	
	\cauhoi
	{Trên dây dẫn kim loại có một dòng điện không đổi chạy qua có cường độ $\SI{1.6}{mA}$. Trong 1 phút, số electron chuyển qua một tiết diện thẳng của dây là
		\begin{mcq}(4)
			\item $\SI{6e20}{}$.
			\item $\SI{6e19}{}$.
			\item $\SI{6e18}{}$.
			\item $\SI{6e17}{}$.
		\end{mcq}
		
	}
	\loigiai
	{	\textbf{Đáp án: D.}
		
		Áp dụng công thức:
		$$I=\dfrac{\Delta q}{\Delta t} = \dfrac{ne}{\Delta t} \Rightarrow n = \SI{6e17}{}.$$
	}
	\item \mkstar{2}
	
	\cauhoi
	{Lực lạ thực hiện một công 840 mJ khi dịch chuyển một điện tích $\SI{7e-2}{C}$ giữa hai cực bên trong một nguồn điện. Suất điện động của nguồn điện này là
		\begin{mcq}(4)
			\item 9 V.
			\item 10 V.
			\item 12 V.
			\item 15 V.
		\end{mcq}
		
	}
	\loigiai
	{	\textbf{Đáp án: C.}
		
		Suất điện động của nguồn:
		$$\calE = \dfrac{A}{q} = \SI{12}{V}.$$
	}
	\item \mkstar{2}
	
	\cauhoi
	{Một dòng điện không đổi, sau 2 phút có điện lượng 24 C chuyển qua một tiết diện thẳng. Cường độ dòng điện đó là
		\begin{mcq}(4)
			\item 12 A.
			\item $\SI{0.083}{A}$.
			\item $\SI{0.2}{A}$.
			\item 48 A.
		\end{mcq}
		
	}
	\loigiai
	{	\textbf{Đáp án: C.}
		
		Cường độ dòng điện:
		$$I=\dfrac{\SI{24}{C}}{\SI{120}{s}} = \SI{0.2}{C/s} = \SI{0.2}{A}.$$
	}
	\item \mkstar{2}
	
	\cauhoi
	{Con số $\SI{1.5}{V}$ ghi trên viên pin cho ta biết
		\begin{mcq}(2)
			\item công suất tiêu thụ của viên pin.
			\item điện trở trong của viên pin.
			\item suất điện động của viên pin.
			\item dòng điện viên pin tạo ra.
		\end{mcq}
		
	}
	\loigiai
	{	\textbf{Đáp án: C.}
		
		Con số $\SI{1.5}{V}$ ghi trên viên pin cho ta biết suất điện động của viên pin.
	}
	\item \mkstar{2}
	
	\cauhoi
	{Dòng điện không đổi chạy qua tiết diện của dây dẫn có cường độ $\SI{1.5}{A}$. Trong khoảng thời gian 3 s, điện lượng chuyển qua tiết diện thẳng của dây là
		\begin{mcq}(4)
			\item $\SI{4.5}{C}$.
			\item $\SI{0.5}{C}$.
			\item 2 C.
			\item 4 C.
		\end{mcq}
		
	}
	\loigiai
	{	\textbf{Đáp án: A.}
		
		Điện lượng chuyển qua tiết diện thẳng của dây:
		$$I=\dfrac{\Delta q}{\Delta t} \Rightarrow \Delta q=\SI{4.5}{C}.$$
	}
	\item \mkstar{2}
	
	\cauhoi
	{Một bộ nguồn gồm hai nguồn điện ($\varepsilon_1 = \SI{5}{V}$, $r_1=\SI{3}{\Omega}$, $\varepsilon_2 = \SI{7}{V}$, $r_2=\SI{5}{\Omega}$) mắc nối tiếp. Suất điện động của bộ nguồn là
		\begin{mcq}(4)
			\item 6 V.
			\item 2 V.
			\item 12 V.
			\item 7 V.
		\end{mcq}
		
	}
	\loigiai
	{	\textbf{Đáp án: C.}
		
		Suất điện động của bộ nguồn mắc nối tiếp:
		$$\varepsilon = \varepsilon_1 + \varepsilon_2 = \SI{12}{V}.$$
	}
	\item \mkstar{2}
	
	\cauhoi
	{Dòng electron đập lên màn đèn hình có độ lớn bằng $\SI{200}{\micro A}$. Có bao nhiêu electron đập vào màn hình trong mỗi giây?
		\begin{mcq}(4)
			\item $\SI{8.5e14}{}$.
			\item $\SI{12.5e14}{}$.
			\item $\SI{1.25e14}{}$.
			\item $\SI{2.5e14}{}$.
		\end{mcq}
		
	}
	\loigiai
	{	\textbf{Đáp án: B.}
		
		Số electron đập vào màn hình trong mỗi giây:
		$$I=n|e| \Rightarrow n = \dfrac{I}{|e|} = \SI{12.5e14}{}.$$
	}
	\item \mkstar{2}
	
	\cauhoi
	{Công của lực lạ khi làm dịch chuyển điện lượng $q=\SI{1.5}{C}$ trong nguồn điện từ cực âm đến cực dương của nó là 18 J. Suất điện động của nguồn điện đó là
		\begin{mcq}(4)
			\item $\SI{1.2}{V}$.
			\item $\SI{12}{V}$.
			\item $\SI{2.7}{V}$.
			\item $\SI{27}{V}$.
		\end{mcq}
		
	}
	\loigiai
	{	\textbf{Đáp án: B.}
		
		Suất điện động của nguồn:
		$$\varepsilon = \dfrac{A}{q} = \SI{12}{V}.$$
	}
	\item \mkstar{2}
	
	\cauhoi
	{Một điện lượng bằng $\SI{0.5}{C}$ chạy trong một dây dẫn trong thời gian $\SI{0.5}{s}$. Cường độ dòng điện trong mạch bằng
		\begin{mcq}(4)
			\item $\SI{0.25}{A}$.
			\item $\SI{0.1}{A}$.
			\item 1 A.
			\item $\SI{0.02}{A}$.
		\end{mcq}
		
	}
	\loigiai
	{	\textbf{Đáp án: C.}
		
		Cường độ dòng điện trong mạch:
		$$I=\dfrac{\Delta q}{\Delta t} = \SI{1}{C/s} = \SI{1}{A}.$$
	}
	\item \mkstar{2}
	
	\cauhoi
	{Số electron dịch chuyển qua tiết diện thẳng của dây dẫn trong khoảng thời gian 2 s là $\SI{6.25e18}{}$ hạt. Cho $q_e=\SI{-1.6e-19}{C}$, dòng điện qua dây dẫn có cường độ là
		\begin{mcq}(4)
			\item 2 A.
			\item 1 A.
			\item $\SI{0.5}{A}$.
			\item $\SI{0.512}{A}$.
		\end{mcq}
		
	}
	\loigiai
	{	\textbf{Đáp án: C.}
		
		Cường độ dòng điện:
		$$I=\dfrac{\Delta q}{\Delta t} = \dfrac{n |q_e|}{\Delta t} = \SI{0.5}{A}.$$
	}
	\item \mkstar{2}
	
	\cauhoi
	{Đoạn mạch gồm điện trở $R_1=\SI{100}{\Omega}$ mắc song song với điện trở $R_2=\SI{300}{\Omega}$. Điện trở tương đương toàn mạch là
		\begin{mcq}(4)
			\item $R=\SI{75}{\Omega}$.
			\item $R=\SI{100}{\Omega}$.
			\item $R=\SI{150}{\Omega}$.
			\item $R=\SI{400}{\Omega}$.
		\end{mcq}
		
	}
	\loigiai
	{	\textbf{Đáp án: A.}
		
		Điện trở tương đương toàn mạch của đoạn mạch mắc song song:
		$$R=\dfrac{R_1 R_2}{R_1 + R_2} = \SI{75}{\Omega}.$$
	}
	\item \mkstar{2}
	
	\cauhoi
	{Xét một dòng điện không đổi có cường độ $I$ chạy qua một dây dẫn kim loại. Biết rằng điện lượng chuyển qua tiết diện thẳng của dây dẫn mỗi phút là 150 C. Cường độ dòng điện này là
		\begin{mcq}(4)
			\item $\SI{0.8}{A}$.
			\item $\SI{2.5}{A}$.
			\item $\SI{0.4}{A}$.
			\item $\SI{1.25}{A}$.
		\end{mcq}
		
	}
	\loigiai
	{	\textbf{Đáp án: B.}
		
		Cường độ dòng điện:
		$$I=\dfrac{\Delta q}{\Delta t} = \SI{2.5}{C/s} = \SI{2.5}{A}.$$
	}
	\item \mkstar{3}
	
	\cauhoi
	{Điện tích của electron là $\SI{-1.6e-19}{C}$, điện lượng chuyển qua tiết diện thẳng của dây dẫn trong 30 s là 15 C. Số electron chuyển qua tiết diện thẳng của dây dẫn trong 1 s là
		\begin{mcq}(4)
			\item $\SI{3.125e18}{}$.
			\item $\SI{9.375e19}{}$.
			\item $\SI{7.895e19}{}$.
			\item $\SI{2.632e18}{}$.
		\end{mcq}
		
	}
	\loigiai
	{	\textbf{Đáp án: A.}
		
		Cường độ dòng điện:
		$$I=\dfrac{\Delta q}{\Delta t} = \SI{0.5}{A}.$$
		
		Đây cũng là điện lượng chuyển qua tiết diện thẳng của dây dẫn trong $\SI{1}{s}$. Suy ra số electron là
		$$n=\dfrac{I}{e} = \SI{3.125e18}{}.$$
	}
	\item \mkstar{3}
	
	\cauhoi
	{Một dòng điện không đổi chạy qua dây tóc của một bóng đèn có cường độ $I=\SI{1.6}{mA}$. Tính điện lượng và số electron dịch chuyển qua tiết diện thẳng của dây tóc bóng đèn trong thời gian 5 phút.
		\begin{mcq}(2)
			\item $\SI{0.48}{C}$ và $\SI{3e17}{}$ electron.
			\item $\SI{0.48}{C}$ và $\SI{3e18}{}$ electron.
			\item $\SI{0.28}{C}$ và $\SI{4e17}{}$ electron.
			\item $\SI{0.28}{C}$ và $\SI{4e17}{}$ electron.
		\end{mcq}
		
	}
	\loigiai
	{	\textbf{Đáp án: B.}
		
		Điện lượng chuyển qua trong 5 phút:
		$$\Delta q = I \Delta t = \SI{0.48}{C}.$$
		
		Số electron chuyển qua trong 5 phút:
		$$n=\dfrac{\Delta q}{e} = \SI{3e18}{}.$$
	}
	\item \mkstar{3}
	
	\cauhoi
	{Một dòng điện không đổi có cường độ 3 A thì sau một khoảng thời gian, có điện lượng là 4 C chuyển qua một tiết diện thẳng. Nếu cũng trong khoảng thời gian đó, với dòng điện có cường độ $\SI{4.5}{A}$ thì có điện lượng chuyển qua tiết diện thẳng đó là
		\begin{mcq}(4)
			\item 4 C.
			\item 8 C.
			\item $\SI{4.5}{C}$.
			\item 6 C.
		\end{mcq}
		
	}
	\loigiai
	{	\textbf{Đáp án: D.}
		
		Lập tỉ số:
		$$\dfrac{I_2}{I_2} = \dfrac{q_1}{q_2} \Rightarrow q_2 = \SI{6}{C}.$$
	}
	\item \mkstar{3}
	
	\cauhoi
	{Qua một nguồn điện không đổi, để chuyển một điện lượng 10 C thì lực lạ phải sinh công 20 mJ. Để chuyển một điện lượng 15 C thì lực lạ phải sinh công là
		\begin{mcq}(4)
			\item 10 mJ.
			\item 15 mJ.
			\item 20 mJ.
			\item 30 mJ.
		\end{mcq}
		
	}
	\loigiai
	{	\textbf{Đáp án: D.}
		
		Lập tỉ số:
		$$\dfrac{A_1}{A_2}=\dfrac{q_1}{q_2} \Rightarrow A_2 = \SI{30}{mJ}.$$
	}
	\item \mkstar{4}
	
	\cauhoi
	{Một dòng điện không đổi có cường độ $I=\SI{4.8}{A}$ chạy qua một dây kim loại có tiết diện thẳng $S=\SI{1}{cm^2}$. Tính vận tốc trung bình trong chuyển động định hướng của electron, biết mật độ electron tự do là $n=\SI{3e28}{m^{-3}}$.
		\begin{mcq}(4)
			\item 10 mm/s.
			\item $\SI{0.01}{mm/s}$.
			\item $\SI{0.1}{mm/s}$.
			\item 1 mm/s.
		\end{mcq}
		
	}
	\loigiai
	{	\textbf{Đáp án: B.}
		
		Dựa vào đơn vị của các đại lượng, ta suy ra được công thức để tính vận tốc trung bình $v$:
		$$I=nvqS \Rightarrow v = \SI{0.01}{mm/s}.$$
	}
\end{enumerate}

\whiteBGstarEnd

\loigiai
{
	\begin{center}
		\textbf{BẢNG ĐÁP ÁN}
	\end{center}
	\begin{center}
		\begin{tabular}{|m{2.8em}|m{2.8em}|m{2.8em}|m{2.8em}|m{2.8em}|m{2.8em}|m{2.8em}|m{2.8em}|m{2.8em}|m{2.8em}|}
			\hline
			1.B  & 2.B  & 3.A  & 4.D  & 5.C  & 6.C  & 7.C  & 8.A  & 9.C  & 10.B  \\
			\hline
			11.B  & 12.C  & 13.C  & 14.A  & 15.B  & 16.A  & 17.B  & 18.D  & 19.D  & 20.B  \\
			\hline
		\end{tabular}
	\end{center}
}
\section{Tự luận}
\begin{enumerate}[label=\bfseries Câu \arabic*:]
	\item \mkstar{1}
	
	\cauhoi{
		Nêu định nghĩa của cường độ dòng điện. Ý nghĩa của đơn vị cường độ dòng điện (Ampe).
	}
	
	\loigiai{
		
		Cường độ dòng điện là đại lượng đặc trưng cho tác dụng mạnh, yếu của dòng điện. Nó được xác định bằng thương số của điện lượng $\Delta q$ dịch chuyển qua tiết diện thẳng của vật dẫn trong khoảng thời gian $\Delta t$ và khoảng thời gian $\Delta t$ đó.
		$$I=\dfrac{\Delta q}{\Delta t}.$$
		
		Đơn vị đo cường độ dòng điện là Ampe: 1 Ampe là cường độ dòng điện chạy trong dây dẫn khi có điện lượng là 1 Cu-lông chạy qua tiết diện dây trong 1 giây.
	}
	
	\item \mkstar{2}
	
	\cauhoi{
		Một ắcquy có suất điện động 24 V và điện trở trong $\SI{2}{\Omega}$ mắc vào mạch ngoài có điện trở $R=\SI{6}{\Omega}$. Tính hiệu điện thế mạch ngoài khi mạch hở.
	}
	
	\loigiai{
		
		Khi mạch hở thì hiệu điện thế mạch ngoài là suất điện động của nguồn:
		$$U=\calE = \SI{24}{V}.$$
	}
	\item \mkstar{3}
	
	\cauhoi{
		Một nguồn điện có suất điện động 8 V. Khi mắc nguồn điện này với một bóng đèn thành mạch kín thì nó cung cấp một dòng điện có cường độ $I$, biết công của nguồn điện trong thời gian 10 phút là 10 J. Tính cường độ dòng điện chạy trong mạch kín.
	}
	
	\loigiai{
		
		Công suất của nguồn:
		$$\calP = \dfrac{A}{t} = \dfrac{1}{60}\ \text{W}.$$
		
		Cường độ dòng điện:
		$$I=\dfrac{\calP}{\calE} \approx \SI{2.08}{mA}.$$
	}
	\item \mkstar{4}
	
	\cauhoi{
		Một dây dẫn hình trụ tiết diện ngang $S=\SI{10}{mm^2}$ có dòng điện $I=\SI{2}{A}$ chạy qua. Hạt mang điện trong dây dẫn là electron tự do có điện tích có độ lớn là $e=\SI{1.6e-19}{C}$. Biết vận tốc trung bình của electron trong chuyển động có hướng là $\SI{0.1}{mm/s}$. Tính mật độ hạt electron trong dây dẫn.
	}
	
	\loigiai{
		
		Áp dụng công thức:
		$$I=nvqS \Rightarrow n = \SI{1.25e28}{m^{-3}}.$$
	}
	\item \mkstar{4}
	
	\cauhoi{
		Người ta mắc hai cực của nguồn điện với một biến trở có thể thay đổi từ 0 đến vô cực. Khi giá trị của biến trở rất lớn thì hiệu điện thế giữa hai cực của nguồn điện là $\SI{4.5}{V}$. Giảm giá trị của biến trở đến khi cường độ dòng điện trong mạch là 2 A thì hiệu điện thế giữa hai cực của nguồn điện là 4 V. Tính suất điện động và điện trở trong của nguồn điện.
	}
	
	\loigiai{
		
		Khi giá trị của biến trở rất lớn thì hiệu điện thế giữa hai đầu mạch cũng là suất điện động của nguồn: $$\calE = \SI{4.5}{V}.$$
		
		Áp dụng công thức:
		$$\calE=I_r + U \Rightarrow r = \SI{0.25}{\Omega}.$$
	}
	
\end{enumerate}