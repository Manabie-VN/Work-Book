\whiteBGstarBegin
\setcounter{section}{0}
\section{Trắc nghiệm}
\begin{enumerate}[label=\bfseries Câu \arabic*:]
	
	\item \mkstar{1}
	
	\cauhoi
	{Một ngẫu lực gồm có hai lực $\vec F_1$ và $\vec F_2$ có $F_1 = F_2 = F$ và có cánh tay đòn $d$. Momen ngẫu lực này là
		\begin{mcq}(2)
			\item $(F_1-F_2)d$.
			\item $2Fd$.
			\item $Fd$.
			\item Chưa xác định.
		\end{mcq}
	}
	
	\loigiai
	{	\textbf{Đáp án: C.}
			
	}
	\item \mkstar{1}
	
	\cauhoi
	{Chọn ra vế tiếp theo của câu sau đây: Ngẫu lực là hai lực song song,
		\begin{mcq}
			\item cùng chiều, độ lớn bằng nhau và cùng tác dụng vào một vật.
			\item ngược chiều, độ lớn bằng nhau và cùng tác dụng vào một vật.
			\item cùng chiều, độ lớn bằng nhau và tác dụng vào hai vật khác nhau.
			\item ngược chiều, độ lớn bằng nhau và tác dụng vào hai vật khác nhau.
		\end{mcq}
	}
	
	\loigiai
	{	\textbf{Đáp án: B.}
		
		Ngẫu lực là hai lực song song, ngược chiều, độ lớn bằng nhau và cùng tác dụng vào một vật.
	}
	
	\item \mkstar{2}
	
	\cauhoi
	{Hai lực của một ngẫu lực có độ lớn $F=\SI{5}{N}$. Cánh tay đòn của ngẫu lực $d=\SI{20}{cm}$. Momen của ngẫu lực này là
		\begin{mcq}(4)
			\item $\SI{100}{N.m}$.
			\item $\SI{2}{N.m}$.
			\item $\SI{0.5}{N.m}$.
			\item $\SI{1}{N.m}$.
		\end{mcq}
	}
	
	\loigiai
	{	\textbf{Đáp án: D.}
		
	Momen của ngẫu lực:
	$$M=Fd = \SI{1}{N.m}$$
	}
	\item \mkstar{2}
	
	\cauhoi
	{Hai lực của ngẫu lực có độ lớn $F=\SI{20}{N}$, khoảng cách giữa hai giá của ngẫu lực là $d=\SI{30}{cm}$. Momen của ngẫu lực có độ lớn bằng
			\begin{mcq}(4)
		\item $\SI{0.6}{N.m}$.
		\item $\SI{600}{N.m}$.
		\item $\SI{6}{N.m}$.
		\item $\SI{60}{N.m}$.
	\end{mcq}
}

\loigiai
{	\textbf{Đáp án: C.}

Momen của ngẫu lực:
$$M=Fd = \SI{6}{N.m}$$
}
	\item \mkstar{2}

\cauhoi
{Hai lực của ngẫu lực có độ lớn $F=\SI{40}{N}$, khoảng cách giữa hai giá của ngẫu lực là $d=\SI{30}{cm}$. Momen của ngẫu lực có độ lớn bằng
	\begin{mcq}(4)
		\item $\SI{18}{N.m}$.
		\item $\SI{40}{N.m}$.
		\item $\SI{10}{N.m}$.
		\item $\SI{12}{N.m}$.
	\end{mcq}
}

\loigiai
{	\textbf{Đáp án: D.}
	
	Momen của ngẫu lực:
	$$M=Fd = \SI{12}{N.m}$$
}
\end{enumerate}

\whiteBGstarEnd

\loigiai
{
	\begin{center}
		\textbf{BẢNG ĐÁP ÁN}
	\end{center}
	\begin{center}
		\begin{tabular}{|m{2.8em}|m{2.8em}|m{2.8em}|m{2.8em}|m{2.8em}|m{2.8em}|m{2.8em}|m{2.8em}|m{2.8em}|m{2.8em}|}
			\hline
			1.C  & 2.B  & 3.D  & 4.C  & 5.D  & & & & &  \\
			\hline
			
		\end{tabular}
	\end{center}
}
\section{Tự luận}
\begin{enumerate}[label=\bfseries Câu \arabic*:]
	\item \mkstar{1}
	
	\cauhoi{
		Ngẫu lực là gì? Nêu tác dụng của ngẫu lực đối với một vật rắn.
	}
	
	\loigiai{
		Hệ hai lực song song, ngược chiều, có độ lớn bằng nhau cùng tác dụng vào một vật gọi là ngẫu lực.
		
		Tác dụng của ngẫu lực:
		\begin{itemize}
			\item Trường hợp vật không có trục quay cố định: Ngẫu lực sẽ làm cho vật quay quanh trọng tâm. Nếu có trục quay đi qua trọng tâm thì trục quay này không chịu tác dụng lực.
			\item Trường hợp vật có trục quay cố định: Ngẫu lực làm cho vật quay quanh trục cố định. Trọng tâm vật cũng quay quanh trục quay, gây ra lực tác dụng lên trục quay đó, có thể làm cho trục quay biến dạng.
		\end{itemize}
	}
	
	\item \mkstar{2}
	
	\cauhoi
	{Hai lực của một ngẫu lực có độ lớn  $F=\SI{5.0}{N}$ Cánh tay đòn của ngẫu lực $d=\SI{20}{cm}$. Tính momen của ngẫu lực.
	}
	
	\loigiai
	{
	Momen của ngẫu lực:
	$$M=Fd = \SI{1}{N.m}$$
	}
	\item \mkstar{3}
	
	\cauhoi
	{Chứng minh rằng momen của ngẫu lực không phụ thuộc vào vị trí của trục quay vuông góc với mặt phẳng chứa ngẫu lực.
	}
	
	\loigiai
	{Gọi O là vị trí của trục quay. Ta có momen của ngẫu lực:
		$$M=F_1d_1 + F_2d_2 = F (d_1 + d_2) = Fd$$
		
	Momen ngẫu lực chỉ phụ thuộc vào $d$ là khoảng cách giữa hai giá của hai lực, không phụ thuộc vào vị trí O của trục quay.
	}
	\item \mkstar{4}
	
	\cauhoi
	{Một vật rắn phẳng, mỏng, có dạng là một tam giác đều ABC, mỗi cạnh là $a=\SI{10}{cm}$. Người ta tác dụng một ngẫu lực nằm trong mặt phẳng của tam giác. Biết các lực vuông góc với cạnh AC có độ lớn $\SI{10}{N}$ và đặt vào hai đỉnh của A và B. Tính momen của ngẫu lực.
	}
	
	\loigiai
	{Ta có đường cao của tam giác đều cũng chính là đường trung trực, nên:
		$$d=\dfrac{\text{AC}}{2} = \SI{5}{cm} = \SI{0.05}{m}$$
		
	Momen ngẫu lực:
	$$M=Fd = \SI{0.5}{Nm}$$
	}
	\item \mkstar{4}

\cauhoi
{Một vật rắn phẳng, mỏng, có dạng là một hình vuông ABCD, mỗi cạnh là $a=\SI{10}{cm}$. Người ta tác dụng một ngẫu lực nằm trong mặt phẳng của hình vuông. Biết các lực vuông góc với đường chéo AD có độ lớn $\SI{10}{N}$ và đặt vào hai đỉnh của A và D. Tính momen của ngẫu lực.
}

\loigiai
{Ta có đường chéo của hình vuông:
	$$d=\sqrt{a^2 + a^2} = \SI{14.14}{cm} = \SI{0.14}{m}$$
	
	Momen ngẫu lực:
	$$M=Fd = \SI{1.41}{Nm}$$
}
\end{enumerate}