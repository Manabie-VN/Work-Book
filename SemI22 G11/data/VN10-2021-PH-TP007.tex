\whiteBGstarBegin
\setcounter{section}{0}
\section{Trắc nghiệm}
\begin{enumerate}[label=\bfseries Câu \arabic*:]
	
		\item \mkstar{2}
	
	\cauhoi
	{Hành khách A đứng trên toa tàu, nhìn qua cửa sổ thấy hành khách B ở toa tàu bên cạnh. Hai toa tàu đang đỗ trên hai đường tàu trong sân ga. Bỗng A thấy B chuyển động về phía sau. Tình huống nào sau đây chắc chắn \textbf{không} xảy ra?
		\begin{mcq}
			\item Cả hai toa tàu cùng chạy về phía trước, B chạy nhanh hơn.
			\item Cả hai toa tàu cùng chạy về phía trước, A chạy nhanh hơn.
			\item Toa A chạy về phía trước, toa B đứng yên.
			\item Toa A đứng yên, toa B chạy về phía sau.
		\end{mcq}
		
	}
	\loigiai
	{	\textbf{Đáp án: A.}
		
	Chỉ có 1 khả năng không xảy ra là cả hai toa tàu cùng chạy về phía trước, B chạy nhanh hơn.
	}

	\item \mkstar{2}
	
	\cauhoi
	{Một chiếc thuyền đi trong nước yên lăng với vận tốc có độ lớn $v_1$, vận tốc dòng chảy của nước so với bờ sông có độ lớn $v_2$. Nếu người lái thuyền hướng mũi thuyền dọc theo dòng nước từ hạ nguồn lên thượng nguồn con sông thì người đứng trên bờ sẽ thấy như thế nào?
		\begin{mcq}
			\item Thuyền trôi về phía thượng nguồn nếu $v_1>v_2$.
			\item Thuyền trôi về phía hạ lưu nếu $v_1>v_2$.
			\item Thuyền đứng yên nếu $v_1<v_2$.
			\item Thuyền trôi về phía hạ lưu nếu $v_1=v_2$.
		\end{mcq}
		
	}
	\loigiai
	{	\textbf{Đáp án: A.}
	
	Nếu người lái thuyền hướng mũi thuyền dọc theo dòng nước từ hạ nguồn lên thượng nguồn con sông thì người đứng trên bờ sẽ thấy thuyền trôi về phía thượng nguồn nếu $v_1>v_2$.
	}
	\item \mkstar{3}
	
	\cauhoi
	{Khi nước yên lặng, một người bơi với tốc độ $\SI{4}{km/h}$. Khi bơi xuôi dòng từ A đến B mất 30 phút và ngược dòng từ B đến A mất 48 phút. A và B cách nhau
		\begin{mcq}(4)
			\item $\SI{2.46}{km}$.
			\item $\SI{4.32}{km}$.
			\item $\SI{2.78}{km}$.
			\item $\SI{1.98}{km}$.
		\end{mcq}
		
	}
	\loigiai
	{	\textbf{Đáp án: A.}
		
	Khi bơi xuôi dòng, $t_1=\SI{0.5}{h}$:
	$$t_1 = \dfrac{s}{v_\text{người} + v_\text{nước}}$$
	
	Khi bơi ngược dòng, $t_2=\SI{0.8}{h}$:
	$$t_2=\dfrac{s}{v_\text{người} - v_\text{nước}}$$
	
	Suy ra $v_\text{nước} = \SI{0.923}{km/h}$, $s=(v_\text{người} + v_\text{nước}) \cdot t_1 = \SI{2.46}{km}$.
	}
	
	

	\item \mkstar{3}
	
	\cauhoi
	{Một chiếc phà chạy xuôi dòng từ A đến B mất 3 giờ, khi chạy về mất 6 giờ. Nếu phà tắt máy trôi theo dòng nước từ A đến B thì mất
		\begin{mcq}(4)
			\item 13 giờ.
			\item 12 giờ.
			\item 11 giờ.
			\item 10 giờ.
		\end{mcq}
	}
	\loigiai
	{	\textbf{Đáp án: B.}
		
	Lập tỉ số:
	$$\dfrac{t_1}{t_2} = \dfrac{1}{2} = \dfrac{v_\text p - v_\text n}{v_\text p + v_\text n} \Rightarrow \dfrac{v_\text p}{v_\text n} = \dfrac{3}{1}$$
	
	Tiếp tục lập tỉ số:
	$$\dfrac{t_1}{t_3} = \dfrac{v_\text n}{v_\text p + v_\text n} = \dfrac{1}{4}$$
	
	Vậy $t_3=12\ \text h$.
	}

	\item \mkstar{4}
	
	\cauhoi
	{Từ hai bến trên bờ sông, một ca nô và một chiếc thuyền chèo đồng thời khởi hành theo hướng ngược nhau. Sau khi gặp nhau, chiếc ca nô quay ngược lại, còn chiếc thuyền thả trôi sông. Kết quả là thuyền và ca nô trở về vị trí xuất phát cùng lúc. Biết rằng tỉ số vận tốc giữa vận tốc của ca nô với vận tốc dòng chảy là 10. Tỉ số vận tốc giữa vận tốc của thuyền khi chèo và vận tốc của dòng chảy là (chọn đáp án gần đúng nhất)
		\begin{mcq}(4)
			\item $\SI{3.1}{}$.
			\item $\SI{2.5}{}$.
			\item $\SI{2.2}{}$.
			\item $\SI{2.8}{}$.
		\end{mcq}
	}
	\loigiai
	{	\textbf{Đáp án: C.}
		
	Để lúc sau chiếc thuyền thả trôi sông quay về được điểm xuất phát ban đầu thì lúc đầu, thuyền chèo ngược dòng nước.
	
	Vận tốc thuyền lúc đầu:
	$$v_1=v_\text t - v_\text n$$
	
	Vận tốc ca nô lúc đầu:
	$$v_2 = v_\text c + v_\text n$$
	
	Khi gặp nhau:
	$$t_1 = t_2 \Rightarrow \dfrac{s_1}{v_1} = \dfrac{s_2}{v_2} \Rightarrow \dfrac{s_1}{v_\text t - v_\text n} = \dfrac{s_2}{v_\text c + v_\text n}$$
	
	Vận tốc thuyền lúc sau:
	$$v_1' = v_\text n$$
	
	Vận tốc ca nô lúc sau:
	$$v_2' = v_\text c - v_\text n$$
	
	Cả hai trở về điểm xuất phát cùng lúc:
	$$t_1' = t_2' \Rightarrow \dfrac{s_1}{v_1'} = \dfrac{s_2}{v_2'} \Rightarrow \dfrac{s_1}{v_\text n} = \dfrac{s_2}{v_\text c - v_\text n}$$
	
	Mà theo dữ kiện thì $v_\text c = 10 v_\text n$. Suy ra:
	
	$$\dfrac{s_1}{v_\text t - v_\text n} = \dfrac{s_2}{11 v_\text n}$$
	
	$$\dfrac{s_1}{v_\text n} = \dfrac{s_2}{9 v_\text n}$$
	
	$$\Rightarrow \dfrac{v_\text n}{v_\text t - v_\text n} = \dfrac{9 v_\text n}{11 v_\text n} \Rightarrow \dfrac{v_\text t}{v_\text n} = \SI{2.2222}{}$$
	}
\end{enumerate}

\whiteBGstarEnd

\loigiai
{
	\begin{center}
		\textbf{BẢNG ĐÁP ÁN}
	\end{center}
	\begin{center}
		\begin{tabular}{|m{2.8em}|m{2.8em}|m{2.8em}|m{2.8em}|m{2.8em}|m{2.8em}|m{2.8em}|m{2.8em}|m{2.8em}|m{2.8em}|}
			\hline
			1.A  & 2.A  & 3.A  & 4.B  & 5.C  & & & &  &   \\
			\hline
			
		\end{tabular}
	\end{center}
}
\section{Tự luận}
\begin{enumerate}[label=\bfseries Câu \arabic*:]
	\item \mkstar{1}
	
	\cauhoi{
		Trình bày công thức cộng vận tốc trong trường hơp các chuyển động cùng phương, cùng chiều; cùng phương, ngược chiều.
	}
	
	\loigiai{
		
	Công thức cộng vận tốc:
	$$\overrightarrow{v_{13}} =\overrightarrow{v_{12}}+\overrightarrow{v_{23}}, $$
	trong đó:
\begin{itemize}
	\item $\overrightarrow{v_{13}}$ là vận tốc tuyệt đối;
	\item $\overrightarrow{v_{12}}$ là vận tốc tương đối;
	\item $\overrightarrow{v_{23}}$ là vận tốc kéo theo.
\end{itemize}

	Trường hợp các vật chuyển động cùng phương, cùng chiều:
	$$v_{13} = v_{12}+v_{23}$$
	
	Trường hợp các vật chuyển động cùng phương, ngược chiều:
	$$v_{13} = |v_{12} - v_{23}|$$
	}
	
	\item \mkstar{2}
	
	\cauhoi{Một ô tô A chạy đều trên một đường thẳng với vận tốc $\SI{40}{km/h}$. Một ô tô B đuổi theo ô tô A với vận tốc $\SI{60}{km/h}$. Xác định vận tốc của ô tô B đối với ô tô A và của ô tô A đối với ô tô B.}
	\loigiai{
	Chọn chiều dương là chiều chuyển động của hai xe.
	
	$\overrightarrow{v_\text{AD}}$ là vận tốc của xe A đối với đất;
	
	$\overrightarrow{v_\text{BD}}$ là vận tốc của xe B đối với đất;
	
	$\overrightarrow{v_\text{BA}}$ là vận tốc của xe B đối với xe A.
	
	Theo công thức cộng vận tốc:
	$$\overrightarrow{v_\text{AB}} = \overrightarrow{v_\text{AD}}+\overrightarrow{v_\text{DB}} \Rightarrow \overrightarrow{v_\text{AB}} = \overrightarrow{v_\text{AD}} - \overrightarrow{v_\text{BD}}$$
	
	Do hai xe chuyển động ngược chiều nên
	$$v_\text{AB} = 40-60=-20\ \text{km/h}$$
	
	Vậy $v_\text{BA} = 20\ \text{km/h}$.
	}
	\item \mkstar{3}
	
	\cauhoi
	{A ngồi trên một toa tàu chuyển động với vận tốc $\SI{15}{km/h}$ đang rời ga. B ngồi trên một toa tàu khác đang chuyển động với vận tốc $\SI{10}{km/h}$ đang đi ngược chiều vào ga. Hai đường tàu song song với nhau. Tính vận tốc của B đối với A.
	}
	\loigiai
	{Chọn chiều dương là chiều chuyển động của tàu A.
		
		$\overrightarrow{v_\text{AD}}$ là vận tốc của tàu A đối với đất;
		
		$\overrightarrow{v_\text{BD}}$ là vận tốc của tàu B đối với đất;
		
		$\overrightarrow{v_\text{BA}}$ là vận tốc của tàu B đối với tàu A.
		
		Theo công thức cộng vận tốc:
		$$\overrightarrow{v_\text{BD}} = \overrightarrow{v_\text{BA}}+\overrightarrow{v_\text{AD}} \Rightarrow \overrightarrow{v_\text{BA}} = \overrightarrow{v_\text{BD}} - \overrightarrow{v_\text{AD}}=\overrightarrow{v_\text{BD}} + \overrightarrow{-v_\text{AD}}$$
		
		Do A và B chuyển động ngược chiều nên
		$$v_\text{AB} = v_\text{BD} + v_\text{DA} = -10-15 = \SI{-25}{km/h}$$
		
		Vận tốc của tàu B đối với tàu A có độ lớn $\SI{25}{km/h}$ và ngược chiều so với chiều chuyển động của tàu A.
	}
	\item \mkstar{4}
	
	\cauhoi
	{Một hành khách ngồi ở cửa sổ một chiếc tàu A đang chạy trên đường ray với vận tốc $v_1=\SI{72}{km/h}$, nhìn chiếc tàu B chạy ngược chiều ở đường ray bên cạnh qua một thời gian nào đó. Nếu tàu B chạy cùng chiều, thì người khách đó nhận thấy thời gian tàu B chạy qua mình lâu gấp 3 lần so với trường hợp trước. Tính vận tốc của tàu B.
	}
	\loigiai
	{Thời gian tàu B chạy qua cùng chiều lâu gấp 3 lần thời gian tàu B chạy qua ngược chiều tàu A:
		$$\dfrac{t_1}{t_3} = 3 \Rightarrow \dfrac{v_3}{v_1} = 3 \Rightarrow \dfrac{|v_\text A - v_\text B|}{v_\text{A}+v_\text B = 3} \Rightarrow v_\text B = \SI{-36}{km/h}; v_\text B = \SI{-144}{km/h}$$
		
	Vậy vận tốc của tàu B là $\SI{36}{km/h}$ hoặc $\SI{144}{km/h}$ và ngược chiều với vận tốc của tàu A.
	}
	\item \mkstar{4}
	
	\cauhoi
	{Trong một siêu thị, người ta có đặt hệ thống cầu thang cuốn để đưa hành khách lên. Khi hành khách đứng yên trên thang thì thời gian thang cuốn đưa lên là $t_1=1\ \text{phút}$. Khi thang đứng yên, thì hành khách đi lên cầu thang này phải đi bộ và mất một khoảng thời gian là $t_2=3\ \text{phút}$. Nếu hành khách vừa đi lên thang cuốn trong khi thang đang hoạt động thì tiêu tốn một khoảng thời gian là bao nhiêu?
	}
	\loigiai
	{Thời gian đi bộ gấp 3 lần thời gian thang cuốn đưa lên:
		$$\dfrac{t_2}{t_1} = 3 \Rightarrow \dfrac{v_1}{v_2} = 3$$
		
	Lập tỉ số:
	$$\dfrac{t_1}{t_3} = \dfrac{v_3}{v_1} = \dfrac{v_1+v_2}{v_1} = \dfrac{4/3 v_1}{v_1} = \dfrac{4}{3}$$
	
	Vậy nếu hành khách vừa đi lên thang cuốn trong khi thang đang hoạt động thì tiêu tốn một khoảng thời gian là $t_3=\SI{45}{s}$.
	}
\end{enumerate}