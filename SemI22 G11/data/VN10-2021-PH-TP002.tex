\whiteBGstarBegin
\setcounter{section}{0}
\section{Trắc nghiệm}
\begin{enumerate}[label=\bfseries Câu \arabic*:]
	\item \mkstar{1}
	
	\cauhoi
	{Trường hợp nào sau đây có thể coi vật là chất điểm?
		\begin{mcq}
			\item Trái Đất trong chuyển động tự quay quanh mình nó. 
			\item Hai hòn bi lúc va chạm với nhau. 
			\item Hai người nhảy cầu lúc đang rơi xuống nước. 
			\item Giọt nước mưa lúc đang rơi. 
		\end{mcq}
	}
	
	\loigiai
	{		\textbf{Đáp án: D.}
		
	Giọt nước mưa lúc đang rơi có thể coi như là chất điểm.
		
	}
\item \mkstar{1}

\cauhoi
{Một người ngồi trên xe đi từ TP HCM ra Đà Nẵng, nếu lấy vật làm mốc là tài xế đang lái xe thì vật chuyển động là
	\begin{mcq}
		\item xe ô tô. 
		\item cột đèn bên đường. 
		\item bóng đèn trên xe. 
		\item hành khách đang ngồi trên xe. 
	\end{mcq}
}

\loigiai
{		\textbf{Đáp án: B.}
	
	Một người ngồi trên xe đi từ TP HCM ra Đà Nẵng, nếu lấy vật làm mốc là tài xế đang lái xe thì vật chuyển động là cột đèn bên đường.
	
}
	\item \mkstar{2}
	
	\cauhoi
	{Một người đứng bên đường quan sát chiếc ô tô chạy qua trước mặt. Dấu hiệu nào cho biết ô tô đang chuyển động? 
		\begin{mcq}
			\item Khói phụt ra từ ống thoát khí đặt dưới gầm xe. 
			\item Khoảng cách giữa xe và người đó thay đổi. 
			\item Bánh xe quay tròn. 
			\item Tiếng nổ của động cơ vang lên. 
		\end{mcq}
	}
	
	\loigiai
	{\textbf{Đáp án: B.}
		
	Một người đứng bên đường quan sát chiếc ô tô chạy qua trước mặt. Dấu hiệu cho biết ô tô đang chuyển động: Khoảng cách giữa xe và người đó thay đổi. 
		
		
	}
		\item \mkstar{2}
	
	\cauhoi
	{Một người chỉ cho một người khách du lịch như sau: "Ông hãy đi dọc theo phố này đến một bờ hồ lớn. Đứng tại đó, nhìn sang bên kia hồ theo hướng Tây Bắc, ông sẽ thấy tòa nhà của khách sạn S." Người chỉ đường đã xác định vị trí của khách sạn theo cách nào?
		\begin{mcq}
			\item Cách dùng đường đi và vật làm mốc. 
			\item Cách dùng các trục tọa độ. 
			\item Dùng cả hai cách A và B. 
			\item Không dùng cả hai cách A và B. 
		\end{mcq}
	}
	
	\loigiai
	{\textbf{Đáp án: C.}
		
	Đi dọc theo phố này đến một bờ hồ lớn là cách dùng đường đi và vật làm mốc.
	
	Đứng ở bờ hồ, nhìn sang hướng Tây Bắc là cách dùng các trục tọa độ.
		
		
	}
	\item \mkstar{2}
	
	\cauhoi
	{Trong các cách chọn hệ trục tọa độ và mốc thời gian dưới đây, cách nào là thích hợp nhất để xác định vị trí của một máy bay đang bay trên đường dài?
		\begin{mcq}
			\item Khoảng cách đến ba sân bay lớn; $t=0$ là lúc máy bay cất cánh.
			\item Khoảng cách đến ba sân bay lớn; $t=0$ là lúc 0 giờ quốc tế.
			\item Kinh độ, vĩ độ địa lý và độ cao của máy bay; $t=0$ là lúc máy bay cất cánh.
			\item Kinh độ, vĩ độ địa lý và độ cao của máy bay; $t=0$ là lúc 0 giờ quốc tế.
		\end{mcq}
	}
	
	\loigiai
	{\textbf{Đáp án: D.}		
		
	}
	
\end{enumerate}

\whiteBGstarEnd

\loigiai
{
	\begin{center}
		\textbf{BẢNG ĐÁP ÁN}
	\end{center}
	\begin{center}
		\begin{tabular}{|m{2.8em}|m{2.8em}|m{2.8em}|m{2.8em}|m{2.8em}|m{2.8em}|m{2.8em}|m{2.8em}|m{2.8em}|m{2.8em}|}
			\hline
			1.D  & 2.B  & 3.B  & 4.C  & 5.D  &   &  &  &  &  \\
			\hline
			
		\end{tabular}
	\end{center}
}
\section{Tự luận}
\begin{enumerate}[label=\bfseries Câu \arabic*:]
\item \mkstar{1}

\cauhoi{
Hãy cho biết các khái niệm sau: chuyển động cơ, chất điểm, quỹ đạo chuyển động, vật mốc, hệ tọa độ, mốc thời gian.
}

\loigiai{
	\begin{itemize}
		\item Chuyển đông cơ là sự thay đổi vị trí của một vật so với các vật khác theo thời gian;
		\item Chất điểm là vật chuyển động có kích thước rất nhỏ so với độ dài đường đi (hoặc so với những khoảng cách mà ta đề cập đến);
		\item Quỹ đạo chuyển động là tập hợp tất cả các vị trí của một chất điểm chuyển động;
		\item Vật mốc là vật mà ta chọn để xác định được chính xác vị trí của vật bằng cách dùng một thước đo chiều dài từ vật làm mốc đến vật;
		\item Hệ tọa độ là hệ trục tọa độ vuông góc mà ta chọn để xác định được chính xác vị trí của vật trong không gian, hệ tọa độ luôn gắn với vật mốc;
		\item Mốc thời gian là thời điểm mà ta bắt đầu đo thời gian.
	\end{itemize}
}

\item \mkstar{2}

\cauhoi{Hãy chỉ ra sự khác nhau giữa hệ tọa độ và hệ quy chiếu.}

\loigiai{
Hệ tọa độ gồm các trục tọa độ và vật mốc. Khi các trục vuông góc với nhau thì ta gọi đó là hệ tọa độ Đề-các.

Hệ quy chiếu bao gồm vật mốc, hệ tọa độ gắn với vật mốc, gốc thời gian và đồng hồ.
}
\item \mkstar{3}

\cauhoi
{Hãy cho biết các tọa độ của điểm M nằm chính giữa một bức tường hình chữ nhật ABCD có cạnh AB = 5 m, AD = 4 m. Lấy trục $\text O x$ dọc theo AB, trục $\text O y$ dọc theo AD.
}

\loigiai
{Điểm M nằm chính giữa bức tường hình chữ nhật ABCD nên M là giao điểm của hai đường chéo của hình chữ nhật.
	
	Tọa độ điểm M: $x_\text M = \dfrac{\text{AB}}{2} = \SI{2.5}{m}$, $y_\text M = \dfrac{\text{AD}}{2} = \SI{2}{m}$
}
\item \mkstar{4}

\cauhoi
{Nếu lấy mốc thời gian là lúc 5 giờ 15 phút thì sau ít nhất bao lâu, kim phút đuổi kịp kim giờ?
}

\loigiai
{
Kim phút đuổi kịp kim giờ khi hai kim trùng nhau.

Ta biết, khi kim phút quay được một vòng thì kim giờ quay được 1/12 vòng. Vậy hiệu vận tốc của hai kim là $1-1/12=11/12$ vòng.

Vào lúc 5 giờ 15 phút, kim giờ cách mốc thứ 5 là $1/4 \cdot 1/12 = 1/48$ vòng.

Khoảng cách giữa kim phút và kim giờ là $2/12+1/48=9/48$ vòng.

Thời gian để kim phút đuổi kịp kim giờ là $(9/48 :11/12) \cdot 60=$ 12 phút 16 giây.
}
\item \mkstar{4}

\cauhoi
{Hai ô tô chuyển động thẳng đều, khởi hành đồng thời từ 2 địa điểm cách nhau 20 km. Nếu đi ngược chiều nhau thì sau 15 phút chúng gặp nhau. Nếu đi cùng chiều thì sau 30 phút chúng đuổi kịp nhau. Tìm vận tốc của hai xe đó.
}

\loigiai
{	Khi đi ngược chiều, hai xe gặp nhau khi
	$$s = s_1 + s_2 \Rightarrow \SI{20}{km} = v_1 t_1 + v_2 t_2$$
	
	Khi đi cùng chiều, hai xe gặp nhau khi
	$$s=|s_1 - s_2| \Rightarrow \SI{20}{km} = |v_1 t_1' - v_2t_2'|$$
	
	Giải hệ trên, ta được $\SI{20}{km/h}$ và $\SI{60}{km/h}$.
}
\end{enumerate}