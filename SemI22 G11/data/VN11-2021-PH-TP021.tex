\whiteBGstarBegin
\setcounter{section}{0}
\section{Trắc nghiệm}
\begin{enumerate}[label=\bfseries Câu \arabic*:]
	
	
	\item \mkstar{1}
	
	\cauhoi
	{Môi trường nào dưới đây \textbf{không} chứa điện tích tự do?
		\begin{mcq}(2)
			\item Nước sông.
			\item Nước biển.
			\item Nước mưa.
			\item Nước cất.
		\end{mcq}
		
	}
	\loigiai
	{	\textbf{Đáp án: D.}
		
		Nước cất không chứa các điện tích tự do.
	}
	\item \mkstar{1}
	
	\cauhoi
	{Chọn câu đúng nhất. Bản chất dòng điện trong chất điện phân là
		\begin{mcq}
			\item dòng ion dương và dòng ion âm chuyển động có hướng theo hai chiều ngược nhau.
			\item dòng ion dương dịch chuyển theo chiều điện trường.
			\item dòng electron dịch chuyển ngược chiều điện trường.
			\item dòng ion âm dịch chuyển ngược chiều điện trường.
		\end{mcq}
		
	}
	\loigiai
	{	\textbf{Đáp án: A.}
		
		Bản chất dòng điện trong chất điện phân là dòng ion dương và dòng ion âm chuyển động có hướng theo hai chiều ngược nhau.
	}
	\item \mkstar{2}
	
	\cauhoi
	{Một bình điện phân đựng dung dịch $\ce{AgNO_3}$, cường độ dòng điện chạy qua bình điện phân là $I=\SI{1}{A}$. Cho $A_\text{Ag} = \SI{108}{}$, $n_\text{Ag} = 1$. Lượng $\ce{Ag}$ bám vào cathode trong thời gian 16 phút 5 giây là
		\begin{mcq}(4)
			\item $\SI{1.08}{g}$.
			\item $\SI{0.54}{g}$.
			\item $\SI{1.08}{mg}$.
			\item $\SI{0.54}{mg}$.
		\end{mcq}
		
	}
	\loigiai
	{	\textbf{Đáp án: A.}
		
		Theo định luật Fa-ra-đây:
		$$m=\dfrac{1}{F} \dfrac{A}{n} It = \SI{1.08}{g}.$$
	}
	\item \mkstar{2}
	
	\cauhoi
	{Khi điện phân dương cực tan, nếu tăng cường độ dòng điện và thời gian điện phân lên 2 lần thì khối lượng chất giải phóng ra ở điện cực
		\begin{mcq}(2)
			\item giảm 4 lần.
			\item tăng 4 lần.
			\item không đổi.
			\item tăng 2 lần.
		\end{mcq}
		
	}
	\loigiai
	{	\textbf{Đáp án: B.}
		
		Theo định luật Fa-ra-đây:
		$$m=\dfrac{1}{F} \dfrac{A}{n} I t$$
		
		Nếu $I$ tăng 2 lần, $t$ tăng 2 lần thì $m$ tăng 4 lần.
	}
	\item \mkstar{2}
	
	\cauhoi
	{Đặt một hiệu điện thế $U$ không đổi vào hai cực của bình điện phân. Xét trong cùng một khoảng thời gian, nếu kéo hai cực của bình ra xa để cho khoảng cách giữa chúng tăng 2 lần thì khối lượng chất được giải phóng ra ở điện cực sẽ
		\begin{mcq}(2)
			\item tăng 2 lần.
			\item giảm 2 lần.
			\item tăng 4 lần.
			\item giảm 4 lần.
		\end{mcq}
		
	}
	\loigiai
	{	\textbf{Đáp án: B.}
		
		Khi tăng khoảng cách giữa 2 điện cực lên 2 lần thì điện trở của bình điện phân tăng 2 lần, dẫn đến cường độ dòng điện $I$ qua bình giảm 2 lần.
		
		Vậy $m$ giảm 2 lần.
	}
	\item \mkstar{2}
	
	\cauhoi
	{Một bình điện phân dung dịch $\ce{CuSO_4}$ có anode làm bằng đồng, điện trở của bình điện phân $R=\SI{8}{\Omega}$, được mắc vào hai cực của bộ nguồn $\calE = \SI{9}{V}$, $r=\SI{1}{\Omega}$. Khối lượng đồng bám vào cathode sau thời gian 5 giờ là
		\begin{mcq}(4)
			\item $\SI{5}{g}$.
			\item $\SI{10.5}{g}$.
			\item $\SI{5.97}{g}$.
			\item $\SI{11.94}{g}$.
		\end{mcq}
		
	}
	\loigiai
	{	\textbf{Đáp án: C.}
		
		Cường độ dòng điện tính theo định luật Ôm toàn mạch:
		$$I=\dfrac{\calE}{R + r} = \SI{1}{A}.$$
		
		Áp dụng định luật Fa-ra-đây:
		$$m=\dfrac{1}{F} \dfrac{A}{n} I t = \SI{5.97}{g}.$$
	}
	\item \mkstar{2}
	
	\cauhoi
	{Một bình điện phân chứa dung dịch bạc nitrate có điện trở là $\SI{5}{g}$. Anode của bình bằng bạc và hiệu điện thế đặt vào hai điện cực của bình là $\SI{20}{V}$. Tính khối lượng bạc bám vào cathode sau 32 phút 10 giây.
		\begin{mcq}(4)
			\item $\SI{8.64}{g}$.
			\item $\SI{4.64}{g}$.
			\item $\SI{3.55}{g}$.
			\item $\SI{8.44}{g}$.
		\end{mcq}
		
	}
	\loigiai
	{	\textbf{Đáp án: A.}
		
		Cường độ dòng điện qua bình điện phân:
		$$I=\dfrac{U}{R} = \SI{4}{A}.$$
		
		Áp dụng định luật Fa-ra-đây:
		$$m=\dfrac{1}{F} \dfrac{A}{n} I t = \SI{8.64}{g}.$$
	}
	\item \mkstar{2}
	
	\cauhoi
	{Một bình điện phân dung dịch $\ce{CuSO_4}$ có anode làm bằng đồng, điện trở của bình điện phân $R=\SI{8}{\Omega}$, được mắc vào hai cực của bộ nguồn $\calE = \SI{9}{V}$, $r=\SI{1}{\Omega}$. Cần điện phân trong bao lâu để có một lượng $\SI{5.97}{g}$ $\ce{Cu}$ bám vào cathode?
		\begin{mcq}(2)
			\item 4 giờ 30 phút.
			\item 5 giờ 30 phút.
			\item 5 giờ.
			\item 4 giờ.
		\end{mcq}
		
	}
	\loigiai
	{	\textbf{Đáp án: C.}
		
		Cường độ dòng điện tính theo định luật Ôm toàn mạch:
		$$I=\dfrac{\calE}{R + r} = \SI{1}{A}.$$
		
		Áp dụng định luật Fa-ra-đây:
		$$m=\dfrac{1}{F} \dfrac{A}{n} I t \Rightarrow t = \SI{18003}{s} \approx \SI{5}{h}.$$
	}
	\item \mkstar{2}
	
	\cauhoi
	{Cho dòng điện chạy qua bình điện phân đựng dung dịch của muối nikel, có anode làm bằng nickel. Biết nguyên tử khối và hóa trị của nickel lần lượng là $\SI{58.71}{g/mol}$ và $2$. Trong thời gian 1 giờ, dòng điện $\SI{10}{A}$ đã sản ra một khối lượng nickel bằng
		\begin{mcq}(4)
			\item $\SI{8}{g}$.
			\item $\SI{10.95}{g}$.
			\item $\SI{12.35}{g}$.
			\item $\SI{15.27}{g}$.
		\end{mcq}
		
	}
	\loigiai
	{	\textbf{Đáp án: B.}
		
		Áp dụng định luật Fa-ra-đây:
		$$m=\dfrac{1}{F} \dfrac{A}{n} I t = \SI{10.95}{g}.$$
	}
	\item \mkstar{2}
	
	\cauhoi
	{Một bình điện phân đựng dung dịch bạc nitrate với anode làm bằng bạc. Điện trở của bình điện phân là $R=\SI{2}{\Omega}$. Hiệu điện thế đặt vào hai cực của nguồn là $U=\SI{10}{V}$. Cho $A=\SI{108}{g/mol}$ và $n=1$. Khối lượng bạc bám vào cực âm sau 2 giờ là
		\begin{mcq}(4)
			\item $\SI{40.3}{g}$.
			\item $\SI{40.3}{kg}$.
			\item $\SI{8.04}{g}$.
			\item $\SI{8.04}{kg}$.
		\end{mcq}
		
	}
	\loigiai
	{	\textbf{Đáp án: A.}
		
		Cường độ dòng điện qua bình điện phân:
		$$I=\dfrac{U}{R} = \SI{5}{A}.$$
		
		Áp dụng định luật Fa-ra-đây:
		$$m=\dfrac{1}{F} \dfrac{A}{n} I t = \SI{40.3}{g}.$$
	}
	\item \mkstar{2}
	
	\cauhoi
	{Một bình điện phân đựng dung dịch $\ce{AgNO_3}$, cường độ dòng điện chạy qua bình điện phân là $I=\SI{6}{A}$. Cho $A_\text{Ag} = 108$, $n_\text{Ag} = 1$. Lượng $\ce{Ag}$ bám vào cathode sau thời gian 49 phút 40 giây là
		\begin{mcq}(4)
			\item $\SI{2}{g}$.
			\item $\SI{20}{g}$.
			\item $\SI{200}{g}$.
			\item $\SI{2}{kg}$.
		\end{mcq}
		
	}
	\loigiai
	{	\textbf{Đáp án: B.}
		
		Thep định luật Fa-ra-đây:
		$$m=\dfrac{1}{F} \dfrac{A}{n} I t = \SI{20}{g}.$$
	}
	\item \mkstar{3}
	
	\cauhoi
	{Cho dòng điện chạy qua bình điện phân chứa dung dịch $\ce{CuSO_4}$ có anode làm bằng $\ce{Cu}$. Biết rằng đương lượng điện hóa của đồng là $k=\dfrac{A}{Fn} = \SI{3.3e-7}{kg/C}$. Để trên cathode xuất hiện $\SI{0.33}{kg}$ đồng thì điện tích chuyển qua bình phải bằng
		\begin{mcq}(4)
			\item $\SI{e5}{C}$.
			\item $\SI{e6}{C}$.
			\item $\SI{5e6}{C}$.
			\item $\SI{e7}{C}$.
		\end{mcq}
		
	}
	\loigiai
	{	\textbf{Đáp án: B.}
		
		Áp dụng định luật Fa-ra-đây:
		$$m=kq \Rightarrow q = \SI{e6}{C}.$$
	}
	\item \mkstar{3}
	
	\cauhoi
	{Để giải phóng lượng $\ce{Cl}$ và $\ce{H}$ từ $\SI{7.6}{g}$ $\ce{HCl}$ bằng dòng điện $\SI{5}{A}$ thì phải cần thời gian điện phân là bao lâu? Biết rằng đương lượng điện hóa của $\ce{H}$ và $\ce{Cl}$ lần lượt là $k_1 = \SI{0.1045e-7}{kg/C}$ và $k_2 = \SI{3.67e-7}{kg/C}$.
		\begin{mcq}(4)
			\item $\SI{1.5}{h}$.
			\item $\SI{1.3}{h}$.
			\item $\SI{1.1}{h}$.
			\item $\SI{1.0}{h}$.
		\end{mcq}
		
	}
	\loigiai
	{	\textbf{Đáp án: C.}
		
		Áp dụng định luật Fa-ra-đây:
		$$m=(k_1 + k_2) I t \Rightarrow t = \SI{4027}{s} \approx \SI{1.1}{h}.$$
	}
	\item \mkstar{3}
	
	\cauhoi
	{Một nguồn gồm 30 pin mắc thành 3 nhóm nối tiếp, mỗi nhóm có 10 pin mắc song song, mỗi pin có suất điện động $\SI{0.9}{V}$ và điện trở trong $\SI{0.6}{\Omega}$. Bình điện phân dung dịch $\ce{CuSO_4}$ có điện trở $\SI{205}{\Omega}$ mắc vào hai cực của bộ nguồn. Trong thời gian 50 phút, khối lượng đồng bám vào cathode là
		\begin{mcq}(4)
			\item $\SI{0.013}{g}$.
			\item $\SI{0.13}{g}$.
			\item $\SI{1.3}{g}$.
			\item $\SI{13}{g}$.
		\end{mcq}
		
	}
	\loigiai
	{	\textbf{Đáp án: A.}
		
		Suất điện động của bộ nguồn:
		$$\calE_\text{b} = 3 \calE = \SI{2.7}{V}.$$
		
		Điện trở trong của bộ nguồn:
		$$r_\text{b} = \dfrac{3r}{10} = \SI{0.18}{\Omega}.$$
		
		Áp dụng định luật Ôm toàn mạch:
		$$I=\dfrac{\calE_\text b}{R + r_\text b} = \SI{0.01316}{A}.$$
		
		Áp dụng định luật Fa-ra-đây:
		$$m=\dfrac{1}{F} \dfrac{A}{n} I t = \SI{0.013}{g}.$$
	}
	\item \mkstar{4}
	
	\cauhoi
	{Chiều dày của lớp nickel phủ lên một tấm kim loại là $d=\SI{0.05}{mm}$ sau khi điện phân trong 30 phút. Diện tích mặt phủ của tấm kim loại là $\SI{30}{cm^2}$. Cho biết nickel có khối lượng riêng là $\rho = \SI{8.9e3}{kg/m^3}$, nguyên tử khối của nickel là $A=\SI{58}{g/mol}$ và hóa trị của nickel là $n=2$. Cường độ dòng điện qua bình điện phân là
		\begin{mcq}(4)
			\item $\SI{2.5}{\micro A}$.
			\item $\SI{2.5}{mA}$.
			\item $\SI{0.25}{A}$.
			\item $\SI{2.5}{A}$.
		\end{mcq}
		
	}
	\loigiai
	{	\textbf{Đáp án: B.}
		
		Khối lượng nickel giải phóng ra ở điện cực tính theo 2 công thức:
		$$m = \rho V = \rho d S$$
		$$m=\dfrac{1}{F} \dfrac{A}{n} I t$$
		
		Suy ra:
		$$I=\dfrac{\rho d S F n}{A t} = \SI{2.5}{mA}.$$
	}
\end{enumerate}

\whiteBGstarEnd

\loigiai
{
	\begin{center}
		\textbf{BẢNG ĐÁP ÁN}
	\end{center}
	\begin{center}
		\begin{tabular}{|m{2.8em}|m{2.8em}|m{2.8em}|m{2.8em}|m{2.8em}|m{2.8em}|m{2.8em}|m{2.8em}|m{2.8em}|m{2.8em}|}
			\hline
			1.D  & 2.A  & 3.A  & 4.B  & 5.B  & 6.C  & 7.A  & 8.C  & 9.B  & 10.A  \\
			\hline
			11.B  & 12.B  & 13.C  & 14.A  & 15.B  & & & & & \\
			\hline
		\end{tabular}
	\end{center}
}
\section{Tự luận}
\begin{enumerate}[label=\bfseries Câu \arabic*:]
	\item \mkstar{1}
	
	\cauhoi{
		Nội dung của thuyết điện li là gì? Anion thường là phần nào của phân tử?
	}
	
	\loigiai{
		
		Nội dung của thuyết điện li: Trong dung dịch, các hợp chất hóa học như axit, bazơ và muối bị phân li (một phần hay toàn bộ) thành các nguyên tử hay các nhóm nguyên tử tích điện gọi là ion, ion có thể chuyển động tự do trong dung dịch và trở thành hạt tải điện.
		
		Anion là các ion âm, gốc axit, hay nhóm $\ce{OH^-}$.
	}
	
	\item \mkstar{2}
	
	\cauhoi{
		Hãy nói rõ hạt tải điện nào mang dòng điện trên các phần khác nhau của mạch điện có chứa bình điện phân?
		\begin{enumerate}
			\item Dây dẫn và điện cực kim loại;
			\item Ở sát bề mặt hai điện cực;
			\item Ở trong lòng chất lỏng điện phân.
		\end{enumerate}
	}
	
	\loigiai{
		
			\begin{enumerate}
			\item Dây dẫn và điện cực kim loại:
			
			Hạt tải điện là electron tự do.
			\item Ở sát bề mặt hai điện cực:
			
			Ở sát bề mặt anode, hạt tải điện là các ion âm.
			
			Ở sát bề mặt cathode, hạt tải điện là các ion dương.
			\item Ở trong lòng chất lỏng điện phân:
			
			Hạt tải điện là các ion dương và âm.
		\end{enumerate}
	}
	\item \mkstar{3}
	
	\cauhoi{
		Hai bình điện phân mắc nối tiếp trong một mạch điện. Bình 1 chứa dung dịch $\ce{CuSO_4}$ có cực dương bằng $\ce{Cu}$, bình hai chứa dung dịch $\ce{AgNO_3}$ có cực dương bằng $\ce{Ag}$. Sau một thời gian điện phân, khối lượng cực dương của cả hai bình tăng lên $\SI{2.8}{g}$.
		\begin{enumerate}
			\item Tính khối lượng cực dương tăng lên của mỗi bình.
			\item Tính thời gian điện phân, biết cường độ dòng điện qua mạch là $I=\SI{0.5}{A}$.
		\end{enumerate}
	
		Cho $A_\text{Cu} = \SI{64}{g/mol}$, hóa trị của $\ce{Cu}$ là 2, $A_\text{Ag} = \SI{108}{g/mol}$, hóa trị của $\ce{Ag}$ là 1.
	}
	
	\loigiai{
		
		\begin{enumerate}
			\item Tính khối lượng cực dương tăng lên của mỗi bình:
			
			Lập tỉ lệ:
			$$\dfrac{m_1}{m_2} = \dfrac{A_1 n_2}{A_2 n_1} = \dfrac{8}{27}.$$
			
			Mà theo đề bài thì $m_1 + m_2 = \SI{2.8}{g}$. Kết hợp hai phương trình trên, tính được $m_1 = \SI{0.64}{g}$ và $m_2 = \SI{1.16}{g}$.
			\item Tính thời gian điện phân, biết cường độ dòng điện qua mạch là $I=\SI{0.5}{A}$:
			
			Thời gian điện phân:
			$$t=\dfrac{m_1 F n_1}{A_1 I} = \SI{3860}{s}.$$
		\end{enumerate}
	}
	\item \mkstar{4}
	
	\cauhoi{
		Tốc độ chuyển động có hướng của ion $\ce{Na^+}$ và $\ce{Cl^-}$ trong nước có thể tính theo công thức: $v=\mu E$, trong đó $E$ là cường độ điện trường, $\mu$ có giá trị đối với $\ce{Na^+}$ và $\ce{Cl^-}$ lần lượt là $\SI{4.5e-8}{m^2/(Vs)}$ và $\SI{6.8e-8}{m^2/(Vs)}$. Tính điện trở suất của dung dịch $\ce{NaCl}$ nồng độ $\SI{0.1}{mol/l}$, cho rằng toàn bộ các phân tử $\ce{NaCl}$ đều phân li thành ion.
	}
	
	\loigiai{
		
		Điện trở của một khối vật dẫn có thể được tính bởi $R=\dfrac{U}{I}$ hoặc $R=\rho \dfrac{l}{S}$, suy ra:
		$$\rho = \dfrac{US}{I l} = \dfrac{E S}{I},$$
		với $E=\dfrac{U}{l}$ là cường độ điện trường.
		
		Gọi $v_\text{Na}$ và $v_\text{Cl}$ lần lượt là tốc độ có hướng của ion $\ce{Na^+}$ và $\ce{Cl^-}$, gọi $n$ là mật độ ion. Ta có:
		$$I=eS(v_\text{Na} + v_\text{Cl}) n = e S (\mu_\text{Na} + \mu_\text{Cl}) n E \Rightarrow \rho = \dfrac{E S}{I} = \dfrac{1}{e n (\mu_\text{Na} + \mu_\text{Cl})}.$$
		
		Đổi $n=\SI{0.1}{mol/l}= \SI{6.023e25}{m^{-3}}$. Tính được:
		$$\rho = \dfrac{1}{e n (\mu_\text{Na} + \mu_\text{Cl})} = \SI{0.9183}{\Omega m}.$$
	}
	\item \mkstar{4}
	
	\cauhoi{
		Người ta muốn bóc một lớp đồng dày $d=\SI{10}{\micro m}$ trên một bản đồng diện tích $S=\SI{1}{cm^2}$ bằng phương pháp điện phân. Cho cường độ dòng điện qua bình điện phân là $\SI{0.01}{A}$. Tính thời gian cần thiết để bóc được lớp đồng, cho biết đồng có khối lượng riêng là $\rho = \SI{8900}{kg/m^3}$.
	}
	
	\loigiai{
		
		Đổi $d=\SI{10}{\micro m} = \SI{10e-6}{m}$, $S=\SI{1}{cm^2}=\SI{e-4}{m^2}$.
		
		Khối lượng đồng phải bóc ra là
		$$m=\rho V = \rho S d = \SI{8.9e-6}{kg}.$$
		
		Theo công thức Fa-ra-đây:
		$$m=\dfrac{1}{F} \dfrac{A}{n} I t \Rightarrow t=\SI{2684}{s}.$$
	}
	
\end{enumerate}