\whiteBGstarBegin
\setcounter{section}{0}
\section{Trắc nghiệm}
\begin{enumerate}[label=\bfseries Câu \arabic*:]
	
	\item \mkstar{1}
	
	\cauhoi
	{Khi nói về lực hấp dẫn giữa hai chất điểm, phát biểu nào sau đây \textbf{sai?}
		\begin{mcq}
			\item Lực hấp dẫn có phương trùng với đường thẳng nối hai chất điểm.
			\item Lực hấp dẫn có điểm đặt tại mỗi chất điểm.
			\item Lực hấp dẫn của hai chất điểm là cặp lực trực đối.
			\item Lực hấp dẫn của hai chất điểm là cặp lực cân bằng.
		\end{mcq}
	}
	
	\loigiai
	{	\textbf{Đáp án: D.}
		
Lực hấp dẫn của hai chất điểm không phải là cặp lực cân bằng, vì điểm đặt của chúng khác nhau.
	}
	\item \mkstar{2}
	
	\cauhoi
	{Hiện tượng thủy triều xảy ra là do
		\begin{mcq}
			\item chuyển động của các dòng hải lưu.
			\item Trái Đất quay quanh Mặt Trời.
			\item lực hấp dẫn của Mặt Trăng - Mặt Trời.
			\item lực hấp dẫn của Mặt Trăng - Trái Đất.
		\end{mcq}
	}
	\loigiai
	{	\textbf{Đáp án: D.}
		
Hiện tượng thủy triều xảy ra là do lực hấp dẫn của Mặt Trăng - Trái Đất.
	}

	\item \mkstar{3}
	
	\cauhoi
	{Hai tàu thủy, mỗi chiếc có khối lượng $\SI{50000}{}$ tấn ở cách nhau $\SI{1}{km}$. Lấy $g=\SI{10}{m/s^2}$. So sánh lực hấp dẫn giữa chúng với trọng lượng của một quả cân có khối lượng $\SI{20}{g}$.
		\begin{mcq}(4)
			\item Lớn hơn.
			\item Bằng nhau.
			\item Nhỏ hơn.
			\item Chưa thể biết.
		\end{mcq}
	}
	
	\loigiai
	{	\textbf{Đáp án: C.}
		
	Lực hấp dẫn giữa hai tàu thủy:
	$$F_\text{hd}=G\dfrac{m_1m_2}{r^2} = \SI{0.16675}{N}$$
	
	Trọng lượng của 1 quả cân có khối lượng $\SI{20}{g}$:
	$$P=mg=\SI{0.2}{N}$$
	
	Vậy lực hấp dẫn giữa hai tàu thủy nhỏ hơn trọng lượng của 1 quả cân khối lượng $\SI{20}{g}$.
	}
	
	\item \mkstar{3}
	
	\cauhoi
	{Một vật ở trên mặt đất có trọng lượng $\SI{9}{N}$. Khi ở một điểm cách tâm Trái Đất $3R$ (với $R$ là bán kính Trái Đất) thì có trọng lượng bằng
		\begin{mcq}(4)
			\item $\SI{81}{N}$.
			\item $\SI{27}{N}$.
			\item $\SI{3}{N}$.
			\item $\SI{1}{N}$.
		\end{mcq}
	}
	
	\loigiai
	{	\textbf{Đáp án: D.}	
		
Lập tỉ số:
$$\dfrac{P_1}{P_2} = \dfrac{(3R)^2}{R^2} = 9$$

Vậy $P_2 = \SI{1}{N}$.
	}

	\item \mkstar{4}
	
	\cauhoi
	{Coi khoảng cách trung bình giữa tâm Trái Đất và tâm Mặt Trăng gấp 60 lần bán kính Trái Đất, khối lượng Mặt Trăng nhỏ hơn khối lượng Trái Đất 81 lần. Xét vật $M$ nằm trên đường thẳng nối tâm Trái Đất và tâm Mặt Trăng mà ở đó lực hấp dẫn của Trái Đất và của Mặt Trăng tác dụng với nó cân bằng nhau. So với bán kính Trái Đất, khoảng cách từ $M$ đến tâm Trái Đất gấp bao nhiêu lần?
		\begin{mcq}(4)
			\item $\SI{56.5}{}$ lần.
			\item $\SI{54}{}$ lần.
			\item $\SI{48}{}$ lần.
			\item $\SI{32}{}$ lần.
		\end{mcq}
	}
	
	\loigiai
	{	\textbf{Đáp án: B.}
		
	Gọi $m_0$ là khối lượng Mặt Trăng, suy ra $81m_0$ là khối lượng Trái Đất.
	
	Gọi $R_0$ là bán kính Trái Đất, suy ra $60R_0$ là khoảng cách giữa tâm Trái Đất và tâm Mặt Trăng.
	
	Gọi $R$ là khoảng cách từ Trái Đất đến vật $M$, suy ra $60R_0 - R$ là khoảng cách từ Mặt Trăng đến vật $M$.
	
	Để lực hấp dẫn giữa Trái Đất - vật $M$ và lực hấp dẫn giữa Mặt Trăng - vật $M$ cân bằng nhau:
	$$F_1 = F_2 \Rightarrow \dfrac{81m_0}{R^2} = \dfrac{m_0}{(60R_0 - R)^2} \Rightarrow \dfrac{81}{R^2} = \dfrac{1}{(60R_0 - R)^2} \Rightarrow R=54 R_0; R=67,5 R_0$$
	}
	
	
\end{enumerate}

\whiteBGstarEnd

\loigiai
{
	\begin{center}
		\textbf{BẢNG ĐÁP ÁN}
	\end{center}
	\begin{center}
		\begin{tabular}{|m{2.8em}|m{2.8em}|m{2.8em}|m{2.8em}|m{2.8em}|m{2.8em}|m{2.8em}|m{2.8em}|m{2.8em}|m{2.8em}|}
			\hline
			1.D  & 2.D  & 3.C  & 4.D  & 5.B  &  & & &  &  \\
			\hline
			
		\end{tabular}
	\end{center}
}
\section{Tự luận}
\begin{enumerate}[label=\bfseries Câu \arabic*:]
	\item \mkstar{1}
	
	\cauhoi{
		Phát biểu định luật vạn vật hấp dẫn và viết hệ thức của lực hấp dẫn.
	}
	
	\loigiai{
		
	Định luật vạn vật hấp dẫn: Lực hấp dẫn giữa hai chất điểm bất kì tỉ lệ thuận với tích hai khối lượng của chúng và tỉ lệ nghịch với bình phương khoảng cách giữa chúng.
	
	Công thức:
	$$F_\text{hd}=G\dfrac{m_1m_2}{r^2},$$
	trong đó:
	\begin{itemize}
		\item $m_1$, $m_2$ là khối lượng của hai chất điểm;
		\item $r$ là khoảng cách giữa chúng;
		\item $G=\SI{6.67e-11}{N\cdot m^2 / kg^2}$ gọi là hằng số hấp dẫn.
	\end{itemize}
	}
	
	\item \mkstar{2}
	
	\cauhoi
	{Tại sao gia tốc rơi tự do và trọng lượng của vật càng lên cao lại càng giảm?
	}
	
	\loigiai
	{Lực hấp dẫn giữa hai vật:
		$$F_\text{hd}=G\dfrac{mM}{r^2}$$
		
	Vật có khối lượng $m$ nhỏ hơn nhiều so với Trái Đất, ta xem lực tác dụng của Trái Đất lên vật là $F=P=mg$ gọi là trọng lực.
	
	Vật ở gần mặt đất thì:
	$$P=mg=G\dfrac{mM}{R^2}$$
	
	Vật ở độ cao $h$ thì:
	$$P'=mg'=G\dfrac{mM}{(R+h)^2}$$
	
	Ta được:
	$$\dfrac{g}{g'} = \dfrac{R^2}{(R+h)^2}$$
	
	Vậy càng lên cao ($h$ càng lớn) thì $g$ càng giảm và $P$ càng giảm.
	}
	\item \mkstar{3}
	
	\cauhoi
	{Một vật khối lượng 1 kg ở trên mặt đất có trọng lượng 10 N. Khi chuyển vật tới một điểm cách tâm Trái Đất $2R$ (với $R$ là bán kính Trái Đất) thì nó có trọng lượng là bao nhiêu?	}
	
	\loigiai
	{Ta có độ lớn của trọng lực (trọng lượng):
		$$P=G\dfrac{mM}{(R+h)^2}$$
		
		Tại mặt đất thì
		$$P_1 = G\dfrac{mM}{R^2} = 10\ \text N$$
		
		Tại một điểm cách tâm Trái Đất $2R$ thì
		$$P_2=G\dfrac{mM}{(2R)^2} = G\dfrac{mM}{4R^2}$$
		
		Suy ra:
		$$\dfrac{P_2}{P_1} = \dfrac{1}{4} \Rightarrow P_2 = \dfrac{P_1}{4} = \SI{2.5}{N}$$
	}
	\item \mkstar{4}
	
	\cauhoi
	{Hai quả cầu có khối lượng là $m_1=\SI{400}{g}$ và $m_2=\SI{200}{g}$. Khoảng cách giữa tâm hai quả cầu là $\SI{60}{m}$. Tại M nằm trên đường thẳng nối tâm hai quả cầu có vật khối lượng $m$. Biết độ lớn lực hút của $m_1$ tác dụng lên $m$ bằng 8 lần lực hút của $m_2$ tác dụng lên $m$. Tính khoảng cách giữa $m$ và $m_1$.
	}
	
	\loigiai
	{Lập tỉ số:
		$$\dfrac{F_1}{F_2} = 8 \Rightarrow \dfrac{m_1 m}{x^2} = 8 \dfrac{m_2 m}{(60-x)^2} \Rightarrow x = \SI{20}{cm}$$
		
	Vậy điểm $m$ cách $m_1$ một khoảng $\SI{20}{cm}$.
	}
	\item \mkstar{4}
	
	\cauhoi
	{Biết gia tốc rơi tự do ở đỉnh và chân một ngọn núi lần lượt là $\SI{9.809}{m/s^2}$ và $\SI{9.810}{m/s^2}$. Coi Trái Đất là đồng chất và chân núi cách tâm Trái Đất $\SI{6370}{km}$. Tính chiều cao của ngọn núi.
	}
	
	\loigiai
	{Lập tỉ số:
		$$\dfrac{F_1}{F_2}=\dfrac{9809}{9810} \Rightarrow \dfrac{R^2}{(R+h)^2} = \dfrac{9809}{9810} \Rightarrow h = \SI{0.3247}{km} = \SI{324.7}{m}$$
	}
\end{enumerate}