\whiteBGstarBegin
\setcounter{section}{0}
\section{Trắc nghiệm}
\begin{enumerate}[label=\bfseries Câu \arabic*:]
	
	
	\item \mkstar{1}
	
	\cauhoi
	{Người ta gọi silic là chất bán dẫn vì
		\begin{mcq}
			\item nó không phải là kim loại, cũng không phải là điện môi.
			\item hạt tải điện trong đó có thể là electron và lỗ trống.
			\item điện trở suất của nó rất nhạy với nhiệt độ, tạp chất và các tác nhân ion khác.
			\item Cả 3 lí do trên.
		\end{mcq}
		
	}
	\loigiai
	{	\textbf{Đáp án: D.}
		
	}
	\item \mkstar{1}
	
	\cauhoi
	{Phát biểu nào dưới dây về transistor là chính xác?
		\begin{mcq}
			\item Một bán dẫn loại p kẹp giữa hai lớp bán dẫn loại n là transistor n-p-n.
			\item Một bán dẫn loại n mỏng kẹp giữa hai lớp bán dẫn loại p không thể xem là transistor.
			\item Một lớp bán dẫn loại p mỏng kẹp giữa hai lớp bán dẫn loại n luôn có khả năng là mạch khuếch đại.
			\item Trong transistor n-p-n, bao giờ mật độ hạt tải điện vùng Emitor cũng cao hơn vùng Base.
		\end{mcq}
		
	}
	\loigiai
	{	\textbf{Đáp án: D.}
	
	}
	\item \mkstar{1}
	
	\cauhoi
	{Silic tinh khiết thì
		\begin{mcq}
			\item dẫn điện tốt ở moi nhiệt độ.
			\item chỉ dẫn điện tốt khi ở nhiệt độ thấp.
			\item có liên kết đồng hóa trị giữa hai nguyên tử.
			\item chỉ có một loại hạt tải điện là electron.
		\end{mcq}
		
	}
	\loigiai
	{	\textbf{Đáp án: C.}
		
	}
	\item \mkstar{1}
	
	\cauhoi
	{Pin mặt trời là một nguồn điện biến đổi năng lượng từ
		\begin{mcq}(2)
			\item nhiệt năng thành điện năng.
			\item quang năng thành điện năng.
			\item cơ năng thành điện năng.
			\item hóa năng thành điện năng.
		\end{mcq}
		
	}
	\loigiai
	{	\textbf{Đáp án: B.}
		
	}
	\item \mkstar{1}
	
	\cauhoi
	{Lớp tiếp giáp p-n có
		\begin{mcq}
			\item tác dụng ngăn cản các electron từ p sang n.
			\item tác dụng ngăn cản các electron từ n sang p.
			\item tính dẫn điện một chiều từ n sang p.
			\item tính dẫn điện một chiều từ p sang n.
		\end{mcq}
		
	}
	\loigiai
	{	\textbf{Đáp án: C.}
		
	}
	\item \mkstar{1}
	
	\cauhoi
	{Cách pha tạp nào sau đây tạo ra bán dẫn loại p?
		\begin{mcq}(2)
			\item Si pha tạp As.
			\item Si pha tạp B.
			\item Si pha tạp S.
			\item Si pha tạp Pb.
		\end{mcq}
		
	}
	\loigiai
	{	\textbf{Đáp án: B.}
		
	}
	\item \mkstar{1}
	
	\cauhoi
	{Hạt mang điện chủ yếu trong bán dẫn loại n là
		\begin{mcq}(2)
			\item lỗ trống.
			\item electron.
			\item electron và lỗ trống.
			\item electron và ion dương.
		\end{mcq}
		
	}
	\loigiai
	{	\textbf{Đáp án: B.}
		
	}
	\item \mkstar{1}
	
	\cauhoi
	{Chọn câu đúng. Diode bán dẫn được dùng để
		\begin{mcq}
			\item nấu chảy kim loại.
			\item chỉnhh lưu dòng điện xoay chiều.
			\item tạo ra dòng điện xoay chiều.
			\item biến đổi dòng điện một chiều thành dòng điện xoay chiều.
		\end{mcq}
		
	}
	\loigiai
	{	\textbf{Đáp án: B.}
		
	}
	\item \mkstar{1}
	
	\cauhoi
	{Bán dẫn chứa tạp chất nhận là bán dẫn có đặc điểm:
		\begin{mcq}
			\item có mật độ lỗ trống rất lớn so với mật độ electron.
			\item có mật độ electron rất lớn so với mật độ lỗ trống.
			\item chỉ có một loại hạt tải điện là lỗ trống.
			\item chỉ có một loại hạt tải điện là electron.
		\end{mcq}
		
	}
	\loigiai
	{	\textbf{Đáp án: A.}
		
	}
	\item \mkstar{1}
	
	\cauhoi
	{Transistor là dụng cụ bán dẫn có số lớp chuyển tiếp p-n trong nó là
		\begin{mcq}(4)
			\item 4 lớp.
			\item 2 lớp.
			\item 3 lớp .
			\item 1 lớp.
		\end{mcq}
		
	}
	\loigiai
	{	\textbf{Đáp án: B.}
		
	}

\end{enumerate}

\whiteBGstarEnd

\loigiai
{
	\begin{center}
		\textbf{BẢNG ĐÁP ÁN}
	\end{center}
	\begin{center}
		\begin{tabular}{|m{2.8em}|m{2.8em}|m{2.8em}|m{2.8em}|m{2.8em}|m{2.8em}|m{2.8em}|m{2.8em}|m{2.8em}|m{2.8em}|}
			\hline
			1.D  & 2.D  & 3.C  & 4.B  & 5.C  & 6.B  & 7.B  & 8.B  & 9.A  & 10.B  \\
			\hline
			
		\end{tabular}
	\end{center}
}
\section{Tự luận}
\begin{enumerate}[label=\bfseries Câu \arabic*:]
	\item \mkstar{1}
	
	\cauhoi{
		Chất bán dẫn là gì? Nêu bản chất dòng điện trong chất bán dẫn.
	}
	
	\loigiai{
		
		Chất bán dẫn là chất có điện trở suất nằm trong khoảng trung gian giữa kim loại và chất điện môi.
		
		Dòng điện trong chất bán dẫn là dòng các electron dẫn chuyển động ngược chiều điện trường và dòng các lỗ trống chuyển động cùng chiều điện trường.
	}
	
	\item \mkstar{2}
	
	\cauhoi{
		So sánh chất bán dẫn loại n và loại p theo các tiêu chí được cho trong bảng sau:
		\begin{longtable}{|m{8em}|m{14em}|m{14em}|}
			\hline
			& \thead{Bán dẫn loại n} & \thead{Bán dẫn loại p} \\
			\hline
			\textbf{Định nghĩa} & & \\
			\hline
			\textbf{Hạt tải đa số} & & \\
			\hline
			\textbf{Tạp chất} & & \\
			\hline
		\end{longtable}
	}
	
	\loigiai{
		
		\begin{longtable}{|m{8em}|m{14em}|m{14em}|}
			\hline
			& \thead{Bán dẫn loại n} & \thead{Bán dẫn loại p} \\
			\hline
			\textbf{Định nghĩa} & Là chất bán dẫn mà hạt tải điện đa số trong đó mang điện âm. & Là chất bán dẫn mà hạt tải điện đa số trong đó mang điện dương.\\
			\hline
			\textbf{Hạt tải đa số} & Các electron. & Các lỗ trống. \\
			\hline
			\textbf{Tạp chất} & Tạp chất cho (donor): Sinh ra electron dẫn, thường là những nguyên tố có 5 electron hóa trị như $\ce{P}$, $\ce{As}$, $\ldots$. & Tạp chất nhận (aceptor): Nhận electron (đồng nghĩa sinh ra lỗ trống), thường là những nguyên tố có 5 electron hóa trị như $\ce{B}$, $\ce{Al}$, $\ldots$.\\
			\hline
		\end{longtable}
	}
	\item \mkstar{3}
	
	\cauhoi{
		Ở nhiệt độ phòng, trong bán dẫn Si tinh khiết có số cặp điện tử-lỗ trống bằng $\SI{e-13}{}$ lần số nguyên tử Si. Tìm số cặp điện tử-lỗ trống có trong $\SI{2}{mol}$ nguyên tử Si.
	}
	
	\loigiai{
		
		Số nguyên tử $\ce{Si}$ có trong $\SI{2}{\mol}$ là
		$$N = 2 N_\text{A} = \SI{1.205e24}{}.$$
		
		Số cặp điện tử-lỗ trống bằng $\SI{e-13}{}$ lần số nguyên tử, suy ra số cặp điện tử-lỗ trống là
		$$\SI{e-13}{} \cdot N = \SI{1.205e11}{}.$$
	}
	\item \mkstar{3}
	
	\cauhoi{
		Ở nhiệt độ phòng, trong bán dẫn Si tinh khiết có số cặp điện tử-lỗ trống bằng $\SI{e-13}{}$ lần số nguyên tử Si. Nếu pha P và Si với tỉ lệ một phần triệu thì số hạt tải điện tăng lên bao nhiêu lần?
	}
	
	\loigiai{
		
		Gọi $N_0$ là số nguyên tử $\ce{Si}$ có trong chất bán dẫn.
		
		Ở nhiệt độ phòng, trong bán dẫn $\ce{Si}$ tinh khiết, số cặp điện tử-lỗ trống bằng $\SI{e-13}{} N_0$. Tức là số hạt tải điện bằng:
		$$N=\SI{e-13}{} N_0.$$
		
		Khi pha một nguyên tử $\ce{P}$ vào bán dẫn $\ce{Si}$ tinh khiết sẽ tạo ra thêm 1 electron tự do. Với tỉ lệ pha tạp là một phần triệu thì số hạt tải điện tăng thêm là
		$$\Delta N = \SI{e-6}{} N_0.$$
		
		Vậy số hạt tại điện tăng thêm gấp $\dfrac{\Delta N}{N} = \SI{5e6}{}$ lần.
	}
	
\end{enumerate}