\whiteBGstarBegin
\setcounter{section}{0}
\section{Trắc nghiệm}
\begin{enumerate}[label=\bfseries Câu \arabic*:]
	
	
	\item \mkstar{1}
	
	\cauhoi
	{Phát biểu nào là chính xác? Hạt tải điện trong kim loại là
		\begin{mcq}
			\item các electron của nguyên tử.
			\item electron ở lớp trong cùng của nguyên tử.
			\item các electron hóa trị đã bay tự do ra khỏi tinh thể.
			\item các electron hóa trị chuyển động tự do trong mạng tinh thể.
		\end{mcq}
		
	}
	\loigiai
	{	\textbf{Đáp án: D.}
		
		Hạt tải điện trong kim loại là các electron hóa trị chuyển động tự do trong mạng tinh thể.
	}
	\item \mkstar{1}
	
	\cauhoi
	{Phát biểu nào là chính xác? Các kim loại đều
		\begin{mcq}
			\item dẫn điện tốt, có điện trở suất không thay đổi.
			\item dẫn điện tốt, có điện trở suất thay đổi theo nhiệt độ.
			\item dẫn điện tốt như nhau, có điện trở suất thay đổi theo nhiệt độ.
			\item dẫn điện tốt, có điện trở suất thay đổi theo nhiệt độ giống nhau.
		\end{mcq}
		
	}
	\loigiai
	{	\textbf{Đáp án: B.}
		
		Các kim loại đều dẫn điện tốt, có điện trở suất thay đổi theo nhiệt độ.
	}
	\item \mkstar{2}
	
	\cauhoi
	{Chuyển động của electron trong vật dẫn bằng kim loại khi có điện trường ngoài có đặc điểm:
		\begin{mcq}
			\item cùng hướng với điện trường ngoài.
			\item kết hợp chuyển động nhiệt và chuyển động có hướng.
			\item theo một phương duy nhất.
			\item hỗn loạn.
		\end{mcq}
		
	}
	\loigiai
	{	\textbf{Đáp án: B.}
		
	}

	\item \mkstar{2}
	
	\cauhoi
	{Nguyên nhân gây ra điện trở của kim loại là do
		\begin{mcq}
			\item sự va chạm của các electron với các ion dương ở các nút mạng.
			\item sự va chạm của các ion dương ở các nút mạng với nhau.
			\item sự va chạm của các electron với nhau.
			\item sự va chạm của các ion âm ở các nút mạng với nhau.
		\end{mcq}
		
	}
	\loigiai
	{	\textbf{Đáp án: A.}
		
	}
	\item \mkstar{2}
	
	\cauhoi
	{Khi tăng nhiệt độ của kim loại thì sẽ làm tăng điện trở của kim loại ấy, nguyên nhân gây ra hiện tượng này là
		\begin{mcq}
			\item số lượng va chạm của các electron dẫn với các ion ở nút mạng trong tinh thể tăng.
			\item số electron dẫn bên trong mạng tinh thể giảm.
			\item số ion ở nút mạng bên trong mạng tinh thể tăng.
			\item số nguyên tử kim loại bên trong mạng tinh thể tăng.
		\end{mcq}
		
	}
	\loigiai
	{	\textbf{Đáp án: A.}
		
	}
	\item \mkstar{2}
	
	\cauhoi
	{Nguyên nhân gây ra hiện tượng tỏa nhiệt trong dây dẫn khi có dòng điện chạy qua là do
		\begin{mcq}
			\item năng lượng của chuyển động có hướng của electron truyền cho ion dương khi va chạm.
			\item năng lượng dao động của ion dương truyền cho electron khi va chạm.
			\item năng lượng chuyển động có hướng của electron truyền cho ion âm khi va chạm.
			\item năng lượng của chuyển động có hướng của electron, ion âm truyền cho ion dương khi va chạm.
		\end{mcq}
		
	}
	\loigiai
	{	\textbf{Đáp án: A.}
		
	}
	\item \mkstar{2}
	
	\cauhoi
	{Khi đường kính của khối kim loại đồng chất tăng 2 lần thì điện trở của khối kim loại
		\begin{mcq}(2)
			\item giảm 4 lần.
			\item giảm 2 lần.
			\item tăng 2 lần.
			\item tăng 4 lần.
		\end{mcq}
		
	}
	\loigiai
	{	\textbf{Đáp án: A.}
		
		Điện trở của khối kim loại:
		$$R= \rho \dfrac{l}{S} = \rho \dfrac{l}{\pi \dfrac{d^2}{4}}.$$
		
		Khi $d$ tăng 2 lần thì $R$ giảm 4 lần.
	}
	\item \mkstar{2}
	
	\cauhoi
	{Khi tăng đồng thời chiều dài của một sợi dây đồng chất lên 2 lần, đồng thời giảm tiết diện của dây đi 2 lần thì điện trở của dây
		\begin{mcq}(2)
			\item không đổi.
			\item tăng 2 lần.
			\item giảm 4 lần.
			\item tăng 4 lần.
		\end{mcq}
		
	}
	\loigiai
	{	\textbf{Đáp án: D.}
		
		Điện trở của dây kim loại:
		$$R=\rho \dfrac{l}{S}.$$
		
		Khi $l$ tăng 2 lần, đồng thời $S$ giảm 2 lần thì $R$ tăng 4 lần.
	}
	\item \mkstar{2}
	
	\cauhoi
	{Khi chiều dài của khối kim loại đồng chất tiết diện đều tăng 2 lần thì điện trở suất của kim loại đó
		\begin{mcq}(2)
			\item không đổi.
			\item tăng 2 lần.
			\item giảm 2 lần.
			\item chưa thể kết luận.
		\end{mcq}
		
	}
	\loigiai
	{	\textbf{Đáp án: A.}
		
		Điện trở suất của kim loại không phụ thuộc vào chiều dài.
	}
	\item \mkstar{2}
	
	\cauhoi
	{Phát biểu nào sau đây đúng? Khi cho hai thanh kim loại có bản chất khác nhau tiếp xúc với nhau thì
		\begin{mcq}
			\item có sự khuếch tán electron từ thanh có nhiều electron sang thanh có ít electron hơn.
			\item có sự khuếch tán ion từ kim loại này sang kim loại kia.
			\item có sự khuếch tán electron từ kim loại có mật độ electron lớn sang kim loại có mật độ electron nhỏ hơn.
			\item không có hiện tượng gì xảy ra.
		\end{mcq}
		
	}
	\loigiai
	{	\textbf{Đáp án: C.}
		
	}
	\item \mkstar{2}
	
	\cauhoi
	{Một sợi dây bằng nhôm có điện trở $\SI{120}{\Omega}$ ở nhiệt độ $\SI{20}{\celsius}$, điện trở của sợi dây đó ở $\SI{179}{\celsius}$ là $\SI{204}{\Omega}$. Hệ số nhiệt điện trở của nhôm là
		\begin{mcq}(4)
			\item $\SI{4.8e-3}{K^{-1}}$.
			\item $\SI{4.4e-3}{K^{-1}}$.
			\item $\SI{4.3e-3}{K^{-1}}$.
			\item $\SI{4.1e-3}{K^{-1}}$.
		\end{mcq}
		
	}
	\loigiai
	{	\textbf{Đáp án: A.}
		
		Áp dụng công thức tính điện trở $R=R_0 (1+ \alpha (t-t_0))$, ta có:
		$$\dfrac{R_1}{R_2} = \dfrac{1+ \alpha t_1}{1+ \alpha t_2} \Rightarrow \alpha = \SI{4.8e-3}{K^{-1}}.$$
	}
	\item \mkstar{2}
	
	\cauhoi
	{Một sợi dây đồng có điện trở $\SI{74}{\Omega}$ ở $\SI{50}{\celsius}$. Cho biết hệ số nhiệt điện trở của đồng là $\SI{4.1e-3}{K^{-1}}$. Điện trở của sợi dây đồng đó ở $\SI{100}{\celsius}$ là
		\begin{mcq}(4)
			\item $\SI{86.6}{\Omega}$.
			\item $\SI{89.2}{\Omega}$.
			\item $\SI{95}{\Omega}$.
			\item $\SI{82}{\Omega}$.
		\end{mcq}
		
	}
	\loigiai
	{	\textbf{Đáp án: A.}
		
		Áp dụng công thức tính điện trở $R=R_0 (1+ \alpha (t-t_0))$, ta có:
		$$\dfrac{R_1}{R_2} = \dfrac{1 + \alpha t_1}{1 + \alpha t_2} \Rightarrow R_2 \approx \SI{86.6}{\Omega}.$$
	}
	\item \mkstar{2}
	
	\cauhoi
	{Một mối hàn của một cặp nhiệt điện có hệ số nhiệt điện $\alpha_T = \SI{65}{\micro V / K}$ được đặt trong không khí ở $\SI{20}{\celsius}$, còn mối hàn kia được nung nóng đến nhiệt độ $\SI{232}{\celsius}$. Suất điện động nhiệt điện của cặp nhiệt điện đó là
		\begin{mcq}(4)
			\item $\calE = \SI{13.00}{m V}$.
			\item $\calE = \SI{13.58}{mV}$.
			\item $\calE  = \SI{13.98}{mV}$.
			\item $\calE = \SI{13.78}{mV}$.
		\end{mcq}
		
	}
	\loigiai
	{	\textbf{Đáp án: D.}
		
		Suất điện động nhiệt điện:
		$$\calE = \alpha_T (T_2 - T_1) = \SI{13780}{\micro V} = \SI{13.78}{mV}.$$
	}
	\item \mkstar{2}
	
	\cauhoi
	{Một mối hàn của một cặp nhiệt điện có hệ số $\alpha_T  = \SI{48}{\micro V / K}$ được đặt trong không khí ở $\SI{20}{\celsius}$, còn mối hàn kia được nung nóng đến $t$ (độ C). Suất điện động nhiệt điện của cặp nhiệt điện đó là $\calE = \SI{6}{mV}$. Tìm $t$.
		\begin{mcq}(4)
			\item $\SI{125}{\celsius}$.
			\item $\SI{398}{K}$.
			\item $\SI{145}{\celsius}$.
			\item $\SI{418}{\celsius}$.
		\end{mcq}
		
	}
	\loigiai
	{	\textbf{Đáp án: C.}
		
		Nhiệt độ $t$ là
		$$\calE = \alpha_t (T-T_0) \Rightarrow T = \SI{418}{K} \Rightarrow t = \SI{145}{\celsius}.$$
	}
	\item \mkstar{2}
	
	\cauhoi
	{Một mối hàn của một cặp nhiệt điện có hệ số $\alpha_T$ được đặt trong không khí ở $\SI{20}{\celsius}$, còn mối hàn kia được nung nóng đến $\SI{500}{\celsius}$. Suất điện động nhiệt điện của cặp nhiệt điện khi đó là $\calE = \SI{6}{mV}$. Tính $\alpha_T$.
		\begin{mcq}(4)
			\item $\SI{1.25e-4}{V/K}$.
			\item $\SI{12.5}{\micro V / K}$.
			\item $\SI{1.25}{\micro V / K}$.
			\item $\SI{1.25}{mV / K}$.
		\end{mcq}
		
	}
	\loigiai
	{	\textbf{Đáp án: B.}
		
		Hệ số $\alpha_T$ là
		$$\calE = \alpha_T (T - T_0) \Rightarrow \alpha_T = \SI{12.5}{\micro V / K}.$$
	}
	\item \mkstar{3}
	
	\cauhoi
	{Khi hiệu điện thế giữa hai cực của bóng đèn là $U_1 = \SI{20}{mV}$ thì cường độ dòng điện chạy qua đèn là $I_1= \SI{8}{mA}$, nhiệt độ dây tóc bóng đèn là $t_1 = \SI{25}{\celsius}$. Khi sáng bình thường, hiệu điện thế giữa hai cực của bóng đèn là $U_2 = \SI{240}{V}$ thì cường độ dòng điện chạy qua đèn là $I_2 = \SI{8}{A}$. Biết hệ số nhiệt điện trở $\alpha = \SI{4.2e-3}{K^{-1}}$. Nhiệt độ $t_2$ của dây tóc bóng đèn khi sáng bình thường là
		\begin{mcq}(4)
			\item $\SI{2600}{\celsius}$.
			\item $\SI{3649}{\celsius}$.
			\item $\SI{2644}{K}$.
			\item $\SI{2919}{\celsius}$.
		\end{mcq}
		
	}
	\loigiai
	{	\textbf{Đáp án: D.}
		
		Điện trở của dây tóc bóng đèn ở nhiệt độ $t_1$:
		$$R_1 = \dfrac{U_1}{I_1} = \SI{2.5}{\Omega}.$$
		
		Điện trở của dây tóc bóng đèn ở nhiệt độ $t_2$:
		$$R_2 = \dfrac{U_2}{I_2} = \SI{30}{\Omega}.$$
		
		Lập tỉ lệ, ta có:
		$$\dfrac{R_1}{R_2} = \dfrac{1 + \alpha t_1}{1 + \alpha t_2} \Rightarrow t_2 = \SI{2919}{\celsius}.$$
	}
	\item \mkstar{3}
	
	\cauhoi
	{Trong dây dẫn kim loại có dòng điện không đổi chạy qua có cường độ $\SI{16}{mA}$. Trong 1 phút số lượng electron chuyển qua một tiết diện thẳng của dây là
		\begin{mcq}(2)
			\item $\SI{6e19}{}$ electron.
			\item $\SI{6e18}{}$ electron.
			\item $\SI{6e20}{}$ electron.
			\item $\SI{6e17}{}$ electron.
		\end{mcq}
		
	}
	\loigiai
	{	\textbf{Đáp án: B.}
		
		Số electron chuyển qua tiết diện thẳng trong 1 phút ($t=\SI{60}{s}$) là
		$$n_e = \dfrac{It}{|e|} = \SI{6e18}{}.$$
	}
	\item \mkstar{3}
	
	\cauhoi
	{Một bóng đèn ghi $\SI{220}{V} - \SI{40}{W}$ có dây tóc làm bằng von-fram. Điện trở của dây tóc ở $\SI{20}{\celsius}$ là $\SI{121}{\Omega}$. Tính nhiệt độ của bóng đèn khi sáng bình thường, biết rằng điện trở của dây tóc bóng đèn tăng theo hàm bậc nhất với nhiệt độ. Cho hệ số nhiệt điện trở của von-fram là $\alpha = \SI{4.5e-3}{K^{-1}}$.
		\begin{mcq}(4)
			\item $\SI{2200}{\celsius}$.
			\item $\SI{1919}{\celsius}$.
			\item $\SI{2121}{\celsius}$.
			\item $\SI{2222}{\celsius}$.
		\end{mcq}
		
	}
	\loigiai
	{	\textbf{Đáp án: A.}
		
		Điện trở của dây tóc bóng đèn khi sáng bình thường:
		$$R=\dfrac{U_\text{đm}^2}{\calP_\text{đm}} = \SI{1210}{\Omega}.$$
		
		Lập tỉ lệ, ta có:
		$$\dfrac{R_1}{R_2} = \dfrac{1+ \alpha t_1}{1 + \alpha t_2} \Rightarrow t_2 = \SI{2200}{\celsius}.$$
	}
	\item \mkstar{3}
	
	\cauhoi
	{Một bóng đèn ghi $\SI{120}{V} - \SI{60}{W}$ khi sáng bình thường ở nhiệt độ $\SI{2500}{\celsius}$ có điện trở gấp 10 lần so với điện trở của nó ở nhiệt độ $\SI{20}{\celsius}$. Tính điện trở của dây tóc bóng đèn ở nhiệt độ $\SI{20}{\celsius}$.
		\begin{mcq}(4)
			\item $\SI{24}{\Omega}$.
			\item $\SI{22}{\Omega}$.
			\item $\SI{14}{\Omega}$.
			\item $\SI{12}{\Omega}$.
		\end{mcq}
		
	}
	\loigiai
	{	\textbf{Đáp án: A.}
		
		Điện trở của dây tóc bóng đèn khi sáng bình thường:
		$$R_2 = \dfrac{U_\text{đm}^2}{\calP_\text{đm}} = \SI{240}{\Omega}.$$
		
		Điện trở của dây tóc bóng đèn ở $\SI{20}{\celsius}$:
		$$R_1 = \dfrac{R_1}{10}= \SI{24}{\Omega}.$$
	}
	\item \mkstar{4}

\cauhoi
{Một cặp nhiệt điện có hệ số nhiệt điện động là $\alpha_T = \SI{52e-6}{V / K}$, điện trở trong $r=\SI{0.5}{\Omega}$. Nối cặp nhiệt điện này với điện kế G có điện trở $R_\text{G} = \SI{20}{\Omega}$. Đặt một mối hàn của cặp nhiệt điện này trong không khí ở $\SI{24}{\celsius}$ và đưa mối hàn thứ hai vào trong lò điện thì thấy dòng điện qua điện kế G là $\SI{1.52}{mA}$. Nhiệt độ trong lò điện khi đó là
	\begin{mcq}(4)
		\item $\SI{304}{\celsius}$.
		\item $\SI{623}{\celsius}$.
		\item $\SI{3120}{\celsius}$.
		\item $\SI{3100}{\celsius}$.
	\end{mcq}
	
}
\loigiai
{	\textbf{Đáp án: B.}
	
	Suất điện động nhiệt điện:
	$$\calE = I (R+r) = \SI{0.03116}{V}.$$
	
	Nhiệt độ của lò điện:
	$$\calE = \alpha_T (T_1 - T_2) \Rightarrow T_1 \approx \SI{896}{K} \Rightarrow t_1 \approx \SI{623}{\celsius}.$$
}
\end{enumerate}

\whiteBGstarEnd

\loigiai
{
	\begin{center}
		\textbf{BẢNG ĐÁP ÁN}
	\end{center}
	\begin{center}
		\begin{tabular}{|m{2.8em}|m{2.8em}|m{2.8em}|m{2.8em}|m{2.8em}|m{2.8em}|m{2.8em}|m{2.8em}|m{2.8em}|m{2.8em}|}
			\hline
			1.D  & 2.B  & 3.B  & 4.A  & 5.A  & 6.A  & 7.A  & 8.D  & 9.A  & 10.C  \\
			\hline
			11.A  & 12.A  & 13.D  & 14.C  & 15.B  & 16.D  & 17.B  & 18.A  & 19.A  & 20.B  \\
			\hline
		\end{tabular}
	\end{center}
}
\section{Tự luận}
\begin{enumerate}[label=\bfseries Câu \arabic*:]
	\item \mkstar{1}
	
	\cauhoi{
		Hạt tải điện trong kim loại là loại electron nào? Mật độ của chúng vào cỡ nào?
	}
	
	\loigiai{
		
		Hạt tải điện trong kim loại là electron tự do. Mật độ của chúng vào cỡ $10^{28}$ electron trên mét khối. Mật độ này rất cao nên kim loại dẫn điện rất tốt.
	}
	
	\item \mkstar{2}
	
	\cauhoi{
		Vì sao điện trở của kim loại tăng khi nhiệt độ tăng?
	}
	
	\loigiai{
		
		Khi nhiệt độ tăng, các ion kim loại ở nút mạng tinh thể dao động mạnh dẫn đến độ mất trật tự của mạng tinh thể kim loại tăng, làm tăng sự cản trở chuyển động của các electron tự do. Vì vậy, khi nhiệt độ tăng, điện trở suất của kim loại tăng, dẫn đến điện trở của kim loại tăng.
	}
	\item \mkstar{3}
	
	\cauhoi{
		Một bóng đèn $\SI{220}{V} - \SI{100}{W}$ khi sáng bình thường thì nhiệt độ của dây tóc bóng đèn là $\SI{2000}{\celsius}$. Xác định điện trở của dây tóc bóng đèn khi thắp sáng bình thường và khi không thắp sáng, biết rằng khi không thắp sáng thì nhiệt độ của dây tóc là $\SI{20}{\celsius}$. Cho biết dây tóc bóng đèn làm bằng von-fram có hệ số nhiệt điện trở là $\alpha = \SI{4.5e-3}{K^{-1}}$.
	}
	
	\loigiai{
		
		Điện trở của bóng đèn khi thắp sáng (ở $\SI{2000}{\celsius}$), cũng là điện trở của đèn khi đèn sáng bình thường:
		$$R = \dfrac{U_\text{đm}^2}{\calP_\text{đm}} = \SI{484}{\Omega}.$$
		
		Điện trở bóng đèn khi không sáng (ở $\SI{20}{\celsius}$):
		$$R_0 = \dfrac{R}{1 + \alpha (t-t_0)} = \SI{48.8}{\Omega}.$$
	}
	\item \mkstar{4}
	
	\cauhoi{
		Khối lượng mol của nguyên tử đồng là $\SI{64e-3}{kg/mol}$. Khối lượng riêng của đồng là $\SI{8.9e3}{kg/m^3}$. Biết rằng mỗi nguyên tử đồng đóng góp 1 electron dẫn.
		\begin{enumerate}
		\item Tính mật độ electron tự do trong đồng.
		\item Một dây tải điện bằng đồng, tiết diện $\SI{10}{mm^2}$, mang dòng điện $\SI{10}{A}$. Tính tốc độ trôi của electron dẫn trong dây dẫn đó.
		\end{enumerate}
	}
	
	\loigiai{
		
		\begin{enumerate}
			\item Tính mật độ electron tự do trong đồng.
			
			Vì mỗi nguyên tử đồng đóng góp 1 electron dẫn nên số electron tự do trong $\SI{1}{mol}$ đồng là
			$$N_e = N_\text{A} = \SI{6.02e23}{}.$$
			
			Thể tích của $\SI{1}{mol}$ đồng:
			$$V=\dfrac{m}{D} = \dfrac{\SI{64e-3}{kg/mol}}{\SI{8.9e3}{kg/m^3}} = \SI{7.19e-6}{m^3 / mol}.$$
			
			Mật độ electron tự do trong đồng:
			$$n_e = \dfrac{N_e}{V} = \SI{8.38e28}{m^{-3}}.$$
			\item Một dây tải điện bằng đồng, tiết diện $\SI{10}{mm^2}$, mang dòng điện $\SI{10}{A}$. Tính tốc độ trôi của electron dẫn trong dây dẫn đó.
			
			Số electron tự do đi qua tiết diện $S$ của dây dẫn trong $\SI{1}{s}$:
			$$N= vS n_e.$$
			
			Cường độ dòng điện qua dây dẫn:
			$$I=e N = ev S n_e.$$
			
			Suy ra:
			$$v=\dfrac{I}{e S n_e} = \SI{7.46e-5}{m/s}.$$
		\end{enumerate}
	}
	\item \mkstar{4}
	
	\cauhoi{
		Để mắc đường dây tải điện từ địa điểm A đến địa điểm B, ta cần $\SI{1000}{kg}$ dây đồng. Muốn thay dây đồng bằng dây nhôm mà vẫn đảm bảo chất lượng truyền điện, ít nhất phải dùng bao nhiêu kg nhôm? Cho biết khối lượng riêng của đồng là $\SI{8900}{kg/m^3}$, của nhôm là $\SI{2700}{kg/m^3}$.
	}
	
	\loigiai{
		
		Để đảm bảo chất lượng truyền điện thì tổng điện trở trên đường dây đối với dây nhôm và dây đông phải bằng nhau. Áp dụng công thức $R=\rho \dfrac{l}{S}$, ta có:
		$$\rho_\text{đồng} \dfrac{l_\text{đồng}}{S_\text{đồng}} = \rho_\text{nhôm} \dfrac{l_\text{nhôm}}{S_\text{nhôm}}.$$
		
		Mà chiều dài dây từ A đến B là không đổi, nên:
		$$\dfrac{S_\text{đồng}}{S_\text{nhôm}} = \dfrac{\rho_\text{đồng}}{\rho_\text{nhôm}}.$$
		
		Mà khối lượng dây $m=DV$, ta có:
		$$\dfrac{m_\text{đồng}}{m_\text{nhôm}} = \dfrac{D_\text{đồng} S_\text{đồng}}{D_\text{nhôm} S_\text{nhôm}} = \dfrac{D_\text{đồng} \rho_\text{đồng}}{D_\text{nhôm} \rho_\text{nhôm}}.$$
		
		Suy ra $m_\text{nhôm} = \SI{493.7}{kg}$.
	}
	
\end{enumerate}