\whiteBGstarBegin
\setcounter{section}{0}
\section{Trắc nghiệm}
\begin{enumerate}[label=\bfseries Câu \arabic*:]
	
	
	\item \mkstar{1}
	
	\cauhoi
	{Phát biểu nào sau đây là \textbf{không} đúng?
		\begin{mcq}
			\item Hạt electron là hạt mang điện tích âm, độ lớn $\SI{1.6e-19}{C}$.
			\item Hạt electron là hạt có khối lượng $m=\SI{9.1e-31}{kg}$.
			\item Nguyên tử trung hòa có thể mất hoặc nhận thêm electron để trở thành ion.
			\item Electron không thể chuyển từ vật này sang vật khác.
		\end{mcq}
		
	}
	\loigiai
	{	\textbf{Đáp án: D.}
		
		Electron có thể chuyển từ vật này sang vật khác, làm cho vật bị nhiễm điện.
	}
	\item \mkstar{1}
	
	\cauhoi
	{Nhận định nào dưới đây là \textbf{sai}?
		\begin{mcq}
			\item Hai điện tích cùng loại thì đẩy nhau.
			\item Hai điện tích khác loại thì hút nhau.
			\item Hai thanh nhựa giống nhau, sau khi cọ xát với len dạ, đưa chúng lại gần thì chúng sẽ hút nhau.
			\item Hai thanh thủy tinh sau khi cọ xát vào lụa, đưa chúng lại gần thì chúng sẽ đẩy nhau.
		\end{mcq}
		
	}
	\loigiai
	{	\textbf{Đáp án: C.}
		
		Hai thanh nhựa giống nhau, sau khi cọ xát với len dạ, đưa chúng lại gần thì chúng sẽ đẩy nhau vì chúng nhiễm điện cùng dấu.
	}
	\item \mkstar{2}
	
	\cauhoi
	{Cọ xát thanh ebonic vào miếng dạ, thanh ebonic tích điện âm vì
		\begin{mcq}
			\item electron chuyển từ thanh ebonic sang dạ.
			\item electron chuyển từ dạ sang thanh ebonic.
			\item proton chuyển từ dạ sang thanh ebonic.
			\item proton chuyển từ thanh ebonic sang dạ.
		\end{mcq}
		
	}
	\loigiai
	{	\textbf{Đáp án: B.}
		
		Cọ xát thanh ebonic vào miếng dạ, thanh ebonic tích điện âm vì electron chuyển từ dạ sang thanh ebonic.
	}
	\item \mkstar{2}
	
	\cauhoi
	{Đưa một thanh kim loại trung hòa về điện đặt trên một giá cách điện lại gần một quả cầu tích điện dương. Sau khi đưa thanh kim loại ra thật xa quả cầu thì thanh kim loại
		\begin{mcq}(2)
			\item có hai nửa tích điện trái dấu.
			\item tích điện dương.
			\item tích điện âm.
			\item trung hòa về điện.
		\end{mcq}
		
	}
	\loigiai
	{	\textbf{Đáp án: D.}
		
		Đưa một thanh kim loại trung hòa về điện đặt trên một giá cách điện lại gần một quả cầu tích điện dương. Sau khi đưa thanh kim loại ra thật xa quả cầu thì thanh kim loại trung hòa về điện.
	}
	\item \mkstar{2}
	
	\cauhoi
	{Hạt nhân của một nguyên tử oxi trung hòa có 8 proton và 9 nơtron. Số electron của nguyên tử oxi này là
		\begin{mcq}(4)
			\item 9.
			\item 16.
			\item 17.
			\item 8.
		\end{mcq}
		
	}
	\loigiai
	{	\textbf{Đáp án: D.}
		
		Trong 1 nguyên tử trung hòa thì số electron bằng số proton. Vậy số electron của nguyên tử oxi này là 8.
	}
	\item \mkstar{2}
	
	\cauhoi
	{Hai quả cầu bằng kim loại có bán kính như nhau, mang điện tích cùng dấu, một quả cầu đặc, một quả cầu rỗng. Ta cho hai quả cầu tiếp xúc với nhau thì
		\begin{mcq}
			\item điện tích của hai quả cầu bằng nhau.
			\item điện tích của quả cầu đặc lớn hơn điện tích của quả cầu rỗng.
			\item điện tích của quả cầu rỗng lớn hơn điện tích của quả cầu đặc.
			\item hai quả cầu đều trở nên trung hòa về điện.
		\end{mcq}
		
	}
	\loigiai
	{	\textbf{Đáp án: A.}
		
	}
	\item \mkstar{2}
	
	\cauhoi
	{Một thanh ebonic khi cọ xát với tấm dạ (cả 2 đặt cô lập về điện so với các vật khác) thì nó nhiễm điện $\SI{-3e-8}{C}$. Khi đó tấm dạ nhiễm điện
		\begin{mcq}(4)
			\item $\SI{-3e-8}{C}$.
			\item $\SI{-1.5e-8}{C}$.
			\item $\SI{3e-8}{C}$.
			\item $0$.
		\end{mcq}
		
	}
	\loigiai
	{	\textbf{Đáp án: C.}
		
		Theo định luật bảo toàn điện tích thì tấm dạ nhiễm điện $\SI{3e-8}{C}$.
	}
	\item \mkstar{2}
	
	\cauhoi
	{Nếu truyền cho một quả cầu trung hòa về điện $\SI{5e5}{}$ electron thì quả cầu mang điện tích là
		\begin{mcq}(4)
			\item $\SI{8e-14}{C}$.
			\item $\SI{-8e-14}{C}$.
			\item $\SI{-1.6e-24}{C}$.
			\item $\SI{1.6e-24}{C}$.
		\end{mcq}
		
	}
	\loigiai
	{	\textbf{Đáp án: B.}
		
		Điện tích của $\SI{5e5}{}$ electron:
		$$\SI{5e5}{} \cdot \SI{-1.6e-19}{C} = \SI{-8e-14}{C}.$$
	}
	\item \mkstar{2}
	
	\cauhoi
	{Nếu truyền cho một quả cầu trung hòa về điện $\SI{5e5}{}$ proton thì quả cầu mang điện tích là
		\begin{mcq}(4)
		\item $\SI{8e-14}{C}$.
		\item $\SI{-8e-14}{C}$.
		\item $\SI{-1.6e-24}{C}$.
		\item $\SI{1.6e-24}{C}$.
	\end{mcq}
		
	}
	\loigiai
	{	\textbf{Đáp án: A.}
		
		Điện tích của $\SI{5e5}{}$ proton:
		$$\SI{5e5}{} \cdot \SI{1.6e-19}{C} = \SI{8e-14}{C}.$$
	}
	\item \mkstar{2}
	
	\cauhoi
	{Tổng số proton và electron của một nguyên tử trung hòa về điện có thể là số nào sau đây?
		\begin{mcq}(4)
			\item 11.
			\item 13.
			\item 15.
			\item 16.
		\end{mcq}
		
	}
	\loigiai
	{	\textbf{Đáp án: D.}
		
		Trong nguyên tử trung hòa về điện, số electron và số proton bằng nhau, cộng lại phải là số chẵn (16).
	}
	\item \mkstar{2}
	
	\cauhoi
	{Nếu một nguyên tử đang có điện tích $\SI{-1.6e-19}{C}$ nhận thêm 1 electron thì nó sẽ trở thành
		\begin{mcq}(2)
			\item ion dương.
			\item nguyên tử trung hòa.
			\item ion âm.
			\item electron.
		\end{mcq}
		
	}
	\loigiai
	{	\textbf{Đáp án: C.}
		
		Nguyên tử đang có điện tích âm, khi nhận thêm 1 electron nữa thì nó vẫn có điện tích âm (ion âm).
	}
	\item \mkstar{2}
	
	\cauhoi
	{Vật bị nhiễm điện khi cọ xát là do
		\begin{mcq}
			\item electron chuyển từ vật này sang vật khác.
			\item vật bị nóng lên.
			\item các điện tích tự do được tạo ra trong vật.
			\item các điện tích bị mất đi.
		\end{mcq}
		
	}
	\loigiai
	{	\textbf{Đáp án: A.}
		
		Vật bị nhiễm điện khi cọ xát là do electron chuyển từ vật này sang vật khác.
	}
	\item \mkstar{2}
	
	\cauhoi
	{Đưa một cái đũa nhiễm điện lại gần những mẩu giấy nhỏ thì thấy mẩu giấy bị hút về phía đũa. Sau khi giấy chạm vào đũa thì
		\begin{mcq}
			\item mẩu giấy càng bị hút chặt vào đũa.
			\item mẩu giấy bị nhiễm điện tích trái dấu với đũa.
			\item mẩu giấy trở nên trung hòa về điện nên bị đũa đẩy ra.
			\item mẩu giấy bị đẩy ra khỏi đũa do nhiễm điện cùng dấu với đũa.
		\end{mcq}
		
	}
	\loigiai
	{	\textbf{Đáp án: D.}
		
		Đưa một cái đũa nhiễm điện lại gần những mẩu giấy nhỏ thì thấy mẩu giấy bị hút về phía đũa. Sau khi giấy chạm vào đũa thì mẩu giấy bị đẩy ra khỏi đũa do nhiễm điện cùng dấu với đũa.
	}
	\item \mkstar{2}
	
	\cauhoi
	{Một quả cầu nhôm rỗng bị nhiễm điện thì điện tích của quả cầu
		\begin{mcq}
			\item chỉ phân bố ở mặt trong của quả cầu.
			\item chỉ phân bố ở mặt ngoài của quả cầu.
			\item phân bố ở cả mặt trong và mặt ngoài của quả cầu.
			\item phân bố ở mặt trong nếu quả cầu nhiễm điện dương, phân bố ở mặt ngoài nếu quả cầu nhiễm điện âm.
		\end{mcq}
		
	}
	\loigiai
	{	\textbf{Đáp án: B.}
		
		Điện tích quả cầu chỉ phân bố ở mặt ngoài của quả cầu.
	}
	\item \mkstar{2}
	
	\cauhoi
	{Điện tích tập trung ở vị trí nào là nhiều nhất trong cột thu lôi?
		\begin{mcq}
			\item Ở chân cột thu lôi.
			\item Ở đỉnh cột thu lôi.
			\item Ở chính giữa cột thu lôi.
			\item Mọi vị trí của cột thu lôi đều có phân bố điện tích như nhau.
		\end{mcq}
		
	}
	\loigiai
	{	\textbf{Đáp án: B.}
		
		Điện tích tập trung ở đầu nhọn, tức là ở đỉnh cột thu lôi.
	}
	\item \mkstar{3}
	
	\cauhoi
	{Cho hai quả cầu nhỏ trung hòa điện đặt trong không khí, cách nhau $\SI{40}{cm}$. Giả sử có $\SI{4e12}{}$ electron từ quả cầu này di chuyển sang quả cầu kia. Khi đó, hai quả cầu sẽ
		\begin{mcq}(2)
			\item hút nhau.
			\item đẩy nhau.
			\item không hút cũng không đẩy.
			\item Chưa thể kết luận được.
		\end{mcq}
		
	}
	\loigiai
	{	\textbf{Đáp án: A.}
		
		Với 2 quả cầu trung hòa điện, khi có electron di chuyển từ quả cầu này sang quả cầu kia thì 1 quả nhiễm điện âm, 1 quả nhiễm điện dương. Do đó chúng hút nhau.
	}
	\item \mkstar{3}
	
	\cauhoi
	{Có 2 hạt bụi trong không khí, mỗi hạt chứa $\SI{5e8}{}$ electron, giữa hai hạt bụi cách nhau $\SI{1}{cm}$. Lực đẩy tĩnh điện giữa hai hạt bụi đó là
		\begin{mcq}(4)
			\item $\SI{5.76e-5}{N}$.
			\item $\SI{5.76e-6}{N}$.
			\item $\SI{5.76e-7}{N}$.
			\item $\SI{5.76e-9}{N}$.
		\end{mcq}
		
	}
	\loigiai
	{	\textbf{Đáp án: C.}
		
		Ta có $q_1=q_2=\SI{5e8}{} \cdot \SI{1.6e-19}{C} = \SI{80e-12}{C}$.
		
		Vậy:
		$$F=k\dfrac{q_1q_2}{r^2} = \SI{5.76e-7}{N}.$$
	}
	\item \mkstar{3}
	
	\cauhoi
	{Cho hai quả cầu kim loại kích thước giống nhau mang điện tích $\SI{-26.5}{\micro C}$ và $\SI{5.9}{\micro C}$ tiếp xúc với nhau, sau đó tách chúng ra. Điện tích của mỗi quả cầu sau khi tách có giá trị lần lượt là
		\begin{mcq}(2)
			\item $\SI{-16.2}{\micro C}$ và $\SI{16.2}{\micro C}$.
			\item $\SI{-16.2}{\micro C}$ và $\SI{-16.2}{\micro C}$.
			\item $\SI{-10.3}{\micro C}$ và $\SI{-10.3}{\micro C}$.
			\item $\SI{-10.3}{\micro C}$ và $\SI{10.3}{\micro C}$.
		\end{mcq}
		
	}
	\loigiai
	{	\textbf{Đáp án: C.}
		
		Ta có $q_1=-q_2$.
		
		Bảo toàn điện tích: $q_1' = q_2' = \dfrac{q_1+q_2}{2} = \SI{-10.3}{\micro C}$.
	}
	\item \mkstar{3}
	
	\cauhoi
	{Số phân tử khí có trong $\SI{1}{cm^3}$ khí hidro ở điều kiện tiêu chuẩn là
		\begin{mcq}(4)
			\item $\SI{1.69e19}{}$.
			\item $\SI{2.69e19}{}$.
			\item $\SI{1.69e20}{}$.
			\item $\SI{2.69e20}{}$.
		\end{mcq}
		
	}
	\loigiai
	{	\textbf{Đáp án: B.}
		
		Một mol khí hidro có $N_\text{A} = \SI{6.02e24}{}$ phân tử khí.
		
		Mà $\SI{1}{cm^3}$ khí hidro tương ứng với $\SI{e-3}{dm^3}$. Số mol là
		$$\dfrac{10^{-3} \cdot 1}{22,4} = \SI{44.64e-6}{mol}.$$
		
		Tổng số phân tử khí hidro là
		$$\SI{44.64e-6}{} \cdot \SI{6.02e23}{} = \SI{2.69e19}{}.$$
	}
	\item \mkstar{4}
	
	\cauhoi
	{Tổng điện tích dương và tổng điện tích âm trong $\SI{1}{cm^3}$ khí hidro ở điều kiện tiêu chuẩn là
		\begin{mcq}(2)
			\item $\SI{4.3e3}{C}$ và $\SI{-4.3e3}{C}$.
			\item $\SI{8.6e3}{C}$ và $\SI{-8.6e3}{C}$.
			\item $\SI{4.3}{C}$ và $\SI{-4.3}{C}$.
			\item $\SI{8.6}{C}$ và $\SI{-8.6}{C}$.
		\end{mcq}
		
	}
	\loigiai
	{	\textbf{Đáp án: C.}
		
		Một mol khí hidro có $N_\text{A} = \SI{6.02e24}{}$ phân tử khí, mà mỗi phân tử hidro có 1 điện tích âm và 1 điện tích dương.
		
		Mà $\SI{1}{cm^3}$ khí hidro tương ứng với $\SI{e-3}{dm^3}$. Số mol là
		$$\dfrac{10^{-3} \cdot 1}{22,4} = \SI{44.64e-6}{mol}.$$
		
		Tổng số phân tử khí hidro là
		$$\SI{44.64e-6}{} \cdot \SI{6.02e23}{} = \SI{2.69e19}{}.$$
		
		Vậy tổng điện tích âm là
		$$\SI{2.69e19}{} \cdot \SI{-1.6e-19}{C} = \SI{-4.3}{C}.$$
		
		Tổng điện tích dương là
		$$\SI{2.69e19}{} \cdot \SI{1.6e-19}{C} = \SI{4.3}{C}.$$
	}
\end{enumerate}

\whiteBGstarEnd

\loigiai
{
	\begin{center}
		\textbf{BẢNG ĐÁP ÁN}
	\end{center}
	\begin{center}
		\begin{tabular}{|m{2.8em}|m{2.8em}|m{2.8em}|m{2.8em}|m{2.8em}|m{2.8em}|m{2.8em}|m{2.8em}|m{2.8em}|m{2.8em}|}
			\hline
			1.D  & 2.C  & 3.B  & 4.D  & 5.D & 6.A  & 7.C  & 8.B  & 9.A  & 10.D  \\
			\hline
			11.C  & 12.A  & 13.D  & 14.B  & 15.B  & 16.A  & 17.C  & 18.C  & 19.B  & 20.C  \\
			\hline
		\end{tabular}
	\end{center}
}
\section{Tự luận}
\begin{enumerate}[label=\bfseries Câu \arabic*:]
	\item \mkstar{1}
	
	\cauhoi{
		Phát biểu định luật bảo toàn điện tích.
	}
	
	\loigiai{
		
		Trong một hệ cô lập về điện, tổng đại số các điện tích là không đổi
		$$\Sigma q = \text{hằng số}.$$
	}
	
	\item \mkstar{2}
	
	\cauhoi{
		Hãy vận dụng thuyết electron để giải thích các hiện tượng nhiễm điện.
	}
	
	\loigiai{
		
		\begin{itemize}
			\item Nhiễm điện do cọ xát: khi cọ xát thước nhựa vào vải, do ma sát nên một số electron ở bề mặt của thước nhựa đã dịch chuyển sang bên vải làm thước nhựa nhiễm điện;
			\item Nhiễm điện do tiếp xúc: do có sự dịch chuyển electron giữa quả cầu tích điện và quả cầu không tích điện, làm cho cả 2 quả cầu đều tích điện;
			\item Nhiễm điện do hưởng ứng: các electron tự do bị đẩy ra một đầu thanh, làm cho một đầu nhiễm điện âm, đầu còn lại nhiễm điện dương. Khi đưa thanh ra xa quả cầu nhiễm điện thì các electron tự do sắp xếp lại như trước, làm cho thanh trở nên trung hòa về điện.
		\end{itemize}
	}
	\item \mkstar{3}
	
	\cauhoi{
		Hai quả cầu nhỏ giống hệt nhau bằng kim loại đặt tại A và B trong không khí, có điện tích lần lượt là $q_1=\SI{-3.2e-7}{C}$, $q_2=\SI{2.4e-7}{C}$, cách nhau $\SI{12}{cm}$. Cho hai quả cầu tiếp xúc với nhau rồi đặt về chỗ cũ. Xác định lực tương tác giữa hai quả cầu lúc này.
	}
	
	\loigiai{
		
		Khi hai quả cầu tiếp xúc với nhau thì
		$$q_1' = q_2' = \dfrac{q_1 + q_2}{2} = \SI{-4e-8}{C}.$$
		
		Lực tương tác giữa hai quả cầu:
		$$F=k \dfrac{|q_1' q_2'|}{r^2} = \SI{e-3}{N}.$$
	}
	\item \mkstar{4}
	
	\cauhoi{
		Hai quả cầu nhỏ giống nhau bằng kim loại, mỗi quả cầu có khối lượng $\SI{5}{g}$, được treo vào cùng một điểm O bằng hai sợi chỉ không dãn, dài $\SI{10}{cm}$. Hai quả cầu tiếp xúc với nhau. Tích điện cho một quả cầu thì thấy hai quả cầu đẩy nhau đến khi hai dây treo hợp với nhau một góc $60^\circ$. Điện tích mà ta đã truyền cho quả cầu có độ lớn là bao nhiêu?
	}
	
	\loigiai{
		
		Tại vị trí cân bằng của mỗi quả cầu, ta có:
		$$\tan \dfrac{\alpha}{2} = \dfrac{F}{P} = \dfrac{kq^2}{r^2 mg}.$$
		
		Với $r=2l \sin \dfrac{\alpha}{2}$, ta được
		$$q=\SI{1.8e-7}{C}.$$
		
		Vì $q$ là điện tích của mỗi quả cầu sau khi tách ra, nên $2q$ là điện tích đã truyền cho một quả cầu lúc trước khi chúng tách ra. Vậy điện tích mà ta đã truyền cho quả cầu có độ lớn là $\SI{3.58e-7}{C}$.
	}
	\item \mkstar{4}
	
	\cauhoi{
		Hai quả cầu kim loại nhỏ giống nhau mang các điện tích $q_1$, $q_2$ đặt trong không khí và cách nhau một khoảng $r=\SI{20}{cm}$. Chúng hút nhau bằng một lực $F=\SI{3.6e-4}{N}$. Cho hai quả cầu tiếp xúc nhau rồi lại đưa về khoảng cách cũ thì chúng đẩy nhau bằng một lực $F' = \SI{2.025e4}{N}$. Tính điện tích $q_1$ và $q_2$.
	}
	
	\loigiai{
		
		Khi hai quả cầu chưa tiếp xúc nhau thì
		$$F=k \dfrac{|q_1q_2|}{r^2} \Rightarrow |q_1 q_2| = \SI{1.6e-15}{C^2}.$$
		
		Lực tương tác là lực hút nên $q_1 q_2 = \SI{-1.6e-15}{C^2}$.
		
		Sau khi hai quả cầu tiếp xúc nhau thì
		$$q_1' = q_2' = \dfrac{q_1 + q_2}{2}.$$
		
		Mà $F'=k \dfrac{|q_1' q_2'|}{r^2} \Rightarrow |q_1' q_2'| = \SI{9e-16}{C^2} \Rightarrow q_1' = \sqrt{|q_1' q_2'|} = \SI{3e-8}{C}$.
		
		Vậy $\dfrac{q_1+q_2}{2} = \SI{3e-8}{C}$.
		
		Ta được hệ phương trình:
		\begin{equation*}
			\begin{cases}
				q_1q_2 &= \SI{-1.6e-15}{C^2}\\
				q_1+q_2 &= \SI{6e-8}{C}
			\end{cases}
		\end{equation*}
	hoặc
		\begin{equation*}
			\begin{cases}
				q_1q_2 &= \SI{-1.6e-15}{C^2}\\
				q_1+q_2 &= \SI{-6e-8}{C}
			\end{cases}
		\end{equation*}
	
	Giải các hệ trên sẽ rút ra được 4 cặp nghiệm cần tìm.
	}
	
\end{enumerate}