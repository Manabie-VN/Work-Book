\whiteBGstarBegin
\setcounter{section}{0}
\section{Trắc nghiệm}
\begin{enumerate}[label=\bfseries Câu \arabic*:]
	
	
	\item \mkstar{1}
	
	\cauhoi
	{Chọn câu đúng. Điện năng tiêu thụ được đo bằng
		\begin{mcq}(2)
			\item Vôn kế.
			\item Công tơ điện.
			\item Ampe kế.
			\item Tĩnh điện kế.
		\end{mcq}
		
	}
	\loigiai
	{	\textbf{Đáp án: B.}
		
	}
	\item \mkstar{1}
	
	\cauhoi
	{Công suất điện được đo bằng đơn vị nào sau đây?
		\begin{mcq}(4)
			\item Jun (J).
			\item Oát (W).
			\item Niutơn (N).
			\item Culông (C).
		\end{mcq}
		
	}
	\loigiai
	{	\textbf{Đáp án: B.}
		
	}
	\item \mkstar{2}
	
	\cauhoi
	{Một bếp điện $\SI{115}{V}\ -\ \SI{1}{kW}$ bị cắm nhầm vào mạng điện 230 V được nối qua cầu chì chịu được dòng điện tối đa 15 A. Khi đó
		\begin{mcq}
			\item bếp điện sẽ có công suất tỏa nhiệt nhỏ hơn định mức.
			\item bếp điện sẽ có công suất tỏa nhiệt bằng định mức.
			\item bếp điện sẽ có công suất tỏa nhiệt lớn hơn định mức.
			\item dòng điện sẽ làm chảy cầu chì.
		\end{mcq}
		
	}
	\loigiai
	{	\textbf{Đáp án: D.}
		
		Dòng điện khi đó là
		$$I=\dfrac{U}{R} = \dfrac{U}{\dfrac{U_\text{đm}^2}{\calP}} \approx \SI{17.4}{A}.$$
		
		Vì giá trị trên vượt mức dòng điện tối đa, do đó sẽ làm chảy cầu chì.
	}
	\item \mkstar{2}
	
	\cauhoi
	{Hai bóng đèn Đ1 ($\SI{220}{V}\ -\ \SI{25}{W}$), Đ2 ($\SI{220}{V}\ -\ \SI{100}{W}$) khi sáng bình thường thì
		\begin{mcq}
			\item cường độ dòng điện qua Đ1 lớn gấp hai lần cường độ dòng điện qua Đ2.
			\item cường độ dòng điện qua Đ2 lớn gấp bốn lần cường độ dòng điện qua Đ1.
			\item cường độ dòng điện qua Đ1 bằng cường độ dòng điện qua Đ2.
			\item điện trở của Đ2 lớn gấp bốn lần điện trở của Đ1.
		\end{mcq}
		
	}
	\loigiai
	{	\textbf{Đáp án: B.}
		
		Khi sáng bình thường thì hiệu điện thế đặt vào hai đầu bóng đèn là $\SI{220}{V}$.
		
		Lập tỉ số:
		$$\dfrac{I_1}{I_2} = \dfrac{\calP_1}{\calP_2} = \dfrac{1}{4}.$$
		
		Vậy cường độ dòng điện qua Đ2 lớn gấp 4 lần cường độ dòng điện qua Đ1.
	}
	\item \mkstar{2}
	
	\cauhoi
	{Để đun sôi 2 lít nước bằng một ấm điện, người ta đã dùng hết $\SI{0.25}{}$ số điện. Điều này có nghĩa là
		\begin{mcq}(2)
			\item ta đã dùng $\SI{0.25}{kW/h}$ điện năng.
			\item ta đã dùng $\SI{0.25}{kWh}$ điện năng.
			\item ta đã dùng $\SI{0.25}{kW}$ điện năng.
			\item ta đã dùng $\SI{1.8e6}{J}$ điện năng.
		\end{mcq}
		
	}
	\loigiai
	{	\textbf{Đáp án: B.}
		
		$\SI{0.25}{}$ số điện ($\SI{0.25}{}$ kí điện) nghĩa là $\SI{0.25}{kWh}$ điện năng.
	}
	\item \mkstar{2}
	
	\cauhoi
	{Nếu đặt vào hai đầu điện trở $R$ một hiệu điện thế $U_1$ thì công suất của mạch là 10 W. Nếu đặt vào hai đầu điện trở $R$ một hiệu điện thế $U_2=2U_1$ thì công suất của mạch là
		\begin{mcq}(4)
			\item 5 W.
			\item 20 W.
			\item 40 W.
			\item 10 W.
		\end{mcq}
		
	}
	\loigiai
	{	\textbf{Đáp án: C.}
		
		Lập tỉ lệ:
		$$\dfrac{\calP_1}{\calP_2} = \dfrac{U_1^2}{U_2^2}=\dfrac{1}{4} \Rightarrow \calP_2 = 4\calP_1 = \SI{40}{W}.$$
	}
	\item \mkstar{2}
	
	\cauhoi
	{Trên một bóng đèn dây tóc có ghi $\SI{12}{V}\ -\ \SI{1.25}{A}$. Kết luận nào dưới đây là \textbf{sai}?
		\begin{mcq}
			\item Bóng đèn này luôn có công suất là 15 W khi hoạt động.
			\item Bóng đèn này chỉ có công suất là 15 W khi mắc vào hiệu điện thế 12 V.
			\item Bóng đèn này tiêu thụ điện năng 15 J trong 1 s khi hoạt động bình thường.
			\item Bóng đèn này có điện trở $\SI{9.6}{\Omega}$ khi hoạt động bình thường.
		\end{mcq}
		
	}
	\loigiai
	{	\textbf{Đáp án: A.}
		
		Khi bóng đèn hoạt động dưới hiệu điện thế $\SI{12}{V}$ thì tiêu thụ công suất định mức là
		$$\calP_\text{đm} = U_\text{đm} I_\text{đm} = \SI{15}{W}.$$
		
		Khi bóng đèn hoạt động dưới hiệu điện thế khác $\SI{12}{V}$ thì tiêu thụ công suất khác với công suất định mức.
		
		Vậy khi nói rằng "bóng đèn này luôn có công suất là $\SI{15}{W}$ khi hoạt động" là sai. 
	}
	\item \mkstar{2}
	
	\cauhoi
	{Trong đoạn mạch chỉ có điện trở thuần, với thời gian như nhau, nếu cường độ dòng điện giảm 2 lần thì nhiệt lượng tỏa ra trên mạch
		\begin{mcq}(2)
			\item tăng 2 lần.
			\item giảm 4 lần.
			\item tăng 4 lần.
			\item giảm còn $\dfrac{1}{2}$.
		\end{mcq}
		
	}
	\loigiai
	{	\textbf{Đáp án: B.}
		
		Nhiệt lượng tỏa ra:
		$$Q=I^2 Rt$$
		
		Khi $I$ giảm 2 lần thì $Q$ giảm 4 lần.
	}
	\item \mkstar{2}
	
	\cauhoi
	{Một bóng đèn loại 6 V - 6 W. Điện trở của bóng đèn này là
		\begin{mcq}(4)
			\item $\SI{1}{\Omega}$.
			\item $\SI{3}{\Omega}$.
			\item $\SI{4}{\Omega}$.
			\item $\SI{6}{\Omega}$.
		\end{mcq}
		
	}
	\loigiai
	{	\textbf{Đáp án: D.}
		
		Điện trở của bóng đèn:
		$$R=\dfrac{U_\text{đm}^2}{\calP_{\text{đm}}} = \SI{6}{\Omega}.$$
	}
	\item \mkstar{2}
	
	\cauhoi
	{Điện trở $R_1$ tiêu thụ một công suất $\calP$ khi được mắc vào hiệu điện thế $U$ không đổi. Nếu mắc nối tiếp $R_1$ với một điện trở $R_2$ rồi mắc vào hiệu điện thế $U$ nói trên thì công suất tiêu thụ trên $R_1$ so với lúc đầu sẽ
		\begin{mcq}(2)
			\item giảm.
			\item không đổi.
			\item tăng.
			\item có thể tăng hoặc giảm.
		\end{mcq}
		
	}
	\loigiai
	{	\textbf{Đáp án: A.}
		
		Khi mắc nối tiếp $R_1$ và $R_2$ thì điện trở tương đương của mạch tăng: $R=R_1 + R_2$.
		
		Khi đó cường độ dòng điện qua mạch giảm $\left(I=\dfrac{U}{R}\right)$, công suất tiêu thụ trên $R_1$ giảm $(\calP_1 = I^2 R_1)$.
	}
	\item \mkstar{2}
	
	\cauhoi
	{Dòng điện không đổi có cường độ 2 A chạy qua một điện trở $\SI{200}{\Omega}$. Nhiệt lượng tỏa ra trên điện trở đó trong 40 s là
		\begin{mcq}(4)
			\item 20 kJ.
			\item 30 kJ.
			\item 32 kJ.
			\item 16 kJ.
		\end{mcq}
		
	}
	\loigiai
	{	\textbf{Đáp án: C.}
		
		Nhiệt lượng tỏa ra trên điện trở:
		$$Q=I^2 R t = \SI{32000}{J} = \SI{32}{kJ}.$$
	}
	\item \mkstar{2}
	
	\cauhoi
	{Có hai điện trở gồm $R_1$ và $R_2$, với $R_1=2R_2$, mắc nối tiếp vào đoạn mạch có hiệu điện thế không đổi. Goi công suất tỏa nhiệt trên điện trở $R_1$ là $\calP_1$, công suất tỏa nhiệt trên điện trở $R_2$ là
		\begin{mcq}(4)
			\item $\calP_2 = 2 \calP_1$.
			\item $\calP_2 = \calP_1$.
			\item $\calP_2 = \dfrac{1}{2} \calP_1$.
			\item $\calP_2 = 4 \calP_1$.
		\end{mcq}
		
	}
	\loigiai
	{	\textbf{Đáp án: C.}
		
		Lập tỉ lệ:
		$$\dfrac{\calP_1}{\calP_2} = \dfrac{R_1}{R_2} = \dfrac{2}{1} \Rightarrow \calP_2 = \dfrac{1}{2} \calP_1.$$
	}
	\item \mkstar{2}
	
	\cauhoi
	{Một nguồn điện có suất điện động 3 V khi mắc với một bóng đèn tạo thành một mạch kín thì có dòng điện có cường độ là $\SI{0.2}{A}$. Khi đó, công suất của nguồn điện này là
		\begin{mcq}(4)
			\item 10 W.
			\item 30 W.
			\item $\SI{0.9}{W}$.
			\item $\SI{0.1}{W}$.
		\end{mcq}
		
	}
	\loigiai
	{	\textbf{Đáp án: C.}
		
		Công suất của nguồn điện:
		$$\calP = \calE I = \SI{0.9}{W}.$$
	}
	\item \mkstar{2}
	
	\cauhoi
	{Tính điện năng tiêu thụ khi dòng điện có cường độ 1 A chạy qua dây dẫn trong 1 giờ. Biết hiệu điện thế giữa hai đầu dây dẫn này là 6 V.
		\begin{mcq}(4)
			\item $\SI{21600}{J}$.
			\item $\SI{216000}{J}$.
			\item $\SI{2160}{J}$.
			\item $\SI{21600}{J}$.
		\end{mcq}
		
	}
	\loigiai
	{	\textbf{Đáp án: A.}
		
		Điện năng tiêu thụ:
		$$A=UI t = \SI{21600}{J}.$$
	}
	\item \mkstar{2}
	
	\cauhoi
	{Tính công suất tiêu thụ khi dòng điện có cường độ 1 A chạy qua dây dẫn trong 1 giờ. Biết hiệu điện thế giữa hai đầu dây dẫn này là 6 V.
		\begin{mcq}(4)
			\item $\SI{0.6}{W}$.
			\item $\SI{6}{W}$.
			\item $\SI{60}{W}$.
			\item $\SI{600}{W}$.
		\end{mcq}
		
	}
	\loigiai
	{	\textbf{Đáp án: B.}
		
		Công suất tiêu thụ:
		$$\calP = UI = \SI{6}{W}.$$
	}
	\item \mkstar{3}
	
	\cauhoi
	{Khi hai điện trở giống nhau mắc song song vào một hiệu điện thế $U$ không đổi thì công suất tiêu thụ của chúng là 20 W. Nếu mắc chúng nối tiếp nhau rồi mắc vào hiệu điện thế như trên thì công suất tiêu thụ lúc này là
		\begin{mcq}(4)
			\item 5 W.
			\item 10 W.
			\item 40 W.
			\item 80 W.
		\end{mcq}
		
	}
	\loigiai
	{	\textbf{Đáp án: A.}
		
		Do $U$ không đổi nên ta có tỉ số:
		$$\dfrac{\calP_1}{\calP_2} = \dfrac{R_2}{R_1} = \dfrac{2R}{R/2} = 4.$$
		
		Vậy $\calP_2 = \dfrac{\calP_1}{4} = \SI{5}{W}$.
	}
	\item \mkstar{3}
	
	\cauhoi
	{Hai bóng đèn có công suất định mức bằng nhau, hiệu điện thế định mức của chúng lần lượt là $U_1=\SI{110}{V}$ và $U_2=\SI{220}{V}$. Tỉ số điện trở của chúng là
		\begin{mcq}(4)
			\item $\dfrac{R_1}{R_2} = \dfrac{1}{2}$.
			\item $\dfrac{R_1}{R_2} = \dfrac{2}{1}$.
			\item $\dfrac{R_1}{R_2} = \dfrac{1}{4}$.
			\item $\dfrac{R_1}{R_2} = \dfrac{4}{1}$.
		\end{mcq}
		
	}
	\loigiai
	{	\textbf{Đáp án: C.}
		
		Do $\calP$ không đổi nên ta có tỉ số:
		$$\dfrac{R_1}{R_2} = \dfrac{U_1^2}{U_2^2} = \dfrac{1}{4}.$$
	}
	\item \mkstar{3}
	
	\cauhoi
	{Một chiếc pin điện thoại có ghi ($\SI{3.6}{V}\ -\ \SI{900}{mAh}$). Điện thoại sau khi sạc đầy, pin có thể dùng để nghe gọi liên tục trong $\SI{4.5}{h}$. Bỏ qua mọi hao phí, công suất điện tiêu thụ trung bình của điện thoại trong quá trình đó là
		\begin{mcq}(4)
			\item $\SI{3.6}{W}$.
			\item $\SI{0.36}{W}$.
			\item $\SI{0.72}{W}$.
			\item $\SI{7.2}{W}$.
		\end{mcq}
		
	}
	\loigiai
	{	\textbf{Đáp án: C.}
		
		Điện năng của pin có được khi đã sạc đầy là
		$$A = \SI{3.6}{V} \cdot \SI{900e-3}{A} \cdot \SI{3600}{s} = \SI{11664}{J}.$$
		
		Công suất tiêu thụ trung bình:
		$$\calP = \dfrac{A}{t} = \SI{0.72}{W}.$$
	}
	\item \mkstar{3}
	
	\cauhoi
	{Một ắcquy có suất điện động là 12 V, sinh ra một công 720 J khi dịch chuyển điện tích ở bên trong giữa hai cực của nó khi ắcquy này phát điện. Biết thời gian dịch chuyển điện lượng này là 5 phút. Cường độ dòng điện chạy qua ắcquy khi đó là
		\begin{mcq}(4)
			\item $\SI{0.2}{A}$.
			\item 2 A.
			\item $\SI{1.2}{A}$.
			\item 12 A.
		\end{mcq}
		
	}
	\loigiai
	{	\textbf{Đáp án: A.}
		
		Áp dụng công thức:
		$$A=\calE I t \Rightarrow I = \SI{0.2}{A}.$$
	}
	\item \mkstar{4}
	
	\cauhoi
	{Một quạt điện (loại đứng) sử dụng dòng điện với hiệu điện thế 220 V và dòng điện chạy qua quạt có cường độ $\SI{1.41}{A}$. Tính số tiền điện phải trả cho chiếc quạt này trong 30 ngày, mỗi ngày sử dụng 4 giờ. Cho biết đơn giá điện cho mỗi kWh điện là 1720 đồng và coi như quạt luôn hoạt động bình thường.
		\begin{mcq}(4)
			\item 62 000 đồng.
			\item 64 025 đồng.
			\item 32 000 đồng.
			\item 34 000 đồng.
		\end{mcq}
		
	}
	\loigiai
	{	\textbf{Đáp án: B.}
		
		Điện năng quạt sử dụng trong 1 giây:
		$$A=UI = \SI{310.2}{J}.$$
		
		Điện năng quạt sử dụng trong 4 giờ (mỗi ngày):
		$$A'=A \cdot \SI{14400}{s} = \SI{4466880}{J}.$$
		
		Điện năng quạt sử dụng trong 30 ngày:
		$$A'' = A' \cdot 30 = \SI{134e6}{J} = \SI{37.224}{kWh}.$$
		
		Số tiền điện phải trả là $\SI{64025}{}$ đồng.
	}
\end{enumerate}

\whiteBGstarEnd

\loigiai
{
	\begin{center}
		\textbf{BẢNG ĐÁP ÁN}
	\end{center}
	\begin{center}
		\begin{tabular}{|m{2.8em}|m{2.8em}|m{2.8em}|m{2.8em}|m{2.8em}|m{2.8em}|m{2.8em}|m{2.8em}|m{2.8em}|m{2.8em}|}
			\hline
			1.B  & 2.B  & 3.D  & 4.B  & 5.B  & 6.C  & 7.A  & 8.B  & 9.D  & 10.A  \\
			\hline
			11.C  & 12.C  & 13.C  & 14.A  & 15.B  & 16.A  & 17.C  & 18.C  & 19.A  & 20.B  \\
			\hline
		\end{tabular}
	\end{center}
}
\section{Tự luận}
\begin{enumerate}[label=\bfseries Câu \arabic*:]
	\item \mkstar{1}
	
	\cauhoi{
		Điện năng mà một mạch điện tiêu thụ được đo bằng công do lực nào thực hiện? Viết công thức tính điện năng tiêu thụ và công suất điện của một đoạn mạch khi có dòng điện chạy qua.
	}
	
	\loigiai{
		
		Điện năng mà một đoạn mạch tiêu thụ được đo bằng công của lực điện thực hiện.
		
		Công thức tính điện năng tiêu thụ: $A=qU = UI t$.
		
		Công thức tính công suất điện: $\calP = \dfrac{A}{t} = UI$.
	}
	
	\item \mkstar{2}
	
	\cauhoi{
		Tính điện năng tiêu thụ và công suất điện khi dòng điện có cường độ 1 A chạy qua dây dẫn trong 1 giờ, biết hiệu điện thế giữa hai đầu dây dẫn là 6 V.
	}
	
	\loigiai{
		
			Điện năng tiêu thụ của đoạn mạch:
		$$A=UI t = \SI{21600}{J}.$$
		
		Công suất điện:
		$$\calP = UI = \SI{6}{W}.$$
	}
	\item \mkstar{3}
	
	\cauhoi{
		Trên nhãn của một ấm điện có ghi 220 V - 1000 W.
		\begin{enumerate}
			\item Cho biết ý nghĩa của các chỉ số trên;
			\item Sử dụng ấm điện với hiệu điện thế 220 V để đun sôi 2 lít nước từ $\SI{25}{\celsius}$. Tính thời gian để đun sôi nước, biết hiệu suất của ấm là $90\%$ và nhiệt dung riêng của nước là $\SI{4190}{J/(kg K)}$.
		\end{enumerate}
	}
	
	\loigiai{
		
	\begin{enumerate}
		\item Cho biết ý nghĩa của các chỉ số trên;
		
		Số $\SI{220}{V}$ là điện áp định mức của ấm.
		
		Số $\SI{1000}{W}$ là công suất định mức của ấm khi hoạt động bình thường ở $\SI{220}{V}$.
		
		\item Sử dụng ấm điện với hiệu điện thế 220 V để đun sôi 2 lít nước từ $\SI{25}{\celsius}$. Tính thời gian để đun sôi nước, biết hiệu suất của ấm là $90\%$ và nhiệt dung riêng của nước là $\SI{4190}{J/(kg K)}$.
		
		Đổi $\SI{2}{l} \rightarrow \SI{2}{kg}$.
		
		Nhiệt lượng cần thiết để đun nước:
		$$Q_1 = mc\Delta t = \SI{628500}{J}.$$
		
		Năng lượng thực tế cần thiết để cung cấp cho ấm:
		$$Q_1 = \dfrac{Q_1}{H} = \SI{698333.3}{J}.$$
		
		Thời gian đun nước:
		$$\calP = \dfrac{Q_2}{t} \Rightarrow t = \SI{698.3}{s}.$$
	\end{enumerate}
	}
	\item \mkstar{4}
	
	\cauhoi{
		Hai bóng đèn có công suất định mức lần lượt là 25 W và 100 W đều làm việc bình thường ở hiệu điện thế 110 V. Khi mắc nối tiếp hai bóng đèn này vào mạng điện 220 V thì đèn nào sẽ dễ hỏng hơn?
	}
	
	\loigiai{
		
		Cường độ dòng điện định mức qua đèn 1:
		$$I_1 = \dfrac{\calP_1}{U_1} = \xsi{\dfrac{5}{2}}{A}.$$
		
		Cường độ dòng điện định mức qua đèn 2:
		$$I_2 = \dfrac{\calP_2}{U_2} = \xsi{\dfrac{20}{2}}{A}.$$
		
		Điện trở của đèn 1:
		$$R_1 = \dfrac{U_1^2}{\calP_1} = \SI{484}{\Omega}.$$
		
		Điện trở của đèn 2:
		$$R_2 = \dfrac{U_2^2}{\calP_2} = \SI{212}{\Omega}.$$
		
		Nếu mắc hai bóng đèn này vào mạng điện $\SI{220}{V}$ thì cường độ dòng điện qua mạch là
		$$I=\dfrac{U}{R_1 + R_2} = \xsi{\dfrac{8}{22}}{A}.$$
		
		Do $I_1 > I_2$ nên đèn 1 (đèn công suất $\SI{25}{W}$) sẽ dễ hỏng hơn.
	}
	\item \mkstar{4}
	
	\cauhoi{
		Giả sử hiệu điện thế đặt vào hai đầu bóng đèn 220 V - 100 W đột ngột tăng lên tới 240 V thì công suất điện của bóng đèn khi đó tăng hay giảm bao nhiêu phần trăm so với công suất định mức của nó? Cho rằng điện trở của bóng đèn luôn không đổi.
	}
	
	\loigiai{
		
		Điện trở của bóng đèn:
		$$R=\dfrac{U^2}{\calP} = \SI{484}{\Omega}.$$
		
		Khi hiệu điện thế tăng lên tới $\SI{240}{V}$ thì công suất của đèn khi đó là
		$$\calP' = \dfrac{U'^2}{R} = \SI{119}{W}.$$
		
		Phần trăm tăng lên của công suất:
		$$\Delta P = \dfrac{\calP' - \calP}{\calP} = \SI{19}{\percent}.$$
	}
	
\end{enumerate}