\whiteBGstarBegin

\begin{enumerate}[label=\bfseries Câu \arabic*:]
	
	\item \mkstar{1}
	
	\cauhoi
	{Câu nào sau đây là đúng khi nói về sự rơi?
		\begin{mcq}
			\item Khi không có sức cản vật nặng rơi nhanh hơn vật nhẹ.
			\item Ở cùng một nơi, mọi vật rơi tự do có cùng gia tốc.
			\item Khi rơi tự do, vật nào ở độ cao lớn hơn sẽ rơi với gia tốc lớn hơn.
			\item Vận tốc của vật chạm đất không phụ thuộc vào độ cao của vật khi rơi.
		\end{mcq}
	}
	
	\loigiai
	{\textbf{Đáp án: B.}
		
		Gia tốc rơi tự do không phụ thuộc vào khối lượng của vật, chỉ phụ thuộc vào vĩ độ địa lý, độ cao và cấu trúc địa chất nơi đó nên ở cùng một nơi, mọi vật rơi tự do có cùng gia tốc.
		
		
	}
	\item \mkstar{1}
	
	\cauhoi
	{Chuyển động của vật nào dưới đây được coi là chuyển động tròn đều?
		\begin{mcq}
			\item Chuyển động của bánh xe ôtô khi đang hãm phanh.
			\item Chuyển động quay của cánh quạt khi quay ổn định.
			\item Chuyển động quay của các điểm treo các ghế ngồi trên chiếc đu quay.
			\item Chuyển động quay của cánh quạt khi vừa tắt điện.
		\end{mcq}
		
	}
	\loigiai
	{\textbf{Đáp án: B.}
		
		Chuyển động quay của cánh quạt khi quay ổn định là chuyển động tròn đều.
		
	}
	\item \mkstar{1}
	
	\cauhoi
	{Các công thức liên hệ giữa tốc độ góc $\omega$ với chu kỳ $T$ và giữa tốc độ góc $\omega$ với tần số $f$ trong chuyển động tròn đều là gì?
		\begin{mcq}(2)
			\item $\omega=2\pi T$; $\omega=\dfrac{2\pi}{f}$.
			\item $\omega=\dfrac{2\pi}{T}$; $\omega=2\pi f$.
			\item $\omega=2\pi T$; $\omega=2\pi f$. 
			\item $\omega=\dfrac{2\pi}{T}$; $\omega=\dfrac{2\pi}{f}$.
		\end{mcq}
		
	}
	\loigiai
	{\textbf{Đáp án: B.}
		
		Công thức liên hệ giữa tốc độ góc $\omega$ với chu kỳ $T$ là $\omega=\dfrac{2\pi}{T}$.
		Công thức liên hệ giữa giữa tốc độ góc $\omega$ với tần số $f$ trong chuyển động tròn đều là $\omega=2\pi f$.
	
	}
	
	\item \mkstar{2}
	
	\cauhoi
	{Người ta thả một vật rơi tự do, sau $\SI{5}{\second}$ vật chạm đất, $g=\SI{9.8}{\meter/\second^2}$. Độ cao thả vật là
		\begin{mcq}(4)
			\item $\SI{122.5}{\meter}$.
			\item $\SI{61.25}{\meter}$.
			\item $\SI{254}{\meter}$.
			\item $\SI{183,75}{\meter}$.
		\end{mcq}
	}
	\loigiai
	{\textbf{Đáp án: A.}	
		
		Độ cao lúc thả vật là
		$h=s=\dfrac{1}{2}gt^2=\dfrac{1}{2}\cdot\SI{9.8}{\meter/\second^2}\cdot(\SI{5}{\second})^2=\SI{122.5}{\meter}.$
		
		
	}
	\item \mkstar{2}
	
	\cauhoi
	{Cùng một lúc tại mái nhà, bi A được thả rơi còn bi B được ném theo phương ngang. Bi A có khối lượng gấp đôi bi B. Bỏ qua sức cản của không khí thì
		\begin{mcq}
			\item bi A chạm đất trước bi B.
			\item bi B chạm đất trước bi A.
			\item cả hai chạm đất cùng lúc.
			\item chưa đủ điều kiện để kết luận bi A hay bi B chạm đất trước.
		\end{mcq}
		
	}
	\loigiai
	{\textbf{Đáp án: C.}
		
		Sự rơi tự do nhanh hay chậm không phụ thuộc vào khối lượng $\left( t=\sqrt{\dfrac{2h}{g}}\right)$.
		
		Do đó hai bi chạm đất cùng lúc.
		
		
	}
	
	\item \mkstar{2}
	
	\cauhoi
	{Một bánh xe có đường kính $\SI{100}{\centi\meter}$ lăn đều với vận tốc $\SI{36}{\kilo\meter/\hour}$. Gia tốc hướng tâm của một điểm trên vành bánh xe có độ lớn
		\begin{mcq}(4)
			\item $\SI{200}{\meter/\second^2}$.
			\item $\SI{400}{\meter/\second^2}$.
			\item $\SI{100}{\meter/\second^2}$.
			\item $\SI{300}{\meter/\second^2}$.
		\end{mcq}
	}
	\loigiai
	{\textbf{Đáp án: A.}
		
		Đổi đơn vị: $\SI{100}{\centi\meter}=\SI{1}{\meter}$; $\SI{36}{\kilo\meter/\hour}=\SI{10}{\meter/\second}$
		
		Gia tốc hướng tâm của một điểm trên vành bánh xe có độ lớn:
		$$a_\text{ht}=\dfrac{v^2}{R}=\SI{200}{\meter/\second^2}.$$
		
		
	}
	\item \mkstar{2}
	
	\cauhoi
	{Thả hai vật rơi tự do đồng thời từ hai độ cao $h_1$ và $h_2$. Biết rằng thời gian chạm đất của vật thứ nhất bằng 2 lần của vật thứ hai. Tỉ số $\dfrac{h_1}{h_2}$ là
		\begin{mcq}(4)
			\item $\dfrac{h_1}{h_2}=4$.
			\item $\dfrac{h_1}{h_2}=\dfrac{1}{2}$.
			\item $\dfrac{h_1}{h_2}=2$.
			\item $\dfrac{h_1}{h_2}=\dfrac{1}{4}$.
		\end{mcq}
	}
	\loigiai
	{\textbf{Đáp án: A.}
		
		Tỉ số $\dfrac{h_1}{h_2}$ có thể được suy ra từ việc lập tỉ số độ cao của hai vật: $$\dfrac{h_1}{h_2}=\dfrac{\dfrac{1}{2}gt_1^2}{\dfrac{1}{2}gt_2^2}=\dfrac{t_1^2}{t_2^2}=\left( \dfrac{t_1}{t_2}\right)^2=4.$$
		
		
	}
	\item \mkstar{2}
	
	\cauhoi
	{Một đĩa tròn bán kính $\SI{20}{\centi\meter}$ quay đều quanh trục của nó. Đĩa quay hết 1 vòng mất $\SI{0.2}{\second}$. Tốc độ dài $v$ của một điểm nằm ở mép đĩa bằng
		\begin{mcq}(4)
			\item $\SI{4,71}{\meter/\second}$. 
			\item $\SI{3,14}{\meter/\second}$. 
			\item $\SI{6,28}{\meter/\second}$. 
			\item $\SI{7,85}{\meter/\second}$. .
		\end{mcq}
		
	}
	\loigiai
	{\textbf{Đáp án: C.}
		
		Tốc độ dài $v$ của một điểm trên vành ngoài xe:
		$v=r\omega=r\dfrac{\Delta \alpha}{\Delta t}=\SI{0,2}{\meter}\dfrac{2\pi}{\SI{0.2}{\second}}=\SI{6,28}{\meter/\second}$
		
		
	}
	\item \mkstar{2}
	
	\cauhoi
	{Một động cơ xe máy có trục quay 1200 vòng/ phút. Tốc độ góc của chuyển động quay là bao nhiêu?
		\begin{mcq}(4)
			\item $\SI{125,7}{\radian/\second}$.
			\item $\SI{188,5}{\radian/\second}$.
			\item $\SI{62,8}{\radian/\second}$.
			\item $\SI{7200}{\radian/\second}$.
		\end{mcq}
	}
	\loigiai
	{\textbf{Đáp án: A.}
		
		Tốc độ góc của chuyển động quay là:
		$\omega$ = 1200 vòng/ phút = $1200\cdot\dfrac{2\pi}{60}\,\SI{}{\radian/\second}=\SI{125,7}{\radian/\second}$
		
		
	}
	\item \mkstar{2}
	
	\cauhoi
	{Một bánh xe có bán kính $\SI{100}{\centi\meter}$ lăn đều với vận tốc $\SI{54}{\kilo\meter/\hour}$. Gia tốc hướng tâm của một điểm trên vành bánh xe có độ lớn
		\begin{mcq}(4)
			\item $\SI{225}{\meter/\second^2}$.
			\item $\SI{400}{\meter/\second^2}$.
			\item $\SI{100}{\meter/\second^2}$.
			\item $\SI{300}{\meter/\second^2}$.
		\end{mcq}
	}
	\loigiai
	{\textbf{Đáp án: A.}
		
		Gia tốc hướng tâm của một điểm trên vành bánh xe có độ lớn:
		$$a_\text{ht}=\dfrac{v^2}{R}=\SI{225}{\meter/\second^2}.$$
		
		
	}
	\item \mkstar{2}
	
	\cauhoi
	{Một hòn đá thả rơi tự do từ đỉnh toà nhà $25$ tầng nó chạm đất trong thời gian $5\ \text{s}$. Lấy $g=10\ \text{m/s}^2$. Trong giây đầu tiên hòn đá đã đi qua số tầng của toà nhà là
		\begin{mcq}(4)
			\item $1$.
			\item $2$.
			\item $3$.
			\item $5$.
		\end{mcq}
	}
	\loigiai
	{\textbf{Đáp án: A.}
		
		Gọi chiều cao mỗi tầng nhà là $h$. 
		
		Suy ra $25h=g\dfrac{5^2}{2}\Rightarrow h=5\ \text{m}$.
		
		Gọi $n$ là số tầng trong giây đầu hòn đá rơi được.
		
		Suy ra $n\cdot 5=g\cdot \dfrac{1^2}{g}\Rightarrow n=1$.
		
		
	}
	\item \mkstar{3}
	
	\cauhoi
	{Từ đỉnh tháp hai vật A và B được thả rơi tự do. Biết B được thả rơi sau A $1\ \text{s}$. Lấy $g=10\ \text{m/s}^2$. Khoảng cách giữa A và B tại thời điểm sau khi B rơi được $2\ \text{s}$ là
		\begin{mcq}(4)
			\item $5\ \text{m}$.
			\item $10\ \text{m}$.
			\item $20\ \text{m}$.
			\item $25\ \text{m}$.
		\end{mcq}
		
	}
	\loigiai
	{\textbf{Đáp án: D.}
		
		Tại thời điểm sau khi B rơi được $2\ \text{s}$, A đã rơi được $3\ \text{s}$
		
		Suy ra, khoảng cách giữa A và B là
		
		$$\Delta h=g\cdot\dfrac{t_1^2}{2}-g\cdot\dfrac{t_2^2}{2}=\dfrac{10}{2}\cdot \left( 3^2-2^2\right)=25\ \text{m}$$
		
		
		
	}
	\item \mkstar{3}
	
	\cauhoi
	{Một người thả vật rơi tự do, vật chạm đất có $v=\SI{36}{\meter/\second}$, $g=\SI{10}{\meter/\second^2}$. Độ cao của vật sau khi thả được $\SI{3}{\second}$ là
		\begin{mcq}(4)
			\item $\SI{64.8}{\meter}$
			\item $\SI{19.8}{\meter}$
			\item $\SI{86.4}{\meter}$
			\item $\SI{45.0}{\meter}$
		\end{mcq}
	}
	\loigiai
	{\textbf{Đáp án: B.}
		
		Độ cao nơi thả vật:
		$$v=\sqrt{2gh}\Rightarrow h = \SI{64.8}{\meter}.$$
		Quãng đường vật rơi $\SI{3}{\second}$ đầu tiên là
		$$s_3=\dfrac{1}{2}gt^2=\dfrac{1}{2}\cdot\SI{10}{\meter/\second^2}\cdot(\SI{3}{\second})^2=\SI{45}{\meter}.$$
		Độ cao của vật lúc này:
		$$h=s-s_3=\SI{64.8}{\meter}-\SI{45}{\meter}=\SI{19.8}{\meter}.$$
		
		
	}
	\item \mkstar{3}
	
	\cauhoi
	{Các giọt nước mưa đang rơi từ mái nhà xuống sau những khoảng thời gian bằng nhau. Khi giọt thứ nhất chạm đất thì giọt thứ năm bắt đầu rơi, lúc đó khoảng cách giữa giọt thứ nhất và giọt thứ hai là $\SI{14}{\meter}$. Lấy $g=\SI{10}{\meter/\second^2}$. Độ cao mái nhà là
		\begin{mcq}(4)
			\item $\SI{32}{\meter}$.
			\item $\SI{9}{\meter}$.
			\item $\SI{56}{\meter}$.
			\item $\SI{16}{\meter}$.
		\end{mcq}
	}
	\loigiai
	{\textbf{Đáp án: A.}
		
		Gọi $\Delta t$ khoảng thời gian bằng nhau các giọt nước mưa đang rơi từ mái nhà xuống.
		
		Quãng đường giọt thứ nhất rơi cho đến khi chạm đất là
		$$s_1=\dfrac{1}{2}g(4\Delta t)^2.$$
		Quãng đường giọt thứ hai rơi cho đến khi giọt thứ nhất chạm đất là
		$$s_2=\dfrac{1}{2}g(3\Delta t)^2.$$
		Theo đề bài ta có:
		$$\Delta s= s_1-s_2=\dfrac{7}{2}g(\Delta t)^2=\SI{14}{\meter}\Rightarrow \Delta t=\dfrac{\sqrt{10}}{5}\,\text{s}.$$
		Độ cao mái nhà là
		$$h=s_1=\dfrac{1}{2}g(4\Delta t)^2=\SI{32}{\meter}.$$
		
		
	}
	\item \mkstar{3}
	
	\cauhoi
	{Một vệ tinh nhân tạo ở độ cao $\SI{250}{\kilo\meter}$ bay quanh Trái Đất theo một quỹ đạo tròn. Chu kì của vệ tinh là 88 phút. Tính tốc độ góc và gia tốc hướng tâm của vệ tinh. Cho bán kính Trái Đất là $\SI{6400}{\kilo\meter}$.
		\begin{mcq}(4)
			\item $\SI{9,41}{ \meter/\second^2}$.
			\item $\SI{9,48}{ \meter/\second^2}$.
			\item $\SI{8,72}{ \meter/\second^2}$.
			\item $\SI{10,05}{ \meter/\second^2}$.
		\end{mcq}
	}
	\loigiai
	{	\textbf{Đáp án: A.}
		
		Khoảng cách từ vệ tinh đến tâm Trái Đất: 
		$$r=\SI{250}{\kilo\meter}+\SI{6400}{\kilo\meter} =\SI{6650}{\kilo\meter}=\SI{6650000}{\meter}.$$
		
		Tốc độ góc của vệ tinh:
		$$T=\frac{2\pi}{\omega} \Rightarrow \omega = \frac{\pi}{2640}\ \text{rad/s}.$$ 
		
		Gia tốc hướng tâm của vệ tinh: 
		
		$$a_{ht}=\omega^2 \cdot r \approx \SI{9,41}{ \meter/\second^2}.$$
		
		
	
	}
	\item \mkstar{3}
	
	\cauhoi
	{Một hòn đá rơi tự do từ cửa sổ một toà nhà cao tầng. Sau đó $1\ \text{s}$ tại ban công phía dưới cách cửa sổ trên của toà nhà $20\ \text{m}$ có một hòn đá khác cũng rơi tự do. Biết cả hai hòn đá cùng chạm đất đồng thời. Lấy $g=10\ \text{m/s}^2$. Chiều cao của cửa sổ toà nhà trên so với đất là	
		\begin{mcq}(4)
			\item $\text{25,31}\ \text{m}$.
			\item $\text{31,25}\ \text{m}$.
			\item $\text{51,25}\ \text{m}$.
			\item $\text{35,31}\ \text{m}$.
		\end{mcq}
	}
	\loigiai
	{\textbf{Đáp án: B.}	
		
		Ta có: $h=g\cdot \dfrac{t^2}{2}; \ h-20=g\cdot \dfrac{\left(t-1 \right)^2 }{2}$.
		
		$\Rightarrow t=\text{2,5}\ \text{s}; \ h=\text{31,25}\ \text{m}$.  
		
		
	}
	\item \mkstar{3}
	
	\cauhoi
	{Một vật rơi tự do từ độ cao $250\ \text{m}$. Tỉ số quãng đường vật rơi được trong $2\ \text{s}$ đầu, $2\ \text{s}$ sau và $2\ \text{s}$ cuối cùng là
		\begin{mcq}(4)
			\item $1:4:9$.
			\item $1:2:4$.
			\item $1:3:5$.
			\item $1:2:3$.
		\end{mcq}
	}
	\loigiai
	{\textbf{Đáp án: C.}
		
		Rơi tự do là chuyển động nhanh dần đều với $V_0=0$  nên khi thời gian chuyển động trên các đoạn đường liên tiếp bằng nhau (cùng bằng $2\ \text{s}$) thì: $\Delta s_1:\Delta s_2:\Delta s_3=1:3:5$.
		
		
	}
	\item \mkstar{4}
	
	\cauhoi
	{Một viên bi A được thả rơi từ độ cao $\SI{30}{\meter}$. Cùng lúc đó, một viên bi B được bắn theo phương thẳng đứng từ dưới đất lên với $\SI{25}{\meter/\second}$ tới va chạm vào bi A. Chọn trục O$y$ thẳng đứng, gốc O ở mặt đất, chiều dương hướng lên, gốc thời gian lúc 2 viên bi bắt đầu chuyển động, $g=\SI{10}{\meter/\second^2}$. Bỏ qua sức cản không khí. Thời điểm và tọa độ 2 viên bi gặp nhau là
		\begin{mcq}(2)
			\item $\SI{1.2}{\second}$ và $\SI{22.8}{\meter}$.
			\item $\SI{1.6}{\second}$ và $\SI{11.4}{\meter}$.
			\item $\SI{1.4}{\second}$ và $\SI{8.8}{\meter}$.
			\item $\SI{1.8}{\second}$ và $\SI{1.6}{\meter}$.
		\end{mcq}
	}
	\loigiai
	{\textbf{Đáp án: A.}
		
		Phương trình chuyển động của viên bi A là
		$$y_{\text{A}}=y_{0\text{A}}+v_{0\text{A}}t+\dfrac{1}{2}gt^2=30-5t^2 \textrm{ (m, s)}.$$
		Phương trình chuyển động của viên bi B là
		$$y_{\text{B}}=y_{0\text{B}}+v_{0\text{B}}t+\dfrac{1}{2}gt^2=25t-5t^2\textrm{ (m, s)}.$$
		Hai viên bi gặp nhau khi chúng có cùng tọa độ:
		$$y_{\text{A}}=y_{\text{B}}\Rightarrow 30-5t^2=25t-5t^2 \Rightarrow t=\SI{1.2}{\second}.$$
		Tọa độ hai viên bi khi gặp nhau là
		$$y_{\text{A}}=y_{\text{B}}=30-5t^2=\SI{22.8}{\meter}.$$
		
		
	}
	\item \mkstar{4}
	
	\cauhoi
	{Hai viên bi sắt được thả rơi cùng độ cao cách nhau một khoảng thời gian $\SI{0,5}{\second}$. Lấy $g = \SI{10}{\meter/\second^2}$. Khoảng cách giữa hai viên bi sau khi viên thứ nhất rơi được $\SI{1,5}{\second}$ là
		\begin{mcq}(4)
			\item $\SI{12,5}{\meter}$.
			\item $\SI{6,25}{\meter}$.
			\item $\SI{5,0}{\meter}$.
			\item $\SI{2,5}{\meter}$.
		\end{mcq}
	}
	\loigiai
	{\textbf{Đáp án: B.}
		
		Chọn gốc thời gian là lúc thả viên bi 1. Viên bi 2 được thả sau $\SI{0,5}{\second}$ nên:
		$$t_2=t_1-\SI{0,5}{\second}$$
		Quãng đường viên bi 1 đi được:
		$$s_1=\frac{1}{2}gt_1^2.$$
		Quãng đường viên bi 2 đi được:
		$$s_2=\frac{1}{2}g(t_1-0,5)^2$$
		Lấy $s_1-s_2=\dfrac{1}{2}gt_1^2-\dfrac{1}{2}g(t_1-0,5)^2=\SI{6,25}{\meter}$.
		
		
	}
	\item \mkstar{4}
	
	\cauhoi
	{Quãng đường vật rơi trong giây thứ $n$ là $h$. Quãng đường mà nó rơi trong giây tiếp theo là
		\begin{mcq}(4)
			\item $h$.
			\item $h+\dfrac{g}{2}$.
			\item $h-g$.
			\item $h+g$.
		\end{mcq}
		
		
	}
	\loigiai
	{\textbf{Đáp án: D.}
		
		Quãng đường vật rơi được trong giây thứ $n$ bằng quãng đường vật rơi được từ ban đầu cho đến giây $n$ trừ cho quãng đường vật rơi được từ ban đầu cho đến giây $n-1$: $$h=\dfrac{1}{2}gn^2-\dfrac{1}{2}g(n-1)^2=\dfrac{g}{2}(2n-1)$$
		
		Quãng đường vật rơi được trong giây thứ $n+1$ bằng quãng đường vật rơi được từ ban đầu cho đến giây $n+1$ trừ cho quãng đường vật rơi được từ ban đầu cho đến giây $n$:
		$$h'=\dfrac{1}{2}g(n+1)^2-\dfrac{1}{2}gn^2 = \dfrac{g}{2}(2n+1)$$
		
		Vậy $h'=h+g$.
		
		
	}
\end{enumerate}

\whiteBGstarEnd

\loigiai{\begin{center}
		\textbf{BẢNG ĐÁP ÁN}
	\end{center}
	\begin{center}
		\begin{tabular}{|m{2.8em}|m{2.8em}|m{2.8em}|m{2.8em}|m{2.8em}|m{2.8em}|m{2.8em}|m{2.8em}|m{2.8em}|m{2.8em}|}
			\hline
			1. B  & 2.B  & 3.B  & 4.A  & 5.C  & 6.A  & 7.A  & 8.C  & 9.A  & 10.A  \\
			\hline
			11.A  & 12.D  & 13.B  & 14.A  & 15.A  & 16.B  & 17.C  & 18.A  & 19.B  & 20.D  \\
			\hline
			
		\end{tabular}
\end{center}}