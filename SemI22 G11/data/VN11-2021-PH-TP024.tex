\whiteBGstarBegin
\setcounter{section}{0}
\section{Trắc nghiệm}
\begin{enumerate}[label=\bfseries Câu \arabic*:]
	
	
	\item \mkstar{1}
	
	\cauhoi
	{Câu nào dưới đây nói về chất điện phân là \textbf{không} đúng?
		\begin{mcq}
			\item Chất điện phân khi có dòng điện chạy qua sẽ giải phóng các chất ở các điện cực.
			\item Trong dung dịch, các phân tử axit, muối, bazơ đều bị phân li thành các ion.
			\item Một số chất rắn khi nóng chảy cũng là chất điện phân.
			\item Chất điện phân nhất thiết phải là dung dịch của các chất tan được trong dung môi.
		\end{mcq}
		
	}
	\loigiai
	{	\textbf{Đáp án: D.}
		
		Chất điện phân không nhất thiết phải là dung dịch của các chất tan được trong dung môi.
	}
	\item \mkstar{1}

\cauhoi
{Câu nào sau đây khi nói về dòng điện trong chất điện phân là đúng?
	\begin{mcq}
		\item Khi có dòng điện chạy qua bình điện phân thì các electron và ion âm đi về phía anode, còn các ion dương đi về cathode.
		\item Khi có dòng điện chạy qua bình điện phân thì các electron đi về phía anode, các ion dương đi về cathode.
		\item Khi có dòng điện chạy qua bình điện phân thì các ion âm đi về phía anode, các ion dương đi về cathode.
		\item Khi có dòng điện chạy qua bình điện phân thì các ion âm và electron đi về phía cathode, các ion dương đi về anode.
	\end{mcq}
	
}
\loigiai
{	\textbf{Đáp án: C.}
	
	Khi có dòng điện chạy qua bình điện phân thì các ion âm đi về phía anode, các ion dương đi về cathode.
}
	\item \mkstar{1}

\cauhoi
{Câu phát biểu nào sau đây \textbf{sai}?
	\begin{mcq}
		\item Dòng điện trong kim loại là dòng chuyển dời có hướng của các electron ngược chiều điện trường.
		\item Dòng điện trong chất điện phân là dòng chuyển dời có hướng của các ion dương theo chiều điện trường, của các ion âm ngược chiều điện trường.
		\item Dòng điện trong chất khí là dòng chuyển dời có hướng của các ion dương theo chiều điện trường, của các ion âm ngược chiều điện trường.
		\item Dòng điện trong chân không là dòng chuyển dời có hướng của các electron phát xạ từ cathode bị nung nóng dưới tác dụng của điện trường.
	\end{mcq}
	
}
\loigiai
{	\textbf{Đáp án: C.}
	
	Dòng điện trong chất khí là dòng chuyển dời có hướng của các ion dương theo chiều điện trường, của các ion âm và các electron ngược chiều điện trường. Các hạt tải điện này là do chất khí bị ion hóa sinh ra.
}
	\item \mkstar{1}

\cauhoi
{Câu nào sau đây \textbf{sai} khi nói về tính dẫn điện của chất điện phân?
	\begin{mcq}
		\item Dòng điện trong chất điện phân tuân theo định luật Ôm.
		\item Độ dẫn điện của chất điện phân tăng khi nhiệt độ tăng.
		\item Điện trở của chất điện phân giảm khi nhiệt độ tăng.
		\item Khi ắc quy được nạp điện, dòng điện qua ắc quy cũng là dòng điện trong chất điện phân.
	\end{mcq}
	
}
\loigiai
{	\textbf{Đáp án: A.}
	
	Dòng điện trong chất điện phân chỉ tuân theo định luật Ôm khi xảy ra hiện tượng dương cực tan.
}
	\item \mkstar{2}

\cauhoi
{Phát biểu nào sau đây là đúng?
	\begin{mcq}
		\item Dòng điện trong kim loại cũng như trong chân không đều là dòng chuyển động có hướng của các electron, ion dương và ion âm.
		\item Dòng điện trong kim loại là dòng chuyển động có hướng của các electron, dòng điện trong chân không và trong chất khí đều là dòng chuyển động có hướng của các ion dương và ion âm.
		\item Dòng điện trong kim loại và trong chân không đều là dòng chuyển động có hướng của các electron.
		\item Dòng điện trong kim loại và dòng điện trong chất khí là dòng chuyển động có hướng của các ion.
	\end{mcq}
	
}
\loigiai
{	\textbf{Đáp án: C.}
	
	Dòng điện trong kim loại và trong chân không đều là dòng chuyển động có hướng của các electron.
}
	\item \mkstar{2}

\cauhoi
{Nguyên nhân gây ra điện trở của kim loại là
	\begin{mcq}
		\item do sự va chạm của các electron với các ion dương ở các nút mạng.
		\item do sự va chạm của các ion dương ở các nút mạng với nhau.
		\item do sự va chạm của các electron với nhau.
		\item Cả B và C đều đúng.
	\end{mcq}
	
}
\loigiai
{	\textbf{Đáp án: A.}
	
	Nguyên nhân gây ra điện trở của kim loại là do sự va chạm của các electron với các ion dương ở các nút mạng.
}
	\item \mkstar{2}

\cauhoi
{Phát biểu nào dưới đây về sự phụ thuộc của cường độ dòng điện $I$ vào hiệu điện thế $U$ giữa hai cực tụ điện chứa chất khí trong quá trình dẫn điện không tự lực là \textbf{không} đúng?
	\begin{mcq}
		\item Với mọi giá trị của $U$, $I$ luôn tăng tỉ lệ với $U$.
		\item Với $U$ nhỏ, $I$ tăng theo $U$.
		\item Với $U$ đủ lớn, $I$ đạt giá trị bão hòa.
		\item Với $U$ quá lớn, $I$ tăng nhanh theo $U$.
	\end{mcq}
	
}
\loigiai
{	\textbf{Đáp án: A.}
	
	Dòng điện $I$ tỉ lệ với $U$ đến khi $I$ đạt bão hòa thì $I$ không tăng nữa. Khi đó $I$ không còn tỉ lệ với $U$.
}
	\item \mkstar{2}

\cauhoi
{Hai thanh kim loại được nối với nhau bởi hai đầu mối hàn tạo thành một mạch kín, hiện tượng nhiệt điện chỉ xảy ra khi
	\begin{mcq}
		\item hai thanh kim loại có bản chất khác nhau và nhiệt độ ở hai đầu mối hàn bằng nhau.
		\item hai thanh kim loại có bản chất khác nhau và nhiệt độ ở hai đầu mối hàn khác nhau.
		\item hai thanh kim loại có bản chất giống nhau và nhiệt độ ở hai đầu mối hàn bằng nhau.
		\item hai thanh kim loại có bản chất giống nhau và nhiệt độ ở hai đầu mối hàn khác nhau.
	\end{mcq}
	
}
\loigiai
{	\textbf{Đáp án: B.}
	
	Hai thanh kim loại được nối với nhau bởi hai đầu mối hàn tạo thành một mạch kín, hiện tượng nhiệt điện chỉ xảy ra khi hai thanh kim loại có bản chất khác nhau và nhiệt độ ở hai đầu mối hàn khác nhau.
}
	\item \mkstar{2}

\cauhoi
{Khi cho dòng điện chạy qua một sợi dây thép có hệ số nhiệt điện trở là $\SI{0.004}{K^{-1}}$ thì điện trở của nó tăng gấp đôi. Nhiệt độ của sợi dây này đã tăng thêm
	\begin{mcq}(4)
		\item $\SI{800}{\celsius}$.
		\item $\SI{250}{\celsius}$.
		\item $\SI{300}{\celsius}$.
		\item $\SI{80}{\celsius}$.
	\end{mcq}
	
}
\loigiai
{	\textbf{Đáp án: B.}
	
	Áp dụng công thức:
	$$R=R_0 [1+ \alpha (t-t_0)] \Rightarrow t-t_0 = \SI{250}{\celsius}.$$
}
	\item \mkstar{2}

\cauhoi
{Ở $\SI{20}{\celsius}$, điện trở suất của bạc là $\SI{1.62e-8}{\Omega m}$. Biết hệ số nhiệt điện trở của bạc là $\SI{4.1e-3}{K^{-1}}$. Ở $\SI{330}{K}$ thì điện trở suất của bạc là
	\begin{mcq}(2)
		\item $\SI{1.866e-8}{\Omega m}$.
		\item $\SI{3.679e-8}{\Omega m}$.
		\item $\SI{3.812e-8}{\Omega m}$.
		\item $\SI{4.151e-8}{\Omega m}$.
	\end{mcq}
	
}
\loigiai
{	\textbf{Đáp án: A.}
	
	Ta có $t-t_0 = \SI{330}{K} - \SI{293}{K} = \SI{37}{K}$.
	
	Áp dụng công thức:
	$$\rho = \rho_0 [1+ \alpha (t-t_0)] = \SI{1.866e-8}{\Omega m}.$$
}
	
\end{enumerate}

\whiteBGstarEnd

\loigiai
{
	\begin{center}
		\textbf{BẢNG ĐÁP ÁN}
	\end{center}
	\begin{center}
		\begin{tabular}{|m{2.8em}|m{2.8em}|m{2.8em}|m{2.8em}|m{2.8em}|m{2.8em}|m{2.8em}|m{2.8em}|m{2.8em}|m{2.8em}|}
			\hline
			1.D  & 2.C  & 3.C  & 4.A  & 5.C  & 6.A  & 7.A  & 8.B  & 9.B  & 10.A  \\
			\hline
			
		\end{tabular}
	\end{center}
}
\section{Tự luận}
\begin{enumerate}[label=\bfseries Câu \arabic*:]
	\item \mkstar{2}
	
	\cauhoi{
	Một sợi dây đồng có điện trở $\SI{74}{\Omega}$ ở $\SI{50}{\celsius}$, có hệ số nhiệt điện trở là $\SI{4.1e-3}{K^{-1}}$. Tính điện trở của sợi dây đó ở $\SI{100}{\celsius}$.
	}
	
	\loigiai{
		
		Điện trở của sợi dây ở $\SI{100}{\celsius}$:
		$$\dfrac{R_2}{R_1} = \dfrac{1 + \alpha t_2}{1+ \alpha t_1} \Rightarrow R_2 = \SI{86.6}{\Omega}.$$
	}
	
	\item \mkstar{2}
	
	\cauhoi{
		Một mối hàn của một cặp nhiệt điện có hệ số $\alpha_T = \SI{48}{\micro V / K}$ được đặt trong không khí ở $\SI{20}{\celsius}$, còn mối hàn kia được nung nóng đến nhiệt độ $t \SI{}{\celsius}$, suất điện động của cặp nhiệt điện khi đó là $\calE = \SI{6}{mV}$. Xác định nhiệt độ của mối hàn còn lại.
	}
	
	\loigiai{
		
		Áp dụng công thức tính suất điện động nhiệt điện:
		$$\calE = \alpha_T (T_1 - T_2) \Rightarrow T_1 = \SI{418}{K} \Rightarrow t_1 = \SI{145}{\celsius}.$$
	}
	\item \mkstar{2}
	
	\cauhoi{
		Một bình điện phân chứa dung dịch bạc nitrate có điện trở là $\SI{5}{\Omega}$. Anode của bình bằng bạc và hiệu điện thế đặt vào hai điện cực của bình là $\SI{20}{V}$. Tính khối lượng bạc bám vào cathode sau 32 phút 10 giây.
	}
	
	\loigiai{
		
		Cường độ dòng điện qua bình điện phân:
		$$I=\dfrac{U}{R} = \SI{4}{A}.$$
		
		Áp dụng định luật Fa-ra-đây để tính khối lượng bạc bám vào cathode:
		$$m=\dfrac{1}{F} \dfrac{A}{n} I t = \SI{8.64}{g}.$$
	}
	\item \mkstar{3}
	
	\cauhoi{
		Để giải phóng lượng $\ce{Cl}$ và $\ce{H}$ từ $\SI{7.6}{g}$ $\ce{HCl}$ bằng dòng điện $\SI{5}{A}$ thì phải cần thời gian điện phân là bao lâu? Biết rằng đương lượng điện của $\ce{H}$ và $\ce{Cl}$ lần lượt là $k_1=\SI{0.1045e-7}{kg/C}$ và $k_2 = \SI{3.6e-7}{kg/C}$.
	}
	
	\loigiai{
		
		Áp dụng định luật Fa-ra-đây:
		$$m=(k_1 + k_2) It$$
		
		Suy ra thời gian cần phải điện phân là
		$$t=\SI{4027}{s} \approx \SI{1.1}{h}.$$
	}
	
\end{enumerate}