\whiteBGstarBegin
\setcounter{section}{0}
\section{Trắc nghiệm}
\begin{enumerate}[label=\bfseries Câu \arabic*:]
	\item \mkstar{1}
	
	\cauhoi
	{Trong các chuyển động sau, chuyển động nào là đều?
		\begin{mcq}
			\item Chuyển động của quả dừa rơi từ trên cây xuống. 
			\item Chuyển đông của Mặt Trăng quanh Trái Đất. 
			\item Chuyển động của đầu cánh quạt.
			\item Chuyển động thực của chiếc xe buýt ở trên đường. 
		\end{mcq}
	}
	
	\loigiai
	{		\textbf{Đáp án: B.}
		
		Chuyển đông của Mặt Trăng quanh Trái Đất là chuyển động đều.
		
	}
	\item \mkstar{2}
	
	\cauhoi
	{Tàu Thống Nhất TN1 đi từ ga Huế vào ga Sài Gòn mất $\SI{20}{h}$. Biết vận tốc trung bình của tàu là $\SI{15}{m/s}$. Hỏi chiều dài của đường ray từ Huế vào Sài Gòn?
		\begin{mcq}(4)
			\item $\SI{3000}{km}$. 
			\item $\SI{1080}{km}$.
			\item $\SI{1000}{km}$.
			\item $\SI{1333}{km}$.
		\end{mcq}
	}
	
	\loigiai
	{		\textbf{Đáp án: B.}
		
	Chiều dài của đường ray từ Huế vào Sài Gòn:
	$$s=vt=\SI{1080}{km}$$
		
	}
	\item \mkstar{2}
	
	\cauhoi
	{Trong trận đấu giữa Đức và Áo ở EURO 2008, tiền vệ Mai-cơn BaLack của đội tuyển Đức sút phạt cách khung thành của đội Áo $\SI{30}{m}$. Các chuyên gia tính được vận tốc trung bình của quả phạt đó lên tới $\SI{108}{km/}$. Hỏi thời gian bay của quả bóng là bao nhiêu?
		\begin{mcq}(4)
			\item $\SI{1}{s}$. 
			\item $\SI{36}{s}$. 
			\item $\SI{1.5}{s}$.  
			\item $\SI{3.6}{s}$. 
		\end{mcq}
	}
	
	\loigiai
	{\textbf{Đáp án: A.}
		
		Thời gian bay của quả bóng:
		$$t=\dfrac{s}{v} = 1\ \text s$$
		
	}
	\item \mkstar{2}
	
	\cauhoi
	{Hưng đạp xe lên dốc dài $\SI{100}{m}$ với vận tốc 2 m/s, sau đó xuống dốc dài $\SI{140}{m}$ hết 30 s. Hỏi vận tốc trung bình của Hưng trên cả hai đoạn đường?
		\begin{mcq}(4)
			\item $\SI{50}{m/s}$. 
			\item $\SI{8}{m/s}$. 
			\item $\SI{4.67}{m/s}$. 
			\item $\SI{3}{m/s}$.
		\end{mcq}
	}
	
	\loigiai
	{\textbf{Đáp án: D.}
		
		Thời gian trên đoạn đường thứ nhất:
		$$t_1=\dfrac{s_1}{v_1} = 50\ \text s$$
		
		Vận tốc trung bình trên cả hai đoạn đường:
		$$v=\dfrac{s_1+s_2}{t_1+t_2} = 3\ \text {m/s}$$
		
	}
	\item \mkstar{2}
	
	\cauhoi
	{
		Một học sinh vô địch trong giải điền kinh ở nội dung chạy cự li $\SI{1000}{m}$ với thời gian là 2 phút 5 giây. Vận tốc của học sinh đó là
		\begin{mcq}(4)
			\item $\SI{40}{m/s}$. 
			\item $\SI{8}{m/s}$. 
			\item $\SI{4.88}{m/s}$. 
			\item $\SI{120}{m/s}$.
		\end{mcq}
	}
	
	\loigiai
	{\textbf{Đáp án: B.}
		
		Vận tốc của học sinh đó là
		$$v=\dfrac{s}{t} = \SI{8}{m/s}$$
		
	}
	
	\item \mkstar{2}
	
	\cauhoi
	{Một học sinh đi bộ từ nhà đến trường trên đoạn đường $\SI{0.9}{km}$ trong thời gian 10 phút. Vận tốc trung bình của học sinh đó là
		\begin{mcq}(4)
			\item $\SI{15}{m/s}$. 
			\item $\SI{1.5}{m/s}$. 
			\item $\SI{5.4}{km/h}$. 
			\item $\SI{0.9}{km/h}$. 
		\end{mcq}
		
	}
	
	\loigiai
	{\textbf{Đáp án: C.}
		
		Vận tốc trung bình của học sinh:
		$$v=\dfrac{s}{t} = \SI{5.4}{km/h}$$	
		
	}
	\item \mkstar{3}
	
	\cauhoi
	{Một xe máy di chuyển giữa hai địa điểm A và B. Vận tốc trong 1/2 thời gian đầu là 30 km/h và trong 1/2 thời gian sau là 15 m/s. Vận tốc trung bình của ô tô trên cả đoạn đường là
		\begin{mcq}(4)
			\item $\SI{42}{km/h}$. 
			\item $\SI{22.5}{km/h}$. 
			\item $\SI{36}{km/h}$.
			\item $\SI{54}{km/h}$. 
		\end{mcq}
		
	}
	
	\loigiai
	{\textbf{Đáp án: A.}
		
		Vận tốc trung bình của ô tô trên cả đoạn đường:
		$$v=\dfrac{s_1+s_2}{t_1+t_2} = \dfrac{s_1+s_2}{2t_1} = \dfrac{s_1+s_2}{2t_2} = \dfrac{t_1v_1+t_1v_2}{2t_1} = \dfrac{v_1+v_2}{2} = \SI{42}{km/h}$$
		
	}
	
	\item \mkstar{3}
	
	\cauhoi
	{Một xe máy di chuyển giữa hai địa điểm A và B. Vận tốc trong 1/2 quãng đường đầu là 30 km/h và trong 1/2 quãng đường sau là 15 m/s. Vận tốc trung bình của ô tô trên cả đoạn đường là
		\begin{mcq}(4)
			\item $\SI{42}{km/h}$. 
			\item $\SI{22.5}{km/h}$. 
			\item $\SI{38.6}{km/h}$.
			\item $\SI{54.4}{km/h}$. 
		\end{mcq}
	}
	
	\loigiai
	{	\textbf{Đáp án: C.}
		
	Vận tốc trung bình của ô tô trên cả đoạn đường:
	$$v=\dfrac{s_1+s_2}{t_1+t_2} = \dfrac{2s_1}{t_1+t_2} = \dfrac{2s_2}{t_1+t_2} = \dfrac{2s_1}{\dfrac{s_1}{v_1}+\dfrac{s_1}{v_2}} = \dfrac{2}{\dfrac{1}{v_1}+\dfrac{1}{v_2}} = \SI{38.6}{km/h}$$
		
	}
	\item \mkstar{2}
	
	\cauhoi
	{Một ô tô lên dốc với vận tốc 16 km/h. Khi xuống lại dốc đó, ô tô này chuyển động nhanh gấp đôi so với khi lên dốc. Vận tốc trung bình của ô tô trên cả hai đoạn đường là
		\begin{mcq}(4)
			\item $\SI{24}{km/h}$.
			\item $\SI{32}{km/h}$.
			\item $\SI{21.33}{km/h}$.
			\item $\SI{16}{km/h}$.
		\end{mcq}
	}
	
	\loigiai
	{\textbf{Đáp án: C.}	
		
	Vận tốc trung bình của ô tô trên cả đoạn đường:
$$v=\dfrac{s_1+s_2}{t_1+t_2} = \dfrac{2s_1}{t_1+t_2} = \dfrac{2s_2}{t_1+t_2} = \dfrac{2s_1}{\dfrac{s_1}{v_1}+\dfrac{s_1}{v_2}} = \dfrac{2}{\dfrac{1}{v_1}+\dfrac{1}{v_2}} = \SI{21.33}{km/h}$$
		
	}
	\item \mkstar{3}
	
	\cauhoi
	{Một xe đạp đi từ A đến B, trong đó nửa quãng đường đầu xe đi với vận tốc 20 km/h, nửa còn lại đi với vận tốc 30 km/h. Hỏi vận tốc trung bình của xe đạp trên cả đoạn đường?
		
		\begin{mcq}(4)
			\item 25 km/h.	
			\item 24 km/h.	
			\item 50 km/h.	
			\item 10 km/h.	
		\end{mcq}
		
	}
	
	\loigiai
	{\textbf{Đáp án: B.}
		
			Vận tốc trung bình của xe đạp trên cả đoạn đường:
		$$v=\dfrac{s_1+s_2}{t_1+t_2} = \dfrac{2s_1}{t_1+t_2} = \dfrac{2s_2}{t_1+t_2} = \dfrac{2s_1}{\dfrac{s_1}{v_1}+\dfrac{s_1}{v_2}} = \dfrac{2}{\dfrac{1}{v_1}+\dfrac{1}{v_2}} = \SI{24}{km/h}$$
		
	}
	\item \mkstar{3}
	
	\cauhoi
	{Một ô tô đi từ Huế vào Đà Nẵng với vận tốc trung bình 48 km/h. Nửa quãng đường đầu ô tô có vận tốc trung bình là 40 km/h. Nửa quãng đường sau ô tô có vận tốc trung bình là
		
		\begin{mcq}(4)
			\item 50 km/h.	
			\item 44 km/h.	
			\item 60 km/h.	
			\item 68 km/h.	
		\end{mcq}
		
	}
	
	\loigiai
	{\textbf{Đáp án: C.}
		
		Vận tốc trung bình của ô tô trên cả đoạn đường:
	$$v=\dfrac{s_1+s_2}{t_1+t_2} = \dfrac{2s_1}{t_1+t_2} = \dfrac{2s_2}{t_1+t_2} = \dfrac{2s_1}{\dfrac{s_1}{v_1}+\dfrac{s_1}{v_2}} = \dfrac{2}{\dfrac{1}{v_1}+\dfrac{1}{v_2}} \Rightarrow v_1 = \SI{60}{km/h}$$
	}
	\item \mkstar{3}

\cauhoi
{Một ô tô đi từ Huế vào Đà Nẵng với vận tốc trung bình 48 km/h. Nửa thời gian đầu ô tô có vận tốc trung bình là 40 km/h. Nửa thời sau ô tô có vận tốc trung bình là
	
	\begin{mcq}(4)
		\item 54 km/h.	
		\item 56 km/h.	
		\item 60 km/h.	
		\item 58 km/h.	
	\end{mcq}
	
}

\loigiai
{\textbf{Đáp án: C.}
	
	Vận tốc trung bình của ô tô là
	
	$$v=\dfrac{s_1+s_2}{t_1+t_2} = \dfrac{s_1+s_2}{2t_1} = \dfrac{s_1+s_2}{2t_2} = \dfrac{t_1v_1+t_1v_2}{2t_1} = \dfrac{v_1+v_2}{2} \Rightarrow v_1 = \SI{56}{km/h}$$
}
	\item \mkstar{3}
	
	\cauhoi
	{Một đoàn tàu hỏa đi từ ga Hà Nội đến Huế. Nửa thời gian đầu tàu đi với vận tốc 70 km/h, nửa thời gian còn lại tàu đi với vận tốc trung bình $v_2$. Biết vận tốc trung bình trên cả đoạn đường là 60 km/h. Tính $v_2$.
		
		\begin{mcq}(4)
			\item $\SI{60}{km/h}$.
			\item $\SI{50}{km/h}$.	
			\item $\SI{58.33}{km/h}$.	
			\item $\SI{55}{km/h}$.	
		\end{mcq}
		
	}
	
	\loigiai
	{\textbf{Đáp án: B.}
		
		Vận tốc trung bình của tàu hỏa là
		
		$$v=\dfrac{s_1+s_2}{t_1+t_2} = \dfrac{s_1+s_2}{2t_1} = \dfrac{s_1+s_2}{2t_2} = \dfrac{t_1v_1+t_1v_2}{2t_1} = \dfrac{v_1+v_2}{2} \Rightarrow v_2 = \SI{50}{km/h}$$
	}
	\item \mkstar{3}
	
	\cauhoi
	{Bắn một viên bi lên trên mặt phẳng nghiêng, sau đó lăn xuống với vận tốc 6 cm/s. Biết vận tốc trung bình của viên bi cả lên và xuống là 4 cm/s. Hỏi vận tốc của viên bi khi đi lên?
		
		\begin{mcq}(4)
			\item $\SI{3}{cm/s}$.	
			\item $\SI{3}{m/s}$.	
			\item $\SI{5}{cm/s}$.
			\item $\SI{5}{m/s}$.	
		\end{mcq}
		
	}
	
	\loigiai
	{\textbf{Đáp án: A.}
		
			Vận tốc trung bình của viên bi trên cả đoạn đường:
		$$v=\dfrac{s_1+s_2}{t_1+t_2} = \dfrac{2s_1}{t_1+t_2} = \dfrac{2s_2}{t_1+t_2} = \dfrac{2s_1}{\dfrac{s_1}{v_1}+\dfrac{s_1}{v_2}} = \dfrac{2}{\dfrac{1}{v_1}+\dfrac{1}{v_2}} \Rightarrow v_1 = \SI{3}{cm/s}$$
	}
	\item \mkstar{3}
	
	\cauhoi
	{Hai bến sông A và B cách nhau 24 km, dòng nước chảy đều theo hướng từ A đến B với vận tốc 6 km/h. Một ca nô đi từ A đến B mất 1 giờ, thì đi từ B đến A mất bao lâu? Cho rằng công suất của ca nô trên cả 2 đoạn đều như nhau.
		
		\begin{mcq}(4)
			\item $\SI{1.5}{h}$
			\item $\SI{1.25}{s}$
			\item $\SI{2}{h}$	
			\item $\SI{2.5}{h}$
		\end{mcq}
		
	}
	
	\loigiai
	{\textbf{Đáp án: C.}
		
	Gọi $v_1$ là vận tốc dòng nước, $v_2$ là vận tốc ca nô.
	
	Khi đi từ A đến B là xuôi dòng:
	$$t_1 = \dfrac{s}{v_1+v_2} \Rightarrow v_2=\SI{18}{km/h}$$
	
	Khi đi từ B đến A là ngược dòng:
	$$t_2=\dfrac{s}{v_1-v_2} = \SI{2}{h}$$
	}
	\item \mkstar{3}
	
	\cauhoi
	{Một người đi xe máy từ A đến B cách nhau 400 m. Nửa quãng đường đầu, xe đi với vận tốc $v_1$, nửa quãng đường sau xe đi với vận tốc $v_2$ bằng một nửa $v_1$. Hãy tính $v_1$ để người đó đi từ A đến B trong 1 phút.
		
		\begin{mcq}(4)
			\item $\SI{5}{m/s}$.	
			\item $\SI{40}{km/h}$.	
			\item $\SI{7.5}{m/s}$.	
			\item $\SI{36}{km/h}$.	
		\end{mcq}
		
	}
	
	\loigiai
	{\textbf{Đáp án: D.}
		
	Thời gian của người đi xe máy trên nửa đoạn đường đầu:
	$$t_1 = \dfrac{s_1}{v_1}$$
	
	Thời gian của người đi xe máy trên nửa đoạn đường sau:
	$$t_2=\dfrac{s_2}{v_2}$$
	
	Ta có $t=t_1+t_2=\SI{60}{s}$, suy ra:
	$$\dfrac{s_1}{v_1} +\dfrac{s_2}{v_2} = 60 \Rightarrow \dfrac{s}{2v_1}+\dfrac{s}{2\cdot{v_1}{2}} \Rightarrow v_1 = \SI{10}{m/s} = \SI{36}{km/h}$$
	}
	\item \mkstar{4}
	
	\cauhoi
	{Một người đi xe đạp trên đoạn đường AB. Nửa đoạn đường đầu người ấy đi với vận tốc $v_1=\SI{20}{km/h}$. Trong quãng thời gian còn lại thì một nửa quãng thời gian đầu người đó đi với vận tốc $v_2=\SI{10}{km/h}$, một nửa quãng thời gian cuối người đó đi với vận tốc $v_3=\SI{5}{km/h}$. Tính vận tốc trung bình trên cả đoạn đường AB. 
		
		\begin{mcq}(4)
			\item $\SI{10.9}{km/h}$.	
			\item $\SI{11.67}{km/h}$.	
			\item $\SI{7.5}{km/h}$.	
			\item $\SI{15}{km/h}$.	
		\end{mcq}
		
	}
	
	\loigiai
	{\textbf{Đáp án: A.}
		
	Vận tốc trung bình trên cả đoạn đường:
	$$v=\dfrac{2}{\dfrac{1}{v_1}+\dfrac{1}{v_2}}$$
	
	Vận tốc trung bình trên nửa đoạn đường thứ hai:
	$$v_2=\dfrac{v_2'+v_2''}{2} = \SI{7.5}{km/h}$$
	
	Vậy vận tốc trung bình trên cả đoạn đường:
	$$v=\SI{10.9}{km/h}$$
	}
	\item \mkstar{4}

\cauhoi
{Một chiếc ô tô chạy từ điểm A đến điểm B. Nửa đoạn đường đầu ô tô đi với vận tốc $v_1=\SI{40}{km/h}$. Trong quãng thời gian còn lại thì một nửa quãng thời gian đầu ô tô đi với vận tốc $v_2=\SI{50}{km/h}$, một nửa quãng thời gian cuối ô tô đi với vận tốc $v_3=\SI{60}{km/h}$. Tính vận tốc trung bình của ô tô trên cả đoạn đường AB. 
	
	\begin{mcq}(4)
		\item $\SI{50.9}{km/h}$.	
		\item $\SI{55.7}{km/h}$.	
		\item $\SI{57.4}{km/h}$.	
		\item $\SI{58.5}{km/h}$.	
	\end{mcq}
	
}

\loigiai
{\textbf{Đáp án: C.}
	
	Vận tốc trung bình trên cả đoạn đường:
	$$v=\dfrac{2}{\dfrac{1}{v_1}+\dfrac{1}{v_2}}$$
	
	Vận tốc trung bình trên nửa đoạn đường thứ hai:
	$$v_2=\dfrac{v_2'+v_2''}{2} = \SI{55}{km/h}$$
	
	Vậy vận tốc trung bình trên cả đoạn đường:
	$$v=\SI{57.4}{km/h}$$
}
	\item \mkstar{4}

\cauhoi
{Một chiếc ô tô chạy từ điểm A đến điểm B. Nửa quãng thời gian đầu ô tô đi với vận tốc $v_1=\SI{40}{km/h}$. Trong quãng thời gian còn lại thì một nửa quãng thời gian đầu ô tô đi với vận tốc $v_2=\SI{50}{km/h}$, một nửa quãng thời gian cuối ô tô đi với vận tốc $v_3=\SI{60}{km/h}$. Tính vận tốc trung bình của ô tô trên cả đoạn đường AB. 
	
	\begin{mcq}(4)
		\item $\SI{45.5}{km/h}$.	
		\item $\SI{47.5}{km/h}$.	
		\item $\SI{57.5}{km/h}$.	
		\item $\SI{55.5}{km/h}$.	
	\end{mcq}
	
}

\loigiai
{\textbf{Đáp án: B.}
	
	Vận tốc trung bình trên cả đoạn đường:
	$$v=\dfrac{v_1+v_2}{2}$$
	
	Vận tốc trung bình trên nửa quãng thời gian thứ hai:
	$$v_2=\dfrac{v_2'+v_2''}{2} = \SI{55}{km/h}$$
	
	Vậy vận tốc trung bình trên cả đoạn đường:
	$$v=\SI{47.5}{km/h}$$
}
	\item \mkstar{4}

\cauhoi
{Một chiếc ô tô chạy từ điểm A đến điểm B. Nửa đoạn đường đầu ô tô đi với vận tốc $v_1=\SI{40}{km/h}$. Trong nửa đoạn đường còn lại thì một nửa đoạn đường đầu ô tô đi với vận tốc $v_2=\SI{50}{km/h}$, một nửa đoạn đường cuối ô tô đi với vận tốc $v_3=\SI{60}{km/h}$. Tính vận tốc trung bình của ô tô trên cả đoạn đường AB. 
	
	\begin{mcq}(4)
		\item $\SI{55.25}{km/h}$.	
		\item $\SI{50.15}{km/h}$.	
		\item $\SI{48.15}{km/h}$.	
		\item $\SI{46.15}{km/h}$.	
	\end{mcq}
	
}

\loigiai
{\textbf{Đáp án: D.}
	
	Vận tốc trung bình trên cả đoạn đường:
	$$v=\dfrac{2}{\dfrac{1}{v_1}+\dfrac{1}{v_2}}$$
	
	Vận tốc trung bình trên nửa đoạn đường thứ hai:
	$$v_2=\dfrac{2}{\dfrac{1}{v_2'}+\dfrac{1}{v_2''}} = \SI{54.54}{km/h}$$
	
	Vậy vận tốc trung bình trên cả đoạn đường:
	$$v=\SI{46.15}{km/h}$$
}
	
\end{enumerate}

\whiteBGstarEnd

\loigiai
{
	\begin{center}
		\textbf{BẢNG ĐÁP ÁN}
	\end{center}
	\begin{center}
		\begin{tabular}{|m{2.8em}|m{2.8em}|m{2.8em}|m{2.8em}|m{2.8em}|m{2.8em}|m{2.8em}|m{2.8em}|m{2.8em}|m{2.8em}|}
			\hline
			1.B  & 2.B  & 3.A  & 4.D  & 5.B  & 6.C  & 7.A  & 8.C  & 9.C  & 10.B  \\
			\hline
			11.C  & 12.C  & 13.B  & 14.A  & 15.C  & 16.D  & 17.A  & 18.C  & 19.B  & 20.D  \\
			\hline
			
		\end{tabular}
	\end{center}
}
\section{Tự luận}
\begin{enumerate}[label=\bfseries Câu \arabic*:]
	\item \mkstar{1}
	
	\cauhoi{
		Trong những chuyển động sau, chuyển động nào là đều? Không đều?
		\begin{enumerate}
			\item Chuyển động của đầu cánh quạt máy khi quạt đang chạy ổn định;
			\item Chuyển động của ô tô khi khởi hành;
			\item Chuyển động của xe đạp khi xuống dốc;
			\item Chuyển động của tàu hỏa khi rời ga.
		\end{enumerate}
	}
	
	\loigiai{
	
	\begin{enumerate}
	\item Chuyển động của đầu cánh quạt máy khi quạt đang chạy ổn định là chuyển động đều;
	\item Chuyển động của ô tô khi khởi hành là chuyển động không đều;
	\item Chuyển động của xe đạp khi xuống dốc là chuyển động không đều;
	\item Chuyển động của tàu hỏa khi rời ga là chuyển động không đều.
\end{enumerate}
	}
	
	\item \mkstar{2}
	
	\cauhoi{Chuyển động của ô tô trên đường từ Hà Nội đến Hải Phòng là chuyển động đều hay không đều? Tại sao? Khi nói ô tô chạy từ Hà Nội đến Hải Phòng với vận tốc $\SI{50}{km/h}$ là nói tới vận tốc nào?}
	\loigiai{
		
		Chuyển động của ô tô trên đường từ Hà Nội đến Hải Phòng là chuyển động không đều, vì ô tô có thể dừng đỗ, tăng giảm vận tốc nhiều lần trên đoạn đường.
		
		Khi nói ô tô chạy từ Hà Nội đến Hải Phòng với vận tốc $\SI{50}{km/h}$ là nói tới vận tốc trung bình.
	}
	\item \mkstar{2}
	
	\cauhoi
	{Một người đi xe đạp thả dốc xuống một cái dốc dài $\SI{120}{m}$ hết $\SI{30}{s}$. Khi hết dốc, xe lăn tiếp trên quãng đường nằm ngang dài $\SI{60}{m}$ trong $\SI{24}{s}$ rồi dừng lại. Tính vận tốc trung bình của xe đạp trên quãng đường xuống dốc, trên đoạn đường nằm ngang và trên cả hai đoạn đường.
	}
	
	\loigiai
	{Vận tốc trung bình trên quãng đường xuống dốc:
		$$v_1 = \dfrac{s_1}{t_1} = \SI{4}{m/s}$$
		
		Vận tốc trung bình trên quãng đường nằm ngang:
		$$v_2 = \dfrac{s_2}{t_2} = \SI{2.5}{m/s}$$
		
		Vận tốc trung bình trên cả hai đoạn đường:
		$$v= \dfrac{s_1 + s_2}{t_1 + t_2} = \SI{3.3}{m/s}$$
	}
	\item \mkstar{2}
	
	\cauhoi
	{Một đoàn tàu chuyển động trong 5 giờ với vận tốc trung bình là $\SI{30}{km/h}$. Tính quãng đường tàu đi được.
	}
	
	\loigiai
	{Quãng đường tàu đi được là:
		$$s=vt=150\ \text{km}$$
	}
	\item \mkstar{2}
	
	\cauhoi
	{Đào đi bộ từ nhà đến trường, quãng đường đầu dài $\SI{200}{m}$ Đào đi mất 1 phút 40 giây, quãng đường còn lại dài $\SI{300}{m}$ Đào đi mất 100 giây. Tính vận tốc trung bình của Đào trên cả hai đoạn đường.
	}
	
	\loigiai
	{
		Vận tốc trung bình trên cả hai đoạn đường:
		$$v= \dfrac{s_1 + s_2}{t_1 + t_2} = \SI{2.5}{m/s}$$
	}
\end{enumerate}