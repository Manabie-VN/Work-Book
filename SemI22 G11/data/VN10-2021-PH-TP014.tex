\whiteBGstarBegin
\setcounter{section}{0}
\section{Trắc nghiệm}
\begin{enumerate}[label=\bfseries Câu \arabic*:]
	
	\item \mkstar{1}
	
	\cauhoi
	{Khi nói về lực đàn hồi của lò xo, phát biểu nào sau đây \textbf{sai?}
		\begin{mcq}
			\item Lực đàn hồi luôn có chiều ngược với chiều biến dạng của lò xo.
			\item Trong giới hạn đàn hồi, lực đàn hồi luôn tỉ lệ thuận với độ biến dạng.
			\item Khi lò xo bị dãn, lực đàn hồi có phương dọc theo trục lò xo.
			\item Lò xo luôn lấy lại được hình dạng ban đầu khi thôi tác dụng lực.
		\end{mcq}
	}
	
	\loigiai
	{	\textbf{Đáp án: D.}
		
	Lò xo chỉ lấy lại được hình dạng ban đầu khi bị biến dạng trong giới hạn đàn hồi.
	}
	\item \mkstar{2}
	
	\cauhoi
	{Một vật có khối lượng $\SI{200}{g}$ được treo vào một lò xo thẳng đứng thì chiều dài của lò xo là $\SI{20}{cm}$. Biết khi chưa treo vật thì lò xo dài $\SI{18}{cm}$. Lấy $g=\SI{10}{m/s^2}$. Độ cứng của lò xo này là
		\begin{mcq}(4)
			\item $\SI{200}{N/m}$.
			\item $\SI{150}{N/m}$.
			\item $\SI{100}{N/m}$.
			\item $\SI{50}{N/m}$.
		\end{mcq}
	}
	\loigiai
	{	\textbf{Đáp án: C.}
		
	Khi lò xo treo thẳng đứng thì
	$$k|\Delta l| = mg \Rightarrow k =\dfrac{mg}{\Delta l} = \SI{100}{N/m}$$
	
	}
	
	\item \mkstar{3}
	
	\cauhoi
	{Hai lò xo A và B có chiều dài tự nhiên bằng nhau. Độ cứng của lò xo A là $\SI{100}{N/m}$. Khi kéo hai lò xo với cùng lực $F$ thì lò xo A dãn $\SI{2}{cm}$, lò xo B dãn $\SI{1}{cm}$. Độ cứng của lò xo B là
		\begin{mcq}(4)
	\item $\SI{200}{N/m}$.
	\item $\SI{150}{N/m}$.
	\item $\SI{100}{N/m}$.
	\item $\SI{50}{N/m}$.
\end{mcq}
	}
	
	\loigiai
	{	\textbf{Đáp án: A.}
		
	Lập tỉ lệ:
	$$\dfrac{k_\text A}{k_\text B} = \dfrac{|\Delta l_\text B|}{|\Delta l_\text A|} = \SI{200}{N/m}$$
	}
	
	\item \mkstar{3}
	
	\cauhoi
	{Một lò xo nằm ngang có chiều dài tự nhiên là $\SI{40}{cm}$, khi bị nén lò xo dài $\SI{35}{cm}$ và lực đàn hồi khi đó bằng $\SI{2}{N}$. Khi lực đàn hồi của lò xo bị nén là $\SI{5}{N}$ thì lò xo có chiều dài
		\begin{mcq}(4)
			\item $\SI{35}{cm}$.
			\item $\SI{32.5}{cm}$.
			\item $\SI{25}{cm}$.
			\item $\SI{27.5}{cm}$.
		\end{mcq}
	}
	
	\loigiai
	{	\textbf{Đáp án: D.}	
		
		Lập tỉ số:
		$$\dfrac{F_\text{đh 1}}{F_\text{đh 2}} = \dfrac{|\Delta l_1|}{|\Delta l_2|} \Rightarrow |\Delta l_2| = \SI{12.5}{cm}$$
		
		Vậy $l_2 = \SI{27.5}{cm}$.
	}
	
	\item \mkstar{4}
	
	\cauhoi
	{Một lò xo đầu trên gắn cố định. Nếu treo vật nặng khối lượng $\SI{600}{g}$ vào một đầu thì lò xo có chiều dài $\SI{23}{cm}$. Nếu treo thêm vật nặng $\SI{800}{g}$ thì lò xo có chiều dài $\SI{24}{cm}$. Biết khi treo cả hai vật trên thì lò xo vẫn ở trong giới hạn đàn hồi. Lấy $g=\SI{10}{m/s^2}$. Độ cứng của lò xo là
		\begin{mcq}(4)
	\item $\SI{200}{N/m}$.
	\item $\SI{400}{N/m}$.
	\item $\SI{600}{N/m}$.
	\item $\SI{800}{N/m}$.
\end{mcq}
	}
	
	\loigiai
	{	\textbf{Đáp án: D.}
		
	Khi treo vật có $m_1=\SI{0.6}{kg}$ thì $l_1 = l_0 + \Delta l_1 = \SI{23}{cm}$. Suy ra:
	
	$$l_0 = l_1 - \dfrac{m_1 g}{k}$$
	
	Khi treo 2 vật có $m_1+m_2 = \SI{1.4}{kg}$ thì $l_2 = l_0 + \Delta l_2 = \SI{24}{cm}$. Suy ra:
	
	$$l_0 = l_2 - \dfrac{(m_1+m_2)g}{k}$$
	
	Vậy $$l_1 - \dfrac{m_1 g}{k} =l_2 - \dfrac{(m_1+m_2)g}{k} \Rightarrow k = \SI{800}{N/m} $$
	
	
	}
	
	
\end{enumerate}

\whiteBGstarEnd

\loigiai
{
	\begin{center}
		\textbf{BẢNG ĐÁP ÁN}
	\end{center}
	\begin{center}
		\begin{tabular}{|m{2.8em}|m{2.8em}|m{2.8em}|m{2.8em}|m{2.8em}|m{2.8em}|m{2.8em}|m{2.8em}|m{2.8em}|m{2.8em}|}
			\hline
			1.D  & 2.C  & 3.A  & 4.D  & 5.D  &  & & &  &  \\
			\hline
			
		\end{tabular}
	\end{center}
}
\section{Tự luận}
\begin{enumerate}[label=\bfseries Câu \arabic*:]
	\item \mkstar{1}
	
	\cauhoi{
		Phát biểu định luật Húc.
	}
	
	\loigiai{
		
		Định luật Húc: Trong giới hạn đàn hồi, độ lớn của lực đàn hồi của lò xo tỉ lệ thuận với độ biến dạng của lò xo:
		$$F_\text{đh} = k |\Delta l|,$$
		trong đó:
		\begin{itemize}
			\item $k$ là độ cứng của lò xo (hay còn gọi là hệ số đàn hồi), đơn vị $\SI{}{N/m}$;
			\item $\Delta l$ là độ biến dạng của lò xo, đơn vị $\SI{}{m}$.
		\end{itemize}
	}
	
	\item \mkstar{2}
	
	\cauhoi
	{Nêu những đặc điểm (về phương, chiều, điểm đặt) của lực đàn hồi của
		\begin{enumerate}
			\item lò xo;
			\item dây cao su, dây thép;
			\item mặt phẳng tiếp xúc.
		\end{enumerate}
	}
	
	\loigiai
	{\begin{enumerate}
			\item Lò xo;
			
			Điểm đặt: 2 đầu lò xo;
			
			Phương: trùng với trục lò xo;
			
			Chiều: ngược chiều biến dạng của lò xo.
			
			\item Dây cao su, dây thép;
			
			Điểm đặt: 2 đầu;
			
			Phương: cùng phương với lực gây biến dạng;
			
			Chiều: hướng từ hai đầu dây vào phần giữa của sợi dây.
			
			\item Mặt phẳng tiếp xúc.
			
			Điểm đặt: tại mặt tiếp xúc;
			
			Phương: vuông góc với mặt tiếp xúc;
			
			Chiều: hướng ra ngoài mặt phẳng tiếp xúc.
		\end{enumerate}
	}
	\item \mkstar{3}
	
	\cauhoi
	{Treo một vật có trọng lượng $\SI{2.0}{N}$ vào một cái lò xo, lò xo dãn ra $\SI{10}{mm}$. Treo một vật khác có trọng lượng chưa biết vào lò xo, nó dãn ra $\SI{80}{mm}$.
	\begin{enumerate}
		\item Tính độ cứng của lò xo;
		\item Tính trọng lượng chưa biết.
	\end{enumerate}

	}
	
	\loigiai
	{	\begin{enumerate}
			\item Tính độ cứng của lò xo;
			
	Khi treo vật có trọng lượng $P_1=\SI{2.0}{N}$ thì $\Delta l_1 = \SI{10e-3}{m}$. Khi đó trọng lực và lưc đàn hồi cân bằng nên
	$$F_\text{đh 1} = P_1 = k |\Delta l_1| \Rightarrow k = \SI{200}{N/m}$$
	
			\item Tính trọng lượng chưa biết.
			
	$$P_2 = F_\text{đh 2} =k |\Delta l_2| = \SI{16}{N} $$
		\end{enumerate}
	}
	\item \mkstar{4}
	
	\cauhoi
	{Một lò xo khối lượng không đáng kể, độ cứng $\SI{100}{N/m}$ và có chiều dài tự nhiên $\SI{40}{cm}$. Giữ đầu trên của lò xo cố định và buộc vào đầu dưới của lò xo một vật nặng khối lượng $\SI{500}{g}$, sau đó lại buộc thêm vào điểm chính giữa của lò xo đã bị dãn thêm một vật thứ hai khối lượng $\SI{500}{g}$. Lấy $g=\SI{10}{m/s^2}$. Tính chiều dài của lò xo khi đó.
	}
	
	\loigiai
	{Chiều dài của lò xo khi treo vật thứ nhất:
		$$l_1 = l_0 + \Delta l_1 = l_0 + \dfrac{m_1g}{k} = \SI{45}{cm}$$
		
	Treo thêm vật thứ hai vào điểm chính giữa của lò xo đang dãn ($l_2 = \dfrac{l_1}{2} = \SI{22.5}{cm}$) thì tương ứng với cắt lò xo ra còn một nửa 
	$$k_1l_1= k_2 l_2 \Rightarrow k_2 = \dfrac{k_1l_1}{l_2} = \SI{200}{N/m}$$
	
	Độ biến dạng thêm của lò xo khi treo vật thứ hai:
	$$\Delta l_2 = \SI{2.5}{cm}$$
	
	Vậy với chiều dài khi theo vật thứ nhất là $\SI{45}{cm}$ mà còn biến dạng thêm một đoạn $\SI{2.5}{cm}$ thì chiều dài lò xo lúc này là $\SI{47.5}{cm}$
	
	}
	\item \mkstar{4}
	
	\cauhoi
	{Một lò xo có độ cứng $\SI{100}{N/m}$ được treo thẳng đứng vào một điểm cố định, đầu dưới gắn với vật khối lượng $\SI{1}{kg}$. Vật được đặt trên giá đỡ D. Ban đầu giá đỡ D đứng yên và lò xo dãn $\SI{1}{cm}$. Cho D chuyển động nhanh dần đều hướng xuống với gia tốc $\SI{1}{m/s^2}$. Bỏ qua mọi ma sát và lực cản. Lấy $g=\SI{10}{m/s^2}$. Tính quãng đường mà giá đỡ đi được từ lúc bắt đầu chuyển động đến khi vật rời khỏi giá đỡ và tốc độ của vật khi đó.
	}
	
	\loigiai
	{Độ biến dạng của lò xo khi treo vật:
		$$\Delta l = \SI{10}{cm}$$
		
		Mà khi đặt trên D thì lò xo bị dãn $\SI{1}{cm}$, nên lò xo cần dãn thêm $\SI{9}{cm}$ nữa thì vật sẽ rời khỏi D.
		
		Vậy quãng đường mà D đi xuống thêm được là $s=\SI{8}{cm}$.
		
	Áp dụng công thức:
	$$v^2 -0 = 2aS \Rightarrow v = \SI{40}{cm/s}$$
	}
\end{enumerate}