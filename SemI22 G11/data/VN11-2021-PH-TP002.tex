\whiteBGstarBegin
\setcounter{section}{0}
\section{Trắc nghiệm}
\begin{enumerate}[label=\bfseries Câu \arabic*:]
	
	
	\item \mkstar{1}
	
	\cauhoi
	{Có hai điện tích điểm $q_1$ và $q_2$, chúng đẩy nhau. Khẳng định nào sau đây là đúng?
		\begin{mcq}(2)
			\item $q_1>0$ và $q_2<0$.
			\item $q_1<0$ và $q_2>0$.
			\item $q_1 \cdot q_2 >0$.
			\item $q_1 \cdot q_2 <0$.
		\end{mcq}
		
	}
	\loigiai
	{	\textbf{Đáp án: C.}
		
		Hai điện tích điểm $q_1$ và $q_2$ đẩy nhau khi và chỉ khi chúng cùng dấu:
		$$q_1 \cdot q_2 >0.$$
	}
	\item \mkstar{1}
	
	\cauhoi
	{Nhận định nào dưới đây là \textbf{sai}?
		\begin{mcq}
			\item Hai điện tích cùng loại thì đẩy nhau.
			\item Hai điện tích khác loại thì hút nhau.
			\item Hai thanh nhựa giống nhau, sau khi cọ xát với len dạ, đưa chúng lại gần chúng sẽ hút nhau.
			\item Hai thanh thủy tinh sau khi cọ xát vào lụa, đưa chúng lại gần thì chúng sẽ đẩy nhau.
		\end{mcq}
		
	}
	\loigiai
	{	\textbf{Đáp án: C.}
		
		Hai thanh nhựa gióng nhau, sau khi cọ xát với len dạ, đưa chúng lại gần thì chúng sẽ đẩy nhau và chúng nhiễm điện cùng dấu.
	}
\item \mkstar{2}

\cauhoi
{Hai điện tích đặt gần nhau, nếu giảm khoảng cách giữa chúng đi 2 lần thì lực tương tác giữa 2 vật sẽ
	\begin{mcq}(4)
		\item tăng lên 2 lần.
		\item giảm đi 2 lần.
		\item tăng lên 4 lần.
		\item giảm đi 4 lần.
	\end{mcq}
	
}
\loigiai
{	\textbf{Đáp án: C.}
	
	$F=k \dfrac{q_1q_2}{r^2}$.
	
	$F'=k\dfrac{q_1q_2}{(0,5r)^2}$.
	
	$F'=4F$.
}
\item \mkstar{2}

\cauhoi
{Nếu tăng khoảng cách giữa hai điện tích điểm lên 3 lần thì lực tương tác tĩnh điện giữa chúng sẽ
	\begin{mcq}(4)
		\item tăng lên 3 lần.
		\item giảm đi 3 lần.
		\item tăng lên 9 lần.
		\item giảm đi 9 lần.
	\end{mcq}
	
}
\loigiai
{	\textbf{Đáp án: D.}
	
	$F=k\dfrac{q_1q_2}{r^2}$.
	
	$F'=k \dfrac{q_1q_2}{(3r)^2}$.
	
	$F'=F/9$.
}
\item \mkstar{2}

\cauhoi
{Bốn quả cầu kim loại kích thước giống nhau mang điện tích $\SI{2.3}{\micro C}$, $\SI{-264e-7}{C}$, $\SI{-5.9}{\micro C}$, $\SI{3.6e-5}{C}$. Cho 4 quả cầu đồng thời tiếp xúc nhau sau đó tách chúng ra. Tìm điện tích mỗi quả cầu
	\begin{mcq}(4)
		\item $\SI{0.5}{\micro C}$.
		\item $\SI{1.5}{\micro C}$.
		\item $\SI{2.5}{\micro C}$.
		\item $\SI{3.5}{\micro C}$.
	\end{mcq}
	
}
\loigiai
{	\textbf{Đáp án: B.}
	
	Bảo toàn điện tích: $q_1' = q_2' = q_3' = q_4' = \dfrac{q_1 + q_2 + q_3 + q_4}{4} = \SI{1.5}{\micro C}$.
}
\item \mkstar{2}

\cauhoi
{Lực hút tĩnh điện giữa hai điện tích là $\SI{2e-6}{N}$. Khi đưa chúng xa nhau thêm $\SI{2}{cm}$ thì lực hút là $\SI{5e-7}{N}$. Khoảng cách ban đầu giữa chúng là
	\begin{mcq}(4)
		\item 1 cm.
		\item 2 cm.
		\item 3 cm.
		\item 4 cm.
	\end{mcq}
	
}
\loigiai
{	\textbf{Đáp án: B.}
	
	Lập tỉ số:
	
	$$\dfrac{F_1}{F_2} = \dfrac{4}{1} = \dfrac{(R+\SI{0.02}{m})^2}{R^2} \Rightarrow R=\SI{2}{cm}.$$
}
\item \mkstar{2}

\cauhoi
{Hai điện tích điểm đặt trong chân không cách nhau một khoảng $\SI{4}{cm}$ thì đẩy nhau một lực bằng $\SI{9e-5}{N}$. Để lực đẩy giữa chúng là $\SI{1.6e-4}{N}$ thì khoảng cách giữa chúng là
	\begin{mcq}(4)
		\item 1 cm.
		\item 2 cm.
		\item 3 cm.
		\item 4 cm.
	\end{mcq}
	
}
\loigiai
{	\textbf{Đáp án: C.}
	
	Lập tỉ số:
	$$\dfrac{F_1}{F_2} = \dfrac{9}{16} = \dfrac{R_2^2}{R_1^2} \Rightarrow R_2 = \SI{3}{cm}.$$
}
\item \mkstar{2}

\cauhoi
{Xét tương tác của hai điện tích điểm trong một môi trường xác định. Khi lực đẩy Cu-lông tăng 2 lần thì hằng số điện môi
	\begin{mcq}(2)
		\item tăng 2 lần.
		\item vẫn không đổi.
		\item giảm 2 lần.
		\item giảm 4 lần.
	\end{mcq}
	
}
\loigiai
{	\textbf{Đáp án: B.}
	
	Hằng số điện môi của một môi trường là không thay đổi.
}
\item \mkstar{2}

\cauhoi
{Hai điện tích điểm $q_1=\SI{3}{\micro C}$ và $q_2=\SI{-3}{\micro C}$ đặt trong dầu $(\varepsilon = 2)$ cách nhau một khoảng $r=\SI{3}{cm}$. Lực tương tác giữa hai điện tích đó là
	\begin{mcq}(2)
		\item lực hút, độ lớn $F=\SI{45}{N}$.
		\item lực đẩy, độ lớn $F=\SI{45}{N}$.
		\item lực hút, độ lớn $F=\SI{90}{N}$.
		\item lực đẩy, độ lớn $F=\SI{90}{N}$.
	\end{mcq}
	
}
\loigiai
{	\textbf{Đáp án: A.}
	
	Vì $q_1$ và $q_2$ trái dấu nên lực tĩnh điện giữa chúng là lực hút.
	
	Độ lớn:
	$$F=k \dfrac{|q_1 q_2|}{\varepsilon r^2} = \SI{45}{N}.$$
}
\item \mkstar{2}

\cauhoi
{Hai quả cầu nhỏ có điện tích $\SI{e-7}{C}$ và $\SI{4e-7}{C}$ tương tác với nhau bằng một lực $\SI{0.1}{N}$ trong chân không. Khoảng cách giữa chúng là
	\begin{mcq}(4)
		\item $r=\SI{0.6}{cm}$.
		\item $r=\SI{0.6}{m}$.
		\item $r=\SI{6}{m}$.
		\item $r=\SI{6}{cm}$.
	\end{mcq}
	
}
\loigiai
{	\textbf{Đáp án: D.}
	
	Áp dụng định luật Cu-lông:
	$$F=k \dfrac{|q_1 q_2|}{r^2} \Rightarrow r = \SI{6}{cm}.$$
}
\item \mkstar{2}

\cauhoi
{Hai điện tích điểm nằm yên trong chân không tương tác với nhau một lực $F$. Giảm mỗi điện tích đi một nửa, đồng thời cũng giảm khoảng cách đi một nửa thì lực tương tác giữa chúng
	\begin{mcq}(2)
		\item không đổi.
		\item tăng gấp đôi.
		\item giảm một nửa.
		\item giảm 4 lần.
	\end{mcq}
	
}
\loigiai
{	\textbf{Đáp án: A.}
	
	Lập tỉ lệ:
	$$\dfrac{F_1}{F_2} = \dfrac{q_1 q_2 \cdot (r/2)^2}{(q_1/2 \cdot q_2/2) \cdot r^2}=1.$$
}
\item \mkstar{2}

\cauhoi
{Hai điện tích điểm $q_1$, $q_2$ khi đặt trong không khí chúng hút nhau bằng lực $F$. Khi đưa chúng vào trong dầu có hằng số điện môi $\varepsilon = 2$ thì lực tương tác giữa chúng là
	\begin{mcq}(4)
		\item $F$.
		\item $2F$.
		\item $F/2$.
		\item $F/4$.
	\end{mcq}
	
}
\loigiai
{	\textbf{Đáp án: C.}
	
	Lập tỉ số:
	$$\dfrac{F_1}{F_2} = \dfrac{\varepsilon}{1} = 2 \Rightarrow F_2 = \dfrac{F_1}{2}.$$
}
\item \mkstar{2}

\cauhoi
{Hai điện tích điểm được đặt cố định trong không khí thì lực Cu-lông giữa chúng là $\SI{12}{N}$. Khi cho chúng vào một môi trường có hằng số điện môi khác 1 thì lực tương tác giữa chúng là $\SI{4}{N}$. Hằng số điện môi của môi trường đó là
	\begin{mcq}(4)
		\item $\varepsilon = 9$.
		\item $\varepsilon = 3$.
		\item $\varepsilon = 1/9$.
		\item $\varepsilon = 1/3$.
	\end{mcq}
	
}
\loigiai
{	\textbf{Đáp án: B.}
	
	Lập tỉ số:
	$$\dfrac{F_1}{F_2} = \dfrac{\varepsilon}{1} = 3 \Rightarrow \varepsilon = 3.$$
}
\item \mkstar{2}

\cauhoi
{Cho hai điện tích điểm đặt trong chân không. Khi khoảng cách giữa hai điện tích là $r$ thì lực tương tác điện giữa chúng là $F$. Khi khoảng cách giữa hai điện tích là $3r$ thì lực tương tác điện giữa chúng là
	\begin{mcq}(4)
		\item $F/9$.
		\item $F/3$.
		\item $3F$.
		\item $9F$.
	\end{mcq}
	
}
\loigiai
{	\textbf{Đáp án: A.}
	
	Lập tỉ lệ:
	$$\dfrac{F_1}{F_2} = \dfrac{r_2^2}{r_1^2} = 9 \Rightarrow F_2 = \dfrac{F_1}{9}.$$
}
\item \mkstar{1}

\cauhoi
{Hai điện tích điểm đặt trong chân không cách nhau một đoạn $\SI{4}{cm}$ thì chúng hút nhau bằng lực $\SI{e-5}{N}$. Để lực hút giữa chúng là $\SI{2.5e-6}{N}$ thì chúng phải đặt cách nhau
	\begin{mcq}(4)
		\item 6 cm.
		\item 8 cm.
		\item $\SI{2.5}{cm}$.
		\item 5 cm.
	\end{mcq}
	
}
\loigiai
{	\textbf{Đáp án: B.}
	
	Lập tỉ lệ:
	$$\dfrac{F_1}{F_2} = \dfrac{r_2^2}{r_1^2} = 4 \Rightarrow r_2 = \SI{8}{cm}.$$
}
\item \mkstar{3}

\cauhoi
{Hai điện tích điểm đặt trong không khí cách nhau $\SI{12}{cm}$, lực tương tác giữa chúng bằng $\SI{10}{N}$. Đặt chúng vào trong dầu cách nhau $\SI{8}{cm}$ thì lực tương tác giữa chúng vẫn bằng $\SI{10}{N}$. Hằng số điện môi của dầu là
	\begin{mcq}(4)
		\item $\SI{1.51}{}$.
		\item $\SI{2.01}{}$.
		\item $\SI{2.25}{}$.
		\item $\SI{3.41}{}$.
	\end{mcq}
	
}
\loigiai
{	\textbf{Đáp án: C.}
	
	$F=k \dfrac{q_1 q_2}{r^2}$.
	
	$F'=k \dfrac{q_1 q_2}{\varepsilon r'^2}$.
	
	Mà $F'=F$ nên $\varepsilon = \dfrac{r^2}{r'^2} = \SI{2.25}{}$.
}
\item \mkstar{3}

\cauhoi
{Có 2 hạt bụi trong không khí, mỗi hạt chứa $\SI{5e8}{}$ electron, giữa hai hạt bụi cách nhau $\SI{2}{cm}$. Lực đẩy tĩnh điện giữa hai hạt bụi đó là
	\begin{mcq}(4)
		\item $\SI{1.44e-5}{N}$.
		\item $\SI{1.44e-6}{N}$.
		\item $\SI{1.44e-7}{N}$.
		\item $\SI{1.44e-9}{N}$.
	\end{mcq}
	
}
\loigiai
{	\textbf{Đáp án: C.}
	
	Ta có $q_1=q_2=\SI{5e8}{} \cdot \SI{1.6e-19}{} = \SI{80e-12}{C}$.
	
	Vậy:
	$$F=k \dfrac{q_1 q_2}{r^2} = \SI{1.44e-7}{N}.$$
}
\item \mkstar{3}

\cauhoi
{Hai quả cầu giống nhau mang điện tích có độ lớn như nhau, khi đưa chúng lại gần nhau thì chúng hút nhau. Cho chúng tiếp xúc nhau, sau đó tách chúng ra một khoảng nhỏ thì chúng
	\begin{mcq}(2)
		\item hút nhau.
		\item đẩy nhau.
		\item có thể hút hoặc đẩy.
		\item không còn tương tác hút hay đẩy.
	\end{mcq}
	
}
\loigiai
{	\textbf{Đáp án: D.}
	
	Ta có $q_1=-q_2$.
	
	Bảo toàn điện tích: $q_1' = q_2' = \dfrac{q_1+q_2}{2} = 0$.
	
	Do đó, khi tách chúng ra một khoảng nhỏ thì chúng không còn tương tác hút hay đẩy.
}
\item \mkstar{3}

\cauhoi
{Hai điện tích dương $q_1=q$ và $q_2=4q$ đặt tại hai điểm A, B trong không khí, cách nhau $\SI{12}{cm}$. Gọi M là điểm đặt điện tích $q_0$ mà tại đó lực tĩnh điện tổng hợp tác dụng lên $q_0$ bằng 0. Điểm M cách $q_1$ một khoảng là
	\begin{mcq}(4)
		\item 8 cm.
		\item 6 cm.
		\item 4 cm.
		\item 3 cm.
	\end{mcq}
	
}
\loigiai
{	\textbf{Đáp án: C.}
	
	Ta có:
	$$F_1 = F_2 \Rightarrow \dfrac{q_1}{r_1^2} = \dfrac{q_2}{r_2^2} \Rightarrow \dfrac{q}{r_1^2} = \dfrac{4q}{r_2^2} \Rightarrow r_1=\dfrac{r_2}{2}.$$
	
	Mà $r_1+r_2=\SI{12}{cm}$ (do M nằm bên trong đoạn thẳng AB), nên:
	$$r_1=\SI{4}{cm};\ r_2=\SI{8}{cm}.$$
}
\item \mkstar{4}

\cauhoi
{Hai quả cầu nhỏ giống nhau, có cùng khối lượng $\SI{2.5}{g}$, điện tích $\SI{5e-7}{C}$ được treo tại cùng một điểm bằng hai dây mảnh. Do lực đẩy tĩnh điện hai quả cầu tách ra xa nhau một đoạn $\SI{60}{cm}$, lấy $g=\SI{10}{m/s^2}$. Góc lệch của dây so với phương thẳng đứng là
	\begin{mcq}(4)
		\item $\alpha = 4^\circ$.
		\item $\alpha = 14^\circ$.
		\item $\alpha = 24^\circ$.
		\item $\alpha = 34^\circ$.
	\end{mcq}
	
}
\loigiai
{	\textbf{Đáp án: B.}
	
	Gọi $\alpha$ là góc hợp bởi dây treo và phương thẳng đứng.
	
	Khi cân bằng: $\vec P + \vec F_\text{đ} + \vec T = 0$.
	
	Ta có: $\tan \alpha = \dfrac{F_\text{đ}}{P} = \dfrac{k \cdot \dfrac{|q_1 q_2|}{r^2}}{mg} = \dfrac{1}{4}$.
	
	Suy ra $\alpha = 14^\circ$.
}
\end{enumerate}

\whiteBGstarEnd

\loigiai
{
	\begin{center}
		\textbf{BẢNG ĐÁP ÁN}
	\end{center}
	\begin{center}
		\begin{tabular}{|m{2.8em}|m{2.8em}|m{2.8em}|m{2.8em}|m{2.8em}|m{2.8em}|m{2.8em}|m{2.8em}|m{2.8em}|m{2.8em}|}
			\hline
			1.C  & 2.C  & 3.C  & 4.D  & 5.B  & 6.B  & 7.C  & 8.B  & 9.A  & 10.D  \\
			\hline
			11.A  & 12.C  & 13.B  & 14.A  & 15.B  & 16.C  & 17.C  & 18.D  & 19.C  & 20.B  \\
			\hline
		\end{tabular}
	\end{center}
}
\section{Tự luận}
\begin{enumerate}[label=\bfseries Câu \arabic*:]
	\item \mkstar{1}
	
	\cauhoi{
		Nêu các cách nhiễm điện cho một vật và  ví dụ cho mỗi cách.
	}
	
	\loigiai{
		
		Có 3 cách nhiễm điện cho một vật:
		\begin{itemize}
			\item Nhiễm điện do cọ xát: Cọ xát thước nhựa vào vải len, ta thấy thước nhựa có thể hút được các vật nhẹ như giấy;
			\item Nhiễm điện do tiếp xúc: Cho thanh kim loại không nhiễm điện chạm vào quả cầu đã nhiễm điện thì thanh kim loại nhiễm điện cùng dấu với quả cầu. Đưa thanh kim loại ra xa quả cầu thì thanh kim loại vẫn còn nhiễm điện;
			\item Nhiễm điện do hưởng ứng: Đưa thanh kim loại không nhiễm điện lại gần quả cầu nhiễm điện nhưng không tiếp xúc, thì thanh kim loại bị nhiễm điện. Đầu thanh gần quả cầu nhiễm điện trái dấu với quả cầu, đầu thanh xa quả cầu nhiễm điện cùng dấu với quả cầu. Đưa thanh kim loại ra xa thì nó trở lại trạng thái trung hòa.
		\end{itemize}
	}
	
	\item \mkstar{2}
	
	\cauhoi{
		Khoảng cách giữa một proton và một electron là $r=\SI{5e-9}{m}$, coi rằng proton và electron là điện tích điểm. Lực tương tác giữa chúng là bao nhiêu?
	}
	
	\loigiai{
		
		Áp dụng công thức:
		$$F=k\dfrac{|q_1q_2|}{r^2} = \SI{9.216e-8}{N}.$$
	}
\item \mkstar{3}

\cauhoi{
	Hai điện tích $q_1=\SI{4e-7}{C}$, $q_2=\SI{-4e-7}{C}$ đặt cố định tại hai điểm A và B cách nhau $\text{AB} = a = \SI{3}{cm}$, trong không khí. Hãy xác định lực điện tổng hợp tác dụng lên điện tích $q_3=\SI{4e-7}{C}$ đặt tại điểm C (nằm trên đường thẳng đi qua A và B), với:
	\begin{enumerate}
	\item $\text{CA} = \SI{2}{cm}$; $\text{CB} = \SI{1}{cm}$.
	\item $\text{CA} = \SI{2}{cm}$; $\text{CB} = \SI{5}{cm}$.
	\item $\text{CA} = \text{CB} = \SI{1.5}{cm}$.
\end{enumerate}
}

\loigiai{
	Gọi $\vec F_1$, $\vec F_2$ lần lượt là lực điện của $q_1$, $q_2$ tác dụng lên $q_3$, lực điện tổng hợp tác dụng lên $q_3$ là $\vec F = \vec F_1 + \vec F_2$.
	\begin{enumerate}
		\item $\text{CA} = \SI{2}{cm}$; $\text{CB} = \SI{1}{cm}$.
		
		Vì $\text{AC} + \text{CB} = \text{AB}$ nên điểm C nằm trong đoạn AB. Vì $q_1$ và $q_3$ đều là điện tích dương nên $\vec F_1$ là lực đẩy, còn $q_2$ và $q_3$ trái dấu nên $\vec F_2$ là lực hút.
		
		Do $\vec F_1$ và $\vec F_2$ cùng chiều, do đó lực điện tổng hợp $\vec F$ cùng chiều với $\vec F_1$ và $\vec F_2$, độ lớn:
		$$F=F_1 + F_2 = \SI{18}{N}.$$
		
		\item $\text{CA} = \SI{2}{cm}$; $\text{CB} = \SI{5}{cm}$.
		
		Vì $\text{CB} - \text{CA} = \text{AB}$ nên điểm C nằm ngoài đoạn AB và gần A hơn.
		
		Do $\vec F_1$ và $\vec F_2$ ngược chiều nên lực điện tổng hợp $\vec F$ có chiều hướng ra xa A, độ lớn:
		$$F=F_1 - F_2 = \SI{3.024}{N}.$$
		
		\item $\text{CA} = \text{CB} = \SI{1.5}{cm}$.
		
		Vì $\text{AC} = \text{CB}$ nên điểm C là trung điểm của đoạn AB.
		
		Lực điện tổng hợp $\vec F$ bằng tổng hợp của lực $\vec F_1$ và $\vec F_2$ cùng phương, ngược chiều, cùng độ lớn, do đó tổng hợp lực bằng 0:
		$$F = 0.$$
	\end{enumerate}
}
\item \mkstar{4}

\cauhoi{
	Cho hai điện tích $q_1=\SI{3e-7}{C}$, $q_2=\SI{1.2e-6}{C}$ không cố định. đặt cách nhau một đoạn $a=\SI{6}{cm}$ trong chân không. Người ta đặt thêm một điện tích thứ ba $q_3$ để hệ ba điện tích cân bằng. Hãy xác định vị trí và độ lớn của $q_3$.
}

\loigiai{
	
	Muốn cho hệ ba điện tích nằm cân bằng thì lực điện tác dụng lên điện tích $q_3$ phải bằng 0. Vị trí điểm C phải nằm giữa A, B và được xác định bằng cách:
	$$F_{13} = F_{23}.$$
	
	Đặt $x=\text{CA}$, suy ra:
	$$k \dfrac{|q_1 q_3|}{x^2} = k \dfrac{|q_2 q_3|}{(a-x)^2} \Rightarrow q_1 (a-x)^2 = q_2 x^2.$$
	
	Tính được $x=\SI{2}{cm}$. Ngoài ra, để $q_1$ cũng nằm cân bằng thì
	$$F_{31} = F_{21} \Rightarrow k \dfrac{|q_1q_3|}{x^2} = k \dfrac{|q_1 q_2|}{a^2} \Rightarrow q_3 = \xsi{-\dfrac{4}{3}\cdot 10^{-7}}{C}.$$
}
\item \mkstar{4}

\cauhoi{
	Hai quả cầu nhỏ có cùng khối lượng $m$, cùng điện tích $q$, được treo trong không khí vào cùng một điểm O bằng hai sợi dây mảnh (khối lượng không đáng kể, cách điện, không dãn, chiều dài $l$). Do lực đẩy tĩnh điện, hai quả cầu cách nhau một khoảng $r$ ($r << l$).
	\begin{enumerate}
		\item Tính điện tích của mỗi quả cầu.
		\item Áp dụng số với $m=\SI{1.2}{g}$, $l=\SI{1}{m}$, $r=\SI{6}{cm}$. Lấy $g=\SI{10}{m/s^2}$. Coi như $\sin \alpha \approx \tan \alpha$.
	\end{enumerate}
}

\loigiai{
	
	\begin{enumerate}
		\item Tính điện tích của mỗi quả cầu.
		
		Để quả cầu cân bằng thì:
		$$\vec P + \vec F_\text{đ} + \vec T = 0 \Rightarrow \vec F = \vec P + \vec F_\text{đ}.$$
		
		Mà $\tan \alpha = \dfrac{F_\text{đ}}{P} = \dfrac{kq^2}{mgr^2}$.
		
		Mặt khác, vì $\sin \alpha \approx \tan \alpha$ nên:
		$$\sin \alpha \approx \tan \alpha = \dfrac{r}{2l} \Rightarrow \dfrac{r}{2l} = \dfrac{kq^2}{mgr^2} \Rightarrow |q| = \sqrt{\dfrac{mgr^3}{2lk}}.$$
		\item Áp dụng số với $m=\SI{1.2}{g}$, $l=\SI{1}{m}$, $r=\SI{6}{cm}$. Lấy $g=\SI{10}{m/s^2}$. Coi như $\sin \alpha \approx \tan \alpha$.
		
		$$|q| = \SI{1.2e-8}{C}.$$
	\end{enumerate}
}

\end{enumerate}