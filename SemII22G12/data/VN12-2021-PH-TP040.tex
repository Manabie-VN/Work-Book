\whiteBGstarBegin
\setcounter{section}{0}
\section{Tính chất và cấu tạo hạt nhân}
\begin{enumerate}[label=\bfseries Câu \arabic*:]
	\item \mkstar{1} [Trích đề thi năm 2007]
	
	\cauhoi
	{Hạt nhân Triti ($\ce{^3_1 T}$) có
			\begin{mcq}(2)
				\item 3 nuclôn, trong đó có 1 prôtôn.	
				\item 3 nơtrôn và 1 prôtôn.	
				\item 3 nuclôn, trong dó có 1 nơtrôn.	
				\item 3 prôtôn và 1 nơtrôn.
			\end{mcq}
	}
	
	\loigiai
	{		\textbf{Đáp án: A.}
		
		Hạt nhân Triti ($\ce{^3_1 T}$) có 3 nuclôn, trong đó có 1 prôtôn.
		
	}
	
	\item \mkstar{1} [Trích đề thi năm 2010]
	
	\cauhoi
	{So với hạt nhân $\ce{^{29}_{14} Si}$, hạt nhân $\ce{^{40}_{20} Ca}$ có nhiều hơn
		\begin{mcq}(2)
			\item 11 nơtrôn và 6 prôtôn.	
			\item 5 nơtrôn và 6 prôtôn.	
			\item 6 nơtrôn và 5 prôtôn.	
			\item 5 nơtrôn và l2 prôtôn.
		\end{mcq}
	}
	
	\loigiai
	{		\textbf{Đáp án: B.}
		
		Hạt nhân $\ce{^{29}_{14} Si}$ có $Z=14$ proton, $N=29-14=15$ nơtron.
		
		Hạt nhân $\ce{^{40}_{20} Ca}$ có $Z=20$ proton, $N=40-20=20$ nơtron.
		
		Vậy $\ce{^{40}_{20} Ca}$ có nhiều hơn 6 proton và 5 nơtron so với $\ce{^{29}_{14} Si}$ 
		
	}
	
	\item \mkstar{1} [Trích đề thi năm 2007]
	
	\cauhoi
	{Phát biểu nào là \textbf{sai}?
		\begin{mcq}
			\item Các đồng vị phóng xạ đều không bền.	
			\item Các nguyên tử mà hạt nhân có cùng số prôtôn nhưng có số nơtrôn khác nhau gọi là đồng vị.	
			\item Các đồng vị của cùng một nguyên tố có số nơtrôn khác nhau nên tính chất hóa học khác nhau.
			\item Các đồng vị của cùng một nguyên tố có cùng vị trí trong bảng hệ thống tuần hoàn. 
		\end{mcq}
	}
	
	\loigiai
	{		\textbf{Đáp án: C.}
		
		Các đồng vị của cùng một nguyên tố có tính chất hóa học giống nhau.
		
	}
		
\item \mkstar{1} [Trích đề thi năm 2011]

\cauhoi
{Theo thuyết tương đối, một êlectron có động năng bằng một nửa năng lượng nghỉ của nó thì êlectron này chuyển động với tốc độ bằng
	\begin{mcq} (4)
		\item $\text{2,24}\cdot 10^8\ \text{m/s}$.	
		\item $\text{2,75}\cdot 10^8\ \text{m/s}$.	
		\item $\text{1,67}\cdot 10^8\ \text{m/s}$.	
		\item $\text{2,59}\cdot 10^8\ \text{m/s}$.
	\end{mcq}
}

\loigiai
{		\textbf{Đáp án: A.}
	
	Động năng của hạt theo thuyết tương đối:
	$$W_\text{đ} =\dfrac{m_0c^2}{2}=E-E_0=  m_0 \left(\dfrac{1}{1-\sqrt{\dfrac{v^2}{c^2}}}-1\right)c^2 \Rightarrow v= \SI{2.24e8}{m/s}$$
	
}
\item \mkstar{2}

\cauhoi
{Cho biết khối lượng hạt nhân $\ce{^{234}_{92} U}$ là 233,9904 u. Biết khối lượng của hạt prôtôn và nơtrôn lần lượt là $m_{\ce{p}}= \text{1,007276}\ \text{u}$ và $m_{\ce{n}}= \text{l,008665}\ \text{u}$. Độ hụt khối của hạt nhân $\ce{^{234}_{92} U}$ bằng
	\begin{mcq}(4)
		\item 1,909422 u.	
		\item 3,460 u.	
		\item 0.	
		\item 2,056 u.
	\end{mcq}
}

\loigiai
{		\textbf{Đáp án: A.}
	
	Độ hụt khối:
	$$\Delta m = Zm_{\ce{p}} + (A-Z) m_{\ce{n}} - m_{\ce{X}} = \SI{1.909442}{u}$$
	
}
	\item \mkstar{2}
	
	\cauhoi
	{Urani tự nhiên gồm 3 đồng vị chính là $\ce{^{238} U}$ có khối lượng nguyên tử 238,0508 u (chiếm $\text{99,27}\%$), $\ce{^{235} U}$ có khối lượng nguyên tử 235,0439 u (chiếm 0,72$\%$), $\ce{^{234} U}$ có khối lượng nguyên tử 234,0409 u (chiếm 0,01$\%$). Tính khối lượng trung bình.
		\begin{mcq}(4)
			\item 238,0887 u.	
			\item 238,0587 u.	
			\item 237,0287 u.	
			\item 238,0287 u.
		\end{mcq}
	}
	
	\loigiai
	{		\textbf{Đáp án: D.}
		
		Khối lượng trung bình:
		$$M=\SI{238.0508}{u} \cdot \SI{99.27}{\percent} + \SI{235.0439}{u} \cdot \SI{0.72}{\percent} + \SI{234.0409}{u} \cdot \SI{0.01}{\percent} = \SI{238.0287}{u}$$
		
	}
	\item \mkstar{2}
	
	\cauhoi
	{Nitơ tự nhiên có khối lượng nguyên tử là 14,0067 u gồm 2 đồng vị là $\ce{^{14} N}$ và $\ce{^{15} N}$ có khối lượng nguyên tử lần lượt là 14,00307 u và 15,00011 u. Phần trăm của $\ce{^{15} N}$ trong nitơ tự nhiên bằng
		\begin{mcq}(4)
			\item 0,36$\%$.	
			\item 0,59$\%$.	
			\item 0,43$\%$.	
			\item 0,68$\%$.
		\end{mcq}
	}
	
	\loigiai
	{		\textbf{Đáp án: A.}
		
		Gọi $x$ là phần trăm của $\ce{^15 N}$. Ta có khối lượng trung bình:
		$$M=\SI{14.0067}{u}=\SI{14.00307}{u} \cdot \xsi{100-x}{\percent} + \SI{15.00011}{u} \cdot \xsi{x}{\percent} \Rightarrow x= \SI{0.36}{\percent}$$
		
	}
	\item \mkstar{2} [Trích đề thi năm 2010]
	
	\cauhoi
	{Một hạt có khối lượng nghỉ $m_0$. Theo thuyết tương đối, động năng của hạt này khi chuyển động với tốc độ $\text{0,6}c$ ($c$ là tốc độ ánh sáng trong chân không) là
		\begin{mcq}(4)
			\item $\text{0,36}\ m_0c^2$.	
			\item $\text{1,25}\ m_0c^2$.	
			\item $\text{0,225}\ m_0c^2$.	
			\item $\text{0,25}\ m_0c^2$.
		\end{mcq}
	}
	
	\loigiai
	{		\textbf{Đáp án: D.}
		
		Động năng của hạt khi chuyển động với tốc độ $\text{0,6}c$ theo thuyết tương đối:
		$$W_\text{đ} =E-E_0=  m_0 \left(\dfrac{1}{1-\sqrt{\dfrac{v^2}{c^2}}}-1\right)c^2=\text{0,25}\ m_0c^2$$
		
	}
	\item \mkstar{3}
	
	\cauhoi
	{Tính số hạt prôtôn $\ce{^1_1 p}$ có trong 9 gam nước tinh khiết, biết rằng hyđro là đồng vị $\ce{^1_1 H}$ và ôxy là đồng vị $\ce{^{16}_8 O}$.
		\begin{mcq}(4)
			\item $3\cdot 10^2$.	
			\item $3\cdot 10^{24}$.	
			\item $2\cdot 10^{24}$.	
			\item $2\cdot 10^{20}$.
		\end{mcq}
	}
	
	\loigiai
	{		\textbf{Đáp án: B.}
		
		Số phân tử nước có trong $\SI{9}{g}$ nước tinh khiết:
		$$N=\dfrac{m}{M} N_\text{A} = \SI{3.01e23}{}$$
		
		Mỗi phân tử nước có 2 nguyên tử $\ce{H}$ và 1 nguyên tử $\ce{O}$ nên số hạt nhân $\ce{^1_1 H}$ là $\SI{6.02e23}{}$, số hạt nhân $\ce{^16_8 O}$ là $\SI{3.01e23}{}$.
		
		Tổng số proton:
		$$1 \cdot \SI{6.02e23}{} + 8 \cdot \SI{3.01e23}{} = \SI{3.01e24}{}$$
		
	}
	
\item \mkstar{3} [Trích đề thi năm 2007]

\cauhoi
{Biết số Avôgađrô  là $N_{\text{A}} = \text{6,02} \cdot 10^{23}\ \text{g}/\text{mol}$ và khối lượng mol của uran $\ce{^{238}_{92} U}$ bằng 238 g/mol. Số nơtrôn có trong 119 gam uran $\ce{^{238}_{92} U}$ xấp xỉ bằng
	\begin{mcq} (4)
		\item $\text{8,8}\cdot 10^{25}$.	
		\item $\text{1,2}\cdot 10^{25}$.	
		\item $\text{2,2}\cdot 10^{25}$.	
		\item $\text{4,4}\cdot 10^{25}$.
	\end{mcq}
}

\loigiai
{		\textbf{Đáp án: D.}
	
	Số nơtron trong 1 hạt $\ce{^{238}_{92} U}$ là $N=A-Z = 146$ hạt.
	
	Số hạt $\ce{^{238}_{92} U}$ có trong $\SI{119}{g}$ là $\dfrac{m}{M}N_\text A = \SI{3.011e23}{}$
	
	Vậy số hạt nơtron có trong $\SI{119}{g}$ là $\SI{3.011e23}{} \cdot 146 = \SI{4.4e25}{}$.
	
}


\end{enumerate}

\loigiai
{
	\begin{center}
		\textbf{BẢNG ĐÁP ÁN}
	\end{center}
	\begin{center}
		\begin{tabular}{|m{2.8em}|m{2.8em}|m{2.8em}|m{2.8em}|m{2.8em}|m{2.8em}|m{2.8em}|m{2.8em}|m{2.8em}|m{2.8em}|}
			\hline
			1.A  & 2.B  & 3.C  & 4.A  & 5.A  & 6.D  & 7.A & 8.D & 9.B & 10.D \\
			\hline
			
		\end{tabular}
	\end{center}
}

\section{Năng lượng liên kết của hạt nhân. Phản ứng hạt nhân}
\begin{enumerate}[label=\bfseries Câu \arabic*:]
	\item \mkstar{1}
	
	\cauhoi
	{Hạt nhân $\ce{^{90}_{60} Zr}$ có năng lượng liên kết là 783 MeV. Năng lượng liên kết riêng của hạt nhân này là
		\begin{mcq}(2)
			\item l9,6 MeV/nuclôn.	
			\item 6,0 MeV/nuclôn.	
			\item 8,7 MeV/nuclôn.	
			\item 15,6 MeV/nuclôn.
		\end{mcq}
	}
	
	\loigiai
	{		\textbf{Đáp án: C.}
		
		Năng lượng liên kết riêng:
		$$E_{\text{lkr}} = \dfrac{E_{\text{lk}}}{A} = \SI{8.7}{MeV}$$
		
	}
\item \mkstar{1}

\cauhoi
{Cho hạt prôtôn bắn vào các hạt nhân $\ce{^9_4 Be}$ đang đứng yên, người ta thấy các hạt tạo thành gồm $\ce{^4_2 He}$ và hạt nhân X. Hạt nhân X có cấu tạo gồm
	\begin{mcq}(2)
		\item 3 prôtôn và 3 nơtrôn.	
		\item 3 prôtôn và 6 nơtrôn.	
		\item 2 prôtôn và 2 nơtrôn.
		\item 2 prôtôn và 3 nơtrôn.
	\end{mcq}
}

\loigiai
{		\textbf{Đáp án: A.}
	
	Phương trình phản ứng:
	$$\ce{^1_1 p} + \ce{^9_4 Be} \longrightarrow \ce{^4_2 He} + X$$
	
	Áp dụng định luật bảo toàn số khối và bảo toàn điện tích, tìm được hạt X có 6 nuclon, 3 proton, suy ra có 3 nơtron.
	
}
\item \mkstar{2}

\cauhoi
{Cho khối lượng của: prôtôn; nơtrôn và hạt nhân $^4_2He$ lần lượt là: 1,0073 u; 1,0087u và  4,0015u. Lấy $1\ \text{uc}^2 =\text{931,5}\ \text{MeV}$. Năng lượng liên kết của hạt nhân $^4_2He$ là
	\begin{mcq}(4)
		\item 18,3 eV.	
		\item 30,21 MeV.	
		\item 14,21 MeV.	
		\item 28,41 MeV.
	\end{mcq}
}

\loigiai
{		\textbf{Đáp án: D.}
	
	Độ hụt khối:
	$$\Delta m = Z m_p + (A-Z) m_n - m_{\ce{X}} = \SI{0.0305}{u}$$
	
	Năng lượng liên kết:
	$$E_{\text{lk}} = \Delta m c^2 = \SI{28.41}{MeV}$$
	
}
\item \mkstar{2} [Trích đề thi năm 2010]

\cauhoi
{Cho khối lượng của prôtôn; nơtrôn; $\ce{^{40}_{18} Ar}$; $\ce{^6_3 Li}$ lần lượt 1à: 1,0073 u; 1,0087 u; 39,9525 u; 6,0145 u và l u = 931,5 $\text{MeV/c}^2$. So với năng lượng liên kết riêng của hạt nhân $\ce{^6_3 Li}$ thì năng lượng liên kết riêng của hạt $\ce{^{40}_{18} Ar}$ 
	\begin{mcq}(2)
		\item lớn hơn một lượng là 5,20 MeV.	
		\item lớn hơn một lượng là 3,42 MeV.	
		\item nhỏ hơn một lượng là 3,42 MeV.	
		\item nhỏ hơn một lượng là 5,20 MeV.
	\end{mcq}
}

\loigiai
{		\textbf{Đáp án: B.}
	
	Năng lượng liên kết riêng của $\ce{^6_3 Li}$:
	$$E_{\text{lkr1}} = \dfrac{E_{\text{lk1}}}{A_1} = \SI{5.2}{MeV}$$
	
	Năng lượng liên kết riêng của $\ce{^40_18 Ar}$:
	$$E_{\text{lkr2}} = \dfrac{E_{\text{lk2}}}{A_2} = \SI{8.62}{MeV}$$
	
	Vậy năng lượng liên kết riêng của hạt $\ce{^{40}_{18} Ar}$  lớn hơn $\SI{3.42}{MeV}$.
	
}
	\item \mkstar{2}

\cauhoi
{Cho phản ứng hạt nhân $\ce{^9_4 Be} + \alpha \longrightarrow \ce{^{12}_6 C} + n$, trong đó khối lượng các hạt tham gia và tạo thành trong phản ứng là $m_{\alpha} = \text{4,0015}\ \text{u}$; $m_{\ce{Be}} = \text{9,0122}\ \text{u}$; $m_{\ce C} =\text{12,0000}\ \text{u}$; $m_n = \text{1,0087}\ \text{u}$ và $1\ \text{u} =\text{931,5}\ \text{MeV/c}^2$. Phản ứng hạt nhân này
	\begin{mcq}(2)
		\item thu vào 4,66 MeV.	
		\item tỏa ra 4,66 MeV.	
		\item thu vào 6,46 MeV.	
		\item tỏa ra 6,46 MeV.
	\end{mcq}
}

\loigiai
{		\textbf{Đáp án: B.}
	
	Năng lượng tỏa ra hoặc thu vào:
	$$Q=(m_\text{t} - m_\text{s})c^2 = \SI{4.6575}{MeV}$$
	
	Vì $Q>0$ nên phản ứng tỏa năng lượng.
	
}

\item \mkstar{2}

\cauhoi
{Cho phản ứng hạt nhân $\ce{^{27}_{13} Al} +\alpha \longrightarrow \ce{^{30}_{15} P} +n$, trong đó khối lượng các hạt tham gia và tạo thành trong phản ứng là $m_{\alpha} =\text{4,0016}\ \text{u}$; $m_{\ce{Al}} =\text{26,9743}\ \text{u}$; $m_{\ce{P}}=\text{29,9701}\ \text{u}$; $m_n=\text{1,0087}\ \text{u}$ và $1\ \text{u} =\text{931,5}\ \text{MeV/c}^2$. Phản ứng hạt nhân này
	\begin{mcq}(2)
		\item thu vào 2,7 MeV.	
		\item tỏa ra 2,7 MeV.	
		\item thu vào 4,3 MeV.	
		\item tỏa ra 4,3 MeV.
	\end{mcq}
}

\loigiai
{		\textbf{Đáp án: A.}
	
	Năng lượng tỏa ra hoặc thu vào:
	$$Q=(m_\text{t} - m_\text{s})c^2 = \SI{-2.7}{MeV}$$
	
	Vì $Q<0$ nên phản ứng thu năng lượng.
	
}

\item \mkstar{2} [Trích đề thi năm 2012]

\cauhoi
{Tổng hợp hạt nhân heli $\ce{^4_2 He}$ từ phản ứng hạt nhân $\ce{^1_1 H} + \ce{^7_3 Li} \longrightarrow \ce{^4_2 He} + \text{X}$. Mỗi phản ứng trên tỏa năng lượng 17,3 MeV. Năng lượng tỏa ra khi tổng hợp được 0,5 mol heli là
	\begin{mcq}(4)
		\item 2,6$\cdot 10^{24}$ MeV.	
		\item 2,4$\cdot 10^{24}$ MeV.	
		\item 5,2$\cdot 10^{24}$ MeV.	
		\item 1,3$\cdot 10^{24}$ MeV.
	\end{mcq}
}

\loigiai
{		\textbf{Đáp án: A.}
	
	Phương trình phản ứng:
	$$\ce{^1_1 H} + \ce{^7_3 Li} \longrightarrow \ce{^4_2 He} + \ce{^4_2 He}$$
	
	Vậy mỗi phản ứng tổng hợp được 2 hạt heli.
	
	Trong 0,5 mol có số hạt heli là:
	$$N=\nu N_\text{A} = \SI{3.01e23}{}$$
	
	Suy ra có $\dfrac{N}{2} = \SI{1.505e23}{}$ phản ứng xảy ra. Tính được có $\SI{1.505e23}{} \cdot \SI{17.3}{MeV} = \SI{2.6e24}{MeV}$ năng lượng tỏa ra.
	
}

\item \mkstar{2} [Trích đề thi năm 2009]

\cauhoi
{Cho phản ứng hạt nhân: $\ce{^3_1 T}+ \ce{^2_1D} \longrightarrow \ce{^4_2 He} + \text{X}$. Lấy độ hụt khối của hạt nhân $\ce T$, hạt nhân $\ce D$, hạt nhân He lần lượt là 0,009106 u; 0,002491 u; 0,030382 u và $1\ \text{u} =\text{931,5}\ \text{MeV/c}^2$. Năng lượng tỏa ra của phản ứng xấp xỉ bằng
	\begin{mcq}(4)
		\item 21,076 MeV.	
		\item 200,025 MeV.	
		\item 17,498 MeV.	
		\item 15,017 MeV.
	\end{mcq}
}

\loigiai
{		\textbf{Đáp án: C.}
	
	Hạt nhân X là nơtron $\ce{^1_0 n}$ nên độ hụt khối bằng 0.
	
	Năng lượng tỏa ra của phản ứng:
	$$Q = (\Delta m_{\text s} - \Delta m_{\text t}) c^2 = \SI{17.498}{MeV}$$
	
}

	\item \mkstar{3}
	
	\cauhoi
	{Do sự phát bức xạ nên mỗi ngày (86400 s) khối lượng Mặt Trời giảm một lượng $\text{3,744}\cdot 10^{14}\ \text{kg}$. Biết vận tốc ánh sáng trong chân không là $3\cdot 10^8\ \text{m/s}$. Công suất bức xạ (phát xạ) trung bình của Mặt Trời bằng
		\begin{mcq} (4)
			\item $\text{6,9}\cdot 10^{15}\ \text{MW}$.	
			\item $\text{3,9}\cdot 10^{20}\ \text{MW}$.	
			\item $\text{4,9}\cdot 10^{40}\ \text{MW}$.	
			\item $\text{5,9}\cdot 10^{10}\ \text{MW}$.
		\end{mcq}
	}
	
	\loigiai
	{		\textbf{Đáp án: B.}
		
		Năng lượng Mặt Trời tỏa ra trong 1 ngày:
		$$E=\Delta m c^2 = \SI{3.37e31}{J}$$
		
		Công suất phát xạ trung bình của Mặt Trời (với $t=\SI{86400}{s}$):
		$$\calP = \dfrac{E}{t} = \SI{3.9e26}{W} = \SI{3.9e20}{MW}$$
		
	}
	
	
	
	\item \mkstar{3} [Trích đề thi năm 2007]
	
	\cauhoi
	{Cho khối lượng của hạt nhân $C^{12}$ là  $m_C  = \text{l2,00000}\ \text{u}$; $m_p = \text{l,00728}\ \text{u}$; $m_n =\text{1,00867}\ \text{u}$, $1\ \text{u}=\text{1,66058}\cdot 10^{-27}\ \text{kg}$;  $1\ \text{eV}= \text{1,6}\cdot 10^{-19}\ \text{J}$; $c = 3\cdot 10^8\ \text{m/s}$. Năng lượng tối thiểu để tách hạt nhân $\ce{^12_6 C}$ thành các nuclôn riêng biệt là
		\begin{mcq}(4)
			\item 72,7 MeV.	
			\item 89,4 MeV.	
			\item 44,7 MeV.	
			\item 8,94 MeV.
		\end{mcq}
	}
	
	\loigiai
	{		\textbf{Đáp án: B.}
		
		Độ hụt khối:
		$$\Delta m = Z m_p + (A-Z)m_n - m_{\ce{X}} = \SI{0.0957}{u} = \SI{1.5892e-28}{kg}$$
		
		Năng lượng liên kết = năng lượng tối thiểu để tách hạt nhân $\ce{^12_6 C}$ thành các nuclon riêng biệt:
		$$E_{\text{lk}} = \Delta m c^2 = \SI{1.43e-11}{J} = \SI{89.4}{MeV}$$
		
	}
	
	\item \mkstar{3}
	
	\cauhoi
	{Cho năng lượng liên kết riêng của hạt nhân $\ce{^{56}_{26} Fe}$ là 8,8 MeV. Biết khối lượng của hạt prôtôn và nơtrôn lần lượt là $m_p = \text{1,007276}\ \text{u}$ và $m_n = \text{1,008665}\ \text{u}$, trong đó $1\ \text{u} = \text{931,5}\ \text{MeV/c}^2$. Khối lượng hạt nhân  là
		\begin{mcq}(4)
			\item 55,9200 u.	
			\item 56,4396 u	
			\item 55,9921 u.	
			\item 56,3810 u.
		\end{mcq}
	}
	
	\loigiai
	{		\textbf{Đáp án: A.}
		
		Năng lượng liên kết:
		$$E_{\text{lk}} = E_{\text{lk}} A = \SI{492.8}{MeV}$$
		
		Độ hụt khối:
		$$\Delta m = \dfrac{E_{\text{lk}}}{c^2} = \SI{0.529}{u}$$
		
		Khối lượng hạt nhân:
		$$\Delta m = Z m_p + (A-Z) m_n - m_{\text{X}} \Rightarrow m_{\text{X}} = \SI{55.92}{u}$$
		
	}
	
	
	\item \mkstar{3} [Trích đề thi năm 2010]
	
	\cauhoi
	{Cho ba hạt nhân X, Y và Z có số nuclôn tương ứng là $\text{A}_{\text{X}}$, $\text{A}_{\text{Y}}$, $\text{A}_{\text{Z}}$ với $\text{A}_{\text{X}} = 2\text{A}_{\text{Y}} = \text{0,5}\ \text{A}_{\text{Z}}$. Biết năng lượng liên kết của từng hạt nhân tương ứng là $\Delta E_{\text{X}}$, $\Delta E_{\text{Y}}$, $\Delta E_{\text{Z}}$ với $\Delta E_{\text{Z}} < \Delta E_{\text{X}} < \Delta E_{\text{Y}}$. Sắp xếp các hạt nhân này theo thứ tự tính bền vững giảm dần là
		\begin{mcq}(4)
			\item Y, X, Z.	
			\item Y, Z, X.	
			\item X, Y, Z.	
			\item Z, X, Y.
		\end{mcq}
	}
	
	\loigiai
	{		\textbf{Đáp án: A.}
		
		Chuẩn hóa $A_{\ce{X}} = 1$, ta được số nuclon các hạt tương ứng là $A_{\ce{Y}} = \dfrac{1}{2}$, $A_{\ce{Z}} = 2$.
		
		Ta có $\Delta E_{\ce{Z}} < \Delta E_{\ce{X}}$ mà $A_{\ce{Z}} > A_{\ce{X}}$, nên năng lượng liên kết riêng của Z nhỏ hơn X.
		
		
		Ta có $\Delta E_{\ce{X}} < \Delta E_{\ce{Y}}$ mà $A_{\ce{X}} > A_{\ce{Y}}$, nên năng lượng liên kết riêng của X nhỏ hơn Y.
		
		Vậy năng lượng liên kết riêng của Z nhỏ hơn X, của X nhỏ hơn Y. Do đó theo thứ tự tính bền vững giảm dần là Y, X, Z.
	}
	
	
	
	
	
	\item \mkstar{3}
	
	\cauhoi
	{Urani 238 sau một loạt phóng xạ $\alpha$ và biến thành chì. Phương trình của phản ứng là: $\ce{^{238}_{92} U} \longrightarrow \ce{^{206} _{82} Pb} + x \ce{^4_2 He} + y \ce{^0_{-1} \beta^{-}}$. $y$ có giá trị là
		\begin{mcq}(4)
			\item $y = 4$.	
			\item $y = 5$.	
			\item $y = 6$.	
			\item $y = 8$.
		\end{mcq}
	}
	
	\loigiai
	{		\textbf{Đáp án: C.}
		
		Áp dụng định luật bảo toàn số khối, ta có:
		$$238 = 206 + 4x \Rightarrow x = 8$$
		
		Áp dụng định luật bảo toàn điện tích, ta có:
		$$92 = 82 + 8 \cdot 2 + y \cdot (-1) \Rightarrow y = 6$$
		
	}
	
	\item \mkstar{3}
	
	\cauhoi
	{Trong phản ứng sau đây: $n+ \ce{^{235}_{92} U} \longrightarrow \ce{^{95}_{46} Mo} + \ce{^{139}_{57} La} + 2\text{X} + 12\beta^{-}$. Hạt X là
		\begin{mcq}(4)
			\item Electrôn.	
			\item Prôtôn.	
			\item Hêli.	
			\item Nơtrôn.
		\end{mcq}
	}
	
	\loigiai
	{		\textbf{Đáp án: B.}
		
		Áp dụng định luật bảo toàn số khối, ta có:
		$$1+235=95+139+2\cdot A_{\text{X}} + 7 \cdot 0 \Rightarrow A_{\text{X}} = 1$$
		
		Áp dụng định luật bảo toàn điện tích, ta có:
		$$0 + 92 = 46 + 57 + 2 \cdot Z_{\text{X}} + 7 \cdot (-1) \Rightarrow Z_{\text{X}} = 1$$
		
		Vậy X là hạt proton.
		
	}
	

	\item \mkstar{3}
	
	\cauhoi
	{Cho phản ứng hạt nhân $\ce{^{235}_{92} U} + n \longrightarrow \ce{^{94}_{38} Sr} + \ce{^{140}_{54} Xe} + 2n$. Biết năng lượng liên kết riêng của các hạt nhân trong phản ứng: $\ce{U}$ bằng 7,59 MeV; $\ce{Sr}$ bằng 8,59 MeV và $\ce{Xe}$ bằng 8,29 MeV. Năng lượng tỏa ra của phản ứng là
		\begin{mcq}(4)
			\item 148,4 MeV.	
			\item 144,8 MeV.	
			\item 418,4 MeV.	
			\item 184,4 MeV.
		\end{mcq}
	}
	
	\loigiai
	{		\textbf{Đáp án: D.}
		
		Năng lượng liên kết của $\ce{^{235}_{92} U}$:
		$$E_1 = \SI{7.59}{MeV} \cdot 235 = \SI{1783.65}{MeV}$$
		
		Năng lượng liên kết của $\ce{^{94}_{38} Sr}$:
		$$E_2 = \SI{8.59}{MeV} \cdot 94 = \SI{807.46}{MeV}$$
		
		Năng lượng liên kết của $\ce{^{140}_{54} Xe}$:
		$$E_3 = \SI{8.29}{MeV} \cdot 140 = \SI{1160.6}{MeV}$$
		
		Năng lượng tỏa ra của phản ứng:
		$$Q=E_\text{s} - E_\text{t} = \SI{184.41}{MeV}$$
		
	}
	\item \mkstar{3}
	
	\cauhoi
	{Cho phản ứng hạt nhân sau $\ce{^2_1 D} + \ce{^2_1 D} \longrightarrow \ce{^3_2 He} + n + \text{3,25}\ \text{MeV}$. Biết độ hụt khối của $\ce{^2_1 H}$ là $\Delta m_{\ce D} = \text{0,0024}\ \text{u}$; và $1\ \text{u} = 931\ \text{MeV/c}^2$. Năng lượng liên kết của hạt nhân $\ce{^3_2 He}$ là
		\begin{mcq}(4)
			\item 7,7188 MeV.	
			\item 77,188 MeV.	
			\item 771,88 MeV.	
			\item 7,7188 eV.
		\end{mcq}
	}
	
	\loigiai
	{		\textbf{Đáp án: A.}
		
	Độ hụt khối của He:
		$$\SI{3.25}{MeV} = (\Delta m_\text{s} - \Delta m_\text{t} )c^2 \Rightarrow \Delta m_{\ce{^3_2 He}} = \SI{8.3e-3}{u}$$
		
		Năng lượng liên kết của He:
		$$E_{lk} = \Delta m c^2 = \SI{7.7188}{MeV}$$
		
	}
\item \mkstar{3}

\cauhoi
{Một hạt $\alpha$ bắn vào hạt nhân $\ce{^{27}_{13} Al}$ đứng yên tạo ra hạt nơtron và hạt X. Cho $m_{\alpha}= \text{4,0016}\ \text{u}$; $m_n = \text{1,00866}\ \text{u}$; $m_{\ce{Al}} = \text{26,9744}\ \text{u}$; $m_{\text{X}} =\text{29,9701}\ \text{u}$; $1\ \text{u} = \text{931,5}\ \text{MeV/c}^2$. Các hạt nơtron và X có động năng là 4 MeV và 1,8 MeV. Động năng của hạt $\alpha$ là
	\begin{mcq}(4)
		\item 5,87 MeV.	
		\item 8,37 MeV.	
		\item 7,87 MeV.	
		\item 7,27 MeV.
	\end{mcq}
}

\loigiai
{		\textbf{Đáp án: B.}
	
	Năng lượng tỏa ra hoặc thu vào của phản ứng:
	$$Q=(m_\text{t} - m_\text{s})c^2 = \SI{-2.57}{MeV}$$
	
	Động năng của hạt $\alpha$:
	$$Q=K_{\text s} - K_{\text{t}} \Rightarrow K_{\text X} + K_n - K_\alpha \Rightarrow\ K_\alpha = \SI{8.37}{MeV}$$
	
}
\item \mkstar{4}

\cauhoi
{Hạt $\ce{^{234}_{92} U}$ đang đứng yên thì bị vỡ thành hạt $\alpha$ và hạt $\ce{^{230}_{90} Th}$. Cho $m_{\alpha}=\text{4,0015}\ \text{u}$; $m_{\ce{Th}}=\text{229,9737}\ \text{u}$ và $1\ \text{u} = \text{931,5}\ \text{MeV/c}^2$. Phản ứng không bức xạ sóng gamma. Động năng của hạt $\alpha$ sinh ra bằng 4,0 MeV. Khối lượng hạt nhân $\ce{^{234}_{92} U}$ bằng
	\begin{mcq}(4)
		\item 233,9796 u.	
		\item 234,0032 u.	
		\item 233,6796 u.	
		\item 233,7965 u.
	\end{mcq}
}

\loigiai
{		\textbf{Đáp án: A.}
	
	Áp dụng định luật bảo toàn động lượng:
	$$0 = \vec p_{\alpha} + \vec p_{\ce{Th}} \Rightarrow p_{\alpha} = p_{\ce{Th}}$$
	
	Từ $p^2 = 2mK$, suy ra:
	$$2m_{\alpha} K_{\alpha} = 2 m_{\ce{Th}} K_{\ce{Th}} \Rightarrow K_{\ce{Th}} = \SI{0.07}{MeV}$$
	
	Áp dụng kết hợp 2 công thức tính năng lượng tỏa ra hoặc thu vào:
	$$Q=(m_{\text t} - m_{\text s})c^2 = K_\text{s} - K_{\text t} \Rightarrow (m_{\text t} - m_{\text s})c^2 = \SI{4.07}{MeV} \Rightarrow m_{\ce{U}} = \SI{233.9796}{MeV}$$
	
}

\item \mkstar{4} [Trích đề thi năm 2011] 

\cauhoi
{Bắn một prôtôn vào hạt nhân $\ce{^7_3 Li}$ đứng yên. Phản ứng tạo ra hai hạt nhân X giống nhau bay ra với cùng tốc độ theo các phương hợp với phương tới của prôtôn các góc bằng nhau là $60^\circ$. Lấy khối lượng của mỗi hạt nhân tính theo đơn vị u bằng số khối của nó. Tỉ số giữa tốc độ của prôtôn và tốc độ của hạt nhân X là
	\begin{mcq}(4)
		\item $4$.	
		\item $\dfrac{1}{4}$.	
		\item $2$.	
		\item $\dfrac{1}{2}$.
	\end{mcq}
}

\loigiai
{		\textbf{Đáp án: A.}
	
	Hạt X là $\ce{^4_2 He}$.
	
	Áp dụng định luật bảo toàn động lượng:
	$$\vec p_{\text t} = \vec p_{\text s} \Rightarrow \vec p_p = 2 \vec p_\text{X} \Rightarrow p_p = 2 p_\text{X} \cos 60^\circ$$
	
	Mà $p^2 = 2mK$m suy ra:
	$$p_p^2 = 4 p_\text{X}^2 \cdot \dfrac{1}{4} \Rightarrow m_p K_p = m_\text{X} K_\text{X} \Rightarrow \dfrac{K_p}{K_\text{X}} = \dfrac{m_\text{X}}{m_p} = 4$$
	
}
\item \mkstar{4}

\cauhoi
{Bắn hạt nhân $\alpha$ có động năng 18 MeV vào hạt nhân $\ce{^{14}_7 N}$ đứng yên ta có phản ứng $$ \alpha + \ce{^{14}_7 N} \longrightarrow \ce{^1_1 p} + \ce{^{17}_8 O} $$ Biết các hạt nhân sinh ra có cùng véctơ vận tốc. Cho $m_{\alpha} = \text{4,0015}\ \text{u}$; $m_{\ce p}=\text{1,0073}\ \text{u}$; $m_{\ce N}=\text{13,9992}\ \text{u}$; $m_{\ce O}=\text{16,9947}\ \text{u}$; và $1\ \text{u} = \text{931,5}\ \text{MeV/c}^2$. Động năng của hạt prôtôn sinh ra có giá trị bằng
	\begin{mcq}(4)
		\item 0,111 MeV.	
		\item 0,222 MeV.	
		\item 0,333 MeV.	
		\item 0,938 MeV.
	\end{mcq} 
}

\loigiai
{		\textbf{Đáp án: B.}
	
	Năng lượng tỏa ra hoặc thu vào của phản ứng:
	$$Q=(m_{\text t} - m_{\text s})c^2 = \SI{-1.21}{MeV}$$
	
	Vì các hạt sinh ra có cùng vectơ vận tốc nên $\vec v_{\ce{p}} = \vec v_{\ce{O}}$, suy ra $\vec p_{\ce{p}} = \dfrac{1}{17} \vec p_{\ce{O}}$. Áp dụng định luật bảo toàn động lượng:
	$$\vec p_{\alpha} = \vec p_{\ce{p}} + \vec p_{\ce{O}} = 18 \vec p_{\ce{p}}$$
	
	Mà $p^2 =2mK$, suy ra:
	$$2 m_{\alpha} K_{\alpha} = 18^2\cdot 2m_{\ce{p}} K_{\ce{p}} \Rightarrow K_{\ce{p}} = \SI{0.222}{MeV}$$
	
}
\item \mkstar{4} [Trích đề thi năm 2013]

\cauhoi
{Dùng một hạt $\alpha$ có động năng 7,7 MeV bắn vào hạt nhân $\ce{^{14}_7 N}$ đang đứng yên gây ra phản ứng $$\alpha + \ce{^{14}_7 N} \longrightarrow \ce{^1_1 p} + \ce{^{17}_8 O}$$ Hạt prôtôn bay ra theo phương vuông góc với phương bay tới của hạt $\alpha$. Cho khối lượng các hạt nhân: $m_{\alpha} = \text{4,0015}\ \text{u}$; $m_p=\text{1,0073}\ \text{u}$; $m_{\ce{N}}=\text{13,9992}\ \text{u}$; $m_{\ce{O}}=\text{16,9947}\ \text{u}$; và $1\ \text{u} = \text{931,5}\ \text{MeV/c}^2$. Động năng của hạt nhân $\ce{p}$ là
	\begin{mcq}(4)
		\item 6,145 MeV.	
		\item 2,214 MeV.	
		\item 11,857 MeV.	
		\item 2,075 Mev.
	\end{mcq}
}

\loigiai
{		\textbf{Đáp án: C.}
	
	Áp dụng định luật bảo toàn động lượng:
	$$\vec p_\alpha = \vec p_{\ce{p}} + \vec p_{\ce{O}} \Rightarrow p_{\ce{O}}^2 = p_{\alpha}^2 + p_{\ce{p}}^2$$
	
	Với $p^2=2mK$, ta được:
	$$2m_{\ce{O}} K_{\ce{O}} = 2m_{\alpha} K_{\alpha} + 2m_{\ce{p}} K_{\ce{p}}$$
	
	Mặt khác, từ công thức tính năng lượng tỏa ra hoặc thu vào:
	$$Q=(m_{\text t} - m_\text{s})c^2 = K_{\ce{p}} + K_{\ce{O}} - K_\alpha \Rightarrow K_{\ce{p}} + K_{\ce{O}} - K_\alpha = \SI{-1.21}{MeV}$$
	
	Giải hệ 2 phương trình trên, tính được $K_{\ce{p}} = \SI{11.86}{MeV}$, $K_{\ce{O}} = \SI{4.93}{MeV}$.
	
}
\end{enumerate}

\loigiai
{
	\begin{center}
		\textbf{BẢNG ĐÁP ÁN}
	\end{center}
	\begin{center}
		\begin{tabular}{|m{2.8em}|m{2.8em}|m{2.8em}|m{2.8em}|m{2.8em}|m{2.8em}|m{2.8em}|m{2.8em}|m{2.8em}|m{2.8em}|}
			\hline
			1.C  & 2.A  & 3.D  & 4.B  & 5.B  & 6.A  & 7.A & 8.C & 9.B & 10.B \\
			\hline
			11.A  & 12.A  & 13.C  & 14.B  & 15.D  & 16.A  & 17.B & 18.A & 19.A & 20.B \\
			\hline
			21.C  &  &  &  &  &  & & & & \\
			\hline
			
		\end{tabular}
	\end{center}
}

\section{Phóng xạ}
\begin{enumerate}[label=\bfseries Câu \arabic*:]
	\item \mkstar{1}
	
	\cauhoi
	{Cho phản ứng hạt nhân $^{\text{A}}_{\text{Z}} \text{A} \longrightarrow ^{\text{A}}_{\text{Z+1}} \text{B} + \text{X}$, X là
		\begin{mcq} (4)
			\item hạt $\alpha$.	
			\item hạt $\beta^{-}$.	
			\item hạt $\beta^{+}$ .	
			\item hạt phôtôn.
		\end{mcq}
	}
	
	\loigiai
	{		\textbf{Đáp án: B.}
		
		Áp dụng định luật bảo toàn số khối và bảo toàn điện tích, tìm được X là $\ce{^0_{-1}} e$, nghĩa là hạt $\beta^-$.
		
	}
		\item \mkstar{1}
	
	\cauhoi
	{Cho 2 gam $\ce{^{60}_{27} Co}$ tinh khiết có phóng xạ $\beta^{-}$ với chu kỳ bán rã là 5,33 năm. Sau 15 năm, khối lượng $\ce{^{60}_{27} Co}$ còn lại là
		\begin{mcq}(4)
			\item  0,284 g.
			\item  0,842 g.
			\item  0,482 g.
			\item  0,248 g.
		\end{mcq}
	}
	
	\loigiai
	{		\textbf{Đáp án: A.}
		
		Khối lượng $\ce{^{60}_{27} Co}$ còn lại sau 15 năm:
		$$m=m_0 2^{-\frac{t}{T}} = \SI{0.284}{g}$$
		
	}
	
	\item \mkstar{1}
	
	\cauhoi
	{Gọi $\Delta t$ là khoảng thời gian để số hạt nhân của một chất phóng xạ giảm 4 lần. Sau $2\Delta t$  thì số hạt nhân còn lại bằng bao nhiêu phần trăm ban đầu?
		\begin{mcq}(4)
			\item  25,25$\%$.	
			\item  93,75$\%$.	
			\item  13,5$\%$.	
			\item  6,25$\%$.
		\end{mcq}
	}
	
	\loigiai
	{		\textbf{Đáp án: D.}
		
		Sau 1 chu kì bán rã thì số hạt nhân còn lại giảm 2 lần, suy ra $\Delta t = 2T$ thì số hạt nhân còn lại giảm 4 lần, sau $2\Delta t = 4T$ thì số hạt nhân còn lại là
		$$N=N_0 2^{-\frac{4T}{T}} = \SI{6.25}{\percent}$$
		
	}
	\item \mkstar{2}
	
	\cauhoi
	{Ban đầu có 100 g lượng chất phóng xạ $\ce{^{60}_{27} Co}$ với chu kì bán rã $T=\text{5,33}$ năm. Sau 25 năm, khối lượng và số hạt Coban còn lại bao nhiêu?
		\begin{mcq}(2)
			\item $m=\text{3,873}\ \text{g}$; $N=\text{0,389} \cdot 10^{23}$ hạt.	
			\item $m=\text{2,873}\ \text{g}$; $N=\text{0,286} \cdot 10^{23}$ hạt.	 
			\item $m=\text{4,873}\ \text{g}$; $N=\text{0,490} \cdot 10^{23}$ hạt.		
			\item  $m=\text{3,365}\ \text{g}$; $N=\text{0,338} \cdot 10^{23}$ hạt.	
		\end{mcq}
	}
	
	\loigiai
	{		\textbf{Đáp án: A.}
		
		Số hạt $\ce{Co}$ có trong $\SI{100}{g}$:
		$$N_0 = \dfrac{m}{M} N_\text{A} = \SI{1e24}{}$$
		
		Số hạt $\ce{Co}$ còn lại sau 25 năm:
		$$N=N_0 2^{-\frac{t}{T}} = \SI{3.89e22}{}$$
		
		Khối lượng $\ce{Co}$ còn lại sau 25 năm:
		$$m=m_0 2^{-\frac{t}{T}} = \SI{3,873}{g}$$
		
	}\item \mkstar{2}

\cauhoi
{Chu kì bán rã của chất phóng xạ $\ce{^{90}_{38} Sr}$ là 20 năm. Sau 80 năm có bao nhiêu phần trăm chất phóng xạ đó phân rã thành chất khác?
\begin{mcq}(4)
	\item  6,25$\%$.	
	\item  12,5$\%$.	
	\item  87,5$\%$.	
	\item  93,75$\%$.
\end{mcq}
}

\loigiai
{		\textbf{Đáp án: D.}

Số phần trăm chất phóng xạ bị phân rã:
$$\dfrac{\Delta N}{N_0} = 1-2^{-\frac{t}{T}} = \SI{93.75}{\percent}$$

}

\item \mkstar{3}

\cauhoi
{Một hạt $\ce{^{226} Ra}$ phân rã chuyển thành hạt nhân $\ce{^{222} Rn}$. Xem khối lượng bằng số khối. Nếu có $\SI{226}{g}$ $\ce{^{226} Ra}$ thì sau 2 chu kì bán rã khối lượng $\ce{^{222} Rn}$ tạo thành là:
\begin{mcq}(4)
	\item 55,5 g.	
	\item 56,5 g.	
	\item 169,5 g.	
	\item 166,5 g.
\end{mcq}
}

\loigiai
{		\textbf{Đáp án: D.}

Vì khối lượng chất trước và sau phân rã khác nhau nên ta tính thông qua số hạt.
Số hạt $\ce{Ra}$ ban đầu:
$$N_0 = \dfrac{m}{M} N_\text{A} = \SI{6.02e23}{}$$
Số hạt $\ce{Rn}$ tạo thành:
$$\Delta N = N_0 (1-2^{-\frac{t}{T}}) = \SI{4.515e23}{}$$
Khối lượng $\ce{Rn}$ tạo thành:
$$\Delta m = \Delta N \cdot \SI{222}{u} = \SI{1e26}{u} = \SI{166.5}{g}$$
}


	\item \mkstar{3}
	
	\cauhoi
	{Chu kỳ bán rã của hai chất phóng xạ A, B lần lượt là 20 phút và 40 phút. Ban đầu hai chất phóng xạ có số hạt nhân bằng nhau. Sau 80 phút thì tỉ số các hạt A và B bị phân rã là
		\begin{mcq}(4)
			\item $\dfrac{4}{5}$.	
			\item $\dfrac{5}{4}$.	
			\item $4$.	
			\item $\dfrac{1}{4}$.
		\end{mcq}
	}
	
	\loigiai
	{		\textbf{Đáp án: B.}
		
		Ta có $N_{0 1} = N_{0 2}$, sau $t=80$ phút thì số hạt bị phân rã là:
		$$\dfrac{\Delta N_1}{\Delta N_2} = \dfrac{1-2^{-\frac{t}{T_1}}}{1-2^{-\frac{t}{T_2}}}=\dfrac{5}{4}$$
		
	}

	\item \mkstar{3}
	
	\cauhoi
	{Chất polonium $\ce{^{210}_{84} Po}$ phóng xạ anpha ($\alpha$) và chuyển thành chì $\ce{^{206}_{82} Pb}$ với chu kỳ bán rã là 138,4 ngày. Biết tại điều kiện tiêu chuẩn, mỗi mol khí chiếm một thể tích là $\text{22,4}\ l$. Nếu ban đầu có 5 g chất $\ce{^{210}_{84} Po}$ tinh khiết thì thể tích khí $\ce{He}$ ở điều kiện tiêu chuẩn sinh ra sau một năm là
		\begin{mcq} (4)
			\item $\text{0,484}\ l$.
			\item $\text{8,44}\ l$.
			\item $\text{0,884}\ l$.
			\item $\text{0,448}\ l$.
			
		\end{mcq}
	}
	
	\loigiai
	{		\textbf{Đáp án: D.}
		
		Số hạt $\ce{Po}$ có trong 5 g:
		$$N_0 = \dfrac{m}{M} N_\text{A} = \SI{1.434e22}{}$$
		
		Sau 1 năm (365 ngày), số hạt $\ce{He}$ tạo thành cũng là số hạt $\ce{Po}$ bị phân rã:
		$$\Delta N = N_0 (1-2^{-\frac{t}{T}}) = \SI{1.2e22}{}$$
		
		Số mol $\ce{He}$ tạo thành:
		$$\nu = \dfrac{\Delta N}{N_\text{A}} = \SI{0.02}{mol}$$
		
		Thể tích $\ce{He}$ tạo thành:
		$$V=\nu \cdot \SI{22.4}{} = \SI{0.448}{l}$$
		
	}
\item \mkstar{3} [Trích đề thi năm 2009]

\cauhoi
{Lấy chu kì bán rã của pôlôni $\ce{^{210}_{84} Po}$ là 138 ngày và $N_{\text{A}} = \text{6,02} \cdot 10^{23}\ \text{mol}^{-1}$. Độ phóng xạ của 42 mg Pôlôni là 
	\begin{mcq}(4)
		\item $7 \cdot 10^{12}\ \text{Bq}$.
		\item $7 \cdot 10^{10}\ \text{Bq}$.
		\item $7 \cdot 10^{14}\ \text{Bq}$.
		\item $7 \cdot 10^{9}\ \text{Bq}$.
	\end{mcq}
}

\loigiai
{		\textbf{Đáp án: A.}
	
	Hằng số phóng xạ:
	$$\lambda = \dfrac{\ln 2}{T} = \SI{5.813e-8}{s^{-1}}$$
	
	Số hạt có trong 42 mg $\ce{Po}$:
	$$N = \dfrac{m}{M} N_\text{A} = \SI{1.204e20}{}$$
	
	Độ phóng xạ:
	$$H=\lambda N \approx \SI{7e12}{Bq}$$
	
}

	\item \mkstar{4}
	
	\cauhoi
	{Một chất phóng xạ $\ce{^{210}_{84} Po}$ có chu kỳ bán rã là 138 ngày, ban đầu mẫu chất phóng xạ nguyên chất. Sau thời gian $t$ ngày thì số prôtôn có trong mẫu phóng xạ còn lại là $N_1$. Tiếp sau đó $\Delta t$ ngày thì số nơtrôn có trong mẫu phóng xạ còn lại là $N_2$, biết $N_1=\text{1,158}N_2$. Giá trị của $\Delta t$ gần đúng bằng
		\begin{mcq}(4)
			\item  140 ngày	
			\item  130 ngày 	
			\item  120 ngày	
			\item  110 ngày
		\end{mcq}
	}
	
	\loigiai
	{		\textbf{Đáp án: D.}
		
		Gọi số proton ban đầu có trong mẫu là $84x$, suy ra số nơtron ban đầu có trong mẫu là $126x$.
		
		Sau $t$ ngày, số proton còn lại là $N_1$ thì:
		$$N_1 = 84x \cdot 2^{-\frac{t}{T}} \Rightarrow 2^{-\frac{t}{T}} = \dfrac{N1}{84x}$$
		
		Sau $t+\Delta t$ ngày, số nơtron còn lại là $N_2$ thì:
		$$N_2 = 126x \cdot 2^{-\frac{t+\Delta t}{T}}=126x \cdot \dfrac{N1}{84x} \cdot 2^{-\frac{\Delta t}{T}}=\dfrac{126N_1}{84} \cdot 2^{-\frac{\Delta t}{T}}$$
		
		Ta có $\dfrac{N_1}{N_2} = 1,158$, suy ra:
		$$\dfrac{N_1}{\dfrac{126N_1}{84} \cdot 2^{-\frac{\Delta t}{T}}} = 1,156 \Rightarrow \Delta t = 110$$
		
	}
	





\item \mkstar{4} [Trích đề thi năm 2018]

\cauhoi
{Pôlôni $\ce{^{210}_{84} Po}$ là chất phóng xạ $\alpha$. Ban đầu có một mẫu $\ce{^{210}_{84} Po}$  nguyên chất. Khối lượng $\ce{^{210}_{84} Po}$ trong mẫu ở các thời điểm $t=t_0$, $t=t_0 +2\Delta t$ và $t=t_0 +3\Delta t (\Delta t>0)$ có giá trị lần lượt là $m_0$, $\SI{8}{g}$ và $\SI{1}{g}$. Giá trị của $m_0$ là
	\begin{mcq}(4)
		\item 256 g.	
		\item 128 g.	
		\item 64 g.	
		\item 512 g.
	\end{mcq}
}

\loigiai
{		\textbf{Đáp án: D.}
	
	Xét hai thời điểm là $t_0 + 2\Delta t$ và $t_0+3\Delta t$, ta có:
	$$\SI{8}{g} = m_0 2^{-\frac{t_0 + 2\Delta t}{T}}=m_0 2^{-\frac{t_0}{T}} 2^{-\frac{2\Delta t}{T}}=m_0 2^{-\frac{t_0}{T}} \left(2^{-\frac{\Delta t}{T}}\right)^2$$
	$$\SI{1}{g} = m_0 2^{-\frac{t_0 + 3\Delta t}{T}}=m_0 2^{-\frac{t_0}{T}} 2^{-\frac{3\Delta t}{T}}=m_0 2^{-\frac{t_0}{T}} \left(2^{-\frac{\Delta t}{T}}\right)^3$$
	
	Chia theo vế, ta được:
	$$\dfrac{1}{8} = 2^{-\frac{\Delta t}{T}} \Rightarrow \left(2^{-\frac{\Delta t}{T}}\right)^2 = \dfrac{1}{64}$$
	
	Thay vào phương trình $\SI{8}{g} = m_0 2^{-\frac{t_0 + 2\Delta t}{T}}=m_0 2^{-\frac{t_0}{T}} 2^{-\frac{2\Delta t}{T}}=m_0 2^{-\frac{t_0}{T}} \left(2^{-\frac{\Delta t}{T}}\right)^2$, ta được $m_0 2^{-\frac{t_0}{T}}=\dfrac{8}{1/64} = \SI{512}{g}$. Dựa vào công thức $m=m_02^{-\frac{t}{T}}$, ta thấy đây cũng đồng thời là khối lượng $\ce{Po}$ ban đầu ở thời điểm $t=t_0$.
	
}

\item \mkstar{4}

\cauhoi
{Hạt nhân X phóng xạ biến đổi thành hạt nhân bên Y. Ban đầu ($t = 0$) có một mẫu chất X nguyên chất. Tại từng thời điểm $t_1$ và $t_2$ thì tỉ số giữa số hạt nhân Y và số hạt nhân X ở trong mẫu tương ứng là 2 và 3. Tại thời điểm $t_3 =2t_1 +3t_2$, tỉ số đó là
	\begin{mcq}(4)
		\item 17.	
		\item 575. 	
		\item 107.	
		\item 72.
	\end{mcq}
}

\loigiai
{		\textbf{Đáp án: B.}
	
	Hạt nhân X còn lại trong mẫu là $N$, hạt nhân Y tạo thành là số đã bị phân rã là $\Delta N$.
	
	Xét tại thời điểm $t_1$:
	$$\dfrac{1}{2}=\dfrac{N}{\Delta N} = \dfrac{2^{-\frac{t_1}{T}}}{1-2^{-\frac{t_1}{T}}} \Rightarrow 2^{-\frac{t_1}{T}} = \dfrac{1}{3}$$
	
	Xét tại thời điểm $t_2$:
	$$\dfrac{1}{3}=\dfrac{N}{\Delta N} = \dfrac{2^{-\frac{t_2}{T}}}{1-2^{-\frac{t_2}{T}}} \Rightarrow 2^{-\frac{t_2}{T}} = \dfrac{1}{4}$$
	
	Xét tại thời điểm $t_3 =2t_1 +3t_2$:
	$$\dfrac{N}{\Delta N} = \dfrac{2^{-\frac{t_3}{T}}}{1-2^{-\frac{t_3}{T}}}=\dfrac{2^{-\frac{2t_1 +3t_2}{T}}}{1-2^{-\frac{2t_1 +3t_2}{T}}}=\dfrac{\left(2^{-\frac{t_1}{T}}\right)^2\cdot \left(2^{-\frac{t_2}{T}}\right)^3}{1-\left(2^{-\frac{t_1}{T}}\right)^2\cdot \left(2^{-\frac{t_2}{T}}\right)^3}=\dfrac{1}{575}$$
	
	Vậy tỉ số cần tìm là $\dfrac{\Delta N}{N} = 575$.
	
}



\item \mkstar{4}

\cauhoi
{Đồng vị $\ce{^{210}_{84} Po}$ phóng xạ $\alpha$ tạo thành chì $\ce{^{206}_{86} Pb}$. Ban đầu trong một mẫu chất $\ce{Po}$ có khối lượng 1 mg. Tại thời điểm $t_1$ tỉ lệ giữa số hạt $\ce{Pb}$ và số hạt $\ce{Po}$ trong mẫu là 7 : 1. Tại thời điểm $t_2=t_1 + 414$ ngày thì tỉ lệ đó là 63:1. Chu kỳ phóng xạ của $\ce{Po}$ là
	\begin{mcq}(4)
		\item 138,0 ngày.	
		\item 138,4 ngày.	
		\item 137,8 ngày.	
		\item 138,5 ngày.
	\end{mcq}
}

\loigiai
{		\textbf{Đáp án: A.}
	
	Số hạt $\ce{Po}$ có trong $\SI{1}{mg}$:
	$$N_0 = \dfrac{m}{M} N_text{A} = \SI{2.87e18}{}$$
	
	Số hạt $\ce{Pb}$ tạo thành cũng là số hạt $\ce{Po}$ bị phân rã. Tại thời điểm $t_1$:
	$$\dfrac{\Delta N}{N} = \dfrac{7}{1} = \dfrac{1-2^{-\frac{t_1}{T}}}{2^{-\frac{t_1}{T}}} \Rightarrow 2^{-\frac{t_1}{T}} = \dfrac{1}{8}$$
	
	Tại thời điểm $t_2=t_1+414$:
	$$\dfrac{\Delta N}{N} = \dfrac{63}{1} = \dfrac{1-2^{-\frac{t_2}{T}}}{2^{-\frac{t_2}{T}}}=\dfrac{1-2^{-\frac{t_1+414}{T}}}{2^{-\frac{t_1+414}{T}}}=\dfrac{1-2^{-\frac{t_1}{T}}\cdot 2^{-{\frac{414}{T}}}}{2^{-\frac{t_1}{T}}\cdot 2^{-{\frac{414}{T}}}}$$
	
	Suy ra $2^{-{\frac{414}{T}}} = \dfrac{1}{8}$, vậy $T=138$ ngày.
	
}


\end{enumerate}
\loigiai
{
	\begin{center}
		\textbf{BẢNG ĐÁP ÁN}
	\end{center}
	\begin{center}
		\begin{tabular}{|m{2.8em}|m{2.8em}|m{2.8em}|m{2.8em}|m{2.8em}|m{2.8em}|m{2.8em}|m{2.8em}|m{2.8em}|m{2.8em}|}
			\hline
			1.B  & 2.A  & 3.D  & 4.A  & 5.D  & 6.D  & 7.B & 8.D & 9.A & 10.D \\
			\hline
			11.D  & 12.B  & 13.A  & & & & & & & \\
			\hline
			
		\end{tabular}
	\end{center}
}

\whiteBGstarEnd