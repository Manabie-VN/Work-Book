\whiteBGstarBegin
\setcounter{section}{0}
\section{Lý thuyết: Điện từ trường}
\begin{enumerate}[label=\bfseries Câu \arabic*:]
	
%=======================
	\item \mkstar{1} [4]
	
	\cauhoi
	{Khi nói về điện từ trường, phát biểu nào sau đây là sai?
		\begin{mcq}(1)
			\item Nếu tại một nơi có từ trường biến thiên theo thời gian thì tại đó xuất hiện điện trường xoáy. 
			\item Điện trường và từ trường là hai mặt thể hiện khác nhau của một trường duy nhất gọi là điện từ trường.
			\item Trong quá trình lan truyền điện từ trường, vectơ cường độ điện trường và vectơ cảm úng từ tại mọi điểm luôn vuông góc nhau.
			\item Điện trường không lan truyền được trong điện môi.
		\end{mcq}
	}
	
	\loigiai
	{		\textbf{Đáp án: D.}
		
Điện trường lan truyền cả trong môi trường điện môi lẫn chân không.
		
	}
	
%=======================	
	\item \mkstar{1} [10]
	
	\cauhoi
	{Khi một từ trường biến thiên không đều và không tắt theo thời gian sẽ sinh ra
		\begin{mcq}(2)
			\item điện trường xoáy.      
			\item từ trường đều. 
			\item dòng điện không đổi. 
			\item điện trường đều. 
		\end{mcq}
	}
	
	\loigiai
	{		\textbf{Đáp án: A.}
		
Từ trường biến thiên sinh ra điện trường xoáy.
		
	}

\end{enumerate}

\loigiai
{
	\begin{center}
		\textbf{BẢNG ĐÁP ÁN}
	\end{center}
	\begin{center}
		\begin{tabular}{|m{2.8em}|m{2.8em}|m{2.8em}|m{2.8em}|m{2.8em}|m{2.8em}|m{2.8em}|m{2.8em}|m{2.8em}|m{2.8em}|}
			\hline
			01.D  & 02.A  &  &   &   &   &  &  &  &  \\
			\hline
			
		\end{tabular}
	\end{center}
}

\section{Lý thuyết: Sóng điện từ và truyền sóng điện từ}
\begin{enumerate}[label=\bfseries Câu \arabic*:]

%=======================	
	\item \mkstar{1} [9] %Cau 1
	
	\cauhoi
	{Đặc điểm không phải đặc điểm chung của sóng cơ và sóng điện từ là
		\begin{mcq} (2)
			\item là sóng ngang.
			\item truyền được trong chân không.
			\item bị nhiễu xạ khi gặp vật cản.
			\item mang năng lượng
		\end{mcq}
	}
	
	\loigiai
	{		\textbf{Đáp án: B.}
		
Sóng cơ không truyền được trong chân không.
		
	}

%=======================	
	\item \mkstar{1} [3] %Cau 11
	
	\cauhoi
	{Trong sóng điện từ, điện trường và từ trường tại một điểm luôn dao động
		\begin{mcq}(2)
			\item lệch pha nhau $\dfrac{\pi}{4}$.
			\item cùng pha nhau.
			\item lệch pha nhau $\dfrac{\pi}{2}$.
			\item ngược pha nhau.
		\end{mcq}
	}
	
	\loigiai
	{		\textbf{Đáp án: B.}
		
Trong sóng điện từ, điện trường và từ trường tại một điểm luôn dao động cùng pha nhau.
		
	}

%=======================	
	\item \mkstar{1} [7] %Cau 3
	
	\cauhoi
	{Sóng điện từ
		\begin{mcq}(1)
			\item có điện trường và từ trường tại một điểm dao động cùng phương. 
			\item là sóng dọc hoặc sóng ngang tùy vào môi trường truyền sóng.
			\item không truyền được trong chân không.
			\item là điện từ trường lan truyền trong không gian theo thời gian.
		\end{mcq}
	}
	
	\loigiai
	{		\textbf{Đáp án: D.}
		
Sóng điện từ là điện từ trường lan truyền trong không gian theo thời gian.
		
	}

%=======================	
	\item \mkstar{1} [7] %Cau 4
	
	\cauhoi
	{Một sóng điện từ có tần số $\SI{100}{MHz}$ truyền với tốc độ $\xsi{3.10^{8}}{m/s}$ có bước sóng là
		\begin{mcq}(4)
			\item $\SI{0,3}{m}$ 
			\item $\SI{3}{m}$. 
			\item $\SI{30}{m}$. 
			\item $\SI{300}{m}$. 
		\end{mcq}
	}
	
	\loigiai
	{		\textbf{Đáp án: B.}
		
Bước sóng của sóng điện từ cho bởi:
$$
\lambda = \dfrac{v}{f} = \dfrac{3.10^{8}}{100.10^{6}} = \SI{3}{m}.
$$
	}

%=======================	
	\item \mkstar{1} [12] %Cau 5
	
	\cauhoi
	{Sóng điện từ là 
		\begin{mcq}(1)
			\item điện từ trường lan truyền trong không gian.         
			\item sóng cơ và truyền được trong chất lỏng.
			\item sóng dọc và truyền được trong chân không.  
			\item sóng ngang và không thể truyền trong chân không.
		\end{mcq}
	}
	
	\loigiai
	{		\textbf{Đáp án: A.}
		
Sóng điện từ là điện từ trường lan truyền trong không gian.   
    }

%=======================
	\item \mkstar{1} [10] %Cau 6
	
	\cauhoi
	{Sóng điện từ
		\begin{mcq}(1)
			\item là sóng dọc hoặc sóng ngang.
			\item là điện từ trường lan truyền trong không gian.
			\item có thành phần điện trường và thành phần từ trường tại một điểm dao động cùng phương.
			\item không truyền được trong chân không.
		\end{mcq}
	}
	
	\loigiai
	{		\textbf{Đáp án: B.}
		
Sóng điện từ à điện từ trường lan truyền trong không gian.
		
	}
	
%=======================
	\item \mkstar{1} [4] %Cau 7
	
	\cauhoi
	{Khi nói về sóng điện từ, phát biểu nào sau đây là sai?
		\begin{mcq}(1)
			\item Sóng điện từ mang năng lượng. 
			\item Sóng điện từ tuân theo các quy luật giao thoa, nhiễu xạ. 
			\item Sóng điện từ là sóng ngang. 
			\item Sóng điện từ không truyền được trong chân không.
		\end{mcq}
	}
	
	\loigiai
	{		\textbf{Đáp án: D.}

Sóng điện từ truyền được trong chân không.
		
	}

%=======================		
	\item \mkstar{1} [7] % Cau 8
	
	\cauhoi
	{Đối với sự lan truyền sóng điện từ thì
		\begin{mcq}(1)
			\item vectơ cường độ điện trường cùng phương với phương truyền sóng còn vectơ cảm ứng từ vuông góc với vectơ cường độ điện trường.
			\item vectơ cảm ứng từ cùng phương với phương truyền sóng còn vectơ cường độ điện trường vuông góc với vectơ cảm ứng từ. 
			\item vectơ cường độ điện trường và vectơ cảm ứng từ luôn vuông góc với phương truyền sóng.
			\item vectơ cường độ điện trường và vectơ cảm ứng từ luôn cùng phương với phương truyền sóng.
		\end{mcq}
	}
	
	\loigiai
	{		\textbf{Đáp án: C.}
		
Đối với sự lan truyền sóng điện từ thì vectơ cường độ điện trường và vectơ cảm ứng từ luôn vuông góc với phương truyền sóng.
		
	}
	
%=======================
	\item \mkstar{1} [2] %Cau 9
	
	\cauhoi
	{Sóng điện từ được dùng trong lò vi ba (vi sóng) để hâm nóng, nấu chín thức ăn là
		\begin{mcq}(4)
			\item tia tử ngoại. 
			\item sóng cực ngắn. 
			\item tia hồng ngoại. 
			\item sóng ngắn. 
		\end{mcq}
	}
	
	\loigiai
	{		\textbf{Đáp án: B.}
		
Sóng điện từ được dùng trong lò vi ba (vi sóng) để hâm nóng, nấu chín thức ăn là sóng cực ngắn.
		
	}

%=======================
	\item \mkstar{1} [3] %Cau 10
	
	\cauhoi
	{Trong chân không, các bức xạ được sắp xếp theo tần số tăng dần là
		\begin{mcq}(1)
			\item Tia hồng ngoại, ánh sáng tím, tia tử ngoại, tia Rơn-ghen.
			\item Ánh sáng tím, tia hồng ngoại, tia tử ngoại, tia Rơn-ghen. 
			\item Tia Rơn-ghen, tia tử ngoại, ánh sáng tím, tia hồng ngoại. 
			\item Tia hồng ngoại, ánh sáng tím, tia Rơn-ghen, tia tử ngoại.
		\end{mcq}
	}
	
	\loigiai
	{		\textbf{Đáp án: A.}
		
Trong chân không, các bức xạ được sắp xếp theo tần số tăng dần là tia hồng ngoại, ánh sáng tím, tia tử ngoại, tia Rơn-ghen.
		
	}

%=======================	
	\item \mkstar{2} [7] %Cau 2
	
	\cauhoi
	{Một sóng điện từ truyền qua điểm M trong không gian. Cường độ điện trường và cảm ứng từ tại M biến thiên điều hòa với giá trị cực đại lần lượt là $E_0$ và $B_0$. Khi cảm ứng từ tại M bằng $ 0,5B_0$ thì cường độ điện trường tại đó có độ lớn là
		\begin{mcq}(4)
			\item $2E_0$.
			\item $E_0$.
			\item $0,25E_0$. 
			\item $0,5E_0$. 
		\end{mcq}
	}
	
	\loigiai
	{		\textbf{Đáp án: D.}
		
Vì điện trường và từ trường cùng pha nên khi cảm ứng từ bằng $0,5B_0$ thì cường độ cũng điện trường bằng $0,5E_0$.
		
	}


%=======================	
	\item \mkstar{2} [3]
	
	\cauhoi
	{Một sóng điện từ có tần  số $\xsi{15.10^{6}}{Hz}$ truyền trong môi trường với tốc độ $\xsi{2,25.10^{8}}{m/s}$. Trong môi trường đó, sóng điện từ có bước sóng là
		\begin{mcq}(4)
			\item $\SI{45}{m}$.
			\item $\SI{15}{m}$.
			\item $\SI{7,5}{m}$.
			\item $\SI{6,7}{m}$. 
		\end{mcq}
	}
	
	\loigiai
	{		\textbf{Đáp án: B.}
		
Bước sóng của sóng điện từ cho bởi
$$
\lambda=\dfrac{v}{f}=\dfrac{2,25.10^{8}}{15.10^{16}}= \SI{15}{m}.
$$
		
	}

%=======================	
	\item \mkstar{3} [3]
	
	\cauhoi
	{Tại một điểm có sóng điện từ truyền qua, cường độ điện tường biến thiên theo phương trình $E=E_{0} \cos \left(2 \pi \cdot 10^{8} t-\dfrac{2 \pi}{3}\right)\left(E_{0}>0, t\right.$ tính bằng $s$). Kể từ lúc $t=0$, thời điểm đầu tiên để cảm ứng từ tại điểm đó bằng không là
		\begin{mcq}(4)
			\item $\xsi{\dfrac{10^{-8}}{9}}{s}$.
			\item $\xsi{\dfrac{10^{-8}}{12}}{s}$.
			\item $\xsi{\dfrac{10^{-8}}{8}}{s}$.
			\item $\xsi{\dfrac{10^{-8}}{6}}{s}$.
		\end{mcq}
	}
	
	\loigiai
	{		\textbf{Đáp án: B.}
		
Chu kì của sóng điện từ cho bởi
$$
T=\dfrac{2 \pi}{\omega}=\dfrac{2 \pi}{2 \pi \cdot 10^{8}}= \xsi{10^{-8}}{s}.
$$
Vì cảm ứng từ và cường độ điện trường luôn cùng pha nhau, nên thời điểm đầu tiên cảm ứng từ bằng không cũng chính là thời điểm đầu tiên cường độ điện trường bằng không.

Từ đường tròn pha, ta xác định được thời điểm đầu tiên cường độ điện trường bằng không cho bởi:
$$
t=\dfrac{T}{12}= \xsi{\dfrac{10^{-8}}{12}}{s}.
$$
	}
	
\end{enumerate}

\loigiai
{
	\begin{center}
		\textbf{BẢNG ĐÁP ÁN}
	\end{center}
	\begin{center}
		\begin{tabular}{|m{2.8em}|m{2.8em}|m{2.8em}|m{2.8em}|m{2.8em}|m{2.8em}|m{2.8em}|m{2.8em}|m{2.8em}|m{2.8em}|}
			\hline
			01.B  & 02.B  & 03.D  & 04.B  & 05.A  & 06.B  & 07.D & 08.C & 09.B & 10.A \\
			\hline
			11.D  & 12.B  & 13.B  &       &       &       &      &      &      & \\
			\hline
			
		\end{tabular}
	\end{center}
}

\section{Lý thuyết: Nguyên tắc thông tin liên lạc bằng sóng vô tuyến}
\begin{enumerate}[label=\bfseries Câu \arabic*:]

%=======================	
	\item \mkstar{1} [1]
	
	\cauhoi
	{Máy phát thanh vô tuyến đơn giản không có bộ phận nào sau đây?
		\begin{mcq}(4)
			\item Mạch biến điệu.
			\item Micro.
			\item Mạch tách sóng.
			\item Mạch khuếch đại.
		\end{mcq}
	}
	
	\loigiai
	{		\textbf{Đáp án: C.}
		
Máy phát thanh vô tuyến đơn giản không có bộ phận mạch tách sóng.
	} 
	
%=======================    	
	\item \mkstar{1} [1]
	
	\cauhoi
	{Sóng vô tuyến điện không bị tần điện li hấp thụ, phản xạ và truyền thẳng qua được tầng điện li có bước sóng là
		\begin{mcq}(4)
			\item $\SI{20}{m}$.
			\item $\SI{2}{m}$. 
			\item $\SI{2000}{m}$. 
			\item $\SI{200}{m}$. 
		\end{mcq}
	}
	
	\loigiai
	{		\textbf{Đáp án: B.}
		
Sóng vô tuyến không bị tần điện li hấp thụ, phản xạ và truyền thẳng qua được tầng điện li là sóng cực ngắn có bước sóng vào cỡ vài mét.
		
	}

%=======================	
	\item \mkstar{1} [3]
	
	\cauhoi
	{Ở đảo Song Tử Tây (là đảo lớn thứ hai của quần đảo Trường Sa do Việt Nam quản lí), để có thể xem được các chương trình truyền hình phát sóng qua vệ tinh, người ta dùng anten thu sóng trực tiếp từ vệ tinh, qua bộ xử lí tín hiệu rồi đưa ra màn hình. Sóng điện từ mà anten trực tiếp thu từ vệ tinh là loại 
		\begin{mcq}(4)
			\item sóng ngắn. 
			\item sóng cực ngắn. 
			\item sóng trung. 
			\item sóng dài. 
		\end{mcq}
	}
	
	\loigiai
	{		\textbf{Đáp án: B.}
		
Sóng vô tuyến dùng để truyền phát tín hiệu qua vệ tinh là sóng cực ngắn.
		
	}

%=======================	
	\item \mkstar{1} [7]
	
	\cauhoi
	{Trong “máy bắn tốc độ” của cảnh sát giao thông sử dụng để đo tốc độ của phương tiện giao thông
		\begin{mcq}(1)
			\item chỉ có máy thu sóng vô tuyến.	
			\item có cả máy phát và thu sóng vô tuyến.
			\item chỉ có máy phát sóng vô tuyến.	
			\item không có máy phát và thu sóng vô tuyến.
		\end{mcq}
	}
	
	\loigiai
	{		\textbf{Đáp án: B.}
		
Trong “máy bắn tốc độ” của cảnh sát giao thông sử dụng để đo tốc độ của phương tiện giao thông có cả máy phát và thu sóng vô tuyến.
		
	}

%=======================	
	\item \mkstar{1} [9]
	
	\cauhoi
	{Khi sử dụng máy thu thanh vô tuyến điện, người ta vặn nút dò đài là để
		\begin{mcq}(1)
			\item thay đổi tần số của sóng tới. 
			\item tách tín hiệu cần thu ra khỏi sóng mang cao tần.
			\item thay đổi tần số riêng của mạch chọn sóng.
			\item khuếch đại tín hiệu thu được.
		\end{mcq}
	}
	
	\loigiai
	{		\textbf{Đáp án: C.}
		
Khi sử dụng máy thu thanh vô tuyến điện, người ta vặn nút dò đài là để thay đổi tần số riêng của mạch chọn sóng.
		
	}
	
%=======================	
	\item \mkstar{1} [9]
	
	\cauhoi
	{Trong sơ đồ của một máy phát sóng vô tuyến điện không có mạch
		\begin{mcq}(2)
			\item biến điệu. 
			\item khuếch đại. 
			\item tách sóng. 
			\item phát động cao tần. 
		\end{mcq}
	}
	
	\loigiai
	{		\textbf{Đáp án: C.}
		
Trong sơ đồ của một máy phát sóng vô tuyến điện không có mạch tách sóng. 
		
	}

%=======================
	\item \mkstar{3} [10]
	
	\cauhoi
	{Một mạch dao động ở máy vào của một máy thu thanh gồm cuộn thuần cảm có độ tự cảm $\SI{3}{\mu H}$ và tụ điện có điện dung biến thiên trong khoảng từ $\SI{10}{pF}$ đến $ \SI{500}{pF} $ . Biết rằng, muốn thu được sóng điện từ thì tần số riêng của mạch dao động phải bằng tần số của sóng điện từ cần thu (để có cộng hưởng). Trong không khí, tốc độ truyền sóng điện từ là $ \xsi{3.10^{8}}{m/s} $, máy thu này có thể thu được sóng điện từ có bước sóng trong khoảng

		\begin{mcq}(2)
			\item từ $ \SI{100}{m} $ đến $ \SI{730}{m} $.
			\item từ $ \SI{10,32}{m} $ đến $ \SI{73}{m} $.
			\item từ $ \SI{1,24}{m} $ đến $ \SI{73}{m} $.
			\item từ $ \SI{10}{m} $ đến $ \SI{730}{m} $. 
		\end{mcq}
	}
	
	\loigiai
	{		\textbf{Đáp án: B.}
		
Bước sóng của mạch dao động cho bời:
$$
\lambda=2 \pi c \sqrt{L C}.
$$
Với $C_{1}=10 p F$, ta thu được bước sóng
$$
\lambda_{1}=2 \pi c \sqrt{L C_{1}}=2 \pi \cdot 3.10^{8} \sqrt{3.10^{-6} \cdot 10.10^{-12}}= \SI{10,32}{m}.
$$
Với $C_{2}=500 p F$, ta thu được bước sóng
$$
\lambda_{1}=2 \pi c \sqrt{L C_{2}}=2 \pi \cdot 3.10^{8} \sqrt{3.10^{-6} \cdot 500.10^{-12}}= \SI{73}{m}.
$$
Vậy máy này có thể thu được bước sóng trong khoảng từ $\SI{10,32}{m}$ đến \SI{73}{m}.
		
	}
	
\end{enumerate}

\loigiai
{
	\begin{center}
		\textbf{BẢNG ĐÁP ÁN}
	\end{center}
	\begin{center}
		\begin{tabular}{|m{2.8em}|m{2.8em}|m{2.8em}|m{2.8em}|m{2.8em}|m{2.8em}|m{2.8em}|m{2.8em}|m{2.8em}|m{2.8em}|}
			\hline
			01.C  & 02.C  & 03.B  & 04.B  & 05.B  & 06.C  & 07.B &  &  &  \\
			\hline
			
		\end{tabular}
	\end{center}
}

\whiteBGstarEnd