%\setcounter{chapter}{1}
\chapter{Luyện tập: Mẫu nguyên tử Bohr}
\section{Bài toán trạng thái dừng, quỹ đạo dừng}

\begin{enumerate}
	\item \textbf{[Trích đề thi THPT QG năm 2008]} Trong nguyên tử Hidro, bán kính Bo là $r_0=\text{5,3}\cdot 10^{-11}\ \text{m}$. Bán kính quỹ đạo dừng N là
	\begin{mcq}(2)
		\item $\text{47,7}\cdot 10^{-11}\ \text{m}.$
		\item $\text{21,2}\cdot 10^{-11}\ \text{m}.$
		\item $\text{84,4}\cdot 10^{-11}\ \text{m}.$
		\item $\text{132,5}\cdot 10^{-11}\ \text{m}.$
	\end{mcq}
	
	
	\item \textbf{[Trích đề thi THPT QG năm 2011]} Trong nguyên tử Hidro, bán kính Bo là $r_0=\text{5,3}\cdot 10^{-11}\ \text{m}$.   Ở một trạng thái kích thích của nguyên tử Hidro, electron chuyển động trên quỹ đạo dừng có bán kính là $r=\text{2,12}\cdot 10^{-10}\ \text{m}$.   Quỹ đạo đó có tên gọi là 
	
	\begin{mcq}(2)
		\item quỹ đạo dừng L.
		\item quỹ đạo dừng O.
		\item quỹ đạo dừng N.
		\item quỹ đạo dừng M.
	\end{mcq}
	
	\item   Vận dụng mẫu nguyên tử Rutherford cho nguyên tử Hidro. Cho hằng số điện $k=9\cdot 10^9\ \text{Nm}^2/\text{C}^2$, hằng số điện tích nguyên tố $e=\text{1,6}\cdot 10^{-19}\ \text{C}$ và khối lượng của electron $m_\text{e}=\text{9,1}\cdot 10^{-31}\ \text{kg}$. Khi electron chuyển động trên quỹ đạo tròn bán kính $r = \text{2,12} \cdot 10^{-10}\ \text{m}$ thì tốc độ chuyển động của electron xấp xỉ bằng
	
	\begin{mcq}(4)
		\item $\text{1,1}\cdot 10^6\ \text{m/s}.$
		\item $\text{1,4}\cdot 10^6\ \text{m/s}.$
		\item $\text{2,2}\cdot 10^6\ \text{m/s}.$
		\item $\text{3,3}\cdot 10^6\ \text{m/s}.$
	\end{mcq}
	
	\item  \textbf{[Trích đề thi THPT QG năm 2010]} Theo mẫu nguyên tử Bohr, bán kính quỹ đạo K của electron trong nguyên tử Hidro là $r_0$. Khi electron chuyển từ quỹ đạo N về quỹ đạo L thì bán kính quỹ đạo giảm bớt
	
	\begin{mcq}(4)
		\item $12r_0$
		\item $16r_0$.
		\item $9r_0$.
		\item $4r_0$.
	\end{mcq}
	
	\item \textbf{[Trích đề thi THPT QG năm 2012]} Theo mẫu nguyên tử Bo, trong nguyên tử Hidro, chuyển động của electron quanh hạt nhân là chuyển động tròn đều. Tỉ số giữa tốc độ của electron trên quỹ đạo K và tốc độ của electron trên quỹ đạo M bằng
	
	\begin{mcq}(4)
		\item $9.$
		\item $3.$
		\item $4.$
		\item $2.$
	\end{mcq}
	
	\item \textbf{[Trích đề thi THPT QG năm 2014]} Theo mẫu Bo về nguyên tử Hidro, nếu lực tương tác tĩnh điện giữa electron và hạt nhân khi electron chuyển động trên quỹ đạo dừng L là $F$ thì khi electron chuyển động trên quỹ đạo dừng N, lực này sẽ là
	\begin{mcq}(4)
		\item $\dfrac{F}{16}.$
		\item $\dfrac{F}{25}.$
		\item $\dfrac{F}{9}.$
		\item $\dfrac{F}{4}.$
	\end{mcq}
\end{enumerate}

\section{Bài toán về sự hấp thụ và bức xạ}

\begin{enumerate}
	\item  \textbf{[Trích đề thi THPT QG năm 2010]} Khi electron ở quỹ đạo dừng thứ $n$ thì năng lượng của nguyên tử Hidro được tính theo công thức $E_n=-\dfrac{\text{13,6}}{n^2}\ \text{eV}$  . Khi electron trong nguyên tử Hidro chuyển từ quỹ đạo dừng $n = 3$ sang quỹ đạo dừng $n = 2$ thì nguyên tử Hidro phát ra phôtôn ứng với bức xạ có bước sóng bằng
	\begin{mcq}(2)
		\item $\text{0,4350}\ \mu\text{m}$.
		\item $\text{0,6576}\ \mu\text{m}$.
		\item $\text{0,4102}\ \mu\text{m}$.
		\item $\text{0,4861}\ \mu\text{m}$.
	\end{mcq}
	
	
	\item \textbf{[Trích đề thi THPT QG năm 2012]} Theo mẫu nguyên tử Bo, trong nguyên tử Hidro, khi electron chuyển từ quỹ đạo P về quỹ đạo K thì nguyên tử phát ra phôtôn ứng với bức xạ có tần số $f_1$. Khi electron chuyển từ quỹ đạo P về quỹ đạo L thì nguyên tử phát ra phôtôn ứng với bức xạ có tần số $f_2$. Nếu electron chuyển từ quỹ đạo L về quỹ đạo K thì nguyên tử phát ra phôtôn ứng với bức xạ có tần số
	
	\begin{mcq}(2)
		\item $f_3=f_1-f_2$.
		\item $f_3=f_1+f_2$.
		\item $f_3=\sqrt{f^2_1+f^2_2}$.
		\item $f_3=\dfrac{f_1\cdot f_2}{f_1+f_2}$
	\end{mcq}
	
	\item  Cho biết giá trị hằng số $h=\text{6,625}\cdot 10^{-34}\ \text{Js}$, $c=3\cdot 10^8\ \text{m/s}$ và $\text{eV}=\text{1,6}\cdot 10^{-19}\ \text{J}$. Khi electron trong nguyên tử hidro chuyển từ quỹ đạo dừng có năng lượng $E_m=-\text{0,85}\ \text{eV}$ sang quỹ đạo dừng có năng lượng $E_n=-\text{13,6}\ \text{eV}$ thì nguyên tử phát bức xạ điện từ có bước sóng 
	
	\begin{mcq}(4)
		\item $\text{0,0974}\ \mu\text{m}$.
		\item $\text{0,4340}\ \mu\text{m}$.
		\item $\text{0,4860}\ \mu\text{m}$.
		\item $\text{0,6563}\ \mu\text{m}$.
	\end{mcq}
	
	\item   Cho biết giá trị hằng số $h=\text{6,625}\cdot 10^{-34}\ \text{Js}$ và độ lớn điện tích của lectron $e=\text{1,6}\cdot 10^{-19}\ \text{C}$. Khi electron trong nguyên tử hidro chuyển từ quỹ đạo dừng có năng lượng  $E_m=-\text{1,514}\ \text{eV}$ sang quỹ đạo dừng có năng lượng  $E_m=-\text{3,407}\ \text{eV}$ thì nguyên tử phát bức xạ điện từ có tần số
	
	\begin{mcq}(4)
		\item $\text{2,571}\cdot 10^{13}\ \text{Hz}.$
		\item$\text{4,572}\cdot 10^{14}\ \text{Hz}.$
		\item $\text{3,879}\cdot 10^{13}\ \text{Hz}.$
		\item $\text{6,542}\cdot 10^{13}\ \text{Hz}.$
	\end{mcq}
	
	\item Trong nguyên tử hidro, electron từ quỹ đạo L chuyển về quỹ đạo K có năng lượng $E_K= -\text{13,6}\ \text{eV}$. Bước sóng bức xạ phát ra bằng $\text{0,1218}\ \mu\text{m}$. Mức năng lượng ứng với quỹ đạo L bằng 
	
	\begin{mcq}(4)
		\item $\text{3,2}\ \text{eV}.$
		\item $-\text{3,4}\ \text{eV}.$
		\item $-\text{4,1}\ \text{eV}.$
		\item $-\text{5,6}\ \text{eV}.$
	\end{mcq}
	
	\item  Nguyên tử hydro ở trạng thái cơ bản có mức năng lượng bằng $-\text{13,6}\ \text{eV}.$ Để chuyển lên trạng thái dừng có mức năng lượng $-\text{3,4}\ \text{eV}$ thì nguyên tử hidro phải hấp thụ một photon có năng lượng là 
	\begin{mcq}(4) 
		\item $\text{10,2}\ \text{eV}.$
		\item $-\text{10,2}\ \text{eV}.$
		\item $\text{17}\ \text{eV}.$
		\item $\text{4}\ \text{eV}.$
	\end{mcq}
	
	\item Đối với nguyên tử hidro, khi elextron chuyển từ quỹ đạo M về quỹ đạo K thì nguyên tử phát ra photon có bước sóng $\text{0,1026}\ \mu\text{m}$. Cho $h=\text{6,625}\cdot 10^{-34}\ \text{Js}$ và độ lớn điện tích của electron $e=\text{1,6}\cdot 10^{-19}\ \text{C}$. Năng lượng của photon là
	\begin{mcq}(4)
		\item $\text{1,21}\ \text{eV}.$
		\item $\text{11,2}\ \text{eV}.$
		\item $\text{12,1}\ \text{eV}.$
		\item $\text{121}\ \text{eV}.$
	\end{mcq}
	
	\item Cho $h=\text{6,625}\cdot 10^{-34}\ \text{Js}$, $c=3\cdot 10^8\ \text{m/s}$. Mức năng lượng của các quỹ đạo dừng của nguyên tử hidro lần lượt từ trong ra ngoài là $-\text{13,6}\ \text{eV}; \ -\text{3,4}\ \text{eV};\ -\text{1,5}\ \text{eV},...$ với $E_n=-\dfrac{\text{13,6}}{n^2}\ \text{eV}, \ n=1, \ 2, \ 3,...$   Khi electron chuyển từ mức năng lượng ứng với $n = 3$ về $n = 1$ thì sẽ phát ra bức xạ có tần số là 
	\begin{mcq}(4)
		\item $\text{2,9}\cdot 10^{14}\ \text{Hz}.$
		\item$\text{2,9}\cdot 10^{15}\ \text{Hz}.$
		\item $\text{2,9}\cdot 10^{16}\ \text{Hz}.$
		\item $\text{2,9}\cdot 10^{17}\ \text{Hz}.$
	\end{mcq}
\end{enumerate}

\begin{center}
	\textbf{ĐÁP ÁN}
	
\end{center}

\textbf{1. Bài toán trạng thái dừng, quỹ đạo dừng}

\begin{longtable}[\textwidth]{|p{0.1\textwidth}|p{0.1\textwidth}|p{0.1\textwidth}|p{0.1\textwidth}|p{0.1\textwidth}|p{0.1\textwidth}|p{0.1\textwidth}|p{0.1\textwidth}|}
	% --- first head
	\hline%\hspace{2 pt}
	\multicolumn{1}{|c|}{\textbf{Câu 1}} & \multicolumn{1}{c|}{\textbf{Câu 2}} & \multicolumn{1}{c|}{\textbf{Câu 3}} &
	\multicolumn{1}{c|}{\textbf{Câu 4}} &
	\multicolumn{1}{c|}{\textbf{Câu 5}} &
	\multicolumn{1}{c|}{\textbf{Câu 6}} &
	\multicolumn{1}{c|}{\textbf{}} &
	\multicolumn{1}{c|}{\textbf{}}\\
	\hline
	C. &A. &A. &A. &B. &A. & & \\
	\hline
\end{longtable}


\textbf{2. Bài toán về sự hấp thụ và bức xạ}

\begin{longtable}[\textwidth]{|p{0.1\textwidth}|p{0.1\textwidth}|p{0.1\textwidth}|p{0.1\textwidth}|p{0.1\textwidth}|p{0.1\textwidth}|p{0.1\textwidth}|p{0.1\textwidth}|}
	% --- first head
	\hline%\hspace{2 pt}
	\multicolumn{1}{|c|}{\textbf{Câu 1}} & \multicolumn{1}{c|}{\textbf{Câu 2}} & \multicolumn{1}{c|}{\textbf{Câu 3}} &
	\multicolumn{1}{c|}{\textbf{Câu 4}} &
	\multicolumn{1}{c|}{\textbf{Câu 5}} &
	\multicolumn{1}{c|}{\textbf{Câu 6}} &
	\multicolumn{1}{c|}{\textbf{Câu 7}} &
	\multicolumn{1}{c|}{\textbf{Câu 8}}\\
	\hline
	B. &A. &A. &B. &B. &A. &C. &B. \\
	\hline
\end{longtable}











