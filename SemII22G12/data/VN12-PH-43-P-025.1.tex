%\setcounter{chapter}{1}
\chapter{Luyện tập: Sơ lược về LASER}
\begin{enumerate}
	\item Tia laze không có đặc điểm nào dưới đây ?	
	\begin{mcq}
		\item Độ đơn sắc cao. 
		\item Độ định hướng cao.
		\item Cường độ lớn.
		\item Công suất lớn.
	\end{mcq}
	
	\item Laze rubi hoạt động theo nguyên tắc nào dưới đây?
	\begin{mcq}
		\item Dựa vào sự phát xạ cảm ứng.
		\item Tạo ra sự đảo lộn mật độ.
		\item Dựa vào sự tái hợp giữa êlectron và lỗ trống.
		\item Sử dụng buồng cộng hưởng.
	\end{mcq}
	
	\item Một laze He – Ne phát ánh sáng có bước sóng $\text{632,8}\ \text{nm}$ và có công suất đầu ra là $\text{2,3}\ \text{mW}$. Số photon phát ra trong mỗi phút là
	\begin{mcq}(4)
		\item $22\cdot 10^{15}$.
		\item $24\cdot 10^{15}$.
		\item $\text{44}\cdot 10^{16}$.
		\item $44\cdot 10^{15}$.
	\end{mcq}
	
	\item Một laze rubi phát ra ánh sáng có bước sóng $\text{694,4}\ \text{nm}$. Nếu xung laze được phát ra trong $t$ (s) và năng lượng giải phóng bởi mỗi xung là $Q = \text{0,15}\ \text{J}$ thì số photon trong mỗi xung là
	
	
	\begin{mcq}(4)
		\item $\text{22}\cdot 10^{16}$.
		\item $\text{24}\cdot 10^{17}$. 
		\item $\text{5,24}\cdot 10^{17}$.
		\item $\text{5,44}\cdot 10^{15}$.
	\end{mcq}
	
	\item 
	Phát biểu \textbf{sai}?
	Các đặc điểm quan trọng của laze là
	\begin{mcq}
		\item tính đơn sắc cao (do độ sai lệch tỉ đối về tần số của chùm tia laze là rất nhỏ).
		\item tia laze là chùm sáng không kết hợp (do các photon trong chùm tia có khác tần số và khác pha).
		\item tính định hướng cao (do tia laze là chùm sáng song song).
		\item cường độ lớn.
	\end{mcq}
	
	\item Để đo khoảng cách từ Trái Đất lên Mặt Trăng người ta dùng một tia laze phát ra những xung ánh sáng có bước sóng $\text{0,52}\ \mu\text{m}$, chiếu về phía Mặt Trăng. Thời gian kéo dài mỗi xung là $10^{-7}$ (s) và công suất của chùm laze là 100000 MW. Số phôtôn chứa trong mỗi xung là
	\begin{mcq}(2)
		\item $\text{2,62}\cdot 10^{22}$ hạt.
		\item $\text{2,62}\cdot 10^{15}$ hạt.
		\item $\text{2,62}\cdot 10^{29}$ hạt.
		\item $\text{5,2}\cdot 10^{20}$ hạt.
	\end{mcq}
	
	\item Một đèn laze có công suất phát sáng 1W phát ánh sáng đơn sắc có bước sóng $\text{0,7}\ \mu\text{m}$. Cho $h = \text{6,625}\cdot 10^{-34}\ \text{Js}$, $c = 3\cdot 10^8\ \text{m/s}$. Số phôtôn của nó phát ra trong 1 giây là
	\begin{mcq}(4)
		\item $\text{3,52}\cdot 10^{16}$.
		\item  $\text{3,52}\cdot 10^{19}$.
		\item  $\text{3,52}\cdot 10^{18}$.
		\item  $\text{3,52}\cdot 10^{20}$.
	\end{mcq}
	
	
	
	\item Chọn phát biểu đúng.
	\begin{mcq}
		\item Trong y học: làm dao mổ trong phẩu thuật tinh vi như mắt, mạch máu…
		\item Trong thông tin liên lạc: dùng trong liên lạc vô tuyến định vị, liên lạc vệ tinh...
		\item Trong công nghiệp: dùng khoan cắt, tôi…với độ chính xác cao.
		\item Cả ba ý trên.
	\end{mcq}
	
	\item Màu của laze phát ra
	\begin{mcq}
		\item màu trắng.
		\item hỗn hợp màu đơn sắc.
		\item hỗn hợp nhiều màu đơn sắc.
		\item màu đơn sắc.
	\end{mcq}
	
	\item Phát biểu \textbf{sai} về tia laze
	\begin{mcq}
		\item tia laze có tính định hướng cao.
		\item tia laze bị tán sắc khi đi qua lăng kính.
		\item tia laze là chùm sáng kết hợp.
		\item tai laze có cường độ lớn.
	\end{mcq}
	
\end{enumerate}
\begin{center}
	\textbf{Đáp án}
	
\end{center}

\begin{longtable}[\textwidth]{|p{0.1\textwidth}|p{0.1\textwidth}|p{0.1\textwidth}|p{0.1\textwidth}|p{0.1\textwidth}|p{0.1\textwidth}|p{0.1\textwidth}|p{0.1\textwidth}|}
	% --- first head
	\hline%\hspace{2 pt}
	\multicolumn{1}{|c}{\textbf{Câu 1}} & \multicolumn{1}{|c|}{\textbf{Câu 2}} & \multicolumn{1}{c|}{\textbf{Câu 3}} &
	\multicolumn{1}{c|}{\textbf{Câu 4}} &
	\multicolumn{1}{c|}{\textbf{Câu 5}} &
	\multicolumn{1}{c|}{\textbf{Câu 6}} &
	\multicolumn{1}{c|}{\textbf{Câu 7}} &
	\multicolumn{1}{c|}{\textbf{Câu 8}}\\
	\hline
	D. &A. &C. &C. &B. &A. &B. &D.\\
	\hline
	
	\multicolumn{1}{|c|}{\textbf{Câu 9}} & \multicolumn{1}{c|}{\textbf{Câu 10}} & \multicolumn{1}{c|}{\textbf{}} &
	\multicolumn{1}{c|}{\textbf{}} &
	\multicolumn{1}{c|}{\textbf{}} &
	\multicolumn{1}{c|}{\textbf{}} &
	\multicolumn{1}{c|}{\textbf{}} &
	\multicolumn{1}{c|}{} \\
	\hline
	D. &B. & & & & & &\\
	\hline		
\end{longtable}









