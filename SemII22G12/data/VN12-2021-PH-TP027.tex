\whiteBGstarBegin
\setcounter{section}{0}
\begin{enumerate}[label=\bfseries Câu \arabic*:]
	\item \mkstar{1}
	
	\cauhoi
	{Trường hợp nào sau đây có thể coi vật là chất điểm?
		\begin{mcq}
			\item Trái Đất trong chuyển động tự quay quanh mình nó. 
			\item Hai hòn bi lúc va chạm với nhau. 
			\item Hai người nhảy cầu lúc đang rơi xuống nước. 
			\item Giọt nước mưa lúc đang rơi. 
		\end{mcq}
	}
	
	\loigiai
	{		\textbf{Đáp án: D.}
		
		Giọt nước mưa lúc đang rơi có thể coi như là chất điểm.
		
	}
	\item \mkstar{1}
	
	\cauhoi
	{Một người ngồi trên xe đi từ TP HCM ra Đà Nẵng, nếu lấy vật làm mốc là tài xế đang lái xe thì vật chuyển động là
		\begin{mcq}
			\item xe ô tô. 
			\item cột đèn bên đường. 
			\item bóng đèn trên xe. 
			\item hành khách đang ngồi trên xe. 
		\end{mcq}
	}
	
	\loigiai
	{		\textbf{Đáp án: B.}
		
		Một người ngồi trên xe đi từ TP HCM ra Đà Nẵng, nếu lấy vật làm mốc là tài xế đang lái xe thì vật chuyển động là cột đèn bên đường.
		
	}
	\item \mkstar{2}
	
	\cauhoi
	{Một người đứng bên đường quan sát chiếc ô tô chạy qua trước mặt. Dấu hiệu nào cho biết ô tô đang chuyển động? 
		\begin{mcq}
			\item Khói phụt ra từ ống thoát khí đặt dưới gầm xe. 
			\item Khoảng cách giữa xe và người đó thay đổi. 
			\item Bánh xe quay tròn. 
			\item Tiếng nổ của động cơ vang lên. 
		\end{mcq}
	}
	
	\loigiai
	{\textbf{Đáp án: B.}
		
		Một người đứng bên đường quan sát chiếc ô tô chạy qua trước mặt. Dấu hiệu cho biết ô tô đang chuyển động: Khoảng cách giữa xe và người đó thay đổi. 
		
		
	}
	\item \mkstar{2}
	
	\cauhoi
	{Một người chỉ cho một người khách du lịch như sau: "Ông hãy đi dọc theo phố này đến một bờ hồ lớn. Đứng tại đó, nhìn sang bên kia hồ theo hướng Tây Bắc, ông sẽ thấy tòa nhà của khách sạn S." Người chỉ đường đã xác định vị trí của khách sạn theo cách nào?
		\begin{mcq}
			\item Cách dùng đường đi và vật làm mốc. 
			\item Cách dùng các trục tọa độ. 
			\item Dùng cả hai cách A và B. 
			\item Không dùng cả hai cách A và B. 
		\end{mcq}
	}
	
	\loigiai
	{\textbf{Đáp án: C.}
		
		Đi dọc theo phố này đến một bờ hồ lớn là cách dùng đường đi và vật làm mốc.
		
		Đứng ở bờ hồ, nhìn sang hướng Tây Bắc là cách dùng các trục tọa độ.
		
		
	}
	\item \mkstar{2}
	
	\cauhoi
	{Trong các cách chọn hệ trục tọa độ và mốc thời gian dưới đây, cách nào là thích hợp nhất để xác định vị trí của một máy bay đang bay trên đường dài?
		\begin{mcq}
			\item Khoảng cách đến ba sân bay lớn; $t=0$ là lúc máy bay cất cánh.
			\item Khoảng cách đến ba sân bay lớn; $t=0$ là lúc 0 giờ quốc tế.
			\item Kinh độ, vĩ độ địa lý và độ cao của máy bay; $t=0$ là lúc máy bay cất cánh.
			\item Kinh độ, vĩ độ địa lý và độ cao của máy bay; $t=0$ là lúc 0 giờ quốc tế.
		\end{mcq}
	}
	
	\loigiai
	{\textbf{Đáp án: D.}		
		
	}
	
\end{enumerate}

\whiteBGstarEnd

\loigiai
{
	\begin{center}
		\textbf{BẢNG ĐÁP ÁN}
	\end{center}
	\begin{center}
		\begin{tabular}{|m{2.8em}|m{2.8em}|m{2.8em}|m{2.8em}|m{2.8em}|m{2.8em}|m{2.8em}|m{2.8em}|m{2.8em}|m{2.8em}|}
			\hline
			1.D  & 2.B  & 3.B  & 4.C  & 5.D  &   &  &  &  &  \\
			\hline
			
		\end{tabular}
	\end{center}
}