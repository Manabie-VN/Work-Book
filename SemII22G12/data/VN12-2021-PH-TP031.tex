\whiteBGstarBegin
\setcounter{section}{0}
\section{Lý thuyết: Hiện tượng quang điện trong}
\begin{enumerate}[label=\bfseries Câu \arabic*:]
	
	%=======================================
	\item \mkstar{1} [1]
	
	\cauhoi
	{Hiện tượng chất bán dẫn trở nên dẫn điện tốt hơn khi được chiếu sáng thích hợp được gọi là hiện tượng
		\begin{mcq}(4)
			\item phát quang. 
			\item quang dẫn. 
			\item quang điện ngoại. 
			\item quang điện trong. 
		\end{mcq}
	}
	
	\loigiai
	{		\textbf{Đáp án: B.}
		
		Hiện tượng chất bán dẫn trở nên dẫn điện tốt hơn khi được chiếu sáng thích hợp được gọi là hiện tượng quang dẫn.
	}

%=======================================
\item \mkstar{1} [1]

\cauhoi
{Chất quang dẫn CdTe có năng lượng kích hoạt (năng lượng cần thiết để giải phóng một electron liên kết thành một electron dẫn) là $\SI{1,51}{eV}$. Giới hạn quang dẫn của chất này bằng
	\begin{mcq}(4)
		\item $\SI{0,902}{\mu m}$. 
		\item $\SI{0,789}{\mu m}$. 
		\item $\SI{0,822}{\mu m}$. 
		\item $\SI{0,826}{\mu m}$. 
	\end{mcq}
}

\loigiai
{		\textbf{Đáp án: C.}
	
	Giới hạn quang dẫn của chất này bằng
	$$
	\lambda_{0} = \dfrac{hc}{A} = \SI{0,822}{\mu m}.
	$$
}

%=======================================
\item \mkstar{1} [2]

\cauhoi
{Hiện tượng quang điện trong xảy ra khi chất nào sau đây được chiếu sáng thích hợp?
	\begin{mcq}(4)
		\item Fe. 
		\item Zn. 
		\item Ge. 
		\item Cu. 
	\end{mcq}
}

\loigiai
{		\textbf{Đáp án: C.}
	
	Hiện tượng quang điện trong xảy ra khi chất bán dẫn được chiếu sáng thích hợp.
}

%=======================================
\item \mkstar{1} [2]

\cauhoi
{Pin quang điện là nguồn điện trong đó
	\begin{mcq}(1)
		\item cơ năng được trực tiếp biến đổi thành điện năng. 
		\item nhiệt năng được biến đổi trực tiếp thành điện năng. 
		\item hóa năng được trực tiếp biến đổi thành điện năng. 
		\item quang năng được trực tiếp biến đổi thành điện năng. 
	\end{mcq}
}

\loigiai
{		\textbf{Đáp án: D.}
	
	Pin quang điện là nguồn điện trong đó quang năng được trực tiếp biến đổi thành điện năng.
}

%=======================================
\item \mkstar{1} [2]

\cauhoi
{Chọn câu đúng. Điện trở của một quang điện trở thì
	\begin{mcq}(2)
		\item có giá trị rất nhỏ. 
		\item có giá trị không đổi. 
		\item có giá trị rất lớn. 
		\item có giá trị thay đổi được. 
	\end{mcq}
}

\loigiai
{		\textbf{Đáp án: D.}
	
	Điện trở của một quang điện trở giảm nhanh khi có ánh sáng thích hợp chiếu vào.
}

%=======================================
\item \mkstar{1} [2]

\cauhoi
{Ở trên các đoạn đường cao tốc, các bóng đèn được gắn với một thiết bị là quang điện trở. Cứ khi trời tối thì cấc bóng đèn sáng. Đó là ứng dụng của hiện tượng
	\begin{mcq}(2)
		\item quang - phát quang. 
		\item quang dẫn. 
		\item nhiệt điện. 
		\item quang điện ngoài. 
	\end{mcq}
}

\loigiai
{		\textbf{Đáp án: B.}
	
	Ở trên các đoạn đường cao tốc, các bóng đèn được gắn với một thiết bị là quang điện trở. Cứ khi trời tối thì các bóng đèn sáng. Đó là ứng dụng của hiện tượng quang dẫn.
}

%=======================================
\item \mkstar{1} [2]

\cauhoi
{Chọn phát biểu đúng. Hiện tượng quang điện trong là hiện tượng 
	\begin{mcq}(1)
		\item giải phóng electron khỏi một chất bằng cách bằng phá ion. 
		\item giải phóng electron khỏi mối liên kết trong chất bán dẫn khi bị chiếu sáng thích hợp. 
		\item giải phóng electron khỏi kim loại bằng cách đốt nóng. 
		\item bứt electron ra khỏi bề mặt kim loại khi bị chiếu sáng. 
	\end{mcq}
}

\loigiai
{		\textbf{Đáp án: B.}
	
	Hiện tượng quang điện trong là hiện tượng giải phóng electron khỏi mối liên kết trong chất bán dẫn khi có ánh sáng thích hợp chiếu vào.
}

%=======================================
\item \mkstar{1} [3]

\cauhoi
{Điều nào \textbf{sai} về Pin quang điện:
	\begin{mcq}(1)
		\item Hiệu suất của Pin vào khoảng 10\%. 
		\item Đế kim loại trở thành cực dương của Pin. 
		\item Biến quang năng thành điện năng. 
		\item Hoạt động dựa trên hiện tượng quang điện trong. 
	\end{mcq}
}

\loigiai
{		\textbf{Đáp án: B.}
	
	Trong pin quang điện, đế kim loại là cực âm của pin.
}

%=======================================
\item \mkstar{1} [4]

\cauhoi
{Chọn phát biểu \textbf{đúng}:
	\begin{mcq}(1)
		\item Quang trở là một linh kiện bán dẫn hoạt động dựa trên hiện tượng quang điện ngoài. 
		\item Quang trở là một linh kiện bán dẫn hoạt động dựa trên hiện tượng quang điện trong.
		\item Điện trở của quang trở tăng nhanh khi quang trở được chiếu sáng. 
		\item Điện trở của quang trở tăng nhanh khi quang trở được chiếu sáng bởi bước sóng ngắn.
	\end{mcq}
}

\loigiai
{		\textbf{Đáp án: B.}
	
	Quang trở là một linh kiện bán dẫn hoạt động dựa trên hiện tượng quang điện trong.
}

%=======================================
\item \mkstar{1} [10]

\cauhoi
{Khi chiếu chùm sáng có bước sóng thích hợp vào khối chất bán dẫn thì
	\begin{mcq}(1)
		\item mật độ hạt dẫn điện trong khối bán dẫn tăng nhanh. 
		\item nhiệt độ khối bán dẫn giảm nhanh. 
		\item mật độ electron trong khối bán dẫn giảm mạnh. 
		\item cấu trúc tinh thể trong khối bán dẫn thay đổi. 
	\end{mcq}
}

\loigiai
{		\textbf{Đáp án: A.}
	
	Khi chiếu chùm sáng có bước sóng thích hợp vào khối chất bán dẫn thì mật độ hạt dẫn điện trong khối bán dẫn tăng nhanh. 
}

%=======================================
\item \mkstar{1} [12]

\cauhoi
{Thiết bị nào dưới đây hoạt động dựa vào hiện tượng quang điện trong? 
	\begin{mcq}(2)
		\item Máy quang phổ. 
		\item Pin năng lượng mặt trời. 
		\item Laze. 
		\item Ống Cu-lít-giơ. 
	\end{mcq}
}

\loigiai
{		\textbf{Đáp án: B.}
	
	Pin mặt trời hoạt động dựa trên hiện tượng quang điện trong.
}

%=======================================
\item \mkstar{1} [5]

\cauhoi
{Dụng cụ nào dưới đây không làm từ chất bán dẫn?
	\begin{mcq}(4)
		\item Quang điện trở. 
		\item Điôt chỉnh lưu. 
		\item Cặp nhiệt điện. 
		\item Pin mặt trời. 
	\end{mcq}
}

\loigiai
{		\textbf{Đáp án: C.}
	
	Cặp nhiệt điện không làm từ chất bán dẫn.
}

%=======================================
\item \mkstar{1} [7]

\cauhoi
{Pin quang điện hiện nay được chế tạo dựa trên hiện tượng vật lí nào sau đây?
	\begin{mcq}(4)
		\item Quang điện trong. 
		\item Quang điện ngoài. 
		\item Lân quang. 
		\item Huỳnh quang. 
	\end{mcq}
}

\loigiai
{		\textbf{Đáp án: A.}
	
	Pin quang điện hiện nay được chế tạo dựa trên hiện tượng quang điện trong.
}

%=======================================
\item \mkstar{1} [21]

\cauhoi
{Theo định nghĩa, hiện tượng quang điện trong là
	\begin{mcq}(1)
		\item hiện tượng quang điện xảy ra ở bên trong một khối kim loại.
		\item hiện tượng quang điện xảy ra ở bên trong một khối điện môi.
		\item nguyên nhân sinh ra hiện tượng quang dẫn. 
		\item sự giải phóng các electron liên kết để chúng trở thành electron dẫn nhờ tác dụng của một bức xạ điện từ. 
	\end{mcq}
}

\loigiai
{		\textbf{Đáp án: D.}
	
	Theo định nghĩa, hiện tượng quang điện trong là sự giải phóng các electron liên kết để chúng trở thành electron dẫn nhờ tác dụng của một bức xạ điện từ.
}
	
\end{enumerate}

\loigiai
{
	\begin{center}
		\textbf{BẢNG ĐÁP ÁN}
	\end{center}
	\begin{center}
		\begin{tabular}{|m{2.8em}|m{2.8em}|m{2.8em}|m{2.8em}|m{2.8em}|m{2.8em}|m{2.8em}|m{2.8em}|m{2.8em}|m{2.8em}|}
			\hline
			01.B  & 02.C  & 03.C  & 04.D  & 05.D  & 06.B  & 07.B & 08.B & 09.B & 10.A \\
			\hline
			11.B  & 12.C  & 13.A  & 14.D  &  &  & &  &  &  \\
			\hline
			
		\end{tabular}
	\end{center}
}

\section{Dạng bài: Điều kiện xảy ra hiện tượng quang điện trong}
\begin{enumerate}[label=\bfseries Câu \arabic*:]
	
%=======================================
\item \mkstar{1} [4]

\cauhoi
{Một chất quang dẫn có giới hạn quang dẫn là $\SI{0,62}{\mu m}$. Chiếu vào bán dẫn đó lần lượt các chùm bức xạ đơn sắc $f_{1} = \SI{4,5e14}{Hz}$, $f_{2} = \SI{5,0e13}{Hz}$, $f_{3} = \SI{6,5e13}{Hz}$, $f_4 = \SI{6,0e14}{Hz}$ thì hiện tượng quang điện sẽ xảy ra với
	\begin{mcq}(2)
		\item chùm bức xạ $f_{1}$. 
		\item chùm bức xạ $f_{2}$. 
		\item chùm bức xạ $f_{3}$. 
		\item chùm bức xạ $f_{4}$. 
	\end{mcq}
}

\loigiai
{		\textbf{Đáp án: D.}
	
	Tần số tương ứng với giới hạn quang điện là
	$$
	f_{0} = \dfrac{c}{\lambda_{0}} = \SI{4,8e14}{Hz}.
	$$
	Hiện tượng quang điện chỉ xảy ra khi $f \geq f_{0}$. Vậy nên chỉ có chùm bức xạ $f_{4}$ có thể gây ra hiện tượng quang điện.
}

	
\end{enumerate}

\loigiai
{
	\begin{center}
		\textbf{BẢNG ĐÁP ÁN}
	\end{center}
	\begin{center}
		\begin{tabular}{|m{2.8em}|m{2.8em}|m{2.8em}|m{2.8em}|m{2.8em}|m{2.8em}|m{2.8em}|m{2.8em}|m{2.8em}|m{2.8em}|}
			\hline
			01.D  &  &  &  &  &  & &  & &  \\
			\hline
			
		\end{tabular}
	\end{center}
}

\whiteBGstarEnd