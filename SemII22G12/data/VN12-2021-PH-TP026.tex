\whiteBGstarBegin
\setcounter{section}{0}
\section{Lý thuyết: Máy quang phổ}
\begin{enumerate}[label=\bfseries Câu \arabic*:]
	
	%========================================
	\item \mkstar{1} [2]
	
	\cauhoi
	{Khe sáng của ống chuẩn trực của máy quang phổ được đặt tại
		\begin{mcq}(1)
			\item một điểm trên trục chính. 
			\item quang tâm của thấu kính hội tụ.
			\item tiêu điểm ảnh của thấu kính hội tụ. 
			\item tiêu điểm vật chính của thấu kính hội tụ. 
		\end{mcq}
	}
	
	\loigiai
	{		\textbf{Đáp án: D.}
		
		Khe sáng của ống chuẩn trực của máy quang phổ được đặt tại tiêu điểm vật chính của thấu kính hội tụ.
	}

%========================================
\item \mkstar{1} [3]

\cauhoi
{Nguyên tắc hoạt động của máy quang phổ lăng kính dựa vào hiện tượng
	\begin{mcq}(2)
		\item tán sắc ánh sáng.
		\item phản xạ toàn phần. 
		\item khúc xạ ánh sáng.
		\item giao thoa ánh sáng. 
	\end{mcq}
}

\loigiai
{		\textbf{Đáp án: A.}
	
	Nguyên tắc hoạt động của máy quang phổ lăng kính dựa vào hiện tượng tán sắc ánh sáng.
}

%========================================
\item \mkstar{1} [4]

\cauhoi
{Ống chuẩn trực trong máy quang phổ có tác dụng 
	\begin{mcq}(1)
		\item tạo ra chùm tia hội tụ. 
		\item tạo ra chùm sáng song song. 
		\item tạo ra chùm tia phân kì. 
		\item tách chùm sáng phức tạp thành nhiều thành phần. 
	\end{mcq}
}

\loigiai
{		\textbf{Đáp án: B.}
	
	Ống chuẩn trực trong máy quang phổ có tác dụng tạo ra chùm sáng song song. 
}

%========================================
\item \mkstar{1} [13]

\cauhoi
{Máy quang phổ lăng kính gồm các bộ phận chính
	\begin{mcq}(1)
		\item ống chuẩn trực, thấu kính, buồng tối. 
		\item ống chuẩn trực, hệ tán sắc, lăng kính. 
		\item thấu kính phân kì, hệ tán sắc, buồng tối. 
		\item ống chuẩn trực, hệ tán sắc, buồng tối. 
	\end{mcq}
}

\loigiai
{		\textbf{Đáp án: D.}
	
	Máy quang phổ lăng kính gồm các bộ phận chính ống chuẩn trực, hệ tán sắc, buồng tối.
}

%========================================
\item \mkstar{1} [9]

\cauhoi
{Bộ phận có tác dụng phân tích chùm sáng phức tạp thành phần đơn sắc trong máy quang phổ lăng kính là 
	\begin{mcq}(4)
		\item buồng tối. 
		\item tấm kính ảnh. 
		\item lăng kính. 
		\item ống chuẩn trực. 
	\end{mcq}
}

\loigiai
{		\textbf{Đáp án: C.}
	
	Bộ phận có tác dụng phân tích chùm sáng phức tạp thành phần đơn sắc trong máy quang phổ lăng kính là lăng kính. 
}
	
\end{enumerate}

\loigiai
{
	\begin{center}
		\textbf{BẢNG ĐÁP ÁN}
	\end{center}
	\begin{center}
		\begin{tabular}{|m{2.8em}|m{2.8em}|m{2.8em}|m{2.8em}|m{2.8em}|m{2.8em}|m{2.8em}|m{2.8em}|m{2.8em}|m{2.8em}|}
			\hline
			01.D  & 02.A  & 03.B  & 04.D  & 05.C  &  & &  &  &  \\
			\hline
			
		\end{tabular}
	\end{center}
}

\section{Lý thuyết: Các loại quang phổ}
\begin{enumerate}[label=\bfseries Câu \arabic*:]
	
	%========================================
	\item \mkstar{1} [2]
	
	\cauhoi
	{Quang phổ liên tục của một nguồn sáng phụ thuộc vào
		\begin{mcq}(1)
			\item nồng độ cấu tạo chất của nguồn sáng. 
			\item trạng thái cấu tạo của nguồn sáng. 
			\item nhiệt độ của nguồn sáng. 
			\item thành phần cấu tạo của nguồn sáng. 
		\end{mcq}
	}
	
	\loigiai
	{		\textbf{Đáp án: C.}
		
		Quang phổ liên tục của một nguồn sáng phụ thuộc vào thành phần cấu tạo của nguồn sáng. 
	}

%========================================
\item \mkstar{1} [3]

\cauhoi
{Để xác định nhiệt độ của một nguồn sáng ta dựa vào
	\begin{mcq}(2)
		\item quang phổ vạch. 
		\item quang phổ vạch phát xạ. 
		\item quang phổ vạch hấp thụ. 
		\item quang phổ liên tục. 
	\end{mcq}
}

\loigiai
{		\textbf{Đáp án: D.}
	
	Để xác định nhiệt độ của một nguồn sáng ta dựa vào quang phổ liên tục. 
}

%========================================
\item \mkstar{1} [12]

\cauhoi
{Quang phổ liên tục được phát ra khi nung nóng
	\begin{mcq}(1)
		\item chất rắn, chất lỏng, chất khí. 
		\item chất rắn, chất lỏng, chất khí có khối lượng riêng lớn. 
		\item chất rắn và chất lỏng. 
		\item chất rắn. 
	\end{mcq}
}

\loigiai
{		\textbf{Đáp án: B.}
	
	Quang phổ liên tục được phát ra khi nung nóng chất rắn, chất lỏng, chất khí có khối lượng riêng lớn. 
}

%========================================
\item \mkstar{1} [10]

\cauhoi
{Ứng dụng của việc khảo sát quang phổ liên tục là xác định
	\begin{mcq}(1)
		\item thành phần hóa học của một vật nào đó. 
		\item nhiệt độ và thành phần  cấu tạo hóa học của một vật nào đó. 
		\item hình dạng và cấu tạo của vật phát sáng. 
		\item nhiệt độ của các vật có nhiệt độ cao. 
	\end{mcq}
}

\loigiai
{		\textbf{Đáp án: D.}
	
	Ứng dụng của việc khảo sát quang phổ liên tục là xác địnhnhiệt độ của các vật có nhiệt độ cao. 
}

%========================================
\item \mkstar{1} [12]

\cauhoi
{Khi nói về quang phổ liên tục và quang phổ vạch phát xạ, phát biểu nào dưới đây \textbf{không} đúng? 
	\begin{mcq}(1)
		\item Quang phổ liên tục của các chất khác nhau nhưng ở cùng nhiệt độ thì hoàn toàn giống nhau. 
		\item Nguồn phát ra quang phổ vạch phát xạ là các chất khí có áp suất thấp khi bị kích thích. 
		\item Dựa vào quang phổ vạch phát xạ có thể xác định được thành phần cấu tạo của nguồn sáng. 
		\item Dựa vào quang phổ liên tục có thể xác định được thành phần cấu tạo của nguồn sáng. 
	\end{mcq}
}

\loigiai
{		\textbf{Đáp án: D.}
	
	Không thể xác định được thành phần cấu tạo của nguồn sáng từ quang phổ liên tục.
}

%========================================
\item \mkstar{1} [13]

\cauhoi
{Khi nói về quang phổ vạch phát xạ, phát biểu nào sau đây là \textbf{sai}?
	\begin{mcq}(1)
		\item Quang phổ vạch phát xạ của các nguyên tố hóa học khác nhau là khác nhau. 
		\item Hệ thống vạch phát xạ của một nguyên tố là hệ thống những vạch sáng riêng lẻ, ngăn cách nhau bằng những khoảng tối. 
		\item Trong quang phổ vạch phát xạ của hiđrô, ở vùng ánh sáng nhìn thấy có bốn vạch đặc trưng là đỏ, lam, chàm, tím. 
		\item Quang phổ vạch phát xạ phát ra do chất rắn và chất lỏng bị nung nóng.. 
	\end{mcq}
}

\loigiai
{		\textbf{Đáp án: D.}
	
	Quang phổ phát ra do chất rắn và chất lỏng bị nung nóng là quang phổ liên tục.
}

%========================================
\item \mkstar{1} [13]

\cauhoi
{Để thu được quang phổ vạch hấp thụ thì
	\begin{mcq}(1)
		\item nhiệt độ của đám khí hay hơi phải nhỏ hơn nhiệt độ của nguồn phát quang phổ liên tục. 
		\item Nhiệt độ của đám khí hay hơi phải lớn hơn nhiệt độ của nguồn phát quang phổ liên tục. 
		\item nhiệt độ của đám khí hay hơi hấp thụ phải lớn. 
		\item áp suất của đám khí hấp thụ phải lớn. 
	\end{mcq}
}

\loigiai
{		\textbf{Đáp án: A.}
	
	Để thu được quang phổ vạch hấp thụ thì nhiệt độ của đám khí hay hơi phải nhỏ hơn nhiệt độ của nguồn phát quang phổ liên tục. 
}
	
\end{enumerate}

\loigiai
{
	\begin{center}
		\textbf{BẢNG ĐÁP ÁN}
	\end{center}
	\begin{center}
		\begin{tabular}{|m{2.8em}|m{2.8em}|m{2.8em}|m{2.8em}|m{2.8em}|m{2.8em}|m{2.8em}|m{2.8em}|m{2.8em}|m{2.8em}|}
			\hline
			01.D  & 02.A  & 03.B  & 04.D  & 05.C  &  & & & &  \\
			\hline
			
		\end{tabular}
	\end{center}
}

\section{Lý thuyết: Tia hồng ngoại}
\begin{enumerate}[label=\bfseries Câu \arabic*:]
	
	%=======================================
	\item \mkstar{1} [2]
	
	\cauhoi
	{Khi nói về tia hồng ngoại, phát biểu nào dưới đây là \textbf{sai}?
		\begin{mcq}(1)
			\item Tia hồng ngoại có khả năng làm phát quang một số chất. 
			\item Tia hồng ngoại có khả năng gây ra một số phản ứng hóa học. 
			\item Tác dụng nổi bật nhất của tia hồng ngoại là tác dụng nhiệt. 
			\item Tia hồng ngoại cũng có thể biến điệu được như sóng điện từ cao tần. 
		\end{mcq}
	}
	
	\loigiai
	{		\textbf{Đáp án: A.}
		
		Tia hồng ngoại không có khả năng gây phát quang.
	}

%=======================================
\item \mkstar{1} [6]

\cauhoi
{Để đo thân nhiệt của một người mà không cần tiếp xúc trực tiếp, ta dùng máy đo thân nhiệt điện tử. Máy này tiếp nhận năng lượng bức xạ phát ra từ người cần đo. Nhiệt độ của người càng cao thì máy tiếp nhận năng lượng càng lớn. Bức xạ chủ yếu mà máy nhận được từ người thuộc miền 
	\begin{mcq}(4)
		\item tia Y. 
		\item hồng ngoại. 
		\item tia X. 
		\item tử ngoại. 
	\end{mcq}
}

\loigiai
{		\textbf{Đáp án: B.}
	
	Nhiệt là tính chất đặc trưng của bức xạ hồng ngoại nên bức xạ chủ yếu mà máy thu được từ người thuộc miền hồng ngoại.
}

%=======================================
\item \mkstar{1} [7]

\cauhoi
{Tính chất nổi bật của tia hồng ngoại là
	\begin{mcq}(2)
		\item tác dụng nhiệt.
		\item làm ion hóa không khí. 
		\item tác dụng lên kính ảnh. 
		\item khả năng đâm xuyên. 
	\end{mcq}
}

\loigiai
{		\textbf{Đáp án: A.}
	
	Tính chất nổi bật của tia hồng ngoại là tác dụng nhiệt. 
}

\end{enumerate}

\loigiai
{
	\begin{center}
		\textbf{BẢNG ĐÁP ÁN}
	\end{center}
	\begin{center}
		\begin{tabular}{|m{2.8em}|m{2.8em}|m{2.8em}|m{2.8em}|m{2.8em}|m{2.8em}|m{2.8em}|m{2.8em}|m{2.8em}|m{2.8em}|}
			\hline
			01.A & 02.B  & 03.A  &  &  &  &  & &  &  \\
			\hline
			
		\end{tabular}
	\end{center}
}

\section{Lý thuyết: Tia tử ngoại}
\begin{enumerate}[label=\bfseries Câu \arabic*:]
	
	%=======================================
	\item \mkstar{1} [7]
	
	\cauhoi
	{Tia tử ngoại có ứng dụng nào sau đây?
		\begin{mcq}(1)
			\item Phát hiện các vết nứt trên bề mặt sản phẩm kim loại. 
			\item Kiểm tra hành lý khách hàng đi máy bay. 
			\item Chụp ảnh bề mặt trái đất từ vệ tinh. 
			\item Chiếu điện, chụp điện. 
		\end{mcq}
	}
	
	\loigiai
	{		\textbf{Đáp án: A.}
		
		Tia tử ngoại có công dụng phát hiện các vết nứt trên bề mặt sản phẩm kim loại. 
	}

%=======================================
\item \mkstar{1} [1]

\cauhoi
{Nước hấp thụ được tia nào sau đây?
	\begin{mcq}(4)
		\item Tia tử ngoại. 
		\item Tia hồng ngoại. 
		\item Tia X. 
		\item Tia gamma. 
	\end{mcq}
}

\loigiai
{		\textbf{Đáp án: A.}
	
	Nước hấp thụ tốt tia tử ngoại.
}

%=======================================
\item \mkstar{1} [2]

\cauhoi
{Tia được dùng để khử khuẩn, tiệt trùng dụng cụ y tế
	\begin{mcq}(4)
		\item Tia tử ngoại. 
		\item Tia hồng ngoại. 
		\item Tia X. 
		\item Tia laser.
	\end{mcq}
}

\loigiai
{		\textbf{Đáp án: A.}
	
	Tia được dùng để khử khuẩn, tiệt trùng dụng cụ y tế là tia tử ngoại.
}

%=======================================
\item \mkstar{1} [31]

\cauhoi
{Để kiểm tra vết nứt trên bề mât sản phẩm kim loại, người ta dùng
	\begin{mcq}(2)
		\item tia tử ngoại. 
		\item tia Rơnghen. 
		\item tia hồng ngoại. 
		\item ánh sáng nhìn thấy. 
	\end{mcq}
}

\loigiai
{		\textbf{Đáp án: A.}
	
	Để kiểm tra vết nứt trên bề mât sản phẩm kim loại, người ta dùng tia tử ngoại.
}

%=======================================
\item \mkstar{1} [4]

\cauhoi
{Tầng ozon là tầng áo giáp bảo vệ cho con người và sinh vật trên mặt đất khỏi bị tác dụng hủy diệt của
	\begin{mcq}(1)
		\item tia tử ngoại trong ánh sáng mặt trời.
		\item tia hồng ngoại trong ánh sáng mặt trời. 
		\item tia đơn sắc màu đỏ trong ánh sáng mặt trời. 
		\item tia đơn sắc màu tím trong ánh sáng mặt trời. 
	\end{mcq}
}

\loigiai
{		\textbf{Đáp án: A.}
	
	Tầng ozon là tầng áo giáp bảo cho con người và sinh vật trên mặt đất khỏi bị tác dụng hủy diệt của tia tử ngoại trong ánh sáng mặt trời.
}

%=======================================
\item \mkstar{1} [4]

\cauhoi
{Phát biểu nào sau đây là \textbf{đúng}?
	\begin{mcq}(1)
		\item Tia hồng ngoại có tần số cao hơn tần số của tia sáng vàng. 
		\item Tia tử ngoại có bước sóng lớn hơn bước sóng của tia sáng đỏ. 
		\item Bức xạ tử ngoại có tần số cao hơn bức xạ hồng ngoại. 
		\item Bức xạ tử ngoại có chu kì lớn hơn chu kì của bức xạ tử ngoại. 
	\end{mcq}
}

\loigiai
{		\textbf{Đáp án: C.}
	
	Phát biểu \textbf{đúng} là bức xạ tử ngoại có tần số cao hơn bức xạ hồng ngoại. 
}

%=======================================
\item \mkstar{1} [10]

\cauhoi
{Tia tử ngoại được dùng
	\begin{mcq}(1)
		\item để tìm vết nứt trên bề mặt sản phẩm bằng kim loại. 
		\item trong y tế để chụp điện, chiếu điện. 
		\item để chụp ảnh bề mặt Trái Đất từ vệ tinh. 
		\item để tìm khuyết tật bên trong sản phẩm bằng kim loại. 
	\end{mcq}
}

\loigiai
{		\textbf{Đáp án: A.}
	
	Tia tử ngoại được dùng để tìm vết nứt trên bề mặt sản phẩm bằng kim loại.
}
	
\end{enumerate}

\loigiai
{
	\begin{center}
		\textbf{BẢNG ĐÁP ÁN}
	\end{center}
	\begin{center}
		\begin{tabular}{|m{2.8em}|m{2.8em}|m{2.8em}|m{2.8em}|m{2.8em}|m{2.8em}|m{2.8em}|m{2.8em}|m{2.8em}|m{2.8em}|}
			\hline
			01.A  & 02.A  & 03.A  & 04.A  & 05.A  & 06.C  & 07.A & & & \\
			\hline
			
		\end{tabular}
	\end{center}
}

\whiteBGstarEnd