%\setcounter{chapter}{1}
\chapter{Luyện tập: Hiện tượng quang - phát quang}
\begin{enumerate}
	\item Muốn một chất phát quang ra ánh sáng khả kiến có bước sóng $\lambda$ lúc được chiếu sáng thì
	\begin{mcq}
		\item phải kích thích bằng ánh sáng có bước sóng $\lambda$.
		\item phải kích thích bằng ánh sáng có bước sóng nhỏ hơn $\lambda$.
		\item phải kích thích bằng ánh sáng có bước sóng lớn hơn $\lambda$.
		\item phải kích thích bằng tia hồng ngoại.
	\end{mcq}
	
	\item Chọn câu trả lời\textbf{ sai} khi nói về sự phát quang?
	
	\begin{mcq}
		\item Sự huỳnh quang của chất khí, chất lỏng và sự lân quang của các chất rắn gọi là sự phát quang.
		\item Đèn huỳnh quang là việc áp dụng sự phát quang của các chất rắn.
		\item Sự phát quang còn được gọi là sự phát lạnh.
		\item Khi chất khí được kích thích bởi ánh sáng có tần số $f$, sẽ phát ra ánh sáng có tần số $f'$ với $f'>f$.
	\end{mcq}
	
	\item Phát biểu nào sau đây \textbf{sai} khi nói về hiện tượng huỳnh quang?
	\begin{mcq}
		\item Hiện tượng huỳnh quang là hiện tượng phát quang của các chất khí bị chiến ánh sáng kích thích.
		\item Khi tắt ánh sáng kích thích thì hiện tượng huỳnh quang còn kéo dài khoảng cách thời gian trước khi tắt.
		\item Phôtôn phát ra từ hiện tượng huỳnh quang bao giờ cũng nhỏ hơn năng lượng phôtôn của ánh sáng kích thích.
		\item Huỳnh quang còn được gọi là sự phát sáng lạnh.
	\end{mcq}
	
	\item Phát biểu nào sau đây sai khi nói về hiện tượng lân quang?
	\begin{mcq}
		\item Sự phát sáng của các tinh thể khi bị chiếu sáng thích hợp được gọi là hiện tượng lân quang.
		\item  Nguyên nhân chính của sự lân quang là do các tinh thể phản xạ ánh sáng chiếu vào nó.
		\item  Ánh sáng lân quang có thể tồn tại rất lâu sau khi tắt ánh sáng kích thích.
		\item Hiện tượng lân quang là hiện tượng phát quang của chất rắn.
	\end{mcq}
	
	\item Khi chiếu chùm tia tử ngoại vào một ống nghiệm đựng dung dịch fluorexêin thì thấy dung dịch này phát ra ánh sáng màu lục. Đó là hiện tượng
	\begin{mcq}
		\item phản xạ ánh sáng.
		\item quang – phát quang.	
		\item hóa – phát quang. 
		\item tán sắc ánh sáng.
	\end{mcq}
	
	\item Nếu ánh sáng kích thích là ánh sáng màu lam thì ánh sáng huỳnh quang không thể là ánh sáng nào dưới đây?
	\begin{mcq}
		\item Ánh sáng đỏ.
		\item Ánh sáng lục.
		\item Ánh sáng chàm.
		\item Ánh sáng lam.
	\end{mcq}
	
	\item Sự phát sáng của nguồn sáng nào dưới đây gọi là sự phát quang?
	\begin{mcq}
		\item  Ngọn nến.
		\item  Đèn pin.
		\item Con đom đóm.
		\item Ngôi sao băng.
	\end{mcq}
	
	\item Trong trường hợp nầo dưới đây có sự quang – phát quang?
	\begin{mcq}
		\item Ta nhìn thấy màu xanh của một biển quảng cáo lúc ban ngày.
		\item  Ta nhìn thấy ánh sáng lục phát ra từ đầu các cọc tiêu trên đường núi khi có ánh sáng đèn ô-tô chiếu vào.
		\item Ta nhìn thấy ánh sáng của một ngọn đèn đường.
		\item Ta nhìn thấy ánh sáng đỏ của một tấm kính đỏ.
	\end{mcq}
	
	\item  Chiếu bức xạ có bước sóng $\text{0,22}\ \mu\text{m}$ và một chất phát quang thì nó phát ra ánh sáng có bước sóng $\text{0,55}\ \mu\text{m}$.
	Nếu số photon ánh sáng kích thích chiếu vào là 500 thì số photon ánh sáng phát ra là 4. Tính tỉ số công suất của ánh sáng phát quang và ánh sáng kích thích?
	\begin{mcq}
		\item $\text{0,2}\%.$
		\item  $\text{0,3}\%.$
		\item $\text{0,32}\%.$
		\item $\text{2}\%.$
	\end{mcq}
	
	\item Sự phát sáng của vật nào dưới đây là sự phát quang?
	\begin{mcq}
		\item Tia lửa điện.
		\item Hồ quang.
		\item Bóng đèn ống.
		\item Bóng đèn pin.
	\end{mcq}
\end{enumerate}
\begin{center}
	\textbf{ĐÁP ÁN}
	
\end{center}

\begin{longtable}[\textwidth]{|p{0.1\textwidth}|p{0.1\textwidth}|p{0.1\textwidth}|p{0.1\textwidth}|p{0.1\textwidth}|p{0.1\textwidth}|p{0.1\textwidth}|p{0.1\textwidth}|}
	% --- first head
	\hline%\hspace{2 pt}
	\multicolumn{1}{|c|}{\textbf{Câu 1}} & \multicolumn{1}{c|}{\textbf{Câu 2}} & \multicolumn{1}{c|}{\textbf{Câu 3}} &
	\multicolumn{1}{c|}{\textbf{Câu 4}} &
	\multicolumn{1}{c|}{\textbf{Câu 5}} &
	\multicolumn{1}{c|}{\textbf{Câu 6}} &
	\multicolumn{1}{c|}{\textbf{Câu 7}} &
	\multicolumn{1}{c|}{\textbf{Câu 8}}\\
	\hline
	B. &D. &B. &B. &B. &C. &C. &B.\\
	\hline
	
	\multicolumn{1}{|c|}{\textbf{Câu 9}} & \multicolumn{1}{c|}{\textbf{Câu 10}} & \multicolumn{1}{c|}{\textbf{}} &
	\multicolumn{1}{c|}{\textbf{}} &
	\multicolumn{1}{c|}{\textbf{}} &
	\multicolumn{1}{c|}{\textbf{}} &
	\multicolumn{1}{c|}{\textbf{}} &
	\multicolumn{1}{c|}{} \\
	\hline
	C. &C. & & & & & &\\
	\hline		
\end{longtable}









