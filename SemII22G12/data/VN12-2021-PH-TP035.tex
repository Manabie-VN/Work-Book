\whiteBGstarBegin
\setcounter{section}{0}
\section{Hiện tượng quang điện. Thuyết lượng tử ánh sáng}
\begin{enumerate}[label=\bfseries Câu \arabic*:]
	\item \mkstar{3} 
	
	\cauhoi
	{Trong chân không, một bức xạ đơn sắc có bước sóng $\lambda =\text{0,6}\ \mu\text{m}$. Cho biết giá trị hằng số $h=\text{6,625}\cdot 10^{-34}\ \text{Js}$, $c=3\cdot 10^8\ \text{m/s}$ và $e=\text{1,6}\cdot 10^{-19}\ \text{C}$. Lượng tử năng lượng của ánh sáng này có giá trị
		\begin{mcq}(4)
			\item $\text{5,3}\ \text{eV}.$ 
			\item $\text{2,07}\ \text{eV}.$ 
			\item $\text{1,2}\ \text{eV}.$ 
			\item $\text{3,71}\ \text{eV}.$ 
		\end{mcq}
	}
	
	\loigiai
	{		\textbf{Đáp án: B.}
		
Lượng tử năng lượng của ánh sáng này có giá trị cho bởi:
$$
	\varepsilon = \dfrac{hc}{\lambda} = \SI{2,07}{eV}.
$$
		
	}
	
	\item \mkstar{3} 
	
	\cauhoi
	{Trong môi trường nước có chiết suất bằng $\dfrac{4}{3}$, một bức xạ đơn sắc có bước sóng bằng $\lambda =\text{0,6}\ \mu\text{m}$. Cho biết giá trị hằng số $h=\text{6,625}\cdot 10^{-34}\ \text{Js}$, $c=3\cdot 10^8\ \text{m/s}$ và $e=\text{1,6}\cdot 10^{-19}\ \text{C}$. Lượng tử năng lượng của ánh sáng này có giá trị
		\begin{mcq}(4)
			\item $\text{2,76}\ \text{eV}.$ 
			\item $\text{2,07}\ \text{eV}.$ 
			\item $\text{1,2}\ \text{eV}.$ 
			\item $\text{1,55}\ \text{eV}.$ 
		\end{mcq}
	}
	
	\loigiai
	{		\textbf{Đáp án: D.}
	
Bước sóng ánh sáng trong chân không có giá trị:
$$
	\lambda = n \lambda_{0} = \SI{0,8}{\mu m}.
$$
		
Lượng tử ánh sáng này có giá trị:
$$
	\varepsilon = \dfrac{hc}{\lambda_{0}} = \SI{1,55}{eV}.
$$
		
	}
	
	\item \mkstar{3} 
	
	\cauhoi
	{Công thoát êlectron của một kim loại là $A=\text{1,88}\ \text{eV}$. Cho biết giá trị hằng số $h=\text{6,625}\cdot 10^{-34}\ \text{Js}$, $c=3\cdot 10^8\ \text{m/s}$ và $e=\text{1,6}\cdot 10^{-19}\ \text{C}$. Giới hạn quang điện của kim loại này có giá trị là
		\begin{mcq}(4)
			\item $\text{550}\ \text{nm}.$
			\item $\text{220}\ \text{nm}.$ 
			\item $\text{1057}\ \text{nm}.$ 
			\item $\text{661}\ \text{nm}.$ 
		\end{mcq}
	}
	
	\loigiai
	{		\textbf{Đáp án: D.}
		
Giới hạn quang điện của kim loại này cho bởi:
$$
	\lambda_{0} = \dfrac{hc}{A} = \SI{661}{nm}.
$$
		
	}
	
	\item \mkstar{3} 
	
	\cauhoi
	{Giới hạn quang điện của một kim loại là $\lambda =\text{0,278}\ \mu\text{m}$. Cho biết giá trị hằng số $h=\text{6,625}\cdot 10^{-34}\ \text{Js}$, $c=3\cdot 10^8\ \text{m/s}$ và $e=\text{1,6}\cdot 10^{-19}\ \text{C}$. Công thoát electron của kim loại này có giá trị là
		\begin{mcq}(4)
			\item $\text{4,47}\ \text{eV}.$ 
			\item $\text{3,54}\ \text{eV}.$ 
			\item $\text{2,73}\ \text{eV}.$ 
			\item $\text{3,09}\ \text{eV}.$ 
		\end{mcq}
	}
	
	\loigiai
	{		\textbf{Đáp án: A.}
		
Công thoát cho bởi:
$$
	A = \dfrac{hc}{\lambda_{0}} = \SI{4,47}{eV}.
$$
		
	}
	
	\item \mkstar{3}
	
	\cauhoi
	{Biết công thoát $A$ của $\ce{Ca};\ \ce{K}$; $\ \ce{Ag};\ \ce{Cu}$ lần lượt là $\text{2,89}\ \text{eV};\ \text{2,26}\ \text{eV};$ $\ \text{4,78} eV;\ \text{4,14}\ \text{eV}$. Chiếu ánh sáng có bước sóng $\text{0,33}\ \mu\text{m}$ vào bề mặt các kim loại trên. Hiện tượng quang điện không xảy ra với các kim loại nào sau đây?
		\begin{mcq}(4)
			\item $\ce{Ag};\ \ce{Cu}.$ 
			\item $\ce{K};\ \ce{Cu}.$ 
			\item $\ce{Ca};\ \ce{Ag}.$ 
			\item $\ce{K};\ \ce{Ca}.$
		\end{mcq}
	}
	
	\loigiai
	{		\textbf{Đáp án: D.}
		
Hiện tượng quang điện không xảy ra khi:
$$
	\varepsilon \leq A.
$$
Trong đó, lượng tử năng lượng cho bởi:
$$
	\varepsilon = \dfrac{hc}{\lambda} = \SI{3,763}{eV}.
$$
Vậy hiện tượng quang điện không xảy ra với kim loại $ \ce{Ca} $ và $ \ce{K} $.
	}
	
\item \mkstar{3} 
	
	\cauhoi
	{Một bản kim loại có công thoát electron bằng $\text{4,47}\ \text{eV}$. Chiếu ánh sáng kích thích có bước sóng bằng $\text{0,14}\ \mu\text{m}$(trong chân không). Cho biết giá trị hằng số $h=\text{6,625}\cdot 10^{-34}\ \text{Js}$, $c=3\cdot 10^8\ \text{m/s}$ và $e=\text{1,6}\cdot 10^{-19}\ \text{C}, \ m_\text{e}=\text{9,1}\cdot 10^{-31}\ \text{kg}$. Động năng ban đầu cực đại và vận tốc ban đầu của electron quang điện lần lượt là
		\begin{mcq}(2)
			\item $\text{7,04}\cdot 10^{-19}\ \text{J};\ \text{1,24}\cdot 10^{6}\ \text{m/s}.$
			\item $\text{3,25}\ \text{eV};\ \text{2,43}\cdot 10^{6}\ \text{m/s}.$ 
			\item $\text{5,37}\cdot 10^{-19}\ \text{J};\ \text{2,43}\cdot 10^{6}\ \text{m/s}.$ 
			\item $\text{4,40}\ \text{eV};\ \text{1,24}\cdot 10^{6}\ \text{m/s}.$ 
		\end{mcq}
	}
	
	\loigiai
	{		\textbf{Đáp án: D.}

Năng lượng của photon cho bởi:
$$
	\varepsilon = \dfrac{hc}{\lambda} = \SI{8.87}{eV}.
$$
Áp dụng hệ thức Einstein ta có:
$$
	\varepsilon = A + K \rightarrow K = \SI{4,4}{eV}.
$$
Vận tốc ban đầu của electron cho bởi:
$$
	K = \dfrac{1}{2}mv^{2} \rightarrow v = \SI{1,24 e6}{m/s}.
$$
	}
	
	\item \mkstar{3} 
	
	\cauhoi
	{Chiếu đồng thời hai bức xạ có bước sóng $\text{0,452}\ \mu \text{m}$   và $\text{0,243}\ \mu\text{m}$ vào một tấm kim loại có giới hạn quang điện là $\text{0,5}\ \mu\text{m}$. Cho biết giá trị hằng số $h=\text{6,625}\cdot 10^{-34}\ \text{Js}$, $c=3\cdot 10^8\ \text{m/s}$ và $e=\text{1,6}\cdot 10^{-19}\ \text{C}, \ m_\text{e}=\text{9,1}\cdot 10^{-31}\ \text{kg}$. Vận tốc ban đầu cực đại của các electron quang điện bằng
		\begin{mcq}(4)
			\item $\text{9,61}\cdot 10^{5}\ \text{m/s}.$ 
			\item $\text{1,34}\cdot 10^{6}\ \text{m/s}.$ 
			\item $\text{2,29}\cdot 10^{4}\ \text{m/s}.$ 
			\item $\text{9,24}\cdot 10^{3}\ \text{m/s}.$ 
		\end{mcq}
	}
	
	\loigiai
	{		\textbf{Đáp án: A.}
		
Ta thấy $ \lambda_{1} = \SI{0,452}{\mu m} < \lambda_{2} = \SI{0,243}{\mu m}$. Vậy nên lượng tử năng lượng có giá trị lớn nhất cho bởi:
$$
	\varepsilon_{2} = \dfrac{hc}{\lambda_{2}} = \SI{5,11}{eV}.
$$
Giới hạn quang điện của tấm kim loại cho bởi:
$$
	A = \dfrac{hc}{\lambda_{0}} = \SI{2,484}{eV}.
$$
Áp dụng hệ thức Einstein ta có:
$$
	\varepsilon = A + K \rightarrow K = \SI{2,626}{eV}.
$$
Vận tốc ban đầu của electron cho bởi:
$$
	K = \dfrac{1}{2}mv^{2} \rightarrow v = \SI{9,61 e5}{m/s}.
$$		
	}
	
	\item \mkstar{3} 
	
	\cauhoi
	{Chiếu lần lượt hai bức xạ điện từ có bước sóng $\lambda_1$ và $\lambda_2$ với $\lambda_2=2\lambda_1$  vào một tấm kim loại thì tỉ số động năng ban đầu cực đại của quang electron bứt ra khỏi kim loại là 9. Giới hạn quang điện của kim loại là $\lambda_0$. Tỉ số $\dfrac{\lambda_0}{\lambda_1}$ bằng
		\begin{mcq}(4)
			\item $\dfrac{16}{9}.$ 
			\item 2. 
			\item $\dfrac{16}{7}.$ 
			\item $\dfrac{8}{17}.$ 
		\end{mcq}
	}
	
	\loigiai
	{		\textbf{Đáp án: D.}

Vì $ \lambda_{2} = 2 \lambda_{1} $ nên suy ra $ \varepsilon_{1} = 2 \varepsilon_{2} $. \\
Mặt khác tỉ lệ giữa động năng ban đầu cực đại bằng $ 9 $ tương đương $ K_{1} = 9 K_{2}$. \\
Áp dụng hệ thức Einstein cho bức xạ $ \lambda_{1} $:
$$
	\varepsilon_{1} = A + K_{1} \rightarrow \varepsilon_{1} = A + 9 K_{2}.
$$
Áp dụng hệ thức Einstein cho bức xạ $ \lambda_{2} $:
$$
	\varepsilon_{2} = A + K_{2} \rightarrow 18\varepsilon_{1} = 9A + 9 K_{2}.
$$
Từ hai phương trình trên ta có:
$$
17 \varepsilon_{1} = 8A \rightarrow \dfrac{\varepsilon_{1}}{A} = \dfrac{\lambda_{0}}{\lambda_{1}} = \dfrac{8}{17}.
$$		
	}
	
	\item \mkstar{3} 
	
	\cauhoi
	{Một ngọn đèn phát ra ánh sáng đơn sắc có $\lambda=\text{0,6}\ \mu\text{m}$ sẽ phát ra bao nhiêu photon trong 10 s nếu công suất đèn là $P = 10\ \text{W}$.
		\begin{mcq}(4)
			\item $\text{3,0189}\cdot 10^{20}.$ 
			\item $\text{6}\cdot 10^{20}.$ 
			\item $\text{3,0189}\cdot 10^{19}.$ 
			\item $\text{6,04}\cdot 10^{16}.$ 
		\end{mcq}
	}
	
	\loigiai
	{		\textbf{Đáp án: C.}

Năng lượng của một photon là
$$
	\varepsilon = \dfrac{hc}{\lambda} = \SI{3,3125 e-19}{J}.
$$		
Số photon phát ra là
$$
	N = \dfrac{P}{\varepsilon} = \num{3,0189 e19}.
$$
		
	}
	
	\item \mkstar{3} 
	
	\cauhoi
	{Chiếu một chùm bức xạ vào tế bào quang điện có catot làm bằng $\ce{Na}$ thì cường độ dòng quang điện bão hòa là $3\ \mu\text{A}$ . Số electron bị bứt ra khỏi catot trong hai phút là bao nhiêu?
		\begin{mcq}(4)
			\item $\text{3,25}\cdot 10^{15}.$ 
			\item $\text{2,35}\cdot 10^{14}.$
			\item $\text{2,25}\cdot 10^{15}.$ 
			\item $\text{4,45}\cdot 10^{15}.$ 
		\end{mcq}
	}
	
	\loigiai
	{		\textbf{Đáp án: C.}
		
Số electron phát ra cho bởi:
$$
	N = \dfrac{q}{e} = \dfrac{It}{e} = \num{2,25 e15}.
$$
		
	}
	
\item \mkstar{1} 
	
	\cauhoi
	{Hiện tượng bứt electron ra khỏi kim loại, khi chiếu ánh sáng kích thích có bước sóng thích hợp lên kim loại được gọi là
		\begin{mcq}(2)
			\item hiện tượng bức xạ. 
			\item hiện tượng phóng xạ. 
			\item hiện tượng quang dẫn. 
			\item hiện tượng quang điện. 
		\end{mcq}
	}
	
	\loigiai
	{		\textbf{Đáp án: D.}
		
Hiện tượng bứt electron ra khỏi kim loại, khi chiếu ánh sáng kích thích có bước sóng thích hợp lên kim loại được gọi là hiện tượng quang điện. 
		
	}
	
	\item \mkstar{1} 
	
	\cauhoi
	{Hiện tượng quang điện là hiện tượng electron bứt ra khỏi bề mặt của tấm kim loại khi 
		\begin{mcq}(1)
			\item có ánh sáng thích hợp chiếu vào nó. 
			\item tấm kim loại bị nung nóng. 
			\item tấm kim loại bị nhiễm điện do tiếp xúc với vật nhiễm điện khác. 
			\item tấm kim loại được đặt trong điện trường đều. 
		\end{mcq}
	}
	
	\loigiai
	{		\textbf{Đáp án: A.}
		
Hiện tượng quang điện là hiện tượng electron bứt ra khỏi bề mặt của tấm kim loại khi có ánh sáng thích hợp chiếu vào nó.  
		
	}
	
	\item \mkstar{1} 
	
	\cauhoi
	{Nếu chiếu một chùm tia hồng ngoại vào tấm kẽm tích điện âm, thì
		\begin{mcq}(1)
			\item tấm kẽm mất dần điện tích dương. 
			\item tấm kẽm mất dần điện tích âm.	
			\item tấm kẽm trở nên trung hoà về điện.
			\item điện tích âm của tấm kẽm không đổi.
		\end{mcq}
	}
	
	\loigiai
	{		\textbf{Đáp án: D.}
		
Nếu chiếu một chùm tia hồng ngoại vào tấm kẽm tích điện âm, thì điện tích âm của tấm kẽm không đổi. Do tia hồng ngoại không gây ra hiện tượng quang điện trên tấm kẽm.
		
	}
	
	\item \mkstar{1} 
	
	\cauhoi
	{Giới hạn quang điện của mỗi kim loại là
		\begin{mcq}(1)
			\item bước sóng của ánh sáng kích thích chiếu vào kim loại. 
			\item công thoát của các electron ở bề mặt kim loại đó. 
			\item bước sóng giới hạn của ánh sáng kích thích để gây ra hiện tượng quang điện kim loại đó. 
			\item hiệu điện thế hãm. 
		\end{mcq}
	}
	
	\loigiai
	{		\textbf{Đáp án: C.}
		
Giới hạn quang điện của mỗi kim loại là bước sóng giới hạn của ánh sáng kích thích để gây ra hiện tượng quang điện kim loại đó. 
		
	}
	
	\item \mkstar{1} 
	
	\cauhoi
	{Giới hạn quang điện của mỗi kim loại là
		\begin{mcq}(1)
			\item bước sóng dài nhất của bức xạ chiếu vào kim loại đó để gây ra hiện tượng quang điện. 
			\item bước sóng ngắn nhất của bức xạ chiếu vào kim loại đó để gây ra hiện tượng quang điện.	
			\item công nhỏ nhất dùng để bứt electron ra khỏi kim loại đó.	
			\item công lớn nhất dùng để bứt electron ra khỏi kim loại đó.
		\end{mcq}
	}
	
	\loigiai
	{		\textbf{Đáp án: A.}
		
Giới hạn quang điện của mỗi kim loại là  bước sóng dài nhất của bức xạ chiếu vào kim loại đó để gây ra hiện tượng quang điện. 
		
	}
	
\item \mkstar{1} 
	
	\cauhoi
	{Giới hạn quang điện tuỳ thuộc vào
		\begin{mcq}(1)
			\item bản chất của kim loại. 
			\item điện áp giữa anôt và catôt của tế bào quang điện.
			\item bước sóng của ánh sáng chiếu vào catôt. 
			\item điện trường giữa anôt và catôt. 
		\end{mcq}
	}
	
	\loigiai
	{		\textbf{Đáp án: A.}
		
Giới hạn quang điện tuỳ thuộc vào bản chất của kim loại. 
		
	}
	
	\item \mkstar{1} 
	
	\cauhoi
	{Để gây được hiệu ứng quang điện, bức xạ dọi vào kim loại được thoả mãn điều kiện là 
		\begin{mcq}(1)
			\item tần số lớn hơn giới hạn quang điện. 
			\item tần số nhỏ hơn giới hạn quang điện. 
			\item bước sóng nhỏ hơn giới hạn quang điện. 
			\item bước sóng lớn hơn giới hạn quang điện.
		\end{mcq}
	}
	
	\loigiai
	{		\textbf{Đáp án: C.}
		
Để gây được hiệu ứng quang điện, bức xạ dọi vào kim loại được thoả mãn điều kiện là bước sóng nhỏ hơn giới hạn quang điện. 
		
	}
	
	\item \mkstar{1} 
	
	\cauhoi
	{Khi chiếu sóng điện từ xuống bề mặt tấm kim loại, hiện tượng quang điện xảy ra nếu
		\begin{mcq}(1)
			\item sóng điện từ có nhiệt độ đủ cao. 
			\item sóng điện từ có bước sóng thích hợp.
			\item sóng điện từ có cường độ đủ lớn.
			\item sóng điện từ phải là ánh sáng nhìn thấy được. 
		\end{mcq}
	}
	
	\loigiai
	{		\textbf{Đáp án: B.}
		
Khi chiếu sóng điện từ xuống bề mặt tấm kim loại, hiện tượng quang điện xảy ra nếu sóng điện từ có bước sóng thích hợp.
		
	}
	
	\item \mkstar{1} 
	
	\cauhoi
	{Trong trường hợp nào dưới đây có thể xảy ra hiện tượng quang điện? Ánh sáng Mặt Trời chiếu vào
		\begin{mcq}(2)
			\item mặt nước biển. 
			\item lá cây. 
			\item mái ngói.
			\item tấm kim loại không sơn. 
		\end{mcq}
	}
	
	\loigiai
	{		\textbf{Đáp án: D.}
		
Khi ánh sáng mặt trời chiếu vào tấm kim loại không sơn thì có thể xảy ra hiện tượng quang điện.
		
	}
	
	\item \mkstar{1} 
	
	\cauhoi
	{Lần lượt chiếu hai bức xạ có bước sóng $\lambda_1=\text{0,75}\ \mu \text{m}$ và $\lambda_2=\text{0,25}\ \mu \text{m}$ vào một tấm kẽm có giới hạn quang điện $\lambda_0=\text{0,35}\ \mu \text{m}$. Bức xạ nào gây ra hiện tượng quang điện?
		\begin{mcq}(1)
			\item Cả hai bức xạ.
			\item Chỉ có bức xạ $\lambda_1$.
			\item Chỉ có bức xạ $\lambda_2$. 
			\item Không có bức xạ nào trong hai bức xạ đó. 
		\end{mcq}
	}
	
	\loigiai
	{		\textbf{Đáp án: C.}
		
Chỉ có bức xạ $ \lambda_{2} < \lambda_{0}  $ có thể gây ra hiện tượng quang điện.
		
	}
	
\item \mkstar{1} 
	
	\cauhoi
	{Electron quang điện bị bứt ra khỏi bề mặt kim loại khi bị chiếu sáng nếu
		\begin{mcq}(1)
			\item cường độ của chùm sáng rất lớn. 
			\item bước sóng của ánh sáng lớn. 
			\item tần số ánh sáng nhỏ. 
			\item bước sóng nhỏ hơn hay bằng một giới hạn xác định. 
		\end{mcq}
	}
	
	\loigiai
	{		\textbf{Đáp án: D.}
		
Electron quang điện bị bứt ra khỏi bề mặt kim loại khi bị chiếu sáng nếu bước sóng nhỏ hơn hay bằng một giới hạn xác định. 
		
	}
	
\end{enumerate}

\loigiai
{
	\begin{center}
		\textbf{BẢNG ĐÁP ÁN}
	\end{center}
	\begin{center}
		\begin{tabular}{|m{2.8em}|m{2.8em}|m{2.8em}|m{2.8em}|m{2.8em}|m{2.8em}|m{2.8em}|m{2.8em}|m{2.8em}|m{2.8em}|}
			\hline
			01.B  & 02.D  & 03.D  & 04.A  & 05.D  & 06.D  & 07.A & 08.D & 09.C & 10.C \\
			\hline
			11.D  & 12.A  & 13.D  & 14.C  & 15.A  & 16.A  & 17.C & 18.B & 19.D & 20.C \\
			\hline
			21.D  &  &  &  &  &  & &  &  &  \\
			\hline
			
		\end{tabular}
	\end{center}
}

\section{Hiện tượng quang điện trong}
\begin{enumerate}[label=\bfseries Câu \arabic*:]
	\item \mkstar{1} 
	
	\cauhoi
	{Phát biểu nào sau đây là đúng?
		\begin{mcq}(1)
			\item Hiện tượng quang điện trong là hiện tượng bứt êlectron ra khỏi bề mặt kim loại khi chiếu vào kim loại ánh sáng có bước sóng thích hợp. 
			\item Hiện tượng quang điện trong là hiện tượng êlectron bị bắn ra khỏi kim loại khi kim loại bị đốt nóng. 
			\item Hiện tượng quang điện trong là hiện tượng êlectron liên kết được giải phóng thành êlectron dẫn khi chất bán dẫn được chiếu bằng bức xạ thích hợp. 
			\item Hiện tượng quang điện trong là hiện tượng điện trở của chất bán dẫn tăng lên khi chiếu ánh sáng thích hợp vào chất bán dẫn. 
		\end{mcq}
	}
	
	\loigiai
	{		\textbf{Đáp án: C.}
		
Hiện tượng quang điện trong là hiện tượng êlectron liên kết được giải phóng thành êlectron dẫn khi chất bán dẫn được chiếu bằng bức xạ thích hợp. 
		
	}
	
	\item \mkstar{1} 
	
	\cauhoi
	{Khi chiếu vào chất CdS ánh sáng đơn sắc có bước sóng nhỏ hơn giới hạn quang điện trong của chất này thì điện trở của nó
		\begin{mcq}(2)
			\item không thay đổi. 
			\item luôn tăng. 
			\item giảm đi. 
			\item lúc tăng, lúc giảm. 
		\end{mcq}
	}
	
	\loigiai
	{		\textbf{Đáp án: C.}
		
Khi chiếu vào bức xạ có bước sóng nhỏ hơn giới hạn quang điện thì xảy ra hiện tượng quang dẫn làm điện trở của chất bán dẫn giảm đi.
	}
	
	\item \mkstar{1} 
	
	\cauhoi
	{Chọn câu trả lời đúng?
		\begin{mcq}(1)
			\item Quang dẫn là hiện tuợng dẫn điện của chất bán dẫn lúc được chiếu sáng.
			\item Quang dẫn là hiện tượng kim loại phát xạ êlectron lúc được chiếu sáng. 
			\item Quang dẫn là hiện tượng điện trở của một chất giảm rất nhiều khi hạ nhiệt độ xuống rất thấp. 
			\item Quang dẫn là hiện tượng bứt quang êlectron ra khỏi bề mặt chất bán dẫn. 
		\end{mcq}
	}
	
	\loigiai
	{		\textbf{Đáp án: A.}
		
Quang dẫn là hiện tuợng dẫn điện của chất bán dẫn lúc được chiếu sáng.
		
	}
	
	\item \mkstar{1} 
	
	\cauhoi
	{Khi chiếu sóng điện từ vào chất bán dẫn, hiện tượng quang điện trong xảy ra nếu
		\begin{mcq}(1)
			\item sóng điện từ có nhiệt độ cao.
			\item sóng điện từ có bước sóng thích hợp. 
			\item sóng điện từ có cường độ đủ lớn. 
			\item sóng điện từ phải là ánh sáng nhìn thấy được. 
		\end{mcq}
	}
	
	\loigiai
	{		\textbf{Đáp án: B.}
		
Khi chiếu sóng điện từ vào chất bán dẫn, hiện tượng quang điện trong xảy ra nếu sóng điện từ có bước sóng thích hợp. 
		
	}
	
	\item \mkstar{1} 
	
	\cauhoi
	{Điểm giống nhau giữa hiện tượng quang điện trong và hiện tượng quang điện ngoài là
		\begin{mcq}(1)
			\item cùng được ứng dụng để chế tạo pin quang điện. 
			\item khi hấp thu phôtôn có năng lượng thích hợp thì êlectron sẽ bứt ra khỏi bề mặt của khối chất. 
			\item chỉ xảy ra khi êlectron hấp thu một phôtôn có năng lượng đủ lớn. 
			\item chỉ xảy ra khi tần số của ánh sáng kích thích nhỏ hơn một giá trị nhất định.
		\end{mcq}
	}
	
	\loigiai
	{		\textbf{Đáp án: C.}
		
Điểm giống nhau giữa hiện tượng quang điện trong và hiện tượng quang điện ngoài là chỉ xảy ra khi êlectron hấp thu một phôtôn có năng lượng đủ lớn. 
		
	}
	
\item \mkstar{1} 
	
	\cauhoi
	{Chọn phát biểu đúng về hiện tượng quang điện trong?
		\begin{mcq}(1)
			\item Có bước sóng giới hạn nhỏ hơn bước sóng giới hạn của hiện tượng quang điện ngoài.
			\item Ánh sáng kích thích phải là ánh sáng tử ngoại.
			\item Có thể xảy ra khi được chiếu bằng bức xạ hồng ngoại. 
			\item Có thể xảy ra đối với cả kim loại. 
		\end{mcq}
	}
	
	\loigiai
	{		\textbf{Đáp án: C.}
		
Hiện tượng quang điện trong có thể xảy ra được với bức xạ hồng ngoại.
		
	}
	
	\item \mkstar{1} 
	
	\cauhoi
	{Pin quang điện
		\begin{mcq}(1)
			\item là dụng cụ biến đổi trực tiếp quang năng thành điện năng. 
			\item là dụng cụ biến nhiệt năng thành điện năng. 
			\item hoạt động dựa vào hiện tượng quang điện ngoài.
			\item là dụng cụ có điện trở tăng khi được chiếu sáng. 
		\end{mcq}
	}
	
	\loigiai
	{		\textbf{Đáp án: A.}
		
Pin quang điện là dụng cụ biến đổi trực tiếp quang năng thành điện năng. 
		
	}
	
	\item \mkstar{1}
	
	\cauhoi
	{Chọn phát biểu đúng:
		\begin{mcq}(1)
			\item Trong pin quang điện, năng lượng Mặt Trời được biến đổi toàn bộ thành điện năng. 
			\item Suất điện động của một pin quang điện chỉ xuất hiện khi pin được chiếu sáng. 
			\item Theo định nghĩa, hiện tượng quang điện trong là nguyên nhân sinh ra hiện tượng quang dẫn. 
			\item Bước sóng ánh sáng chiếu vào khối bán dẫn càng lớn thì điện trở của khối này càng nhỏ. 
		\end{mcq}
	}
	
	\loigiai
	{		\textbf{Đáp án: C.}
		
Theo định nghĩa, hiện tượng quang điện trong là nguyên nhân sinh ra hiện tượng quang dẫn. 
		
	}
	
	\item \mkstar{1} 
	
	\cauhoi
	{Chọn ý \textbf{sai}. 
	
	Pin quang điện
		\begin{mcq}(1)
			\item là pin chạy bằng năng lượng ánh sáng. 
			\item biến đổi trực tiếp quang năng thành điện năng. 
			\item hoạt động dựa trên quang điện trong. 
			\item có hiệu suất cao (khoảng trên $50\%$). 
		\end{mcq}
	}
	
	\loigiai
	{		\textbf{Đáp án: D.}
		
Pin quang điện thực tế có hiệu suất rất thấp. 
		
	}
	
	\item \mkstar{1} 
	
	\cauhoi
	{Trong hiện tượng quang điện trong: Năng lượng cần thiết để giải phóng một electron liên kết thành electron tự do là $\varepsilon$ thì bước sóng dài nhất của ánh kích thích gây ra được hiện tượng quang điện trong bằng
		\begin{mcq}(4)
			\item $\dfrac{hc}{\varepsilon}$. 
			\item $\dfrac{h\varepsilon}{c}$. 
			\item $\dfrac{\varepsilon}{hc}$. 
			\item $\dfrac{c}{h\varepsilon}$. 
		\end{mcq}
	}
	
	\loigiai
	{		\textbf{Đáp án: A.}
		
Bước sóng dài nhất để giải phóng một electron liên kết thành một electron tự do cho bởi:
$$
	\lambda = \dfrac{hc}{\varepsilon}.
$$
		
	}
	
\end{enumerate}

\loigiai
{
	\begin{center}
		\textbf{BẢNG ĐÁP ÁN}
	\end{center}
	\begin{center}
		\begin{tabular}{|m{2.8em}|m{2.8em}|m{2.8em}|m{2.8em}|m{2.8em}|m{2.8em}|m{2.8em}|m{2.8em}|m{2.8em}|m{2.8em}|}
			\hline
			01.C  & 02.C  & 03.A  & 04.B  & 05.C  & 06.C  & 07.A & 08.C & 09.D & 10.A \\
			\hline
			
		\end{tabular}
	\end{center}
}

\section{Hiện tượng quang - phát quang}
\begin{enumerate}[label=\bfseries Câu \arabic*:]
	\item \mkstar{1} 
	
	\cauhoi
	{Muốn một chất phát quang ra ánh sáng khả kiến có bước sóng $\lambda$ lúc được chiếu sáng thì
		\begin{mcq}(1)
			\item phải kích thích bằng ánh sáng có bước sóng $\lambda$. 
			\item phải kích thích bằng ánh sáng có bước sóng nhỏ hơn $\lambda$.
			\item phải kích thích bằng ánh sáng có bước sóng lớn hơn $\lambda$.
			\item phải kích thích bằng tia hồng ngoại. 
		\end{mcq}
	}
	
	\loigiai
	{		\textbf{Đáp án: B.}
		
Muốn một chất phát quang ra ánh sáng khả kiến có bước sóng $\lambda$ lúc được chiếu sáng thì phải kích thích bằng ánh sáng có bước sóng nhỏ hơn $\lambda$.

	}
	
	\item \mkstar{1} 
	
	\cauhoi
	{Chọn câu trả lời\textbf{ sai} khi nói về sự phát quang?
		\begin{mcq}(1)
			\item Sự huỳnh quang của chất khí, chất lỏng và sự lân quang của các chất rắn gọi là sự phát quang. 
			\item Đèn huỳnh quang là việc áp dụng sự phát quang của các chất rắn. 
			\item Sự phát quang còn được gọi là sự phát xạ lạnh.
			\item Khi chất khí được kích thích bởi ánh sáng có tần số $f$, sẽ phát ra ánh sáng có tần số $f'$ với $f'>f$.
		\end{mcq}
	}
	
	\loigiai
	{		\textbf{Đáp án: D.}
		
Khi chất khí được kích thích bởi ánh sáng có tần số $f$, sẽ phát ra ánh sáng có tần số $f'$ với $f'<f$.
		
	}
	
	\item \mkstar{1}
	
	\cauhoi
	{Phát biểu nào sau đây \textbf{sai} khi nói về hiện tượng huỳnh quang?
		\begin{mcq}(1)
			\item Hiện tượng huỳnh quang là hiện tượng phát quang của các chất khí bị chiến ánh sáng kích thích. 
			\item Khi tắt ánh sáng kích thích thì hiện tượng huỳnh quang còn kéo dài khoảng cách thời gian trước khi tắt. 
			\item Phôtôn phát ra từ hiện tượng huỳnh quang bao giờ cũng nhỏ hơn năng lượng phôtôn của ánh sáng kích thích. 
			\item Huỳnh quang còn được gọi là sự phát sáng lạnh.
		\end{mcq}
	}
	
	\loigiai
	{		\textbf{Đáp án: B.}
		
Khi tắt ánh sáng kích thích thì hiện tượng huỳnh quang còn kéo dài khoảng cách thời gian trước khi tắt là sai.
		
	}
	
	\item \mkstar{1} 
	
	\cauhoi
	{Phát biểu nào sau đây sai khi nói về hiện tượng lân quang?
		\begin{mcq}(1)
			\item Sự phát sáng của các tinh thể khi bị chiếu sáng thích hợp được gọi là hiện tượng lân quang.
			\item Nguyên nhân chính của sự lân quang là do các tinh thể phản xạ ánh sáng chiếu vào nó. 
			\item Ánh sáng lân quang có thể tồn tại rất lâu sau khi tắt ánh sáng kích thích. 
			\item Hiện tượng lân quang là hiện tượng phát quang của chất rắn. 
		\end{mcq}
	}
	
	\loigiai
	{		\textbf{Đáp án: B.}
		
Nguyên nhân chính của sự lân quang là do các tinh thể phản xạ ánh sáng chiếu vào nó là sai. Cơ chế ở đây là phát xạ chứ không phải phản xạ.
		
	}
	
	\item \mkstar{1} 
	
	\cauhoi
	{Khi chiếu chùm tia tử ngoại vào một ống nghiệm đựng dung dịch fluorexêin thì thấy dung dịch này phát ra ánh sáng màu lục. Đó là hiện tượng
		\begin{mcq}(2)
			\item phản xạ ánh sáng. 
			\item quang – phát quang.
			\item hóa – phát quang. 
			\item tán sắc ánh sáng. 
		\end{mcq}
	}
	
	\loigiai
	{		\textbf{Đáp án: B.}
		
Khi chiếu chùm tia tử ngoại vào một ống nghiệm đựng dung dịch fluorexêin thì thấy dung dịch này phát ra ánh sáng màu lục. Đó là hiện tượng quang - quang phát quang.
		
	}
	
\item \mkstar{1} 
	
	\cauhoi
	{Nếu ánh sáng kích thích là ánh sáng màu lam thì ánh sáng huỳnh quang không thể là ánh sáng nào dưới đây?
		\begin{mcq}(2)
			\item Ánh sáng đỏ. 
			\item Ánh sáng lục. 
			\item Ánh sáng chàm. 
			\item Ánh sáng lam. 
		\end{mcq}
	}
	
	\loigiai
	{		\textbf{Đáp án: C.}
		
Nếu ánh sáng kích thích là ánh sáng màu lam thì ánh sáng huỳnh quang không thể là ánh sáng màu chàm.
		
	}
	
	\item \mkstar{1} 
	
	\cauhoi
	{Trong trường hợp nào dưới đây có sự quang – phát quang?
		\begin{mcq}(1)
			\item Ta nhìn thấy màu xanh của một biển quảng cáo lúc ban ngày. 
			\item Ta nhìn thấy ánh sáng lục phát ra từ đầu các cọc tiêu trên đường núi khi có ánh sáng đèn ô-tô chiếu vào. 
			\item Ta nhìn thấy ánh sáng của một ngọn đèn đường.
			\item Ta nhìn thấy ánh sáng đỏ của một tấm kính đỏ.
		\end{mcq}
	}
	
	\loigiai
	{		\textbf{Đáp án: B.}
		
Ta nhìn thấy ánh sáng lục phát ra từ đầu các cọc tiêu trên đường núi khi có ánh sáng đèn ô-tô chiếu vào có sự quang - phát quang. 
		
	}
	
	\item \mkstar{1} 
	
	\cauhoi
	{Chiếu bức xạ có bước sóng $\text{0,22}\ \mu\text{m}$ và một chất phát quang thì nó phát ra ánh sáng có bước sóng $\text{0,55}\ \mu\text{m}$.
	Nếu số photon ánh sáng kích thích chiếu vào là 500 thì số photon ánh sáng phát ra là 4. Tính tỉ số công suất của ánh sáng phát quang và ánh sáng kích thích?
		\begin{mcq}(4)
			\item $\text{0,2}\%.$
			\item $\text{0,3}\%.$ 
			\item $\text{0,32}\%.$ 
			\item $\text{2}\%.$ 
		\end{mcq}
	}
	
	\loigiai
	{		\textbf{Đáp án: C.}
		
Tỉ số công suất ánh sáng phát quang và ánh sáng kích thích cho bởi:
$$
	\dfrac{P'}{P} = \dfrac{N'\varepsilon'}{N\varepsilon} = \dfrac{N' \lambda}{N \lambda'} = \num{0,32} \%.
$$
		
	}
	
	\item \mkstar{1} 
	
	\cauhoi
	{Sự phát sáng của vật nào dưới đây là sự phát quang?
		\begin{mcq}(2)
			\item Tia lửa điện. 
			\item Hồ quang. 
			\item Bóng đèn ống. 
			\item Bóng đèn pin. 
		\end{mcq}
	}
	
	\loigiai
	{		\textbf{Đáp án: C.}
		
Sự phát sáng của một bóng đèn ống là sự phát quang.
		
	}
	

	
\end{enumerate}

\loigiai
{
	\begin{center}
		\textbf{BẢNG ĐÁP ÁN}
	\end{center}
	\begin{center}
		\begin{tabular}{|m{2.8em}|m{2.8em}|m{2.8em}|m{2.8em}|m{2.8em}|m{2.8em}|m{2.8em}|m{2.8em}|m{2.8em}|m{2.8em}|}
			\hline
			01.B  & 02.D  & 03.B  & 04.B  & 05.B  & 06.C  & 07.B & 08.C & 09.C & \\
			\hline
			
		\end{tabular}
	\end{center}
}

\section{Mẫu nguyên tử Bo}
\begin{enumerate}[label=\bfseries Câu \arabic*:]
	\item \mkstar{3} 
	
	\cauhoi
	{Trong nguyên tử Hidro, bán kính Bo là $r_0=\text{5,3}\cdot 10^{-11}\ \text{m}$. Bán kính quỹ đạo dừng N là
		\begin{mcq}(4)
			\item $\text{47,7}\cdot 10^{-11}\ \text{m}.$ 
			\item $\text{21,2}\cdot 10^{-11}\ \text{m}.$ 
			\item $\text{84,4}\cdot 10^{-11}\ \text{m}.$ 
			\item $\text{132,5}\cdot 10^{-11}\ \text{m}.$ 
		\end{mcq}
	}
	
	\loigiai
	{		\textbf{Đáp án: C.}
		
Trong nguyên tử Hidro, bán kính quỹ dạo dừng $ N $ cho bởi:
$$
	r_{N} = 4^{2} \cdot r_{0} = \SI{84,4 e-11}{m}.
$$
		
	}
	
	\item \mkstar{3} 
	
	\cauhoi
	{Trong nguyên tử Hidro, bán kính Bo là $r_0=\text{5,3}\cdot 10^{-11}\ \text{m}$.   Ở một trạng thái kích thích của nguyên tử Hidro, electron chuyển động trên quỹ đạo dừng có bán kính là $r=\text{2,12}\cdot 10^{-10}\ \text{m}$.   Quỹ đạo đó có tên gọi là 
		\begin{mcq}(2)
			\item quỹ đạo dừng L. 
			\item quỹ đạo dừng O. 
			\item quỹ đạo dừng N. 
			\item quỹ đạo dừng M. 
		\end{mcq}
	}
	
	\loigiai
	{		\textbf{Đáp án: A.}
		
Ta có:
$$
	r_{n} = n^{2} \cdot r_{0} \rightarrow n = \num{2}.
$$
Vậy electron chuyển động trên quỹ dạo dừng $ L $.
		
	}
	
	\item \mkstar{3} 
	
	\cauhoi
	{Vận dụng mẫu nguyên tử Rutherford cho nguyên tử Hidro. Cho hằng số điện $k=9\cdot 10^9\ \text{Nm}^2/\text{C}^2$, hằng số điện tích nguyên tố $e=\text{1,6}\cdot 10^{-19}\ \text{C}$ và khối lượng của electron $m_\text{e}=\text{9,1}\cdot 10^{-31}\ \text{kg}$. Khi electron chuyển động trên quỹ đạo tròn bán kính $r = \text{2,12} \cdot 10^{-10}\ \text{m}$ thì tốc độ chuyển động của electron xấp xỉ bằng
		\begin{mcq}(4)
			\item $\text{1,1}\cdot 10^6\ \text{m/s}.$
			\item $\text{1,4}\cdot 10^6\ \text{m/s}.$ 
			\item $\text{2,2}\cdot 10^6\ \text{m/s}.$ 
			\item $\text{3,3}\cdot 10^6\ \text{m/s}.$ 
		\end{mcq}
	}
	
	\loigiai
	{		\textbf{Đáp án: A.}
		
Tốc độ chuyển động của electron cho bởi:
$$
	v_{n} = \sqrt{\dfrac{k e^{2}}{m r_{n}}} = \SI{1,1 e6}{m/s}.
$$
		
	}
	
	\item \mkstar{3} 
	
	\cauhoi
	{Theo mẫu nguyên tử Bohr, bán kính quỹ đạo K của electron trong nguyên tử Hidro là $r_0$. Khi electron chuyển từ quỹ đạo N về quỹ đạo L thì bán kính quỹ đạo giảm bớt
		\begin{mcq}(4)
			\item $12r_0$. 
			\item $16r_0$. 
			\item $9r_0$. 
			\item $4r_0$. 
		\end{mcq}
	}
	
	\loigiai
	{		\textbf{Đáp án: A.}
		
Bán kính quỹ đạo giảm bớt một lượng cho bởi:
$$
	\Delta r = r_{N} - r_{L} = 4^{2} r_{0} - 2^{2} r_{0} = 12 r_{0}.
$$
		
	}
	
	\item \mkstar{3} 
	
	\cauhoi
	{Theo mẫu nguyên tử Bo, trong nguyên tử Hidro, chuyển động của electron quanh hạt nhân là chuyển động tròn đều. Tỉ số giữa tốc độ của electron trên quỹ đạo K và tốc độ của electron trên quỹ đạo M bằng
		\begin{mcq}(4)
			\item $9.$ 
			\item $3.$ 
			\item $4.$ 
			\item $2.$ 
		\end{mcq}
	}
	
	\loigiai
	{		\textbf{Đáp án: B.}
		
Ta có:
$$
	v_{M} = \dfrac{v_{K}}{3} \rightarrow \dfrac{v_{K}}{v_{M}} = \num{3}.
$$
		
	}
	
\item \mkstar{3} 
	
	\cauhoi
	{Theo mẫu Bo về nguyên tử Hidro, nếu lực tương tác tĩnh điện giữa electron và hạt nhân khi electron chuyển động trên quỹ đạo dừng L là $F$ thì khi electron chuyển động trên quỹ đạo dừng N, lực này sẽ là
		\begin{mcq}(4)
			\item $\dfrac{F}{16}.$ 
			\item $\dfrac{F}{25}.$ 
			\item $\dfrac{F}{9}.$ 
			\item $\dfrac{F}{4}.$
		\end{mcq}
	}
	
	\loigiai
	{		\textbf{Đáp án: A.}
		
Ta có $ F_{N} = \dfrac{F_{K}}{4^{4}} $ và $ F_{L} = \dfrac{F_{K}}{2^{4}} = F $. Từ đó suy ra:
$$
	\dfrac{F_{N}}{F_{L}} = \dfrac{1}{16} \rightarrow F_{N} = \dfrac{F}{16}.
$$
		
	}
	
	\item \mkstar{3} 
	
	\cauhoi
	{Khi electron ở quỹ đạo dừng thứ $n$ thì năng lượng của nguyên tử Hidro được tính theo công thức $E_n=-\dfrac{\text{13,6}}{n^2}\ \text{eV}$  . Khi electron trong nguyên tử Hidro chuyển từ quỹ đạo dừng $n = 3$ sang quỹ đạo dừng $n = 2$ thì nguyên tử Hidro phát ra phôtôn ứng với bức xạ có bước sóng bằng
		\begin{mcq}(4)
			\item $\text{0,4350}\ \mu\text{m}$.
			\item $\text{0,6576}\ \mu\text{m}$.
			\item $\text{0,4102}\ \mu\text{m}$.
			\item $\text{0,4861}\ \mu\text{m}$. 
		\end{mcq}
	}
	
	\loigiai
	{		\textbf{Đáp án: B.}
		
Năng lượng mà photon phát ra cho bởi:
$$
	\varepsilon = E_{3} - E_{2} = \SI{1,9}{eV}.
$$
Bước sóng của photon tương ứng cho bởi:
$$
	\lambda = \dfrac{hc}{\varepsilon} = \SI{0,6576}{\mu m}.
$$
	}
	
	\item \mkstar{3} 
	
	\cauhoi
	{Theo mẫu nguyên tử Bo, trong nguyên tử Hidro, khi electron chuyển từ quỹ đạo P về quỹ đạo K thì nguyên tử phát ra phôtôn ứng với bức xạ có tần số $f_1$. Khi electron chuyển từ quỹ đạo P về quỹ đạo L thì nguyên tử phát ra phôtôn ứng với bức xạ có tần số $f_2$. Nếu electron chuyển từ quỹ đạo L về quỹ đạo K thì nguyên tử phát ra phôtôn ứng với bức xạ có tần số
		\begin{mcq}(4)
			\item $f_3=f_1-f_2$. 
			\item $f_3=f_1+f_2$. 
			\item $f_3=\sqrt{f^2_1+f^2_2}$. 
			\item $f_3=\dfrac{f_1\cdot f_2}{f_1+f_2}$ 
		\end{mcq}
	}
	
	\loigiai
	{		\textbf{Đáp án: A.}
		
Ta có:
$$
	\varepsilon_{PK} = E_{P} - E_{K} = hf_{1}.
$$
Lại có:
$$
	\varepsilon_{PL} = E_{P} - E_{L} = hf_{2}.
$$
Suy ra:
$$
	\varepsilon_{LK} = E_{L} - E_{K} = (E_{P} - E_{K}) - (E_{P} - E_{L}) = \varepsilon_{PK} - \varepsilon_{PL} \rightarrow hf_{3} = hf_{1} - hf_{2}.
$$
Vậy:
$$
	f_{3} = f_{1} - f_{2}.
$$
		
	}
	
	\item \mkstar{3} 
	
	\cauhoi
	{Cho biết giá trị hằng số $h=\text{6,625}\cdot 10^{-34}\ \text{Js}$, $c=3\cdot 10^8\ \text{m/s}$ và $\text{eV}=\text{1,6}\cdot 10^{-19}\ \text{J}$. Khi electron trong nguyên tử hidro chuyển từ quỹ đạo dừng có năng lượng $E_m=-\text{0,85}\ \text{eV}$ sang quỹ đạo dừng có năng lượng $E_n=-\text{13,6}\ \text{eV}$ thì nguyên tử phát bức xạ điện từ có bước sóng 
		\begin{mcq}(4)
			\item $\text{0,0974}\ \mu\text{m}$. 
			\item $\text{0,4340}\ \mu\text{m}$. 
			\item $\text{0,4860}\ \mu\text{m}$.
			\item $\text{0,6563}\ \mu\text{m}$. 
		\end{mcq}
	}
	
	\loigiai
	{		\textbf{Đáp án: A.}
		
Năng lượng của photon phát ra cho bởi:
$$
	\varepsilon = E_{m} - E_{n} = \SI{12,75}{eV}.
$$
Bước sóng của photon phát ra cho bởi:
$$
	\lambda = \dfrac{hc}{\varepsilon} = \SI{0,0974}{\mu m}.
$$
		
	}
	
	\item \mkstar{3} 
	
	\cauhoi
	{Cho biết giá trị hằng số $h=\text{6,625}\cdot 10^{-34}\ \text{Js}$ và độ lớn điện tích của lectron $e=\text{1,6}\cdot 10^{-19}\ \text{C}$. Khi electron trong nguyên tử hidro chuyển từ quỹ đạo dừng có năng lượng  $E_m=-\text{1,514}\ \text{eV}$ sang quỹ đạo dừng có năng lượng  $E_m=-\text{3,407}\ \text{eV}$ thì nguyên tử phát bức xạ điện từ có tần số
		\begin{mcq}(4)
			\item $\text{2,571}\cdot 10^{13}\ \text{Hz}.$ 
			\item $\text{4,572}\cdot 10^{14}\ \text{Hz}.$ 
			\item $\text{3,879}\cdot 10^{13}\ \text{Hz}.$ 
			\item $\text{6,542}\cdot 10^{13}\ \text{Hz}.$ 
		\end{mcq}
	}
	
	\loigiai
	{		\textbf{Đáp án: B.}
		
Năng lượng của photon phát ra cho bởi:
$$
	\varepsilon = E_{m} - E_{n} = \SI{1,893}{eV}.
$$
Tần số của ánh sáng phát ra cho bởi:
$$
	\varepsilon = hf \rightarrow f = \SI{4,572 e14}{Hz}.
$$
		
	}
	
\item \mkstar{3} 
	
	\cauhoi
	{Trong nguyên tử hidro, electron từ quỹ đạo L chuyển về quỹ đạo K có năng lượng $E_K= -\text{13,6}\ \text{eV}$. Bước sóng bức xạ phát ra bằng $\text{0,1218}\ \mu\text{m}$. Mức năng lượng ứng với quỹ đạo L bằng 
		\begin{mcq}(4)
			\item $\text{3,2}\ \text{eV}.$ 
			\item $-\text{3,4}\ \text{eV}.$ 
			\item $-\text{4,1}\ \text{eV}.$ 
			\item $-\text{5,6}\ \text{eV}.$ 
		\end{mcq}
	}
	
	\loigiai
	{		\textbf{Đáp án: B.}
		
Năng lượng của photon phát ra cho bởi:
$$
	\varepsilon = \dfrac{hc}{\lambda} = \SI{10,2}{eV}.
$$
Ta có:
$$
	\varepsilon = E_{L} - E_{K} \rightarrow E_{L} = \SI{-3,4}{eV}.
$$
	}
	
	\item \mkstar{3} 
	
	\cauhoi
	{Nguyên tử hydro ở trạng thái cơ bản có mức năng lượng bằng $-\text{13,6}\ \text{eV}.$ Để chuyển lên trạng thái dừng có mức năng lượng $-\text{3,4}\ \text{eV}$ thì nguyên tử hidro phải hấp thụ một photon có năng lượng là 
		\begin{mcq}(4)
			\item $\text{10,2}\ \text{eV}.$ 
			\item $-\text{10,2}\ \text{eV}.$ 
			\item $\text{17}\ \text{eV}.$. 
			\item $\text{4}\ \text{eV}.$ 
		\end{mcq}
	}
	
	\loigiai
	{		\textbf{Đáp án: A.}
		
Năng lượng photon cần hấp thụ cho bởi:
$$
	\varepsilon = \SI{-3,4}{eV} - ( \SI{-13,6}{eV} ) = \SI{10,2}{eV}.
$$
		
	}
	
	\item \mkstar{3} 
	
	\cauhoi
	{Đối với nguyên tử hidro, khi elextron chuyển từ quỹ đạo M về quỹ đạo K thì nguyên tử phát ra photon có bước sóng $\text{0,1026}\ \mu\text{m}$. Cho $h=\text{6,625}\cdot 10^{-34}\ \text{Js}$ và độ lớn điện tích của electron $e=\text{1,6}\cdot 10^{-19}\ \text{C}$. Năng lượng của photon là
		\begin{mcq}(4)
			\item $\text{1,21}\ \text{eV}.$
			\item $\text{11,2}\ \text{eV}.$
			\item $\text{12,1}\ \text{eV}.$ 
			\item $\text{121}\ \text{eV}.$
		\end{mcq}
	}
	
	\loigiai
	{		\textbf{Đáp án: C.}
		
Năng lượng của photon cho bởi:
$$
	\varepsilon = \dfrac{hc}{\lambda} = \SI{12,1}{eV}.
$$
		
	}
	
	\item \mkstar{3} 
	
	\cauhoi
	{Cho $h=\text{6,625}\cdot 10^{-34}\ \text{Js}$, $c=3\cdot 10^8\ \text{m/s}$. Mức năng lượng của các quỹ đạo dừng của nguyên tử hidro lần lượt từ trong ra ngoài là $-\text{13,6}\ \text{eV}; \ -\text{3,4}\ \text{eV};\ -\text{1,5}\ \text{eV},...$ với $E_n=-\dfrac{\text{13,6}}{n^2}\ \text{eV}, \ n=1, \ 2, \ 3,...$   Khi electron chuyển từ mức năng lượng ứng với $n = 3$ về $n = 1$ thì sẽ phát ra bức xạ có tần số là 
		\begin{mcq}(4)
			\item $\text{2,9}\cdot 10^{14}\ \text{Hz}.$
			\item $\text{2,9}\cdot 10^{15}\ \text{Hz}.$ 
			\item $\text{2,9}\cdot 10^{16}\ \text{Hz}.$ 
			\item $\text{2,9}\cdot 10^{17}\ \text{Hz}.$ 
		\end{mcq}
	}
	
	\loigiai
	{		\textbf{Đáp án: B.}
		
Năng lượng mà photon sẽ phát ra cho bởi:
$$
	\varepsilon = E_{3} - E_{1} = \SI{12,1}{eV}.
$$
Tần số của photon tương ứng cho bởi:
$$
	\varepsilon = hf \rightarrow f = \SI{2,9 e15}{Hz}.
$$
		
	}
	
\end{enumerate}

\loigiai
{
	\begin{center}
		\textbf{BẢNG ĐÁP ÁN}
	\end{center}
	\begin{center}
		\begin{tabular}{|m{2.8em}|m{2.8em}|m{2.8em}|m{2.8em}|m{2.8em}|m{2.8em}|m{2.8em}|m{2.8em}|m{2.8em}|m{2.8em}|}
			\hline
			01.C  & 02.A  & 03.A  & 04.A  & 05.B  & 06.A  & 07.B & 08.A & 09.A & 10.B \\
			\hline
			11.B  & 12.A  & 13.C  & 14.B  &  &  & &  & & \\
			\hline
			
		\end{tabular}
	\end{center}
}

\section{Sơ lược về laser}
\begin{enumerate}[label=\bfseries Câu \arabic*:]

\item \mkstar{1}
	
	\cauhoi
	{Tia laze không có đặc điểm nào dưới đây ?	
		\begin{mcq}(2)
			\item Độ đơn sắc cao. 
			\item Độ định hướng cao. 
			\item Cường độ lớn.
			\item Công suất lớn. 
		\end{mcq}
	}
	
	\loigiai
	{		\textbf{Đáp án: D.}
		
Tia laze không nhất thiết phải có công suất lớn.
		
	}
	
\item \mkstar{1} 
	
	\cauhoi
	{Laze rubi hoạt động theo nguyên tắc nào dưới đây?
		\begin{mcq}(1)
			\item Dựa vào sự phát xạ cảm ứng. 
			\item Tạo ra sự đảo lộn mật độ. 
			\item Dựa vào sự tái hợp giữa êlectron và lỗ trống.
			\item Sử dụng buồng cộng hưởng. 
		\end{mcq}
	}
	
	\loigiai
	{		\textbf{Đáp án: A.}
		
Nguyên tắc hoạt động của laze là dựa vào hiện tượng phát xạ cảm ứng.
		
	}
	
	\item \mkstar{3}
	
	\cauhoi
	{Một laze He – Ne phát ánh sáng có bước sóng $\text{632,8}\ \text{nm}$ và có công suất đầu ra là $\text{2,3}\ \text{mW}$. Số photon phát ra trong mỗi phút là
		\begin{mcq}(4)
			\item $22\cdot 10^{15}$. 
			\item $24\cdot 10^{15}$. 
			\item $\text{44}\cdot 10^{16}$. 
			\item $44\cdot 10^{15}$. 
		\end{mcq}
	}
	
	\loigiai
	{		\textbf{Đáp án: C.}

Năng lượng của photon phát ra cho bởi:
$$
	\varepsilon = \dfrac{hc}{\lambda} = \SI{3,14 e-19}{J}.
$$		
Số photon phát ra trong mỗi phút cho bởi:
$$
	N_{\text{phút}} = \dfrac{P}{\varepsilon} \cdot 60 = \num{44 e16}.
$$
		
	}
	
	\item \mkstar{3} 
	
	\cauhoi
	{Một laze rubi phát ra ánh sáng có bước sóng $\text{694,4}\ \text{nm}$. Nếu xung laze được phát ra trong $t$ (s) và năng lượng giải phóng bởi mỗi xung là $Q = \text{0,15}\ \text{J}$ thì số photon trong mỗi xung là
		\begin{mcq}(4)
			\item $\text{22}\cdot 10^{16}$. 
			\item $\text{24}\cdot 10^{17}$.  
			\item $\text{5,24}\cdot 10^{17}$. 
			\item $\text{5,44}\cdot 10^{15}$.
		\end{mcq}
	}
	
	\loigiai
	{		\textbf{Đáp án: C.}
		
Năng lượng của mỗi photon cho bởi:
$$
	\varepsilon = \dfrac{hc}{\lambda} = \SI{2,862 e-19}{J}.
$$
Số photon phát ra trong mỗi xung cho bởi:
$$
	N = \dfrac{W}{\varepsilon} = \num{5,24 e17}.
$$
		
	}
	
	\item \mkstar{1} 
	
	\cauhoi
	{Để đo khoảng cách từ Trái Đất lên Mặt Trăng người ta dùng một tia laze phát ra những xung ánh sáng có bước sóng $\text{0,52}\ \mu\text{m}$, chiếu về phía Mặt Trăng. Thời gian kéo dài mỗi xung là $10^{-7}$ (s) và công suất của chùm laze là 100000 MW. Số phôtôn chứa trong mỗi xung là
		\begin{mcq}(4)
			\item $\text{2,62}\cdot 10^{22}$ hạt. 
			\item $\text{2,62}\cdot 10^{16}$ hạt.
			\item $\text{2,62}\cdot 10^{29}$ hạt. 
			\item $\text{5,2}\cdot 10^{20}$ hạt. 
		\end{mcq}
	}
	
	\loigiai
	{		\textbf{Đáp án: B.}
		
Năng lượng của photon cho bởi:
$$
	\varepsilon = \dfrac{hc}{\lambda} = \SI{3,822 e-19}{J}.
$$
Số photon phát ra trong một xung cho bởi:
$$
	N_{\text{xung}} = \dfrac{Pt}{\varepsilon} = \num{2,62 e16}.
$$
		
	}
	
\item \mkstar{1} 
	
	\cauhoi
	{Một đèn laze có công suất phát sáng 1W phát ánh sáng đơn sắc có bước sóng $\text{0,7}\ \mu\text{m}$. Cho $h = \text{6,625}\cdot 10^{-34}\ \text{Js}$, $c = 3\cdot 10^8\ \text{m/s}$. Số phôtôn của nó phát ra trong 1 giây là
		\begin{mcq}(4)
			\item $\text{3,52}\cdot 10^{16}$. 
			\item $\text{3,52}\cdot 10^{19}$. 
			\item $\text{3,52}\cdot 10^{18}$.
			\item $\text{3,52}\cdot 10^{20}$. 
		\end{mcq}
	}
	
	\loigiai
	{		\textbf{Đáp án: C.}

Năng lượng của photon cho bởi:
$$
	\varepsilon = \dfrac{hc}{\lambda} = \SI{2,84 e-19}{J}.
$$		
Số photon laze phát ra trong một giây cho bởi:
$$
	N = \dfrac{P}{\varepsilon} = \num{3,52 e18}.
$$
		
	}
	
	\item \mkstar{1} 
	
	\cauhoi
	{Màu của laze phát ra
		\begin{mcq}(2)
			\item màu trắng.
			\item hỗn hợp màu đơn sắc. 
			\item hỗn hợp nhiều màu đơn sắc. 
			\item màu đơn sắc. 
		\end{mcq}
	}
	
	\loigiai
	{		\textbf{Đáp án: D.}
		
Màu của laze phát ra màu đơn sắc.
		
	}
	
	\item \mkstar{1}
	
	\cauhoi
	{Phát biểu \textbf{sai} về tia laze
		\begin{mcq}(1)
			\item tia laze có tính định hướng cao.
			\item tia laze bị tán sắc khi đi qua lăng kính.
			\item tia laze là chùm sáng kết hợp. 
			\item tai laze có cường độ lớn. 
		\end{mcq}
	}
	
	\loigiai
	{		\textbf{Đáp án: B.}
		
Tia laze không bị tán sắc khi đi qua lăng kính.
		
	}
		

\end{enumerate}

\loigiai
{
	\begin{center}
		\textbf{BẢNG ĐÁP ÁN}
	\end{center}
	\begin{center}
		\begin{tabular}{|m{2.8em}|m{2.8em}|m{2.8em}|m{2.8em}|m{2.8em}|m{2.8em}|m{2.8em}|m{2.8em}|m{2.8em}|m{2.8em}|}
			\hline
			01.D  & 02.A  & 03.C  & 04.C  & 05.B  & 06.C  & 07.D & 08.B & & \\
			\hline
			
		\end{tabular}
	\end{center}
}
\whiteBGstarEnd