\whiteBGstarBegin
\setcounter{section}{0}
\section{Lý thuyết: Mẫu nguyên tử Bohr: 2 tiên đề Bohr}
\begin{enumerate}[label=\bfseries Câu \arabic*:]

% Câu No 
		\item \mkstar{1} [5]
	
		\cauhoi
		{Ở trạng thái dừng, mỗi electron chuyển động trên hạt nhân với quỹ đạo dừng có bán kính
		\begin{mcq}(4)
			\item giảm dần. 
			\item giảm rồi tăng.
			\item tăng dần. 
			\item xác định. 
		\end{mcq}
		}
	
		\loigiai
		{		\textbf{Đáp án: D.}

Ở trạng thái dừng, mỗi electron chuyển động trên hạt nhân với quỹ đạo dừng có bán kính xác định.		
		}

% Câu No 
		\item \mkstar{2} [1] 
	
		\cauhoi
		{Xét nguyên tử Hidro theo mẫu nguyên tử Bo. Gọi $ r_{0} $ là bán kính Bo thì quỹ đạo dừng N có bán kính bằng
		\begin{mcq}(4)
			\item $ 4 r_{0} $. 
			\item $ 16 r_{0} $.
			\item $ 2 r_{0} $. 
			\item $ 8 r_{0} $. 
		\end{mcq}
		}
	
		\loigiai
		{		\textbf{Đáp án: B.}	
		
Quỹ đạo dừng N có $ n = 3 $. Nên bán kính quỹ đạo dừng N là $ n^{2} r_{0} = 16 r_{0} $.
		}
		
% Câu No 
		\item \mkstar{2} [1]
	
		\cauhoi
		{Theo lý thuyết Bo, khi chuyển từ trạng thái $ \SI{-3,4}{eV} $ lên trạng thái dừng có năng lượng $ \SI{-0,85}{eV} $ nguyên tử Hidro phải hấp thụ photon có năng lượng
		\begin{mcq}(4)
			\item $ \SI{0,85}{eV} $. 
			\item $ \SI{4,25}{eV} $.
			\item $ \SI{3,4}{eV} $. 
			\item $ \SI{2,55}{eV} $. 
		\end{mcq}
		}
	
		\loigiai
		{		\textbf{Đáp án: D.}
		
Nguyên tử Hidro phải hấp thụ photon có năng lượng là
$$
	\varepsilon = \SI{-0,85}{eV} - \left( \SI{-3,4}{eV} \right) = \SI{2,55}{eV}.
$$	
		}
	
% Câu No 
		\item \mkstar{3} [1]
	
		\cauhoi
		{Theo lí thuyết Bo, năng lượng trong nguyên tử Hidro được xác định bằng công thức $ E_{n} = \xsi{\dfrac{-13,6}{n^{2}}}{eV}$ với $ n = 1, 2, 3, ...\infty $ ứng với các quỹ đạo K, L, M, ... Gọi $ r_{0} $ là bán kính Bo. Khi nguyên tử ở trạng thái dừng có năng lượng $ \SI{-0,544}{eV} $, electron trong nguyên tử đang chuyển động với quỹ đạo có bán kính
		\begin{mcq}(4)
			\item $ 25 r_{0} $. 
			\item $ 5 r_{0} $.
			\item $ 4 r_{0} $. 
			\item $ 16 r_{0} $. 
		\end{mcq}
		}
	
		\loigiai
		{		\textbf{Đáp án: A.}

Ta có:
$$
	E_{n} = \dfrac{-13,6}{n^{2}} \ rightarrow \num{-0,544} = \dfrac{-13,6}{n^{2}} \rightarrow n = 5.
$$		
Khi đó, bán kính của quỹ đạo electron là $ n^{2} r_{0} = 25 r_{0}$.
		}
		
% Câu No 
		\item \mkstar{3} [2]
	
		\cauhoi
		{Xét nguyên tử Hidro theo mẫu nguyên tử Bo. Electron trong nguyên tử chuyển từ quỹ đạo dừng n lên quỹ đạo dừng m thì bán kính quỹ đạo tăng 9 lần. Gọi $ r_{0} $ là bán kính quỹ đạo Bo thì bán kính quỹ đạo dừng m \textbf{không} thể nhận giá trị nào sau đây?
		\begin{mcq}(4)
			\item $ 100 r_{0} $. 
			\item $ 144 r_{0} $.
			\item $ 36 r_{0} $. 
			\item $ 18 r_{0} $. 
		\end{mcq}
		}
	
		\loigiai
		{		\textbf{Đáp án: D.}

Vì khi chuyển từ quỹ đạo dừng n lên quỹ đạo dừng m thì bán kính tăng 9 lần nên
$$
r_{m} = 9 \cdot r_{n} \rightarrow m^{2} \cdot r_{0} = 9 n^{2} \cdot r_{0} \ rightarrow n^{2} = \dfrac{m^{2}}{9}.
$$		
Ta thấy khi $ r_{m} = 100 r_{0} $ thì $ r_{n} = \SI{1111,1}{r_{0}}$.
Vậy $ r_{m} $ không thể nhận giá trị $ 100 r_{0} $
		}
		
% Câu No 
		\item \mkstar{3} [2]
	
		\cauhoi
		{Trong nguyên tử Hidro, bán kính Bo là $ r_{0} = \SI{5,3e-11}{m} $. Ở một trạng thái kích thích của nguyên tử Hidro, electron chuyển động trên quỹ đạo dừng có bán kính $ r = \SI{2,12e-10}{m} $. Quỹ đạo đó có tên quỹ đạo dừng
		\begin{mcq}(4)
			\item L. 
			\item M.
			\item O. 
			\item N. 
		\end{mcq}
		}
		
		
		\loigiai
		{		\textbf{Đáp án: A.}

Ta có:
$$
	r = n^{2} \cdot r_{0} \rightarrow n = \num{2}.
$$
Vậy electron này dang ở trên quỹ đạo L.		
		}
		
% Câu No 
		\item \mkstar{3} [2]
	
		\cauhoi
		{Theo mẫu Bo về nguyên tử Hidro, lực tương tác tĩnh điện giữa electron và hạt nhân khi electron chuyển động trên quỹ đạo dừng K là F. Khi electron chuyển động từ quỹ đạo dừng N về quỹ đạo dừng L thì lực tương tác tĩnh điện giữa electron và hạt nhân tăng thêm
		\begin{mcq}(4)
			\item $ \dfrac{15}{16} F $. 
			\item $ \dfrac{15}{256} F $.
			\item $ 12 F $. 
			\item $ 240 F $. 
		\end{mcq}
		}
	
		\loigiai
		{		\textbf{Đáp án: B.}

Lực tương tác trên quỹ đạo dừng K là
$$
	F = k \cdot \dfrac{e^{2}}{r_{0}^2}
$$ 
Lực tương tác trên quỹ đạo dừng N là
$$
	F_{N} = k \cdot \dfrac{e^{2}}{16^{2} \cdot r_{0}^2} = \dfrac{F}{256}
$$   
Lực tương tác trên quỹ đạo dừng L là
$$
	F_{L} = k \cdot \dfrac{e^{2}}{4^{2} \cdot r_{0}^2} = \dfrac{F}{16}
$$           		
Vậy khi chuyển từ quỹ đạo dừng N về quỹ đạo dừng L thì lực tương tác tĩnh điện giữa electron và hạt nhân tăng thêm $ \dfrac{F}{16} - \dfrac{F}{256} = \dfrac{15}{256} F$.
		}
		
% Câu No 
		\item \mkstar{3} [4]
	
		\cauhoi
		{Theo mẫu nguyên tử Bo, trạng thái dừng của nguyên tử
		\begin{mcq}(1)
			\item có thể là trạng thái cơ bản hoặc trạng thái kích thích. 
			\item chỉ là trạng thái kích thích.
			\item là trạng thái mà các electron nguyên tử ngừng chuyển động. 
			\item chỉ là trạng thái cơ bản. 
		\end{mcq}
		}
	
		\loigiai
		{		\textbf{Đáp án: A.}

Theo mẫu nguyên tử Bo, trạng thái dừng của nguyên tử có thể là trạng thái cơ bản hoặc trạng thái kích thích. 
		}
		
% Câu No 
		\item \mkstar{3} [4]
	
		\cauhoi
		{Nguyên tử H chuyển từ trạng thái kích thích về trạng thái nguyên tử có mức năng lượng thấp hơn phát ra bức xạ có bước sóng $ \lambda = \SI{486}{nm} $. Độ giảm năng lượng của nguyên tử khi phát ra bức xạ là
		\begin{mcq}(4)
			\item $ \SI{4,09e-15}{J} $. 
			\item $ \SI{4,86e-19}{J} $.
			\item $ \SI{4,09e-19}{J} $. 
			\item $ \SI{3,08e-20}{J} $. 
		\end{mcq}
		}
	
		\loigiai
		{		\textbf{Đáp án: C.}
		
Độ giảm năng lượng của nguyên tử khi phát ra bức xạ đúng bằng năng lượng của photon phát ra cho bởi:
$$
	\varepsilon = \dfrac{hc}{\lambda} = \SI{4,09e-19}{J}.
$$
		}
		
% Câu No 
		\item \mkstar{1} [10]
	
		\cauhoi
		{Trạng thái dừng của nguyên tử là
		\begin{mcq}(1)
			\item  trạng thái đứng yên của nguyên tử. 
			\item trạng thái chuyển động đều của nguyên tử.
			\item trạng thái trong đó mọi êlectron của nguyên tử đều không chuyển động đối với hạt nhân.
			\item một trong số các trạng thái có năng lượng xác định, mà nguyên tử có thể tồn tại.
		\end{mcq}
		}
	
		\loigiai
		{		\textbf{Đáp án: D.}

Trạng thái dừng của nguyên tử là một trong số các trạng thái có năng lượng xác định, mà nguyên tử có thể tồn tại.		
		}
		
% Câu No 
		\item \mkstar{3} [10]
	
		\cauhoi
		{Phát biểu nào sau đây là đúng khi nói về mẫu nguyên tử Borh?
		\begin{mcq}(1)
			\item Nguyên tử bức xạ khi chuyển từ trạng thái cơ bản lên trạng thái kích thích. 
			\item Trong các trạng thái dừng, động năng của êlectron trong nguyên tử bằng không.
			\item Khi ở trạng thái cơ bản, nguyên tử có năng lượng cao nhất.
			\item Trạng thái kích thích có năng lượng càng cao thì bán kính quỹ đạo của êlectron càng lớn.
		\end{mcq}
		}
	
		\loigiai
		{		\textbf{Đáp án: D.}

Trạng thái kích thích có năng lượng càng cao thì bán kính quỹ đạo của êlectron càng lớn.		
		}
		
% Câu No 
		\item \mkstar{3} [13]
	
		\cauhoi
		{Tốc độ của electron trong nguyên tử Hidro khi nó tồn tại ở trạng thái dừng M là
		\begin{mcq}(4)
			\item $ \SI{0,73e6}{m/s} $. 
			\item $ \SI{1,09e6}{m/s} $.
			\item $ \SI{0,55e6}{m/s} $. 
			\item $ \SI{0,22e6}{m/s} $. 
		\end{mcq}
		}
	
		\loigiai
		{		\textbf{Đáp án: A.}

Áp dụng định luật II Newton trên phương hướng tâm ta có:
$$
	F = m a_{ht} \rightarrow k \cdot \dfrac{e^{2}}{r_{M}^2} = m \cdot \dfrac{v^{2}}{r_{M}^2} \rightarrow k \cdot \dfrac{e^{2}}{\left( 3^{2} r_{0} \right)^{2}} = m \cdot \dfrac{v^{2}}{ 3^{2} r_{0} } \rightarrow v_{0} = \SI{0,73e6}{m/s}.
$$		
		}
	
\end{enumerate}

\loigiai
{
	\begin{center}
		\textbf{BẢNG ĐÁP ÁN}
	\end{center}
	\begin{center}
		\begin{tabular}{|m{2.8em}|m{2.8em}|m{2.8em}|m{2.8em}|m{2.8em}|m{2.8em}|m{2.8em}|m{2.8em}|m{2.8em}|m{2.8em}|}
			\hline
			01.D  & 02.B  & 03.D  & 04.A   & 05.D  & 06.A  & 07.B & 08.A & 09.C & 10.D \\
			\hline
			11.D  & 12.A  &  &  &  &  & & & &  \\
			\hline
			
		\end{tabular}
	\end{center}
}

\section{Lý thuyết: Quang phổ vạch của nguyên tử hydro}
\begin{enumerate}[label=\bfseries Câu \arabic*:]

% Câu No 
		\item \mkstar{1} [3]
	
		\cauhoi
		{Trong quang phổ vạch của nguyên tử Hidro, ở vùng ánh sáng nhìn thấy có 4 vạch đặc trưng là vạch
		\begin{mcq}(2)
			\item đỏ, lam, chàm, tím. 
			\item đỏ, cam, vàng, tím.
			\item đỏ, cam, chàm, tím. 
			\item đỏ, lục, lam, tím. 
		\end{mcq}
		}
	
		\loigiai
		{		\textbf{Đáp án: A.}

Trong quang phổ vạch của nguyên tử Hidro, ở vùng ánh sáng nhìn thấy có 4 vạch đặc trưng là vạch đỏ, lam, chàm, tím.
		}

% Câu No 
		\item \mkstar{3} [3]
	
		\cauhoi
		{Theo mẫu Bo về nguyên tử Hidro, nếu lực tượng tác tĩnh điện khi electron và hạt nhân khi chuyển động trên quỹ đạo dừng là $ L $ là $ F $ thì khi electron chuyển động trên quỹ đạo dừng $ n $ lực tĩnh điện là $ \dfrac{F}{16} $. Tỉ số giữa bước sóng dài nhất và bước sóng ngắn nhất mà nguyên tử Hidro có thể phát ra bằng
		\begin{mcq}(4)
			\item $ \dfrac{7}{135} $. 
			\item $ \dfrac{3}{128} $.
			\item $ \dfrac{135}{7} $. 
			\item $ \dfrac{128}{3} $. 
		\end{mcq}
		}
	
		\loigiai
		{		\textbf{Đáp án: A.}

Lực tương tác tĩnh điện trên quỹ đạo $ L $ là
$$
	F = k \cdot \dfrac{e^{2}}{\left( 2^{2} r_{0} \right)^2} = k \cdot \dfrac{e^{2}}{16 r_{0}^{2}}.
$$
Lực tương tác tĩnh điện trên quỹ đạo $ n $ là
$$
	\dfrac{F}{16} = k \cdot \dfrac{e^{2}}{\left( n^{2} r_{0} \right)^2} \rightarrow n = \num{4}.
$$
Ta có:
$$
	\dfrac{\lambda_{max}}{\lambda_{min}} = \dfrac{\varepsilon_{max}}{\varepsilon_{min}} = \dfrac{E_{4} - E_{1}}{E_{4} - E_{3}} = \dfrac{\dfrac{-13,6}{4^{4}} - \dfrac{-13,6}{1^{2}}}{\dfrac{-13,6}{4^{2}} - \dfrac{-13,6}{3^{2}}} = \dfrac{135}{7}.
$$
		}
		
% Câu No 
		\item \mkstar{3} [3]
	
		\cauhoi
		{Các mức năng lượng của các trạng thái dừng của nguyên tử Hidro được xác định bằng biểu thức $ E_{n} = \xsi{\dfrac{-13,6}{n^{2}}}{eV} $ với $ n = 1,2,3... $ . Mức năng lượng của nguyên tử Hidro ở trạng thái dừng $ M $ là
		\begin{mcq}(4)
			\item $ \SI{-1,51}{eV} $. 
			\item $ \SI{-0,544}{eV} $.
			\item $ \SI{-0,85}{eV} $. 
			\item $ \SI{-3,4}{eV} $. 
		\end{mcq}
		}
	
		\loigiai
		{		\textbf{Đáp án: A.}

Mức năng lượng của nguyên tử Hidro ở trạng thái dừng $ M $ là
$$
	E_{3} = \dfrac{-13,6}{3^{2}} = \SI{-1,51}{eV}.
$$
		}
		
% Câu No 
		\item \mkstar{3} [3]
	
		\cauhoi
		{Mẫu nguyên tử Bo, bán kính quỹ đạo $ K $ của electron trong nguyên tử Hidro là $ r_{0} $. Khi electron chuyển từ quỹ đạo $ M $ lên quỹ đạo $ N $ thì bán kính quỹ đạo tăng thêm
		\begin{mcq}(4)
			\item $ 7 r_{0} $. 
			\item $ 16 r_{0} $.
			\item $ 27 r_{0} $. 
			\item $ 5 r_{0} $. 
		\end{mcq}
		}
	
		\loigiai
		{		\textbf{Đáp án: A.}

Bán kính quỹ đạo $ r_{M} $ và $ r_{N} $ lần lượt là
$
	\left\{
		\begin{aligned}
			& r_{M} = 3^{2} r_{0} = 9 r_{0} \\
			& r_{N} = 4^{2} r_{0} = 16 r_{0}.
		\end{aligned}
	\right.
$

Vậy khi electron chuyển từ quỹ đạo $ M $ lên quỹ đạo $ N $ thì bán kính quỹ đạo tăng thêm $ 7 r_{0} $.
		}
		
% Câu No 
		\item \mkstar{3} [4]
	
		\cauhoi
		{Trong nguyên tử Hidro, với $ r_{0} $ là bán kính Bo thì bán kính dừng của electron có thể là
		\begin{mcq}(4)
			\item $ 12 r_{0} $. 
			\item $ 26 r_{0} $.
			\item $ 9 r_{0} $. 
			\item $ 10 r_{0} $. 
		\end{mcq}
		}
	
		\loigiai
		{		\textbf{Đáp án: A.}

Bán kính quỹ đạo dừng phải có dạng $ r_{n} = n^{2} r_{0} $. Nên bán kính quỹ đạo dừng chỉ có thể là $ 9 r_{0} $. 
		}
		
% Câu No 
		\item \mkstar{3} [4]
	
		\cauhoi
		{Nguyên tử Hidro chuyển từ trạng thái $ E_{M} = \SI{-1,5}{eV} $ sang trạng thái dừng có năng lượng $ E_{L} = \SI{-3,4}{eV} $ thì nó sẽ
		\begin{mcq}
			\item hấp thụ một photon có năng lượng bằng $ \varepsilon = \SI{1,9e-19}{J} $. 
			\item phát ra một photon có năng lượng bằng $ \varepsilon = \SI{1,9}{eV} $. 
			\item hấp thụ một photon có năng lượng bằng $ \varepsilon = \SI{3,04e-19}{J} $.  
			\item phát ra một photon có năng lượng bằng $ \varepsilon = \SI{3,04}{eV} $.  
		\end{mcq}
		}
	
		\loigiai
		{		\textbf{Đáp án: B.}

Khi chuyển từ mức $ M $ sang mức $ L $ thì năng lượng mà photon phát ra cho bởi:
$$
	\varepsilon = E_{L} - E_{M} = \SI{1,9}{eV}.
$$
		}
		
% Câu No 
		\item \mkstar{3} [4]
	
		\cauhoi
		{Khi electron ở quỹ đạo dừng thứ $ n $ thì năng lượng của nguyên tử Hidro được xác định bởi công thức $ E_{n} = \xsi{\dfrac{-13,6}{n^{2}}}{eV} $ (với $ n= 1,2,3... $). Khi electron trong nguyên tử Hidro đang chuyển từ quỹ đạo dừng $ n = 3 $ về quỹ đạo dừng $ n = 1 $ thì nguyên tử phát ra bức xạ có bước sóng $ \lambda_{1} $. Khi electron chuyển từ quỹ đạo dừng $ n = 5 $ về quỹ đạo dừng $ n = 2 $ thì nguyên tử phát ra bức xạ có bước sóng $ \lambda_{2} $. Mối liên hệ giữa $ \lambda_{2} $ và $ \lambda_{1} $ là
		\begin{mcq}(4)
			\item $ \lambda_{2} = 5 \lambda_{1} $. 
			\item $  27 \lambda_{2} = 128 \lambda_{1} $.
			\item $ \lambda_{2} = 4 \lambda_{1} $. 
			\item $ 189 \lambda_{2} = 800 \lambda_{1} $. 
		\end{mcq}
		}
	
		\loigiai
		{		\textbf{Đáp án: A.}

Mối liên hệ giữa $ \lambda_{2} $ và $ \lambda_{1} $ là
$$
	\dfrac{\lambda_{2}}{\lambda_{1}} = \dfrac{E_{3} - E_{1}}{E_{5} - E_{2}} = \dfrac{800}{189}.
$$
Vậy $ 189 \lambda_{2} = 800 \lambda_{1} $.
		}

% Câu No 
		\item \mkstar{3} [10]
	
		\cauhoi
		{Trong quang phổ vạch của Hidro , bước sóng của vạch thứ nhất trong dãy Laiman ứng với sự chuyển của electron từ quỹ đạo $ L $ về quỹ đạo $ K $ là $ \SI{0,1217}{\mu m} $, vạch thứ nhất của dãy Banme ứng với sự chuyển từ $ M $ sang $ L $  là $ \SI{0,6563}{\mu m} $. Bước sóng của vạch quang phổ thứ hai trong dãy Laiman ứng với sự chuyển từ $ M $ sang $ K $ bằng 
		\begin{mcq}(4)
			\item $ \SI{0,1027}{\mu m} $.  
			\item $ \SI{0,5346}{\mu m} $. 
			\item $ \SI{0,7780}{\mu m} $.   
			\item $ \SI{0,3890}{\mu m} $.   
		\end{mcq}
		}
	
		\loigiai
		{		\textbf{Đáp án: A.}

Vạch quang phổ thứ hai trong dãy Laiman cho bởi:
	\begin{align*}
		\dfrac{hc}{\lambda_{31}} &= E_{3} - E_{1} \\ 
		&= \left( E_{3} - E_{2} \right) + \left( E_{2} - E_{1} \right) \\
		&= \dfrac{hc}{\lambda_{32}} + \dfrac{hc}{\lambda_{21}}.
	\end{align*}
Suy ra,
	\begin{equation*}
		\dfrac{1}{\lambda_{31}} = \dfrac{1}{\lambda_{32}} + \dfrac{1}{\lambda_{21}} \rightarrow \lambda_{32} = \SI{0,1027}{\mu m}.  
	\end{equation*}
		}
		
% Câu No 
		\item \mkstar{3} [12]
	
		\cauhoi
		{Trong nguyên tử Hidro, khi electron chuyển từ quỹ đạo $ P $ có năng lượng $ E_{P} $ về quỹ đạo $ L $ có năng lượng $ E_{L} $ thì phát ra photon có năng lượng $ \varepsilon $. Hệ thức nào dưới đây đúng?
		\begin{mcq}(4)
			\item $ \varepsilon = \dfrac{E_{P} - E_{L}}{hc} $.  
			\item $ \varepsilon = \dfrac{E_{P} + E_{L}}{hc} $. 
			\item $ \varepsilon = E_{P} - E_{L}$.   
			\item $ \varepsilon = E_{L} - E_{P}$.   
		\end{mcq}
		}
	
		\loigiai
		{		\textbf{Đáp án: C.}

Theo tiên đề Bo,
	\begin{equation}
		\varepsilon = E_{P} - E_{L}. 
	\end{equation}
		}
		
% Câu No 
		\item \mkstar{3} [12]
	
		\cauhoi
		{Biết năng lượng của nguyên tử Hidro ở các trạng thái dừng được tính bằng công thức: $ E_{n} = \dfrac{-13,6}{n^{2}} $, với $ n = 1,2,3... $. Nguyên tử Hidro đang ở trạng thái cơ bản, nếu hấp thụ một photon có năng lượng $ \dfrac{1632}{125} $ (eV) thì nó chuyển lên trạng thái dừng có năng lượng 
		\begin{mcq}(4)
			\item $ \xsi{-\dfrac{17}{5}}{eV} $.  
			\item $ \xsi{-\dfrac{17}{20}}{eV} $. 
			\item $ \xsi{-\dfrac{68}{45}}{eV} $.   
			\item $ \xsi{-\dfrac{68}{125}}{eV} $.   
		\end{mcq}
		}
	
		\loigiai
		{		\textbf{Đáp án: C.}

Khi ở trạng thái cơ bản và hấp thụ một photon có năng lượng $ \dfrac{1632}{125} $ (eV) thì nó chuyển lên trạng thái dừng có năng lượng có độ lớn bằng
	\begin{equation*}
		-13,6 + \dfrac{1632}{125} = \xsi{-\dfrac{68}{45}}{eV}
	\end{equation*}
		}

% Câu No 
		\item \mkstar{3} [13]
	
		\cauhoi
		{Khi electron trong nguyên tử Hidro chuyển từ quỹ đạo $ M $ về quỹ đạo $ L $, nguyên tử Hidro phát ra photon có bước sóng $ \SI{0,6563}{\mu m} $. Khi chuyển từ quỹ đạo $ N $ về quỹ đạo $ L $ nguyên tử Hidro phát ra photon có bước sóng $ \SI{0,4861}{\mu m} $. Khi chuyển từ quỹ đạo $ N $ về $ M $, nguyên tử Hidro phát ra bước sóng bằng
		\begin{mcq}(4)
			\item $ \SI{1,8744}{\mu m} $.  
			\item $ \SI{1,1424}{\mu m} $. 
			\item $ \SI{0,5335}{\mu m} $.   
			\item $ \SI{0,1702}{\mu m} $.   
		\end{mcq}
		}
	
		\loigiai
		{		\textbf{Đáp án: A.}

Khi chuyển từ quỹ đạo $ N $ về $ M $, nguyên tử Hidro phát ra bước sóng bằng
	\begin{align*}
		E_{NM} &= E_{N} - E_{M} \\
		&= \left( E_{N} - E_{L} \right) - \left( E_{M} - E_{L} \right) \\
		&= E_{NL} - E_{ML} \\
	\end{align*}
Suy ra,
	\begin{equation*}
		\dfrac{1}{\lambda_{NM}} = \dfrac{1}{\lambda_{NL}} - \dfrac{1}{\lambda_{ML}} \rightarrow \lambda_{NM} = \SI{1,8744}{\mu m}.  
	\end{equation*}
		}
		
% Câu No 
		\item \mkstar{3} [5]
	
		\cauhoi
		{Trong nguyên tử Hidro, bán kính Bo là $ r_{0} = \SI{5,3e-11}{m} $. Bán kính quỹ đạo dừng $ M $ là
		\begin{mcq}(4)
			\item $ \SI{132,5e-11}{m} $. 
			\item $ \SI{47,7e-11}{m} $. 
			\item $ \SI{21,2e-11}{m} $.   
			\item $ \SI{84,8e-11}{m} $.   
		\end{mcq}
		}
	
		\loigiai
		{		\textbf{Đáp án: B.}

Bán kính quỹ đạo dừng $ M $ là
	\begin{equation*}
		r_{M} = 3^{2} r_{0} = \SI{47,7e-11}{m}.
	\end{equation*}
		}

% Câu No 
		\item \mkstar{3} [5]
	
		\cauhoi
		{Trong quang phổ vạch của nguyên tử Hidro, gọi $ d_{1} $ là khoảng cách giữa mức $ L $ và mức $ M $, $ d_{2} $ là khoảng cách giữa mức $ M $ và $ N $. Tỉ số giữa $ d_{2} $ và $ d_{1} $ là 
		\begin{mcq}(4)
			\item $ \num{1,4} $. 
			\item $ \num{1} $. 
			\item $ \num{0,7} $.   
			\item $ \num{2,4} $.   
		\end{mcq}
		}
	
		\loigiai
		{		\textbf{Đáp án: B.}

Ta có:
$$
\left\{
	\begin{aligned}
		d_{1} &= r_{M} - r_{L} = 3^{2} r_{0} - 2^{2} r_{0} = 5 r_{0} \\
		d_{2} &= r_{N} - r_{M} = 4^{2} r_{0} - 3^{2} r_{0} = 7 r_{0}.
	\end{aligned}
\right.
$$
Vậy tỉ số giữa $ d_{2} $ và $ d_{1} $ là $ \num{0,7}. $
		}

% Câu No 
		\item \mkstar{3} [5]
	
		\cauhoi
		{Dãy Ban–me ứng với sự chuyển động electron từ quỹ đạo ở xa hạt nhân về quỹ đạo nào sau đây?
		\begin{mcq}(4)
			\item Quỹ đạo $ K $.
			\item Quỹ đạo $ L $.
			\item Quỹ đạo $ M $. 
			\item Quỹ đạo $ N $. 
		\end{mcq}
		}
	
		\loigiai
		{		\textbf{Đáp án: B.}

Dãy Ban-me ứng với sự chuyển động của elctron từ quỹ đạo ở xa hạt nhân về quỹ  đạo $ L $.
		}	
		
% Câu No 
		\item \mkstar{3} [5]
	
		\cauhoi
		{Các quỹ đạo dừng nguyên tử Hidro có tên K, P, O, L, N, M. Sắp xếp các quỹ đạo theo thứ tự bán kính giảm dần:
		\begin{mcq}(2)
			\item K, L, M, N, O, P.
			\item K, L, N, M, O, P.
			\item P, O, N, M, L, K. 
			\item P, O, M, N, L, K. 
		\end{mcq}
		}
	
		\loigiai
		{		\textbf{Đáp án: C.}

Các quỹ đạo dừng nguyên tử Hidro có tên K, P, O, L, N, M. Sắp xếp các quỹ đạo theo thứ tự bán kính giảm dần là P, O, N, M, L, K. 
		}	

% Câu No 
		\item \mkstar{3} [7]
	
		\cauhoi
		{Xét nguyên tử Hidro theo mẫu nguyên tử Bo. Gọi $ r_{0} $ là bán kính Bo. Trong các quỹ đạo dừng của electron có bán kính lần lượt là $ r_{0}, 4 r_{0}, 9 r_{0}, 16 r_{o} $, bán kính dừng khi electron chuyển động trên quỹ đạo $ L $ là

		\begin{mcq}(4)
			\item $ 9 r_{0} $.
			\item $ 16 r_{0} $.
			\item $ 4 r_{0} $. 
			\item $ r_{0} $. 
		\end{mcq}
		}
	
		\loigiai
		{		\textbf{Đáp án: C.}

bán kính dừng khi electron chuyển động trên quỹ đạo $ L $ là $ 4 r_{0} $. 
		}	

% Câu No 
		\item \mkstar{3} [7]
	
		\cauhoi
		{Khi electron trong nguyên tử Hidro chuyển từ quỹ đạo dừng có mức năng lượng $ E_{M} = \SI{-0,85}{eV} $ sang quỹ đạo dừng có năng lượng $ E_{N} = \SI{-13,60}{eV} $ thì nguyên tử phát bức xạ điện từ có bước sóng
		\begin{mcq}(4)
			\item $ \SI{0,4340}{\mu m} $.    
			\item $ \SI{0,0974}{\mu m} $. 
			\item $ \SI{0,4860}{\mu m} $.     
			\item $ \SI{0,6563}{\mu m} $.  
		\end{mcq}
		}
	
		\loigiai
		{		\textbf{Đáp án: B.}

Nguyên từ phát ra bức xạ điện từ có năng lượng là
$$
	\varepsilon = E_{M} - E_{N} = \SI{12,75}{eV}.
$$
Bức xạ điện từ có bước sóng là
$$
	\lambda = \dfrac{hc}{\varepsilon} = \SI{0,0974}{\mu m}.
$$
		}	
	
% Câu No 
		\item \mkstar{3} [7]
	
		\cauhoi
		{Theo mẫu nguyên tử Bo, nguyên tử Hidro tồn tại ở các trạng thái dừng có năng lượng tương ứng là $ E_{K} = -144E, E_{L} = -36E, E_{M} = -16E, E_{N} = -9E,... $ ($ E $ là hằng số). Khi một nguyên tử Hidro chuyển từ trạng thái dừng có năng lượng $ E_{M} $ về trạng thái dừng có năng lượng $ E_{K} $ thì phát ra một photon có năng lượng
		\begin{mcq}(4)
			\item $ 135E $.    
			\item $ 128E $. 
			\item $ 7E $.   
			\item $ 9E $.  
		\end{mcq}
		}
	
		\loigiai
		{		\textbf{Đáp án: B.}

Photon phát ra có năng lượng là
$$
	\varepsilon = E_{M} - E_{K} = 128E.
$$
		}	
		
% Câu No 
		\item \mkstar{3} [7]
	
		\cauhoi
		{ Một đám nguyên tử Hidro đang ở trạng thái kích thích mà electron chuyển động trên quỹ đạo dừng $ N $. Khi electron chuyển về các quỹ đạo dừng bên trong thì quang phổ vạch phát xạ của đám nguyên tử đó có bao nhiêu vạch?

		\begin{mcq}(4)
			\item $ 3 $.    
			\item $ 1 $. 
			\item $ 6 $.   
			\item $ 4 $.  
		\end{mcq}
		}
	
		\loigiai
		{		\textbf{Đáp án: C.}

Quỹ đạo $ N $ nên ta có $ n = 4 $. Vậy nên số vạch phát ra tối đa là $ \dfrac{n(n-1)}{2} = 6 $.
		}	

\end{enumerate}

\loigiai
{
	\begin{center}
		\textbf{BẢNG ĐÁP ÁN}
	\end{center}
	\begin{center}
		\begin{tabular}{|m{2.8em}|m{2.8em}|m{2.8em}|m{2.8em}|m{2.8em}|m{2.8em}|m{2.8em}|m{2.8em}|m{2.8em}|m{2.8em}|}
			\hline
			01.A  & 02.A  & 03.A  & 04.A   & 05.A  & 06.B  & 07.A & 08.A & 09.C & 10.C \\
			\hline
			11.A  & 12.B  & 13.B  & 14.B   & 15.C  & 16.C  & 17.B & 18.B & 19.C & \\
			\hline
			
		\end{tabular}
	\end{center}
}

\whiteBGstarEnd