\whiteBGstarBegin
\setcounter{section}{0}
\begin{enumerate}[label=\bfseries Câu \arabic*:]
	\item \mkstar{1} [3]
	
	\cauhoi
	{Chùm ánh sáng laze được ứng dụng 
		\begin{mcq}(2)
			\item trong truyền tin bằng vệ tinh. 
			\item làm nguồn phát siêu âm. 
			\item làm dao mổ trong y học. 
			\item trong đầu đọc USB. 
		\end{mcq}
	}
	
	\loigiai
	{		\textbf{Đáp án: C.}
		
		Chùm ánh sáng laze được ứng dụng làm dao mổ trong y học. 
		
	}
	
	\item \mkstar{1} [12]
	
	\cauhoi
	{ Laze không có đặc điểm nào dưới đây? 

		\begin{mcq}(2)
			\item Tính định hướng cao.   
			\item Cường độ lớn. 
			\item Chùm tia laze là chùm phân kì.
			\item Chùm tia laze có tính đơn sắc cao.  
		\end{mcq}
	}
	
	\loigiai
	{		\textbf{Đáp án: C.}
		
	Laze là chùm tia gần như song song.
		
		
	}
	
	\item \mkstar{1} [5]
	
	\cauhoi
	{Tia laze không có tính chất nào dưới đây:
		\begin{mcq} (2)
			\item Tia laze có công suất lớn. 
			\item Tia laze là chùm sáng kết hợp. 
			\item Tia laze có tính đơn sắc cao. 
			\item Tia laze có tính định hướng cao. 
		\end{mcq}
	}
	
	\loigiai
	{		\textbf{Đáp án: A.}
		
		Tia laze không phải lúc nào cũng có công suất lớn.
		
	}
	
	\item \mkstar{1} [5]
	
	\cauhoi
	{Bút laze ta dùng để chỉ bảng thuộc loại laze
		\begin{mcq}(4)
			\item rắn. 
			\item bán dẫn. 
			\item lỏng. 
			\item khí. 
		\end{mcq}
	}
	
	\loigiai
	{		\textbf{Đáp án: B.}
		
		Bút laze ta dùng để chỉ bảng thuộc loại laze bán dẫn.
		
	}
	
\end{enumerate}

\loigiai
{
	\begin{center}
		\textbf{BẢNG ĐÁP ÁN}
	\end{center}
	\begin{center}
		\begin{tabular}{|m{2.8em}|m{2.8em}|m{2.8em}|m{2.8em}|m{2.8em}|m{2.8em}|m{2.8em}|m{2.8em}|m{2.8em}|m{2.8em}|}
			\hline
			01.C  & 02.C  & 03.A  & 04.B  &  &  & & & &  \\
			\hline
			
		\end{tabular}
	\end{center}
}

\whiteBGstarEnd