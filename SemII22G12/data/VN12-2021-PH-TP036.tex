\whiteBGstarBegin
\setcounter{section}{0}
\section{Lý thuyết: Tính chất và cấu tạo hạt nhân}
\begin{enumerate}[label=\bfseries Câu \arabic*:]
	\item \mkstar{1} [1]
	
	\cauhoi
	{Hạt nhân Poloni có 84 proton và 126 nơtron, ký hiệu của hạt nhân này là
		\begin{mcq}(4)
			\item $\ce{^126_42 Po}$. 
			\item $\ce{^210_84 Po}$. 
			\item $\ce{^210_126 Po}$. 
			\item $\ce{^126_84 Po}$. 
		\end{mcq}
	}
	
	\loigiai
	{		\textbf{Đáp án: B.}
		
		Kí hiệu hạt nhân là $\ce{^A_Z X}$, hạt Poloni có $Z=84$ và $A=126+84=210$. Vậy kí hiệu của hạt nhân này là $\ce{^210_84 Po}$.
		
	}
	
	\item \mkstar{1} [2]
	
	\cauhoi
	{Bản chất lực tương tác giữa các nuclon trong hạt nhân là
		\begin{mcq}(2)
			\item lực điện từ.
			\item lực hấp dẫn.
			\item lực tương tác mạnh.
			\item lực tĩnh điện.
		\end{mcq}
	}
	
	\loigiai
	{		\textbf{Đáp án: C.}
		
		Bản chất lực tương tác giữa các nuclon trong hạt nhân là lực tương tác mạnh.
		
	}
	
	\item \mkstar{1} [2]
	
	\cauhoi
	{Hạt nhân nguyên tử $\ce{^A_Z X}$ được cấu tạo gồm
		\begin{mcq}(2)
			\item $Z$ proton và $A$ nơtron.
			\item $Z$ nơtron và $A$ proton.
			\item $Z$ proton và $(A-Z)$ nơtron.
			\item $Z$ nơtron và $(A-Z)$ proton.
		\end{mcq}
	}
	
	\loigiai
	{		\textbf{Đáp án: C.}
		
		Hạt nhân nguyên tử $\ce{^A_Z X}$ được cấu tạo gồm $Z$ proton và $(A-Z)$ nơtron.
		
	}
		\item \mkstar{1} [3]
	
	\cauhoi
	{Trong hạt nhân $\ce{^39_19 K}$ có
		\begin{mcq}(2)
			\item 20 nơtron và 19 proton.
			\item 19 nơtron và 20 proton.
			\item 19 nơtron và 39 proton.
			\item 20 nơtron và 39 proton.
		\end{mcq}
	}
	
	\loigiai
	{		\textbf{Đáp án: A.}
		
		Trong hạt nhân $\ce{^39_19 K}$ có 19 proton và $N=39-19=20$ nơtron.
		
	}
	\item \mkstar{1} [4]

\cauhoi
{Hạt nhân $\ce{^235_92 U}$ có cấu tạo gồm
	\begin{mcq}(2)
		\item 143 proton và 92 nơtron.
		\item 92 proton và 235 nơtron.
		\item 235 proton và 143 nơtron.
		\item 92 proton và 143 nơtron.
	\end{mcq}
}

\loigiai
{		\textbf{Đáp án: D.}
	
	Hạt nhân $\ce{^235_92 U}$ có cấu tạo gồm 92 proton và $N=235-92=143$ nơtron.
	
}
	\item \mkstar{1} [10]

\cauhoi
{Hạt nhân nguyên tử được cấu tạo từ
	\begin{mcq}(2)
		\item các proton.
		\item các nơtron.
		\item các nuclon.
		\item các electron.
	\end{mcq}
}

\loigiai
{		\textbf{Đáp án: C.}
	
	Hạt nhân nguyên tử được cấu tạo từ các nuclon. Có hai loại nuclon là proton và nowtron.
	
}
\item \mkstar{1} [10]

\cauhoi
{Hạt nhân Triti có
	\begin{mcq}(2)
		\item 3 nơtron và 1 proton.
		\item 3 nuclon, trong đó có 1 nơtron.
		\item 3 nuclon, trong đó có 1 proton.
		\item 3 proton và 1 nơtron.
	\end{mcq}
}

\loigiai
{		\textbf{Đáp án: C.}
	
	Hạt nhân Triti $\ce{^3_1 H}$ có 1 proton, 2 nơtron và tổng cộng là 3 nuclon.
	
}
\item \mkstar{1} [12]

\cauhoi
{Hạt nhân $\ce{^27_13 Al}$ có số proton là
	\begin{mcq}(4)
		\item 40.
		\item 14.
		\item 27.
		\item 13.
	\end{mcq}
}

\loigiai
{		\textbf{Đáp án: D.}
	
	Hạt nhân $\ce{^27_13 Al}$ có số proton là 13.
	
}
\item \mkstar{1} [13]

\cauhoi
{Khẳng định nào sau đây đúng?
	\begin{mcq}
		\item Bán kính hạt nhân nguyên tử gần bằng bán kính của nguyên tử đó.
		\item Khối lượng hạt nhân nguyên tử gần bằng khối lượng nguyên tử đó.
		\item Điện tích nguyên tử bằng điện tích hạt nhân.
		\item Có hai loại nuclon và proton và electron.
	\end{mcq}
}

\loigiai
{		\textbf{Đáp án: B.}
	
	Vì khối lượng của electron là rất nhỏ, nên khối lượng hạt nhân nguyên tử gần bằng khối lượng nguyên tử đó.
	
}
\item \mkstar{1} [13]

\cauhoi
{Đồng vị là các nguyên tử mà hạt nhân của chúng có
	\begin{mcq}
		\item số proton bằng nhau, số nơtron khác nhau.
		\item số khối $A$ bằng nhau.
		\item số nơtron bằng nhau, số proton khác nhau.
		\item số nơtron bằng nhau, khối lượng khác nhau.
	\end{mcq}
}

\loigiai
{		\textbf{Đáp án: A.}
	
	Đồng vị là các nguyên tử mà hạt nhân của chúng có cùng số proton, khác nhau số nơtron.
	
}
	\item \mkstar{1} [5]

\cauhoi
{Hạt nhân nguyên tử có 82 proton và 125 nơtron. Hạt nhân nguyên tử này có kí hiệu
	\begin{mcq}(4)
		\item $\ce{^125_82 Pb}$.
		\item $\ce{^82_125 Pb}$.
		\item $\ce{^82_207 Pb}$.
		\item $\ce{^207_82 Pb}$.
	\end{mcq}
}

\loigiai
{		\textbf{Đáp án: D.}
	
	Hạt nhân có $Z=82$ và $A=125+82=207$. Vậy hạt nhân này có kí hiệu $\ce{^207_82 Pb}$.
	
}
	\item \mkstar{1} [5]

\cauhoi
{Các đồng vị hạt nhân của cùng một nguyên tố có cùng
	\begin{mcq}(4)
		\item khối lượng.
		\item nuclon.
		\item số nơtron.
		\item số proton.
	\end{mcq}
}

\loigiai
{		\textbf{Đáp án: D.}
	
	Các đồng vị hạt nhân của cùng một nguyên tố có cùng số proton.
	
}
\item \mkstar{1} [7]

\cauhoi
{Hạt nhân $\ce{^60_27 Co}$ có cấu tạo gồm
	\begin{mcq}(2)
		\item 33 proton và 27 nơtron.
		\item 33 proton và 27 nuclon.
		\item 27 proton và 60 nơtron.
		\item 27 proton và 33 nơtron.
	\end{mcq}
}

\loigiai
{		\textbf{Đáp án: D.}
	
	Hạt nhân $\ce{^60_27 Co}$ có cấu tạo gồm 27 proton và $N=60-27=33$ nơtron.
	
}
\item \mkstar{1} [7]

\cauhoi
{Số nuclon có trong hạt nhân $\ce{^23_11 Na}$ là
	\begin{mcq}(4)
		\item 11.
		\item 12.
		\item 23.
		\item 34.
	\end{mcq}
}

\loigiai
{		\textbf{Đáp án: C.}
	
	Số nuclon có trong hạt nhân $\ce{^23_11 Na}$ là 23.
	
}
\item \mkstar{1} [7]

\cauhoi
{Phát biểu nào sau đây đúng? Đồng vị là các nguyên tử mà
	\begin{mcq}
		\item hạt nhân của chúng có số khối $A$ bằng nhau.
		\item hạt nhân của chúng có số nơtron bằng nhau, số proton khác nhau.
		\item hạt nhân của chúng có số proton bằng nhau, số nơtron khác nhau.
		\item hạt nhân của chúng có khối lượng bằng nhau.
	\end{mcq}
}

\loigiai
{		\textbf{Đáp án: C.}
	
	Đồng vị là các nguyên tử mà hạt nhân của chúng có số proton bằng nhau, số nơtron khác nhau.
	
}
\item \mkstar{1} [9]

\cauhoi
{Trong hạt nhân nguyên tử $\ce{^210_84 Po}$ có
	\begin{mcq}(2)
		\item 210 proton và 84 nơtron.
		\item 84 proton và 210 nơtron.
		\item 126 proton và 84 nơtron.
		\item 84 proton và 126 nơtron.
	\end{mcq}
}

\loigiai
{		\textbf{Đáp án: D.}
	
	Trong hạt nhân nguyên tử $\ce{^210_84 Po}$ có 84 proton và $N=210-84=126$ nơtron.
	
}
	\item \mkstar{2} [4]

\cauhoi
{Biết điện tích của electron là $\SI{-1.6e-19}{C}$. Điện tích của hạt nhân nguyên tử $\ce{^4_2 He}$ là
	\begin{mcq}(4)
		\item $\SI{6.4e-19}{C}$.
		\item $\SI{-6.4e-19}{C}$.
		\item $\SI{3.2e-19}{C}$.
		\item $\SI{-3.2e-19}{C}$.
	\end{mcq}
}

\loigiai
{		\textbf{Đáp án: C.}
	
	Hạt nhân nguyên tử không chứa electron và chỉ chứa hạt proton mang điện. Hạt nhân $\ce{^4_2 He}$ có 2 proton nên điện tích hạt nhân là $q=2|e|=\SI{3.2e-19}{C}$.
	
}
	\item \mkstar{2} [5]

\cauhoi
{Cho hạt nhân $\ce{^A_Z X}$. Gọi số Avogadro là $N_\text A$. Số hạt nhân X có trong $m$ (gam) bằng
	\begin{mcq}(4)
		\item $\dfrac{m N_\text A}{A}$.
		\item $\dfrac{A N_\text A}{m}$.
		\item $mN_\text A$.
		\item $mAN_\text A$.
	\end{mcq}
}

\loigiai
{		\textbf{Đáp án: A.}
	
	Số hạt nhân X có trong $m$ (gam) bằng
	$$\dfrac{m}{M} N_\text{A} = \dfrac{m}{A} N_\text A$$
	
}
	\item \mkstar{3} [3]

\cauhoi
{Biết $N_\text A = \SI{6.02e23}{mol^{-1}}$. Trong $\SI{47.6}{g}$ $\ce{^238_92 U}$ có số nơtron là 
	\begin{mcq}(4)
		\item $\SI{1.758e25}{}$.
		\item $\SI{1.204e23}{}$.
		\item $\SI{2.866e25}{}$.
		\item $\SI{1.107e25}{}$.
	\end{mcq}
}

\loigiai
{		\textbf{Đáp án: A.}
	
	Trong 1 hạt nhân $\ce{^238_92 U}$ có số nơtron là
	$$N=A-Z=146$$
	
	Trong $\SI{47.6}{g}$ $\ce{^238_92 U}$ có số nơtron là
	$$\dfrac{m}{M} N_\text{A} \cdot 146 = \dfrac{m}{A} N_\text{A} \cdot 146 = \SI{1.758e25}{}$$
	
}



	\item \mkstar{3} [4]

\cauhoi
{Số nơtron có trong $\SI{28}{g}$ $\ce{^14_6 C}$ là bao nhiêu? Coi khối lượng của hạt nhân bằng số khối.
	\begin{mcq}(2)
		\item $\SI{9.6352e24}{}$ hạt.
		\item $\SI{5.78112e24}{}$ hạt.
		\item $\SI{1.6856e25}{}$ hạt.
		\item $\SI{8.0836e24}{}$ hạt.
	\end{mcq}
}

\loigiai
{		\textbf{Đáp án: A.}
	
	Trong 1 hạt nhân $\ce{^14_6 C}$ có số nơtron là
	$$N=A-Z=8$$
	
	Trong $\SI{28}{g}$ $\ce{^14_6 C}$ có số nơtron là
	$$\dfrac{m}{M} N_\text{A} \cdot 8 = \dfrac{m}{A} N_\text{A} \cdot 8 = \SI{9.6352e24}{}$$
	
}



	\item \mkstar{3} [5]

\cauhoi
{Biết số Avogadro là $\SI{6.02e23}{mol^{-1}}$, khối lượng mol của $\ce{^238_92 U}$ là $\SI{238}{g/mol}$. Số nơtron trong $\SI{238}{g}$ $\ce{^238_92 U}$ là
	\begin{mcq}(4)
		\item $\SI{8.8e25}{}$.
		\item $\SI{1.2e25}{}$.
		\item $\SI{4.4e25}{}$.
		\item $\SI{2.2e25}{}$.
	\end{mcq}
}

\loigiai
{		\textbf{Đáp án: A.}
	
		Trong 1 hạt nhân $\ce{^238_92 U}$ có số nơtron là
	$$N=A-Z=146$$
	
	Trong $\SI{238}{g}$ $\ce{^238_92 U}$ có số nơtron là
	$$\dfrac{m}{M} N_\text{A} \cdot 146 = \dfrac{m}{A} N_\text{A} \cdot 146 = \SI{8.8e25}{}$$
	
}


\end{enumerate}

\loigiai
{
	\begin{center}
		\textbf{BẢNG ĐÁP ÁN}
	\end{center}
	\begin{center}
		\begin{tabular}{|m{2.8em}|m{2.8em}|m{2.8em}|m{2.8em}|m{2.8em}|m{2.8em}|m{2.8em}|m{2.8em}|m{2.8em}|m{2.8em}|}
			\hline
			1.B  & 2.C  & 3.C  & 4.A  & 5.D  & 6.C  & 7.C & 8.D & 9.B & 10.A \\
			\hline
			11.D  & 12.D  & 13.D  & 14.C  & 15.C  & 16.D  & 17.C & 18.A & 19.A & 20.A \\
			\hline
			21.A  &  &  & & & & & & & \\
			\hline
			
		\end{tabular}
	\end{center}
}

\section{Dạng bài: Độ hụt khối, khối lượng và năng lượng của hạt nhân}
\begin{enumerate}[label=\bfseries Câu \arabic*:]
	\item \mkstar{1} [12]
	
	\cauhoi
	{Các hạt nhân có cùng độ hụt khối thì
		\begin{mcq}
			\item năng lượng liên kết như nhau. 
			\item cùng năng lượng liên kết riêng.
			\item có năng lượng liên kết riêng lớn nếu số khối lớn.
			\item có cùng khối lượng. 
		\end{mcq}
	}
	
	\loigiai
	{		\textbf{Đáp án: A.}
		
		Các hạt nhân có cùng độ hụt khối thì năng lượng liên kết như nhau:
		$$E_\text{lk} = \Delta m c^2$$
		
	}
	
	\item \mkstar{1} [5]
	
	\cauhoi
	{Gọi $m_p$, $m_n$, $m_X$ lần lượt là khối lượng của hạt proton, nơtron và hạt nhân $\ce{^A_Z X}$. Độ hụt khối khi các nuclon ghép lại tạo thành hạt nhân $\ce{^A_Z X}$ là $\Delta m$ được tính bằng biểu thức
		\begin{mcq}(2)
			\item $\Delta m = Z m_p + (A-Z) m_n - m_X$. 
			\item $\Delta m = Z m_p + (A-Z) m_n + Am_X$. 
			\item $\Delta m = Z m_p + (A-Z) m_n + m_X$.
			\item $\Delta m = Z m_p + (A-Z) m_n - Am_X$.
		\end{mcq}
	}
	
	\loigiai
	{		\textbf{Đáp án: A.}
		
		Độ hụt khối khi các nuclon ghép lại tạo thành hạt nhân $\ce{^A_Z X}$ là $\Delta m$ được tính bằng biểu thức $\Delta m = Z m_p + (A-Z) m_n - m_X$. 
		
	}
	\item \mkstar{1} [7]

\cauhoi
{Đơn vị nào dưới đây \textbf{không phải} đơn vị khối lượng hạt nhân?
	\begin{mcq}(4)
		\item $\SI{}{kg}$. 
		\item $\SI{}{MeV/c^2}$.
		\item $\SI{}{u}$.
		\item $\SI{}{MeV/c}$.
	\end{mcq}
}

\loigiai
{		\textbf{Đáp án: D.}
	
	Đơn vị khối lượng hạt nhân gồm $\SI{}{kg}$, $\SI{}{MeV/c^2}$, $\SI{}{u}$. Đơn vị $\SI{}{MeV/c}$ không phải là đơn vị khối lượng hạt nhân.
	
}
	
	\item \mkstar{3} [5]
	
	\cauhoi
	{Một hạt nhân $\ce{^5_3 Li}$ có năng lượng liên kết bằng $\SI{26.3}{MeV}$. Biết khối lượng proton $m_p=\SI{1.0073}{u}$, khối lượng nơtron $m_n=\SI{1.0087}{u}$, $\SI{1}{u}=\SI{931}{MeV/c^2}$. Khối lượng nghỉ của hạt nhân $\ce{^5_3 Li}$ bằng
		\begin{mcq}(4)
			\item $\SI{5.0111}{u}$.
			\item $\SI{5.0675}{u}$.
			\item $\SI{4.7179}{u}$.
			\item $\SI{4.6916}{u}$.
		\end{mcq}
	}
	
	\loigiai
	{		\textbf{Đáp án: A.}
		
		Độ hụt khối của hạt nhân $\ce{^5_3 Li}$:
		$$E_\text{lk} = \Delta m c^2 \Rightarrow \Delta m = \dfrac{\SI{26.3}{MeV}}{\SI{931}{MeV/c^2}} = \SI{0.02825}{u}$$
		
		Khối lượng nghỉ của hạt nhân:
		$$\Delta m = Z m_p + (A-Z)m_n - m_X \Rightarrow m_X = \SI{5.0111}{u}$$
		
	}
	
	\item \mkstar{3} [1]
	
	\cauhoi
	{Theo thuyết tương đối, một vật có khối lượng nghỉ $\SI{100}{g}$ chuyển động với tốc độ $\SI{2.4e8}{m/s}$ thì có động năng bằng
		\begin{mcq}(4)
			\item $\SI{5.76e15}{J}$.
			\item $\SI{2.88e15}{J}$.
			\item $\SI{6e15}{J}$.
			\item $\SI{15e15}{J}$.
		\end{mcq}
	}
	
	\loigiai
	{		\textbf{Đáp án: C.}
		
		Động năng bằng hiệu giữa $E$ và $E_0$:
		$$W_\text{đ} = E-E_0 = mc^2 - m_0 c^2 = m_0 c^2 \left(\dfrac{1}{\sqrt{1-\dfrac{v^2}{c^2}}-1}\right) = \SI{6e15}{J}$$
		
	}
	

	
\end{enumerate}

\loigiai
{
	\begin{center}
		\textbf{BẢNG ĐÁP ÁN}
	\end{center}
	\begin{center}
		\begin{tabular}{|m{2.8em}|m{2.8em}|m{2.8em}|m{2.8em}|m{2.8em}|m{2.8em}|m{2.8em}|m{2.8em}|m{2.8em}|m{2.8em}|}
			\hline
			1.A  & 2.A  & 3.D  & 4.A  & 5.C  & & & & & \\
			\hline
			
		\end{tabular}
	\end{center}
}

\whiteBGstarEnd