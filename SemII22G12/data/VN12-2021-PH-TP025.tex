\whiteBGstarBegin
\setcounter{section}{0}
\section{Lý thuyết: Giao thoa ánh sáng và điều kiện xảy ra giao thoa ánh sáng}
\begin{enumerate}[label=\bfseries Câu \arabic*:]

%========================================
    \item \mkstar{1} [5]
    
	\cauhoi
	{Trong thí nghiệm Y-âng, tại vị trí vân tối thì
		\begin{mcq}(1)
			\item Độ lệch pha của hai sóng từ hai nguồn kết hợp thỏa mãn $\delta \varphi = (2k+1) \dfrac{\pi}{2}$ với $k \in Z$. 
			\item Hai sóng đến từ hai nguồn kết hợp vuông pha với nhau. 
			\item Hiệu quang trình từ hai nguồn kết hợp thỏa mãn $d_{2} - d_{1} = (2k+1)\dfrac{\lambda}{2}$ với $k \in Z$. 
			\item Hiệu quang trình từ hai nguồn kết hợp thỏa mãn $d_{2} - d_{1} = (2k+1)\lambda$ với $k \in Z$.  
		\end{mcq}
	}
	
	\loigiai
	{		\textbf{Đáp án:  C.}
		
Trong thí nghiệm Y-âng, tại vị trí vân tối thì hiệu quang trình từ hai nguồn kết hợp thỏa mãn $d_{2} - d_{1} = (2k+1)\dfrac{\lambda}{2}$ với $k \in Z$. 
		
	}

%========================================
    \item \mkstar{1} [5]
    
	\cauhoi
	{Hiện tượng giao thoa chứng tỏ rằng 
		\begin{mcq}(2)
			\item ánh sáng là sóng ngang. 
			\item ánh sáng là sóng điện từ. 
			\item ánh sáng có bản chất sóng. 
			\item ánh sáng có thể bị tán sắc. 
		\end{mcq}
	}
	
	\loigiai
	{		\textbf{Đáp án: C.}
		
Hiện tượng giao thoa chứng tỏ rằng ánh sáng có bản chất sóng.
		
	}

%========================================
    \item \mkstar{1} [5]
    
	\cauhoi
	{Hiện tượng giao thoa ánh sáng chỉ quan sát được khi hai nguồn ánh sáng là hai nguồn 
		\begin{mcq}(2)
			\item cùng màu sắc. 
			\item đơn sắc. 
			\item cùng cường độ. 
			\item kết hợp. 
		\end{mcq}
	}
	
	\loigiai
	{		\textbf{Đáp án: D.}
		
Hiện tượng giao thoa ánh sáng chỉ quan sát được khi hai nguồn ánh sáng là hai nguồn kết hợp.
		
	}

%========================================
    \item \mkstar{1} [13]
    
	\cauhoi
	{Trong thí nghiệm Y-âng về giao thoa ánh sáng, với k là số nguyên. Công thức dùng để xác định vị trí vân sáng trên màn quan sát là
		\begin{mcq}(2)
			\item $x = \dfrac{a}{D} k \lambda$. 
			\item $x = \dfrac{D}{2a} \lambda$. 
			\item $x= k \dfrac{\lambda D}{a}$. 
			\item $x = \dfrac{D}{a} (k+0,5) \lambda$. 
		\end{mcq}
	}
	
	\loigiai
	{		\textbf{Đáp án: C.}
		
Công thức xác định vị trí vân sáng trong giao thoa khe Y-âng là $x= k \dfrac{\lambda D}{a}$.
		
	}

%========================================
    \item \mkstar{1} [10]
    
	\cauhoi
	{Hiện tượng giao thoa ánh sáng là bằng chứng thực nghiệm chứng tỏ ánh sáng
		\begin{mcq}(2)
			\item là sóng siêu âm. 
			\item có tính chất sóng. 
			\item là sóng dọc. 
			\item có tính chất hạt. 
		\end{mcq}
	}
	
	\loigiai
	{		\textbf{Đáp án: B.}
		
Hiện tượng giao thoa ánh sáng là bằng chứng thực nghiệm chứng tỏ ánh sáng có tính chất sóng.
		
	}

%========================================
    \item \mkstar{1} [3]
    
	\cauhoi
	{Trong thí nghiệm giao thoa ánh sáng khe Young, điều kiện để có hiện tượng giao thoa ánh sáng trên màn quan sát thì sóng ánh sáng do hai nguồn thứ cấp $S_{1}$ và $S_{2}$ phải
		\begin{mcq}(1)
			\item cùng phương, cùng biên độ nhưng có hiệu số pha thay đổi theo thời gian. 
			\item cùng tần số nhưng có hiệu số pha thay đổi theo thời gian. 
			\item cùng biên độ nhưng khác tần số dao động. 
			\item cùng phương, cùng tần số và có hiệu số pha không thay đổi theo thời gian. 
		\end{mcq}
	}
	
	\loigiai
	{		\textbf{Đáp án: D.}

Điều kiện để xảy ra giao thoa là hai nguồn thứ cấp $S_{1}$ và $S_{2}$ phải cùng phương, cùng tần số và có hiệu số pha không thay đổi theo thời gian.
		
	}

%========================================
    \item \mkstar{1} [1]
    
	\cauhoi
	{Hiện tượng nào sau đây liên quan đến tính chất sóng của ánh sáng?
		\begin{mcq}(2)
			\item Hiện tượng quang dẫn. 
			\item Hiện tượng quang phát quang. 
			\item Hiện tượng quang điện trong. 
			\item Hiện tượng giao thoa ánh sáng. 
		\end{mcq}
	}
	
	\loigiai
	{		\textbf{Đáp án: D.}
		
Hiện tượng giao thoa là đặc trưng cho tính chất sóng của ánh sáng.
		
	}
	
%========================================
    \item \mkstar{1} [2]
    
	\cauhoi
	{Điều khẳng định nào sau đây là \textbf{sai} khi nói về bản chất sóng của ánh sáng?
		\begin{mcq}(1)
			\item Ánh sáng có lưỡng tính sóng - hạt. 
			\item Khi tính chất hạt rõ nét, ta dễ quan sát hiện tượng giao thoa ánh sáng. 
			\item Khi ánh sáng có bước sóng càng ngắn, thì khả năng đâm xuyên càng mạnh. 
			\item Khi ánh sáng có bước sóng càng ngắn, thì tính chất hạt càng thể hiện rõ, tính chất sóng càng ít thể hiện. 
		\end{mcq}
	}
	
	\loigiai
	{		\textbf{Đáp án: B.}
		
Khi tính chất hạt rõ nét, ta khó quan sát hiện tượng giao thoa ánh sáng.
		
	}
%========================================
    \item \mkstar{1} [9]
    
	\cauhoi
	{Thực hiện thí nghiệm Y-âng về giao thoa ánh sáng đơn sắc màu đỏ ta quan sát được hệ vân giao thoa trên màn. Nếu thay ánh sáng đơn sắc màu đỏ bằng ánh sáng đơn sắc màu lục và các điều khác của thí nghiệm được giữ nguyên thì 
		\begin{mcq}(2)
			\item khoảng vân tăng lên. 
			\item khoảng vân không thay đổi. 
			\item vị trí vân trung tâm thay đổi. 
			\item khoảng vân giảm xuống. 
		\end{mcq}
	}
	
	\loigiai
	{		\textbf{Đáp án: D.}
		
Khi thay ánh sáng đơn sắc màu đỏ bằng ánh sáng đơn sắc màu lục thì bước sóng giảm xuống. Vì khoảng vân tỉ lệ với bước sóng nên khoảng vân cũng giảm xuống.
		
	}	

\end{enumerate}

\loigiai
{
	\begin{center}
		\textbf{BẢNG ĐÁP ÁN}
	\end{center}
	\begin{center}
		\begin{tabular}{|m{2.8em}|m{2.8em}|m{2.8em}|m{2.8em}|m{2.8em}|m{2.8em}|m{2.8em}|m{2.8em}|m{2.8em}|m{2.8em}|}
			\hline
			01.C  & 02.C  & 03.D  & 04.C  & 05.B  & 06.D  & 07.D & 08.B & 09.D & \\
			\hline
			
		\end{tabular}
	\end{center}
}

\section{Dạng bài: Giao thoa ánh sáng đơn sắc}
\begin{enumerate}[label=\bfseries Câu \arabic*:]
	
%========================================
\item \mkstar{3} [13]

\cauhoi
{Trong thí nghiệm Y-âng về giao thoa ánh sáng. Khi tiến hành trong không khí, người ta đo được khoảng vân là $i = \SI{2}{mm}$. Tiến hành thí nghiệm trong nước, nước có chiết suất tuyệt đối là $n = \dfrac{4}{3}$ thì khoảng vân đo được trong nước là
	\begin{mcq}(4)
		\item $\SI{2,5}{mm}$. 
		\item $\SI{2}{mm}$. 
		\item $\SI{1,5}{mm}$. 
		\item $\SI{1,25}{mm}$. 
	\end{mcq}
}

\loigiai
{		\textbf{Đáp án: C.}
	
	Gọi $i'$ là khoảng vân đo được trong môi trường nước. Ta có
	$$
	i' = \dfrac{i}{n} = \SI{1,5}{mm}.
	$$
}

%========================================
\item \mkstar{3} [9]

\cauhoi
{Trong thí nghiệm Y-âng về giao thoa ánh sáng đơn sắc, khoảng cách giữa hai khe là $\SI{1}{mm}$, khoảng cách từ mặt phẳng chứa hai khe đến màn là $\SI{2}{m}$. Trong hệ vân trên màn, vân sáng bậc 3 cách vân trung tâm là $\SI{2,4}{mm}$. Bước sóng của ánh sáng đơn sắc dùng trong thí nghiệm là 
	\begin{mcq}(4)
		\item $\SI{0,5}{\mu m}$. 
		\item $\SI{0,7}{\mu m}$. 
		\item $\SI{0,4}{\mu m}$. 
		\item $\SI{0,6}{\mu m}$. 
	\end{mcq}
}

\loigiai
{		\textbf{Đáp án: C.}
	
	Vân sáng bậc 3 cách vân trung tâm $3i$ nên
	$$
	3i = \SI{2,4}{mm} \rightarrow i = \SI{0,8}{mm}.
	$$
	Ta có
	$$
	i = \dfrac{\lambda D}{a} \rightarrow \lambda = \SI{0,4}{\mu m}.
	$$
}
	
%========================================
\item \mkstar{3} [9]

\cauhoi
{Thực hiện giao thoa ánh sáng theo khe Y-âng với $a = \SI{2}{mm}$, $D= \SI{1}{m}$, nguồn S phát ra ánh sáng đơn sắc có bước sóng $\lambda = \SI{0,5}{\mu m}$. Khoảng cách từ vân sáng bậc 5 đến vân tối bậc 7 ở hai bên vân sáng trung tâm là
	\begin{mcq}(4)
		\item $\SI{2,875}{mm}$. 
		\item $\SI{11,5}{mm}$. 
		\item $\SI{2,6}{mm}$. 
		\item $\SI{12,5}{mm}$. 
	\end{mcq}
}

\loigiai
{		\textbf{Đáp án: A.}
	
	Khoảng cách giữa vân sáng bậc 5 và vân tối bậc 7 ở hai phía so với vân trung tâm là
	$$
	5i + 6,5i = 11,5i = 11,5 \dfrac{\lambda D}{a} = \SI{2,875}{mm}.
	$$
}
	
%========================================
\item \mkstar{3} [9]

\cauhoi
{Trong thí nghiệm Y-âng về giao thoa ánh sáng đơn sắc, người ta đo được khoảng cách từ vân sáng bậc 2 đến vân sáng bậc 5 cùng một phía so với vân trung tâm là $\SI{3}{mm}$. Số vân sáng quan sát được trên MN đối xứng với nhau qua vân sáng trung tâm có bề rộng là $\SI{13}{mm}$ là
	\begin{mcq}(4)
		\item 9 vân. 
		\item 13 vân. 
		\item 15 vân. 
		\item 11 vân. 
	\end{mcq}
}

\loigiai
{		\textbf{Đáp án: B.}
	
	Khoảng cách từ vân sáng bậc 2 đến vân sáng bậc 5 là $3i$ nên
	$$
	3i = \SI{3}{mm} \rightarrow i = \SI{1}{mm}.
	$$
	Ta có:
	$$
	\dfrac{L}{2i} = \num{6,5}.
	$$
	Số vân sáng trên đoạn MN là 13 vân.
}
	
%========================================
\item \mkstar{3} [23]

\cauhoi
{ Trong thí nghiệm giao thoa khe Y-âng, khoảng cách giữa hai vân sáng cạnh nhau là
	
	\begin{mcq}(4)
		\item $\dfrac{\lambda}{aD}$. 
		\item $\dfrac{ax}{D}$. 
		\item $\dfrac{\lambda a}{D}$. 
		\item $\dfrac{\lambda D}{a}$. 
	\end{mcq}
}

\loigiai
{		\textbf{Đáp án: D.}
	
	Khoảng cách giữa hai vân sáng cạnh nhau là khoảng vân $i = \dfrac{\lambda D}{a}$.		
}
	
%========================================
\item \mkstar{3} [13]

\cauhoi
{Trong một thí nghiệm Y-âng về giao thoa ánh sáng, bước sóng ánh sáng đơn sắc là $\lambda$ (m). Khoảng cách giữa hai khe hẹp là $a$ (m. Khoảng cách từ mặt phẳng phân cách chứa hai khe đến màn là $D$ (m). Vị trí vân tối có tọa độ $x_{k}$ là
	\begin{mcq}(2)
		\item $x_{k} = (2k+1)\dfrac{\lambda D}{a}$. 
		\item $x_{k} = k \dfrac{\lambda D}{a}$. 
		\item $x_{k} = (2k+1) \dfrac{\lambda D}{2a}$. 
		\item $x_{k} = k \dfrac{\lambda D}{2a}$. 
	\end{mcq}
}

\loigiai
{		\textbf{Đáp án: C.}
	
	Vị trí vân tối là 
	$$
	x_{k} = (2k+1) \dfrac{\lambda D}{2a}.
	$$
}
	
%========================================
\item \mkstar{3} [7]

\cauhoi
{Trong thí nghiệm Y-âng về giao thoa ánh sáng, hai khe được chiếu bằng ánh sáng đơn sắc có bước sóng $\SI{0,6}{\mu m}$, khoảng cách giữa hai khe hẹp là $\SI{1}{mm}$, khoảng cách từ mặt phẳng chứa hai khe đến màn quan sát là $\SI{2,5}{m}$. Khoảng vân giao thoa trên màn là
	
	\begin{mcq}(4)
		\item $\SI{1,5}{mm}$.
		\item $\SI{1,0}{mm}$. 	
		\item $\SI{2,0}{mm}$.  	
		\item $\SI{0,5}{mm}$. 
	\end{mcq}
}

\loigiai
{		\textbf{Đáp án: A.}
	
	Khoảng vân trên màn cho bởi
	$$
	i = \dfrac{\lambda D}{a} = \SI{1,5}{mm}.
	$$
}

	
%========================================
\item \mkstar{3} [5]

\cauhoi
{Trong thí nghiệm giao thoa khe Y-âng, nguồn sóng có bước sóng là $\SI{380}{nm}$.  Khoảng cách giữa hai khe hẹp là $\SI{2}{mm}$. Khoảng cách giữa hai khe đến màn là $\SI{2}{m}$. Khoảng vân là
	\begin{mcq}(4)
		\item $\SI{3}{mm}$. 
		\item $\SI{0,38}{mm}$. 
		\item $\SI{0,62}{mm}$. 
		\item $\SI{0,54}{mm}$. 
	\end{mcq}
}

\loigiai
{		\textbf{Đáp án: B.}
	
	Khoảng vân của hệ cho bởi
	$$
	i = \dfrac{\lambda D}{a} = \SI{0,38}{mm}.
	$$
}
	
%========================================
\item \mkstar{3} [5]

\cauhoi
{Khi thực hiện giao thoa với ánh sáng đơn sắc, nếu hai khe Y-âng cách nhau $\SI{1,2}{mm}$ thì khoảng vân là $i = \SI{1,21}{mm}$. Nếu khoảng cách giữa hai khe giảm đi $\SI{0,2}{mm}$ thì khoảng vân sẽ 
	\begin{mcq}(2)
		\item giảm đi $\SI{0,24}{mm}$. 
		\item giảm đi $\SI{0,11}{mm}$. 
		\item tăng thêm $\SI{0,24}{mm}$. 
		\item tăng thêm $\SI{0,11}{mm}$. 
	\end{mcq}
}

\loigiai
{		\textbf{Đáp án: C.}
	
	Ta có:
	$$
	\dfrac{i'}{i} = \dfrac{a}{a'} = \dfrac{a}{a- \Delta a} \rightarrow i' = \SI{1,452}{mm}.
	$$
	Như vậy khoảng vân sẽ tăng thêm $\SI{0,24}{mm}$.
}
	
%========================================
\item \mkstar{3} [13]

\cauhoi
{Trong thí nghiệm Y-âng về giao thoa ánh sáng có $\lambda = \SI{0,75}{\mu m}$, $a = \SI{0,6}{mm}$ và $D = \SI{1,2}{m}$. Tại điểm M trên màn quan sát cách vân trung tâm $\SI{5,25}{mm}$ có vân
	\begin{mcq}(2)
		\item sáng bậc 3. 
		\item tối thứ 3. 
		\item tối thứ 4. 
		\item sáng bậc 4. 
	\end{mcq}
}

\loigiai
{		\textbf{Đáp án: C.}
	
	Khoảng vân $i$ là
	$$
	i = \dfrac{\lambda D}{a} = \SI{1,5}{mm}. 
	$$
	Ta có:
	$$
	\dfrac{x}{i} = 3,5.
	$$
	Vậy tại $x = \SI{5,25}{mm}$ là vân tối thứ 4.
}
	
%========================================
\item \mkstar{3} [13]

\cauhoi
{Trong thí nghiệm Y-âng về giao thoa ánh sáng có $a = \SI{1}{mm}$ và $D = \SI{1}{m}$. Trên màn quan sát ta thấy vân sáng bậc 5 cách vân sáng trung tâm là $\SI{3,0}{mm}$. Bước sóng ánh sáng dùng trong thí nghiệm là
	\begin{mcq}(4)
		\item $\lambda = \SI{0,70}{\mu m}$. 
		\item $\lambda = \SI{0,50}{\mu m}$. 
		\item $\lambda = \SI{0,60}{\mu m}$. 
		\item $\lambda = \SI{0,53}{\mu m}$. 
	\end{mcq}
}

\loigiai
{		\textbf{Đáp án: C.}
	
	Khoảng cách từ vân sáng bậc 5 đến vân sáng trung tâm là
	$$
	\Delta d = 5i \rightarrow  i = \SI{0,6}{mm}.
	$$
	Khoảng vân cho bởi
	$$
	i = \dfrac{\lambda D}{a} \rightarrow \lambda = \SI{0,60}{\mu m}.
	$$
}

%========================================
\item \mkstar{3} [13]

\cauhoi
{Trong thí nghiệm Y-âng về giao thoa ánh sáng với khoảng vân là $i$, khoảng cách từ vân sáng bậc 3 đến vân sáng bậc 7 ở cùng một phía so với vân sáng trung tâm là 
	\begin{mcq}(4)
		\item $3i$. 
		\item $4i$. 
		\item $7i$. 
		\item $5i$. 
	\end{mcq}
}

\loigiai
{		\textbf{Đáp án: B.}
	
	Khoảng cách giữa vân sáng bậc 3 và vân sáng bậc 7 cho bởi
	$$
	\Delta d = 7i - 3i = 4i.
	$$
}
	
%========================================
\item \mkstar{3} [13]

\cauhoi
{Trong thí nghiệm Y-âng về giao thoa ánh sáng có $\lambda = \SI{0,5}{\mu m}$, $a = \SI{1,2}{mm}$ và $D = \SI{2,5}{m}$. Vân tối thứ ba cách vân trung tâm một khoảng bằng
	\begin{mcq}(4)
		\item $\SI{2,1}{mm}$. 
		\item $\SI{3,1}{mm}$. 
		\item $\SI{2,6}{mm}$. 
		\item $\SI{3,6}{mm}$. 
	\end{mcq}
}

\loigiai
{		\textbf{Đáp án: C.}
	
	Vân tối thứ ba cách vân trung tâm một đoạn là
	$$
	x_{t_{3}} = 2,5i = 2,5 \dfrac{\lambda D}{a} = \SI{2,6}{mm}.
	$$
}
	
%========================================
\item \mkstar{3} [13]

\cauhoi
{Trong thí nghiệm Y-âng về giao thoa ánh sáng có $\lambda = \SI{0,5}{\mu m}$, $a = \SI{1}{mm}$ và $D = \SI{2}{m}$. Khoảng vân $i$ bằng
	\begin{mcq}(4)
		\item $\SI{2,5}{mm}$. 
		\item $\SI{5,0}{mm}$. 
		\item $\SI{2,0}{mm}$. 
		\item $\SI{1,0}{mm}$. 
	\end{mcq}
}

\loigiai
{		\textbf{Đáp án: D.}
	
	Khoảng vân $i$ là
	$$
	i = \dfrac{\lambda D}{a} = \SI{1,0}{mm}.
	$$
}

%========================================
\item \mkstar{3} [13]

\cauhoi
{Trong thí nghiệm về giao thoa ánh sáng có $\lambda = \SI{0,7}{\mu m}$, $a = \SI{0,35}{mm}$ và $D = \SI{1}{m}$. Bề rộng của trường giao thoa là $L = \SI{18,2}{mm}$. Số vân sáng quan sát được là
	\begin{mcq}(4)
		\item $11$. 
		\item $8$. 
		\item $10$. 
		\item $9$. 
	\end{mcq}
}

\loigiai
{		\textbf{Đáp án: D.}
	
	Khoảng vân là
	$$
	i = \dfrac{\lambda D}{a} = \SI{2}{mm}.
	$$
	Ta có
	$$
	\dfrac{L}{2i} = \num{4,55}.
	$$
	Số vân quan sát được là 9 vân.
}
	
%========================================
\item \mkstar{3} [13]

\cauhoi
{Đoạn MN trên màn quan sát trong thí nghiệm Y-âng về giao thoa ánh sáng. Khi dùng ánh sáng có bước sóng $\lambda$ thì khoảng cách giữa hai vân sáng liên tiếp là $\SI{1,2}{mm}$. Trên MN có 16 vân sáng quan sát được (tại M, N đều là vân sáng). Đoạn MN có giá trị là
	\begin{mcq}(4)
		\item $\SI{10}{mm}$. 
		\item $\SI{12}{mm}$. 
		\item $\SI{16}{mm}$. 
		\item $\SI{18}{mm}$. 
	\end{mcq}
}

\loigiai
{		\textbf{Đáp án: D.}
	
	Khoảng cách giữa hai vân sáng liên tiếp cũng là khoảng vân nên $i = \SI{1,2}{mm}$.
	Tại M và N là các vân sáng và trên MN có tổng cộng 16 vân sáng quan sát được thì
	$$
	MN = 15i = \SI{18}{mm}.
	$$
}
%========================================
\item \mkstar{3} [12]

\cauhoi
{Thực hiện thí nghiệm Y-âng về giao thoa ánh sáng với khoảng cách giữa hai khe sáng là $\SI{2}{mm}$, khoảng cách từ mặt phẳng chứa hai khe đến màn quan sát là $\SI{2}{m}$, ánh sáng đơn sắc có bước sóng $\SI{0,7}{\mu m}$. Khoảng cách giữa vân sáng và vân tối liền kề là 
	
	\begin{mcq}(4)
		\item $\SI{0,0875}{mm}$. 
		\item $\SI{0,3500}{mm}$.
		\item $\SI{0,7000}{mm}$.
		\item $\SI{0,1750}{mm}$.
	\end{mcq} 
}

\loigiai
{		\textbf{Đáp án: B.}
	
	Khoảng cách giữa vân sáng và vân tối liền kề là $\num{0,5} i = \num{0,5} \dfrac{\lambda D}{a} = \SI{0,3500}{mm}$.		
}

%========================================
\item \mkstar{3} [12]

\cauhoi
{Thực hiện thí nghiệm Y-âng với ánh sáng đơn sắc có bước sóng $\lambda$. Biết khoảng cách giữa hai khe là $a$, khoảng cách từ mặt phẳng chứa hai khe đến màn quan sát là $D$ thì vân sáng bậc 3 cách vân trung tâm 
	\begin{mcq}(4)
		\item $4 \dfrac{\lambda D}{a}$. 
		\item $3 \dfrac{\lambda a}{D}$. 
		\item $3 \dfrac{\lambda D}{a}$. 
		\item $4 \dfrac{\lambda a}{D}$. 
	\end{mcq}
}

\loigiai
{		\textbf{Đáp án: C.}
	
	Vân sáng bậc 3 cách vân trung tâm một khoảng là $3 \dfrac{\lambda D}{a}$.	
}

%========================================
\item \mkstar{3 } [12]

\cauhoi
{Khoảng cách giữa 2 vân sáng liền kề là $\SI{0,5}{mm}$ thì khoảng vân có giá trị 
	\begin{mcq}(4)
		\item $\SI{1,5}{mm}$.
		\item $\SI{1,0}{mm}$. 
		\item $\SI{0,5}{mm}$.
		\item $\SI{0,25}{mm}$.
	\end{mcq}
}

\loigiai
{		\textbf{Đáp án: C.}
	
	Khoảng cách giữa 2 vân sáng liền kề là $\SI{0,5}{mm}$ thì khoảng vân cũng có giá trị $\SI{0,5}{mm}$.	
}
	
%========================================
\item \mkstar{3} [10]

\cauhoi
{Trong thí nghiệm Y-âng về giao thoa với ánh sáng đơn sắc, khoảng cách giữa hai khe là $\SI{1}{mm}$, khoảng cách từ mặt phẳng chứa hai khe đến màn quan sát là $\SI{2}{m}$ và khoảng vân là $\SI{0,8}{mm}$. Cho $c = 3 \cdot 10^{8} \; \si{m/s}$. Tần số ánh sáng đơn sắc dùng trong thí nghiệm là
	
	\begin{mcq}(4)
		\item $\text{5,5} \cdot 10^{14} \si{Hz}$. 
		\item $\text{4,5} \cdot 10^{14} \si{Hz}$. 
		\item $\text{7,5} \cdot 10^{14} \si{Hz}$.
		\item $\text{6,5} \cdot 10^{14} \si{Hz}$. 
	\end{mcq}
}

\loigiai
{		\textbf{Đáp án: C.}
	
	Khoảng vân cho bởi
	$$
	i = \dfrac{\lambda D}{a}  \rightarrow \lambda = \SI{0,4}{\mu m}
	$$
	Tần số ánh sáng cho bởi
	$$
	f = \dfrac{c}{\lambda} = \SI{7,5 e14}{Hz}.
	$$
}
	%========================================
	\item \mkstar{3} [10]
	
	\cauhoi
	{Trong thí nghiệm Y-âng về giao thoa ánh sáng, nguồn sáng phát ra ánh sáng đơn sắc có bước sóng $\SI{500}{nm}$. Khoảng cách giữa hai khe là $\SI{1}{mm}$, khoảng cách từ mặt phẳng chứa hai khe đến màn quan sát là $\SI{1}{m}$. Trên màn, khoảng cách giữa hai vân sáng liên tiếp bằng
		\begin{mcq}(4)
			\item $\SI{0,50}{mm}$.
			\item $\SI{1,0}{mm}$.
			\item $\SI{1,5}{mm}$. 
			\item $\SI{0,75}{mm}$. 
		\end{mcq}
	}
	
	\loigiai
	{		\textbf{Đáp án: A.}
		
		Khoảng cách giữa hai vân sáng là
		$$
		i = \dfrac{\lambda D}{a} = \SI{0,5}{mm}.
		$$
	}
	
%========================================
\item \mkstar{3} [4]

\cauhoi
{Trong thí nghiệm Y-âng về giao thoa ánh sáng đơn sắc có bước sóng $\lambda = \SI{0,6}{\mu m}$. Khoảng cách giữa hai khe là $\SI{1}{mm}$, khoảng cách từ hai khe đến màn là $\SI{2}{m}$. Vân sáng thứ tư cách vân trung tâm một khoảng là
	\begin{mcq}(4)
		\item $\SI{4,2}{mm}$. 
		\item $\SI{3,6}{mm}$. 
		\item $\SI{4,8}{mm}$. 
		\item $\SI{6,0}{mm}$. 
	\end{mcq}
}

\loigiai
{		\textbf{Đáp án: C.}
	
	Vân sáng thứ tư cách vân trung tâm một khoảng là
	$$
	x_{4} = 4 \dfrac{\lambda D}{a} = \SI{4,8}{mm}.
	$$
}

%========================================
\item \mkstar{3} [4]

\cauhoi
{Trong thí nghiệm Y-âng về giao thoa ánh sáng, hai khe cách nhau $\SI{3}{mm}$ được chiếu bằng ánh sáng đơn sắc có bước sóng $\SI{0,6}{\mu m}$. Các vân giao thoa được hứng trên màn cách hai khe $\SI{2}{m}$. Tại điểm M cách vân trung tâm $\SI{1,2}{mm}$ có
	\begin{mcq}(2)
		\item vân sáng bậc 3. 
		\item vân tối thứ 3. 
		\item vân sáng bậc 5. 
		\item vân sáng bậc 4. 
	\end{mcq}
}

\loigiai
{		\textbf{Đáp án: A.}
	
	Khoảng vân là $i = \dfrac{\lambda D}{a} = \SI{0,4}{mm}$.
	Tại điểm M, ta có:
	$$
	\dfrac{x_{M}}{i} = 3.
	$$
	Vậy tại M là vân sáng bậc 3.
}
	
%========================================
\item \mkstar{3} [2]

\cauhoi
{Trong thí nghiệm Y-âng về giao thoa ánh sáng đơn sắc, ta thấy tại điểm M trên màn có vân sáng bậc 5. Dịch chuyển màn quan sát ra xa thêm $\SI{20}{cm}$ thì tại M có vân tối thứ 5 tính từ trung tâm. Trước lúc dịch chuyển, khoảng cách từ màn đến hai khe bằng
	\begin{mcq}(4)
		\item $\SI{2}{m}$. 
		\item $\SI{1,8}{m}$. 
		\item $\SI{1,6}{m}$. 
		\item $\SI{2,2}{m}$. 
	\end{mcq}
}

\loigiai
{		\textbf{Đáp án: B.}
	
	Ban đầu, vị trí của điểm M cho bởi
	$$
	x_{M} = 5 \dfrac{\lambda D}{a}.
	$$
	Lúc sau, vị trí điểm M cho bởi
	$$
	x_{M} = \num{4,5} \dfrac{\lambda (D + \Delta D)}{a}.
	$$
	Từ hai phương trình trên ta được:
	$$
	5D = \num{4,5}(D + \Delta D) \rightarrow D = \SI{1,8}{m}.
	$$
}
	

%========================================
\item \mkstar{3} [3]

\cauhoi
{Trong thí nghiệm Y-âng về giao thoa ánh sáng, hai khe được chiếu bằng ánh sáng đơn sắc có bước sóng $\lambda$. Khoảng cách giữa hai khe sáng là $a$, khoảng cách từ mặt phẳng chứa hai khe đến màn quan sát là $\SI{1}{m}$. Trên màn quan sát, hai vân sáng bậc 3 nằm ở hai điểm M và N. Dịch màn quan sát một đoạn $\SI{50}{cm}$ theo hướng ra xa 2 khe thì số vân sáng trên đoạn MN giảm so với lúc đầu là 
	\begin{mcq}(4)
		\item 2 vân. 
		\item 5 vân. 
		\item 7 vân. 
		\item 4 vân. 
	\end{mcq}
}

\loigiai
{		\textbf{Đáp án: A.}
	
	Vì tại M và N là vị trí vân sáng bậc 3 ở hai bên vân trung tâm nên $\text{MN} = 6i$.
	Ta có:
	$$
	\dfrac{i'}{i} = \dfrac{D'}{D} = \dfrac{D + \Delta D}{D} = 1,5 \rightarrow i' = 1,5i.
	$$
	Ta có:
	$$
	\dfrac{\text{MN}}{2i'} = \dfrac{6i}{3i} = 2 \rightarrow MN = 2i'.
	$$
	Vậy số vân sáng trên đoạn MN lúc sau là 5 vân, giảm 2 vân so với lúc đầu là 7 vân.
}

%=========================================
\item \mkstar{3} [1]

\cauhoi
{Trong thí nghiệm Y-âng về giao thoa ánh sáng, hai khe được chiếu bằng ánh sáng đơn sắc. Khoảng cách giữa hai khe là $\SI{1,2}{mm}$. Khoảng cách từ mặt phẳng chứa hai khe đến màn quan sát là $\SI{1,5}{m}$. Trên màn quan sát, xét đoạn gồm 6 vân sáng liên tiếp cạnh nhau thì hai vân sáng ngoài cùng cách nhau $\SI{3}{mm}$. Bước sóng của ánh sáng đơn sắc này bằng
	\begin{mcq}(4)
		\item $\SI{600}{nm}$. 
		\item $\SI{400}{nm}$. 
		\item $\SI{500}{nm}$. 
		\item $\SI{480}{nm}$. 
	\end{mcq}
}

\loigiai
{		\textbf{Đáp án: D.}
	
	Độ rộng của 6 vân sáng liên tiếp là $5i$, suy ra $i = \SI{0,6}{mm}$.
	Khoảng vân cho bởi:
	$$
	i = \dfrac{\lambda D}{a} \rightarrow \lambda = \SI{480}{nm}.
	$$
}

%========================================
\item \mkstar{3} [13]

\cauhoi
{Trong thí nghiệm Y-âng về giao thoa ánh sáng có $a = \SI{1}{mm}$ và $D = \SI{1}{m}$. Trên màn quan sát thấy vân sáng bậc 5 cách vân sáng trung tâm $\SI{3,0}{mm}$. Bước sóng ánh sáng đơn sắc dùng trong thí nghiệm là
	\begin{mcq}(4)
		\item $\lambda = \SI{0,70}{\mu m}$. 
		\item $\lambda = \SI{0,50}{\mu m}$. 
		\item $\lambda = \SI{0,60}{\mu m}$. 
		\item $\lambda = \SI{0,53}{\mu m}$. 
	\end{mcq}
}

\loigiai
{		\textbf{Đáp án: C.}
	
Vị trí của vân sáng bậc 5 cho bởi:
	$$
	x_{5} = 5 \dfrac{\lambda D}{a} \rightarrow \lambda = \SI{0,6}{\mu m}.
	$$
}
	
%========================================
\item \mkstar{3} [1]

\cauhoi
{Trong thí nghiệm Y-âng về giao thoa ánh sáng, hai khe được chiếu bằng ánh sáng đơn sắc có bước sóng $\SI{0,48}{\mu m}$. Khoảng cách giữa hai khe là $\SI{0.9}{mm}$, khoảng cách từ mặt phẳng chứa hai khe đến màn quan sát là $\SI{1,5}{m}$. Trên màn quan sát, so với vân sáng trung tâm, vân tối thứ ba cách vân sáng trung tâm
	\begin{mcq}(4)
		\item $\SI{2,8}{mm}$. 
		\item $\SI{2,4}{mm}$. 
		\item $\SI{2}{mm}$. 
		\item $\SI{1,6}{mm}$. 
	\end{mcq}
}

\loigiai
{		\textbf{Đáp án: C.}

	Khoảng vân cho bởi
	$$
	i = \dfrac{\lambda D}{a} = \SI{0,8}{mm}.
	$$
	Khoảng cách từ vân trung tâm tới vân tối thứ ba là $2,5i = \SI{2}{mm}$.
}

%========================================
\item \mkstar{4} [32]

\cauhoi
{Trong thí nghiệm Y-âng về giao thoa ánh sáng đơn sắc. Khoảng cách giữa hai khe là $\SI{1}{mm}$ và khoảng cách từ hai khe tới màn quan sát là $\SI{2}{m}$. Trong khoảng bề rộng $\SI{24,5}{mm}$ trên màn quan sát có 25 vân tối. Biết một đầu bề rộng là vân tối, đầu còn lại là vân sáng. Bước sóng của ánh sáng đó là
	\begin{mcq}(4)
		\item $\SI{0,48}{\mu m}$. 
		\item $\SI{0,40}{\mu m}$. 
		\item $\SI{0,50}{\mu m}$. 
		\item $\SI{0,52}{\mu m}$. 
	\end{mcq}
}

\loigiai
{		\textbf{Đáp án: C.}
	
	Một khoảng bề rộng có một đầu là vân sáng, đầu còn lại là vân tối với 25 vân tối thì bề rộng này có kích thước là $24,5i$. Ta có:
	$$
	\num{24,5}i = \SI{24,5}{mm} \rightarrow i = \SI{1}{mm}.
	$$
	Ta có:
	$$
	i = \dfrac{\lambda D}{a} \rightarrow \lambda = \SI{0,50}{\mu m}.
	$$
}

%========================================
\item \mkstar{4} [4]

\cauhoi
{Trong thí nghiệm Y-âng về giao thoa ánh có bước sóng $\lambda$, khoảng cách giữa hai khe là $\SI{1}{mm}$. Ban đầu, tại M cách vị trí trung tâm $\SI{5,25}{mm}$ người ta quan sát được vân sáng bậc 5. Giữ cố định màn chứa hai khe, di chuyển màn quan sát ra xa và dọc theo đường vuông góc với mặt phẳng chứa hai khe một đoạn $\SI{0,75}{m}$ thì ta thấy điểm M chuyển thành vân tối lần thứ hai. Bước sóng $\lambda$ có giá trị là
	\begin{mcq}(4)
		\item $\SI{0,60}{\mu m}$. 
		\item $\SI{0,50}{\mu m}$. 
		\item $\SI{0,70}{\mu m}$. 
		\item $\SI{0,64}{\mu m}$. 
	\end{mcq}
}

\loigiai
{		\textbf{Đáp án: A.}
	
	Vì khoảng cách từ hai khe đến màn tăng lên nên vị trí điểm M sẽ bị dời về vị trí các vân tối gần vân trung tâm hơn. Vì trong quá trình này, vị trí điểm M bị chuyển thành vân tối hai lần nên nó ở vị trí vân tối thứ 4.
	
	Tại vị trí điểm M lúc đầu và lúc sau, ta có:
	$$
	x_{M} = 5 \dfrac{\lambda D}{a} = 3,5 \dfrac{\lambda (D + \Delta D)}{a} \rightarrow D = \SI{1,75}{m}.
	$$
	Tại vị trí điểm M lúc đầu, ta có:
	$$
	x_{M} = 5 \dfrac{\lambda D}{a} \rightarrow \lambda = \SI{0,60}{\mu m}.
	$$
}
	
\end{enumerate}

\loigiai
{
	\begin{center}
		\textbf{BẢNG ĐÁP ÁN}
	\end{center}
	\begin{center}
		\begin{tabular}{|m{2.8em}|m{2.8em}|m{2.8em}|m{2.8em}|m{2.8em}|m{2.8em}|m{2.8em}|m{2.8em}|m{2.8em}|m{2.8em}|}
			\hline
			01.C  &  02.C  & 03.A  & 04.B  & 05.D  & 06.C  & 07.A  & 08.B  & 09.C  & 10.C \\
			\hline
			11.C  &  12.B  & 13.C  & 14.D  & 15.D  & 16.D  & 17.B  & 18.C  & 19.C  & 20.C \\
			\hline
			21.A  &  22.C  & 23.A  & 24.B  & 25.A  & 26.D  & 27.C  & 28.C  & 29.C  & 30.A \\
			\hline
			
		\end{tabular}
	\end{center}
}

\section{Dạng bài: Giao thoa nhiều ánh sáng đơn sắc}
\begin{enumerate}[label=\bfseries Câu \arabic*:]

%========================================
\item \mkstar{4} [7]

\cauhoi
{Trong thí nghiệm Y-âng về giao thoa ánh sáng, khoảng cách giữa hai khe là $\SI{1,2}{mm}$, khoảng cách từ hai khe đến màn là $\SI{2,0}{m}$. Người ta chiếu đồng thời hai bức xạ đơn sắc $\lambda_{1} = \SI{0,48}{\mu m}$ và $\lambda_{2} = \SI{0,6}{\mu m}$. Khoảng cách ngắn nhất giữa hai vân sáng bức xạ trùng nhau là
	\begin{mcq}(4)
		\item $\SI{6}{mm}$. 
		\item $\SI{2,4}{mm}$. 
		\item $\SI{4,8}{mm}$. 
		\item $\SI{4}{mm}$. 
	\end{mcq}
}

\loigiai
{		\textbf{Đáp án: D.}
	
	Khoảng vân trùng cho bởi
	$$
	\lambda_{\equiv} = LCM(\num{0,48};\num{0,60}) = \SI{2,4}{mm}.
	$$
	Khoảng cách ngắn nhất giữa hai vân sáng bức xạ trùng nhau là
	$$
	i_{\equiv} = \dfrac{\lambda_{\equiv} D}{a} = \SI{4}{mm}.
	$$
}

%========================================
\item \mkstar{4} [13]

\cauhoi
{Trong thí nghiệm Y-âng về giao thoa ánh sáng, có $a = \SI{1}{mm}$, $D = \SI{2}{m}$. Nếu chiếu đồng thời hai bức xạ đơn sắc có bước sóng $\lambda_{1}=\SI{0,6}{\mu m}$ ; $\lambda_{2} =\SI{0,5}{\mu m}$ thì trên màn quan sát có những vị trí tại đó có vân sáng của hai bức xạ trùng nhau gọi là vân trùng. Khoảng cách ngắn nhất giữa hai vân trùng bằng
	\begin{mcq}(4)
		\item $\SI{1,2}{mm}$. 
		\item $\SI{6,0}{mm}$. 
		\item $\SI{1,0}{mm}$. 
		\item $\SI{12,0}{mm}$. 
	\end{mcq}
}

\loigiai
{		\textbf{Đáp án: B.}
	
	Bước sóng trùng cho bởi:
	$$
	\lambda_{\equiv} = LCM(\num{0,6};{0,5}) = \SI{3,0}{\mu m}.
	$$
	Khoảng cách ngắn nhất giữa hai vân trùng là khoảng vân trùng
	$$
	i_{\equiv} = \dfrac{\lambda_{\equiv }D}{a} = \SI{6}{mm}.
	$$
}

%========================================
\item \mkstar{4 } [12]

\cauhoi
{Thực hiện thí nghiệm Y-âng về giao thoa ánh sáng bằng hai bức xạ đơn sắc có bước sóng $\lambda_{1}= \SI{0,5}{\mu m}$ và $\lambda_{2}$ thì trong khoảng giữa hai vân sáng liền kề giống màu vân sáng trung tâm có 5 vân sáng của bức xạ có bước sóng $\lambda_{1}$ và 4 vân sáng của bức xạ có bước sóng $\lambda_{2}$ (không kề các vị trí hai vân trùng nhau). Nếu thay bức xạ có bước sóng $\lambda_{1}$ bằng bức xạ khác có bước sóng $\lambda_{3} = \SI{0,65}{\mu m}$ thì trong khoảng giữa hai vân liền kề giống màu vân sáng trung tâm có tổng cộng bao nhiêu vân sáng (không kể các vị trí hai vân trùng)?
	\begin{mcq}(4)
		\item 21. 
		\item 25. 
		\item 27. 
		\item 23. 
	\end{mcq}
}

\loigiai
{		\textbf{Đáp án: D.}
	
Ta có:
$$
	\dfrac{\lambda_{1}}{\lambda_{2}} = \dfrac{5}{6} \rightarrow \lambda_{2} = \SI{0,6}{\mu m}.
$$
Ta có:
$$
\dfrac{\lambda_{3}}{\lambda_{2}} = \dfrac{13}{12}.
$$
Vậy có 11 bức xạ $\lambda_{2}$ và 12 bức xạ $\lambda_{3}$ trong khoảng giữa hai vân sáng giống màu vân trung tâm. Vậy có tổng cộng 23 bức xạ.
}

%========================================
\item \mkstar{4} [10]

\cauhoi
{Trong thí nghiệm giao thoa khe Y-âng, nguồn sáng phát ra đồng thời hai bức xạ có bước sóng lần lượt là $\lambda_{1} = \SI{0,76}{\mu m}$ và $\lambda_{2} = \SI{0,45}{\mu m}$. Biết $a = \SI{1}{mm}$ và $D = \SI{2}{m}$. Khoảng cách nhỏ nhất giữa hai vân sáng quan sát được trên màn là
	\begin{mcq}(4)
		\item $\SI{0,26}{mm}$. 
		\item $\SI{0,62}{mm}$. 
		\item $\SI{1}{mm}$. 
		\item $\SI{1,28}{mm}$. 
	\end{mcq}
}

\loigiai
{		\textbf{Đáp án: B.}
	
	Khoảng cách nhỏ nhất giữa hai vân sáng là khoảng cách giữa hai vị trí vân sáng bậc 1 của $\lambda_{1}$ và $\lambda_{2}$. Ta có:
	$$
	\Delta_{d_{min}} = \dfrac{(\lambda_{1}-\lambda_{2})D}{a} = \SI{0,62}{mm}.
	$$
}

%========================================
\item \mkstar{4} [10]

\cauhoi
{Trong thí nghiệm Y-âng về giao thoa ánh sáng, thực hiện đồng thời với hai ánh sáng đơn sắc có khoảng vân trên màn lần lượt là $\SI{1,2}{mm}$ với $\SI{1,8}{mm}$. Trên màn quan sát, M và N là hai điểm ở cùng phía so với vân sáng trung tâm và cách vân trung tâm lần lượt là $\SI{6}{mm}$ và $\SI{20}{mm}$. Trên đoạn MN số vạch sáng quan sát được là
	\begin{mcq}(4)
		\item $19$. 
		\item $16$. 
		\item $20$. 
		\item $18$. 
	\end{mcq}
}

\loigiai
{		\textbf{Đáp án: D.}
	
	Ta có:
	$$
	x_{M} \leq k_{1}i_{1} \leq x_{N} \rightarrow 5 \leq k_{1} \leq 16,6.                             
	$$
	Vậy có 12 vân sáng $\lambda_{1}$ trên đoạn MN.
	
	Ta có:
	$$
	x_{M} \leq k_{2}i_{2} \leq x_{N} \rightarrow 3,3 \leq k_{2} \leq 11,1.                             
	$$
	Vậy có 8 vân sáng $\lambda_{2}$ trên đoạn MN.
	
	Số vân trùng: 2 vân.
	
	Vây quan sát được 18 vân sáng.
}
	
%========================================
\item \mkstar{4} [1]

\cauhoi
{Trong thí nghiệm Y-âng về giao thoa ánh sáng, nguồn sáng dùng trong thí nghiệm gồm 2 bức xạ là lục có bước sóng $\lambda_{1}$ và tím có bước sóng $\lambda_{2}$. Trên màn quan sát, ta thấy trong khoảng giữa hai vân sáng liên tiếp có màu giống vân sáng trung tâm có 4 vân sáng lục và 6 vân sáng tím. Biết $\lambda_{1} + \lambda_{2} = \SI{960}{nm}$. Giá trị của $\lambda_{1}$ gần bằng 
	\begin{mcq}(4)
		\item $\SI{570}{nm}$. 
		\item $\SI{525}{nm}$. 
		\item $\SI{555}{nm}$. 
		\item $\SI{550}{nm}$. 
	\end{mcq}
}

\loigiai
{		\textbf{Đáp án: C.}
	
	Vì trong khoảng giữa hai vân sáng liên tiếp có màu giống vân trung tâm có 4 vân sáng lục và 6 vân sáng tím nên
	$$
	\dfrac{\lambda_{1}}{\lambda_{2}} = \dfrac{7}{5}  \rightarrow 5\lambda_{1} - 7\lambda_{2} = 0.
	$$
	Kết hợp với phương trình $\lambda_{1} + \lambda_{2} = \SI{960}{nm}$ ta suy ra $\lambda_{1} = \SI{560}{nm}$ và $\lambda_{2} = \SI{400}{nm}$.
}
	
\end{enumerate}

\loigiai
{
	\begin{center}
		\textbf{BẢNG ĐÁP ÁN}
	\end{center}
	\begin{center}
		\begin{tabular}{|m{2.8em}|m{2.8em}|m{2.8em}|m{2.8em}|m{2.8em}|m{2.8em}|m{2.8em}|m{2.8em}|m{2.8em}|m{2.8em}|}
			\hline
			01.D  & 02.B  & 03.D  & 04.B  & 05.D  & 06.C  &  &  &  &  \\
			\hline
			
		\end{tabular}
	\end{center}
}

\section{Dạng bài: Giao thoa ánh sáng trắng}
\begin{enumerate}[label=\bfseries Câu \arabic*:]
	
%========================================
\item \mkstar{2} [4]

\cauhoi
{Khi thực hiện thí nghiệm Y-âng về giao thoa với ánh sáng trắng, ta quan sát thấy
	\begin{mcq}(1)
		\item một dải màu liên tục từ đỏ tới tím. 
		\item vân sáng trắng ở chính giữa, hai bên có những dải màu, với tím ở trong và đỏ ở ngoài. 
		\item vân sáng trắng ở chính giữa, hai bên có những dải màu, với đỏ ở trong và tím ở ngoài. 
		\item các dải sáng trắng xen kẻ với những vạch tối.. 
	\end{mcq}
}

\loigiai
{		\textbf{Đáp án: B.}
	
	Khi thực hiện thí nghiệm Y-âng về giao thoa với ánh sáng trắn, ta quan sát thấy vân sáng trắng ở chính giữa, hai bên có những dải màu, với tím ở trong và đỏ ở ngoài.
}

%========================================
\item \mkstar{3} [4]

\cauhoi
{Trong thí nghiệm Y-âng về giao thoa ánh sáng, hai khe được chiếu bằng ánh sáng trắng biến thiên liên tục từ $\SI{0,38}{\mu m}$ đến $\SI{0,75}{\mu m}$. Khoảng cách giữa hai khe là $\SI{0,2}{mm}$. Khoảng cách từ hai khe tới màn là $\SI{2}{m}$. Khoảng cách giữa vân sáng bậc 3 màu đỏ và vân sáng bậc 3 màu tím ở cùng một bên so với vân trung tâm là
	\begin{mcq}(4)
		\item $\SI{11,1}{mm}$. 
		\item $\SI{11,4}{mm}$. 
		\item $\SI{7,4}{mm}$. 
		\item $\SI{2,5}{mm}$. 
	\end{mcq}
}

\loigiai
{		\textbf{Đáp án: A.}
	
	Khoảng cách giữa vân sáng bậc 3 màu đỏ và vân sáng bậc 3 màu tím là
	$$
	\Delta d = 3(i_{d} - i_{t}) = 3\dfrac{(\lambda_{d} - \lambda_{t})D}{a} = \SI{11,1}{mm}.
	$$
}

%========================================
\item \mkstar{3} [7]

\cauhoi
{Trong thí nghiệm giao thoa với ánh sáng trắng có bước sóng $\lambda$ từ $\SI{0,38}{\mu m}$ đến $\SI{0,76}{\mu m}$. Tại vị trí vân sáng bậc 4 của ánh sáng đỏ có bước sóng $\lambda_{\text{đ}} =\SI{0,75}{\mu m}$ có số vạch sáng nằm trùng với vị trí này là
	\begin{mcq}(4)
		\item 5. 
		\item 4. 
		\item 3. 
		\item 2. 
	\end{mcq}
}

\loigiai
{		\textbf{Đáp án: B.}
	
	Ta có $4\lambda_{d} = k\lambda \rightarrow \lambda = \dfrac{3}{k}$. \\
	Lại có $\SI{0,38}{\mu m} \leq \lambda \leq \SI{0,76}{\mu m} \rightarrow 3,95 \leq k \leq 7,89$.  \\
	Vậy có bốn ánh sáng khác cũng có vân sáng tại vị trí trên.
}
	
%========================================
\item \mkstar{4} [3]

\cauhoi
{Trong thí nghiệm Y-âng về giao thoa ánh sáng, nguồn S phát ra ánh sáng đơn sắc: $\lambda_{1} = \SI{0,4}{\mu m}$ (màu tím); $\lambda_{2} = \SI{0,5}{\mu m}$ (màu lục) và $\lambda_{3} = \SI{0,7}{\mu m}$ (màu đỏ). Số vân sáng màu tím và màu đỏ quan sát được trên màn, nằm giữa hai vân sáng liên tiếp giống màu vân trung tâm là
	\begin{mcq}(2)
		\item 34 vân tím, 19 vân đỏ. 
		\item 24 vân tím, 12 vân đỏ. 
		\item 35 vân tím, 20 vân đỏ. 
		\item 27 vân tím, 18 vân đỏ. 
	\end{mcq}
}

\loigiai
{		\textbf{Đáp án: B.}
	
	Bước sóng vân trùng là
	$$
	\lambda_{\equiv} = \mathrm{LCM}(\lambda_{1}; \lambda_{2}; \lambda_{3}) = \SI{14}{\mu m}.
	$$
	Ta có $\dfrac{\lambda_{\equiv}}{\lambda_{1}} = \num{35}$. Suy ra trong khoảng giữa hai vân sáng cùng màu vân trung tâm có 34 vị trí vân màu tím. \\
	Ta có $\dfrac{\lambda_{\equiv}}{\lambda_{3}} = \num{20}$. Suy ra trong khoảng giữa hai vân sáng cùng màu vân trung tâm có 19 vị trí vân màu đỏ. \\
	Ta có khoảng vân trùng (tím, đỏ) (tím, lục) và (lục, đỏ) lần lượt là: \\
	$\left\{
	\begin{aligned}
		& \lambda_{13} = \mathrm{LCM} (\lambda_{1}; \lambda_{3}) =\SI{2,8}{\mu m} \\
		& \lambda_{1,2} = \mathrm{LCM} (\lambda_{1}; \lambda_{2}) = \SI{2,0}{\mu m} \\
		& \lambda_{2,3} = \mathrm{LCM} (\lambda_{2}; \lambda_{3}) = \SI{3,5}{\mu m}. \\
	\end{aligned} 
	\right.$ \\
	Ta có $\dfrac{\lambda_{\equiv}}{\lambda_{13}} = \num{5}$. Suy ra trong khoảng giữa hai vân sáng cùng màu vân trung tâm có 4 vị trí vân màu tím - đỏ. \\
	Ta có $\dfrac{\lambda_{\equiv}}{\lambda_{12}} = \num{7}$. Suy ra trong khoảng giữa hai vân sáng cùng màu vân trung tâm có 6 vị trí vân màu tím - lục. \\
	Suy ra số vân màu tím quan sát được là $34 - 4 - 6 = \num{24}$ vân sáng.
	Ta có $\dfrac{\lambda_{\equiv}}{\lambda_{23}} = \num{4}$. Suy ra trong khoảng giữa hai vân sáng cùng màu vân trung tâm có 3 vị trí vân màu lục - đỏ. \\
	Suy ra số vân màu tím quan sát được là $19 - 4 - 3 = \num{12}$ vân sáng.
}
	
%========================================
\item \mkstar{4} [2]

\cauhoi
{Trong thí nghiệm Y-âng về giao thoa ánh sáng trắng (có bước sóng thay đối từ $\SI{380}{nm}$ đến $\SI{760}{nm}$), hai khe được chiếu sáng đồng thời bởi hai bức xạ đơn sắc có bước sóng lần lượt là $\lambda_{1}$ và $\lambda_{2}\left(\lambda_{2}>\lambda_{1}\right)$, khoảng cách giữa hai khe là $\SI{0,5}{mm}$, khoảng cách từ hai khe đến màn quan sát là $\SI{2}{m}$. Thấy vân sáng bậc 3 của bức xạ $\lambda_{1}$ trùng với vân sáng bậc $k$ của bức xạ $\lambda_{2}$ và cách vân trung tâm $\SI{6}{mm}$. Giá tri $k$ và $\lambda_{2}$ là
	\begin{mcq}(2)
		\item $k = 2$ và $\lambda_{2} = \SI{0,75}{\mu m}$. 
		\item $k = 2$ và $\lambda_{2} = \SI{4,2}{\mu m}$. 
		\item $k = 1$ và $\lambda_{2} = \SI{1,2}{\mu m}$. 
		\item $k = 1$ và $\lambda_{2} = \SI{4,8}{\mu m}$. 
	\end{mcq}
}

\loigiai
{		\textbf{Đáp án: A.}
	
	Ta có:
	
	$$
	x_{\equiv} = 3 \dfrac{\lambda_{1} D}{a} \rightarrow \lambda_{1} = \SI{0,5}{\mu m}.
	$$
	Ta có:
	
	$$
	3 \lambda_{1} = k \lambda_{2} \rightarrow \lambda_{2} = \dfrac{1500}{k} \; (\si{nm}).
	$$
	Lại có:
	$$
	380 \leq \lambda_{2} \leq 760 \rightarrow 1,97 \leq k \leq 3,95.
	$$
	Vậy $k = 2$ và $\lambda_{2} = \SI{0,75}{\mu m}$.
}   
	
%========================================
\item \mkstar{4} [1]

\cauhoi
{Trong thí nghiệm Y-âng về giao thoa ánh sáng, hai khe được chiếu bằng ánh sáng trắng có bước sóng thay đổi liên tục từ $\SI{380}{nm}$ đến $\SI{760}{nm}$. Trên màn, tại vị trí vân sáng bậc 5 của bức xạ lục có bước sóng $\SI{540}{nm}$, số bức xạ khác cũng cho vân sáng tại đó là
	\begin{mcq}(4)
		\item 3. 
		\item 6. 
		\item 5. 
		\item 4. 
	\end{mcq}
}

\loigiai
{		\textbf{Đáp án: D.}
	
	Ta có:
	$$
	k_{5}\lambda_{l} = k \lambda \rightarrow \lambda = \dfrac{2700}{k} .
	$$
	Mặt khác:
	$$
	380 \leq \lambda \leq 760 \rightarrow 3,5 \leq \lambda \leq 7,1.
	$$
	Vậy có tổng cộng 4 bức xạ khác cũng cho vân sáng tại đó.
}

	
\end{enumerate}

\loigiai
{
	\begin{center}
		\textbf{BẢNG ĐÁP ÁN}
	\end{center}
	\begin{center}
		\begin{tabular}{|m{2.8em}|m{2.8em}|m{2.8em}|m{2.8em}|m{2.8em}|m{2.8em}|m{2.8em}|m{2.8em}|m{2.8em}|m{2.8em}|}
			\hline
			01.B  & 02.A  & 03.B  & 04.B  & 05.A  & 06.D  & &  &  &  \\
			\hline
			
		\end{tabular}
	\end{center}
}

\whiteBGstarEnd