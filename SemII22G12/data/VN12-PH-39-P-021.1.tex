%\setcounter{chapter}{1}
\chapter{Luyện tập: Hiện tượng quang điện. Thuyết lượng tử ánh sáng}
\section{Bài toán thuyết lượng tử ánh sáng}

\begin{enumerate}
	\item Trong chân không, một bức xạ đơn sắc có bước sóng $\lambda =\text{0,6}\ \mu\text{m}$. Cho biết giá trị hằng số $h=\text{6,625}\cdot 10^{-34}\ \text{Js}$, $c=3\cdot 10^8\ \text{m/s}$ và $e=\text{1,6}\cdot 10^{-19}\ \text{C}$. Lượng tử năng lượng của ánh sáng này có giá trị
	\begin{mcq}(4)
		\item $\text{5,3}\ \text{eV}.$
		\item $\text{2,07}\ \text{eV}.$
		\item $\text{1,2}\ \text{eV}.$
		\item $\text{3,71}\ \text{eV}.$
	\end{mcq}
	
	
	\item Trong môi trường nước có chiết suất bằng $\dfrac{4}{3}$, một bức xạ đơn sắc có bước sóng bằng $\lambda =\text{0,6}\ \mu\text{m}$. Cho biết giá trị hằng số $h=\text{6,625}\cdot 10^{-34}\ \text{Js}$, $c=3\cdot 10^8\ \text{m/s}$ và $e=\text{1,6}\cdot 10^{-19}\ \text{C}$. Lượng tử năng lượng của ánh sáng này có giá trị
	
	\begin{mcq}(4)
		\item $\text{2,76}\ \text{eV}.$
		\item $\text{2,07}\ \text{eV}.$
		\item $\text{1,2}\ \text{eV}.$
		\item $\text{1,55}\ \text{eV}.$
	\end{mcq}
\end{enumerate}
\section{Bài toán điều kiện xảy ra hiện tượng quang điện}
\begin{enumerate}
	\item Công thoát êlectron của một kim loại là $A=\text{1,88}\ \text{eV}$. Cho biết giá trị hằng số $h=\text{6,625}\cdot 10^{-34}\ \text{Js}$, $c=3\cdot 10^8\ \text{m/s}$ và $e=\text{1,6}\cdot 10^{-19}\ \text{C}$. Giới hạn quang điện của kim loại này có giá trị là
	\begin{mcq}(4)
		\item $\text{550}\ \text{nm}.$
		\item $\text{220}\ \text{nm}.$
		\item $\text{1057}\ \text{nm}.$
		\item $\text{661}\ \text{nm}.$
	\end{mcq}
	
	\item Giới hạn quang điện của một kim loại là $\lambda =\text{0,278}\ \mu\text{m}$. Cho biết giá trị hằng số $h=\text{6,625}\cdot 10^{-34}\ \text{Js}$, $c=3\cdot 10^8\ \text{m/s}$ và $e=\text{1,6}\cdot 10^{-19}\ \text{C}$. Công thoát electron của kim loại này có giá trị là
	
	\begin{mcq}(4)
		\item $\text{4,47}\ \text{eV}.$
		\item $\text{3,54}\ \text{eV}.$
		\item $\text{2,73}\ \text{eV}.$
		\item $\text{3,09}\ \text{eV}.$
	\end{mcq}
	
	\item \textbf{[Trích đề thi THPT QG 2012]} Biết công thoát $A$ của $\ce{Ca};\ \ce{K}$; $\ \ce{Ag};\ \ce{Cu}$ lần lượt là $\text{2,89}\ \text{eV};\ \text{2,26}\ \text{eV};$ $\ \text{4,78} eV;\ \text{4,14}\ \text{eV}$. Chiếu ánh sáng có bước sóng $\text{0,33}\ \mu\text{m}$ vào bề mặt các kim loại trên. Hiện tượng quang điện không xảy ra với các kim loại nào sau đây?
	
	\begin{mcq}(4)
		\item $\ce{Ag};\ \ce{Cu}.$
		\item $\ce{K};\ \ce{Cu}.$
		\item $\ce{Ca};\ \ce{Ag}.$
		\item $\ce{K};\ \ce{Ca}.$
	\end{mcq}
\end{enumerate}

\section{Bài toán hệ thức Anh-xtanh về hiện tượng quang điện}
\begin{enumerate}
	
	\item Một bản kim loại có công thoát electron bằng $\text{4,47}\ \text{eV}$. Chiếu ánh sáng kích thích có bước sóng bằng $\text{0,14}\ \mu\text{m}$(trong chân không). Cho biết giá trị hằng số $h=\text{6,625}\cdot 10^{-34}\ \text{Js}$, $c=3\cdot 10^8\ \text{m/s}$ và $e=\text{1,6}\cdot 10^{-19}\ \text{C}, \ m_\text{e}=\text{9,1}\cdot 10^{-31}\ \text{kg}$. Động năng ban đầu cực đại và vận tốc ban đầu của electron quang điện lần lượt là
	\begin{mcq}(2)
		\item $\text{7,04}\cdot 10^{-19}\ \text{J};\ \text{1,24}\cdot 10^{6}\ \text{m/s}.$
		\item $\text{3,25}\ \text{eV};\ \text{2,43}\cdot 10^{6}\ \text{m/s}.$
		\item $\text{5,37}\cdot 10^{-19}\ \text{J};\ \text{2,43}\cdot 10^{6}\ \text{m/s}.$
		\item $\text{4,40}\ \text{eV};\ \text{1,24}\cdot 10^{6}\ \text{m/s}.$
	\end{mcq}
	
	\item \textbf{[Trích đề thi THPT QG năm 2009].} Chiếu đồng thời hai bức xạ có bước sóng $\text{0,452}\ \mu \text{m}$   và $\text{0,243}\ \mu\text{m}$ vào một tấm kim loại có giới hạn quang điện là $\text{0,5}\ \mu\text{m}$. Cho biết giá trị hằng số $h=\text{6,625}\cdot 10^{-34}\ \text{Js}$, $c=3\cdot 10^8\ \text{m/s}$ và $e=\text{1,6}\cdot 10^{-19}\ \text{C}, \ m_\text{e}=\text{9,1}\cdot 10^{-31}\ \text{kg}$. Vận tốc ban đầu cực đại của các electron quang điện bằng
	
	\begin{mcq}(4)
		\item $\text{9,61}\cdot 10^{5}\ \text{m/s}.$
		\item $\text{1,34}\cdot 10^{6}\ \text{m/s}.$
		\item $\text{2,29}\cdot 10^{4}\ \text{m/s}.$
		\item $\text{9,24}\cdot 10^{3}\ \text{m/s}.$
	\end{mcq}
	
	\item Chiếu lần lượt hai bức xạ điện từ có bước sóng $\lambda_1$ và $\lambda_2$ với $\lambda_2=2\lambda_1$  vào một tấm kim loại thì tỉ số động năng ban đầu cực đại của quang electron bứt ra khỏi kim loại là 9. Giới hạn quang điện của kim loại là $\lambda_0$. Tỉ số $\dfrac{\lambda_0}{\lambda_1}$ bằng
	\begin{mcq}(4)
		\item $\dfrac{16}{9}.$
		\item 2.
		\item $\dfrac{16}{7}.$
		\item $\dfrac{8}{7}.$
	\end{mcq}
	
	
	
\end{enumerate}
\section{Bài toán về hiệu suất lượng tử}
\begin{enumerate}
	\item Một ngọn đèn phát ra ánh sáng đơn sắc có $\lambda=\text{0,6}\ \mu\text{m}$ sẽ phát ra bao nhiêu photon trong 10 s nếu công suất đèn là $P = 10\ \text{W}$.
	\begin{mcq}(4)
		\item $\text{3,0189}\cdot 10^{20}.$
		\item $\text{6}\cdot 10^{20}.$
		\item $\text{3,0189}\cdot 10^{16}.$
		\item $\text{6,04}\cdot 10^{16}.$
	\end{mcq}
	
	
	\item Chiếu một chùm bức xạ vào tế bào quang điện có catot làm bằng $\ce{Na}$ thì cường độ dòng quang điện bão hòa là $3\ \mu\text{A}$ . Số electron bị bứt ra khỏi catot trong hai phút là bao nhiêu?
	
	\begin{mcq}(4)
		\item $\text{3,25}\cdot 10^{15}.$
		\item $\text{2,35}\cdot 10^{14}.$
		\item $\text{2,25}\cdot 10^{15}.$
		\item $\text{4,45}\cdot 10^{15}.$
	\end{mcq}
\end{enumerate}

\section{Lý thuyết hiện tượng quang điện}
\begin{enumerate}
	\item Hiện tượng bứt electron ra khỏi kim loại, khi chiếu ánh sáng kích thích có bước sóng thích hợp lên kim loại được gọi là
	\begin{mcq}
		\item hiện tượng bức xạ.
		\item hiện tượng phóng xạ.	
		\item hiện tượng quang dẫn.
		\item hiện tượng quang điện.
	\end{mcq}
	
	\item Hiện tượng quang điện là hiện tượng electron bứt ra khỏi bề mặt của tấm kim loại khi 
	\begin{mcq}
		\item có ánh sáng thích hợp chiếu vào nó.
		\item tấm kim loại bị nung nóng.
		\item tấm kim loại bị nhiễm điện do tiếp xúc với vật nhiễm điện khác.
		\item tấm kim loại được đặt trong điện trường đều.
	\end{mcq}
	
	\item Nếu chiếu một chùm tia hồng ngoại vào tấm kẽm tích điện âm, thì
	\begin{mcq}
		\item tấm kẽm mất dần điện tích dương.
		\item tấm kẽm mất dần điện tích âm.	
		\item tấm kẽm trở nên trung hoà về điện.
		\item điện tích âm của tấm kẽm không đổi.
	\end{mcq}
	
	\item Giới hạn quang điện của mỗi kim loại là
	\begin{mcq}
		\item bước sóng của ánh sáng kích thích chiếu vào kim loại.
		\item công thoát của các electron ở bề mặt kim loại đó.
		\item bước sóng giới hạn của ánh sáng kích thích để gây ra hiện tượng quang điện kim loại đó.	
		\item hiệu điện thế hãm.
	\end{mcq}
	
	\item Giới hạn quang điện của mỗi kim loại là
	\begin{mcq}
		\item bước sóng dài nhất của bức xạ chiếu vào kim loại đó để gây ra hiện tượng quang điện.	
		\item bước sóng ngắn nhất của bức xạ chiếu vào kim loại đó để gây ra hiện tượng quang điện.	
		\item công nhỏ nhất dùng để bứt electron ra khỏi kim loại đó.	
		\item công lớn nhất dùng để bứt electron ra khỏi kim loại đó.
	\end{mcq}
	
	\item Giới hạn quang điện tuỳ thuộc vào
	\begin{mcq}
		\item bản chất của kim loại.
		\item điện áp giữa anôt và catôt của tế bào quang điện.
		\item bước sóng của ánh sáng chiếu vào catôt.	
		\item điện trường giữa anôt và catôt.
	\end{mcq}
	
	\item Để gây được hiệu ứng quang điện, bức xạ dọi vào kim loại được thoả mãn điều kiện là 
	\begin{mcq}
		\item tần số lớn hơn giới hạn quang điện.
		\item tần số nhỏ hơn giới hạn quang điện.
		\item bước sóng nhỏ hơn giới hạn quang điện.
		\item bước sóng lớn hơn giới hạn quang điện.
	\end{mcq}
	
	
	\item Khi chiếu sóng điện từ xuống bề mặt tấm kim loại, hiện tượng quang điện xảy ra nếu
	\begin{mcq}
		\item sóng điện từ có nhiệt độ đủ cao.
		\item sóng điện từ có bước sóng thích hợp.	
		\item sóng điện từ có cường độ đủ lớn.
		\item sóng điện từ phải là ánh sáng nhìn thấy được.
	\end{mcq}
	
	\item Trong trường hợp nào dưới đây có thể xảy ra hiện tượng quang điện? Ánh sáng Mặt Trời chiếu vào
	\begin{mcq}
		\item mặt nước biển.
		\item lá cây.	
		\item mái ngói.
		\item tấm kim loại không sơn.
	\end{mcq}
	
	\item Lần lượt chiếu hai bức xạ có bước sóng $\lambda_1=\text{0,75}\ \mu \text{m}$ và $\lambda_2=\text{0,25}\ \mu \text{m}$ vào một tấm kẽm có giới hạn quang điện $\lambda_0=\text{0,35}\ \mu \text{m}$. Bức xạ nào gây ra hiện tượng quang điện?
	\begin{mcq}
		\item Cả hai bức xạ.
		\item Chỉ có bức xạ $\lambda_1$.	
		\item Chỉ có bức xạ $\lambda_2$.
		\item Không có bức xạ nào trong hai bức xạ đó.
	\end{mcq}
	
	\item Electron quang điện bị bứt ra khỏi bề mặt kim loại khi bị chiếu sáng nếu
	\begin{mcq}
		\item cường độ của chùm sáng rất lớn.	
		\item bước sóng của ánh sáng lớn.
		\item tần số ánh sáng nhỏ.	
		\item bước sóng nhỏ hơn hay bằng một giới hạn xác định.
	\end{mcq}
	
	
\end{enumerate}
\begin{center}
	\textbf{ĐÁP ÁN}
	
\end{center}

\textbf{1. Thuyết lượng tử ánh sáng}

\begin{longtable}[\textwidth]{|p{0.1\textwidth}|p{0.1\textwidth}|p{0.1\textwidth}|p{0.1\textwidth}|p{0.1\textwidth}|p{0.1\textwidth}|p{0.1\textwidth}|p{0.1\textwidth}|}
	% --- first head
	\hline%\hspace{2 pt}
	\multicolumn{1}{|c|}{\textbf{Câu 1}} & \multicolumn{1}{c|}{\textbf{Câu 2}} & \multicolumn{1}{c|}{\textbf{}} &
	\multicolumn{1}{c|}{\textbf{}} &
	\multicolumn{1}{c|}{\textbf{}} &
	\multicolumn{1}{c|}{\textbf{}} &
	\multicolumn{1}{c|}{\textbf{}} &
	\multicolumn{1}{c|}{\textbf{}}\\
	\hline
	B. &D. & & & & & & \\
	\hline
\end{longtable}	


\textbf{2. Bài toán điều kiện xảy ra hiện tượng quang điện}
\begin{longtable}[\textwidth]{|p{0.1\textwidth}|p{0.1\textwidth}|p{0.1\textwidth}|p{0.1\textwidth}|p{0.1\textwidth}|p{0.1\textwidth}|p{0.1\textwidth}|p{0.1\textwidth}|}
	% --- first head
	\hline%\hspace{2 pt}
	\multicolumn{1}{|c|}{\textbf{Câu 1}} & \multicolumn{1}{c|}{\textbf{Câu 2}} & \multicolumn{1}{c|}{\textbf{Câu 3}} &
	\multicolumn{1}{c|}{\textbf{}} &
	\multicolumn{1}{c|}{\textbf{}} &
	\multicolumn{1}{c|}{\textbf{}} &
	\multicolumn{1}{c|}{\textbf{}} &
	\multicolumn{1}{c|}{\textbf{}}\\
	\hline
	D. &A. &A. & & & & & \\
	\hline
\end{longtable}	


\textbf{3. Hệ thức Anh-xtanh về hiện tượng quang điện}

\begin{longtable}[\textwidth]{|p{0.1\textwidth}|p{0.1\textwidth}|p{0.1\textwidth}|p{0.1\textwidth}|p{0.1\textwidth}|p{0.1\textwidth}|p{0.1\textwidth}|p{0.1\textwidth}|}
	% --- first head
	\hline%\hspace{2 pt}
	\multicolumn{1}{|c|}{\textbf{Câu 1}} & \multicolumn{1}{c|}{\textbf{Câu 2}} & \multicolumn{1}{c|}{\textbf{Câu 3}} &
	\multicolumn{1}{c|}{\textbf{}} &
	\multicolumn{1}{c|}{\textbf{}} &
	\multicolumn{1}{c|}{\textbf{}} &
	\multicolumn{1}{c|}{\textbf{}} &
	\multicolumn{1}{c|}{\textbf{}}\\
	\hline
	A. &A. &C. & & & & & \\
	\hline
\end{longtable}	


\textbf{4. Bài toán về hiệu suất lượng tử}
\begin{longtable}[\textwidth]{|p{0.1\textwidth}|p{0.1\textwidth}|p{0.1\textwidth}|p{0.1\textwidth}|p{0.1\textwidth}|p{0.1\textwidth}|p{0.1\textwidth}|p{0.1\textwidth}|}
	% --- first head
	\hline%\hspace{2 pt}
	\multicolumn{1}{|c}{\textbf{Câu 1}} & \multicolumn{1}{|c|}{\textbf{Câu 2}} & \multicolumn{1}{c|}{\textbf{}} &
	\multicolumn{1}{c|}{\textbf{}} &
	\multicolumn{1}{c|}{\textbf{}} &
	\multicolumn{1}{c|}{\textbf{}} &
	\multicolumn{1}{c|}{\textbf{}} &
	\multicolumn{1}{c|}{\textbf{}}\\
	\hline
	A. &C. & & & & & & \\
	\hline
\end{longtable}	


\textbf{5. Lý thuyết hiện tượng quang điện}

\begin{longtable}[\textwidth]{|p{0.1\textwidth}|p{0.1\textwidth}|p{0.1\textwidth}|p{0.1\textwidth}|p{0.1\textwidth}|p{0.1\textwidth}|p{0.1\textwidth}|p{0.1\textwidth}|}
	% --- first head
	\hline%\hspace{2 pt}
	\multicolumn{1}{|c}{\textbf{Câu 1}} & \multicolumn{1}{|c|}{\textbf{Câu 2}} & \multicolumn{1}{c|}{\textbf{Câu 3}} &
	\multicolumn{1}{c|}{\textbf{Câu 4}} &
	\multicolumn{1}{c|}{\textbf{Câu 5}} &
	\multicolumn{1}{c|}{\textbf{Câu 6}} &
	\multicolumn{1}{c|}{\textbf{Câu 7}} &
	\multicolumn{1}{c|}{\textbf{Câu 8}}\\
	\hline
	D. &A. &D. &C. &A. &A. &C. &B.\\
	\hline
	
	\multicolumn{1}{|c|}{\textbf{Câu 9}} & \multicolumn{1}{c|}{\textbf{Câu 10}} & \multicolumn{1}{c|}{\textbf{Câu 11}} &
	\multicolumn{1}{c|}{\textbf{}} &
	\multicolumn{1}{c|}{\textbf{}} &
	\multicolumn{1}{c|}{\textbf{}} &
	\multicolumn{1}{c|}{\textbf{}} &
	\multicolumn{1}{c|}{\textbf{}} \\
	\hline
	D. &C. &D. & & & & & \\
	\hline		
\end{longtable}















