\chapter{Tính chất và cấu tạo hạt nhân}
\section{Cấu tạo hạt nhân nguyên tử - Dạng 1: Bài toán liên quan đến cấu tạo hạt nhân}
\begin{enumerate}
	\item {[Trích đề thi THPT QG năm 2007] Hạt nhân Triti ($T^3_1$) có
		\begin{mcq}(2)
			\item 3 nuclôn, trong đó có 1 prôtôn.	
			\item 3 nơtrôn và 1 prôtôn.	
			\item 3 nuclôn, trong dó có 1 nơtrôn.	
			\item 3 prôtôn và 1 nơtrôn.
		\end{mcq}
	}
	\item{[Trích đề thi THPT QG năm 2010] So với hạt nhân $^{29}_{14} Si$, hạt nhân $^{40}_{20}Ca$ có nhiều hơn
		\begin{mcq}(2)
			\item 11 nơtrôn và 6 prôtôn.	
			\item 5 nơtrôn và 6 prôtôn.	
			\item 6 nơtrôn và 5 prôtôn.	
			\item 5 nơtrôn và l2 prôtôn.
		\end{mcq}
	}
	\item{[Trích đề thi THPT QG năm 2007] Phát biểu nào là sai?
		\begin{mcq}
			\item  Các đồng vị phóng xạ đều không bền.	
			\item Các nguyên tử mà hạt nhân có cùng số prôtôn nhưng có số nơtrôn (nơtrôn) khác nhau gọi là đồng vị.	
			\item Các đồng vị của cùng một nguyên tố có số nơtrôn khác nhau nên tính chất hóa học khác nhau.
			\item Các đồng vị của cùng một nguyên tố có cùng vị trí trong bảng hệ thống tuần hoàn. 
		\end{mcq}
	}
	\item{Số hạt prôtôn $^1_1 p$ có trong 9 gam nước tinh khiết biết rằng hyđro là đồng vị $^1_1 H$ và ôxy là đồng vị $^{16}_8 O$ xấp xỉ bằng
		\begin{mcq}(4)
			\item $3\cdot 10^2$.	
			\item $3\cdot 10^{24}$.	
			\item $2\cdot 10^{24}$.	
			\item $2\cdot 10^{20}$.
		\end{mcq}
	}
	\item {[Trích đề thi THPT QG năm 2007] Biết số Avôgađrô  là $N_{\text{A}} = \text{6,02} \cdot 10^{23}\ \text{g}/\text{mol}$ và khối lượng mol của uran $^{238}_{92}U$ bằng 238 g/mol. Số nơtrôn có trong 119 gam uran $^{238}_{92}U$ xấp xỉ bằng
		\begin{mcq} (4)
			\item $\text{8,8}\cdot 10^{25}$.	
			\item $\text{1,2}\cdot 10^{25}$.	
			\item $\text{2,2}\cdot 10^{25}$.	
			\item $\text{4,4}\cdot 10^{25}$.
		\end{mcq}
	}
	\item {Uran tự nhiên gồm 3 đồng vị chính là $^{238} U$ có khối lượng nguyên tử 238,0508 u (chiếm $\text{99,27}\%$), $^{235} U$ có khối lượng nguyên tử 235,0439 u (chiếm 0,72$\%$), $^{234} U$ có khối lượng nguyên tử 234,0409 u (chiếm 0,01$\%$). Tính khối lượng trung bình.
		\begin{mcq}(4)
			\item 238,0887 u.	
			\item 238,0587 u.	
			\item 237,0287 u.	
			\item 238,0287 u.
		\end{mcq}
	}
	\item {Nitơ tự nhiên có khối lượng nguyên tử là 14,0067 u gồm 2 đồng vị là $^{14} N$ và $^{15} N$ có khối lượng nguyên tử lần lượt là 14,00307 u và 15,00011 u. Phần trăm của $^{15} N$ trong nitơ tự nhiên bằng
		\begin{mcq}(4)
			\item 0,36$\%$.	
			\item 0,59$\%$.	
			\item 0,43$\%$.	
			\item 0,68$\%$.
		\end{mcq}
	}
\end{enumerate}

\textbf{ĐÁP ÁN}
\begin{longtable}[\textwidth]{|p{0.1\textwidth}|p{0.1\textwidth}|p{0.1\textwidth}|p{0.1\textwidth}|p{0.1\textwidth}|p{0.1\textwidth}|p{0.1\textwidth}|p{0.1\textwidth}|}
	% --- first head
	\hline%\hspace{2 pt}
	\multicolumn{1}{|c|}{\textbf{Câu 1}} & \multicolumn{1}{c|}{\textbf{Câu 2}} & \multicolumn{1}{c|}{\textbf{Câu 3}} &
	\multicolumn{1}{c|}{\textbf{Câu 4}} &
	\multicolumn{1}{c|}{\textbf{Câu 5}} &
	\multicolumn{1}{c|}{\textbf{Câu 6}} &
	\multicolumn{1}{c|}{\textbf{Câu 7}} &
	\multicolumn{1}{c|}{\textbf{Câu 8}} \\
	\hline
	A.&B. &C. &B. &D. &D. &A. &	\\
	\hline
\end{longtable}

\section{Dạng 2: Bài toán liên quan đến năng lượng nghỉ}
\begin{enumerate}
	\item {[Trích đề thi THPT QG năm 2007] Do sự phát bức xạ nên mỗi ngày (86400 s) khối lượng Mặt Trời giảm một lượng $\text{3,744}\cdot 10^{14}\ \text{kg}$. Biết vận tốc ánh sáng trong chân không là $3\cdot 10^8\ \text{m/s}$. Công suất bức xạ (phát xạ) trung bình của Mặt Trời bằng
		\begin{mcq} (4)
			\item $\text{6,9}\cdot 10^{15}\ \text{MW}$.	
			\item $\text{3,9}\cdot 10^{20}\ \text{MW}$.	
			\item $\text{4,9}\cdot 10^{40}\ \text{MW}$.	
			\item $\text{5,9}\cdot 10^{10}\ \text{MW}$.
		\end{mcq}
	}
	\item{[Trích đề thì THPT QG năm 2010] Một hạt có khối lượng nghỉ $m_0$. Theo thuyết tương đối, động năng của hạt này khi chuyển động với tốc độ $\text{0,6}c$ ($c$ là tốc độ ánh sáng trong chân không) là
		\begin{mcq}(4)
			\item $\text{0,36}\ m_0c^2$.	
			\item $\text{1,25}\ m_0c^2$.	
			\item $\text{0,225}\ m_0c^2$.	
			\item $\text{0,25}\ m_0c^2$.
		\end{mcq}
	}
	\item{[Trích đề thi THPT QG năm 2011] Theo thuyết tương đối, một êlectron có động năng bằng một nửa năng lượng nghỉ của nó thì êlectron này chuyển động với tốc độ bằng
		\begin{mcq} (4)
			\item $\text{2,24}\cdot 10^8\ \text{m/s}$.	
			\item $\text{2,75}\cdot 10^8\ \text{m/s}$.	
			\item $\text{1,67}\cdot 10^8\ \text{m/s}$.	
			\item $\text{2,59}\cdot 10^8\ \text{m/s}$.
		\end{mcq}
	}
\end{enumerate}
\textbf{ĐÁP ÁN}
\begin{longtable}[\textwidth]{|p{0.1\textwidth}|p{0.1\textwidth}|p{0.1\textwidth}|p{0.1\textwidth}|p{0.1\textwidth}|p{0.1\textwidth}|p{0.1\textwidth}|p{0.1\textwidth}|}
	% --- first head
	\hline%\hspace{2 pt}
	\multicolumn{1}{|c|}{\textbf{Câu 1}} & \multicolumn{1}{c|}{\textbf{Câu 2}} & \multicolumn{1}{c|}{\textbf{Câu 3}} &
	\multicolumn{1}{c|}{\textbf{Câu 4}} &
	\multicolumn{1}{c|}{\textbf{Câu 5}} &
	\multicolumn{1}{c|}{\textbf{Câu 6}} &
	\multicolumn{1}{c|}{\textbf{Câu 7}} &
	\multicolumn{1}{c|}{\textbf{Câu 8}} \\
	\hline
	B & D & A & && & &	\\
	\hline
\end{longtable}

\section{Dạng 3: Bài toán liên quan đến độ hụt khối, năng lượng liên kết hạt nhân}
\begin{enumerate}
	\item {Cho biết khối lượng hạt nhân $^{234}_{92}U$ là 233,9904 u. Biết khối lượng của hạt prôtôn và nơtrôn lần lượt là $m_p= \text{1,007276}\ \text{u}$ và $m_n= \text{l,008665}\ \text{u}$. Độ hụt khối của hạt nhân $^{234}_{92}U$ bằng
		\begin{mcq}(4)
			\item 1,909422 u.	
			\item 3,460 u.	
			\item 0.	
			\item 2,056 u.
		\end{mcq}
	}
	\item {Cho khối lượng của: prôtôn; nơtrôn và hạt nhân $^4_2He$ lần lượt là: 1,0073 u; 1,0087u và  4,0015u. Lấy $1\ \text{uc}^2 =\text{931,5}\ \text{MeV}$. Năng lượng liên kết của hạt nhân $^4_2He$ là
		\begin{mcq}(4)
			\item 18,3 eV.	
			\item 30,21 MeV.	
			\item 14,21 MeV.	
			\item 28,41 MeV.
		\end{mcq}
	}
	\item{[Trích đề thi THPT QG năm 2007] Cho khối lượng của hạt nhân $C^{12}$ là  $m_C  = \text{l2,00000}\ \text{u}$; $m_p = \text{l,00728}\ \text{u}$; $m_n =\text{1,00867}\ \text{u}$, $1\ \text{u}=\text{1,66058}\cdot 10^{-27}\ \text{kg}$;  $1\ \text{eV}= \text{1,6}\cdot 10^{-19}\ \text{J}$; $c = 3\cdot 10^8\ \text{m/s}$. Năng lượng tối thiểu để tách hạt nhân $C^{12}$ thành các nuclôn riêng biệt là
		\begin{mcq}(4)
			\item 72,7 MeV.	
			\item 89,4 MeV.	
			\item 44,7 MeV.	
			\item 8,94 MeV.
		\end{mcq}
	}
	\item{Cho năng lượng liên kết riêng của hạt nhân $^{56}_{26} Fe$là 8,8 MeV. Biết khối lượng của hạt prôtôn và nơtrôn lần lượt là $m_p = \text{1,007276}\ \text{u}$ và $m_n = \text{1,008665}\ \text{u}$, trong đó $1\ \text{u} = \text{931,5}\ \text{MeV/c}^2$. Khối lượng hạt nhân   là
		\begin{mcq}(4)
			\item 55,9200 u.	
			\item 56,4396 u	
			\item 55,9921 u.	
			\item 56,3810 u.
		\end{mcq}
	}
	\item{[Trích đề thi THPT QG năm 2018] Hạt nhân $^{90}_{60} Zr$ có năng lượng liên kết là 783 MeV. Năng lượng liên kết riêng của hạt nhân này là
		\begin{mcq}(2)
			\item l9,6 MeV/nuclôn.	
			\item 6,0 MeV/nuclôn.	
			\item 8,7 MeV/nuclôn.	
			\item 15,6 MeV/nuclôn.
		\end{mcq}
	}
	\item{[Trích đề thi THPT QG năm 2010] Cho ba hạt nhân X, Y và Z có số nuclôn tương ứng là $\text{A}_{\text{X}}$, $\text{A}_{\text{Y}}$, $\text{A}_{\text{Z}}$ với $\text{A}_{\text{X}} = 2\text{A}_{\text{Y}} = \text{0,5}\ \text{A}_{\text{Z}}$. Biết năng lượng liên kết của từng hạt nhân tương ứng là $\Delta E_{\text{X}}$, $\Delta E_{\text{Y}}$, $\Delta E_{\text{Z}}$ với $\Delta E_{\text{Z}} < \Delta E_{\text{X}} < \Delta E_{\text{Y}}$. Sắp xếp các hạt nhân này theo thứ tự tính bền vững giảm dần là
		\begin{mcq}(4)
			\item Y, X, Z.	
			\item Y, Z, X.	
			\item X, Y, Z.	
			\item Z, X, Y.
		\end{mcq}
	}
	\item{[Trích đề thi THPT QG năm 2010] Cho khối lượng của prôtôn; nơtrôn; $^{40}_{18} Ar$; $^6_3 Li$ lần lượt 1à: 1,0073 u; 1,0087 u; 39,9525 u; 6,0145 u và l u = 931,5 $\text{MeV/c}^2$. So với năng lượng liên kết riêng của hạt nhân $^6_3 Li$ thì năng lượng liên kết riêng của hạt $^{40}_{18} Ar$. 
		\begin{mcq}(2)
			\item lớn hơn một lượng là 5,20 MeV.	
			\item lớn hơn một lượng là 3,42 MeV.	
			\item nhỏ hơn một lượng là 3,42 MeV.	
			\item nhỏ hơn một lượng là 5,20 MeV.
		\end{mcq}
	}
\end{enumerate}
\textbf{ĐÁP ÁN}
\begin{longtable}[\textwidth]{|p{0.1\textwidth}|p{0.1\textwidth}|p{0.1\textwidth}|p{0.1\textwidth}|p{0.1\textwidth}|p{0.1\textwidth}|p{0.1\textwidth}|p{0.1\textwidth}|}
	% --- first head
	\hline%\hspace{2 pt}
	\multicolumn{1}{|c|}{\textbf{Câu 1}} & \multicolumn{1}{c|}{\textbf{Câu 2}} & \multicolumn{1}{c|}{\textbf{Câu 3}} &
	\multicolumn{1}{c|}{\textbf{Câu 4}} &
	\multicolumn{1}{c|}{\textbf{Câu 5}} &
	\multicolumn{1}{c|}{\textbf{Câu 6}} &
	\multicolumn{1}{c|}{\textbf{Câu 7}} &
	\multicolumn{1}{c|}{\textbf{Câu 8}} \\
	\hline
	A.&D. &B. &B. &C. &A. &C. &	\\
	\hline
	
\end{longtable}



