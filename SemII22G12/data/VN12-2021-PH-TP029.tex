\whiteBGstarBegin
\setcounter{section}{0}
\section{Tán sắc ánh sáng}
\begin{enumerate}[label=\bfseries Câu \arabic*:]
	\item \mkstar{1} 
	
	\cauhoi
	{Gọi $n_\text{đ}$, $n_\text{t}$ và $n_\text{v}$ lần lượt là chiết suất của một môi trường trong suốt đối với các ánh sáng đơn sắc đỏ, tím và vàng. Sắp xếp nào sau đây đúng?
		\begin{mcq}(2)
			\item $n_\text{đ}<n_\text{v}<n_\text{t}$. 
			\item $n_\text{v}>n_\text{đ}>n_\text{t}$. 
			\item $n_\text{đ}>n_\text{t}>n_\text{v}$. 
			\item $n_\text{t}>n_\text{đ}>n_\text{v}$. 
		\end{mcq}
	}
	
	\loigiai
	{		\textbf{Đáp án: A.}
		
Chiết suất màu tím là lớn nhất, màu đỏ là nhỏ nhất.
		
	}
	
	\item \mkstar{2}
	
	\cauhoi
	{Bước sóng của ánh sáng đỏ trong không khí là $\SI{0.64}{\micro \meter}$. Tính bước sóng của ánh sáng đỏ trong nước biết chiết suất của nước đối với ánh sáng đỏ là $\dfrac{4}{3}$.
		\begin{mcq}(4)
			\item $\SI{0.24}{\micro \meter}$. 
			\item $\SI{0.48}{\micro \meter}$. 
			\item $\SI{0.36}{\micro \meter}$. 
			\item $\SI{0.54}{\micro \meter}$. 
		\end{mcq}
	}
	
	\loigiai
	{		\textbf{Đáp án: B.}
		
	Bước sóng của ánh sáng đỏ:
	$$
		\lambda_{n} = \dfrac{\lambda_{kk}}{n_{n}} = \SI{0.48}{\micro \meter}.
	$$
		
	}
	
	\item \mkstar{3} 
	
	\cauhoi
	{Một lăng kính có góc chiết quang là $\ang{60}$. Biết chiết suất của lăng kính đối với ánh sáng đỏ là $\SI{1.5}{}$. Chiếu tia sáng màu đỏ vào mặt bên của lăng kính với góc tới $\ang{60}$. Góc lệch của tia ló và tia tới là
		\begin{mcq}(4)
			\item $\ang{\SI{60.0}{}}$. 
			\item $\ang{\SI{40.0}{}}$. 
			\item $\ang{\SI{38.8}{}}$. 
			\item $\ang{\SI{42.1}{}}$. 
		\end{mcq}
	}
	
	\loigiai
	{		\textbf{Đáp án: C.}
		
Tại mặt bên thứ nhất của lăng kính:
$$
	\sin i_{1} = n \sin r_{1} \rightarrow r_{1} = \ang{\num{35,26}}.
$$
Ta có:
$$
	A = r_{1} + r_{2} \rightarrow r_{2} = \ang{\num{24,74}}.
$$
Tại mặt bên thứ hai của lăng kính:
$$
	\sin i_{2} = n \sin r_{2} \rightarrow r_{2} = \ang{\num{38,8}}.
$$
	}
	
	\item \mkstar{3} 
	
	\cauhoi
	{Chiếu một tia sáng đơn sắc màu vàng từ không khí (chiết suất coi như bằng 1 đối với mọi ánh sáng) vào mặt phẳng phân cách của một khối chất rắn trong suốt với góc tới $\ang{60}$ thì thấy tia phản xạ trở lại không khí vuông góc với tia khúc xạ đi vào khối chất rắn. Chiết suất của chất rắn trong suốt đó đối với ánh sáng màu vàng là
		\begin{mcq}(4)
			\item $1$. 
			\item $\sqrt 5$. 
			\item $\sqrt 2$. 
			\item $\sqrt 3$. 
		\end{mcq}
	}
	
	\loigiai
	{		\textbf{Đáp án: D.}
		

Theo định luật phản xạ ánh sáng:
$$
	i' = i = \ang{60}.
$$
Tia phản xạ vuông góc với tia khúc xạ nên: 
$$
	r + i' = \ang{\num{90}} \rightarrow r = \ang{30}.
$$
Áp dụng định luật khúc xạ ánh sáng:
$$
	n = \dfrac{\sin i}{\sin r} = \sqrt{3}.
$$
	}
	
	\item \mkstar{4}
	
	\cauhoi
	{Một tia sáng trắng chiếu tới mặt bên của một lăng kính thuỷ tinh tam giác đều. Tia ló màu vàng qua lăng kính có góc lệch cực tiểu. Biết chiết suất của lăng kính đối với ánh sáng vàng, ánh sáng tím lần lượt là $n_\text{v} = \SI{1.50}{}; n_\text{t}=\SI{1.52}{}$. Góc tạo bởi tia ló màu vàng và tia ló màu tím có giá trị xấp xỉ bằng
		\begin{mcq}(4)
			\item $\ang{\SI{20.77}{}}$. 
			\item $\ang{\SI{48.59}{}}$. 
			\item $\ang{\SI{40.46}{}}$. 
			\item $\ang{\SI{10.73}{}}$. 
		\end{mcq}
	}
	
	\loigiai
	{		\textbf{Đáp án: A.}
		
Trong trường hợp ánh sáng vàng, ta có góc lệch cực tiểu:
$$
	r_{1_v} = r_{2_v} = A/2 = \ang{\num{30}}.
$$
Tại các mặt bên của lăng kính với ánh áng vàng:
$$
	\sin i_{1} = \sin i_{2_v} = n_{v} \sin r_{1_v} \rightarrow i_{1} = i_{2_v} = \ang{\num{48,59}}.
$$
Tại mặt bên thứ nhất của lăng kính với ánh sáng tím:
$$
	\sin i_{1} = n_{t} \sin r_{1_t} \rightarrow r_{1_t} = \ang{\num{29,57}}.
$$
Ta có:
$$
	A = r_{1_t} + r_{2_t} \rightarrow r_{2_t} = \ang{\num{30,43}}.
$$
Tại mặt bên thứ hai của lăng kính với ánh sáng tím:
$$
	\sin i_{2_t} = n_{t} \sin r_{2_t} \rightarrow i_{2_t} = \ang{\num{50,34}}.
$$
Vậy, góc hợp bởi tia lò màu vàng và tia ló màu tím có giá trị là
$$
	| i_{2_v} - i_{2_t} | = \ang{\num{20,77}}.
$$
		
	}
	
	\item \mkstar{3} 
	
	\cauhoi
	{Một lăng kính có góc chiết quang $\ang{\SI{6.0}{}}$ (coi là góc nhỏ) được đặt trong không khí. Chiếu một chùm ánh sáng trắng song song, hẹp vào mặt bên của lăng kính theo phương vuông góc với mặt phẳng phân giác của góc chiết quang, rất gần cạnh của lăng kính. Đặt một màn ảnh E sau lăng kính, vuông góc với phương của chùm tia tới và cách mặt phẳng phân giác của góc chiết quang $\SI{1.2}{\meter}$. Chiết suất của lăng kính đối với ánh sáng đỏ là $n_\text{đ}=\SI{1.642}{}$ và đối với ánh sáng tím là $n_\text{t}=\SI{1.685}{}$. Độ rộng từ màu đỏ đến màu tím của quang phổ liên tục quan sát được trên màn là
		\begin{mcq}(4)
			\item $\SI{5.4}{\milli \meter}$. 
			\item $\SI{36.9}{\milli \meter}$. 
			\item $\SI{4.5}{\milli \meter}$. 
			\item $\SI{10.1}{\milli \meter}$. 
		\end{mcq}
	}
	
	\loigiai
	{		\textbf{Đáp án: A.}
		
Gọi $ \Delta D $ là độ rộng từ màu đỏ đến màu tím của quang phổ liên tục quan sát được trên màn: 
$$
	\Delta D = (n_{t} - n_{\text{đ}})\cdot A \cdot D = \SI{5,4}{mm}.
$$
		
	}
	
	\item \mkstar{2}
	
	\cauhoi
	{Phát biểu nào trong các phát biểu dưới đây là đúng khi nói về hiện tượng tán sắc ánh sáng và ánh sáng đơn sắc?
		\begin{mcq}(1)
			\item Hiện tượng tán sắc ánh sáng là hiện tượng khi qua lăng kính, chùm ánh sáng trắng không những là bị lệch về phía đáy mà còn bị tách ra thành chiều chùm sáng có màu sắc khác nhau. 
			\item Mỗi ánh sáng đơn sắc có một màu nhất định. 
			\item Trong quang phổ của ánh sáng trắng có vô số các ánh sáng đơn sắc khác nhau. 
			\item Cả A, B, C đều đúng. 
		\end{mcq}
	}
	
	\loigiai
	{		\textbf{Đáp án: D.}
		
Tất cả nhận định A, B, C đều đúng.
		
	}
	
	\item \mkstar{2} 
	
	\cauhoi
	{Phát biểu nào sau đây là \textbf{không đúng}?
		\begin{mcq}
			\item Ánh sáng trắng là tập hợp của vô số các ánh sáng đơn sắc có màu biến đổi liên tục từ đỏ đến tím. 
			\item Chiết suất của chất làm lăng kính đối với các ánh sáng đơn sắc là khác nhau. 
			\item Ánh sáng đơn sắc không bị tán sắc khi đi qua lăng kính. 
			\item Khi chiếu một chùm ánh sáng Mặt Trời đi qua một cặp hai môi trường trong suốt thì tia tím bị lệch về phía mặt phân cách hai môi trường nhiều hơn tia đỏ. 
		\end{mcq}
	}
	
	\loigiai
	{		\textbf{Đáp án: D.}
		
Tia tím có lệch về mặt phân cách giữa hai môi trường hay không còn phụ thuộc vào bản chất của hai môi trường.
		
	}
	
	\item \mkstar{2}
	
	\cauhoi
	{Khi ánh sáng trắng bị tán sắc thì
		\begin{mcq}(2)
			\item màu đỏ lệch nhiều nhất. 
			\item màu tím lệch nhiều nhất. 
			\item màu tím lệch ít nhất. 
			\item ánh sáng trắng tách ra thành 7 màu. 
		\end{mcq}
	}
	
	\loigiai
	{		\textbf{Đáp án: B.}
		
Khi ánh sáng trắng bị tán sắc thì màu tím lệch nhiều nhất. 
		
	}
	
\end{enumerate}

\loigiai
{
	\begin{center}
		\textbf{BẢNG ĐÁP ÁN}
	\end{center}
	\begin{center}
		\begin{tabular}{|m{2.8em}|m{2.8em}|m{2.8em}|m{2.8em}|m{2.8em}|m{2.8em}|m{2.8em}|m{2.8em}|m{2.8em}|m{2.8em}|}
			\hline
			01.A & 02.B  & 03.C  & 04.D  & 05.A  & 06.A  & 07.D & 08.D & 09.B & \\
			\hline
			
		\end{tabular}
	\end{center}
}

\section{Giao thoa ánh sáng}
\begin{enumerate}[label=\bfseries Câu \arabic*:]
	\item \mkstar{1} 
	
	\cauhoi
	{Trong thí nghiệm Y-âng, vân tối thứ nhất xuất hiện ở trên màn tại các vị trí cách vân sáng trung tâm là
		\begin{mcq}(4)
			\item $i/4$. 
			\item $i/2$. 
			\item $i$. 
			\item $2i$. 
		\end{mcq}
	}
	
	\loigiai
	{		\textbf{Đáp án: B.}
		
Vị trí vân tối thứ nhất là $ i/2 $.
		
	}
	
	\item \mkstar{2}
	
	\cauhoi
	{Khoảng cách từ vân sáng bậc 4 bên này đến vân sáng bậc 5 bên kia so với vân sáng trung tâm là
		\begin{mcq}(4)
			\item $7i$. 
			\item $8i$. 
			\item $9i$. 
			\item $10i$. 
		\end{mcq}
	}
	
	\loigiai
	{		\textbf{Đáp án: C.}
		
Khoảng cách từ vân sáng bậc 4 bên này đến vân sáng bậc 5 bên kia là $ 9i $.
	}
	
	\item \mkstar{2} 
	
	\cauhoi
	{Khoảng cách từ vân sáng bậc 5 đến vân sáng bậc 9 ở cùng phía so với vân sáng trung tâm là
		\begin{mcq}(4)
			\item $4i$. 
			\item $5i$. 
			\item $14i$. 
			\item $13i$. 
		\end{mcq}
	}
	
	\loigiai
	{		\textbf{Đáp án: A.}
		
Khoảng cách từ vân sáng bậc 5 đến vân sáng bậc 9 ở cùng phía so với vân sáng trung tâm là $ 4i $.
	}
	
	\item \mkstar{2} 
	
	\cauhoi
	{Trong thí nghiệm Y-âng về giao thoa ánh sáng, khoảng cách giữa hai khe sáng là $\SI{0.2}{\milli \meter}$, khoảng cách từ hai khe sáng đến màn ảnh là $D=\SI{1}{\meter}$, khoảng vân đo được là $i=\SI{2}{\milli \meter}$. Bước sóng của ánh sáng là
		\begin{mcq}(4)
			\item $\SI{0.4}{\micro \meter}$. 
			\item $\SI{4}{\micro \meter}$. 
			\item $\SI{0.4e-3}{\micro \meter}$. 
			\item $\SI{0.4e-4}{\micro \meter}$. 
		\end{mcq}
	}
	
	\loigiai
	{		\textbf{Đáp án: A.}
		
Khoảng vân cho bởi:
$$
	i = \dfrac{\lambda D}{a} \rightarrow \lambda = \SI{0,4}{\mu m}.
$$	
	}
	
	\item \mkstar{2}
	
	\cauhoi
	{Trong thí nghiệm Y-âng về giao thoa ánh sáng, biết $a=\SI{0.4}{\milli \meter}$, $D=\SI{1.2}{\meter}$, nguồn S phát ra bức xạ đơn sắc có bước sóng $\lambda = \SI{600}{\nano \meter}$. Khoảng cách giữa 2 vân sáng liên tiếp trên màn là
		\begin{mcq}(4)
			\item $\SI{1.6}{\milli \meter}$. 
			\item $\SI{1.2}{\milli \meter}$. 
			\item $\SI{1.8}{\milli \meter}$. 
			\item $\SI{1.4}{\milli \meter}$. 
		\end{mcq}
	}
	
	\loigiai
	{		\textbf{Đáp án: C.}
		
Khoảng cách giữa hai vân sáng liên tiếp là khoảng vân
$$
	i = 2 \cdot \dfrac{\lambda D}{a} = \SI{1,8}{mm}.
$$
		
	}
	
\item \mkstar{3} 
	
	\cauhoi
	{Trong thí nghiệm Y-âng về giao thoa ánh sáng, biết $a=\SI{5}{\milli \meter}$, $D=\SI{2}{\meter}$. Khoảng cách giữa 6 vân sáng liên tiếp là $\SI{1.5}{\milli \meter}$. Bước sóng của ánh sáng đơn sắc là
		\begin{mcq}(4)
			\item $\SI{0.6}{\micro \meter}$. 
			\item $\SI{0.9}{\micro \meter}$. 
			\item $\SI{0.7}{\micro \meter}$. 
			\item $\SI{0.75}{\micro \meter}$. 
		\end{mcq}
	}
	
	\loigiai
	{		\textbf{Đáp án: D.}
	
Khoảng cách giữa 6 vân sáng liên tiếp cho bởi:
$$
	\Delta d = 5 \rightarrow i = \SI{0,3}{mm}.
$$
		
Bước sóng của ánh sáng cho bởi:
$$
	i = \dfrac{\lambda D}{a} \rightarrow \lambda = \SI{0,75}{\micro m}
$$
		
	}
	
\item \mkstar{3} 
	
	\cauhoi
	{Trong thí nghiệm Y-âng về giao thoa ánh sáng, các khe sáng được chiếu bằng ánh sáng đơn sắc. Khoảng cách giữa hai khe là $\SI{2}{\milli \meter}$, khoảng cách từ hai khe đến màn là $\SI{4}{\meter}$. Khoảng cách giữa 5 vân sáng liên tiếp đo được là $\SI{4.8}{\milli \meter}$. Toạ độ của vân sáng bậc 3 là
		\begin{mcq}(4)
			\item $\pm \SI{9.6}{\milli \meter}$. 
			\item $\pm \SI{4.8}{\milli \meter}$. 
			\item $\pm \SI{3.6}{\milli \meter}$. 
			\item $\pm \SI{2.4}{\milli \meter}$. 
		\end{mcq}
	}
	
	\loigiai
	{		\textbf{Đáp án: C.}
		
Khoảng cách giữa 5 vân sáng liên tiếp cho bởi:
$$
	\Delta d = 4i \rightarrow i = \SI{1,2}{mm}.
$$
Tọa độ của vân sáng bậc 3 là
$$
	x_{3} = 3i = \SI{3,6}{mm}.
$$
		
	}
	
\item \mkstar{3} 
	
	\cauhoi
	{Trong thí nghiệm Y-âng, khoảng cách giữa hai khe là $a=\SI{2}{\milli \meter}$, khoảng cách từ hai khe đến màn là $D=\SI{2}{\meter}$. Vân sáng thứ 3 cách vân sáng trung tâm $\SI{1.8}{\milli \meter}$. Bước sóng ánh sáng đơn sắc dùng trong thí nghiệm là
		\begin{mcq}(4)
			\item $\SI{0.4}{\micro \meter}$. 
			\item $\SI{0.55}{\micro \meter}$. 
			\item $\SI{0.5}{\micro \meter}$. 
			\item$\SI{0.6}{\micro \meter}$. 
		\end{mcq}
	}
	
	\loigiai
	{		\textbf{Đáp án: D.}
		
Vị trí vân sáng thứ ba là
$$
	x_{3} = 3i \rightarrow i = \SI{0,6}{mm}.
$$
Bước sóng ánh sáng đơn sắc dùng trong thí nghiệm là
$$
	i = \dfrac{\lambda D}{a} \rightarrow \lambda = \SI{0,6}{\mu m}.
$$
		
	}
	
\item \mkstar{1} 
	\cauhoi
	{Trong thí nghiệm Y-âng về giao thoa ánh sáng, khoảng cách giữa hai khe là $a=\SI{2}{\milli \meter}$, khoảng cách từ hai khe đến màn là $D=\SI{2}{\meter}$, ánh sáng đơn sắc có bước sóng $\SI{0.5}{\micro \meter}$. Khoảng cách từ vân sáng bậc 1 đến vân sáng bậc 10 là
		\begin{mcq}(4)
			\item $\SI{4.5}{\milli \meter}$. 
			\item $\SI{5.5}{\milli \meter}$. 
			\item $\SI{4.0}{\milli \meter}$. 
			\item $\SI{5.0}{\milli \meter}$. 
		\end{mcq}
	}
	
	\loigiai
	{		\textbf{Đáp án: A.}
		
Khoảng cách từ vân sáng bậc 1 đến vân sáng bậc 10 là
$$
	\Delta d = 10i - i = 9i = 9 \cdot \dfrac{\lambda D}{a} = \SI{4,5}{mm}.
$$
		
	}
	
\item \mkstar{1} 
	
	\cauhoi
	{Trong thí nghiệm về giao thoa ánh sáng, khoảng cách giữa 2 khe hẹp là $a=\SI{1}{\milli \meter}$, từ 2 khe đến màn ảnh là $D=\SI{1}{\meter}$. Dùng ánh sáng đỏ có bước sóng $\lambda_\text{đ}=\SI{0.75}{\micro \meter}$, khoảng cách từ vân sáng thứ 4 đến vân sáng thứ 10 ở cùng phía so với vân trung tâm là
		\begin{mcq}(4)
			\item $\SI{2.8}{\milli \meter}$. 
			\item $\SI{3.6}{\milli \meter}$. 
			\item $\SI{4.5}{\milli \meter}$. 
			\item $\SI{5.2}{\milli \meter}$. 
		\end{mcq}
	}
	
	\loigiai
	{		\textbf{Đáp án: C.}
		
Khoảng cách từ vân sáng thứ 4 đến vân sáng thứ 10 ở cùng phía so với vân trung tâm là
$$
	\Delta d = 10i - 4i = 6i = 6 \cdot \dfrac{\lambda D}{a} = \SI{4,5}{mm}.
$$
		
	}
	
\item \mkstar{3} 
	
	\cauhoi
	{Ánh sáng đơn sắc trong thí nghiệm Y-âng là $\SI{0.5}{\micro \meter}$. Khoảng cách từ hai nguồn đến màn là $\SI{1}{\meter}$, khoảng cách giữa hai nguồn là $\SI{2}{\milli \meter}$. Khoảng cách giữa vân sáng bậc 3 và vân tối bậc 5 ở hai bên so với vân trung tâm là
		\begin{mcq}(4)
			\item $\SI{0.375}{\milli \meter}$. 
			\item $\SI{1.875}{\milli \meter}$. 
			\item $\SI{18.75}{\milli \meter}$. 
			\item $\SI{3.75}{\milli \meter}$. 
		\end{mcq}
	}
	
	\loigiai
	{		\textbf{Đáp án: B.}
		
Khoảng cách giữa vân sáng bậc 3 và vân tối bậc 5 ở hai bên so với bên trung tâm là
$$
	\Delta d = 3i + \num{4,5}i = \num{7,5}i = \num{7,5} \cdot \dfrac{\lambda D}{a} = \SI{1,875}{mm}.
$$
		
	}
	
\item \mkstar{3} 
	
	\cauhoi
	{Trong thí nghiệm Y-âng về giao thoa của ánh sáng đơn sắc, hai khe hẹp cách nhau $\SI{1}{\milli \meter}$, mặt phẳng chứa hai khe cách màn quan sát $\SI{1.5}{\meter}$. Khoảng cách giữa 5 vân sáng liên tiếp là $\SI{3.6}{\milli \meter}$. Bước sóng của ánh sáng dùng trong thí nghiệm này bằng
		\begin{mcq}(4)
			\item $\SI{0.48}{\micro \meter}$. 
			\item $\SI{0.40}{\micro \meter}$. 
			\item $\SI{0.60}{\micro \meter}$. 
			\item $\SI{0.76}{\micro \meter}$. 
		\end{mcq}
	}
	
	\loigiai
	{		\textbf{Đáp án: C.}
	
Khoảng cách giữa 5 vân sáng liên tiếp là
$$
	\Delta d = 4i \rightarrow i = \SI{0,9}{mm}.
$$
Bước sóng ánh sáng dùng trong thí nghiệm là
$$
	i = \dfrac{\lambda D}{a} \rightarrow \lambda = \SI{0,6}{\mu m}.
$$
		
	}
	
	\item \mkstar{3} 
	
	\cauhoi
	{Trong thí nghiệm Y-âng về giao thoa với ánh sáng đơn sắc, khoảng cách giữa hai khe là $\SI{1}{\milli \meter}$, khoảng cách từ mặt phẳng chứa hai khe đến màn quan sát là $\SI{2}{\meter}$ và khoảng vân là $\SI{0.8}{\milli \meter}$. Cho $c=\SI{3e8}{\meter / \second}$. Tần số ánh sáng đơn sắc dùng trong thí nghiệm là
		\begin{mcq}(4)
			\item $\SI{5.5e14}{\hertz}$. 
			\item $\SI{4.5e14}{\hertz}$. 
			\item $\SI{7.5e14}{\hertz}$. 
			\item $\SI{6.5e14}{\hertz}$. 
		\end{mcq}
	}
	
	\loigiai
	{		\textbf{Đáp án: C.}
		
Khoảng vân cho bởi:
$$
	i = \dfrac{\lambda D}{a} \rightarrow \lambda = \SI{0,4}{\mu m}.
$$
Tần số ánh sáng cho bởi:
$$
	f = \dfrac{c}{\lambda} = \SI{7,5 e14}{Hz}.
$$
		
	}
	
\item \mkstar{3} 
	
	\cauhoi
	{Trong thí nghiệm giao thoa ánh sáng dùng hai khe Y-âng, hai khe được chiếu bằng ánh sáng có bước sóng $\lambda=\SI{0.5}{\micro \meter}$, biết $\text S_1 \text S_2 = a = \SI{0.5}{\milli \meter}$, khoảng cách từ mặt phẳng chứa hai khe đến màn quan sát là $D=\SI{1}{\meter}$. Tại điểm M cách vân trung tâm một khoảng $x=\SI{3.5}{\milli \meter}$, có vân sáng hay vân tối, bậc mấy?
		\begin{mcq}(2)
			\item Vân sáng bậc 3. 
			\item Vân tối thứ 4. 
			\item Vân sáng bậc 4. 
			\item Vân tối thứ 2. 
		\end{mcq}
	}
	
	\loigiai
	{		\textbf{Đáp án: B.}
		
Khoảng vân cho bởi:
$$
	i = \dfrac{\lambda D}{a} = \SI{1}{mm}.
$$
Ta có:
$$
	\dfrac{x}{i} = \num{3,5}.
$$
Vậy tại $ x $ là vân tối thứ tư.
	}
	
	\item \mkstar{3} 
	
	\cauhoi
	{Giao thoa ánh sáng đơn sắc của Y-âng có $\lambda=\SI{0.5}{\micro \meter}$, $a=\SI{0.5}{\milli \meter}$, $D=\SI{2}{\meter}$. Tại M cách vân trung tâm $\SI{7}{\milli \meter}$ và tại điểm N cách vân trung tâm $\SI{10}{\milli \meter}$ thì
		\begin{mcq}(2)
			\item M, N đều là vân sáng. 
			\item M là vân tối, N là vân sáng. 
			\item M, N đều là vân tối. 
			\item M là vân sáng, N là vân tối. 
		\end{mcq}
	}
	
	\loigiai
	{		\textbf{Đáp án: B.}
		
Khoảng vân cho bởi
$$
	i = \dfrac{\lambda D}{a} = \SI{2}{mm}.
$$
Ta có $ \dfrac{x_{M}}{i}=\num{3,5} $ và $ \dfrac{x_{N}}{i}=\num{5} $. Nên tại $ M $ là vân tối thứ 4 và tại $ N $ là vân sáng bậc 5.
		
	}
	
	\item \mkstar{3} 
	
	\cauhoi
	{Trong thí nghiệm giao thoa ánh sáng, khi $a=\SI{2}{\milli \meter}$, $D=\SI{2}{\meter}$, $\lambda=\SI{0.6}{\micro \meter}$ thì khoảng cách giữa hai vân sáng bậc 4 ở hai bên vân trung tâm là
		\begin{mcq}(4)
			\item $\SI{4.8}{\milli \meter}$. 
			\item $\SI{1.2}{\centi \meter}$. 
			\item $\SI{2.4}{\milli \meter}$. 
			\item $\SI{4.8}{\centi \meter}$. 
		\end{mcq}
	}
	
	\loigiai
	{		\textbf{Đáp án: A.}
		
Khoảng cách giữa hai vân sáng bậc 4 ở hai bên trung tâm cho bởi:
$$
	\Delta d = 8i = 8 \cdot \dfrac{\lambda D}{a} = \SI{4,8}{mm}.
$$
		
	}
	
	\item \mkstar{3} 
	
	\cauhoi
	{Một nguồn sáng đơn sắc S cách hai khe Y-âng $\SI{0.2}{\milli \meter}$ phát ra một bức xạ đơn sắc có $\lambda=\SI{0.64}{\micro \meter}$. Hai khe cách nhau $a=\SI{3}{\milli \meter}$, màn cách hai khe $\SI{3}{\meter}$. Trường giao thoa trên màn có bề rộng $\SI{12}{\milli \meter}$. Số vân tối quan sát được trên màn là
		\begin{mcq}(4)
			\item $16$. 
			\item $17$. 
			\item $18$. 
			\item $19$. 
		\end{mcq}
	}
	
	\loigiai
	{		\textbf{Đáp án: C.}
		
Khoảng vân cho bởi:
$$
	i = \dfrac{\lambda D}{a} = \SI{0,64}{mm}.
$$
Ta có:
$$
	\dfrac{L}{2i} = \num{9,375}.
$$
Số vân tối quan sát được trên màn là $ \num{18} $.
	}
	
\item \mkstar{3} 
	
	\cauhoi
	{Trong thí nghiệm Y-âng về giao thoa ánh sáng, khoảng cách giữa hai khe là $\SI{1.5}{\milli \meter}$, khoảng cách từ hai khe đến màn là $\SI{3}{\meter}$, người ta đo được khoảng cách giữa vân sáng bậc 2 đến vân sáng bậc 5 ở cùng phía với nhau so với vân sáng trung tâm là $\SI{3}{\milli \meter}$. Tìm số vân sáng quan sát được trên vùng giao thoa đối xứng có bề rộng $\SI{11}{\milli \meter}$.
		\begin{mcq}(4)
			\item $9$. 
			\item $10$. 
			\item $11$. 
			\item $12$. 
		\end{mcq}
	}
	
	\loigiai
	{		\textbf{Đáp án: C.}
		
Khoảng cách giữa vân sáng bậc 2 và vân sáng bậc 5 cho bởi:
$$
	\Delta d = 5i - 2i = 3i \rightarrow i = \SI{1}{mm}.
$$
Ta có:
$$
	\dfrac{L}{2i} = \num{5,5}.
$$
Số vân sáng quan sát được trên vùng giao thoa là $ \num{11} $.
		
	}
	
	\item \mkstar{3}
	
	\cauhoi
	{Người ta thực hiện giao thoa ánh sáng đơn sắc với hai khe Y-âng cách nhau $\SI{0.5}{\milli \meter}$, khoảng cách giữa hai khe đến màn là $\SI{2}{\meter}$, ánh sáng dùng có bước sóng $\lambda=\SI{0.5}{\micro \meter}$. Bề rộng của trường giao thoa đối xứng là $\SI{18}{\milli \meter}$. Số vân sáng, vân tối lần lượt là
		\begin{mcq}(2)
			\item $N_1=11, N_2=12$. 
			\item $N_1=7, N_2=8$. 
			\item $N_1=9, N_2=10$. 
			\item $N_1=13, N_2=14$. 
		\end{mcq}
	}
	
		
	\loigiai
	{		\textbf{Đáp án: C.}

Khoảng vân cho bởi:
$$
	i = \dfrac{\lambda D}{a} = \SI{2}{mm}.
$$
Ta có:
$$
	\dfrac{L}{2i}=\num{4,5}.
$$
Vậy trên trường giao thoa có $ 9 $ vân sáng và $ 10 $ vân tối.
		
	}
	

	
	\item \mkstar{3} 
	
	\cauhoi
	{Trong thí nghiệm Y-âng về giao thoa ánh sáng, nguồn sáng đơn sắc có $\lambda=\SI{0.5}{\micro \meter}$, khoảng cách giữa hai khe là $a=\SI{2}{\milli \meter}$. Trong khoảng MN trên màn với $\text{MO}=\text{ON}=\SI{5}{\milli \meter}$ có 11 vân sáng mà hai mép M và N là hai vân sáng. Khoảng cách từ hai khe đến màn quan sát là
		\begin{mcq}(4)
			\item $D=\SI{2}{\meter}$. 
			\item $D=\SI{2.4}{\meter}$. 
			\item $D=\SI{3}{\meter}$. 
			\item $D=\SI{4}{\meter}$. 
		\end{mcq}
	}
	
	\loigiai
	{		\textbf{Đáp án: D.}
		
Ta có:
$$
	MN = 10i \rightarrow i = \SI{1}{mm}.
$$
Ta có:
$$
	i = \dfrac{\lambda D}{a} \rightarrow D = \SI{4}{m}.
$$
		
	}
	
\item \mkstar{3} 
	
	\cauhoi
	{Bề rộng vùng giao thoa (đối xứng) quan sát được trên màn là $\text{MN}=\SI{30}{\milli \meter}$, khoảng cách giữa hai vân tối liên tiếp bằng $\SI{2}{\milli \meter}$. Trên MN quan sát thấy 
		\begin{mcq}(2)
			\item 16 vân tối, 15 vân sáng. 
			\item 15 vân tối, 16 vân sáng. 
			\item 14 vân tối, 15 vân sáng. 
			\item 16 vân tối, 16 vân sáng. 
		\end{mcq}
	}
	
	\loigiai
	{		\textbf{Đáp án: B.}
		
Khoảng cách giữa hai vân tối liên tiếp cũng chính là khoảng vân. Vậy nên $ i = \SI{2}{mm} $. \\
Ta có:
$$
	\dfrac{L}{2i} = \num{7,5}.
$$
Vậy số vân sáng là $ 15 $ vân. Số vân tối là $ 16 $ vân.
	}
	
	\item \mkstar{3}
	
	\cauhoi
	{Trong thí nghiệm giao thoa khe Young, khoảng cách giữa hai khe $\text F _1 \text F_2$ là $a=\SI{2}{\milli \meter}$, khoảng cách từ hai khe $\text F_1 \text F_2$ đến màn là $D=\SI{1.5}{\meter}$, dùng ánh sáng đơn sắc có bước sóng $\lambda=\SI{0.6}{\micro \meter}$. Xét trên khoảng MN, với $\text{MO}=\SI{5}{\milli \meter}$, $\text{ON}=\SI{10}{\milli \meter}$ (O là vị trí vân sáng trung tâm), MN nằm cùng phía vân sáng trung tâm. Số vân sáng trong đoạn MN là
		\begin{mcq}(4)
			\item $11$. 
			\item $12$. 
			\item $13$. 
			\item $15$. 
		\end{mcq}
	}
	
	\loigiai
	{		\textbf{Đáp án: A.}
		
Khoảng vân cho bởi:
$$
	i = \dfrac{\lambda D}{a} = \SI{0,45}{mm}.
$$
Ta có:
$$
	x_{N} \leq ki \leq x_{M} \rightarrow \num{11,1} \leq k \leq \num{22,2}.
$$
Vậy có $ 11 $ vân sáng năm trên đoạn $ MN $.
	}
	
	\item \mkstar{3}
	
	\cauhoi
	{Trong thí nghiệm giao thoa khe Young, khoảng cách giữa hai khe $\text F _1 \text F_2$ là $a=\SI{2}{\milli \meter}$, khoảng cách từ hai khe $\text F_1 \text F_2$ đến màn là $D=\SI{1.5}{\meter}$, dùng ánh sáng đơn sắc có bước sóng $\lambda=\SI{0.6}{\micro \meter}$. Xét trên khoảng MN, với $\text{MO}=\SI{5}{\milli \meter}$, $\text{ON}=\SI{10}{\milli \meter}$ (O là vị trí vân sáng trung tâm), MN nằm khác phía vân sáng trung tâm. Số vân sáng trong đoạn MN là
		\begin{mcq}(4)
			\item $31$. 
			\item $32$. 
			\item $33$. 
			\item $34$. 
		\end{mcq}
	}
	
	\loigiai
	{		\textbf{Đáp án: D.}
		
Khoảng vân cho bởi:
$$
	i = \dfrac{\lambda D}{a} = \SI{0,45}{\mu m}.
$$
Ta có:
$$
	-x_{M} \leq ki \leq x_{N} \rightarrow \num{-11,1} \leq k \leq \num{22,2}.
$$
Vậy có $ 34 $ vân sáng nằm trong đoạn MN.
	}
	
	\item \mkstar{3} 
	
	\cauhoi
	{Trong thí nghiệm Y-âng về giao thoa ánh sáng, hai khe cách nhau $a=\SI{0.5}{\milli \meter}$ được chiếu sáng bằng ánh sáng đơn sắc. Khoảng cách từ hai khe đến màn quan sát là $\SI{2}{\meter}$. Trên màn quan sát, trong vùng giữa hai điểm M và N mà $\text{MN}=\SI{2}{\centi \meter}$, người ta đếm được có 10 vân tối và thấy tại M và N đều là vân sáng. Bước sóng của ánh sáng đơn sắc dùng trong thí nghiệm này là
		\begin{mcq}(4)
			\item $\SI{0.4}{\micro \meter}$. 
			\item $\SI{0.5}{\micro \meter}$. 
			\item $\SI{0.6}{\micro \meter}$. 
			\item $\SI{0.7}{\micro \meter}$. 
		\end{mcq}
	}
	
	\loigiai
	{		\textbf{Đáp án: B.}
		
Khi tại $ M $ và $ N $ đều là vân sáng, thì số vân tối trên $ MN $ cũng chính là số khoảng vân:
$$
	MN = 10i \rightarrow i = \SI{2}{mm}.
$$
Ta có:
$$
	i = \dfrac{\lambda D}{a} \rightarrow \lambda = \SI{0,5}{\mu m}.
$$
		
	}
	
\item \mkstar{3}
	
	\cauhoi
	{Trong thí nghiệm Y-âng về giao thoa ánh sáng với ánh sáng đơn sắc, khoảng cách giữa hai khe là $\SI{1}{\milli \meter}$, khoảng cách từ hai khe tới màn $\SI{2}{\meter}$. Trong đoạn rộng $\SI{12.5}{\milli \meter}$ trên màn có 13 vân tối biết một đầu là vân tối còn một đầu là vân sáng. Bước sóng của ánh sáng đơn sắc đó là
		\begin{mcq}(4)
			\item $\SI{0.48}{\micro \meter}$. 
			\item $\SI{0.52}{\micro \meter}$. 
			\item $\SI{0.5}{\micro \meter}$. 
			\item $\SI{0.46}{\micro \meter}$. 
		\end{mcq}
	}
	
	\loigiai
	{		\textbf{Đáp án: C.}
		
Trên đoạn có một đầu là vân tối, một đầu là vân sáng. Trên đó lại có $ 13 $ vân tối. Vậy nên:
$$
	\Delta d = 12,5i \rightarrow i = \SI{1}{mm}.
$$
Khoảng vân cho bởi:
$$
	i = \dfrac{\lambda D}{a} \rightarrow \lambda = \SI{0,5}{\mu m}.
$$
		
	}
	
	\item \mkstar{3}
	
	\cauhoi
	{Trong thí nghiệm giao thoa ánh sáng của Y-âng, khoảng cách hai khe là $\SI{0.2}{\milli \meter}$, ánh sáng đơn sắc làm thí nghiệm có bước sóng $\SI{0.6}{\micro \meter}$. Lúc đầu, màn cách hai khe $\SI{1.6}{\meter}$. Tịnh tiến màn theo phương vuông góc mặt phẳng chứa hai khe một đoạn $d$ thì tại vị trí vân sáng bậc 3 lúc đầu trùng vân sáng bậc 2. Màn được tịnh tiến
		\begin{mcq}(2)
			\item xa hai khe $\SI{150}{\centi \meter}$. 
			\item gần hai khe $\SI{80}{\centi \meter}$. 
			\item xa hai khe $\SI{80}{\centi \meter}$. 
			\item gần hai khe $\SI{150}{\centi \meter}$. 
		\end{mcq}
	}
	
	\loigiai
	{		\textbf{Đáp án: C.}
		
Tại vị trí vân sáng bậc $ 3 $ lúc đầu là vân sáng bậc $ 2 $. Suy ra:
$$
	3 \cdot \dfrac{\lambda D}{a} = 2 \cdot \dfrac{\lambda (D+d)}{a} \rightarrow d = \SI{80}{cm}.
$$
		
	}
	
	\item \mkstar{3} 
	
	\cauhoi
	{Trong thí nghiệm Y-âng, khoảng cách giữa 9 vân sáng liên tiếp là $L$. Dịch chuyển màn $\SI{36}{\centi \meter}$ theo phương vuông góc với màn thì khoảng cách giữa 11 vân sáng liên tiếp cũng là $L$. Khoảng cách giữa màn và hai khe lúc đầu là
		\begin{mcq}(4)
			\item $\SI{1.8}{\meter}$. 
			\item $\SI{2}{\meter}$. 
			\item $\SI{2.5}{\meter}$. 
			\item $\SI{1.5}{\meter}$. 
		\end{mcq}
	}
	
	\loigiai
	{		\textbf{Đáp án: A.}
		
Ban đầu, ta có: $ L = 8i. $ Lúc sau, ta có: $ L = 10i' $. Suy ra:
$$
	8 \cdot \dfrac{\lambda D}{a} = 10 \cdot \dfrac{\lambda (D-d)}{a} \rightarrow D = \SI{1,8}{m}.
$$
			
	}
	
	\item \mkstar{3} 
	
	\cauhoi
	{Thực hiện giao thoa ánh sáng đơn sắc với khoảng cách từ mặt phẳng chứa hai khe đến màn hứng vân giao thoa là $D=\SI{2}{\meter}$ và tại ví trí M đang có vân sáng bậc 4. Cần phải thay đổi khoảng cách $D$ nói trên một khoảng bao nhiêu thì tại M có vân tối thứ 6?
		\begin{mcq}(2)
			\item giảm đi $\xsi{2/9}{\meter}$. 
			\item tăng thêm $\xsi{8/11}{\meter}$. 
			\item tăng thêm $\xsi{0.4}{\meter}$. 
			\item giảm $\xsi{6/11}{\meter}$. 
		\end{mcq}
	}	
	
	\loigiai
	{		\textbf{Đáp án: D.}
		
Ta có:
$$
	x_{M} = 4 \cdot \dfrac{\lambda D}{a} = 5,5 \cdot \dfrac{\lambda (D-d)}{a} \rightarrow d = \xsi{6/11}{m}.
$$
		
	}
	
	\item \mkstar{3} 
	
	\cauhoi
	{Trong thí nghiệm giao thoa Y-âng, nguồn S phát ánh sáng đơn sắc có bước sóng $\lambda$ người ta đặt màn quan sát cách mặt phẳng hai khe một khoảng $D$ thì khoảng vân $i=\SI{1}{\milli \meter}$. Khi khoảng cách từ màn quan sát đến mặt phẳng hai khe lần lượt là $D+\Delta D$ hoặc $D-\Delta D$ thì khoảng vân thu được trên màn tương tứng là $2i$ và $i$. Nếu khoảng cách từ màn quan sát đến mặt phẳng hai khe là $D+3\Delta D$ thì khoảng vân trên màn là
		\begin{mcq}(4)
			\item $\SI{3}{\milli \meter}$. 
			\item $\SI{4}{\milli \meter}$. 
			\item $\SI{2}{\milli \meter}$. 
			\item $\SI{2.5}{\milli \meter}$. 
		\end{mcq}
	}
	
	\loigiai
	{		\textbf{Đáp án: C.}
		
Ban đầu:
$$
	i_{1} = \dfrac{\lambda D}{a} = 1.
$$
Lúc sau:
$$
	i_{2} = \dfrac{\lambda (D + \Delta D)}{a} = 2i. 
$$
Và:
$$
i_{2}' = \dfrac{\lambda (D - \Delta D)}{a} = i. 
$$
Suy ra:
$$
	3\Delta D = D.
$$		
Vậy:
$$
	i_{3} = \dfrac{\lambda (D + 3\Delta D)}{a} = 2i = \SI{2}{mm}.
$$
	}
	
	\item \mkstar{3} 
	
	\cauhoi
	{Trong thí nghiệm Y-âng về giao thoa ánh sáng, hai khe được chiếu bằng ánh sáng đơn sắc có bước sóng $\SI{0.6}{\micro \meter}$. Khoảng cách giữa hai khe sáng là $\SI{1}{\milli \meter}$, khoảng cách từ mặt phẳng chứa hai khe đến màn quan sát là $\SI{1.5}{\meter}$. Trên màn quan sát, hai vân sáng bậc 4 nằm ở hai điểm M và N. Dịch màn quan sát một đoạn $\SI{50}{\centi \meter}$ theo hướng ra xa hai khe Y-âng thì số vân sáng quan sát trên đoạn MN giảm so với lúc đầu là
		\begin{mcq}(4)
			\item 7 vân. 
			\item 4 vân.  
			\item 6 vân. 
			\item 2 vân. 
		\end{mcq}
	}
	
	\loigiai
	{		\textbf{Đáp án: D.}
		
Ta có:
$$
	\dfrac{i'}{i} =  \dfrac{D + \Delta D}{D} = \dfrac{4}{3} \rightarrow i' = \dfrac{4}{3} i.
$$
Ban đầu, tại $ M $ và $ N $ là vị trí của vân sáng bậc 4 nên $ MN = 8i $.			
Ta có:
$$
	\dfrac{MN}{2i'} = \dfrac{8i}{2\dfrac{2}{3}i} = \num{3}.
$$
Vậy số vân sáng trên đoạn $ MN $ lúc sau là $ 7 $ vân, giảm đi $ 2 $ vân so với lúc ban đầu.
	}
	
	\item \mkstar{3} 
	
	\cauhoi
	{Trong thí nghiệm Y-âng về giao thoa ánh sáng, hai khe được chiếu bằng ánh sáng đơn sắc $\lambda$, màn quan sát cách mặt phẳng hai khe một khoảng không đổi $D$, khoảng cách giữa hai khe có thể thay đổi (nhưng $\text S_1$ và $\text S_2$ luôn cách đều S). Xét điểm M trên màn, lúc đầu là vân sáng bậc 4, nếu lần lượt giảm hoặc tăng khoảng cách $\text S_1 \text S_2$ một lượng $\Delta a$ thì tại đó là vân sáng bậc $k$ và bậc $3k$. Nếu tăng khoảng cách $\text S_1 \text S_2$ thêm $2\Delta a$ thì tại M là
		\begin{mcq}(2)
			\item vân sáng bậc 7. 
			\item vân sáng bậc 8. 
			\item vân tối thứ 7. 
			\item vân tối thứ 8. 
		\end{mcq}
	}
	
	\loigiai
	{		\textbf{Đáp án: B.}
		
Ta có:
$$
	x_{M} = 4 \cdot \dfrac{\lambda D}{a} = k \dfrac{\lambda (D + \Delta a)}{a} = 3k \cdot \dfrac{\lambda (D - \Delta a)}{a} \rightarrow D = 2\Delta a.
$$
Vậy khi tăng khoảng cách giữa hai khe thêm $ 2\Delta a $, ta có:
$$
	x_{M} = 4 \cdot \dfrac{D}{a} = n \cdot \dfrac{D}{2\Delta a + a} \rightarrow n = \num{8}.
$$
		
	}
	
	\item \mkstar{3} 
	
	\cauhoi
	{Trong thí nghiệm Y-âng về giao thoa ánh sáng, hai khe được chiếu bằng ánh sáng trắng có bước sóng từ $\SI{0.38}{\micro \meter}$ đến $\SI{0.76}{\micro \meter}$. Khoảng cách giữa hai khe là $\SI{0.8}{\milli \meter}$, khoảng cách từ mặt phẳng chứa hai khe đến màn quan sát là $\SI{1.2}{\meter}$. Độ rộng quang phổ bậc 3 (nằm về một phía so với vân sáng trung tâm) là
		\begin{mcq}(4)
			\item $\SI{0.57}{\milli \meter}$. 
			\item $\SI{1.14}{\milli \meter}$. 
			\item $\SI{1.71}{\milli \meter}$. 
			\item $\SI{2.36}{\milli \meter}$. 
		\end{mcq}
	}
	
	\loigiai
	{		\textbf{Đáp án: C.}
		
Độ rộng quang phổ bậc 3 là
$$
	\Delta d_{3} = 3 \cdot \dfrac{(\lambda_{\text{đ}} - \lambda_{\text{t}})D}{a} = \SI{1,71}{mm}.
$$
		
	}
	
	\item \mkstar{3} 
	
	\cauhoi
	{Trong thí nghiệm giao thoa ánh sáng khe Y-âng, khoảng cách giữa hai khe $\text S_1, \text S_2$ bằng $\SI{1}{\milli \meter}$, khoảng cách từ hai khe đến màn quan sát là $D=\SI{2}{\meter}$. Chiếu vào 2 khe bằng chùm sáng trắng có bước sóng $\lambda$ ($\SI{0.38}{\micro \meter} \leq \lambda \leq \SI{0.76}{\micro \meter}$). Bề rộng đoạn chồng chập của quang phổ bậc 5 và quang phổ bậc 7 trên trường giao thoa bằng
		\begin{mcq}(4)
			\item $\Delta x= \SI{0.76}{\milli \meter}$. 
			\item $\Delta x= \SI{2.28}{\milli \meter}$. 
			\item $\Delta x= \SI{1.14}{\milli \meter}$. 
			\item $\Delta x= \SI{1.44}{\milli \meter}$. 
		\end{mcq}
	}
	
	\loigiai
	{		\textbf{Đáp án: B.}
		
Bề rộng đoạn chồng chập của quang phổ bậc 5 và quang phổ bậc 7 trên trường giao thoa bằng
$$
	\Delta d = 5 \cdot \dfrac{\lambda_{max} D}{a} - 7 \cdot \dfrac{\lambda_{min} D}{a} = \SI{2.28}{\milli \meter}. 
$$
			
	}
	
	\item \mkstar{3} 
	
	\cauhoi
	{Thực hiện giao thoa với ánh sáng trắng có bước sóng $\SI{0.4}{\micro \meter} \leq \lambda \leq \SI{0.7}{\micro \meter}$. Hai khe cách nhau $\SI{2}{\milli \meter}$, màn hứng vân giao thoa cách hai khe $\SI{2}{\meter}$. Tại điểm M cách vân trung tâm $\SI{3.3}{\milli \meter}$ có bao nhiêu ánh sáng đơn sắc cho vân sáng tại đó?
		\begin{mcq}(4)
			\item $5$. 
			\item $3$. 
			\item $4$. 
			\item $2$. 
		\end{mcq}
	}
	
	\loigiai
	{		\textbf{Đáp án: C.}
		
Điều kiện một vân bức xạ cho vân sáng tại $ M $ là
$$
	x_{M} = k \cdot \dfrac{\lambda D}{a} \rightarrow \lambda = \xsi{\dfrac{3,3}{k}}{\mu m}.
$$
Lại có:
$$
	\SI{0.4}{\micro \meter} \leq \lambda \leq \SI{0.7}{\micro \meter} \rightarrow \num{4,7} \leq k \leq \num{8,3}.
$$
Vậy có $ 4 $ bức xạ cho vân sáng tại $ M $.
		
	}
	
	\item \mkstar{1} 
	
	\cauhoi
	{Hai khe Y-âng cách nhau $\SI{1}{\milli \meter}$ được chiếu sáng bằng ánh sáng trắng ($\SI{0.4}{\micro \meter} \leq \lambda \leq \SI{0.76}{\micro \meter}$), khoảng cách từ hai khe đến màn là $\SI{1}{\meter}$. Tại điểm A trên màn cách vân trung tâm $\SI{2}{\milli \meter}$ có các bức xạ cho vân sáng có bước sóng
		\begin{mcq}(2)
			\item $\SI{0.40}{\micro \meter}$, $\SI{0.50}{\micro \meter}$ và $\SI{0.66}{\micro \meter}$. 
			\item $\SI{0.44}{\micro \meter}$, $\SI{0.50}{\micro \meter}$ và $\SI{0.66}{\micro \meter}$. 
			\item $\SI{0.40}{\micro \meter}$, $\SI{0.44}{\micro \meter}$ và $\SI{0.50}{\micro \meter}$. 
			\item$\SI{0.40}{\micro \meter}$, $\SI{0.44}{\micro \meter}$ và $\SI{0.66}{\micro \meter}$. 
		\end{mcq}
	}
	
	\loigiai
	{		\textbf{Đáp án: A.}
		
Điều kiện một vân bức xạ cho vân sáng tại $ M $ là
$$
	x_{M} = k \cdot \dfrac{\lambda D}{a} \rightarrow \lambda = \xsi{\dfrac{2}{k}}{\mu m}.
$$
Lại có:
$$
	\SI{0,4}{\mu m} \leq \lambda \leq \SI{0,76}{\mu m} \rightarrow \num{2,63} \leq k \leq \num{5}.
$$
Vậy, các bức xạ có có bước sóng thỏa mãn là $ \SI{0,66}{\mu m} $, $ \SI{0,50}{\mu m} $ và $ \SI{0,4}{\mu m} $.
	}
	
	\item \mkstar{3} 
	
	\cauhoi
	{Thực hiện giao thoa ánh sáng qua khe Y-âng, biết khoảng cách giữa hai khe $\SI{0.5}{\milli \meter}$, khoảng cách từ màn chứa hai khe tới màn quan sát là $\SI{2}{\meter}$. Nguồn S phát ánh sáng trắng gồm vô số bức xạ đơn sắc có bước sóng từ $\SI{0.4}{\micro \meter}$ đến $\SI{0.75}{\micro \meter}$. Hỏi ở đúng vị trí vân sáng bậc 4 của bức xạ đỏ còn có bao nhiêu bức xạ cho vân sáng nằm trùng tại đó?
		\begin{mcq}(4)
			\item 3. 
			\item 4. 
			\item 5. 
			\item 6. 
		\end{mcq}
	}
	
	\loigiai
	{		\textbf{Đáp án: A.}
		
Điều kiện cho vân sáng tại vị trí vân sáng bậc $ 4 $ của bức xạ đỏ:
$$
	4 \cdot \dfrac{\lambda_{/text{đ}} D}{a} = k \cdot \dfrac{\lambda D}{a} \rightarrow \lambda = \xsi{3}{k}.
$$
Ta có:
$$
	\SI{0.4}{\micro \meter} \leq \lambda \leq \SI{0.75}{\micro \meter} \rightarrow \num{4} \leq k \leq \num{7,5}.
$$
Vậy còn có $ 3 $ vân sáng khác nữa cũng trùng tại vị trí vân sáng bậc 4 của bức xạ đỏ.
	}
	
	\item \mkstar{3} 
	
	\cauhoi
	{Trong thí nghiệm giao thoa ánh sáng bằng khe Y-âng. Khoảng cách giữa hai khe kết hợp là $a=\SI{2}{\milli \meter}$, khoảng cách từ hai khe đến màn là $D=\SI{2}{\meter}$. Nguồn S phát ra ánh sáng trắng có bước sóng từ $\SI{380}{\nano \meter}$ đến $\SI{760}{\nano \meter}$. Vùng phủ nhau giữa quang phổ bậc hai và quang phổ bậc ba có bề rộng là
		\begin{mcq}(4)
			\item $\SI{0.76}{\milli \meter}$. 
			\item $\SI{0.38}{\milli \meter}$. 
			\item $\SI{1.14}{\milli \meter}$. 
			\item $\SI{1.52}{\milli \meter}$. 
		\end{mcq}
	}
	
	\loigiai
	{		\textbf{Đáp án: B.}
		
Vùng phủ nhau giữa quang phổ bậc 2 và bậc 3 là
$$
	\Delta d = 2i_{max} - 3i_{min} = \dfrac{(2\lambda_{max} - 3\lambda_{min})D}{a} = \SI{0.38}{\milli \meter}.
$$
		
	}
	
\end{enumerate}

\loigiai
{
	\begin{center}
		\textbf{BẢNG ĐÁP ÁN}
	\end{center}
	\begin{center}
		\begin{tabular}{|m{2.8em}|m{2.8em}|m{2.8em}|m{2.8em}|m{2.8em}|m{2.8em}|m{2.8em}|m{2.8em}|m{2.8em}|m{2.8em}|}
			\hline
			01.B  & 02.C  & 03.A  & 04.A  & 05.C  & 06.D  & 07.C & 08.D & 09.A & 10.C \\
			\hline
			11.B  & 12.C  & 13.C  & 14.B  & 15.B  & 16.A  & 17.C & 18.C & 19.C & 20.D \\
			\hline
			21.B  & 22.A  & 23.D  & 24.B  & 25.C  & 26.C  & 27.A & 28.D & 29.C & 30.D \\
			\hline
			31.B  & 32.C  & 33.B  & 34.C  & 35.A  & 36.A  & 37.B & & & \\
			\hline
			
		\end{tabular}
	\end{center}
}

\section{Các loại quang phổ. Tia hồng ngoại, tia tử ngoại và tia X}
\begin{enumerate}[label=\bfseries Câu \arabic*:]
	\item \mkstar{1} 
	
	\cauhoi
	{Quang phổ vạch phát xạ
		\begin{mcq}
			\item của các nguyên tố khác nhau, ở cùng một nhiệt độ thì như nhau về độ sáng tỉ đối của các vạch. 
			\item là một hệ thống những vạch sáng (vạch màu) riêng lẻ, ngăn cách nhau bởi những khoảng tối. 
			\item do các chất rắn, chất lỏng hoặc chất khí có áp suất lớn phát ra khi bị nung nóng. 
			\item là một dải có màu từ đỏ đến tím nối liền nhau một cách liên tục. 
		\end{mcq}
	}
	
	\loigiai
	{		\textbf{Đáp án: B.}
		
Quang phổ vạch phát xạ là một hệ thống những vạch sáng (vạch màu) riêng lẻ, ngăn cách nhau bởi những khoảng tối.
		
	}
	
	\item \mkstar{1} 
	
	\cauhoi
	{Phát biểu nào sau đây là đúng?
		\begin{mcq}
			\item Chất khí hay hơi ở áp suất thấp được kích thích bằng nhiệt hay bằng điện cho quang phổ liên tục. 
			\item Chất khí hay hơi được kích thích bằng nhiệt hay bằng điện luôn cho quang phổ vạch. 
			\item Quang phổ liên tục của nguyên tố nào thì đặc trưng cho nguyên tố ấy. 
			\item Quang phổ vạch của nguyên tố nào thì đặc trưng cho nguyên tố ấy. 
		\end{mcq}
	}
	
	\loigiai
	{		\textbf{Đáp án: D.}
		
Quang phổ vạch của nguyên tố nào thì đặc trưng cho nguyên tố ấy.
		
	}
	
	\item \mkstar{1} 
	
	\cauhoi
	{Ống chuẩn trực trong máy quang phổ có tác dụng
		\begin{mcq}
			\item tạo ra chùm tia sáng song song. 
			\item tập trung ánh sáng chiếu vào lăng kính. 
			\item tăng cường độ sáng. 
			\item tán sắc ánh sáng. 
		\end{mcq}
	}
	
	\loigiai
	{		\textbf{Đáp án: D.}
		
Quang phổ vạch của nguyên tố nào thì đặc trưng cho nguyên tố ấy. 
		
	}
	
	\item \mkstar{1} 
	
	\cauhoi
	{Khi nói về quang phổ, phát biểu nào sau đây là đúng?
		\begin{mcq}
			\item Các chất rắn bị nung nóng thì phát ra quang phổ vạch. 
			\item Mỗi nguyên tố hóa học có một quang phổ vạch đặc trưng của nguyên tố ấy. 
			\item Các chất khí ở áp suất lớn bị nung nóng thì phát ra quang phổ vạch. 
			\item Quang phổ liên tục của nguyên tố nào thì đặc trưng cho nguyên tố đó. 
		\end{mcq}
	}
	
	\loigiai
	{		\textbf{Đáp án: B.}
		
Mỗi nguyên tố hóa học có một quang phổ vạch đặc trưng của nguyên tố ấy. 
		
	}
	
	\item \mkstar{1} 
	
	\cauhoi
	{Phát biểu nào sau đây là đúng khi nói về quang phổ?
		\begin{mcq}
			\item Quang phổ liên tục của nguồn sáng nào thì phụ thuộc thành phần cấu tạo của nguồn sáng ấy. 
			\item Mỗi nguyên tố hóa học ở trạng thái khí hay hơi nóng sáng dưới áp suất thấp cho một quang phổ vạch riêng, đặc trưng cho nguyên tố đó. 
			\item Để thu được quang phổ hấp thụ thì nhiệt độ của đám khí hay hơi hấp thụ phải cao hơn nhiệt độ của nguồn sáng phát ra quang phổ liên tục. 
			\item Quang phổ hấp thụ là quang phổ của ánh sáng do một vật rắn phát ra khi vật đó được nung nóng. 
		\end{mcq}
	}
	
	\loigiai
	{		\textbf{Đáp án: B.}
		
Mỗi nguyên tố hóa học ở trạng thái khí hay hơi nóng sáng dưới áp suất thấp cho một quang phổ vạch riêng, đặc trưng cho nguyên tố đó. 
		
	}
	
\item \mkstar{1} 
	
	\cauhoi
	{Phát biểu nào sau đây là \textbf{không đúng}?
		\begin{mcq}
			\item Trong máy quang phổ, ống chuẩn trực có tác dụng tạo ra chùm tia sáng song song. 
			\item Trong máy quang phổ, buồng ảnh nằm ở phía sau lăng kính. 
			\item Trong máy quang phổ, lăng kính có tác dụng phân tích chùm ánh sáng phức tạp song song thành các chùm sáng đơn sắc song song. 
			\item Trong máy quang phổ, quang phổ của một chùm sáng thu được trong buồng ảnh luôn là một dải sáng có màu cầu vồng. 
		\end{mcq}
	}
	
	\loigiai
	{		\textbf{Đáp án: D.}
		
Trong máy quang phổ, quang phổ của một chùm sáng thu được trong buồng ảnh luôn là một dải sáng có màu cầu vồng là không đúng.
		
	}
	
	\item \mkstar{1} 
	
	\cauhoi
	{Hiện tượng quang học nào sau đây sử dụng trong máy phân tích quang phổ?
		\begin{mcq}
			\item Hiện tượng khúc xạ ánh sáng. 
			\item Hiện tượng phản xạ ánh sáng. 
			\item Hiện tượng giao thoa ánh sáng. 
			\item Hiện tượng tán sắc ánh sáng. 
		\end{mcq}
	}
	
	\loigiai
	{		\textbf{Đáp án: D.}
		
Hiện tượng quang học sử dụng trong máy phân tích quang phổ là hiện tượng tán sắc ánh sáng.
		
	}
	
	\item \mkstar{1} 
	
	\cauhoi
	{Máy quang phổ là dụng cụ dùng để
		\begin{mcq}(2)
			\item đo bước sóng các vạch quang phổ. 
			\item tiến hành các phép phân tích quang phổ. 
			\item quan sát và chụp quang phổ của các vật. 
			\item phân tích một chùm ánh sáng phức tạp thành những thành phần đơn sắc. 
		\end{mcq}
	}
	
	\loigiai
	{		\textbf{Đáp án: D.}
		
Máy quang phổ là dụng cụ dùng để phân tích một chùm ánh sáng phức tạp thành những thành phần đơn sắc. 
		
	}
	
	\item \mkstar{1} 
	
	\cauhoi
	{Những chất nào sau đây phát ra quang phổ liên tục?
		\begin{mcq}(1)
			\item Chất khí ở nhiệt độ cao. 
			\item Chất rắn ở nhiệt độ thường. 
			\item Hơi kim loại ở nhiệt độ cao. 
			\item Chất khí có áp suất lớn, ở nhiệt độ cao. 
		\end{mcq}
	}
	
	\loigiai
	{		\textbf{Đáp án: B.}
		
Chất rắn ở nhiệt độ thường phát ra quang phổ liên tục.
		
	}
	
	\item \mkstar{1} 
	
	\cauhoi
	{Đặc điểm quan trọng của quang phổ liên tục là
		\begin{mcq}
			\item chỉ phụ thuộc vào thành phần cấu tạo và nhiệt độ của nguồn sáng. 
			\item chỉ phụ thuộc vào thành phần cấu tạo của nguồn sáng và không phụ thuộc vào nhiệt độ của nguồn sáng. 
			\item không phụ thuộc vào thành phần cấu tạo của nguồn sáng và chỉ phụ thuộc vào nhiệt độ của nguồn sáng. 
			\item không phụ thuộc vào thành phần cấu tạo của nguồn sáng và không phụ thuộc vào nhiệt độ của nguồn sáng. 
		\end{mcq}
	}
	
	\loigiai
	{		\textbf{Đáp án: C.}
		
Đặc điểm quan trọng của quang phổ liên tục là không phụ thuộc vào thành phần cấu tạo của nguồn sáng và chỉ phụ thuộc vào nhiệt độ của nguồn sáng. 
		
	}
	
\item \mkstar{1} 
	
	\cauhoi
	{Quang phổ của nguồn sáng nào sau đây không phải là quang phổ liên tục?
		\begin{mcq}(1)
			\item Sợi dây tóc nóng sáng trong bóng đèn. 
			\item Một đèn LED đỏ đang nóng sáng. 
			\item Mặt trời. 
			\item Miếng sắt nung nóng. 
		\end{mcq}
	}
	
	\loigiai
	{		\textbf{Đáp án: B.}
		
Ánh sáng từ mặt trời, sợi dây tóc nóng sáng, miếng sắt nung nóng là quang phổ liên tục.
		
	}
	
	\item \mkstar{1} 
	
	\cauhoi
	{Để nhận biết sự có mặt của nguyên tố hoá học trong một mẫu vật, ta phải nghiên cứu loại quang phổ nào của mẫu đó?
		\begin{mcq}(2)
			\item Quang phổ vạch phát xạ. 
			\item Quang phổ liên tục. 
			\item Quang phổ hấp thụ. 
			\item Cả ba loại quang phổ trên. 
		\end{mcq}
	}
	
	\loigiai
	{		\textbf{Đáp án: A.}
		
Để nhận biết sự có mặt của nguyên tố hoá học trong một mẫu vật, ta phải nghiên cứu loại quang phổ vạch phát xạ.
		
	}
	
	\item \mkstar{1} 
	
	\cauhoi
	{Quang phổ vạch phát xạ được phát ra do
		\begin{mcq}
			\item các chất khi hay hơi ở áp suất thấp khi bị kích thích phát sáng. 
			\item chiếu ánh sáng trắng qua chất khi hay hơi bị nung nóng. 
			\item các chất rắn, lỏng hoặc khí khi bị nung nóng. 
			\item các chất rắn, lỏng hoặc khí có tỉ khối lớn khi bị nung nóng. 
		\end{mcq}
	}
	
	\loigiai
	{		\textbf{Đáp án: A.}
		
Quang phổ vạch phát xạ được phát ra do các chất khi hay hơi ở áp suất thấp khi bị kích thích phát sáng. 
		
	}
	
	\item \mkstar{2} 
	
	\cauhoi
	{Tìm phát biểu sai. Hai nguyên tố khác nhau có đặc điểm quang phổ vạch phát xạ khác nhau về
		\begin{mcq}
			\item số lượng các vạch quang phổ
			\item bề rộng các vạch quang phổ. 
			\item độ sáng tỉ đối giữa các vạch quang phổ. 
			\item màu sắc các vạch và vị trí các vạch màu. 
		\end{mcq}
	}
	
	\loigiai
	{		\textbf{Đáp án: A.}
		
Hai nguyên tố khác nhau có đặc điểm quang phổ vạch phát xạ không nhất thiết khác nhau về số lượng các vạch quang phổ.
		
	}
	
	\item \mkstar{1} 
	
	\cauhoi
	{Phát biểu nào sau đây là \textbf{không đúng}?
		\begin{mcq}
			\item Quang phổ vạch phát xạ của các nguyên tố khác nhau thì khác nhau về số lượng vạch màu, màu sắc vạch, vị trí và độ sáng tỉ đối của các vạch quang phổ. 
			\item Mỗi nguyên tố hoá học ở trạng thái khí hay hơi ở áp suất thấp được kích thích phát sáng có một quang phổ vạch phát xạ đặc trưng. 
			\item Quang phổ vạch phát xạ là những dải màu biến đổi liên tục nằm trên một nền tối. 
			\item Quang phổ vạch phát xạ là một hệ thống các vạch sáng màu nằm riêng rẽ trên một nền tối. 
		\end{mcq}
	}
	
	\loigiai
	{		\textbf{Đáp án: C.}
		
Quang phổ vạch phát xạ là những dải màu biến đổi liên tục nằm trên một nền tối là sai. 
		
	}
	
\item \mkstar{1} 
	
	\cauhoi
	{Chọn câu đúng khi nói về quang phổ liên tục?
		\begin{mcq}
			\item Quang phổ liên tục của một vật phụ thuộc vào bản chất của vật nóng sáng. 
			\item Quang phổ liên tục phụ thuộc vào nhiệt độ của vật nóng sáng. 
			\item Quang phổ liên tục không phụ thuộc vào nhiệt độ và bản chất của vật nóng sáng. 
			\item Quang phổ liên tục phụ thuộc cả nhiệt độ và bản chất của vật nóng sáng. 
		\end{mcq}
	}
	
	\loigiai
	{		\textbf{Đáp án: B.}
		
Quang phổ liên tục phụ thuộc vào nhiệt độ của vật nóng sáng. 
		
	}
	
	\item \mkstar{1} 
	
	\cauhoi
	{Nguồn sáng phát ra quang phổ vạch phát xạ là
		\begin{mcq}(1)
			\item mặt trời. 
			\item khối sắt nóng chảy. 
			\item bóng đèn nê-on của bút thử điện. 
			\item ngọn lửa đèn cồn trên có rắc vài hạt muối. 
		\end{mcq}
	}
	
	\loigiai
	{		\textbf{Đáp án: D.}
		
Nguồn sáng phát ra quang phổ vạch phát xạ là ngọn lửa đèn cồn trên có rắc vài hạt muối. 
		
	}
	
	\item \mkstar{1} 
	
	\cauhoi
	{Khi nói về tia hồng ngoại, phát biểu nào dưới đây là \textbf{sai}?
		\begin{mcq}
			\item Tia hồng ngoại cũng có thể biến điệu được như sóng điện từ cao tần. 
			\item Tia hồng ngoại có khả năng gây ra một số phản ứng hóa học. 
			\item Tia hồng ngoại có tần số lớn hơn tần số của ánh sáng đỏ. 
			\item Tác dụng nổi bật nhất của tia hồng ngoại là tác dụng nhiệt. 
		\end{mcq}
	}
	
	\loigiai
	{		\textbf{Đáp án: C.}
		
Tia hồng ngoại có tần số nhỏ hơn tần số của ánh sáng đỏ.
		
	}
	
	\item \mkstar{1} 
	
	\cauhoi
	{Ánh sáng đơn sắc có tần số $5\cdot 10^{14}\ \text{Hz}$ truyền trong chân không với bước sóng $600\ \text{nm}$. Chiết suất tuyệt đối của một môi trường trong suốt ứng với ánh sáng này là $\text{1,52}$. Tần số của ánh sáng trên khi truyền trong môi trường trong suốt này
		\begin{mcq}
			\item nhỏ hơn $5\cdot 10^{14}\ \text{Hz}$ còn bước sóng bằng $600\ \text{nm}$. 
			\item lớn hơn $5\cdot 10^{14}\ \text{Hz}$ còn bước sóng nhỏ hơn $600\ \text{nm}$. 
			\item vẫn bằng $5\cdot 10^{14}\ \text{Hz}$ còn bước sóng nhỏ hơn $600\ \text{nm}$. 
			\item vẫn bằng $5\cdot 10^{14}\ \text{Hz}$ còn bước sóng lớn hơn $600\ \text{nm}$. 
		\end{mcq}
	}
	
	\loigiai
	{		\textbf{Đáp án: C.}
		
Tần số của ánh sáng không đổi khi truyền từ môi trường này sang môi trường khác nên vẫn là $ 5\cdot 10^{14}\ \text{Hz} $. \\
Bước sóng của ánh sáng trong môi trường cho bởi:
$$
	\lambda' = \dfrac{\lambda}{n} = \SI{395}{nm}.
$$
		
	}
	
	\item \mkstar{1} 
	
	\cauhoi
	{Bức xạ (hay tia) hồng ngoại là bức xạ
		\begin{mcq}(1)
			\item đơn sắc, có màu hồng. 
			\item đơn sắc, không màu ở ngoài đầu đỏ của quang phổ. 
			\item có bước sóng nhỏ dưới $\text{0,4}\ \mu\text{m}$. 
			\item có bước sóng từ $\text{0,76}\ \mu\text{m}$ tới cỡ milimét. 
		\end{mcq}
	}
	
	\loigiai
	{		\textbf{Đáp án: D.}
		
Bức xạ (hay tia) hồng ngoại là bức xạ có bước sóng từ $\text{0,76}\ \mu\text{m}$ tới cỡ milimét. 
		
	}
	
	\item \mkstar{1}
	
	\cauhoi
	{Công dụng phổ biến nhất của tia hồng ngoại là
		\begin{mcq}(2)
			\item Sấy khô, sưởi ấm. 
			\item Chiếu sáng. 
			\item Chụp ảnh ban đêm. 
			\item Chữa bệnh.
		\end{mcq}
	}
	
	\loigiai
	{		\textbf{Đáp án: A.}
		
Công dụng phổ biến nhất của tia hồng ngoại là sấy khô, sưởi ấm. 
		
	}
	
	\item \mkstar{1} 
	
	\cauhoi
	{Bức xạ tử ngoại là bức xạ điện từ
		\begin{mcq}
			\item có màu tím sẫm. 
			\item có tần số thấp hơn so với ánh sáng khả kiến. 
			\item có bước sóng lớn hơn so với bức xạ hồng ngoại. 
			\item có bước sóng nhỏ hơn so với ánh sáng khả kiến. 
		\end{mcq}
	}
	
	\loigiai
	{		\textbf{Đáp án: D.}
		
Bức xạ tử ngoại là bức xạ điện từ có bước sóng nhỏ hơn so với ánh sáng khả kiến. 
		
	}
	
	\item \mkstar{1} 
	
	\cauhoi
	{ Tia hồng ngoại là những bức xạ có
		\begin{mcq}
			\item bản chất là sóng điện từ. 
			\item khả năng ion hoá mạnh không khí. 
			\item khả năng đâm xuyên mạnh, có thể xuyên qua lớp chì dày cỡ cm.
			\item bước sóng nhỏ hơn bước sóng của ánh sáng đỏ. 
		\end{mcq}
	}
	
	\loigiai
	{		\textbf{Đáp án: A.}
		
Tia hồng ngoại là những bức xạ có bản chất là sóng điện từ.
		
	}
	
	\item \mkstar{1} 
	
	\cauhoi
	{Phát biểu nào sau đây là \textbf{không đúng}?
		\begin{mcq}
			\item Vật có nhiệt độ trên $3000^\circ \text{C}$ phát ra tia tử ngoại rất mạnh. 
			\item Tia tử ngoại không bị thuỷ tinh hấp thụ. 
			\item Tia tử ngoại là sóng điện từ có bước sóng nhỏ hơn bước sóng của ánh sáng đỏ.
			\item Tia tử ngoại có tác dụng nhiệt. 
		\end{mcq}
	}
	
	\loigiai
	{		\textbf{Đáp án: B.}
		
Tia tử ngoại bị thủy tinh hấp thụ mạnh.
		
	}
	
	\item \mkstar{1} 
	
	\cauhoi
	{Khi nói về tia tử ngoại, phát biểu nào dưới đây là \textbf{sai}?
		\begin{mcq}
			\item Tia tử ngoại có tác dụng mạnh lên kính ảnh.. 
			\item Tia tử ngoại có bản chất là sóng điện từ. 
			\item Tia tử ngoại có bước sóng lớn hơn bước sóng của ánh sáng tím. 
			\item Tia tử ngoại bị thuỷ tinh hấp thụ mạnh và làm ion hoá không khí. 
		\end{mcq}
	}
	
	\loigiai
	{		\textbf{Đáp án: C.}
		
Tia tử ngoại có bước sóng nhỏ hơn bước sóng ánh sáng tím.
		
	}
	
\item \mkstar{1}
	
	\cauhoi
	{Tìm phát biểu \textbf{sai} về tia hồng ngoại.
		\begin{mcq}
			\item Tia hồng ngoại có bản chất là sóng điện từ. 
			\item Tia hồng ngoại kích thích thị giác làm cho ta nhìn thấy màu hồng. 
			\item vật nung nóng ở nhiệt độ thấp chỉ phát ra tia hồng ngoại. Nhiệt độ của vật trên $500^\circ \text{C}$ mới bắt đầu phát ra ánh sáng khả kiến. 
			\item Tia hồng ngoại nằm ngoài vùng ánh sáng khả kiến, bước sóng của tia hồng ngoại dài hơn bước sóng của ánh sáng đỏ. 
		\end{mcq}
	}
	
	\loigiai
	{		\textbf{Đáp án: B.}
		
Tia hồng ngoại không kích thích thị giác.
		
	}
	
	\item \mkstar{1} 
	
	\cauhoi
	{Chọn câu đúng khi nói về tia X?
		\begin{mcq}
			\item Đa số tia X là sóng điện từ có bước sóng nhỏ hơn bước sóng của tia tử ngoại. 
			\item Tia X do các vật bị nung nóng ở nhiệt độ cao phát ra. 
			\item Tia X có thể được phát ra từ các đèn điện. 
			\item Tia X có thể xuyên qua tất cả mọi vật.
		\end{mcq}
	}
	
	\loigiai
	{		\textbf{Đáp án: A.}
		
Đa số tia X là sóng điện từ có bước sóng nhỏ hơn bước sóng của tia tử ngoại.
		
	}
	
	\item \mkstar{1}
	
	\cauhoi
	{Tia Rơn-ghen hay tia X là sóng điện từ có bước sóng
		\begin{mcq}
			\item lớn hơn tia hồng ngoại. 
			\item nhỏ hơn tia tử ngoại. 
			\item nhỏ quá không đo được. 
			\item vài na-no-mét đến vài mi-li-mét. 
		\end{mcq}
	}
	
	\loigiai
	{		\textbf{Đáp án: B.}
		
Tia Rơn-ghen hay tia X là sóng điện từ có bước sóng nhỏ hơn tia tử ngoại. 
		
	}
	
	\item \mkstar{1} 
	
	\cauhoi
	{Chọn câu \textbf{không đúng}?
		\begin{mcq}
			\item Tia X có khả năng xuyên qua một lá nhôm mỏng.
			\item Tia X có tác dụng mạnh lên kính ảnh. 
			\item Tia X là bức xạ có thể trông thấy được vì nó làm cho một số chất phát quang. 
			\item Tia X là bức xạ có hại đối với sức khỏe con người. 
		\end{mcq}
	}
	
	\loigiai
	{		\textbf{Đáp án: C.}
		
Tia X là bức xạ có thể trông thấy được vì nó làm cho một số chất phát quang là không đúng.
		
	}
	
	\item \mkstar{1} 
	
	\cauhoi
	{\textbf{Chọn phát biểu sai.}
	
	Tia X
		\begin{mcq}
			\item có bản chất là sóng điện từ. 
			\item có năng lượng lớn vì bước sóng lớn. 
			\item không bị lệch phương trong điện trường và từ trường. 
			\item có bước sóng ngắn hơn bước sóng của tia tử ngoại. 
		\end{mcq}
	}
	
	\loigiai
	{		\textbf{Đáp án: B.}

Tia X có năng lượng lớn vì bước sóng nhỏ.
		
	}
	
\end{enumerate}

\loigiai
{
	\begin{center}
		\textbf{BẢNG ĐÁP ÁN}
	\end{center}
	\begin{center}
		\begin{tabular}{|m{2.8em}|m{2.8em}|m{2.8em}|m{2.8em}|m{2.8em}|m{2.8em}|m{2.8em}|m{2.8em}|m{2.8em}|m{2.8em}|}
			\hline
			01.B  & 02.D  & 03.D  & 04.B  & 05.B  & 06.D  & 07.D & 08.D & 09.B & 10.C \\
			\hline
			11.B  & 12.A  & 13.A  & 14.A  & 15.C  & 16.B  & 17.D & 18.C & 19.C & 20.D \\
			\hline
			21.A  & 22.D  & 23.A  & 24.B  & 25.C  & 26.B  & 27.A & 28.B & 29.C & 30.B \\
			\hline
			
		\end{tabular}
	\end{center}
}

\section{Bài toán về tia X}
\begin{enumerate}[label=\bfseries Câu \arabic*:]
	\item \mkstar{1} 
	
	\cauhoi
	{Cho một ống phát tia X có $U_\text{AK}=30\ \text{kV}$. Bỏ qua động năng ban đầu. Cho hằng số điện tích nguyên tố $e=\text{1,6}\cdot 10^{-19}\ \text{C}$ và khối lượng của electron $m_\text{e}=\text{9,1}\cdot 10^{-31}\ \text{kg}$. Tốc độ lớn nhất của electron ngay trước khi đập vào anốt là
		\begin{mcq}(2)
			\item $\text{1,3}\cdot 10^{7}\ \text{m/s}$. 
			\item $\text{1,3}\cdot 10^{6}\ \text{m/s}$. 
			\item $\text{1,03}\cdot 10^{8}\ \text{m/s}$.
			\item $\text{3,1}\cdot 10^{8}\ \text{m/s}$.
		\end{mcq}
	}
	
	\loigiai
	{		\textbf{Đáp án: C.}
		
Động năng lớn nhất của electron trước khi đập vào anốt là
$$
	K_{max} = eU_{AK} = \SI{4,8 e-15}{J}.
$$
Tốc độ lớn nhất của electron trước khi đập vào anốt là
$$
	K_{max} = \dfrac{1}{2}mv_{max}^{2} \rightarrow v_{max} = \SI{1,03 e8}{m/s}.
$$
		
	}
	
	\item \mkstar{3} 
	
	\cauhoi
	{Hiệu điện thế giữa anốt và catốt của một ống Rơnghen là $U_\text{AK}=\text{18,75}\ \text{kV}$. Cho $h=\text{6,625}\cdot 10^{-34}\ \text{Js}$, độ lớn điện tích của lectron $e=\text{1,6}\cdot 10^{-19}\ \text{C}$ và $c=3\cdot 10^8\ \text{m/s}$. Bỏ qua động năng ban đầu của êlectrôn. Bước sóng nhỏ nhất của tia Rơnghen do ống phát ra là
		\begin{mcq}(2)
			\item $\text{0,4625}\cdot 10^{-9}\ \text{m}$. 
			\item $\text{0,6625}\cdot 10^{-9}\ \text{m}$. 
			\item $\text{0,5625}\cdot 10^{-9}\ \text{m}$. 
			\item $\text{0,6625}\cdot 10^{-10}\ \text{m}$. 
		\end{mcq}
	}
	
	\loigiai
	{		\textbf{Đáp án: D.}
		
Bước sóng nhỏ nhất cho bởi:
$$
	\lambda_{min} = \dfrac{hc}{eU_{AK}} = \text{0,6625}\cdot 10^{-10}\ \text{m}.
$$
		
	}
	
	\item \mkstar{3}
	
	\cauhoi
	{Chùm tia X phát ra từ một ống tia X (ống Cu-lít-giơ) có tần số lớn nhất là  $\text{6,4}\ \cdot 10^{18}\ \text{Hz}$. Bỏ qua động năng các êlectron khi bứt ra khỏi catôt. Hiệu điện thế giữa anôt và catôt của ống tia X là
		\begin{mcq}(2)
			\item $U_\text{AK}=\text{13,25}\ \text{kV}$. 
			\item $U_\text{AK}=\text{2,65}\ \text{kV}$. 
			\item $U_\text{AK}=\text{26,50}\ \text{kV}$. 
			\item $U_\text{AK}=\text{5,30}\ \text{kV}$.
		\end{mcq}
	}
	
	\loigiai
	{		\textbf{Đáp án: C.}
		
Hiệu điện thế giữa hai đầu anôt và catôt là
$$
	U_{AK} = \dfrac{hf}{e} = \SI{26,5}{kV}.
$$
		
	}
	
\end{enumerate}

\loigiai
{
	\begin{center}
		\textbf{BẢNG ĐÁP ÁN}
	\end{center}
	\begin{center}
		\begin{tabular}{|m{2.8em}|m{2.8em}|m{2.8em}|m{2.8em}|m{2.8em}|m{2.8em}|m{2.8em}|m{2.8em}|m{2.8em}|m{2.8em}|}
			\hline
			01.C  & 02.D  & 03.C  &  &  &  & & & & \\
			\hline
			
		\end{tabular}
	\end{center}
}

\whiteBGstarEnd