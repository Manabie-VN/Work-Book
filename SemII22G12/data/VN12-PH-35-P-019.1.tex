\chapter{Luyện tập}
\section{Giao thoa ánh sáng đơn sắc - Dạng 1: Khoảng vân - vị trí vân sáng - vân tối}
\begin{enumerate}
	\item
	{
		Trong thí nghiệm Y-âng, vân tối thứ nhất xuất hiện ở trên màn tại các vị trí cách vân sáng trung tâm là
		\begin{mcq}(4)
			\item{$i/4$.}
			\item{$i/2$.}
			\item{$i$.}
			\item{$2i$.}
		\end{mcq}
	}
	\item
	{
		Khoảng cách từ vân sáng bậc 4 bên này đến vân sáng bậc 5 bên kia so với vân sáng trung tâm là
		\begin{mcq}(4)
		\item{$7i$.}
		\item{$8i$.}
		\item{$9i$.}
		\item{$10i$.}
		\end{mcq}
	}
	\item
	{
		Khoảng cách từ vân sáng bậc 5 đến vân sáng bậc 9 ở cùng phía so với vân sáng trung tâm là
		\begin{mcq}(4)
		\item{$4i$.}
		\item{$5i$.}
		\item{$14i$.}
		\item{$13i$.}
		\end{mcq}
	}
	\item
	{
		Trong thí nghiệm Y-âng về giao thoa ánh sáng, khoảng cách giữa hai khe sáng là $\SI{0.2}{\milli \meter}$, khoảng cách từ hai khe sáng đến màn ảnh là $D=\SI{1}{\meter}$, khoảng vân đo được là $i=\SI{2}{\milli \meter}$. Bước sóng của ánh sáng là
		\begin{mcq}(4)
		\item{$\SI{0.4}{\micro \meter}$.}
		\item{$\SI{4}{\micro \meter}$.}
		\item{$\SI{0.4e-3}{\micro \meter}$.}
		\item{$\SI{0.4e-4}{\micro \meter}$.}
		\end{mcq}	
	}
	\item
	{
		Trong thí nghiệm Y-âng về giao thoa ánh sáng, biết $a=\SI{0.4}{\milli \meter}$, $D=\SI{1.2}{\meter}$, nguồn S phát ra bức xạ đơn sắc có bước sóng $\lambda = \SI{600}{\nano \meter}$. Khoảng cách giữa 2 vân sáng liên tiếp trên màn là
		\begin{mcq}(4)
			\item{$\SI{1.6}{\milli \meter}$.}
			\item{$\SI{1.2}{\milli \meter}$.}
			\item{$\SI{1.8}{\milli \meter}$.}
			\item{$\SI{1.4}{\milli \meter}$.}
		\end{mcq}	
	}
	\item
	{
		Trong thí nghiệm Y-âng về giao thoa ánh sáng, biết $a=\SI{5}{\milli \meter}$, $D=\SI{2}{\meter}$. Khoảng cách giữa 6 vân sáng liên tiếp là $\SI{1.5}{\milli \meter}$. Bước sóng của ánh sáng đơn sắc là
		\begin{mcq}(4)
			\item{$\SI{0.65}{\micro \meter}$.}
			\item{$\SI{0.71}{\micro \meter}$.}
			\item{$\SI{0.75}{\micro \meter}$.}
			\item{$\SI{0.69}{\micro \meter}$.}
		\end{mcq}	
	}
	\item
	{
		Trong thí nghiệm Y-âng về giao thoa ánh sáng, các khe sáng được chiếu bằng ánh sáng đơn sắc. Khoảng cách giữa hai khe là $\SI{2}{\milli \meter}$, khoảng cách từ hai khe đến màn là $\SI{4}{\milli \meter}$. Khoảng cách giữa 5 vân sáng liên tiếp đo được là $\SI{4.8}{\milli \meter}$. Toạ độ của vân sáng bậc 3 là
		\begin{mcq}(4)
			\item{$\pm \SI{9.6}{\milli \meter}$.}
			\item{$\pm \SI{4.8}{\milli \meter}$.}
			\item{$\pm \SI{3.6}{\milli \meter}$.}
			\item{$\pm \SI{2.4}{\milli \meter}$.}
		\end{mcq}	
	}
	\item
	{
		Trong thí nghiệm Y-âng, khoảng cách giữa hai khe là $a=\SI{2}{\milli \meter}$, khoảng cách từ hai khe đến màn là $D=\SI{2}{\meter}$. Vân sáng thứ 3 cách vân sáng trung tâm $\SI{1.8}{\milli \meter}$. Bước sóng ánh sáng đơn sắc dùng trong thí nghiệm là
		\begin{mcq}(4)
			\item{$\SI{0.4}{\micro \meter}$.}
			\item{$\SI{0.55}{\micro \meter}$.}
			\item{$\SI{0.5}{\micro \meter}$.}
			\item{$\SI{0.6}{\micro \meter}$.}
		\end{mcq}	
	}
	\item
	{
		Trong thí nghiệm Y-âng về giao thoa ánh sáng, khoảng cách giữa hai khe là $a=\SI{2}{\milli \meter}$, khoảng cách từ hai khe đến màn là $D=\SI{2}{\meter}$, ánh sáng đơn sắc có bước sóng $\SI{0.5}{\micro \meter}$. Khoảng cách từ vân sáng bậc 1 đến vân sáng bậc 10 là
		\begin{mcq}(4)
			\item{$\SI{4.5}{\milli \meter}$.}
			\item{$\SI{5.5}{\milli \meter}$.}
			\item{$\SI{4.0}{\milli \meter}$.}
			\item{$\SI{5.0}{\milli \meter}$.}
		\end{mcq}	
	}
	\item
	{
		Trong thí nghiệm về giao thoa ánh sáng, khoảng cách giữa 2 khe hẹp là $a=\SI{1}{\milli \meter}$, từ 2 khe đến màn ảnh là $D=\SI{1}{\meter}$. Dùng ánh sáng đỏ có bước sóng $\lambda_\text{đ}=\SI{0.75}{\micro \meter}$, khoảng cách từ vân sáng thứ 4 đến vân sáng thứ 10 ở cùng phía so với vân trung tâm là
		\begin{mcq}(4)
			\item{$\SI{2.8}{\milli \meter}$.}
			\item{$\SI{3.6}{\milli \meter}$.}
			\item{$\SI{4.5}{\milli \meter}$.}
			\item{$\SI{5.2}{\milli \meter}$.}
		\end{mcq}	
	}
	\item
	{
		Ánh sáng đơn sắc trong thí nghiệm Y-âng là $\SI{0.5}{\micro \meter}$. Khoảng cách từ hai nguồn đến màn là $\SI{1}{\meter}$, khoảng cách giữa hai nguồn là $\SI{2}{\milli \meter}$. Khoảng cách giữa vân sáng bậc 3 và vân tối bậc 5 ở hai bên so với vân trung tâm là
		\begin{mcq}(4)
			\item{$\SI{0.375}{\milli \meter}$.}
			\item{$\SI{1.875}{\milli \meter}$.}
			\item{$\SI{18.75}{\milli \meter}$.}
			\item{$\SI{3.75}{\milli \meter}$.}
		\end{mcq}	
	}
	\item
	{
	Trong thí nghiệm Y-âng về giao thoa của ánh sáng đơn sắc, hai khe hẹp cách nhau $\SI{1}{\milli \meter}$, mặt phẳng chứa hai khe cách màn quan sát $\SI{1.5}{\meter}$. Khoảng cách giữa 5 vân sáng liên tiếp là $\SI{3.6}{\milli \meter}$. Bước sóng của ánh sáng dùng trong thí nghiệm này bằng
	\begin{mcq}(4)
		\item{$\SI{0.48}{\micro \meter}$.}
		\item{$\SI{0.40}{\micro \meter}$.}
		\item{$\SI{0.60}{\micro \meter}$.}
		\item{$\SI{0.76}{\micro \meter}$.}
	\end{mcq}	
	}
	\item
	{
	Trong thí nghiệm Y-âng về giao thoa với ánh sáng đơn sắc, khoảng cách giữa hai khe là $\SI{1}{\milli \meter}$, khoảng cách từ mặt phẳng chứa hai khe đến màn quan sát là $\SI{2}{\meter}$ và khoảng vân là $\SI{0.8}{\milli \meter}$. Cho $c=\SI{3e8}{\meter / \second}$. Tần số ánh sáng đơn sắc dùng trong thí nghiệm là
	\begin{mcq}(4)
		\item{$\SI{5.5e14}{\hertz}$.}
		\item{$\SI{4.5e14}{\hertz}$.}
		\item{$\SI{7.5e14}{\hertz}$.}
		\item{$\SI{6.5e14}{\hertz}$.}
	\end{mcq}	
	}
	\item
	{
	Trong thí nghiệm giao thoa ánh sáng dùng hai khe Y-âng, hai khe được chiếu bằng ánh sáng có bước sóng $\lambda=\SI{0.5}{\micro \meter}$, biết $\text S_1 \text S_2 = a = \SI{0.5}{\milli \meter}$, khoảng cách từ mặt phẳng chứa hai khe đến màn quan sát là $D=\SI{1}{\meter}$. Tại điểm M cách vân trung tâm một khoảng $x=\SI{3.5}{\milli \meter}$, có vân sáng hay vân tối, bậc mấy?
	\begin{mcq}(2)
		\item{Vân sáng bậc 3.}
		\item{Vân tối thứ 4.}
		\item{Vân sáng bậc 4.}
		\item{Vân tối thứ 2.}
	\end{mcq}		
	}
	\item
	{
	Giao thoa ánh sáng đơn sắc của Y-âng có $\lambda=\SI{0.5}{\micro \meter}$, $a=\SI{0.5}{\milli \meter}$, $D=\SI{2}{\meter}$. Tại M cách vân trung tâm $\SI{7}{\milli \meter}$ và tại điểm N cách vân trung tâm $\SI{10}{\milli \meter}$ thì
	\begin{mcq}(2)
		\item{M, N đều là vân sáng.}
		\item{M là vân tối, N là vân sáng.}
		\item{M, N đều là vân tối.}
		\item{M là vân sáng, N là vân tối.}
	\end{mcq}	
	}
	\item
	{
	Trong thí nghiệm giao thoa ánh sáng, khi $a=\SI{2}{\milli \meter}$, $D=\SI{2}{\meter}$, $\lambda=\SI{0.6}{\micro \meter}$ thì khoảng cách giữa hai vân sáng bậc 4 ở hai bên vân trung tâm là
	\begin{mcq}(4)
		\item{$\SI{4.8}{\milli \meter}$.}
		\item{$\SI{1.2}{\centi \meter}$.}
		\item{$\SI{2.4}{\milli \meter}$.}
		\item{$\SI{4.8}{\centi \meter}$.}
	\end{mcq}	
	}
\end{enumerate}
\hides{\textbf{Đáp án}
\begin{center}
	\begin{tabular}{|m{2.8em}|m{2.8em}|m{2.8em}|m{2.8em}|m{2.8em}|m{2.8em}|m{2.8em}|m{2.8em}|m{2.8em}|m{2.8em}|}
		\hline
		1. B & 2. C & 3. A & 4. A & 5. C & 6. C & 7. C & 8. D & 9. A & 10. C \\
		\hline
		11. A & 12. C & 13. C & 14. B & 15. B & 16. A &&&& \\
		\hline
	\end{tabular}
\end{center}}
\section{Giao thoa ánh sáng đơn sắc - Dạng 2: Tìm số vân sáng - vân tối trên một miền}
\begin{enumerate}
	\item
	{
	Một nguồn sáng đơn sắc S cách hai khe Y-âng $\SI{0.2}{\milli \meter}$ phát ra một bức xạ đơn sắc có $\lambda=\SI{0.64}{\micro \meter}$. Hai khe cách nhau $a=\SI{3}{\milli \meter}$, màn cách hai khe $\SI{3}{\meter}$. Trường giao thoa trên màn có bề rộng $\SI{12}{\milli \meter}$. Số vân tối quan sát được trên màn là
	\begin{mcq}(4)
		\item{$16$.}
		\item{$17$.}
		\item{$18$.}
		\item{$19$.}
	\end{mcq}	
	}
	\item
	{
	Trong thí nghiệm Y-âng về giao thoa ánh sáng, khoảng cách giữa hai khe là $\SI{1.5}{\milli \meter}$, khoảng cách từ hai khe đến màn là $\SI{3}{\meter}$, người ta đo được khoảng cách giữa vân sáng bậc 2 đến vân sáng bậc 5 ở cùng phía với nhau so với vân sáng trung tâm là $\SI{3}{\milli \meter}$. Tìm số vân sáng quan sát được trên vùng giao thoa đối xứng có bề rộng $\SI{11}{\milli \meter}$.
	\begin{mcq}(4)
		\item{$9$.}
		\item{$10$.}
		\item{$11$.}
		\item{$12$.}
	\end{mcq}	
	}
	\item
	{
	Người ta thực hiện giao thoa ánh sáng đơn sắc với hai khe Y-âng cách nhau $\SI{0.5}{\milli \meter}$, khoảng cách giữa hai khe đến màn là $\SI{2}{\meter}$, ánh sáng dùng có bước sóng $\lambda=\SI{0.5}{\micro \meter}$. Bề rộng của trường giao thoa đối xứng là $\SI{18}{\milli \meter}$. Số vân sáng, vân tối lần lượt là
	\begin{mcq}(2)
		\item{$N_1=11, N_2=12$.}
		\item{$N_1=7, N_2=8$.}
		\item{$N_1=9, N_2=10$.}
		\item{$N_1=13, N_2=14$.}
	\end{mcq}	
	}
	\item
	{
	Trong thí nghiệm Y-âng về giao thoa ánh sáng, người ta đo được khoảng vân là $\SI{1.13e3}{\micro \meter}$. Xét 2 điểm M và N cùng một phía so với vân chính giữa, với $\text{OM}=\SI{0.56e4}{\micro \meter}$ và $\text{ON}=\SI{1.288e4}{\micro \meter}$, giữa M và N có bao nhiêu vân tối?
	\begin{mcq}(4)
		\item{$5$.}
		\item{$6$.}
		\item{$7$.}
		\item{$8$.}
	\end{mcq}	
	}
	\item
	{
	Trong thí nghiệm Y-âng về giao thoa ánh sáng, nguồn sáng đơn sắc có $\lambda=\SI{0.5}{\micro \meter}$, khoảng cách giữa hai khe là $a=\SI{2}{\milli \meter}$. Trong khoảng MN trên màn với $\text{MO}=\text{ON}=\SI{5}{\milli \meter}$ có 11 vân sáng mà hai mép M và N là hai vân sáng. Khoảng cách từ hai khe đến màn quan sát là
	\begin{mcq}(4)
		\item{$D=\SI{2}{\meter}$.}
		\item{$D=\SI{2.4}{\meter}$.}
		\item{$D=\SI{3}{\meter}$.}
		\item{$D=\SI{4}{\meter}$.}
	\end{mcq}	
	}
	\item
	{
	Bề rộng vùng giao thoa (đối xứng) quan sát được trên màn là $\text{MN}=\SI{30}{\milli \meter}$, khoảng cách giữa hai vân tối liên tiếp bằng $\SI{2}{\milli \meter}$. Trên MN quan sát thấy 
	\begin{mcq}(2)
		\item{16 vân tối, 15 vân sáng.}
		\item{15 vân tối, 16 vân sáng.}
		\item{14 vân tối, 15 vân sáng.}
		\item{16 vân tối, 16 vân sáng.}
	\end{mcq}	
	}
	\item
	{
	Trong thí nghiệm giao thoa khe Young, khoảng cách giữa hai khe $\text F _1 \text F_2$ là $a=\SI{2}{\milli \meter}$, khoảng cách từ hai khe $\text F_1 \text F_2$ đến màn là $D=\SI{1.5}{\meter}$, dùng ánh sáng đơn sắc có bước sóng $\lambda=\SI{0.6}{\micro \meter}$. Xét trên khoảng MN, với $\text{MO}=\SI{5}{\milli \meter}$, $\text{ON}=\SI{10}{\milli \meter}$ (O là vị trí vân sáng trung tâm), MN nằm cùng phía vân sáng trung tâm. Số vân sáng trong đoạn MN là
	\begin{mcq}(4)
		\item{$11$.}
		\item{$12$.}
		\item{$13$.}
		\item{$15$.}
	\end{mcq}	
	}
	\item
	{
		Trong thí nghiệm giao thoa khe Young, khoảng cách giữa hai khe $\text F _1 \text F_2$ là $a=\SI{2}{\milli \meter}$, khoảng cách từ hai khe $\text F_1 \text F_2$ đến màn là $D=\SI{1.5}{\meter}$, dùng ánh sáng đơn sắc có bước sóng $\lambda=\SI{0.6}{\micro \meter}$. Xét trên khoảng MN, với $\text{MO}=\SI{5}{\milli \meter}$, $\text{ON}=\SI{10}{\milli \meter}$ (O là vị trí vân sáng trung tâm), MN nằm khác phía vân sáng trung tâm. Số vân sáng trong đoạn MN là
		\begin{mcq}(4)
			\item{$31$.}
			\item{$32$.}
			\item{$33$.}
			\item{$35$.}
		\end{mcq}	
	}
	\item
	{
		Trong thí nghiệm Y-âng về giao thoa ánh sáng, hai khe cách nhau $a=\SI{0.5}{\milli \meter}$ được chiếu sáng bằng ánh sáng đơn sắc. Khoảng cách từ hai khe đến màn quan sát là $\SI{2}{\meter}$. Trên màn quan sát, trong vùng giữa hai điểm M và N mà $\text{MN}=\SI{2}{\centi \meter}$, người ta đếm được có 10 vân tối và thấy tại M và N đều là vân sáng. Bước sóng của ánh sáng đơn sắc dùng trong thí nghiệm này là
		\begin{mcq}(4)
			\item{$\SI{0.4}{\micro \meter}$.}
			\item{$\SI{0.5}{\micro \meter}$.}
			\item{$\SI{0.6}{\micro \meter}$.}
			\item{$\SI{0.7}{\micro \meter}$.}
		\end{mcq}	
	}
	\item
	{
		Trong thí nghiệm Y-âng về giao thoa ánh sáng với ánh sáng đơn sắc, khoảng cách giữa hai khe là $\SI{1}{\milli \meter}$, khoảng cách từ hai khe tới màn $\SI{2}{\meter}$. Trong đoạn rộng $\SI{12.5}{\milli \meter}$ trên màn có 13 vân tối biết một đầu là vân tối còn một đầu là vân sáng. Bước sóng của ánh sáng đơn sắc đó là
		\begin{mcq}(4)
			\item{$\SI{0.48}{\micro \meter}$.}
			\item{$\SI{0.52}{\micro \meter}$.}
			\item{$\SI{0.5}{\micro \meter}$.}
			\item{$\SI{0.46}{\micro \meter}$.}
		\end{mcq}	
	}
\end{enumerate}
\hides{\textbf{Đáp án}
\begin{center}
	\begin{tabular}{|m{2.8em}|m{2.8em}|m{2.8em}|m{2.8em}|m{2.8em}|m{2.8em}|m{2.8em}|m{2.8em}|m{2.8em}|m{2.8em}|}
		\hline
		1. C & 2. C & 3. C & 4. B & 5. D & 6. A & 7. A & 8.  & 9. B & 10. C \\
		\hline
	\end{tabular}
\end{center}}
\section{Giao thoa ánh sáng đơn sắc - Dạng 3: Bài toán về sự thay đổi khoảng vân do sự thay đổi khoảng cách}
\begin{enumerate}
	\item
	{
		Trong thí nghiệm giao thoa ánh sáng của Y-âng, khoảng cách hai khe là $\SI{0.2}{\milli \meter}$, ánh sáng đơn sắc làm thí nghiệm có bước sóng $\SI{0.6}{\micro \meter}$. Lúc đầu, màn cách hai khe $\SI{1.6}{\meter}$. Tịnh tiến màn theo phương vuông góc mặt phẳng chứa hai khe một đoạn $d$ thì tại vị trí vân sáng bậc 3 lúc đầu trùng vân sáng bậc 2. Màn được tịnh tiến
		\begin{mcq}(2)
			\item{xa hai khe $\SI{150}{\centi \meter}$.}
			\item{gần hai khe $\SI{80}{\centi \meter}$.}
			\item{xa hai khe $\SI{80}{\centi \meter}$.}
			\item{gần hai khe $\SI{150}{\centi \meter}$.}
		\end{mcq}	
	}
	\item
	{
		Trong thí nghiệm Y-âng, khoảng cách giữa 9 vân sáng liên tiếp là $L$. Dịch chuyển màn $\SI{36}{\centi \meter}$ theo phương vuông góc với màn thì khoảng cách giữa 11 vân sáng liên tiếp cũng là $L$. Khoảng cách giữa màn và hai khe lúc đầu là
		\begin{mcq}(4)
			\item{$\SI{1.8}{\meter}$.}
			\item{$\SI{2}{\meter}$.}
			\item{$\SI{2.5}{\meter}$.}
			\item{$\SI{1.5}{\meter}$.}
		\end{mcq}	
	}
	\item
	{
		Thực hiện giao thoa ánh sáng đơn sắc với khoảng cách từ mặt phẳng chứa hai khe đến màn hứng vân giao thoa là $D=\SI{2}{\meter}$ và tại ví trí M đang có vân sáng bậc 4. Cần phải thay đổi khoảng cách $D$ nói trên một khoảng bao nhiêu thì tại M có vân tối thứ 6?
		\begin{mcq}(2)
			\item{giảm đi $\SI{2/9}{\meter}$.}
			\item{tăng thêm $\SI{8/11}{\meter}$.}
			\item{tăng thêm $\SI{0.4}{\meter}$.}
			\item{giảm $\SI{6/11}{\meter}$.}
		\end{mcq}	
	}
	\item
	{
		Trong thí nghiệm giao thoa Y-âng, nguồn S phát ánh sáng đơn sắc có bước sóng $\lambda$ người ta đặt màn quan sát cách mặt phẳng hai khe một khoảng $D$ thì khoảng vân $i=\SI{1}{\milli \meter}$. Khi khoảng cách từ màn quan sát đến mặt phẳng hai khe lần lượt là $D+\Delta D$ hoặc $D-\Delta D$ thì khoảng vân thu được trên màn tương tứng là $2i$ và $i$. Nếu khoảng cách từ màn quan sát đến mặt phẳng hai khe là $D+3\Delta D$ thì khoảng vân trên màn là
		\begin{mcq}(4)
			\item{$\SI{3}{\milli \meter}$.}
			\item{$\SI{4}{\milli \meter}$.}
			\item{$\SI{2}{\milli \meter}$.}
			\item{$\SI{2.5}{\milli \meter}$.}
		\end{mcq}	
	}
	\item
	{
		Trong thí nghiệm Y-âng về giao thoa ánh sáng, hai khe được chiếu bằng ánh sáng đơn sắc có bước sóng $\SI{0.6}{\micro \meter}$. Khoảng cách giữa hai khe sáng là $\SI{1}{\milli \meter}$, khoảng cách từ mặt phẳng chứa hai khe đến màn quan sát là $\SI{1.5}{\meter}$. Trên màn quan sát, hai vân sáng bậc 4 nằm ở hai điểm M và N. Dịch màn quan sát một đoạn $\SI{50}{\centi \meter}$ theo hướng ra xa hai khe Y-âng thì số vân sáng quan sát trên đoạn MN giảm so với lúc đầu là
		\begin{mcq}(4)
			\item{7 vân.}
			\item{4 vân.}
			\item{6 vân.}
			\item{2 vân.}
		\end{mcq}	
	}
	\item
	{
		Trong thí nghiệm Y-âng về giao thoa ánh sáng, hai khe được chiếu bằng ánh sáng đơn sắc $\lambda$, màn quan sát cách mặt phẳng hai khe một khoảng không đổi $D$, khoảng cách giữa hai khe có thể thay đổi (nhưng $\text S_1$ và $\text S_2$ luôn cách đều S). Xét điểm M trên màn, lúc đầu là vân sáng bậc 4, nếu lần lượt giảm hoặc tăng khoảng cách $\text S_1 \text S_2$ một lượng $\Delta a$ thì tại đó là vân sáng bậc $k$ và bậc $3k$. Nếu tăng khoảng cách $\text S_1 \text S_2$ thêm $2\Delta a$ thì tại M là
		\begin{mcq}(2)
			\item{vân sáng bậc 7.}
			\item{vân sáng bậc 9.}
			\item{vân sáng bậc 8.}
			\item{vân tối thứ 9.}
		\end{mcq}	
	}
\end{enumerate}
\hides{\textbf{Đáp án}
\begin{center}
	\begin{tabular}{|m{2.8em}|m{2.8em}|m{2.8em}|m{2.8em}|m{2.8em}|m{2.8em}|m{2.8em}|m{2.8em}|m{2.8em}|m{2.8em}|}
		\hline
		1. B & 2. A & 3. D & 4. C & 5. A & 6. C &&&& \\
		\hline
	\end{tabular}
\end{center}}
\section{Giao thoa ánh sáng trắng}
\begin{enumerate}
	\item
	{
		Trong thí nghiệm Y-âng về giao thoa ánh sáng, hai khe được chiếu bằng ánh sáng trắng có bước sóng từ $\SI{0.38}{\micro \meter}$ đến $\SI{0.76}{\micro \meter}$. Khoảng cách giữa hai khe là $\SI{0.8}{\milli \meter}$, khoảng cách từ mặt phẳng chứa hai khe đến màn quan sát là $\SI{1.2}{\meter}$. Độ rộng quang phổ bậc 3 (nằm về một phía so với vân sáng trung tâm) là
		\begin{mcq}(4)
			\item{$\SI{0.57}{\milli \meter}$.}
			\item{$\SI{1.14}{\milli \meter}$.}
			\item{$\SI{1.71}{\milli \meter}$.}
			\item{$\SI{2.36}{\milli \meter}$.}
		\end{mcq}	
	}
		\item
	{
		Trong thí nghiệm giao thoa ánh sáng khe Y-âng, khoảng cách giữa hai khe $\text S_1, \text S_2$ bằng $\SI{1}{\milli \meter}$, khoảng cách từ hai khe đến màn quan sát là $D=\SI{2}{\meter}$. Chiếu vào 2 khe bằng chùm sáng trắng có bước sóng $\lambda$ ($\SI{0.38}{\micro \meter} \leq \lambda \leq \SI{0.76}{\micro \meter}$). Bề rộng đoạn chồng chập của quang phổ bậc 5 và quang phổ bậc 7 trên trường giao thoa bằng
		\begin{mcq}(4)
			\item{$\Delta x= \SI{0.76}{\milli \meter}$.}
			\item{$\Delta x= \SI{2.28}{\milli \meter}$.}
			\item{$\Delta x= \SI{1.14}{\milli \meter}$.}
			\item{$\Delta x= \SI{1.44}{\milli \meter}$.}
		\end{mcq}	
	}
		\item
	{
		Thực hiện giao thoa với ánh sáng trắng có bước sóng $\SI{0.4}{\micro \meter} \leq \lambda \leq \SI{0.7}{\micro \meter}$. Hai khe cách nhau $\SI{2}{\milli \meter}$, màn hứng vân giao thoa cách hai khe $\SI{2}{\meter}$. Tại điểm M cách vân trung tâm $\SI{3.3}{\milli \meter}$ có bao nhiêu ánh sáng đơn sắc cho vân sáng tại đó?
		\begin{mcq}(4)
			\item{$5$.}
			\item{$3$.}
			\item{$4$.}
			\item{$2$.}
		\end{mcq}	
	}
		\item
	{
		Hai khe Y-âng cách nhau $\SI{1}{\milli \meter}$ được chiếu sáng bằng ánh sáng trắng ($\SI{0.4}{\micro \meter} \leq \lambda \leq \SI{0.76}{\micro \meter}$), khoảng cách từ hai khe đến màn là $\SI{1}{\meter}$. Tại điểm A trên màn cách vân trung tâm $\SI{2}{\milli \meter}$ có các bức xạ cho vân sáng có bước sóng
		\begin{mcq}(2)
			\item{$\SI{0.40}{\micro \meter}$, $\SI{0.50}{\micro \meter}$ và $\SI{0.66}{\micro \meter}$.}
			\item{$\SI{0.44}{\micro \meter}$, $\SI{0.50}{\micro \meter}$ và $\SI{0.66}{\micro \meter}$.}
			\item{$\SI{0.40}{\micro \meter}$, $\SI{0.44}{\micro \meter}$ và $\SI{0.50}{\micro \meter}$.}
			\item{$\SI{0.40}{\micro \meter}$, $\SI{0.44}{\micro \meter}$ và $\SI{0.66}{\micro \meter}$.}
		\end{mcq}	
	}
		\item
	{
		Thực hiện giao thoa ánh sáng qua khe Y-âng, biết khoảng cách giữa hai khe $\SI{0.5}{\milli \meter}$, khoảng cách từ màn chứa hai khe tới mà quan sát là $\SI{2}{\meter}$. Nguồn S phát ánh sáng trắng gồm vô số bức xạ đơn sắc có bước sóng từ $\SI{0.4}{\micro \meter}$ đến $\SI{0.75}{\micro \meter}$. Hỏi ở đúng vị trí vân sáng bậc 4 của bức xạ đỏ còn có bao nhiêu bức xạ cho vân sáng nằm trùng tại đó?
		\begin{mcq}(4)
			\item{3.}
			\item{4.}
			\item{5.}
			\item{6.}
		\end{mcq}	
	}
		\item
	{
		Trong thí nghiệm giao thoa ánh sáng bằng khe Y-âng. Khoảng cách giữa hai khe kết hợp là $a=\SI{2}{\milli \meter}$, khoảng cách từ hai khe đến màn là $D=\SI{2}{\meter}$. Nguồn S phát ra ánh sáng trắng có bước sóng từ $\SI{380}{\nano \meter}$ đến $\SI{760}{\nano \meter}$. Vùng phủ nhau giữa quang phổ bậc hai và quang phổ bậc ba có bề rộng là
		\begin{mcq}(4)
			\item{$\SI{0.76}{\milli \meter}$.}
			\item{$\SI{0.38}{\milli \meter}$.}
			\item{$\SI{1.14}{\milli \meter}$.}
			\item{$\SI{1.52}{\milli \meter}$.}
		\end{mcq}	
	}
\end{enumerate}
\hides{\textbf{Đáp án}
\begin{center}
	\begin{tabular}{|m{2.8em}|m{2.8em}|m{2.8em}|m{2.8em}|m{2.8em}|m{2.8em}|m{2.8em}|m{2.8em}|m{2.8em}|m{2.8em}|}
		\hline
		1. C & 2. B & 3. C & 4. A & 5. A & 6. B &&&& \\
		\hline
	\end{tabular}
\end{center}}