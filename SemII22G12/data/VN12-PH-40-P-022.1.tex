%\setcounter{chapter}{1}
\chapter{Luyện tập: Hiện tượng quang điện trong}

\begin{enumerate}
	\item Phát biểu nào sau đây là đúng?
	\begin{mcq}
		\item Hiện tượng quang điện trong là hiện tượng bứt êlectron ra khỏi bề mặt kim loại khi chiếu vào kim loại ánh sáng có bước sóng thích hợp.
		\item  Hiện tượng quang điện trong là hiện tượng êlectron bị bắn ra khỏi kim loại khi kim loại bị đốt nóng.
		\item Hiện tượng quang điện trong là hiện tượng êlectron liên kết được giải phóng thành êlectron dẫn khi chất bán dẫn được chiếu bằng bức xạ thích hợp.
		\item Hiện tượng quang điện trong là hiện tượng điện trở của chất bán dẫn tăng lên khi chiếu ánh sáng thích hợp vào chất bán dẫn.
	\end{mcq}
	
	\item Khi chiếu vào chất CdS ánh sáng đơn sắc có bước sóng nhỏ hơn giới hạn quang điện trong của chất này thì điện trở của nó
	\begin{mcq}
		\item không thay đổi.
		\item luôn tăng.
		\item giảm đi.
		\item lúc tăng, lúc giảm.
	\end{mcq}
	
	\item 
	Chọn câu trả lời đúng?
	\begin{mcq}
		\item  Quang dẫn là hiện tuợng dẫn điện của chất bán dẫn lúc được chiếu sáng.
		\item  Quang dẫn là hiện tượng kim loại phát xạ êlectron lúc được chiếu sáng.
		\item Quang dẫn là hiện tượng điện trở của một chất giảm rất nhiều khi hạ nhiệt độ xuống rất thấp.
		\item Quang dẫn là hiện tượng bứt quang êlectron ra khỏi bề mặt chất bán dẫn.
	\end{mcq}
	
	\item Khi chiếu sóng điện từ vào chất bán dẫn, hiện tượng quang điện trong xảy ra nếu 
	\begin{mcq}
		\item sóng điện từ có nhiệt độ cao.
		\item sóng điện từ có bước sóng thích hợp.
		\item sóng điện từ có cường độ đủ lớn.	
		\item sóng điện từ phải là ánh sáng nhìn thấy được.
	\end{mcq}
	
	\item Điểm giống nhau giữa hiện tượng quang điện trong và hiện tượng quang điện ngoài là
	\begin{mcq}
		\item cùng được ứng dụng đề chế tạo pin quang điện.
		\item khi hấp thu phôtôn có năng lượng thích hợp thì êlectron sẽ bứt ra khỏi bề mặt của khối chất.
		\item chỉ xảy ra khi êlectron hấp thu một phôtôn có năng lượng đủ lớn.
		\item chỉ xảy ra khi tần số của ánh sáng kích thích nhỏ hơn một giá trị nhất định.
	\end{mcq}
	
	\item Chọn phát biểu đúng về hiện tượng quang điện trong?
	\begin{mcq}
		\item Có bước sóng giới hạn nhỏ hơn bước sóng giới hạn của hiện tượng quang điện ngoài.
		\item Ánh sáng kích thích phải là ánh sáng tử ngoại.
		\item Có thể xảy ra khi được chiếu bằng bức xạ hồng ngoại.
		\item Có thể xảy ra đối với cả kim loại.
	\end{mcq}
	
	\item Pin quang điện
	\begin{mcq}
		\item là dụng cụ biến đổi trực tiếp quang năng thành điện năng.
		\item là dụng cụ biến nhiệt năng thành điện năng.
		\item hoạt động dựa vào hiện tượng quang điện ngoài.
		\item là dụng cụ có điện trở tăng khi được chiếu sáng.
	\end{mcq}
	
	\item Chọn phát biểu đúng:
	
	\begin{mcq}
		\item Trong pin quang điện, năng lượng Mặt Trời được biến đổi toàn bộ thành điện năng
		\item Suất điện động của một pin quang điện chỉ xuất hiện khi pin được chiếu sáng.
		\item Theo định nghĩa, hiện tượng quang điện trong là nguyên nhân sinh ra hiện tượng quang dẫn.
		\item Bước sóng ánh sáng chiếu vào khối bán dẫn càng lớn thì điện trở của khối này cảng nhỏ.
	\end{mcq}
	
	\item Chọn ý \textbf{sai}. 
	
	Pin quang điện
	\begin{mcq}
		\item là pin chạy bằng năng lượng ánh sáng.
		\item biến đổi trực tiếp quang năng thành điện năng.
		\item hoạt động dựa trên quang điện trong.
		\item có hiệu suất cao (khoảng trên $50\%$).
	\end{mcq}
	
	\item  Trong hiện tượng quang điện trong: Năng lượng cần thiết để giải phóng một electron liên kết thành electron tự do là $\varepsilon$ thì bước sóng dài nhất của ánh kích thích gây ra được hiện tượng quang điện trong bằng
	\begin{mcq}(4)
		\item $\dfrac{hc}{\varepsilon}$.
		\item $\dfrac{h\varepsilon}{c}$.
		\item $\dfrac{\varepsilon}{hc}$.
		\item $\dfrac{c}{h\varepsilon}$.
	\end{mcq}
	
	
\end{enumerate}

\begin{center}
	\textbf{ĐÁP ÁN}
	
\end{center}

\begin{longtable}[\textwidth]{|p{0.1\textwidth}|p{0.1\textwidth}|p{0.1\textwidth}|p{0.1\textwidth}|p{0.1\textwidth}|p{0.1\textwidth}|p{0.1\textwidth}|p{0.1\textwidth}|}
	% --- first head
	\hline%\hspace{2 pt}
	\multicolumn{1}{|c|}{\textbf{Câu 1}} & \multicolumn{1}{c|}{\textbf{Câu 2}} & \multicolumn{1}{c|}{\textbf{Câu 3}} &
	\multicolumn{1}{c|}{\textbf{Câu 4}} &
	\multicolumn{1}{c|}{\textbf{Câu 5}} &
	\multicolumn{1}{c|}{\textbf{Câu 6}} &
	\multicolumn{1}{c|}{\textbf{Câu 7}} &
	\multicolumn{1}{c|}{\textbf{Câu 8}}\\
	\hline
	C. &C. &A. &B. &D. &C. &A. &B.\\
	\hline
	
	\multicolumn{1}{|c|}{\textbf{Câu 9}} & \multicolumn{1}{c|}{\textbf{Câu 10}} & \multicolumn{1}{c|}{\textbf{}} &
	\multicolumn{1}{c|}{\textbf{}} &
	\multicolumn{1}{c|}{\textbf{}} &
	\multicolumn{1}{c|}{\textbf{}} &
	\multicolumn{1}{c|}{\textbf{}} &
	\multicolumn{1}{c|}{} \\
	\hline
	D. &A. & & & & & &\\
	\hline		
\end{longtable}














