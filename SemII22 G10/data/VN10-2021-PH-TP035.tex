\whiteBGstarBegin
\setcounter{section}{0}
\section{Lý thuyết: Nguyên lí 1 nhiệt động lực học}
\begin{enumerate}[label=\bfseries Câu \arabic*:]
	
	\item \mkstar{1} [4]
	
	\cauhoi{
		Nguyên lí I nhiệt động lực học được biểu diễn bởi công thức $\Delta U = Q +A$ với quy ước
		\begin{mcq}(2)
			\item $Q>0$: hệ truyền nhiệt.
			\item $A<0$: hệ nhận công.
			\item $Q<0$: hệ nhận nhiệt.
			\item $A>0$: hệ nhận công.
		\end{mcq}
	}
	
	\loigiai{
		\textbf{Đáp án: D.}
		
		Nguyên lí I nhiệt động lực học được biểu diễn bởi công thức $\Delta U = Q +A$ với quy ước:
		\begin{itemize}
			\item $Q>0$: hệ nhận nhiệt;
			\item $Q<0$: hệ truyền nhiệt;
			\item $A>0$: hệ nhận công;
			\item $A<0$: hệ thực hiện công.
		\end{itemize}
	}
	
	\item \mkstar{1} [4]
	
	\cauhoi{
		Nguyên lí I nhiệt động lực học là sự vận dụng của định luật nào sau đây?
		\begin{mcq}
			\item Định luật bảo toàn khối lượng. 
			\item Định luật bảo toàn động lượng.
			\item Định luật bảo toàn và chuyển hóa năng lượng.
			\item Định luật bảo toàn cơ năng.
		\end{mcq}
		
	}
	
	\loigiai{
		\textbf{Đáp án: C.}
		
		Nguyên lí I nhiệt động lực học là sự vận dụng của định luật bảo toàn và chuyển hóa năng lượng: năng lượng không tự sinh ra và cũng không tự mất đi; năng lượng chỉ chuyển từ dạng này sang dạng khác, từ vật này sang vật khác.
	}
	
	\item \mkstar{1} [24]
	
	\cauhoi{
		Quy ước về dấu nào sau đây phù hợp với công thức $\Delta U = A+Q$ của nguyên lí I nhiệt động lực học?
		\begin{mcq}
			\item Vật thực hiện công thì $A<0$, vật truyền nhiệt thì $Q>0$.
			\item Vật thực hiện công khi $A>0$, vật truyền nhiệt khi $Q<0$.
			\item Vật nhận công khi $A<0$, vật nhận nhiệt khi $Q<0$.
			\item Vật nhận công khi $A>0$, vật nhận nhiệt khi $Q>0$.
		\end{mcq}
	}
	
	\loigiai{
		\textbf{Đáp án: D.}
		
		Nguyên lí I nhiệt động lực học được biểu diễn bởi công thức $\Delta U = Q +A$ với quy ước:
		\begin{itemize}
			\item $Q>0$: hệ nhận nhiệt;
			\item $Q<0$: hệ truyền nhiệt;
			\item $A>0$: hệ nhận công;
			\item $A<0$: hệ thực hiện công.
		\end{itemize}
	}
\item \mkstar{1} [32]

\cauhoi{
	Trong quá trình chất khí tỏa nhiệt và nhận công thì
	\begin{mcq}(2)
		\item $Q<0$ và $A>0$.
		\item $Q>0$ và $A>0$.
		\item $Q>0$ và $A<0$.
		\item $Q<0$ và $A<0$.
	\end{mcq}
}

\loigiai{
	\textbf{Đáp án: A.}
	
	Nguyên lí I nhiệt động lực học được biểu diễn bởi công thức $\Delta U = Q +A$ với quy ước:
	\begin{itemize}
		\item $Q>0$: hệ nhận nhiệt;
		\item $Q<0$: hệ truyền nhiệt;
		\item $A>0$: hệ nhận công;
		\item $A<0$: hệ thực hiện công.
	\end{itemize}
}
\item \mkstar{2} [24]

\cauhoi{
	Khi bị nén lại, khí trong xi lanh nhận được công $\SI{50}{J}$. Trong quá trình đó, khí truyền ra môi trường xung quanh nhiệt lượng $\SI{20}{J}$. Độ biến thiên nội năng của khí là
	\begin{mcq}(4)
		\item $\SI{20}{J}$.
		\item $\SI{30}{J}$.
		\item $\SI{50}{J}$.
		\item $\SI{70}{J}$.
	\end{mcq}
}

\loigiai{
	\textbf{Đáp án: B.}
	
	Khối khí nhận công: $A>0 \Rightarrow A=\SI{50}{J}$.
	
	Khối khí truyền nhiệt: $Q<0 \Rightarrow Q=\SI{-20}{J}$
	
	Độ biến thiên nội năng: $\Delta U = Q+A = \SI{30}{J}$.
}
\item \mkstar{2} [24]

\cauhoi{
	Một lượng khí ở áp suất $\SI{2e5}{Pa}$ có thể tích 10 lít. Sau khi đun nóng đẳng áp khí nở ra và có thể tích 15 lít. Độ lớn công mà khí thực hiện là
	\begin{mcq}(4)
		\item $\SI{3000}{J}$.
		\item $\SI{2000}{J}$.
		\item $\SI{1000}{J}$.
		\item $\SI{5000}{J}$.
	\end{mcq}
}

\loigiai{
	\textbf{Đáp án: C.}
	
	Độ biến thiên thể tích: $\Delta V = \SI{5e-3}{m^3}$.
	
	Độ lớn công mà khí thực hiện trong quá trình đẳng áp
	$$|A|=|p\Delta V| = \SI{1000}{J}$$
	
	Lưu ý theo quy ước dấu của nguyên lý I nhiệt động lực học thì $A<0$ vì khối khí thực hiện công.
}

\item \mkstar{3} [24]

\cauhoi{
	Một cốc nhôm có khối lượng $\SI{100}{g}$ chứa $\SI{400}{g}$ nước ở nhiệt độ $\SI{20}{\celsius}$. Người ta thả vào cốc nước một thìa đồng khối lượng $\SI{150}{g}$ vừa rút ra từ nồi nước sôi $\SI{100}{\celsius}$. Bỏ qua các hao phí nhiệt ra ngoài. Cho biết nhiệt dung riêng của nhôm là $\SI{880}{J/(kg\cdot \text{độ})}$, nhiệt dung riêng của đồng là $\SI{380}{J/(kg\cdot \text{độ})}$ và nhiệt dung riêng của nước là $\SI{4190}{J/(kg\cdot \text{độ})}$. Khi cân bằng nhiệt, nhiệt độ của các vật gần giá trị nào nhất?
	\begin{mcq}(4)
		\item $\SI{20}{\celsius}$.
		\item $\SI{22.5}{\celsius}$.
		\item $\SI{38.5}{\celsius}$.
		\item $\SI{100}{\celsius}$.
	\end{mcq}
}

\loigiai{
	\textbf{Đáp án: B.}
	
	Gọi $t$ là nhiệt độ khi cân bằng nhiệt.
	
	Các vật thu nhiệt là cốc nhôm và nước ở $\SI{20}{\celsius}$, vật tỏa nhiệt là thìa đồng ở $\SI{100}{\celsius}$. Áp dụng phương trình cân bằng nhiệt:
	$$Q_\text{tỏa} = Q_\text{thu} \Rightarrow m_{\ce{Cu}} c_{\ce{Cu}} \Delta t_{\ce{Cu}} = m_{\ce{H_2 O}} c_{\ce{H_2 O}} \Delta t_{\ce{H_2 O}} + m_{\ce{Al}} c_{\ce{Al}} \Delta t_{\ce{Al}}$$
	Suy ra:
	$$\SI{0.15}{kg} \cdot \SI{380}{J/(kg\cdot \text{độ})} \cdot (\SI{100}{\celsius} - t) = \SI{0.4}{kg} \cdot \SI{4190}{J/(kg\cdot \text{độ})} \cdot (t-\SI{20}{\celsius}) + \SI{0.1}{kg} \cdot \SI{880}{J/(kg\cdot \text{độ})} \cdot (t-\SI{20}{\celsius})$$
	
	Vậy $t=\SI{22.5}{\celsius}$.
}
\item \mkstar{1} [3]

\cauhoi{
	Nếu vật đồng thời nhận được công $A$ và nhiệt lượng $Q$ thì nội năng của vật biến thiên là $\Delta U$. Viết biểu thức liên hệ giữa $A$, $Q$ và $\Delta U$. Nội năng của vật lúc này tăng hay giảm? Vì sao?
}

\loigiai{
	Biểu thức liên hệ giữa $A$, $Q$ và $\Delta U$:
	$$\Delta U = Q+A$$
	
	Vật nhận công nên $A>0$, vật nhận nhiệt nên $Q>0$. Suy ra $\Delta U>0$, nội năng tăng.
	
}
	\item \mkstar{2} [2]
	
	\cauhoi{
		\begin{enumerate}[label=\alph*)]
			\item Phát biểu và viết hệ thức của nguyên lí I nhiệt động lực học. Nêu quy ước dấu cảu các đại lượng.
			\item Người ta cung cấp cho chất khí trong xi lanh nằm ngang nhiệt lượng $\SI{2.5}{J}$. Khí nở ra đẩy pit-tông đi một đoạn $\SI{10}{cm}$ với một lực có độ lớn $\SI{15}{N}$. Tính độ biến thiên nội năng của khí.
		\end{enumerate}
	}
	
	\loigiai{		
		\begin{enumerate}[label=\alph*)]
			\item Phát biểu và viết hệ thức của nguyên lí I nhiệt động lực học. Nêu quy ước dấu của các đại lượng.
			
			Độ biến thiên nội năng của hệ bằng tổng công và nhiệt lượng mà hệ nhận được.
			$$\Delta U = A+Q$$
			\begin{itemize}
				\item $Q>0$: hệ nhận nhiệt;
				\item $Q<0$: hệ truyền nhiệt;
				\item $A>0$: hệ nhận công;
				\item $A<0$: hệ thực hiện công.
			\end{itemize}
		
			\item Người ta cung cấp cho chất khí trong xi lanh nằm ngang nhiệt lượng $\SI{2.5}{J}$. Khí nở ra đẩy pit-tông đi một đoạn $\SI{10}{cm}$ với một lực có độ lớn $\SI{15}{N}$. Tính độ biến thiên nội năng của khí.
			
			Khối khí nhận nhiệt: $Q>0 \Rightarrow Q=\SI{2.5}{J}$.
			
			Khối khí nở ra đẩy pit-tông nghĩa là khối khí thực hiện công: $A<0 \Rightarrow A=-Fs=\SI{-1.5}{J}$
			
			Độ biến thiên nội năng: $\Delta U = Q+A = \SI{1}{J}$.
		\end{enumerate}
	}
	
		\item \mkstar{2} [6]
	
	\cauhoi{
		Người ta thả một miếng nhôm có khối lượng $\SI{210}{g}$ đã được đun nóng tới nhiệt độ $\SI{132}{\celsius}$ vào cốc đựng nước ở nhiệt độ $\SI{10}{\celsius}$. Xác định khối lượng của nước biết nhiệt độ của hệ khi bắt đầu có sự cân bằng nhiệt là $\SI{32}{\celsius}$. Biết nhiệt dung riêng của nhôm là $\SI{880}{J/(kg\cdot \text{độ})}$, nhiệt dung riêng của nước là $\SI{4200}{J/(kg\cdot \text{độ})}$. Bỏ qua sự truyền nhiệt ra môi trường bên ngoài.
	}
	
	\loigiai{		
		Vật thu nhiệt là nước ở $\SI{10}{\celsius}$, vật tỏa nhiệt là miếng nhôm ở $\SI{132}{\celsius}$. Áp dụng phương trình cân bằng nhiệt:
		$$Q_\text{thu} = Q_\text{tỏa} \Rightarrow  m_{\ce{H_2 O}} c_{\ce{H_2 O}} \Delta t_{\ce{H_2 O}} = m_{\ce{Al}} c_{\ce{Al}} \Delta t_{\ce{Al}}$$
		Suy ra:
		$$m_{\ce{H_2 O}}  \cdot \SI{4200}{J/(kg\cdot \text{độ})} \cdot (\SI{32}{\celsius}-\SI{10}{\celsius}) = \SI{0.21}{kg} \cdot \SI{880}{J/(kg\cdot \text{độ})} \cdot (\SI{132}{\celsius}-\SI{32}{\celsius})$$
		
		Vậy $m_{\ce{H_2 O}}=\SI{200}{g}$.
	}

	\item \mkstar{2} [14]

\cauhoi{
	Người ta cung cấp cho khối khí trong một xi lanh nằm ngang một nhiệt lượng $\SI{5}{J}$. Khi đó khối khí thực hiện một công $\SI{3}{J}$ đẩy pit-tông. Tính độ biến thiên nội năng của khối khí.
}

\loigiai{		
	Khối khí thực hiện công: $A<0 \Rightarrow A=\SI{-3}{J}$.
	
	Khối khí nhận nhiệt: $Q>0 \Rightarrow Q=\SI{5}{J}$
	
	Độ biến thiên nội năng: $\Delta U = Q+A = \SI{2}{J}$.
}

	\item \mkstar{2} [14]

\cauhoi{
	Đổ $\SI{100}{g}$ nước ở nhiệt độ $\SI{80}{\celsius}$ vào một cái bình bằng nhôm có khối lượng $\SI{800}{g}$ ở nhiệt độ $t\SI{}{\celsius}$, bỏ qua sự trao đổi nhiệt với môi trường ngoài. Sau khi xảy ra cân bằng nhiệt thì nhiệt độ của hệ là $\SI{35}{\celsius}$, biết nhiệt dung riêng của nước và của nhôm lần lượt là $\SI{4200}{J/(kg \cdot K)}$ và $\SI{900}{J/(kg \cdot K)}$. Hãy xác định nhiệt độ ban đầu của bình nhôm.
}

\loigiai{		
	Vật thu nhiệt là bình nhôm, vật tỏa nhiệt là nước ở $\SI{80}{\celsius}$. Áp dụng phương trình cân bằng nhiệt:
	$$Q_\text{tỏa} = Q_\text{thu} \Rightarrow  m_{\ce{H_2 O}} c_{\ce{H_2 O}} \Delta t_{\ce{H_2 O}} = m_{\ce{Al}} c_{\ce{Al}} \Delta t_{\ce{Al}}$$
	Suy ra:
	$$\SI{0.1}{kg}  \cdot \SI{4200}{J/(kg\cdot K)} \cdot (\SI{80}{\celsius}-\SI{35}{\celsius}) = \SI{0.8}{kg} \cdot \SI{900}{J/(kg\cdot \text{độ})} \cdot (\SI{35}{\celsius}-t_{\ce{Al}})$$
	
	Vậy $t_{\ce{Al}}=\SI{8.75}{\celsius}$.
}
	\item \mkstar{2} [17]

\cauhoi{
	Cần truyền cho chất khí một nhiệt lượng bao nhiêu để chất khí đó thực hiện công là $\SI{100}{J}$ và có độ tăng nội năng là $\SI{40}{J}$?
}

\loigiai{
	Nội năng tăng lên $\Delta U >0$, suy ra $\Delta U = \SI{40}{J}$.
	
	Khối khí nhận nhiệt nên $Q>0$.
	
	Khối khí thực hiện công nên $A<0$, suy ra $A=\SI{-100}{J}$.
	
	Áp dụng công thức $\Delta U = A+Q \Rightarrow Q = \SI{140}{J}$.
}

\item \mkstar{2} [27]

\cauhoi{
	Người ta thực hiện công $\SI{300}{J}$ để nén khí trong một xi lanh. Tính độ biến thiên nội năng của khí, biết khí truyền ra môi trường xung quanh nhiệt lượng $\SI{90}{J}$.
}

\loigiai{
	Người ta thực hiện công để nén khí nghĩa là khối khí nhận công: $A>0 \Rightarrow A=\SI{300}{J}$.
	
	Khối khí truyền nhiệt: $Q<0 \Rightarrow Q=\SI{-90}{J}$
	
	Độ biến thiên nội năng: $\Delta U = Q+A = \SI{210}{J}$.
	
}

\item \mkstar{2} [32]

\cauhoi{
	Chất khí trong xi lanh nhận nhiệt hay tỏa nhiệt một lượng là bao nhiêu nếu như người ta thực hiện công $\SI{100}{J}$ lên khối khí và nội năng khối khí tăng thêm $\SI{20}{J}$?
}

\loigiai{
	Nội năng tăng lên $\Delta U >0$, suy ra $\Delta U = \SI{20}{J}$.
	
	Người ta thực hiện công để nén khí nghĩa là khối khí nhận công: $A>0 \Rightarrow A=\SI{100}{J}$.
	
	Áp dụng công thức: $\Delta U = A+Q \Rightarrow Q = \SI{-80}{J}$.
	
	Vì $Q<0$ nên khối khí tỏa nhiệt một nhiệt lượng là $\SI{80}{J}$.
	
}

	\item \mkstar{3} [16]

\cauhoi{
	Một bình nhôm khối lượng $\SI{0.5}{kg}$ chứa $\SI{0.118}{kg}$ nước ở nhiệt độ $\SI{20}{\celsius}$. Người ta thả vào bình một miếng sắt khối lượng $\SI{0.2}{kg}$ ở nhiệt độ $\SI{75}{\celsius}$. Xác định nhiệt độ của nước khi bắt đầu có sự cân bằng nhiệt. Bỏ qua sự truyền nhiệt ra môi trường bên ngoài. Cho nhiệt dung riêng của nhôm là $\SI{920}{J/(kg \cdot K)}$, của nước là $\SI{4180}{J/(kg\cdot K)}$ và của sắt là $\SI{460}{J/(kg \cdot K)}$.
}

\loigiai{		
	Gọi $t$ là nhiệt độ khi cân bằng nhiệt.
	
	Các vật thu nhiệt là cốc nhôm và nước ở $\SI{20}{\celsius}$, vật tỏa nhiệt là miếng sắt ở $\SI{75}{\celsius}$. Áp dụng phương trình cân bằng nhiệt:
	$$Q_\text{tỏa} = Q_\text{thu} \Rightarrow m_{\ce{Fe}} c_{\ce{Fe}} \Delta t_{\ce{Fe}} = m_{\ce{H_2 O}} c_{\ce{H_2 O}} \Delta t_{\ce{H_2 O}} + m_{\ce{Al}} c_{\ce{Al}} \Delta t_{\ce{Al}}$$
	Suy ra:
	$$\SI{0.2}{kg} \cdot \SI{460}{J/(kg\cdot \text{độ})} \cdot (\SI{75}{\celsius} - t) = \SI{0.118}{kg} \cdot \SI{4180}{J/(kg\cdot \text{độ})} \cdot (t-\SI{20}{\celsius}) + \SI{0.5}{kg} \cdot \SI{920}{J/(kg\cdot \text{độ})} \cdot (t-\SI{20}{\celsius})$$
	
	Vậy $t=\SI{24.84}{\celsius}$.
}

	\item \mkstar{3} [16]

\cauhoi{
	Khi truyền nhiệt lượng $\SI{6e6}{J}$ cho chất khí đựng trong một xi lanh hình trụ thì khí nở ra đẩy pit-tông lên. Thể tích khí tăng thêm $\SI{500}{dm^3}$. Hỏi nội năng của khi biến đổi một lượng bằng bao nhiêu? Biết áp suất của khí là $\SI{8e6}{Pa}$ và không đổi trong quá trình dãn nở.
}

\loigiai{		
	Khối khí nhận nhiệt: $Q>0 \Rightarrow Q=\SI{6e6}{J}$.
	
	Khối khí nở ra đẩy pit-tông nghĩa là khối khí thực hiện công: $A<0$.
	
	Trong quá trình đẳng áp thì $|A|=|p\Delta V|$, suy ra
	$ A=-p \Delta V=\SI{-4e6}{J}$
	
	Độ biến thiên nội năng: $\Delta U = Q+A = \SI{2e6}{J}$.
}




\end{enumerate}
\section{Lý thuyết: Nguyên lí 2 nhiệt động lực học}
\begin{enumerate}[label=\bfseries Câu \arabic*:]
		\item \mkstar{2}
	
	\cauhoi{
		Trong các động cơ đốt trong, nguồn lạnh là
		\begin{mcq}(1)
			\item bình ngưng hơi.
			\item hỗn hợp nhiên liệu và không khí cháy trong buồng đốt.
			\item không khí bên ngoài.
			\item hỗn hợp nhiên liệu và không khí cháy trong xi-lanh.
		\end{mcq}
		
	}
	
	\loigiai{
		\textbf{Đáp án: C.}
		
		Trong các động cơ đốt trong, nguồn lạnh là không khí bên ngoài.
	}
	\item \mkstar{2}

\cauhoi{
	Một động cơ nhiệt mỗi giây nhận từ nguồn nóng nhiệt lượng $\SI{4.32e4}{\joule}$ đồng thời nhường cho nguồn lạnh $\SI{3.84e4}{\joule}$. Hiệu suất của động cơ nhiệt là bao nhiêu?
	\begin{mcq}(4)
		\item $\SI{11.11}{\percent}$.
		\item $\SI{12.5}{\percent}$.
		\item $\SI{50}{\percent}$.
		\item $\SI{88.89}{\percent}$.
	\end{mcq}
	
}

\loigiai{
	\textbf{Đáp án: A.}
	
	Xác định nhiệt lượng động cơ nhận từ nguồn nóng $(Q_1)$ và toả ra cho nguồn lạnh $(Q_2)$:
	\begin{align*}
		Q_1&=\SI{4.32e4}{\joule},
		\\ Q_2 &=\SI{3.84e4}{\joule}.
	\end{align*}

	Hiệu suất của động cơ nhiệt:
	\begin{equation*}
		H=\dfrac{Q_1-Q_2}{Q_1}\approx \SI{11.11}{\percent}.
	\end{equation*}
}

\item \mkstar{1} [6]

\cauhoi{
	Phát biểu nguyên lý II nhiệt động lực học theo cách phát biểu của Clau-di-út.
}

\loigiai{
	
	Phát biểu nguyên lý II nhiệt động lực học theo cách phát biểu của Clau-di-út: Nhiệt không thể tự truyền từ một vật sang vật khác nóng hơn.
}

\item \mkstar{2} [26]

\cauhoi{
	Có ý kiến cho rằng, việc sử dụng ô tô, xe máy là một trong những nguyên nhân gây ra hiện tượng Trái Đất ngày càng nóng lên. Dựa vào những kiến thức đã học về Nhiệt động lực học, em hãy cho biết ý kiến này đúng hay sai? Giải thích.
}

\loigiai{
	
	Ý kiến cho rằng "việc sử dụng ô tô, xe máy là một trong những nguyên nhân gây ra hiện tượng Trái Đất ngày càng nóng lên" là ý kiến đúng.
	
	Theo nguyên lý II NĐLH, động cơ nhiệt không thể chuyển hóa tất cả nhiệt lượng nhận được thành công cơ học. Do đó ô tô, xe máy không thể chuyển hóa hoàn toàn nhiệt lượng do nhiện liệu bị đốt cháy tỏa ra thành công cơ học mà có một phần nhiệt lượng tỏa vào trong khí quyển.
}


\end{enumerate}
\whiteBGstarEnd