%\setcounter{chapter}{1}
\chapter{Luyện tập: Động lượng. Định luật bảo toàn động lượng}
\section{Động lượng của vật, hệ vật}
\begin{enumerate}
	\item %câu 1
	Một vật có khối lượng $m=\SI{1}{\kilogram}$ đang chuyển động với vận tốc $v=\SI{2}{\meter/\second}$. Tính động lượng của vật.
	\item %câu 2
	Một vật có khối lượng $\SI{2}{\kilogram}$ và có động lượng $\SI{6}{\kilogram\cdot\meter\cdot\second^{-1}}$. Vật đang chuyển động với vận tốc bao nhiêu?
	\item %câu 3
	Hai vật có khối lượng $m_1=\SI{1}{\kilogram}$, $m_2=\SI{3}{\kilogram}$ chuyển động với các vận tốc $v_1=\SI{3}{\meter/\second}$ và $v_2=\SI{1}{\meter/\second}$. Tìm tổng động lượng (phương, chiều và độ lớn) của hệ trong các trường hợp
	\begin{enumerate}[label=\alph*)]
		\item $\vec{v}_1$ và $\vec{v}_2$ cùng hướng.
		\item $\vec{v}_1$ và $\vec{v}_2$ cùng phương, ngược chiều.
		\item $\vec{v}_1$ và $\vec{v}_2$ vuông góc nhau.
		\item $\vec{v}_1$ và $\vec{v}_2$ hợp nhau một góc $120^\circ$.
	\end{enumerate}
	\item %câu 4
	Một xe có khối lượng 5 tấn bắt đầu hãm phanh chuyển động thẳng chậm dần đều dừng lại hẳn sau $\SI{20}{\second}$ kể từ lúc bắt đầu hãm phanh, trong thời gian đó xe chạy được $\SI{120}{\meter}$. Tính động lượng của xe lúc bắt đầu hãm phanh.
	\item %câu 5
	Từ độ cao $h=\SI{80}{\meter}$, ở thời điểm $t_0=0$ một vật $\SI{200}{\gram}$ được ném ngang với vận tốc ban đầu $v_0=10\sqrt{3}\,\text{m/s}$, gia tốc trọng trường $g=\SI{10}{\meter/\second^2}$. Động lượng của vật ở thời điểm $t=\SI{1}{\second}$?
\end{enumerate}


\textbf{ĐÁP ÁN}
\begin{longtable}[\textwidth]{|m{0.15\textwidth}|m{0.15\textwidth}|m{0.23\textwidth}|m{0.2\textwidth}|m{0.15\textwidth}|}
	% --- first head
	\hline%\hspace{2 pt}
	\multicolumn{1}{|c}{\textbf{Câu 1}} &
	\multicolumn{1}{|c|}{\textbf{Câu 2}}& 
	\multicolumn{1}{c|}{\textbf{Câu 3}} &
	\multicolumn{1}{c|}{\textbf{Câu 4}} &
	\multicolumn{1}{c|}{\textbf{Câu 5}} \\
	\hline
	$\SI{2}{\kilogram\cdot\meter\cdot\second^{-1}}$.&
	$\SI{3}{\meter/\second}$.&
	\begin{enumerate}[label=\alph*)]
		\item $\SI{6}{\kilogram\cdot\meter\cdot\second^{-1}}$.
		\item $\SI{0}{\kilogram\cdot\meter\cdot\second^{-1}}$.
		\item $3\sqrt{2}\,\SI{}{\kilogram\cdot\meter\cdot\second^{-1}}$.
		\item $\SI{3}{\kilogram\cdot\meter\cdot\second^{-1}}$.
	\end{enumerate} &
	$\SI{60000}{\kilogram\cdot\meter\cdot\second^{-1}}$.&
	$\SI{4}{\kilogram\cdot\meter\cdot\second^{-1}}$. \\
	\hline
\end{longtable}	

\section{Độ biến thiên động lượng}
\begin{enumerate}
	\item %câu 1
	Một vật có khối lượng $\SI{1}{\kilogram}$ rơi tự do xuống đất trong khoảng thời gian $\SI{0,5}{\second}$. Độ biến thiên động lượng của vật trong khoảng thời gian đó là bao nhiêu? Lấy $g=\SI{10}{\meter/\second^2}$.
	\item %câu 2
	Một quả bóng có khối lượng $m=\SI{300}{\gram}$ va chạm vào tường và nảy trở lại với cùng tốc độ. Vận tốc bóng trước va chạm là $\SI{5}{\meter/\second}$. Tìm độ biến thiên động lượng?
	\item %câu 3
	Một toa xe khối lượng 10 tấn đang chuyển động trên đường ray nằm ngang với vận tốc không đổi $v=\SI{54}{\kilo\meter/\hour}$. người ta tác dụng lên toa xe một lực hãm theo phương ngang. Tính độ lớn trung bình của lực hãm nếu toa xe dừng lại sau
	\begin{enumerate}[label=\alph*)]
		\item 1 phút 40 giây.
		\item 10 giây.
	\end{enumerate}
	\item %câu 4
	Xác định độ biến thiên động lượng của một vật có khối lượng $\SI{4}{\kilogram}$ sau khoảng thời gian 6 giây. Biết rằng vật chuyển động trên đường thẳng và có phương trình chuyển động là $x=t^2-6t+3$.
	\item %câu 5
	Một viên đạn khối lượng $\SI{10}{\gram}$ đang bay với vận tốc $\SI{600}{\meter/\second}$ thì gặp một bức tường. Đạn xuyên qua tường trong thời gian $\dfrac{1}{100}\,\text{s}$. Sau khi xuyên qua tường, vận tốc của đạn còn $\SI{200}{\meter/\second}$. tính lực cản của tường tác dụng lên viên đạn.
	\item %câu 6
	Một quả bóng $\SI{2,5}{\kilogram}$ đập vào tường với vận tốc $\SI{8,5}{\meter/\second}$ và bị bật ngược trở lại với vận tốc 7,5m/s. Biết thời gian va chạm là $\SI{0,25}{\second}$. Tìm lực mà tường tác dụng lên quả bóng.
\end{enumerate}

\textbf{ĐÁP ÁN}
\begin{longtable}[\textwidth]{|m{0.16\textwidth}|m{0.16\textwidth}|m{0.17\textwidth}|m{0.17\textwidth}|m{0.1\textwidth}|m{0.1\textwidth}|}
	% --- first head
	\hline%\hspace{2 pt}
	\multicolumn{1}{|c}{\textbf{Câu 1}} &
	\multicolumn{1}{|c|}{\textbf{Câu 2}}& 
	\multicolumn{1}{c|}{\textbf{Câu 3}} &
	\multicolumn{1}{c|}{\textbf{Câu 4}} &
	\multicolumn{1}{c|}{\textbf{Câu 5}} &
	\multicolumn{1}{c|}{\textbf{Câu 6}} \\
	\hline
	$\SI{4,9}{\kilogram\cdot\meter\cdot\second^{-1}}$.&
	$\SI{-3}{\kilogram\cdot\meter\cdot\second^{-1}}$.&
	\begin{enumerate}[label=\alph*)]
		\item $\SI{1500}{\newton}$.
		\item $\SI{15000}{\newton}$.
	\end{enumerate} &
	$\SI{240}{\kilogram\cdot\meter\cdot\second^{-1}}$.&
	$\SI{400}{\newton}$.	&
	$\SI{160}{\newton}$.\\
	\hline
\end{longtable}

\section{Định luật bảo toàn động lượng}
\begin{enumerate}
	\item %câu 1
	Một vật $\SI{500}{\gram}$ chuyển động với vận tốc $\SI{4}{\meter/\second}$ không ma sát trên mặt phẳng ngang thì va chạm với vật thứ hai có khối lượng $\SI{300}{\gram}$ đang đứng yên. Sau va chạm, hai vật dính làm một. Tìm vận tốc của hai vật sau va chạm.
	\item %câu 2
	Vật $m_1$ chuyển động với vận tốc $\SI{6}{\meter/\second}$ đến va chạm với vật $m_2$ chuyển động ngược chiều với vận tốc $\SI{2}{\meter/\second}$. Sau va chạm, hai vật bật ngược trở lại với vận tốc $\SI{4}{\meter/\second}$. Tính khối lượng của hai vật biết $m_1+m_2=\SI{ 1,5}{\kilogram}$.
	\item %câu 3
	Vật $\SI{200}{\gram}$ chuyển động với vận tốc $\SI{6}{\meter/\second}$ đến va chạm với vật $\SI{50}{\gram}$ chuyển động với vận tốc $\SI{4}{\meter/\second}$. Sau va chạm vật $\SI{200}{\gram}$ giữ nguyên hướng và chuyển động với vận tốc bằng nửa vận tốc ban đầu. Tính vận tốc của vật còn lại trong các trường hợp sau
	\begin{enumerate}[label=\alph*)]
		\item Trước va chạm hai vật chuyển động cùng chiều.
		\item Trước va chạm hai vật chuyển động ngược chiều.
	\end{enumerate}
	\item %câu 4
	Một viên đạn khối lượng $m_1=\SI{200}{\gram}$ chuyển động thẳng với vận tốc $v_1=\SI{100}{\meter/\second}$, đến va chạm mềm dính vào một bao cát đang đứng yên có khối lượng $m_2=\SI{100}{\gram}$. Tính vận tốc của đạn và bao cát ngay sau va chạm bằng.
	\item %câu 5
	Một khẩu súng nằm ngang khối lượng $m_\text{s}=\SI{1000}{\kilogram}$, bắn một viên đạn khối lượng $m_\text{đ}=\SI{10}{\gram}$. Vận tốc viên đạn ra khỏi nòng súng là $\SI{600}{\meter/\second}$. Độ lớn vận tốc của súng sau khi bắn bằng là bao nhiêu?
	\item %câu 6
	Tên lửa khối lượng 10 tấn chuyển động với vận tốc $\SI{200}{\meter/\second}$ so với Trái Đất, 2 tấn khí phụt ra có vận tốc $\SI{500}{\meter/\second}$ so với tên lửa. Xác định vận tốc của tên lửa sau khi khí phụt ra trong các trường hợp sau
	\begin{enumerate}[label=\alph*)]
		\item Khối khí phụt ra phía sau.
		\item Khối khí phụt ra phía trước.
	\end{enumerate}
\end{enumerate}

\textbf{ĐÁP ÁN}
\begin{longtable}[\textwidth]{|m{0.1\textwidth}|m{0.2\textwidth}|m{0.16\textwidth}|m{0.1\textwidth}|m{0.1\textwidth}|m{0.17\textwidth}|}
	% --- first head
	\hline%\hspace{2 pt}
	\multicolumn{1}{|c}{\textbf{Câu 1}} &
	\multicolumn{1}{|c|}{\textbf{Câu 2}}& 
	\multicolumn{1}{c|}{\textbf{Câu 3}} &
	\multicolumn{1}{c|}{\textbf{Câu 4}} &
	\multicolumn{1}{c|}{\textbf{Câu 5}} &
	\multicolumn{1}{c|}{\textbf{Câu 6}} \\
	\hline
	$\SI{2,5}{\meter/\second}$. &
	$\begin{cases}
		m_1=\SI{0,5625}{\kilogram}\\
		m_2=\SI{0,9375}{\kilogram}.
	\end{cases}$ &
	\begin{enumerate}[label=\alph*)]
		\item $\SI{16}{\meter/\second}$.
		\item $\SI{8}{\meter/\second}$.
	\end{enumerate} &
	$\SI{0,2}{\meter/\second}$. &
	$\SI{-1,5}{\meter/\second}$.	&
	\begin{enumerate}[label=\alph*)]
		\item $\SI{325}{\meter/\second}$.
		\item $\SI{75}{\meter/\second}$.
	\end{enumerate}\\
	\hline
\end{longtable}





