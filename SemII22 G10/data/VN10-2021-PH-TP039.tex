\whiteBGstarBegin
\setcounter{section}{0}
\begin{enumerate}[label=\bfseries Câu \arabic*:]
	
	\item \mkstar{1} [14]
	
	\cauhoi{
		Chiều của lực căng bề mặt chất lỏng có tác dụng
		\begin{mcq}
			\item làm tăng diện tích mặt thoáng của chất lỏng.
			\item làm giảm diện tích mặt thoáng của chất lỏng.
			\item giữ cho diện tích mặt thoáng của chất lỏng luôn ổn định.
			\item giữ cho mặt thoáng của chất lỏng luôn nằm ngang.
		\end{mcq}
	}
	
	\loigiai{
		\textbf{Đáp án: B.}
		
		Chiều của lực căng bề mặt chất lỏng có tác dụng làm giảm diện tích mặt thoáng của chất lỏng.
	}
	
	\item \mkstar{1}
	
	\cauhoi{
		Chọn phương án \textbf{sai} khi nói về lực căng bề mặt của chất lỏng.
		\begin{mcq}
			\item Phụ thuộc vào bản chất của chất lỏng.
			\item Phụ thuộc vào nhiệt độ của chất lỏng.
			\item Tăng khi nhiệt độ tăng.
			\item Có giá trị bằng $f/l$.
		\end{mcq}
		
	}
	
	\loigiai{
		\textbf{Đáp án: C.}
		
		Hệ số căng bề mặt $\sigma$ phụ thuộc vào bản chất và nhiệt độ của chất lỏng ($\sigma$ giảm khi nhiệt độ tăng). Do đó khi nói $\sigma$ tăng khi nhiệt độ tăng là sai.
	}
	
	\item \mkstar{3}
	
	\cauhoi{
		Một vòng dây kim loại có đường kính $\SI{8}{\centi \meter}$ được dìm nằm ngang trong một chậu dầu thô. Khi kéo vòng dây ra khỏi dầu, người ta đo được lực phải tác dụng thêm do lực căng bề mặt là $\SI{9.2e-3}{\newton}$. Hệ số căng bề mặt của dầu trong chậu dầu ấy là
		\begin{mcq}(2)
			\item $\sigma=\SI{18.3e-3}{\newton / \meter}$.
			\item $\sigma=\SI{18.3e-4}{\newton / \meter}$.
			\item $\sigma=\SI{18.3e-5}{\newton / \meter}$.
			\item $\sigma=\SI{18.3e-6}{\newton / \meter}$.
		\end{mcq}
	}
	
	\loigiai{\textbf{Đáp án: A.}
				
		Đổi $d=\SI{8}{\centi \meter} = \SI{8e-2}{\meter}$.
		
		Hệ số căng bề mặt của dầu trong chậu là
		\begin{equation*}
			f=\sigma l = \sigma \cdot 2 \pi d
			\Leftrightarrow \sigma = \dfrac{f}{2\pi d} \approx \SI{18.3e-3}{\newton / \meter}.
		\end{equation*}.
	}
	
\end{enumerate}
\whiteBGstarEnd