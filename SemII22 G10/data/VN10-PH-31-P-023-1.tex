%\setcounter{chapter}{1}
\chapter{Luyện tập: Động năng}

\begin{enumerate}
	\item %câu 1
	Nhận định nào sau đây về động năng là không đúng?
	\begin{mcq}(1)
		\item Động năng là đại lượng vô hướng và luôn dương.
		\item Động năng có tính tương đối, phụ thuộc hệ quy chiếu.
		\item Động năng tỷ lệ thuận với khối lượng và vận tốc của vật.
		\item Động năng là năng lượng của vật đang chuyển động.
	\end{mcq}
	\item %câu 2
	Một ô tô có khối lượng 1,5 tấn đang chuyển động thẳng đều trong 2 giờ xe đi được quãng đường $\SI{72}{\kilo\meter}$. Động năng của ô tô này bằng
	\begin{mcq}(4)
		\item $\SI{972}{\joule}$.
		\item $\SI{150}{\kilo\joule}$.
		\item $\SI{75}{\kilo\joule}$.
		\item $\SI{972}{\kilo\joule}$.
	\end{mcq}
	\item %câu 3
	Một hòn đá có khối lượng $m=\SI{200}{\gram}$ rơi tự do không vận tốc đầu từ một điểm cách mặt đất $\SI{45}{\meter}$, tại nơi có gia tốc trọng trường $g= \SI{10}{\meter/\second^2}$. Động năng của hòn đá ngay trước khi chạm đất là
	\begin{mcq}(4)
		\item $\SI{45}{\joule}$.
		\item $\SI{90}{\joule}$.
		\item $\SI{180}{\joule}$.
		\item $\SI{900}{\joule}$.
	\end{mcq}
	\item %câu 4
	Một ô tô có khối lượng 4 tấn đang chuyển động với vận tốc $\SI{36}{\kilo\meter/\hour}$ thì hãm phanh, sau một thời gian vận tốc giảm còn $\SI{18}{\kilo\meter/\hour}$. Độ biến thiên của động năng của ô tô là
	\begin{mcq}(4)
		\item $\SI{150}{\kilo\joule}$.
		\item $\SI{-150}{\kilo\joule}$.
		\item $\SI{-75}{\kilo\joule}$.
		\item $\SI{75}{\kilo\joule}$.
	\end{mcq}
	\item %câu 5
	Một ô tô có khối lượng $\SI{1600}{\kilogram}$ đang chạy với tốc độ $\SI{54}{\kilo\meter/\hour}$ thì người lái xe nhìn thấy một vật cản trước mặt cách khoảng $\SI{10}{\meter}$. Người đó tắt máy và hãm phanh khẩn cấp với lực hãm không đổi là $\SI{2e4}{\newton}$. Xe dừng lại cách vật cản một khoảng bằng
	\begin{mcq}(4)
		\item $\SI{1,2}{\meter}$.
		\item $\SI{1,0}{\meter}$.
		\item $\SI{1,4}{\meter}$.
		\item $\SI{1,5}{\meter}$.
	\end{mcq}
	\item %câu 6
	Vận động viên Hoàng Xuân Vinh bắn một viên đạn có khối lượng $\SI{100}{\gram}$ bay ngang với vận tốc $\SI{300}{\meter/\second}$ xuyên qua tấm bia bằng gỗ dày $\SI{5}{\centi\meter}$. Sau khi xuyên qua bia gỗ thì đạn có vận tốc $\SI{100}{\meter/\second}$. Tính lực cản của tấm bia gỗ tác dụng lên viên đạn.
	\item %câu 7
	Trung tâm bồi dưỡng kiến thức Hà Nội tổ chức một cuộc thi cho các học viên chạy. Có một học viên có trọng lượng $\SI{700}{\newton}$ chạy đều hết quãng đường $\SI{600}{\meter}$ trong $\SI{50}{\second}$. Tìm động năng của học viên đó. Lấy $g= \SI{10}{\meter/\second^2}$.
	\item %câu 8
	Một vật có khối lượng $\SI{2}{\kilogram}$ trượt qua A với vận tốc $\SI{2}{\meter/\second}$ xuống dốc nghiêng AB dài $\SI{2}{\meter}$, cao $\SI{1}{\meter}$. Biết hệ số ma sát giữa vật và mặt phẳng nghiêng là $\mu=\dfrac{1}{\sqrt{3}}$. Lấy $g= \SI{10}{\meter/\second^2}$.
	\begin{enumerate}[label=\alph*)]
		\item Xác định công của trọng lực, công của lực ma sát thực hiện khi vật chuyển dời từ dinh dốc đến chân dốc.
		\item Xác định vận tốc của vật tại chân dốc B.
		\item Tại chân dốc B vật tiếp tục chuyển dộng trên mặt phẳng nằm ngang BC dài $\SI{2}{\meter}$ thì dừng lại. Xác định hệ số ma sát trên đoạn dường BC này.
	\end{enumerate}
	\item %câu 9
	Một ô tô có khối lượng 2 tấn đang chuyển dộng trên đường thẳng nằm ngang AB dài $\SI{100}{\meter}$, khi qua A vận tốc ô tô là $\SI{10}{\meter/\second}$ và đến B vận tốc của ô tô là $\SI{20}{\meter/\second}$. Biết độ lớn của lực kéo là $\SI{4000}{\newton}$.
	\begin{enumerate}[label=\alph*)]
		\item Tìm hệ số ma sát $\mu_1$ trên đoạn đường AB.
		\item Đến B thì động cơ tắt máy và lên dốc BC dài $\SI{40}{\meter/\second}$ nghiêng $30^\circ$ so với mặt phẳng ngang. Hệ số ma sát trên mặt dốc là $\mu_2=\dfrac{1}{5\sqrt{3}}$. Hỏi xe có lên đến đỉnh dốc C không?
		\item Nếu đến B với vận tốc trên, muốn xe lên dốc và dừng lại tại C thì phải tác dụng lên xe một lực có độ lớn thế nào?
	\end{enumerate}
	\item %câu 10
	Một xe có khối lượng 2 tấn chuyên động trên đoạn AB nằm ngang với vận tốc không đổi $\SI{7,2}{\kilo\meter/\hour}$. Hệ số ma sát giữa xe và mặt đường là $\mu=0,2$, lấy $g= \SI{10}{\meter/\second^2}$.
	\begin{enumerate}[label=\alph*)]
		\item Tính lực kéo của động cơ.
		\item Đến điểm B thì xe tắt máy và xuống dốc BC nghiêng góc $30^\circ$ so với phương ngang, bỏ qua ma sát. Biết vận tốc tại chân C là $\SI{72}{\kilo\meter/\hour}$. Tìm chiều dài dốc BC.
		\item Tại C xe tiếp tục chuyên động trên đoạn đường nằm ngang CD và đi thêm được $\SI{200}{\meter}$ thì dừng lại. Tìm hệ số ma sát trên đoạn CD.
	\end{enumerate}
\end{enumerate}

\textbf{ĐÁP ÁN:}
\begin{longtable}[\textwidth]{|m{0.15\textwidth}|m{0.17\textwidth}|m{0.15\textwidth}|m{0.2\textwidth}|m{0.2\textwidth}|}
	% --- first head
	\hline%\hspace{2 pt}
	\multicolumn{1}{|c}{\textbf{Câu 1}} &
	\multicolumn{1}{|c|}{\textbf{Câu 2}}& 
	\multicolumn{1}{c|}{\textbf{Câu 3}} &
	\multicolumn{1}{c|}{\textbf{Câu 4}} &
	\multicolumn{1}{c|}{\textbf{Câu 5}}   \\
	\hline
	C. &
	C. &
	B. &
	B. &
	B. \\
	\hline
	\multicolumn{1}{|c}{\textbf{Câu 6}} &
	\multicolumn{1}{|c|}{\textbf{Câu 7}}&  
	\multicolumn{1}{c|}{\textbf{Câu 8}} &
	\multicolumn{1}{c|}{\textbf{Câu 9}} &
	\multicolumn{1}{c|}{\textbf{Câu 10}} \\
	\hline
	$\SI{80000}{\newton}$. &
	$\SI{5040}{\joule}$. &
	\begin{enumerate}[label=\alph*)]
		\item $\SI{-20}{\joule}$.
		\item $\SI{2}{\meter/\second}$.
		\item 0,1.
	\end{enumerate} &
	\begin{enumerate}[label=\alph*)]
		\item 0,05.
		\item $\SI{33,333}{\meter}$.
		\item $\SI{2000}{\newton}$.
	\end{enumerate} &
	\begin{enumerate}[label=\alph*)]
		\item $\SI{4000}{\newton}$.
		\item $\SI{39,6}{\meter}$.
		\item 0,1.
	\end{enumerate}\\
	\hline
\end{longtable}


