\whiteBGstarBegin
\setcounter{section}{0}
\section{Cấu tạo chất. Thuyết động học phân tử chất khí}
\begin{enumerate}[label=\bfseries Câu \arabic*:]
	
	\item \mkstar{1}
	
	\cauhoi{
		Trong các câu sau đây, câu nào \textbf{sai}?
		\begin{mcq}(1)
			\item Các chất được cấu tạo một cách gián đoạn.
			\item Các nguyên tử, phân tử đứng sát nhau và giữa chúng không có khoảng cách.
			\item Lực tương tác giữa các phân tử ở thể rắn lớn hơn lực tương tác giữa các phân tử ở thể lỏng và thể khí.
			\item Các nguyên tử, phân tử chất lỏng dao động xung quanh các vị trí cân bằng không cố định.
		\end{mcq}
	}
	
	\loigiai{
		\textbf{Đáp án: B.}
		
		Các nguyên tử, phân tử không đứng sát nhau và giữa chúng có khoảng cách.
	}
	
	\item \mkstar{1}
	
	\cauhoi{
		Tính chất nào sau đây không phải là tính chất của các phân tử khí?
		\begin{mcq}(1)
			\item Có vận tốc trung bình phụ thuộc vào nhiệt độ.
			\item Gây áp suất lên thành bình.
			\item Chuyển động xung quanh vị trí cân bằng.
			\item Chuyển động nhiệt hỗn loạn.
		\end{mcq}
		
	}
	
	\loigiai{
		\textbf{Đáp án: C.}
		
		Các phân tử khí không chuyển động xung quanh vị trí cân bằng, chúng chuyển động hỗn loạn về mọi phía.
	}
	
	\item \mkstar{1}
	
	\cauhoi{
		Chuyển động nào sau đây là chuyển động của riêng các phân tử ở thể lỏng?
		\begin{mcq}(1)
			\item Chuyển động hỗn loạn không ngừng.
			\item Dao động xung quanh các vị trí cân bằng cố định.
			\item Chuyển động hoàn toàn tự do.
			\item Dao động xung quanh các vị trí cân bằng không cố định.
		\end{mcq}
	}
	
	\loigiai{\textbf{Đáp án: D.}
				
		Các phân tử chất lỏng dao động xung quanh các vị trí cân bằng không cố định..
	}
	
	
	\item \mkstar{1}
	
	\cauhoi{
		Câu nào sau đây nói về khí lí tưởng là không đúng?
		\begin{mcq}(1)
			\item Khí lí tưởng là khí mà thể tích của các phân tử có thể bỏ qua.
			\item Khí lí tưởng là khí mà khối lượng của các phân tử có thể bỏ qua.
			\item Khí lí tưởng là khí mà các phân tử chỉ tương tác khi va chạm.
			\item Khí lí tưởng là khí có thể gây áp suất lên thành bình chứa.
		\end{mcq}
	}
	
	\loigiai{\textbf{Đáp án: B.}
		
		Khí lí tưởng là không phải là khí mà khối lượng của các phân tử có thể bỏ qua.
	}
	
	\item \mkstar{1}
	
	\cauhoi{
		Câu nào sau đây nói về lực tương tác phân tử là không đúng?
		\begin{mcq}(1)
			\item Lực tương tác phân tử đáng kể khi các phân tử ở rất gần nhau.
			\item Lực hút phân tử có thể lớn hơn lực đẩy phân tử.
			\item Lực hút phân tử không thể lớn hơn lực đẩy phân tử.
			\item Lực hút phân tữ có thể bằng lực đẩy phân tử.
		\end{mcq}
	}
	
	\loigiai{\textbf{Đáp án: C.}
		
		Lực hút phân tử có thể lớn hơn lực đẩy phân tử nếu các phân tử cách nhau một khoảng cách nhất định.
		
	}

\item \mkstar{1}

\cauhoi{
	Tìm tỉ số khối lượng phân tử nước và nguyên tử Cacbon 12.
}

\loigiai{
	
	Phân tử khối của nước: $M_{\ce{H_2 O}} = 1\cdot 2 + 16 = 18$.
	
	Phân tử khối của Cacbon 12: $M_{\ce{C}} = 12$.
	
	Vậy tỉ số là $\dfrac{3}{2}$.
	
}

\item \mkstar{2}

\cauhoi{
	Tính số lượng phân tử $\text{H}_2\text{O}$ trong $\SI{1}{\gram}$ nước.
}

\loigiai{
	
	Số phân tử nước (với $M=18$): $N=n \cdot N_\text{A} = \dfrac{m}{M} \cdot N_\text A = \SI{3.3456e22}{}$ phân tử.
	
}

\item \mkstar{2}

\cauhoi{
	Hãy xác định
	\begin{enumerate}[label=\alph*)]
		\item Tỉ số khối lượng phân tử nước và nguyên tử cacbon $\text{C}_{12}$.
		\item Số phân tử $\text{H}_2\text{O}$ trong $\SI{2}{\gram}$ nước.
	\end{enumerate}
}

\loigiai{
	\begin{enumerate}[label=\alph*)]
		\item Tỉ số khối lượng phân tử nước và nguyên tử cacbon $\text{C}_{12}$.
		
		Tỉ số bằng $\dfrac{3}{2}$.
		
		\item Số phân tử $\text{H}_2\text{O}$ trong $\SI{2}{\gram}$ nước.
		
		Số phân tử nước (với $M=18$): $N=n \cdot N_\text{A} = \dfrac{m}{M} \cdot N_\text A = \SI{6.6912e22}{}$ phân tử.
	\end{enumerate}
	
	
}

\item \mkstar{2}

\cauhoi{
		Một bình kín chứa $N=3,01\cdot10^{23}$ phân tử khí Heli. Tính khối lượng khí Heli trong bình.
		
		
}

\loigiai{
	Khối lượng khí Heli trong bình (với $M=4$):
	$$N=\dfrac{m}{M}\cdot N_\text{A} \Rightarrow m = \dfrac{N M}{N_\text{A}} = \SI{1.9993}{g}$$
	
}

\item \mkstar{3}

\cauhoi{
	\begin{enumerate}[label=\alph*)]
		\item Tính số phân tử chứa trong $\SI{0,2}{\kilogram}$ nước.
		\item Tính số phân tử chứa trong $\SI{1}{\kilogram}$ không khí nếu như không khí có $22\%$ là oxi và $78\%$ là khí nitơ.
	\end{enumerate}			
}

\loigiai{
	\begin{enumerate}[label=\alph*)]
		\item Tính số phân tử chứa trong $\SI{0,2}{\kilogram}$ nước.
		
		Số phân tử nước (với $M=18$): $N=n \cdot N_\text{A} = \dfrac{m}{M} \cdot N_\text A = \SI{6.6912e24}{}$ phân tử.
		
		\item Tính số phân tử chứa trong $\SI{1}{\kilogram}$ không khí nếu như không khí có $22\%$ là oxi và $78\%$ là khí nitơ.
		
		Khối lượng của $22\%$ oxi trong $\SI{1}{kg}$ không khí:
		$$M_{\ce{O_2}} = 0,22 \cdot \SI{1000}{g} = \SI{220}{g}$$
		
		Khối lượng của $78\%$ nitơ trong $\SI{1}{kg}$ không khí:
		$$M_{\ce{N_2}} = 0,78 \cdot \SI{1000}{g} = \SI{780}{g}$$
		
		Số phân tử oxi (với $M=32$): $N_1=n \cdot N_\text{A} = \dfrac{m}{M} \cdot N_\text A = \SI{4.14e24}{}$ phân tử.
		
		Số phân tử nitơ (với $M=28$): $N_2=n \cdot N_\text{A} = \dfrac{m}{M} \cdot N_\text A = \SI{1.68e25}{}$ phân tử.
		
		Tổng số phân tử: $N=N_1+N_2 = \SI{2.1e25}{}$ phân tử.
	\end{enumerate}			
	
}
\end{enumerate}

\section{Quá trình đẳng nhiệt. Định luật Bôi-lơ - Ma-ri-ốt}
\begin{enumerate}[label=\bfseries Câu \arabic*:]
	\item \mkstar{2}
	
	\cauhoi{
		Nén khí đẳng nhiệt từ thể tích $\SI{9}{l}$ đến thể tích $\SI{6}{l}$ thì thấy áp suất tăng lên một lượng $\Delta p=\SI{40}{\kilo Pa}$. Hỏi áp suất ban đầu của khí là bao nhiêu?
		\begin{mcq}(4)
			\item $\SI{80}{\kilo Pa}$.
			\item $\SI{80}{Pa}$.
			\item $\SI{40}{\kilo Pa}$.
			\item $\SI{40}{Pa}$.
		\end{mcq}
		
	}
	
	\loigiai{
		\textbf{Đáp án: A.}
		
		Đẳng nhiệt (1): $p_1$, $V_1=\SI{9}{l}$ sang (2): $V_2=\SI{6}{l}$, $p_2=p_1+\SI{40}{kPa}$.
		
		Vì quá trình là đẳng nhiệt nên ta có phương trình:
		
		$$p_1V_1=p_2V_2 \Rightarrow p_1 =\dfrac{p_2V_2}{V_1}= \SI{80}{kPa}$$
	}
	
	\item \mkstar{2}
	
	\cauhoi{
		Dưới áp suất $\SI{e5}{Pa}$ một lượng khí có thể tích $\SI{10}{l}$. Tính thể tích của khí đó dưới áp suất $\SI{3e5}{Pa}$. Coi nhiệt độ không đổi trong suốt quá trình.
		\begin{mcq}(4)
			\item $\SI{10}{l}$.
			\item $\SI{3,3}{l}$.
			\item $\SI{50}{l}$.
			\item $\SI{30}{l}$.
		\end{mcq}
		
	}
	
	\loigiai{
		\textbf{Đáp án: B.}
		
		Đẳng nhiệt (1): $p_1=\SI{e5}{Pa}$, $V_1=\SI{10}{l}$ sang (2): $V_2=?$, $p_2=\SI{3e5}{Pa}$
		
		Vì quá trình là đẳng nhiệt nên ta có phương trình:
		
		$$p_1V_1=p_2V_2 \Rightarrow V_2 =\dfrac{p_1V_1}{p_2} = \SI{3.3}{l}$$
	}
	\item \mkstar{3}
	
	\cauhoi{
		Người ta điều chế khí hidro và chứa vào một bình lớn dưới áp suất $\SI{1}{atm}$ ở nhiệt độ $20^\circ\text{C}$. Coi quá trình này là đẳng nhiệt. Tính thể tích khí phải lấy từ bình lớn ra để nạp vào bình nhỏ có thể tích $\SI{20}{l}$ ở áp suất $\SI{25}{atm}$.
		\begin{mcq}(4)
			\item $\SI{250}{l}$.
			\item $\SI{500}{l}$.
			\item $\SI{750}{l}$.
			\item $\SI{1000}{l}$.
		\end{mcq}
	}
	
	\loigiai{
		\textbf{Đáp án: B.}
		
	Đẳng nhiệt (1): $p_1=\SI{1}{atm}$, $V_1=?$ sang (2): $V_2=\SI{20}{l}$, $p_2=\SI{25}{atm}$.
	
	Vì quá trình là đẳng nhiệt nên ta có phương trình:
	
	$$p_1V_1=p_2V_2 \Rightarrow V_1 =\dfrac{p_2V_2}{p_1} = \SI{500}{l}$$
	}
	
	

\item \mkstar{3}

\cauhoi{
	Một lượng khí có $V_1=\SI{3}{l}$, $p_1=\SI{3e5}{Pa}$. Hỏi khi nén $V_2=\dfrac{2}{3}V_1$ thì áp suất của nó là bao nhiêu? Coi nhiệt độ không đôi trong suốt quá trình.
	\begin{mcq}(4)
		\item $\SI{4,5e5}{Pa}$.
		\item $\SI{3e5}{Pa}$.
		\item $\SI{2,1e5}{Pa}$.
		\item $\SI{0,67e5}{Pa}$.
	\end{mcq}
}

\loigiai{
	\textbf{Đáp án: A.}
	
	Đẳng nhiệt (1): $p_1=\SI{3e5}{Pa}$, $V_1=\SI{3}{l}$ sang (2): $V_2=\dfrac{2}{3}V_1=\SI{2}{l}$, $p_2=?$
	
	Vì quá trình là đẳng nhiệt nên ta có phương trình:
	
	$$p_1V_1=p_2V_2 \Rightarrow p_2 =\dfrac{p_1V_1}{V_2} = \SI{4.5e5}{Pa}$$
}

\item \mkstar{3}

\cauhoi{
	Nếu áp suất của một lượng khí tăng thêm $\SI{2,1e5}{Pa}$ thì thể tích giảm $\SI{3}{l}$. Nếu áp suất tăng thêm $\SI{5e5}{Pa}$ thì thể tích giảm $\SI{5}{l}$. Tìm áp suất và thể tích ban đầu của khí, biết nhiệt độ khí không đổi. (Chọn đáp án gần đúng nhất)
	\begin{mcq}(4)
		\item $\SI{4.6e5}{Pa}$, $\SI{10}{l}$.
		\item $\SI{4e5}{Pa}$, $\SI{10}{l}$.
		\item $\SI{4e5}{Pa}$, $\SI{3}{l}$.
		\item $\SI{4.6e5}{Pa}$, $\SI{3}{l}$.
	\end{mcq}
}

\loigiai{
	\textbf{Đáp án: A.}
	
	Đẳng nhiệt (1): $p_1=?$, $V_1=?$ sang (2): $V_2=V_1-\SI{3}{l}$, $p_2=p_1+\SI{2.1e5}{Pa}$
	
	Vì quá trình là đẳng nhiệt nên ta có phương trình:
	
	$$p_1V_1=p_2V_2 \Rightarrow p_1V_1 = (p_1+\SI{2.1e5}{Pa})(V_1-\SI{3}{l}) \Rightarrow -3p_1 + \SI{2.1e5}{}V_1 = \SI{6.3e5}{}$$
	
	Đẳng nhiệt (1): $p_1=?$, $V_1=?$ sang (3): $V_3=V_1-\SI{5}{l}$, $p_3=p_1+\SI{5e5}{Pa}$
	
	Vì quá trình là đẳng nhiệt nên ta có phương trình:
	
	$$p_1V_1=p_3V_3 \Rightarrow p_1V_1 = (p_1+\SI{5e5}{Pa})(V_1-\SI{5}{l}) \Rightarrow -5p_1 + \SI{5e5}{}V_1 = \SI{2.5e6}{}$$
	
	Giải hệ hai phương trình trên, tính được: $p_1=\SI{4.66e5}{Pa}$; $V_1=\SI{9.67}{l}$.
}

\item \mkstar{4}

\cauhoi{
	Mỗi lần bơm đưa được $V_0=\SI{80}{\centi\meter^3}$ không khí vào ruột xe. Sau khi bơm diện tích tiếp xúc của nó với mặt đường là $\SI{30}{\centi\meter^2}$, thể tích ruột xe sau khi bơm là $\SI{2000}{\centi\meter^3}$, áp suất khí quyển là $\SI{1}{atm}$, trọng lượng xe là $\SI{600}{\newton}$. Coi nhiệt độ không đổi trong quá trình bơm. Số lần phải bơm là
\begin{mcq}(4)
	\item 100.
	\item 48.
	\item 240.
	\item 50.
\end{mcq}
	
}

\loigiai{
	\textbf{Đáp án: D.}
	
	Áp suất khí trong xe sau khi bơm:
	$$p_2 = \dfrac{P}{S} =\dfrac{\SI{600}{N}}{\SI{3e-3}{m^2}}= \SI{2e5}{N/m^2} = \SI{2e5}{Pa}$$
	
	Đẳng nhiệt (1): $p_1=\SI{1}{atm} = \SI{e5}{Pa}$, $V_1=?$ sang (2): $V_2=\SI{2000}{cm^3}$, $p_2=\SI{2e5}{Pa}$
	
	Vì quá trình là đẳng nhiệt nên ta có phương trình:
	
	$$p_1V_1=p_2V_2 \Rightarrow V_1 = \dfrac{p_2V_2}{p_1} = \SI{4000}{cm^3}$$
	
	Số lần bơm là: $N=\dfrac{V_1}{V_0} = 50$ lần.
}

\item \mkstar{2}

\cauhoi{
		Một lượng khí xác định ở áp suất $\SI{3}{atm}$, có thể tích là 10 lít. Tính thể tích của khối khí khi nén đẳng nhiệt đến áp suất $\SI{6}{atm}$.
	
}

\loigiai{

	Đẳng nhiệt (1): $p_1=\SI{3}{atm}$, $V_1=\SI{10}{l}$ sang (2): $V_2=?$, $p_2=\SI{6}{atm}$.
	
	Vì quá trình là đẳng nhiệt nên ta có phương trình:
	
	$$p_1V_1=p_2V_2 \Rightarrow V_2 = \dfrac{p_1V_1}{p_2} = \SI{5}{l}$$
}
	
	
	\item \mkstar{3}
	
	\cauhoi{
		Một quả bóng có dung tích $\SI{2,5}{l}$. Người ta bơm không khí ở áp suất khí quyển $\SI{e5}{\newton/\meter^2}$ vào bóng. Mỗi lần bơm được $\SI{125}{\centi\meter^3}$ không khí. Hỏi áp suất của không khí trong quả bóng sau 40 lần bơm? Coi quả bóng trước khi bơm không có không khí và trong thời gian bơm nhiệt độ của không khí không đổi.
	}
	
	\loigiai{
		Đổi $\Delta V = \SI{125}{cm^3} = \SI{0.125}{l}$.
		
		Sau 40 lần bơm thì $\Delta V_{40} = 40\Delta V = \SI{5}{l}$.
		
		Xem quá trình như là đẳng nhiệt nén lượng không khí từ bên ngoài (40 lần bơm) nén vào trong 1 quả bóng có thể tích 2,5 lít: (1): $p_1=\SI{e5}{N/m^2}$, $V_1=\SI{5}{l}$ sang (2): $V_2=\SI{2,5}{l}$, $p_2=?$
		
		Vì quá trình là đẳng nhiệt nên ta có phương trình:
		
		$$p_1V_1=p_2V_2 \Rightarrow p_2 =\dfrac{V_1}{V_2}p_1 = \SI{2e5}{N/m^2}$$
		
	}
\item \mkstar{4}

\cauhoi{
	Một bọt khí khi nổi lên từ một đáy hồ có độ lớn gấp 1,2 lần khi đến mặt nước. Tính độ sâu của đáy hồ biết trọng lượng riêng của nước là $d=\SI{e4}{N/\meter^3}$, áp suất khí quyển là $\SI{e5}{\newton/\meter^2}$.
}

\loigiai{		
	Sự thay đổi áp suất theo độ sâu, gọi $p_1$ là áp suất dưới đáy hồ, $p_2$ là áp suất tại bề mặt. Ta có:
	$$p_1 = p_2 + d h$$
	
	Đẳng nhiệt (1): $p_1=p_2 + dh$, $V_1$ sang (2): $V_2=1,2V_1$, $p_2=\SI{e5}{N/m^2}$.
	
	Vì quá trình là đẳng nhiệt nên ta có phương trình:
	
	$$p_1V_1=p_2V_2 \Rightarrow (10^5 + 10^4 \cdot h) = 1,2 \cdot 10^5 \Rightarrow h = \SI{2}{m}$$
}


\item \mkstar{4}

\cauhoi{
	Một bong bóng khí ở độ sâu $\SI{5}{\meter}$ có thể tích thay đổi như thế nào khi nổi lên mặt nước? Cho áp suất tại mặt nước là $\SI{e5}{Pa}$, khối lượng riêng của nước là $\SI{1000}{\kilogram/\meter^3}$, gia tốc trọng trường là $g=\SI{10}{\meter/\second^2}$.
}

\loigiai{
	Sự thay đổi áp suất theo độ sâu, gọi $p_1$ là áp suất dưới đáy hồ, $p_2$ là áp suất tại bề mặt. Ta có:
	$$p_1 = p_2 + Dgh$$
	
	Đẳng nhiệt (1): $p_1=p_2 + Dgh$, $V_1$ sang (2): $V_2=?$, $p_2=\SI{e5}{Pa}$.
	
	Vì quá trình là đẳng nhiệt nên ta có phương trình:
	
	$$p_1V_1=p_2V_2 \Rightarrow (10^5 + 10^4 \cdot 5) V_1 = V_2 \cdot 10^5 \Rightarrow V_2 = 1,5V_1$$
	
	Vậy thể tích bong bóng tăng 1,5 lần.
}
\end{enumerate}

\section{Quá trình đẳng tích. Định luật Sác-lơ}
\begin{enumerate}[label=\bfseries Câu \arabic*:]
		\item \mkstar{1}
	
	\cauhoi{
		Một bình được nạp khí ở $33^\circ\text{C}$ dưới áp suất $\SI{300}{Pa}$. Sau đó bình được chuyển đến một nơi có nhiệt độ $37^\circ\text{C}$. Tính độ tăng áp suất của khí trong bình.
		\begin{mcq}(4)
			\item $\SI{303,9}{Pa}$.
			\item $\SI{3,9}{Pa}$.
			\item $\SI{336,4}{Pa}$.
			\item $\SI{36,4e5}{Pa}$.
		\end{mcq}
		
	}
	
	\loigiai{
		\textbf{Đáp án: B.}
		
		Đẳng tích (1): $p_1=\SI{300}{Pa}$, $T_1=\SI{306}{K}$ sang (2): $p_2=?$, $T_2=\SI{310}{K}$
		
		Vì quá trình là đẳng tích nên ta có phương trình:
		
		$$\dfrac{p_1}{T_1} = \dfrac{p_2}{T_2} \Rightarrow p_2 =\dfrac{T_2p_1}{T_1} = \SI{303.9}{Pa}$$
		
		Độ tăng áp suất: $\Delta p = p_2 - p_1 = \SI{3.9}{Pa}$
	}

	\item \mkstar{2}
	
	\cauhoi{
		Đun nóng một khối khí được đựng trong một bình kín làm cho nhiệt độ của nó tăng thêm $1^\circ\text{C}$ thì người ta thấy rằng áp suất của khối khí trong bình tăng thêm 1/360 lần áp suất ban đầu. Nhiệt độ ban đầu của khối khí bằng
		\begin{mcq}(4)
			\item $187^\circ\text{C}$.
			\item $360^\circ\text{C}$.
			\item $273^\circ\text{C}$.
			\item $87^\circ\text{C}$.
		\end{mcq}
	}
	
	\loigiai{
		\textbf{Đáp án: D.}
		
		Đẳng tích (1): $p_1$, $T_1=?$ sang (2): $p_2=(1+\dfrac{1}{360})p_1$, $T_2=T_1+\SI{1}{\celsius}$.
		
		Vì quá trình là đẳng tích nên ta có phương trình:
		
		$$\dfrac{p_1}{T_1} = \dfrac{p_2}{T_2} \Rightarrow T_1 = \SI{360}{K}$$
		
		Đổi $t_2 = T_2 - 273 = \SI{87}{\celsius}$.
	}
	


	\item \mkstar{3}
	
	\cauhoi{
		Van an toàn của một nồi áp suất sẽ mở khi áp suất nồi bằng 6 atm. Ở $200^\circ\text{C}$, hơi trong nồi có áp suất $\SI{1,5}{atm}$. Hỏi ở nhiệt độ nào thì van an toàn sẽ mở?
		\begin{mcq}(4)
			\item $\SI{1958}{K}$.
			\item $\SI{120}{K}$.
			\item $120^\circ\text{C}$.
			\item $1619^\circ\text{C}$.
		\end{mcq}
		
	}
	
	\loigiai{
		\textbf{Đáp án: D.}
		
		Van an toàn sẽ mở khi $p_2 = \SI{6}{atm}$ trở lên.
		
		Đẳng tích (1): $p_1=\SI{1.5}{atm}$, $T_1=\SI{473}{K}$ sang (2): $p_2=\SI{6}{atm}$, $T_2=?$.
		
		Vì quá trình là đẳng tích nên ta có phương trình:
		
		$$\dfrac{p_1}{T_1} = \dfrac{p_2}{T_2} \Rightarrow T_2 =\dfrac{T_1p_2}{p_1} = \SI{1892}{K}$$
		
		Đổi $t_2 = T_2 - 273 = \SI{1619}{\celsius}$.
	}

	\item \mkstar{2}
	
	\cauhoi{
		Khí trong một bình kín có nhiệt độ ban đầu là bao nhiêu biết khi áp suất tăng 2 lần thì nhiệt độ trong bình tăng thêm $\SI{313}{K}$? Biết thể tích không đổi.
		\begin{mcq}(4)
			\item $313^\circ\text{C}$.
			\item $40^\circ\text{C}$.
			\item $\SI{156,5}{K}$.
			\item $\SI{40}{K}$.
		\end{mcq}	
		
	}
	
	\loigiai{
		\textbf{Đáp án: B.}
		
		Đẳng tích (1): $p_1$, $T_1=?$ sang (2): $p_2=2p_1$, $T_2=T_1+\SI{313}{K}$.
		
		Vì quá trình là đẳng tích nên ta có phương trình:
		
		$$\dfrac{p_1}{T_1} = \dfrac{p_2}{T_2} \Rightarrow T_1 = \SI{313}{K}$$
		
		Đổi $t_1 = T_1 - 273 = \SI{40}{\celsius}$.
	}

	\item \mkstar{2}
	
	\cauhoi{
		Một bình kín có thể tích không đổi chứa khí lý tưởng ở áp suất $\SI{1,5e5}{Pa}$ và nhiệt độ $20^\circ\text{C}$. Tính áp suất trong bình khi nhiệt độ trong bình tăng lên tới $40^\circ\text{C}$.
		
	}
	
	\loigiai{

		
		Đẳng tích (1): $p_1=\SI{1.5e5}{Pa}$, $T_1=\SI{293}{K}$ sang (2): $p_2=?$, $T_2=\SI{313}{K}$.
		
		Vì quá trình là đẳng tích nên ta có phương trình:
		
		$$\dfrac{p_1}{T_1} = \dfrac{p_2}{T_2} \Rightarrow p_2 = \SI{1.6e5}{Pa}$$
		
	}
	

	\item \mkstar{2}
	
	\cauhoi{
		Một lốp xe được bơm căng không khí có áp suất $\SI{2}{atm}$ và nhiệt độ $20^\circ\text{C}$. Lốp xe chịu được áp suất lớn nhất là $\SI{2,4}{atm}$. Hỏi khi nhiệt độ bên trong lốp xe tăng lên đến $42^\circ\text{C}$ thì lốp xe có bị nổ hay không?
	}
	
	\loigiai{		
		Đẳng tích (1): $p_1=\SI{2}{atm}$, $T_1=\SI{293}{K}$ sang (2): $p_2=?$, $T_2=\SI{315}{K}$.
		
		Vì quá trình là đẳng tích nên ta có phương trình:
		
		$$\dfrac{p_1}{T_1} = \dfrac{p_2}{T_2} \Rightarrow p_2 = \SI{2.15}{atm}$$
		
		Vì $\SI{2.15}{atm} < \SI{2.4}{atm}$ nên lốp xe không bị nổ.
	}

	\item \mkstar{2}
	
	\cauhoi{
		Nung nóng bình thủy tinh có thể tích không đổi chứa không khí tới nhiệt độ $200^\circ\text{C}$. Biết ở thời điểm ban đầu khí trong bình ở điều kiện tiêu chuẩn, tính áp suất khí trong bình sau khi nung nóng.
	}
	
	\loigiai{		
		Đẳng tích (1): điều kiện tiêu chuẩn $p_1=\SI{1}{atm}$, $T_1=\SI{273}{K}$ sang (2): $p_2=?$, $T_2=\SI{473}{K}$.
		
		Vì quá trình là đẳng tích nên ta có phương trình:
		
		$$\dfrac{p_1}{T_1} = \dfrac{p_2}{T_2} \Rightarrow p_2 = \SI{1.73}{atm}$$
	}

	\item \mkstar{2}
	
	\cauhoi{
		Một bình kín thể tích không đổi chứa khí lý tưởng ở nhiệt độ $27^\circ\text{C}$. Hỏi nhiệt độ trong bình tăng thêm một lượng là bao nhiêu, biết áp suất ban đầu và sau khi nhiệt độ thay đổi lần lượt là $\SI{1}{atm}$ và $\SI{2,5}{atm}$?
	}
	
	\loigiai{		
		Đẳng tích (1): $p_1=\SI{1}{atm}$, $T_1=\SI{300}{K}$ sang (2): $p_2=\SI{2.5}{atm}$, $T_2=?$.
		
		Vì quá trình là đẳng tích nên ta có phương trình:
		
		$$\dfrac{p_1}{T_1} = \dfrac{p_2}{T_2} \Rightarrow T_2 = \SI{750}{K}$$
		
		Độ tăng nhiệt độ: $\Delta T = \Delta t  =\SI{450}{K}=\SI{450}{\celsius}$.
	}
	
\end{enumerate}

\section{Phương trình trạng thái của khí lí tưởng}
\begin{enumerate}[label=\bfseries Câu \arabic*:]
	
	\item \mkstar{2}
	
	\cauhoi{
	Một quả bóng có thể tích 2 lít, chứa khí ở $27^\circ\text{C}$ có áp suất $\SI{1}{atm}$. Người ta nung nóng quả bóng đến nhiệt độ $57^\circ\text{C}$ đồng thời giảm thể tích còn  1 lít. Áp suất lúc sau là bao nhiêu?
	\begin{mcq}(4)
		\item $\SI{1}{atm}$.
		\item $\SI{0,47}{atm}$.
		\item $\SI{2,2}{atm}$.
		\item $\SI{0,94}{atm}$.
	\end{mcq}
	}
	
	\loigiai{
		\textbf{Đáp án: C.}
		
		Biến đổi trạng thái (1): $p_1=\SI{1}{atm}$, $V_1=\SI{2}{l}$, $T_1=\SI{300}{K}$ sang (2): $p_2=?$, $V_2=\SI{1}{l}$, $T_2=\SI{330}{K}$.
		
		Áp dụng phương trình trạng thái khí lí tưởng:
		
		$$\dfrac{p_1V_1}{T_1} = \dfrac{p_2V_2}{T_2} \Rightarrow p_2 =\dfrac{p_1V_1T_2}{T_1V_2} = \SI{2.2}{atm}$$
	}

\item \mkstar{2}

\cauhoi{
	Một lượng khí $\text{H}_2$ đựng trong bình có $V_1=\SI{2}{l}$ ở áp suất $\SI{1,5}{atm}$, nhiệt độ $t_1=27^\circ\text{C}$. Đun nóng khí đến $t_2=127^\circ\text{C}$, do bình hở nên một nửa lượng khí thoát ra ngoài. Tính áp suất khí trong bình.
\begin{mcq}(4)
	\item $\SI{3}{atm}$.
	\item $\SI{7,05}{atm}$.
	\item $\SI{4}{atm}$.
	\item $\SI{2,25}{atm}$.
\end{mcq}
}

\loigiai{
	\textbf{Đáp án: ABC.}
	
	Biến đổi trạng thái (1): $p_1=\SI{1.5}{atm}$, $V_1=\SI{2}{l}$, $T_1=\SI{300}{K}$ sang (2): $p_2=?$, $V_2=\SI{1}{l}$, $T_2=\SI{400}{K}$.
	
	Áp dụng phương trình trạng thái khí lí tưởng:
	
	$$\dfrac{p_1V_1}{T_1} = \dfrac{p_2V_2}{T_2} \Rightarrow p_2 =\dfrac{p_1V_1T_2}{T_1V_2} = \SI{4}{atm}$$
}

\item \mkstar{2}

\cauhoi{
	Ở $27^\circ\text{C}$ thể tích của một lượng khí là 6 lít. Thể tích của lượng khí đó ở nhiệt độ $227^\circ\text{C}$ khi áp suất không đổi là bao nhiêu?
	\begin{mcq}(4)
		\item $\SI{3}{l}$.
		\item $\SI{20}{l}$.
		\item $\SI{28,2}{l}$.
		\item $\SI{10}{l}$.
	\end{mcq}
}

\loigiai{
	\textbf{Đáp án: D.}
	
	Đẳng áp (1): $V_1=\SI{6}{l}$, $T_1=\SI{300}{K}$ sang (2): $V_2=?$, $T_2=\SI{500}{K}$.
	
	Vì quá trình là đẳng áp nên ta có phương trình:
	
	$$\dfrac{V_1}{T_1} = \dfrac{V_2}{T_2} \Rightarrow V_2 =\dfrac{V_1T_2}{T_1} = \SI{10}{l}$$
}

\item \mkstar{2}

\cauhoi{
	Một lượng khí đựng trong xilanh có pittông chuyển động được. Các thông số của lượng khí: $\SI{1,5}{atm}$; $\SI{13,5}{l}$; $\SI{300}{K}$. Khi pit tông bị nén, áp suất tăng lên $\SI{3,7}{atm}$, thể tích giảm còn $\SI{10}{l}$. Xác định nhiệt độ khi nén.
	\begin{mcq}(4)
		\item $548,1^\circ\text{C}$.
		\item $275,1^\circ\text{C}$.
		\item $273^\circ\text{C}$.
		\item $450^\circ\text{C}$.
	\end{mcq}
}

\loigiai{
	\textbf{Đáp án: B.}
	
	Biến đổi trạng thái (1): $p_1=\SI{1.5}{atm}$, $V_1=\SI{13.5}{l}$, $T_1=\SI{300}{K}$ sang (2): $p_2=\SI{3.7}{atm}$, $V_2=\SI{10}{l}$, $T_2=?$.
	
	Áp dụng phương trình trạng thái khí lí tưởng:
	
	$$\dfrac{p_1V_1}{T_1} = \dfrac{p_2V_2}{T_2} \Rightarrow T_2 =\dfrac{p_2V_2T_1}{p_1V_1} = \SI{548.1}{K}$$
	
	Đổi $t_2=T_2-273 = \SI{275.1}{\celsius}$
}

\item \mkstar{2}

\cauhoi{
	Trong xilanh của một động cơ đốt trong có $\SI{2}{\deci\meter^3}$ hỗn hợp khí dưới áp suất $\SI{1}{atm}$ và nhiệt độ $47^\circ\text{C}$. Pit tông nén xuống làm cho thể tích của hỗn hợp khí chỉ còn $\SI{0,2}{\deci\meter^3}$ và áp suất tăng lên $\SI{15}{atm}$. Tính nhiệt độ của hỗn hợp khí nén.
	\begin{mcq}(4)
		\item $480^\circ\text{C}$.
		\item $470^\circ\text{C}$.
		\item $\SI{480}{K}$.
		\item $\SI{470}{K}$.
	\end{mcq}
}

\loigiai{
	\textbf{Đáp án: C.}
	
	Biến đổi trạng thái (1): $p_1=\SI{1}{atm}$, $V_1=\SI{2}{dm^3}$, $T_1=\SI{320}{K}$ sang (2): $p_2=\SI{15}{atm}$, $V_2=\SI{0.2}{dm^3}$, $T_2=?$.
	
	Áp dụng phương trình trạng thái khí lí tưởng:
	
	$$\dfrac{p_1V_1}{T_1} = \dfrac{p_2V_2}{T_2} \Rightarrow T_2 =\dfrac{p_2V_2T_1}{p_1V_1} = \SI{480}{K}$$
}
		\item \mkstar{2}
	
	\cauhoi{
		Nén 10 lít khí ở nhiệt độ $27^\circ\text{C}$ để thể tích của nó giảm chỉ còn 4 lít, quá trình nén nhanh nên nhiệt độ tăng đến $260^\circ\text{C}$. Áp suất khí đã tăng bao nhiêu lần?
	}
	
	\loigiai{
		Biến đổi trạng thái (1): $p_1$, $V_1=\SI{10}{l}$, $T_1=\SI{300}{K}$ sang (2): $p_2=?$, $V_2=\SI{4}{l}$, $T_2=\SI{533}{K}$.
		
		Áp dụng phương trình trạng thái khí lí tưởng:
		
		$$\dfrac{p_1V_1}{T_1} = \dfrac{p_2V_2}{T_2} \Rightarrow p_2 = p_1 \dfrac{V_1T_2}{V_2T_1} =4,44 p_1$$
		
		Vậy áp suất tăng 4,44 lần.
		
	}
	\item \mkstar{2}
	
	\cauhoi{
		Trong một động cơ điezen, khối khí có nhiệt độ ban đầu là $32^\circ\text{C}$ được nén để thể tích giảm bằng 1/16 thể tích ban đầu và áp suất tăng bằng 48,5 lần áp suất ban đầu. Tính nhiệt độ khối khí sau khi nén.
	}
	
	\loigiai{
		Biến đổi trạng thái (1): $p_1$, $V_1$, $T_1=\SI{305}{K}$ sang (2): $p_2=48,5p_1$, $V_2=\dfrac{V_1}{16}$, $T_2=?$.
		
		Áp dụng phương trình trạng thái khí lí tưởng:
		
		$$\dfrac{p_1V_1}{T_1} = \dfrac{p_2V_2}{T_2} \Rightarrow T_2 =\dfrac{p_2V_2T_1}{p_1V_1} = \SI{925}{K}$$
		
		Đổi $t_2=T_2 - 273 = \SI{652}{\celsius}$.
		
	}
	
	\item \mkstar{3}
	
	\cauhoi{
	Một bóng thám không được chế tạo để có thể tăng bán kính lên tới $\SI{10}{\meter}$ khi bay ở tầng khí quyển có áp suất $\SI{0,03}{atm}$ và nhiệt độ $\SI{200}{K}$. Hỏi bán kính của bóng khi bơm, biết bóng được bơm khí ở áp suất $\SI{1}{atm}$ và nhiệt độ $\SI{300}{K}$ ?
	}
	
	\loigiai{		
		Thể tích bóng lúc sau: $V_2 = \dfrac{4}{3} \pi R_2^3 = \SI{4188.8}{m^3}$.
		
		Biến đổi trạng thái (1): $p_1=\SI{1}{atm}$, $V_1=?$, $T_1=\SI{300}{K}$ sang (2): $p_2=\SI{0.03}{atm}$, $V_2=\SI{4188.8}{m^3}$, $T_2=\SI{200}{K}$.
		
		Áp dụng phương trình trạng thái khí lí tưởng:
		
		$$\dfrac{p_1V_1}{T_1} = \dfrac{p_2V_2}{T_2} \Rightarrow V_1 =\dfrac{p_2V_2T_1}{p_1T_2} = \SI{188.5}{m^3}$$
		
		Bán kính của bóng khi bơm:
		$$V_1=\dfrac{4}{3}\pi R_1^3 \Rightarrow R_1 = \SI{3.557}{m}$$
	}
	
	
	\item \mkstar{3}
	
	\cauhoi{
	Một bình kín chứa một mol khí Nitơ ở áp suất $\SI{e5}{\newton/\meter^2}$, nhiệt độ $27^\circ\text{C}$. Thể tích bình xấp xỉ bao nhiêu?
	}
	
	\loigiai{
		Áp dụng phương trình Cla-pê-rôn - Men-đê-lê-ep:
		$$pV = \dfrac{m}{M} RT \Rightarrow V = \dfrac{\dfrac{m}{M}RT}{p}=\dfrac{\SI{1}{mol} \cdot 8,34 \cdot \SI{300}{K}}{\SI{e5}{Pa}} = \SI{25e-3}{m^3}$$
	}
	

\end{enumerate}
\whiteBGstarEnd