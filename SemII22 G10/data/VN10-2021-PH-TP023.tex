\whiteBGstarBegin
\setcounter{section}{0}
\section{Lý thuyết: Động lượng của một vật}
\begin{enumerate}[label=\bfseries Câu \arabic*:]
	
\item \mkstar{1} [4]
	
	\cauhoi
	{ Phát biểu nào sau đây \textbf{sai}?
		\begin{mcq}
			\item Động lượng là một đại lượng vectơ.
			\item Xung lượng của lực là một đại lượng vectơ.
			\item Động lượng tỉ lệ với khối lượng của vật.
			\item Động lượng của vật trong chuyển động tròn đều không đổi.
		\end{mcq}
	}
	\loigiai
	{	\textbf{Đáp án: D.}
		
		Vectơ vận tốc trong chuyển động tròn đều thay đổi theo thời gian nên động lượng của vật trong chuyển động tròn đều thay đổi theo thời gian.
	}

\item \mkstar{1} [4]
	
	\cauhoi
	{Động lượng là đại lượng vectơ
		\begin{mcq}
			\item cùng phương, cùng chiều với vectơ vận tốc. 
			\item cùng phương, ngược chiều với vectơ vận tốc.
			\item có phương vuông góc với vectơ vận tốc.
			\item có phương hợp với vectơ vận tốc một góc $\alpha$ bất kì.
		\end{mcq}
		
	}
	\loigiai
	{	\textbf{Đáp án: A.}
		
		Động lượng là đại lượng vectơ cùng phương, cùng chiều với vectơ vận tốc.
	}
	
\item \mkstar{1} [7]
	
	\cauhoi{Động lượng của vật là gì? Viết công thức tính động lượng của vật. Trong hệ SI, đơn vị của động lượng là gì?
	}
	\loigiai
	{		
		Động lượng của vật là đại lượng vectơ bằng tích của khối lượng với vận tốc của vật.
		
		Công thức: $\vec p = m \vec v$.
		
		Đơn vị: $\SI{}{kg.m/s}$.
	}

\item \mkstar{2} [17]
	
	\cauhoi{Một ô tô có khối lượng $\SI{1000}{kg}$, chạy với vận tốc $\SI{54}{km/h}$. Tính động lượng của ô tô.
	}
	\loigiai
	{	Động lượng của ô tô:
		$p = mv = \SI{15000}{kg.m/s}$.
	}

\item \mkstar{2} [23]
	
	\cauhoi{
		Một quả bóng khối lượng $\SI{500}{g}$, chuyển động theo phương ngang với tốc độ $\SI{10}{m/s}$. Tính động lượng của quả bóng.
	}
	\loigiai
	{
		
	Động lượng của quả bóng: $p=mv =\SI{5}{kg.m/s} $.
	}

\item \mkstar{2} [28]
	
	\cauhoi{Một quả bóng nặng $\SI{1}{kg}$ đang đứng yên thì cầu thủ chạy đến sút quả bóng thật mạnh. Quả bóng bay đi với vận tốc $\SI{25}{m/s}$. Tính động lượng quả bóng.
	}
	\loigiai
	{Động lượng của quả bóng: $p=mv=\SI{25}{kg.m/s}$.
	}
\end{enumerate}
\section{Lý thuyết: Tổng động lượng của hệ vật}
\begin{enumerate}[label=\bfseries Câu \arabic*:]

\item \mkstar{2} [6]
	
	\cauhoi{
		Hệ gồm hai vật có khối lượng lần lượt là $m_1=\SI{3}{kg}$, $m_2=\SI{6}{kg}$, chuyển động với vận tốc có độ lớn lần lượt là $v_1=\SI{2}{m/s}$, $v_2=\SI{1}{m/s}$. Tính độ lớn tổng động lượng của hệ trong trường hợp hai vật chuyển động cùng phương ngược chiều.
	}
	
	\loigiai{
		
		Động lượng của hệ: $\vec p = \vec p_1 + \vec p_2$.
		
		Vì hai vật chuyển động cùng phương ngược chiều nên $p=|p_1-p_2| = |m_1v_1 - m_2v_2| = 0$.
	}
	
\end{enumerate}

\section{Lý thuyết: Xung lượng. Độ biến thiên động lượng}
\begin{enumerate}[label=\bfseries Câu \arabic*:]
	
\item \mkstar{1} [4]
	
	\cauhoi{
		Biểu thức khác của định luật II Newton là (liên hệ giữa xung lượng của lực và độ biến thiên động lượng):
		\begin{mcq}(4)
			\item $\vec p = m\vec v$. 
			\item $\Delta \vec v = \vec F  \Delta t$.
			\item $\Delta \vec p = \vec F  \Delta t$.
			\item $\vec F = m \vec a$.
		\end{mcq}
	}
	
	\loigiai{\textbf{Đáp án: C.}
		
		Biểu thức khác của định luật II Newton là (liên hệ giữa xung lượng của lực và độ biến thiên động lượng) $\Delta \vec p = \vec F  \Delta t$.
	}
	
\item \mkstar{3} [23]
	
	\cauhoi{Một quả bóng khối lượng $\SI{500}{g}$, chuyển động theo phương ngang với tốc độ $\SI{10}{m/s}$. Sau khi đập vuông góc vào một bức tường, quả bóng bật trở lại theo phương tới với tốc độ như cũ. Tính độ lớn của độ biến thiên động lượng của quả bóng.
		
	}
	\loigiai{
		
		Độ lớn của độ biến thiên động lượng của quả bóng:
		$|\Delta \vec p | = |\vec p' - \vec p|$.
		
		Vì $\vec p'$ và $\vec p$ cùng phương, ngược chiều nên $|\Delta p| = |-2p| = \SI{10}{kg.m/s}$.
	}
\end{enumerate}

\section{Lý thuyết: Định luật bảo toàn động lượng}
\begin{enumerate}[label=\bfseries Câu \arabic*:]
	
\item \mkstar{2} [4]
	
	\cauhoi{
		Một vật khối lượng $m$ đang chuyển động theo phương ngang với vận tốc $v$ thì va chạm vào vật khối lượng $2m$ đang đứng yên. Sau va chạm, hai vật dính vào nhau và chuyển động với cùng vận tốc. Bỏ qua ma sát, vận tốc của hệ hai vật sau va chạm là
		\begin{mcq}(4)
			\item $\dfrac{v}{3}$. 
			\item $v$.
			\item $3v$.
			\item $\dfrac{v}{2}$.
		\end{mcq}
	}
	
	\loigiai{\textbf{Đáp án: A.}
		
		Áp dụng định luật bảo toàn động lượng trong va chạm mềm (hai vật dính vào nhau sau va chạm):
		$m_1v_1 + m_2 v_2 = (m_1 + m_2) v \Rightarrow m v + 0 = 3m v' \Rightarrow v'=\dfrac{v}{3}$.
	}
	
\item \mkstar{2} [24]
	
	\cauhoi{Một vật khối lượng $m_1=\SI{1}{kg}$ chuyển động thẳng đều trên mặt phẳng nằm ngang với vận tốc $\SI{12}{m/s}$ và va chạm với vật có khối lượng $m_2=\SI{2}{kg}$ đang đứng yên. Sau va chạm, hai vật dính chặt với nhau. Bỏ qua mọi ma sát. Vận tốc của hai vật sau va chạm là
		\begin{mcq}(4)
			\item $\SI{4}{m/s}$. 
			\item $\SI{6}{m/s}$.
			\item $\SI{12}{m/s}$.
			\item $\SI{24}{m/s}$.
		\end{mcq}
	}
	\loigiai{\textbf{Đáp án: A.}
		
		Áp dụng định luật bảo toàn động lượng trong va chạm mềm (hai vật dính vào nhau sau va chạm):
		$m_1v_1 + m_2 v_2 = (m_1 + m_2) v \Rightarrow \SI{1}{kg}\cdot\SI{12}{m/s} + 0 = \SI{3}{kg}\cdot v \Rightarrow v=\SI{4}{m/s}$.
	}
	
\item \mkstar{1} [25]
	
	\cauhoi{Phát biểu và viết biểu thức của định luật bảo toàn động lượng.
		
	}
	\loigiai{
		Động lượng của một hệ cô lập là một đại lượng được bảo toàn.
		
		Biểu thức: $\vec p = \vec p_1 + \vec p_2 + \ldots + \vec p_n = \text{const}$
	}
	
\item \mkstar{2} [15]
	
	\cauhoi{
			Thế nào là chuyển động bằng phản lực? Cho 1 ví dụ.
	}
	\loigiai{
		
		Chuyển động bằng phản lực là chuyển động của một vật tự tạo ra phản lực bằng cách phóng về một hướng một phần khối lượng của nó, để phần kia chuyển động theo hướng ngược lại.
		
		Ví dụ: chuyển động của tên lửa, máy bay phản lực, cung tên, $\ldots$.
	}
	
\item \mkstar{2} [15]
	
	\cauhoi{Thế nào là va chạm mềm? Cho 1 ví dụ.
		
	}
	\loigiai{
		
		Va chạm mềm là va chạm mà sau khi va chạm hai vật gắn chặt vào nhau và chuyển động với cùng vận tốc.
		
		Ví dụ: viên đạn găm vào bao cát, hai xe sau va chạm móc vào nhau, $\ldots$.
	}
	
\item \mkstar{2} [19]
	
	\cauhoi{Một vật khối lượng $\SI{0.8}{kg}$ chuyển động trên mặt phẳng ngang với vận tốc $\SI{12}{m/s}$, đến va chạm với một vật khác có khối lượng $\SI{0.2}{kg}$ đang đứng yên trên mặt phẳng ngang ấy. Sau va chạm hai vật nhập lại làm một và chuyển động với cùng vận tốc. Tính vận tốc của hai vật sau va chạm.
		
	}
	\loigiai{
		
		Áp dụng định luật bảo toàn động lượng trong va chạm mềm (hai vật dính vào nhau sau va chạm):
		$m_1v_1 + m_2 v_2 = (m_1 + m_2) v \Rightarrow \SI{0.8}{kg}\cdot\SI{12}{m/s} + 0 = \SI{1}{kg} \cdot v \Rightarrow v=\SI{9.6}{m/s}$.
	}
	
\item \mkstar{2} [19]
	
	\cauhoi{Một vật khối lượng $\SI{0.6}{kg}$ chuyển động trên mặt phẳng ngang với vận tốc $\SI{12}{m/s}$, đến va chạm với một vật khác có khối lượng $\SI{0.4}{kg}$ đang đứng yên trên mặt phẳng ngang ấy. Sau va chạm hai vật nhập lại làm một và chuyển động với cùng vận tốc. Tính vận tốc của hai vật sau va chạm.
		
	}
	\loigiai{
		
		Áp dụng định luật bảo toàn động lượng trong va chạm mềm (hai vật dính vào nhau sau va chạm):
		$m_1v_1 + m_2 v_2 = (m_1 + m_2) v \Rightarrow \SI{0.6}{kg}\cdot\SI{12}{m/s} + 0 = \SI{1}{kg} \cdot v \Rightarrow v=\SI{7.2}{m/s}$.
	}
	
	
\item \mkstar{2} [25]
	
		\cauhoi{Một vật khối lượng $m_1=\SI{400}{g}$ chuyển động trên mặt phẳng ngang với vận tốc $\SI{18}{km/h}$, đến va chạm với một vật khác có khối lượng $\SI{100}{g}$ đang đứng yên trên mặt phẳng ngang ấy. Sau va chạm hai vật nhập lại làm một và chuyển động với cùng vận tốc. Tính vận tốc của hai vật sau va chạm.
		
	}
	\loigiai{
		
		Áp dụng định luật bảo toàn động lượng trong va chạm mềm (hai vật dính vào nhau sau va chạm):
		$m_1v_1 + m_2 v_2 = (m_1 + m_2) v \Rightarrow \SI{0.4}{kg}\cdot\SI{5}{m/s} + 0 = \SI{0.5}{kg} \cdot v \Rightarrow v=\SI{4}{m/s}$.
	}
\end{enumerate}
\whiteBGstarEnd