\whiteBGstarBegin
\setcounter{section}{0}
\section{Lý thuyết: Thế năng trọng trường}
\begin{enumerate}[label=\bfseries Câu \arabic*:]
	
	\item \mkstar{1} [24]
	
	\cauhoi{
		Dạng năng lượng tương tác giữa Trái Đất và vật là
		\begin{mcq}(2)
			\item động năng.
			\item thế năng trọng trường.
			\item thế năng đàn hồi.
			\item cơ năng.
		\end{mcq}
	}
	
	\loigiai{
		\textbf{Đáp án: B.}
		
		Dạng năng lượng tương tác giữa Trái Đất và vật là thế năng trọng trường.
	}
	
	\item \mkstar{1} [32]
	
	\cauhoi{
		Thế năng trọng trường của một vật
		\begin{mcq}
			\item luôn dương vì độ cao của vật luôn dương. 
			\item có thể âm, dương hoặc bằng không.
			\item không thay đổi nếu vật chuyển động thẳng đều.
			\item không phụ thuộc vào vị trí của vật.
		\end{mcq}
		
	}
	
	\loigiai{
		\textbf{Đáp án: B.}
		
		Thế năng trọng trường của một vật có thể âm, dương hoặc bằng không.
	}

	
	\item \mkstar{1} [7]
	
	\cauhoi{
		Viết biểu thức thế năng trọng trường của vật (có chú thích các đại lượng trong công thức). Chọn gốc thế năng tại mặt đất, khi thả một vật rơi tự do thì thế năng trọng trường của vật tăng hay giảm? (không cần giải thích)
	}
	
	\loigiai{		
		Biểu thức thế năng trọng trường:
		$$W_\text t = mgh,$$
		với:
		\begin{itemize}
			\item $m$ là khối lượng của vật (đơn vị kg);
			\item $g$ là gia tốc trọng trường nơi đặt vật (đơn vị $\SI{}{m/s^2}$);
			\item $h$ là vị trí của vật so với gốc thế năng (đơn vị $\SI{}{m}$).
		\end{itemize}
	
		Chọn gốc thế năng tại mặt đất, khi thả một vật rơi tự do thì thế năng trọng trường của vật giảm.
	}
	
	
	\item \mkstar{2} [13]
	
	\cauhoi{
		Một vật có khối lượng $\SI{2}{kg}$ được thả rơi từ độ cao $\SI{4.5}{m}$ xuống mặt đất, tại nơi có gia tốc trọng trường là $\SI{10}{m/s^2}$. Xác định thế năng của vật.
	}
	
	\loigiai{
		Chọn gốc thế năng tại mặt đất. Thế năng của vật:
		$$W_\text t = mgz = \SI{90}{J}.$$
	}
	
	\item \mkstar{2} [15]
	
	\cauhoi{
		Dùng lực để nâng vật có khối lượng $\SI{180}{g}$ thẳng đứng lên cao một đoạn $\SI{50}{cm}$. Tìm công của trọng lực trong quá trình trên. Lấy $g=\SI{10}{m/s^2}$.
	}
	
	\loigiai{
		Công của trọng lực:
		$$A_P = mgh\cos(180^\circ) = -mgh = \SI{-0.9}{J}.$$
		
	}
\end{enumerate}
\section{Lý thuyết: Thế năng đàn hồi}
\begin{enumerate}[label=\bfseries Câu \arabic*:]
		
	\item \mkstar{1} [29]
	
	\cauhoi{
		Một con lắc lò xo có độ cứng $k$, lò xo bị nén một đoạn $\Delta l$, biểu thức tính thế năng đàn hồi của lò xo là
		\begin{mcq}(4)
			\item $W_\text t =k(\Delta l)^2$. 
			\item $W_\text t = k |\Delta l|$.
			\item $W_\text t = \dfrac{1}{2}k (\Delta l)^2$.
			\item $W_\text t = \dfrac{1}{2} k |\Delta l|$.
		\end{mcq}
		
	}
	
	\loigiai{
		\textbf{Đáp án: C.}
		
		Một con lắc lò xo có độ cứng $k$, lò xo bị nén một đoạn $\Delta l$, biểu thức tính thế năng đàn hồi của lò xo là $W_\text t = \dfrac{1}{2}k (\Delta l)^2$.
	}


	\item \mkstar{2} [8]
	
	\cauhoi{
		Một lò xo nằm ngang ở trạng thái ban đầu không biến dạng có độ cứng là $\SI{100}{N/m}$. Tính thế năng đàn hồi của lò xo khi nó bị giãn ra $\SI{4}{cm}$.
		
	}
	
	\loigiai{
		Thế năng đàn hồi của lò xo:
		$$W_\text t = \dfrac{1}{2} k (\Delta l)^2 = \SI{0.08}{J}.$$
	}

	
\end{enumerate}	
\whiteBGstarEnd