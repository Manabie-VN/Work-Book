\whiteBGstarBegin
\setcounter{section}{0}

\begin{enumerate}[label=\bfseries Câu \arabic*:]
	\item \mkstar{1}
	
	\cauhoi{
		Chất rắn nào dưới đây thuộc loại chất rắn kết tinh?
		\begin{mcq}(2)
			\item Kim loại.
			\item Thủy tinh.
			\item Nhựa đường.
			\item Cao su.
		\end{mcq}
	}
	
	\loigiai{
		\textbf{Đáp án: A.}
		
		Kim loại thuộc chất rắn đa tinh thể, một trong hai loại của chất rắn kết tinh. Trong khi đó, thủy tinh, nhựa đường và cao su là chất rắn vô định hình.
	}

	\item \mkstar{1}
	
	\cauhoi{
		Chất rắn kết tinh không có đặc tính nào sau đây?
		\begin{mcq}
			\item Chất rắn đơn tinh thể có tính dị hướng.
			\item Chất rắn đa tinh thể có tính đẳng hướng.
			\item Ở một áp suất nhất định, chất rắn kết tinh có nhiệt độ nóng chảy xác định, không đổi.
			\item Cấu trúc tinh thể được tạo thành từ cùng một loại hạt thì có tính chất vật lí giống nhau.
		\end{mcq}
	}
	
	\loigiai{
		\textbf{Đáp án: D.}
		
		Đặc tính của chất rắn kết tinh là có nhiệt độ nóng chảy xác định, các chất rắn được tạo thành từ cùng một loại hạt nhưng có cấu trúc tinh thể khác nhau thì tính chất vật lí cũng khác nhau.
	}

	\item \mkstar{1}
	
	\cauhoi{
		Người ta phân loại các chất rắn theo cách nào dưới đây là đúng?
		\begin{mcq}
			\item Chất rắn đơn tinh thể và chất rắn đa tinh thể.
			\item Chất rắn đa tinh thể và chất rắn vô định hình.
			\item Chất rắn vô định hình và chất rắn kết tinh.
			\item Chất rắn kết tinh và chất rắn đơn tinh thể.
		\end{mcq}
	}
	
	\loigiai{
		\textbf{Đáp án: C.}
		
		Chất rắn được chia làm hai loại: kết tinh và vô định hình. Trong đó, chất rắn kết tinh được chia làm hai loại nhỏ hơn: đơn tinh thể và đa tinh thể.
	}

	\item \mkstar{1}
	
	\cauhoi{
		Đặc điểm của chất rắn vô định hình là
		\begin{mcq}
			\item đẳng hướng và không có nhiệt độ nóng chảy xác định.
			\item đẳng hướng và có nhiệt độ nóng chảy xác định.
			\item dị hướng và không có nhiệt độ nóng chảy xác định.
			\item dị hướng và có nhiệt độ nóng chảy xác định.
		\end{mcq}
	}
	
	\loigiai{
		\textbf{Đáp án: A.}
		
		Đặc điểm của chất rắn vô định hình là đẳng hướng và không có dạng hình học xác định, không có nhiệt độ nóng chảy xác định.
	}

	\item \mkstar{2} [4]
	
	\cauhoi{
		Vật nào sau đây \textbf{không} có cấu trúc tinh thể?
		\begin{mcq}(2)
			\item Cốc thủy tinh.
			\item Hạt muối ăn.
			\item Viên kim cương.
			\item Miếng thạch anh.
		\end{mcq}
	}
	
	\loigiai{
		\textbf{Đáp án: A.}
		
		Hạt muối ăn, miếng thạch anh , viên kim cương có cấu trúc đơn tinh thể.
		
		Thủy tinh thuộc chất rắn vô định hình, do đó không có cấu trúc tinh thể.
	}
	

	
	\item \mkstar{1} [14]
	
	\cauhoi{
		Hãy nêu các đặc điểm của chất rắn kết tinh, chất rắn vô định hình. Cho một vài ví dụ mỗi loại.
	}
	
	\loigiai{
		\begin{center}
			\begin{tabular}{|m{8em}|m{14em}|m{14em}|}
				\hline
				& \thead{Chất rắn kết tinh} & \thead{Chất rắn vô định hình} \\
				\hline
				\textbf{Cấu trúc} & Có cấu trúc tinh thể. & Không có cấu trúc tinh thể. \\
				\hline
				\textbf{Dạng hình học} & Có dạng hình học xác định. & Không có dạng hình học xác định. \\
				\hline
				\textbf{Nhiệt độ nóng chảy} & Có nhiệt độ nóng chảy xác định. & Không có nhiệt độ nóng chảy xác định. \\
				\hline
				\textbf{Đặc tính} & Chất rắn đơn tinh thể có tính dị hướng, chất rắn đa tinh thể có tính đẳng hướng. & Có tính đẳng hướng. \\
				\hline
				\textbf{Các ví dụ} & Kim loại, hợp kim, muối ăn, kim cương, \ldots & Thủy tinh, các loại nhựa, cao su, \ldots \\
				\hline
			\end{tabular}
		\end{center}
	}
	
\end{enumerate}
\whiteBGstarEnd