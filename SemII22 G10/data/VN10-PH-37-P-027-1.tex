%\setcounter{chapter}{1}
\chapter{Luyện tập: Quá trình đẳng nhiệt. Định luật Bôi-lơ - Ma-ri-ốt}
\begin{enumerate}
	\item %câu 1
	Người ta điều chế khí hidro và chứa vào một bình lớn dưới áp suất $\SI{1}{atm}$ ở nhiệt độ $20^\circ\text{C}$. Coi quá trình này là đẳng nhiệt. Tính thể tích khí phải lấy từ bình lớn ra để nạp vào bình nhỏ có thể tích $\SI{20}{l}$ ở áp suất $\SI{25}{atm}$.
	\begin{mcq}(4)
		\item $\SI{250}{l}$.
		\item $\SI{250}{l}$.
		\item $\SI{250}{l}$.
		\item $\SI{250}{l}$.
	\end{mcq}
	\item %câu 2
	Nén khí đẳng nhiệt từ thể tích $\SI{9}{l}$ đến thể tích $\SI{6}{l}$ thì thấy áp suất tăng lên một lượng $\Delta p=\SI{40}{\kilo Pa}$. Hỏi áp suất ban đầu của khí là
	\begin{mcq}(4)
		\item $\SI{80}{\kilo Pa}$.
		\item $\SI{80}{Pa}$.
		\item $\SI{40}{\kilo Pa}$.
		\item $\SI{40}{Pa}$.
	\end{mcq}
	\item %câu 3
	Dưới áp suất $\SI{e5}{Pa}$ một lượng khí có thể tích $\SI{10}{l}$. Tính thể tích của khí đó dưới áp suất $\SI{3e5}{Pa}$. Coi nhiệt độ không đổi trong suốt quá trình.
	\begin{mcq}(4)
		\item $\SI{10}{l}$.
		\item $\SI{3,31}{l}$.
		\item $\SI{51}{l}$.
		\item $\SI{301}{l}$.
	\end{mcq}
	\item %câu 4
	Một lượng khí có $V_1=\SI{3}{l}$, $p_1=\SI{3e5}{Pa}$. Hỏi khi nén $V_2=\dfrac{2}{3}V_1$ thì áp suất của nó là bao nhiêu? Coi nhiệt độ không đôi trong suốt quá trình.
	\begin{mcq}(4)
		\item $\SI{4,5e5}{Pa}$.
		\item $\SI{3e5}{Pa}$.
		\item $\SI{2,1e5}{Pa}$.
		\item $\SI{0,67e5}{Pa}$.
	\end{mcq}
	\item %câu 5
	Nếu áp suất của một lượng khí tăng thêm $\SI{2,1e5}{Pa}$ thì thể tích giảm $\SI{3}{l}$. Nếu áp suất tăng thêm $\SI{5e5}{Pa}$ thì thể tích giảm $\SI{5}{l}$. Tìm áp suất và thể tích ban đầu của khí, biết nhiệt độ khí không đổi.
	\begin{mcq}(4)
		\item $\SI{e5}{Pa}$, $\SI{10}{l}$.
		\item $\SI{2e5}{Pa}$, $\SI{10}{l}$.
		\item $\SI{4e5}{Pa}$, $\SI{3}{l}$.
		\item $\SI{3e5}{Pa}$, $\SI{3}{l}$.
	\end{mcq}
	\item %câu 6
	Mỗi lần bơm đưa được $V_0=\SI{80}{\centi\meter^3}$ không khí vào ruột xe. Sau khi bơm diện tích tiếp xúc của nó với mặt đường là $\SI{30}{\centi\meter^2}$, thể tích ruột xe sau khi bơm là $\SI{2000}{\centi\meter^3}$, áp suất khí quyển là $\SI{1}{atm}$, trọng lượng xe là $\SI{600}{\newton}$. Coi nhiệt độ không đổi trong quá trình bơm. Số lần phải bơm là
	\begin{mcq}(4)
		\item 100.
		\item 48.
		\item 240.
		\item 50.
	\end{mcq}
	\item %câu 7
	Một lượng khí xác định ở áp suất $\SI{3}{atm}$, có thể tích là 10 lít. Tính thể tích của khối khí khi nén đẳng nhiệt đến áp suất $\SI{6}{atm}$.
	\item %câu 8
	Một bọt khí khi nổi lên từ một đáy hồ có độ lớn gấp 1,2 lần khi đến mặt nước. Tính độ sâu của đáy hồ biết trọng lượng riêng của nước là: $d=\SI{e4}{N/\meter^3}$, áp suất khi quyển là $\SI{e5}{\newton/\meter^2}$.
	\item %câu 9
	Một bong bóng khí ở độ sâu $\SI{5}{\meter}$ có thể tích thay đổi như thế nào khi nổi lên mặt nước? Cho áp suất tại mặt nước là $\SI{e5}{Pa}$, khối lượng riêng của nước là $\SI{1000}{\kilogram/\meter^3}$, gia tốc trọng trường là $g=\SI{10}{\meter/\second^2}$.
	\item %câu 10
	Một quả bóng có dung tích $\SI{2,5}{l}$. Người ta bơm không khí ở áp suất khí quyển $\SI{e5}{\newton/\meter^2}$ vào bóng. Mỗi lần bơm được $\SI{125}{\centi\meter^3}$ không khí. Hỏi áp suất của không khí trong quả bóng sau 40 lần bơm? Coi quả bóng trước khi bơm không có không khí và trong thời gian bơm nhiệt độ của không khí không đổi.
\end{enumerate}
\textbf{ĐÁP ÁN}
\begin{longtable}[\textwidth]{|p{0.15\textwidth}|p{0.15\textwidth}|p{0.15\textwidth}|p{0.2\textwidth}|p{0.2\textwidth}|}
	% --- first head
	\hline%\hspace{2 pt}
	\multicolumn{1}{|c}{\textbf{Câu 1}} & \multicolumn{1}{|c|}{\textbf{Câu 2}} & \multicolumn{1}{c|}{\textbf{Câu 3}} &
	\multicolumn{1}{c|}{\textbf{Câu 4}} &
	\multicolumn{1}{c|}{\textbf{Câu 5}}  \\
	\hline
	&  & A. & A. & $\SI{1.6e5}{\pascal}$.	\\
	\hline
	
	\multicolumn{1}{|c|}{\textbf{Câu 6}} & \multicolumn{1}{c|}{\textbf{Câu 7}} & \multicolumn{1}{c|}{\textbf{Câu 8}} &
	\multicolumn{1}{c|}{\textbf{Câu 9}} &
	\multicolumn{1}{c|}{\textbf{Câu 10}} \\ 
	\hline
	D.  & $\SI{5}{\litre}$. & $\SI{2}{\meter}$. & Tăng 1,24 lần. & $\SI{2e5}{}$$\SI{}{\newton / \meter \squared}$. \\
	\hline
\end{longtable}			
