%\setcounter{chapter}{1}
\chapter{Luyện tập: Cấu tạo chất. Thuyết động học phân tử chất khí}
\begin{enumerate}
	\item %câu 1
	Trong các câu sau đây, câu nào sai?
	\begin{mcq}(1)
		\item Các chất được cấu tạo một cách gián đoạn.
		\item Các nguyên tử, phân tử đứng sát nhau và giữa chúng không có khoảng cách.
		\item Lực tương tác giữa các phân tử ở thể rắn lớn hơn lực tương tác giữa các phân tử ở thể lỏng và thể khí.
		\item Các nguyên tử, phân tử chất lỏng dao động xung quanh các vị trí cân bằng không cố định.
	\end{mcq}
	\item %câu 2
	Tính chất nào sau đây không phải là tính chất của các phân tử khí?
	\begin{mcq}(1)
		\item Có vận tốc trung bình phụ thuộc vào nhiệt độ.
		\item Gây áp suất lên thành bình.
		\item Chuyển động xung quanh vị trí cân bằng.
		\item Chuyển động nhiệt hỗn loạn.
	\end{mcq}
	\item %câu 3
	Chuyển động nào sau đây là chuyển động của riêng các phân tử ở thể lỏng?
	\begin{mcq}(1)
		\item Chuyển động hỗn loạn không ngừng.
		\item Dao động xung quanh các vị trí cân bằng cố định.
		\item Chuyển động hoàn toàn tự do.
		\item Dao động xung quanh các vị trí cân bằng không cố định.
	\end{mcq}
	\item %câu 4
	Câu nào sau đây nói về khí lí tưởng là không đúng?
	\begin{mcq}(1)
		\item Khí lí tưởng là khí mà thể tích của các phân tử có thể bỏ qua.
		\item Khí lí tưởng là khí mà khối lượng của các phân tử có thể bỏ qua.
		\item Khí lí tưởng là khí mà các phân tử chỉ tương tác khi va chạm.
		\item Khí lí tưởng là khí có thể gây áp suất lên thành bình chứa.
	\end{mcq}
	\item %câu 5
	Câu nào sau đây nói về lực tương tác phân tử là không đúng?
	\begin{mcq}(1)
		\item Lực tương tác phân tử đáng kể khi các phân tử ở rất gần nhau.
		\item Lực hút phân tử có thể lớn hơn lực đẩy phân tử.
		\item Lực hút phân tử không thể lớn hơn lực đẩy phân tử.
		\item Lực hút phân tữ có thể bằng lực đẩy phân tử.
	\end{mcq}
	\item %câu 6
	Tìm tỉ số khối lượng phân tử nước và nguyên tử Cacbon 12.
	\item %câu 7
	Tính số lượng phân tử $\text{H}_2\text{O}$ trong $\SI{1}{\gram}$ nước.
	\item %câu 8
	Hãy xác định
	\begin{enumerate}[label=\alph*)]
		\item Tỉ số khối lượng phân tử nước và nguyên tử cacbon $\text{C}_{12}$.
		\item Số phân tử $\text{H}_2\text{O}$ trong $\SI{2}{\gram}$ nước.
	\end{enumerate}
	\item %câu 9
	Một bình kín chứa $N=3,01\cdot10^{23}$ phân tử khí Heli. Tính khối lượng khí Heli trong bình.
	\item %câu 10
	\begin{enumerate}[label=\alph*)]
		\item Tính số phân tử chứa trong $\SI{0,2}{\kilogram}$ nước.
		\item Tính số phân tử chứa trong $\SI{1}{\kilogram}$ không khí nếu như không khí có $22\%$ là oxi và $78\%$ là khí nitơ.
	\end{enumerate}			
\end{enumerate}

\textbf{ĐÁP ÁN:}

\begin{longtable}[\textwidth]{|m{0.15\textwidth}|m{0.2\textwidth}|m{0.2\textwidth}|m{0.15\textwidth}|m{0.2\textwidth}|}
	% --- first head
	\hline%\hspace{2 pt}
	\multicolumn{1}{|c}{\textbf{Câu 1}} &
	\multicolumn{1}{|c|}{\textbf{Câu 2}}& 
	\multicolumn{1}{c|}{\textbf{Câu 3}} &
	\multicolumn{1}{c|}{\textbf{Câu 4}} &
	\multicolumn{1}{c|}{\textbf{Câu 5}} \\
	\hline
	B. &
	C. &
	D. &
	A. &
	C.
	\\
	\hline
	\multicolumn{1}{|c}{\textbf{Câu 6}} &
	\multicolumn{1}{|c|}{\textbf{Câu 7}}& 
	\multicolumn{1}{c|}{\textbf{Câu 8}} &
	\multicolumn{1}{c|}{\textbf{Câu 9}} &
	\multicolumn{1}{c|}{\textbf{Câu 10}} \\
	\hline
	$\dfrac{3}{2}$.&
	$3,34\cdot 10^{22}$.&
	\begin{enumerate}[label=\alph*)]
		\item $\dfrac{3}{2}$.
		\item $6,69\cdot 10^{22}$.
	\end{enumerate} &
	$\SI{2}{\gram}$. &
	\begin{enumerate}[label=\alph*)]
		\item $6,69\cdot 10^{24}$.
		\item $2,1\cdot 10^{25}$.
	\end{enumerate} 
	\\
	\hline
\end{longtable}



