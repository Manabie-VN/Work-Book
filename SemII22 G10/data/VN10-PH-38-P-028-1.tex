%\setcounter{chapter}{1}
\chapter{Luyện tập: Quá trình đẳng tích. Định luật Sác-lơ}
\begin{enumerate}
	\item %câu 1
	Đun nóng một khối khí được đựng trong một bình kín làm cho nhiệt độ của nó tăng thêm $1^\circ\text{C}$ thì người ta thấy rằng áp suất của khối khí trong bình tăng thêm 1/360 lần áp suất ban đầu. Nhiệt độ ban đầu của khối khí bằng
	\begin{mcq}(4)
		\item $187^\circ\text{C}$.
		\item $360^\circ\text{C}$.
		\item $273^\circ\text{C}$.
		\item $87^\circ\text{C}$.
	\end{mcq}
	\item %câu 2
	Một bình được nạp khí ở $33^\circ\text{C}$ dưới áp suất $\SI{300}{Pa}$. Sau đó bình được chuyển đến một nơi có nhiệt độ $37^\circ\text{C}$. Tính độ tăng áp suất của khí trong bình.
	\begin{mcq}(4)
		\item $\SI{303,9}{Pa}$.
		\item $\SI{3,9}{Pa}$.
		\item $\SI{336,4}{Pa}$.
		\item $\SI{36,4e5}{Pa}$.
	\end{mcq}
	\item %câu 3
	Van an toàn của một nồi áp suất sẽ mở khi áp suất nồi bằng 9 atm. Ở $200^\circ\text{C}$, hơi trong nồi có áp suất $\SI{1,5}{atm}$. Hỏi ở nhiệt độ nào thì van an toàn sẽ mở?
	\begin{mcq}(4)
		\item $\SI{1958}{K}$.
		\item $\SI{120}{K}$.
		\item $120^\circ\text{C}$.
		\item $180^\circ\text{C}$.
	\end{mcq}
	\item %câu 4
	Khí trong bình kín có nhiệt độ là bao nhiêu biết khi áp suất tăng 2 lần thì nhiệt độ trong bình tăng thêm $\SI{313}{K}$, thể tích không đổi.
	\begin{mcq}(4)
		\item $313^\circ\text{C}$.
		\item $40^\circ\text{C}$.
		\item $\SI{156,5}{K}$.
		\item $\SI{40}{K}$.
	\end{mcq}	
	\item %câu 5
	Một bình kín có thể tích không đổi chứa khí lý tưởng ở áp suất $\SI{1,5e5}{Pa}$ và nhiệt độ $20^\circ\text{C}$. Tính áp suất trong bình khi nhiệt độ trong bình tăng lên tới $40^\circ\text{C}$.
	\item %câu 6
	Tính độ tăng áp suất của một bình kín có thể tích không đổi chứa khí ở nhiệt độ $33^\circ\text{C}$ sau đó nung nóng tới nhiệt độ $37^\circ\text{C}$. Cho áp suất ban đầu bên trong bình là $\SI{300}{kPa}$
	\item %câu 7
	Một lốp xe được bơm căng không khí có áp suất $\SI{2}{atm}$ và nhiệt độ $20^\circ\text{C}$. Lốp xe chịu được áp suất lớn nhất là $\SI{2,4}{atm}$. Hỏi khi nhiệt độ bên trong lốp xe tăng lên đến $42^\circ\text{C}$ thì lốp xe có bị nổ hay không?
	\item %câu 8
	Nung nóng bình thủy tinh có thể tích không đổi chứa không khí tới nhiệt độ $200^\circ\text{C}$. Biết ở thời điểm ban đầu khí trong bình ở điều kiện tiêu chuẩn, tính áp suất khí trong bình sau khi nung nóng.
	\item %câu 9
	Một bình kín thể tích không đổi chứa khí lý tưởng ở nhiệt độ $27^\circ\text{C}$. Hỏi nhiệt độ trong bình tăng thêm một lượng là bao nhiêu, biết áp suất ban đầu và sau khi nhiệt độ thay đổi lần lượt là $\SI{1}{atm}$ và $\SI{2,5}{atm}$?
	\item %câu 10
	Một khối khí lý tưởng có thể tích $\SI{10}{l}$, nhiệt độ $\SI{300}{K}$, áp suất $\SI{0,8}{atm}$. Biến đổi qua hai quá trình liên tiếp
	\begin{itemize}
		\item QT1: đẳng tích, nhiệt độ tăng thêm $300^\circ\text{C}$.
		\item QT2: đẳng nhiệt, thể tích sau cùng 25 lít.
	\end{itemize}
	Tìm các thông số còn thiếu ở mỗi trạng thái	
\end{enumerate}
\textbf{ĐÁP ÁN}
\begin{longtable}[\textwidth]{|p{0.15\textwidth}|p{0.15\textwidth}|p{0.15\textwidth}|p{0.15\textwidth}|p{0.25\textwidth}|}
	% --- first head
	\hline%\hspace{2 pt}
	\multicolumn{1}{|c}{\textbf{Câu 1}} & \multicolumn{1}{|c|}{\textbf{Câu 2}} & \multicolumn{1}{c|}{\textbf{Câu 3}} &
	\multicolumn{1}{c|}{\textbf{Câu 4}} &
	\multicolumn{1}{c|}{\textbf{Câu 5}} \\
	\hline
	D.& B. &  & B. & $\SI{1.6e5}{\pascal}$. 	\\
	\hline
	
	\multicolumn{1}{|c|}{\textbf{Câu 6}} & \multicolumn{1}{c|}{\textbf{Câu 7}} & \multicolumn{1}{c|}{\textbf{Câu 8}} &
	\multicolumn{1}{c|}{\textbf{Câu 9}} &
	\multicolumn{1}{c|}{\textbf{Câu 10}} \\ 
	\hline
	$\SI{3.9}{\kilo \pascal}$.  & Không nổ. & $\SI{1.73}{atm}$.&$\SI{450}{\celsius}$. & $p_2=1,6\ \text{atm}$ \newline $p_3=0,64\ \text{atm}$. \\
	\hline
\end{longtable}		
