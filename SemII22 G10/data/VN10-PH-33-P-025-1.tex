%\setcounter{chapter}{1}
\chapter{Luyện tập: Cơ năng}
\section{Cơ năng của vật chuyển động theo đường thẳng}
\begin{enumerate}
	\item %câu 1
	Một học sinh ném một vật có khối lượng $\SI{200}{\gram}$ được ném thẳng đứng lên cao với vận tốc ban đầu $\SI{8}{\meter/\second}$ từ độ cao $\SI{8}{\meter}$ so với mặt đất. Lấy $g= \SI{10}{\meter/\second^2}$. Xác định vận tốc của vật khi $W_\text{đ}=2W_\text{t}$?
	\item %câu 2
	Cho một vật có khối lượng $m$. Truyền cho vật một cơ năng là $\SI{37,5}{\joule}$. Khi vật chuyển động ở độ cao $\SI{3}{\meter}$ vật có $W_\text{đ}=\dfrac{3}{2}W_\text{t}$. Xác định khối lượng của vật và vận tốc của vật ở độ cao đó. Lấy $g= \SI{10}{\meter/\second^2}$.
	\item %câu 3
	Một học sinh chơi đùa ở sân thượng có độ cao $\SI{45}{\meter}$, liền cầm một vật có khối lượng $\SI{100}{\gram}$ thả vật rơi tự do xuống mặt đất. Lấy $g= \SI{10}{\meter/\second^2}$.
	\begin{enumerate}[label=\alph*)]
		\item Tính vận tốc của vật khi vật chạm đất.
		\item Tính độ cao của vật khi $W_\text{đ}=2W_\text{t}$.
		\item Tính vận tốc của vật khi $2W_\text{đ}=5W_\text{t}$.
		\item Xác định vị trí để vận có vận tốc $\SI{20}{\meter/\second}$.
		\item Tại vị trí có độ cao $\SI{20}{\meter}$ vật có vận tốc bao nhiêu?
		\item Khi chạm đất, do đất mềm nên vật bị lún sâu $\SI{10}{\centi\meter}$. Tính lực cản trung bình tác dụng lên vật.
	\end{enumerate}
	\item %câu 4
	Một viên bi khối lượng $m$ chuyến động ngang không ma sát với vận tốc $\SI{2}{\meter/\second}$ rồi đi lên mặt phẳng nghiêng góc nghiêng $30^\circ$.
	\begin{enumerate}[label=\alph*)]
		\item Tính quãng đường $s$ mà viên bi đi được trên mặt phẳng nghiêng.
		\item Ở độ cao nào thì vận tốc của viên bi giảm còn một nửa?
		\item Khi vật chuyển động được quãng đường là $\SI{0,2}{\meter}$ lên mặt phẳng nghiêng thì vật có vận tốc bao nhiêu. Chọn mốc thế năng tại A và giả sử lên đến B vật dùng lại.
	\end{enumerate}
	\item %câu 5
	Thả vật rơi tự do từ độ cao $\SI{45}{\meter}$ so với mặt đất. Bỏ qua sức cản của không khí. Lấy $g= \SI{10}{\meter/\second^2}$.
	\begin{enumerate}[label=\alph*)]
		\item Tính vận tốc của vật khi vật chạm đất.
		\item Tính độ cao của vật khi $W_\text{đ}=2W_\text{t}$.
		\item Khi chạm đất, do đất mềm nên vật bị lún sâu $\SI{10}{\centi\meter}$. Tính lực cản trung bình tác dụng lên vật, cho $m=\SI{100}{\gram}$.
	\end{enumerate}		
\end{enumerate}

\textbf{ĐÁP ÁN:}

\begin{longtable}[\textwidth]{|m{0.15\textwidth}|m{0.15\textwidth}|m{0.17\textwidth}|m{0.15\textwidth}|m{0.15\textwidth}|}
	% --- first head
	\hline%\hspace{2 pt}
	\multicolumn{1}{|c}{\textbf{Câu 1}} &
	\multicolumn{1}{|c|}{\textbf{Câu 2}}& 
	\multicolumn{1}{c|}{\textbf{Câu 3}} &
	\multicolumn{1}{c|}{\textbf{Câu 4}} &
	\multicolumn{1}{c|}{\textbf{Câu 5}}  \\
	\hline
	$\SI{12,2}{\meter/\second}$. &
	$\SI{0,5}{\kilogram}$, $\SI{9,5}{\meter/\second}$. &
	\begin{enumerate}[label=\alph*)]
		\item $\SI{30}{\meter/\second}$.
		\item $\SI{15}{\meter}$.
		\item $\SI{25,4}{\meter/\second}$.
		\item $\SI{20}{\meter}$.
		\item $\SI{22,4}{\meter/\second}$.
		\item $\SI{451}{\newton}$.
	\end{enumerate} &
	\begin{enumerate}[label=\alph*)]
		\item $\SI{0,4}{\meter}$.
		\item $\SI{0,3}{\meter}$.
		\item $\sqrt{2}\,\SI{}{\meter/\second}$.
	\end{enumerate} &
	\begin{enumerate}[label=\alph*)]
		\item $\SI{30}{\meter/\second}$.
		\item $\SI{15}{\meter}$.
		\item $\SI{451}{\newton}$.
	\end{enumerate}\\
	\hline
\end{longtable}


\section{Cơ năng của con lắc đơn}
\begin{enumerate}
	\item %câu 1
	Một con lắc đơn có sợi dây dài $\SI{1}{\meter}$ và vật nặng có khối lượng $\SI{500}{\gram}$. Kéo vật lệch khỏi vị trí cân bằng sao cho cho dây làm với đường thẳng đứng một góc $60^\circ$ rồi thả nhẹ. Lấy $g= \SI{10}{\meter/\second^2}$.
	\begin{enumerate}[label=\alph*)]
		\item Xác định cơ năng của con lắc đơn trong quá trình chuyển động
		\item Tính vận tốc của con lắc khi nó đi qua vị trí mà dây làm với đường thẳng đứng góc $30^\circ$, $45^\circ$ và xác định lực căng của dây ở hai vị trí đó. Lấy $g= \SI{10}{\meter/\second^2}$.
		\item Xác định vị trí để vật có vận tốc $v=\SI{1,8}{\meter/\second}$.
		\item Xác định vận tốc tại vị trí $2W_\text{t}=W_\text{đ}$.
		\item Xác định vị trí để $2W_\text{t}=3W_\text{đ}$. Tính vận tốc và lực căng khi đó.
	\end{enumerate}
	\item %câu 2
	Cho một con lắc đơn gồm có sợi dây dài $\SI{320}{\centi\meter}$ đầu trên cố định đâu dưới treo một vật nặng có khối lượng $\SI{1000}{\gram}$. Khi vật đang ở vị trí cân bằng thì truyền cho vật một vận tốc là $4\sqrt{2}\,\SI{}{\meter/\second}$. Lấy $g= \SI{10}{\meter/\second^2}$. Xác định vị trí cực đại mà vật có thể lên tới?
	\item %câu 3
	Con lắc thử đạn là một bao cát, khối lượng $\SI{19,9}{\kilogram}$, treo vào một sợi dây có chiều dài là $\SI{2}{\meter}$. Khi bắn một đâu dạn khối lượng $\SI{100}{\gram}$ theo phương nằm ngang, thì đầu đạn cắm vào bao cát và nâng bao cát lên cao theo một cung tròn là cho trọng tâm của bao cát sao cho dây treo bao cát hợp với phương thẳng đứng một góc $60^\circ$. Xác định vận tốc $v$ của viên đạn trước lúc va chạm vào bao cát.
	\item %câu 4
	Dây treo vật nặng được kéo nghiêng góc bao nhiêu để khi qua vị trí cân bằng lực căng của dây lớn gấp đôi trọng lực của vật nặng?
	\item %câu 5
	Treo vật $m=\SI{1}{\kilogram}$ vào đầu một sợi dây rồi kéo vật khỏi vị trí cân bằng để dây treo hợp với phương thẳng đứng góc $\alpha_0$. Xác định $\alpha_0$ để khi buông tay, dây không bị đứt trong quá trình chuyển động. Biết dây chịu lực căng tối đa $\SI{16}{\newton}$ và $\alpha_0\leq90^\circ$.
\end{enumerate}

\textbf{ĐÁP ÁN:}

\begin{longtable}[\textwidth]{|m{0.21\textwidth}|m{0.15\textwidth}|m{0.17\textwidth}|m{0.15\textwidth}|m{0.15\textwidth}|}
	% --- first head
	\hline%\hspace{2 pt}
	\multicolumn{1}{|c}{\textbf{Câu 1}} &
	\multicolumn{1}{|c|}{\textbf{Câu 2}}& 
	\multicolumn{1}{c|}{\textbf{Câu 3}} &
	\multicolumn{1}{c|}{\textbf{Câu 4}} &
	\multicolumn{1}{c|}{\textbf{Câu 5}}  \\
	\hline
	\begin{enumerate}[label=\alph*)]
		\item $\SI{2,5}{\joule}$.
		\item $\SI{7,99}{\newton}$, $\SI{5,61}{\newton}$.
		\item $\SI{0,338}{\meter}$.
		\item $\SI{2,581}{\meter/\second}$.
		\item $\SI{2}{\meter/\second}$, $\SI{5,5}{\newton}$.
	\end{enumerate} &
	$\SI{6,4}{\meter}$. &
	$400\sqrt{5}\,\SI{}{\meter/\second}$. &
	$60^\circ$. &
	$\alpha\leq 45^\circ$. 
	\\
	\hline
\end{longtable}