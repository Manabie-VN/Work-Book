	\whiteBGstarBegin
\setcounter{section}{0}
\section{Lý thuyết: Nội năng}
\begin{enumerate}[label=\bfseries Câu \arabic*:]
	
	\item \mkstar{1} [4]
	
	\cauhoi{
		Nội năng của một vật là
		\begin{mcq}
			\item tổng động năng và thế năng của vật.
			\item tổng động năng và thế năng của các phân tử cấu tạo nên vật.
			\item tổng nhiệt lượng và cơ năng mà vật nhận được trong quá trình truyền nhiệt và thực hiện công.
			\item nhiệt lượng vật nhận được trong quá trình truyền nhiệt.
		\end{mcq}
	}
	
	\loigiai{
		\textbf{Đáp án: B.}
		
		Nội năng của một vật là tổng động năng và thế năng của các phân tử cấu tạo nên vật.
	}
	
	\item \mkstar{1} [24]
	
	\cauhoi{
		Nội năng của một vật phụ thuộc vào
		\begin{mcq}(2)
			\item nhiệt độ và khối lượng. 
			\item nhiệt độ và áp suất.
			\item áp suất và thể tích.
			\item nhiệt độ và thể tích.
		\end{mcq}
		
	}
	
	\loigiai{
		\textbf{Đáp án: D.}
		
			Nội năng của một vật là hàm phụ thuộc vào $T$ và $V$ $U=f(T,V)$, nghĩa là phụ thuộc vào nhiệt độ và thể tích.
	}
	


	\item \mkstar{2} [12]
	
	\cauhoi{
		\begin{enumerate}[label=\alph*)]
			\item Nội năng của một vật là gì?
			\item Tại sao nội năng của một khối khí lí tưởng chỉ phụ thuộc vào nhiệt độ của chất khí?
		\end{enumerate}
	}
	
	\loigiai{		
		\begin{enumerate}[label=\alph*)]
			\item Nội năng của một vật là tổng động năng và thế năng của các phân tử cấu tạo nên vật.
			\item Khí lí tưởng là khí mà các phân tử chỉ tương tác nhau khi va chạm, do đó thế năng phân tử có thể bỏ qua. Do đó nội năng của một khối khí lí tưởng không phụ thuộc vào thể tích và chỉ phụ thuộc vào nhiệt độ.
		\end{enumerate}
	}
	
\end{enumerate}
\section{Lý thuyết: Các cách làm thay đổi nội năng}
\begin{enumerate}[label=\bfseries Câu \arabic*:]
		\item \mkstar{2} [24]
	
	\cauhoi{
		Trường hợp nào sau đây làm biến đổi nội năng \textbf{không} do thực hiện công?
		\begin{mcq}(2)
			\item Đun nóng nước bằng bếp.
			\item Làm dãn nở khí trong xi lanh.
			\item Nén khí trong xi lanh.
			\item Cọ xát hai vật vào nhau.
		\end{mcq}
	}
	
	\loigiai{
		\textbf{Đáp án: A.}
		
		Đun nước bằng bếp là cách làm thay đổi nội năng bằng truyền nhiệt, không phải bằng thực hiện công.
	}
	
	
	\item \mkstar{1} [16]
	
	\cauhoi{
	Độ biến thiên nội năng là gì? Nêu hai cách làm thay đổi nội năng, mỗi cách cho 1 ví dụ.
	}
	\loigiai{
		
	Độ biến thiên nội năng là phần nội năng tăng thêm hay giảm bớt đi trong một quá trình.
	
	Có hai cách làm thay đổi nội năng:
	\begin{itemize}
		\item Thực hiện công: làm nóng miếng kim loại bằng ma sát;
		\item Truyền nhiệt: làm nóng miếng kim loại bằng cách nhúng vào nước nóng.
	\end{itemize}
	}

\item \mkstar{2} [3]

\cauhoi{
	Một lượng khí lí tưởng thực hiện quá trình biến đổi đẳng nhiệt. Nội năng của nó tăng, giảm hay không đổi? Nêu hai cách làm biến đổi nội năng của vật.
}

\loigiai{
	
	Trong quá trình biến đổi đẳng nhiệt thì nhiệt độ của khối khí không đổi, do đó nội năng chỉ phụ thuộc vào thể tích. Nếu sau quá trình đẳng nhiệt khối khí nở ra thì khối khí thực hiện công, dẫn đến nội năng giảm; nếu khối khí nén lại thì khối khí nhận công, dẫn đến nội năng tăng.
	
	Hai cách làm biến đổi nội năng của vật là thực hiện công và truyền nhiệt.
}
\end{enumerate}
\whiteBGstarEnd