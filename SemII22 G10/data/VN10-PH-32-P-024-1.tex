%\setcounter{chapter}{1}
\chapter{Luyện tập: Thế năng}
\begin{enumerate}
	\item %câu 1
	Dạng năng lượng tương tác giữa trái đất và vật là
	\begin{mcq}(2)
		\item thế năng đàn hồi.
		\item động năng.
		\item cơ năng.
		\item thế năng trọng trường.
	\end{mcq}
	\item %câu 2
	Một vật nằm yên có thể có
	\begin{mcq}(4)
		\item thế năng.
		\item vận tốc.
		\item động năng.
		\item động lượng.
	\end{mcq}
	\item %câu 3
	Biểu thức nào sau đây không phải biểu thức của thế năng?
	\begin{mcq}(2)
		\item $W_\text{t}=mgh$.
		\item $W=mg(z_2-z_1)$.
		\item $W=Ph$.
		\item $W=\dfrac{1}{2}mgh$.
	\end{mcq}
	\item %câu 4
	Một lò xo có độ cứng $k$, bị kéo giãn ra một đoạn $x$. Thế năng đàn hồi lò xo được tính bằng biểu thức
	\begin{mcq}(4)
		\item $W_\text{t}=\dfrac{kx^2}{2}$.
		\item $W_\text{t}=kx^2$.
		\item $W_\text{t}=\dfrac{kx}{2}$.
		\item $W_\text{t}=\dfrac{k^2x^2}{2}$.
	\end{mcq}
	\item %câu 5
	Một lò xo bị nén $\SI{5}{\centi\meter}$. Biết độ cứng lò xo $k=\SI{100}{\newton/\meter}$, thế năng của lò xo là
	\begin{mcq}(4)
		\item $\SI{0,125}{\joule}$.
		\item $\SI{0,25}{\joule}$.
		\item $\SI{125}{\joule}$.
		\item $\SI{250}{\joule}$.
	\end{mcq}
	\item %câu 6
	Thế năng của vật nặng $\SI{2}{\kilogram}$ ở đáy một giếng sâu $\SI{10}{\meter}$ so với mặt đất tại nơi có gia tốc $g= \SI{10}{\meter/\second^2}$ là bao nhiêu?
	\begin{mcq}(4)
		\item $\SI{-100}{\joule}$.
		\item $\SI{100}{\joule}$.
		\item $\SI{200}{\joule}$.
		\item $\SI{-200}{\joule}$. 
	\end{mcq}
	\item %câu 7
	Một lò xo có độ cứng $\SI{100}{\newton/\meter}$, một đầu cố định, đầu kia gắn với vật nhỏ. Khi lò xo bị nén $\SI{4}{\centi\meter}$ thì thế năng đàn hồi của hệ là
	\begin{mcq}(4)
		\item $\SI{800}{\joule}$.
		\item $\SI{0,08}{\joule}$.
		\item $\SI{8}{\joule}$.
		\item $\SI{80}{\joule}$. 
	\end{mcq}
	\item %câu 8
	Một người có khối lượng $\SI{60}{\kilogram}$ đứng trên mặt đất và cạnh một cái giếng nước, lấy $g= \SI{10}{\meter/\second^2}$.
	\begin{enumerate}[label=\alph*)]
		\item Tính thế năng của người tại A cách mặt đất $\SI{3}{\meter}$ về phía trên và tại đáy giếng cách mặt đất $\SI{5}{\meter}$ với gốc thế năng tại mặt đất.
		\item Nếu lấy mốc thể năng tại đáy giếng, hãy tính lại kết quả câu trên.
	\end{enumerate}
	\item %câu 9
	Một học sinh của trung tâm bồi dưỡng kiến thức Hà Nội thả một vật rơi tự do có khối lượng $\SI{500}{\gram}$ từ độ cao $\SI{45}{\meter}$ so với mặt đất, bỏ qua ma sát với không khí, Tính thế năng của vật tại giây thứ hai so với mặt đất. Cho $g= \SI{10}{\meter/\second^2}$.
	\item %câu 10
	Cho một lò xo nằm ngang có độ cứng $k=\SI{100}{\newton/\meter}$. Công của lực đàn hồi thực hiện khi lò xo bị kéo dãn từ $\SI{2}{\centi\meter}$ đến $\SI{4}{\centi\meter}$ là bao nhiêu?			
\end{enumerate}

\textbf{ĐÁP ÁN:}

\begin{longtable}[\textwidth]{|m{0.15\textwidth}|m{0.15\textwidth}|m{0.17\textwidth}|m{0.15\textwidth}|m{0.15\textwidth}|}
	% --- first head
	\hline%\hspace{2 pt}
	\multicolumn{1}{|c}{\textbf{Câu 1}} &
	\multicolumn{1}{|c|}{\textbf{Câu 2}}& 
	\multicolumn{1}{c|}{\textbf{Câu 3}} &
	\multicolumn{1}{c|}{\textbf{Câu 4}} &
	\multicolumn{1}{c|}{\textbf{Câu 5}}  \\
	\hline
	D. &
	A. &
	D. &
	A. &
	A. \\
	\hline
	\multicolumn{1}{|c}{\textbf{Câu 6}} &
	\multicolumn{1}{|c|}{\textbf{Câu 7}}&  
	\multicolumn{1}{c|}{\textbf{Câu 8}} &
	\multicolumn{1}{c|}{\textbf{Câu 9}} &
	\multicolumn{1}{c|}{\textbf{Câu 10}}  \\
	\hline
	D. &
	B. &
	\begin{enumerate}[label=\alph*)]
		\item $\SI{1800}{\joule}$.
		\item $\SI{4800}{\joule}$.
	\end{enumerate} &
	$\SI{25}{\joule}$.&
	$\SI{-0,06}{\joule}$.
	\\
	\hline
\end{longtable}



