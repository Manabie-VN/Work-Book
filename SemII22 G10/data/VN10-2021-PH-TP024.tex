\whiteBGstarBegin
\setcounter{section}{0}
\section{Lý thuyết: Công cơ học}
\begin{enumerate}[label=\bfseries Câu \arabic*:]
		\item \mkstar{1} [5]
	
	\cauhoi{
		Công là đại lượng
		\begin{mcq}
			\item vô hướng, có thể âm hoặc dương. 
			\item vectơ, có thể âm, dương hoặc bằng 0.
			\item vectơ, có thể âm hoặc dương.
			\item vô hướng, có thể âm, dương hoặc bằng 0.
		\end{mcq}
		
	}
	
	\loigiai{
		\textbf{Đáp án: D.}
		
		Công là đại lượng vô hướng, có thể âm, dương hoặc bằng 0.
	}
	
		\item \mkstar{1} [5]
	
	\cauhoi{
		Trường hợp nào sau đây công của lực bằng không?
		\begin{mcq}
			\item Lực vuông góc với phương chuyển động của vật. 
			\item Lực cùng phương với phương chuyển động của vật.
			\item Lực hợp với phương chuyển động một góc lớn hơn $90^\circ$.
			\item Lực hợp với phương chuyển động một góc nhỏ hơn $90^\circ$.
		\end{mcq}
		
	}
	
	\loigiai{
		\textbf{Đáp án: A.}
		
		Khi lực vuông góc với phương chuyển động của vật thì $\alpha=90^\circ$, khi đó $\cos \alpha =0$, dẫn đến $A=Fs\cos \alpha=0$.
	}

	\item \mkstar{2} [4]
	
	\cauhoi{
		Chọn câu \textbf{sai}.
		\begin{mcq}
			\item Công của lực cản âm vì $90^\circ < \alpha < 180^\circ$.
			\item Công của lực phát động dương vì $90^\circ > \alpha > 0^\circ$.
			\item Vật dịch chuyển theo phương nằm ngang thì công của trọng lực bằng $0$.
			\item Vật dịch chuyển trên mặt phẳng nghiêng thì công của trọng lực bằng $0$.
		\end{mcq}
	}

	\loigiai{
		\textbf{Đáp án: D.}
		
		Vật dịch chuyển trên mặt phẳng nghiêng thì công của trọng lực khác $0$, vì phương của trọng lực không vuông góc với phương của mặt nghiêng.
	}
	
		\item \mkstar{2} [5]
	
	\cauhoi{
		$\SI{1}{kWh}$ tương đương với
		\begin{mcq}(4)
			\item $\SI{3600}{kJ}$. 
			\item $\SI{1000}{kJ}$.
			\item $\SI{600}{kJ}$.
			\item $\SI{60}{kJ}$.
		\end{mcq}
		
	}
	
	\loigiai{
		\textbf{Đáp án: A.}
		
		Đổi đơn vị $\SI{1}{kWh} = \SI{1000}{W} \cdot \SI{1}{h} = \SI{1000}{W}\cdot \SI{3600}{s} = \SI{3600000}{J} =\SI{3600}{kJ}$.
	}

		\item \mkstar{2} [5]
	
	\cauhoi{
		Sử dụng một lực $F=\SI{50}{N}$ tạo với phương ngang một góc $\alpha = 60^\circ$ kéo một vật và làm vật chuyển động thẳng đều trên mặt phẳng nằm ngang. Công của lực kéo khi vật di chuyển được một đoạn đường bằng $\SI{6}{m}$ là
		\begin{mcq}(4)
			\item $\SI{0}{J}$. 
			\item $\SI{260}{J}$.
			\item $\SI{300}{J}$.
			\item $\SI{150}{J}$.
		\end{mcq}
		
	}
	
	\loigiai{
		\textbf{Đáp án: D.}
		
		Công của lực kéo:
		$$A=Fs\cos \alpha  =\SI{150}{J}.$$
	}


		\item \mkstar{2} [24]
	
	\cauhoi{
		Một người kéo một thùng gỗ trượt trên sàn nhà bằng một sợi dây hợp với phương ngang một góc $60^\circ$, lực tác dụng lên dây là $\SI{200}{N}$. Khi thùng gỗ được kéo và trượt một đoạn $\SI{10}{m}$ thì công của lực kéo là
		\begin{mcq}(4)
			\item $\SI{200}{J}$. 
			\item $\SI{1000}{J}$.
			\item $\SI{2000}{J}$.
			\item $\SI{120000}{J}$.
		\end{mcq}
		
	}
	
	\loigiai{
		\textbf{Đáp án: B.}
		
		Công của lực kéo:
		$$A=Fs\cos \alpha = \SI{1000}{J}.$$
	}

			\item \mkstar{2} [29]
	
	\cauhoi{
		Vật nào sau đây \textbf{không} có khả năng sinh công?
		\begin{mcq}
			\item Vật đang rơi tự do xuống mặt đất. 
			\item Dòng nước từ trên cao đổ mạnh xuống làm quay tuabin nước.
			\item Vật đang nằm yên trên mặt đất.
			\item Viên đạn đang bay.
		\end{mcq}
		
	}
	
	\loigiai{
		\textbf{Đáp án: C.}
		
		Vật đang nằm yên trên mặt đất không có khả năng sinh công.
	}
	
	\item \mkstar{2} [8]
	
	\cauhoi{
		Một hành khách kéo đều một vali đi trong nhà ga trên sân bay trên quãng đường dài $\SI{150}{m}$ với lực kéo có độ lớn $\SI{40}{N}$ theo hướng hợp với phương ngang một góc $60^\circ$. Hãy xác định công của lực kéo của người này.
	}

	\loigiai{		
		Công của lực kéo của người: $A=Fs\cos \alpha = \SI{3000}{J}$.
	}
	
	\item \mkstar{2} [13]
	
	\cauhoi{
		Tính công của lực $F=\SI{1200}{N}$ tác dụng lên vật làm vật dịch chuyển quãng đường $\SI{4}{m}$, biết góc hợp bởi chiều của lực và chiều dịch chuyển là $60^\circ$.
	}

	\loigiai{
		Công của lực: $A=Fs\cos \alpha = \SI{2400}{J}$.

	}

		\item \mkstar{3} [20]
	
	\cauhoi{
		Con ngựa kéo chiếc xe với một lực kéo $F=\SI{100}{N}$ theo phương nằm ngang. Chiếc xe chuyển động thẳng đều trên đường nằm ngang với vận tốc $\SI{8}{m/s}$ trong thời gian 5 giây. Tính công của lực kéo của con ngựa ở đoạn đường trên. 
	}
	
	\loigiai{
				Quãng đường con ngựa kéo xe:
				$$s=vt=\SI{40}{m}.$$
				
				Công của lực kéo:
				$$A=Fs=\SI{4000}{J}.$$
	}

	\item \mkstar{3} [22]
	
	\cauhoi{
		Một thùng nước khối lượng $\SI{10}{kg}$ được kéo cho chuyển động thẳng đều lên cao $\SI{5}{m}$ trong thời gian 1 phút 40 giây. Tính công của lực kéo. Lấy $g=\SI{10}{m/s^2}$.
	}
	
	\loigiai{
		Lực kéo thùng nước để thùng chuyển động thẳng đều:
		$$F=P=mg=\SI{100}{N}.$$
		
		Công của lực kéo:
		$$A=Fs=\SI{500}{J}.$$
	}
\end{enumerate}
\section{Lý thuyết: Công suất}
\begin{enumerate}[label=\bfseries Câu \arabic*:]
	\item \mkstar{1} [24]
	
	\cauhoi{
	Công suất được xác định bằng
		\begin{mcq}
			\item giá trị công thực hiện được. 
			\item tích của công và thời gian thực hiện công.
			\item công thực hiện được trên một đơn vị chiều dài.
			\item công thực hiện được trong một đơn vị thời gian.
		\end{mcq}
		
	}
	
	\loigiai{
		\textbf{Đáp án: D.}
		
		Công suất được xác định bằng công thực hiện được trong một đơn vị thời gian.
	}
						\item \mkstar{1} [29]
	
	\cauhoi{
		Đơn vị nào sau đây \textbf{không} phải là đơn vị của công suất?
		\begin{mcq}(4)
			\item Oát (W). 
			\item Jun/giây (J/s).
			\item Mã lực (HP).
			\item Jun (J).
		\end{mcq}
		
	}
	
	\loigiai{
		\textbf{Đáp án: D.}
		
		Jun (J) là đơn vị của công, không phải là đơn vị công suất.
	}

	\item \mkstar{3} [29]
	
	\cauhoi{
		Một động cơ điện cung cấp công suất $\SI{15}{kW}$ cho một cần cẩu nâng vật có khối lượng 1 tấn lên cao $\SI{15}{m}$. Thời gian tối thiểu để thực hiện công việc này bằng bao nhiêu? Lấy $g=\SI{10}{m/s^2}$.
		\begin{mcq}(4)
			\item $\SI{12}{s}$. 
			\item $\SI{10}{s}$.
			\item $\SI{14}{s}$.
			\item $\SI{18}{s}$.
		\end{mcq}
		
	}
	
	\loigiai{
		\textbf{Đáp án: B.}
		
		Áp dụng công thức tính công suất:
		$$\calP = \dfrac{A}{t} \Rightarrow t = \dfrac{A}{\calP}  = \dfrac{mgs}{\calP} = \SI{10}{s}.$$
	}


		\item \mkstar{1} [15]
	
	\cauhoi{
		Nêu khái niệm công suất. Viết biểu thức tính công suất.
	}
	
	\loigiai{		
		Công suất là đại lượng đặc trưng cho tốc độ thực hiện công của người hoặc máy, được xác định bằng công mà lực thực hiện được trong một đơn vị thời gian.
		
		Biểu thức: $\calP=\dfrac{A}{t}$.
		
		Với vật chuyển động thẳng đều với vận tốc $v$ thì $\calP = Fv$. 
	}
	
		\item \mkstar{2} [8]
	
	\cauhoi{
		Một hành khách kéo đều một vali đi trong nhà ga trên sân bay trên quãng đường dài $\SI{150}{m}$ với lực kéo có độ lớn $\SI{40}{N}$ theo hướng hợp với phương ngang một góc $60^\circ$. Hãy xác định công suất của lực kéo của người này trong khoảng thời gian 5 phút.
	}
	
	\loigiai{		
		Công suất của lực kéo của người:
		$$\calP = \dfrac{A}{t} = \dfrac{Fs\cos \alpha}{t}= \SI{10}{W}.$$
	}

	\item \mkstar{3} [8]
	
	\cauhoi{
		Một ô tô chuyển động thẳng đều trên đường nằm ngang với vận tốc $\SI{72}{km/h}$, công suất của động cơ là $\SI{75}{kW}$. Tính lực phát động của động cơ.
	}
	\loigiai{
		
		Công suất của vật chuyển động thẳng đều:
		$$\calP = Fv \Rightarrow F = \dfrac{\calP}{v} = \SI{3750}{N}.$$
	}


	\item \mkstar{3} [29]

\cauhoi{
	Một vật có khối lượng 1,5 tấn được cần cẩu nâng đều lên độ cao $\SI{20}{m}$ trong khoảng thời gian $\SI{20}{s}$. Lấy $g=\SI{10}{m/s^2}$. Tính công suất trung bình của lực nâng của cần cẩu.
}

\loigiai{		
	Lực kéo của cần cẩu để nâng vật đi lên thẳng đều:
	$$F=P=mg=\SI{15000}{N}.$$
	
	Vận tốc nâng:
	$$v=\dfrac{s}{t} = \SI{1}{m/s}.$$
	
	Công suất trung bình của lực nâng của cần cẩu:
	$$\calP = Fv = \SI{15000}{W}.$$
}
\end{enumerate}
\whiteBGstarEnd