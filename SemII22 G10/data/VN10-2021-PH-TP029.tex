\whiteBGstarBegin
\setcounter{section}{0}

	\begin{enumerate}[label=\bfseries Câu \arabic*:]
		
		\item \mkstar{1} [5]
		
		\cauhoi{
			Tính chất nào sau đây \textbf{không} phải là của phân tử ở thể khí?
			\begin{mcq}
				\item Chuyển động càng nhanh thì nhiệt độ của vật càng cao.
				\item Giữa các phân tử có khoảng cách.
				\item Chuyển động không ngừng.
				\item Có lúc đứng yên, có lúc chuyển động.
			\end{mcq}
		}
		
		\loigiai{
			\textbf{Đáp án: D.}
			
			Các phân tử luôn chuyển động hỗn loạn không ngừng, do đó nói các phân tử khí "có lúc đứng yên" là sai.
		}
				\item \mkstar{1} [5]
		
		\cauhoi{
			Nhận xét nào sau đây \textbf{không phù hợp} với khí lí tưởng?
			\begin{mcq}
				\item Các phân tử chuyển động càng nhanh khi nhiệt độ càng cao. 
				\item Khối lượng các phân tử có thể bỏ qua.
				\item Thể tích các phân tử có thể bỏ qua.
				\item Các phân tử chỉ tương tác với nhau khi va chạm.
			\end{mcq}
		}
		
		\loigiai{\textbf{Đáp án: B.}		
			
			"Khối lượng các phân tử có thể bỏ qua" là không phù hợp với khí lí tưởng.
		}
	
		\item \mkstar{1} [24]
		
		\cauhoi{
			Khí nào sau đây \textbf{không phải} là khí lí tưởng?
			\begin{mcq}
				\item Khí mà các phân tử được coi là chất điểm. 
				\item Khí mà các phân tử chuyển động càng nhanh khi nhiệt độ càng cao.
				\item Khí không tuân theo đúng định luật Boyle - Mariotte.
				\item Khí mà lực tương tác giữa các phân tử khí khi không va chạm là không đáng kể.
			\end{mcq}
			
		}
		
		\loigiai{
			\textbf{Đáp án: C.}
			
			Khí lí tưởng có tuân theo đúng định luật Boyle - Mariotte, do đó phát biểu "Khí không tuân theo đúng định luật Boyle - Mariotte" là sai.
		}
		

	
		\item \mkstar{1} [8]
		
		\cauhoi{
			Hãy phát biểu nội dung cơ bản của thuyết động học phân tử chất khí.
		}
		
		\loigiai{		
			\begin{itemize}
				\item Chất khí được cấu tạo từ các phân tử có kích thước rất nhỏ so với khoảng cách giữa chúng;
				\item Các phân tử khí chuyển động hỗn loạn không ngừng, chuyển động này càng nhanh thì nhiệt độ của chất khí càng cao;
				\item Khi chuyển động hỗn loạn các phân tử khí va chạm vào nhau và va chạm vào thành bình gây ra áp suất lên thành bình.
			\end{itemize}
		}
	
		\item \mkstar{1} [11]
		
		\cauhoi{
			Nêu định nghĩa khí lí tưởng.
		}
		
		\loigiai{		
			\begin{itemize}
				\item Cách trả lời 1: Khí lí tưởng là một loại chất khí tưởng tượng chứa các hạt giống nhau có kích thước vô cùng nhỏ so với thể tích của khối khí và không tương tác với nhau, chúng chỉ va chạm đàn hồi với vật chứa bao quanh khối khí.
				\item Cách trả lời 2: Khí lí tưởng là chất khí mà các phân tử khí được coi là các chất điểm và các phân tử chỉ tương tác nhau khi va chạm.
			\end{itemize}.
		}
	
		\item \mkstar{2} [15]
		
		\cauhoi{
			Tại sao chất khí không có hình dạng và thể tích riêng?
		}
		
		\loigiai{		
			Các phân tử khí có kích thước rất nhỏ và cách nhau rất xa. Lực tương tác phân tử rất yếu. Do đó chất khí không có hình dạng riêng và thể tích riêng.
		}
	
		\item \mkstar{2} [15]
		
		\cauhoi{
			Tại sao chất rắn có hình dạng và thể tích riêng?
		}
		
		\loigiai{		
			Các phân tử chất rắn ở rất gần nhau. Lực tương tác phân tử rất mạnh. Do đó chất rắn có hình dạng riêng và thể tích riêng.
		}
		
	\end{enumerate}
\whiteBGstarEnd