\whiteBGstarBegin
\setcounter{section}{0}
\begin{enumerate}[label=\bfseries Câu \arabic*:]
	
	\item \mkstar{1} [4]
	
	\cauhoi{
		Chọn câu trả lời đúng. Kết luận nào sau đây là đúng khi nói về mối liên hệ giữa hệ số nở khối $\beta$ và hệ số nở dài $\alpha$?
		\begin{mcq}(4)
			\item $\beta = 3 \alpha$.
			\item $\beta = \sqrt 3 \alpha$.
			\item $\beta = \alpha^3$.
			\item $\beta = \alpha$.
		\end{mcq}
	}
	
	\loigiai{
		\textbf{Đáp án: A.}
		
		Mối liên hệ giữa hệ số nở khối $\beta$ và hệ số nở dài $\alpha$ $\beta = 3 \alpha$.
	}
		\item \mkstar{1} [24]
	
	\cauhoi{
		Một vật có thể tích $V_0$ ở $t_0 \SI{}{\celsius}$ và thể tích $V$ ở $t \SI{}{\celsius}$. Cho $\alpha$ là hệ số nở dài của vật này. Biểu thức tính thể tích $V$ ở nhiệt độ $t\SI{}{\celsius}$ là
		\begin{mcq}(2)
			\item $V=V_0 \cdot 3\alpha (t-t_0)$.
			\item $V=v_0 + 3\alpha (t-t_0)$.
			\item $V=V_0 \cdot [1+3\alpha (t-t_0)]$.
			\item $V=V_0 (1+3\alpha t)$.
		\end{mcq}
	}
	
	\loigiai{
		\textbf{Đáp án: C.}
		
		Công thức tính độ nở khối:
		$$V=V_0 \cdot [1+\beta (t-t_0)]=V_0 \cdot [1+3\alpha (t-t_0)].$$
	}
	\item \mkstar{2} [24]
	
	\cauhoi{
		Nguyên tắc hoạt động của dụng cụ nào sau đây liên quan tới sự nở vì nhiệt?
		\begin{mcq}(2)
			\item Nồi cơm điện.
			\item Băng kép.
			\item Bếp điện.
			\item Quạt điện.
		\end{mcq}
	}
	
	\loigiai{
		\textbf{Đáp án: B.}
		
		Băng kép là dụng cụ hoạt động dựa trên hiện tượng nở vì nhiệt.
	}
	

	
	\item \mkstar{2} [24]
	
	\cauhoi{
		Một chiếc đũa thủy tinh ở nhiệt độ $\SI{30}{\celsius}$ có chiều dài $\SI{20}{cm}$. Biết hệ số nở dài của thủy tinh là $\SI{9e-6}{K^{-1}}$. Chiều dài của chiếc đũa khi nhiệt độ tăng lên đến $\SI{80}{\celsius}$ là
		\begin{mcq}(4)
			\item $\SI{20.909}{cm}$.
			\item $\SI{20.114}{cm}$.
			\item $\SI{20.009}{cm}$.
			\item $\SI{20.0114}{cm}$.
		\end{mcq}
	}
	
	\loigiai{
		\textbf{Đáp án: C.}
		
		Công thức tính độ nở dài:
		$$l=l_0 (1+\alpha(t-t_0)) = \SI{20.009}{cm}.$$
	}
	
	\item \mkstar{3} [24]
	
	\cauhoi{
		Cho một khối sắt khi ở nhiệt độ $\SI{0}{\celsius}$ có thể tích ban đầu là $V_0$ và có khối lượng riêng là $\SI{7850}{kg/m^3}$. Biết hệ số nở dài của sắt là $\SI{12e-6}{K^{-1}}$. Khối lượng riêng của sắt ở $\SI{200}{\celsius}$ là
		\begin{mcq}(4)
			\item $\SI{7906.5}{kg/m^3}$.
			\item $\SI{7850}{kg/m^3}$.
			\item $\SI{7793.9}{kg/m^3}$.
			\item $\SI{7800}{kg/m^3}$.
		\end{mcq}
	}
	
	\loigiai{
		\textbf{Đáp án: C.}
		
		Theo công thức tính khối lượng riêng: $D=\dfrac{m}{V}$, suy ra $D \sim \dfrac{1}{V}$.
		
		Dựa vào công thức tính độ nở khối:
		$$V=V_0 \cdot [1+3\alpha (t-t_0)] \Rightarrow \dfrac{1}{D}=\dfrac{1}{D_0} \cdot [1+3\alpha (t-t_0)] \Rightarrow D = \SI{7793.9}{kg/m^3}.$$
	}
		\item \mkstar{1} [21]
	
	\cauhoi{
		Phát biểu và viết công thức nở dài của vật rắn (có giải thích các đại lượng). Các ống kim loại dẫn hơi nóng hoặc nước nóng người ta phải làm như thế nào để ống không bị gãy, hỏng do hiện tượng nở dài?
		
	}
	
	\loigiai{
		
		Sự nở dài là sự tăng độ dài của vật rắn khi nhiệt độ tăng.
		
		Công thức tính độ nở dài:
		$$l=l_0 (1+\alpha(t-t_0)).$$
		
		Hay:
		$$\Delta l = l_0 (1+\alpha (t-t_0)).$$
		
		Với:
		\begin{itemize}
			\item $l_0$ là chiều dài của thanh rắn ở nhiệt độ $t_0$;
			\item $l$ là chiều dài của thanh rắn ở nhiệt độ $t$;
			\item $\alpha$ là hệ số nở dài (đơn vị $\SI{}{K^{-1}}$).
		\end{itemize}
		
		Ứng dụng: Các ống kim loại dẫn hơi nóng phải có đoạn uốn cong để khi ống bị nở dài thì đoạn cong này chỉ biến dạng mà không bị gãy.
		
	}
	\item \mkstar{2} [4]
	
	\cauhoi{
		Một thước thẳng bằng thép dài $\SI{0.5}{m}$ ở $\SI{10}{\celsius}$. Tính chiều dài của cây thước ở $\SI{50}{\celsius}$. Biết hệ số nở dài của thép là $\SI{12e-6}{K^{-1}}$.
	
	}
	
	\loigiai{
		
		Công thức tính độ nở dài:
		$$l=l_0 (1+\alpha(t-t_0)) = \SI{0.50024}{m}.$$
	}
	

	
	\item \mkstar{2} [21]
	
	\cauhoi{
		Hai thanh sắt và nhôm có cùng chiều dài $\SI{40}{cm}$ ở nhiệt độ $\SI{20}{\celsius}$. Biết hệ số nở dài của sắt và nhôm lần lượt là $\alpha_{\ce{Fe}} = \SI{11e-6}{K^{-1}}$ và $\alpha_{\ce{Al}} = \SI{24e-6}{K^{-1}}$. Tính độ chênh lệch chiều dài của hai thanh ở nhiệt độ $\SI{350}{\celsius}$.
	}
	
	\loigiai{		
		Chiều dài thanh sắt ở $\SI{350}{\celsius}$:
		$$l_{\ce{Fe}} = l_0 (1+\alpha_{\ce{Fe}} (t-t_0)) = \SI{40.1452}{cm}.$$
		
		Chiều dài thanh nhôm ở $\SI{350}{\celsius}$:
		$$l_{\ce{Al}} = l_0 (1+\alpha_{\ce{Al}} (t-t_0)) = \SI{40.3168}{cm}.$$
		
		Độ chênh lệch chiều dài của hai thanh:
		$$l_{\ce{Al}} - l_{\ce{Fe}} = \SI{0.1716}{cm}.$$
	}
	
		\item \mkstar{2} [31]
	
	\cauhoi{
		Một thanh sắt ở $\SI{20}{\celsius}$ có chiều dài $\SI{800}{mm}$. Khi nhiệt độ tăng lên đến $\SI{220}{\celsius}$ thì thanh sắt này có chiều dài bằng bao nhiêu? Cho biết hệ số nở dài của thanh sắt là $\SI{11e-6}{K^{-1}}$.
	}
	
	\loigiai{		
		Chiều dài thanh sắt ở $\SI{220}{\celsius}$:
		$$l_{\ce{Fe}} = l_0 (1+\alpha_{\ce{Fe}} (t-t_0)) = \SI{801.76}{cm}.$$
	}

\end{enumerate}
\whiteBGstarEnd