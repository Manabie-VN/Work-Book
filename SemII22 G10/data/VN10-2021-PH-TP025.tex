\whiteBGstarBegin
\setcounter{section}{0}
\section{Lý thuyết: Động năng}
\begin{enumerate}[label=\bfseries Câu \arabic*:]
		\item \mkstar{1} [5]
	
	\cauhoi{
		Động năng của một vật khối lượng $m$, chuyển động với vận tốc $v$ là
		\begin{mcq}(4)
			\item $W_\text{đ} = \dfrac{1}{2}mv$.
			\item $W_\text{đ} = \dfrac{1}{2}mv^2$.
			\item $W_\text{đ} = mv^2$.
			\item $W_\text{đ} = 2mv^2$.
		\end{mcq}
	}
	
	\loigiai{
		\textbf{Đáp án: B.}
		
		Động năng của một vật khối lượng $m$, chuyển động với vận tốc $v$ là $W_\text{đ} = \dfrac{1}{2}mv^2$.
	}
	
	\item \mkstar{1} [5]
	
	\cauhoi{
		Chọn phát biểu \textbf{sai}. Động năng của một vật không đổi khi vật chuyển động
		\begin{mcq}(2)
			\item tròn đều.
			\item với gia tốc không đổi.
			\item với vận tốc không đổi.
			\item thẳng đều.
		\end{mcq}
	}
	
	\loigiai{
		\textbf{Đáp án: B.}
		
		Khi vật chuyển động có gia tốc thì vận tốc thay đổi, do đó động năng thay đổi.
	}
	
	\item \mkstar{1} [24]
	
	\cauhoi{
		Động năng của vật tăng khi
		\begin{mcq}
			\item vận tốc của vật có giá trị âm.
			\item vận tốc của vật có giá trị dương.
			\item các lực tác dụng lên vật sinh công âm.
			\item các lực tác dụng lên vật sinh công dương.
		\end{mcq}
	}
	
	\loigiai{
		\textbf{Đáp án: D.}
		
		Các lực tác dụng lên vật sinh công dương thì $v$ tăng, do đó động năng tăng.
	}

	\item \mkstar{2} [5]
	
	\cauhoi{
		Một vật có khối lượng $\SI{0.5}{kg}$ chuyển động với vận tốc $\SI{10}{m/s}$. Động năng của vật bằng
		\begin{mcq}(4)
			\item $\SI{250}{J}$.
			\item $\SI{50}{J}$.
			\item $\SI{5}{J}$.
			\item $\SI{25}{J}$.
		\end{mcq}
	}
	
	\loigiai{
		\textbf{Đáp án: D.}
		
		Động năng của vật: $W_\text{đ} = \dfrac{1}{2}mv^2 = \SI{25}{J}$.
	}
	
	\item \mkstar{2} [5]
	
	\cauhoi{
		Một vận động viên có khối lượng $\SI{80}{kg}$ chạy đều hết quãng đường $\SI{180}{m}$ trong thời gian 40 giây. Động năng của vận động viên đó là
		\begin{mcq}(4)
			\item $\SI{810}{J}$. 
			\item $\SI{360}{J}$.
			\item $\SI{875}{J}$.
			\item $\SI{180}{J}$.
		\end{mcq}
		
	}
	
	\loigiai{
		\textbf{Đáp án: A.}
		
		Vận tốc của vận động viên:
		$$v=\dfrac{s}{t} = \SI{4.5}{m/s}.$$
		
		Động năng của vận động viên:
		$$W_\text{đ} = \dfrac{1}{2} mv^2 = \SI{810}{J}.$$
	}

	\item \mkstar{2} [24]
	
	\cauhoi{
		Khi vận tốc của vật tăng gấp đôi thì
		\begin{mcq}(2)
			\item động lượng của vật tăng gấp đôi.
			\item gia tốc của vật tăng gấp đôi.
			\item động năng của vật tăng gấp đôi.
			\item thế năng của vật tăng gấp đôi.
		\end{mcq}
	}
	
	\loigiai{
		\textbf{Đáp án: A.}
		
		Khi vận tốc của vật tăng gấp đôi thì động lượng của vật tăng gấp đôi, động năng của vật tăng gấp 4.
	}

	\item \mkstar{2} [24]
	
	\cauhoi{
		Một vật có khối lượng $\SI{250}{g}$ đang di chuyển với tốc độ $\SI{10}{m/s}$. Động năng của vật này là
		\begin{mcq}(4)
			\item $\SI{12.5}{J}$.
			\item $\SI{25}{J}$.
			\item $\SI{12500}{J}$.
			\item $\SI{25000}{J}$.
		\end{mcq}
	}
	
	\loigiai{
		\textbf{Đáp án: A.}
		
		Động năng của vật:
		$$W_\text{đ} = \dfrac{1}{2} mv^2  =\SI{12.5}{J}.$$
	}

	\item \mkstar{1} [7]
	
	\cauhoi{
		Động năng của một vật là gì? Nêu biểu thức tính động năng. Động năng là đại lượng vô hướng hay vectơ?
	}
	
	\loigiai{		
		Dạng năng lượng mà một vật có được do nó đang chuyển động gọi là động năng.
		
		Biểu thức tính động năng: $W_\text{đ} = \dfrac{1}{2}mv^2$.
		
		Động năng là đại lượng vô hướng.
	}
	
	
	\item \mkstar{2} [31]
	
	\cauhoi{
		Vật nặng $\SI{1}{kg}$ chuyển động với vận tốc $\SI{8}{m/s}$. Tính động năng của vật.
	}
	
	\loigiai{
		Động năng của vật:
		$$W_\text{đ} = \dfrac{1}{2}mv^2 = \SI{32}{J}.$$
	}

	
\end{enumerate}
\section{Lý thuyết: Biến thiên động năng}
\begin{enumerate}[label=\bfseries Câu \arabic*:]
	
	\item \mkstar{1} [4]
	
	\cauhoi{
		Độ biến thiên động năng của một vật chuyển động bằng
		\begin{mcq}
			\item công của lực ma sát tác dụng lên vật.
			\item công của lực thế (ví dụ: trọng lực) tác dụng lên vật.
			\item công của trọng lực tác dụng lên vật.
			\item công của các lực tác dụng lên vật.
		\end{mcq}
	}

\loigiai{
\textbf{Đáp án: D.}

Độ biến thiên động năng của một vật chuyển động bằng công của các lực tác dụng lên vật.
}

\item \mkstar{1} [20]

\cauhoi{
	Nêu mối liên hệ giữa công của lực tác dụng và độ biến thiên động năng (định lý động năng). Biểu thức của định lý.
}
\loigiai{
	
	Độ biến thiên động năng bằng công của các lực tác dụng vào vật.
	
	Biểu thức: $\dfrac{1}{2} mv^2 - \dfrac{1}{2}mv_0^2 = A$.
}

\item \mkstar{2} [9]

\cauhoi{
	Một ô tô khối lượng $m$ bằng 1 tấn đang chuyển động với vận tốc $v=\SI{20}{m/s}$. Tính độ biến thiên động năng của ô tô khi nó bị hãm tới khi vận tốc còn $\SI{10}{m/s}$.
}
\loigiai{
	
	Độ biến thiên động năng:
	$$\Delta W_\text{đ} = \dfrac{1}{2} m (v^2 - v_0^2) = \SI{-150000}{J}.$$
}

	\item \mkstar{3} [1]
	
	\cauhoi{
		Một xe khối lượng 1 tấn khởi hành không vận tốc đầu, chuyển động nhanh dần đều trên đường nằm ngang. Sau khi đi được quãng đường $\SI{100}{m}$ thì đạt vận tốc $\SI{72}{km/h}$. Biết lực ma sát bằng $5\%$ trọng lượng của xe. Dùng định lý động năng, tính công của lực kéo của động cơ xe.
	}
	\loigiai{
		
		Độ lớn lực ma sát bằng $5\%$ trong lượng xe nên $$F_\text{ms} = \dfrac{5mg}{100} = \SI{500}{N}.$$
		
		Áp dụng định lý động năng:
		$$A_F + A_\text{ms} = \dfrac{1}{2}mv^2 - 0 \Rightarrow A_F = \dfrac{1}{2} mv^2 - (-F_\text{ms}s) = \SI{250000}{J}.$$
	}

\item \mkstar{3} [3]

\cauhoi{
	Một vật nhỏ $m$ được truyền vận tốc đầu $v_0=\SI{5}{m/s}$ tại A để vật trượt trên mặt phẳng ngang $\text{AB} = \SI{3}{m}$. Hệ số ma sát giữa vật và mặt ngang là $\mu=\SI{0.2}{}$. Lấy $g=\SI{10}{m/s^2}$. Tính vận tốc $v$ của vật tại B.
}
\loigiai{
	
	Áp dụng định lý động năng:
	$$A_\text{ms} = \dfrac{1}{2}mv^2 - \dfrac{1}{2}mv_0^2 \Rightarrow -F_\text{ms}s = \dfrac{1}{2}mv^2 - \dfrac{1}{2}mv_0^2 \Rightarrow -\mu g s = \dfrac{1}{2}v^2 - \dfrac{1}{2}v_0^2 \Rightarrow v = \SI{3.6}{m/s}.$$
}



\item \mkstar{3} [13]

\cauhoi{
	Một ô tô có khối lượng 2 tấn đang chạy với vận tốc $\SI{54}{km/h}$ trên đường nằm ngang thì lái xe thấy có chướng ngại vật cách ô tô $\SI{100}{m}$ thì tắt máy, đạp thắng. Biết hệ số ma sát giữa bánh xe và mặt đường là $\mu = 0,1$. Xe ô tô có đâm vào chướng ngại vật không? Cho $g=\SI{10}{m/s^2}$.
}
\loigiai{
	
	Quãng đường xe chạy được cho đến khi dừng hẳn theo lý thuyết định lý động năng là
	$$-F_\text{ms}s = 0 - \dfrac{1}{2}mv^2 \Rightarrow s = \SI{112.5}{m}.$$
	
	Mà thực tế ô tô chỉ cách chướng ngại vật $\SI{100}{m}$, nên ô tô có đâm vào chướng ngại vật trước khi kịp dừng lại.
}

\item \mkstar{3} [25]

\cauhoi{
	Xe ô tô khối lượng 2 tấn bắt đầu khởi hành trên đường thẳng nằm ngang, đi được $\SI{50}{m}$ thì đạt được vận tốc $\SI{54}{km/h}$. Lực kéo của động cơ bằng $\SI{10000}{N}$. Lấy $g=\SI{10}{m/s^2}$.
	\begin{enumerate}[label=\alph*)]
		\item Dùng định lý động năng, tính công của lực ma sát.
		\item Tính độ lớn lực ma sát tác dụng lên xe trong đoạn đường trên.
	\end{enumerate}
}
\loigiai{
\begin{enumerate}[label=\alph*)]
	\item Dùng định lý động năng, tính công của lực ma sát.
	
	Áp dụng định lý động năng:
	$$A_F + A_\text{ms} = \dfrac{1}{2}mv^2 - 0 \Rightarrow A_\text{ms} = \dfrac{1}{2}mv^2 - A= \dfrac{1}{2} mv^2 - Fs= \SI{-275000}{J}.$$
	\item Tính độ lớn lực ma sát tác dụng lên xe trong đoạn đường trên.
	
	Ta có $A_\text{ms} = -F_\text{ms} s \Rightarrow F_\text{ms} = \SI{5500}{N}$.
\end{enumerate}
	
}

\item \mkstar{3} [25]

\cauhoi{
	Xe ô tô khối lượng 1 tấn bắt đầu khởi hành trên đường thẳng nằm ngang, đi được $\SI{50}{m}$ thì đạt được vận tốc $\SI{36}{km/h}$. Cho hệ số ma sát giữa bánh xe và mặt đường là $\SI{0.05}{}$. Lấy $g=\SI{10}{m/s^2}$. Dùng định lý động năng tính công của lực kéo của động cơ, từ đó suy ra độ lớn lực kéo của động cơ.
}
\loigiai{
	
	Áp dụng định lý động năng:
	$$A_F + A_\text{ms} = \dfrac{1}{2}mv^2 - 0 \Rightarrow A_F = \dfrac{1}{2}mv^2 - A_\text{ms} = \dfrac{1}{2}mv^2 - (-F_\text{ms}s) = \SI{75000}{J}.$$
	
	Lực kéo của động cơ:
	$$A_F=Fs \Rightarrow F = \SI{1500}{N}.$$
}

	\end{enumerate}
\whiteBGstarEnd