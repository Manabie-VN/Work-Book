%\setcounter{chapter}{1}
\chapter{Luyện tập: Phương trình trạng thái của khí lí tưởng}
\begin{enumerate}
	\item %câu 1
	Một quả bóng có thể tích 2 lít, chứa khí ở $27^\circ\text{C}$ có áp suất $\SI{1}{atm}$. Người ta nung nóng quả bóng đến nhiệt độ $57^\circ\text{C}$ đồng thời giảm thể tích còn  1 lít. Áp suất lúc sau là bao nhiêu?
	\begin{mcq}(4)
		\item $\SI{1}{atm}$.
		\item $\SI{0,47}{atm}$.
		\item $\SI{2,2}{atm}$.
		\item $\SI{0,94}{atm}$.
	\end{mcq}
	\item %câu 2
	Một lượng khí $\text{H}_2$ đựng trong bình có $V_1=\SI{2}{l}$ ở áp suất $\SI{1,5}{atm}$, nhiệt độ $t_1=27^\circ\text{C}$. Đun nóng khí đến $t_2=127^\circ\text{C}$, do bình hở nên một nửa lượng khí thoát ra ngoài. Tính áp suất khí trong bình.
	\begin{mcq}(4)
		\item $\SI{3}{atm}$.
		\item $\SI{7,05}{atm}$.
		\item $\SI{4}{atm}$.
		\item $\SI{2,25}{atm}$.
	\end{mcq}
	\item %câu 3
	Ở $27^\circ\text{C}$ thể tích của một lượng khí là 6 lít. Thể tích của lượng khí đó ở nhiệt độ $227^\circ\text{C}$ khi áp suất không đổi là bao nhiêu?
	\begin{mcq}(4)
		\item $\SI{3}{l}$.
		\item $\SI{20}{l}$.
		\item $\SI{28,2}{l}$.
		\item $\SI{10}{l}$.
	\end{mcq}
	\item %câu 4
	Một lượng khí đựng trong xilanh có pittông chuyển động được. Các thông số của lượng khí: $\SI{1,5}{atm}$; $\SI{13,5}{l}$; $\SI{300}{K}$. Khi pit tông bị nén, áp suất tăng lên $\SI{3,7}{atm}$, thể tích giảm còn $\SI{10}{l}$. Xác định nhiệt độ khi nén.
	\begin{mcq}(4)
		\item $548,1^\circ\text{C}$.
		\item $275,1^\circ\text{C}$.
		\item $273^\circ\text{C}$.
		\item $450^\circ\text{C}$.
	\end{mcq}
	\item %câu 5
	Trong xilanh của một động cơ đốt trong có $\SI{2}{\deci\meter^3}$ hỗn hợp khí dưới áp suất $\SI{1}{atm}$ và nhiệt độ $47^\circ\text{C}$. Pit tông nén xuống làm cho thể tích của hỗn hợp khí chỉ còn $\SI{0,2}{\deci\meter^3}$ và áp suất tăng lên $\SI{15}{atm}$. Tính nhiệt độ của hỗn hợp khí nén.
	\begin{mcq}(4)
		\item $480^\circ\text{C}$.
		\item $470^\circ\text{C}$.
		\item $\SI{480}{K}$.
		\item $\SI{470}{K}$.
	\end{mcq}
	\item %câu 6
	Một lượng khí đựng trong một xi lanh có pittông chuyển động được. Các thông số trạng thái của lượng khí này là $\SI{2}{atm}$ và $\SI{300}{K}$. Khi pittong nén khí, áp suất của khí tăng lên tới $\SI{3,5}{atm}$, thể tích giảm còn $\SI{12}{l}$. Xác định nhiệt độ của khí nén.
	\item %câu 7
	Một bóng thám không được chế tạo để có thể tăng bán kính lên tới $\SI{10}{\meter}$ khi bay ở tầng khí quyển có áp suất $\SI{0,03}{atm}$ và nhiệt độ $\SI{200}{K}$. Hỏi bán kímh của bóng khi bơm, biết bóng được bơm khí ở áp suất $\SI{1}{atm}$ và nhiệt độ $\SI{300}{K}$ ?
	\item %câu 8
	Một bình kín chứa một mol khí Nitơ ở áp suất $\SI{e5}{\newton/\meter^2}$, nhiệt độ $27^\circ\text{C}$. Thể tích bình xấp xỉ bao nhiêu?
	\item %câu 9
	Nén 10 lít khí ở nhiệt độ $27^\circ\text{C}$ để thể tích của nó giảm chỉ còn 4 lít, quá trình nén nhanh nên nhiệt độ tăng đến $260^\circ\text{C}$. Áp suất khí đã tăng bao nhiêu lần?
	\item %câu 10
	Trong một động cơ điezen, khối khí có nhiệt độ ban đầu là $32^\circ\text{C}$ được nén để thể tích giảm bằng 1/16 thể tích ban đầu và áp suất tăng bằng 48,5 lần áp suất ban đầu. Tính nhiệt độ khối khí sau khi nén.
\end{enumerate}
\textbf{ĐÁP ÁN}
\begin{longtable}[\textwidth]{|p{0.15\textwidth}|p{0.15\textwidth}|p{0.15\textwidth}|p{0.15\textwidth}|p{0.15\textwidth}|}
	% --- first head
	\hline%\hspace{2 pt}
	\multicolumn{1}{|c}{\textbf{Câu 1}} & \multicolumn{1}{|c|}{\textbf{Câu 2}} & \multicolumn{1}{c|}{\textbf{Câu 3}} &
	\multicolumn{1}{c|}{\textbf{Câu 4}} &
	\multicolumn{1}{c|}{\textbf{Câu 5}} \\
	\hline
	C.&  & D. & C. & C. 	\\
	\hline
	
	\multicolumn{1}{|c|}{\textbf{Câu 6}} & \multicolumn{1}{c|}{\textbf{Câu 7}} & \multicolumn{1}{c|}{\textbf{Câu 8}} &
	\multicolumn{1}{c|}{\textbf{Câu 9}} &
	\multicolumn{1}{c|}{\textbf{Câu 10}}  \\ 
	\hline
	  & $^3\sqrt{45}\ \text{m}$. & $\SI{0.025}{\meter \cubed}$.&4,4 lần. & $\SI{92.45}{\kelvin}$.	\\
	\hline
\end{longtable}


